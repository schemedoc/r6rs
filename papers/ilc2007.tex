\documentclass{sigplanconf}

\usepackage{url}

\newcommand{\rn}[1]{R$^{#1}$RS}
\newcommand{\rsix}{\rn{6}}

\begin{document}

\copyrightyear{2007}  

\title{It's All about Being Right:\\ Lessons from the R$^6$RS Process}

\authorinfo{Michael Sperber}
{\textsc{DeinProgramm}}
{\texttt{sperber@deinprogramm.de}}

\maketitle

\conferenceinfo{International Lisp Conference 2007}

\begin{abstract}
  In the Revised Reports on Scheme up to \rn{5}, the language could
  only be changed by unanimous consent. Arguably, this process reached
  its limits with the Revised$^5$ Report on Scheme: Crucial language
  additions such as modules, records and exceptions had little chance
  of reaching unamimous consent, no matter what the specific design.
  While the editors of the Revised$^6$ Report no longer follow this
  rule, standardization is still driven by a strong desire to do the
  right thing. 
\end{abstract}

\section{Introduction}
\label{sec:intro}

\begin{quote}
  The original Revised Report on Scheme was taken as a model for
  future definitions of Scheme, and a self-selected group of so-called
  ``authors'' took on the role of evolving Scheme. The rule they
  adopted was that features could be added only by unanimous consent.
  After a fairly short period [\ldots], the rate of change of Scheme
  slowed down due to this rule. Only peer pressure in a highly
  intellectual group could convince any recalcitrant author to change
  his blackball. As a result there is a widely held belief that
  whenever a feature is added to Scheme, it is clearly the right
  thing.

  \begin{flushright}
    Guy L.\ Steele, Jr., Richard P.\ Gabriel\\ \textit{The Evolution of
      Lisp}~\cite{SteeleGabriel1993}
  \end{flushright}
\end{quote}
%
If all goes according to plan, 2007 will see the \textit{Revised$^6$
  Report on the Algorithmic Language Scheme}~\cite{R6RS} (or \rn{6}), the first
report of the series since 1998.  Many of the most significant changes
in \rn{6} can be traced back to giving up the unanimous-consent rule,
which had served the Revised Reports well until \rn{5}.  This paper
gives a short overview of the history of \rn{6}, its main changes over
\rn{5}, and attempts to motivate why giving up the unanimous-consent
rule was necessary.

\section{History}
\label{sec:history}

This section gives a brief overview of the history of \rn{6}.  The
history of the Revised Report series up to \rn{5} is summarized in the
``Background'' section of \rn{5}~\cite{R5RS}.  Decisions on changes to
the Revised Reports were made at physical meetings, by votes among
those present.  The process started out with rule of consensus, which
required a large majority for changes to pass.  This rule changed to
unanimous consent, which was in place until \rn{5}.  \rn{5} itself
came out in 1998, six years after the 1992 meeting where most of the
changes between \rn{4} and \rn{5} were decided.  By that time, a
disconnect between the process for producing Revised Reports and the
current state of the community had happened: Whereas previously, the
Scheme community was well represented by the authors present at the
meetings, many of the original authors were no longer part of the
Scheme community, and the community had diversified far beyond its
original size.

Richard Kelsey, after having co-edited \rn{5}, organized a workshop in
1998 (co-located with ICFP 1998) to discuss future directions to the
Scheme.  Many proposals for future changes to the language were
discussed, but no clear picture emerged as to how a new Revised Report
might be produced. The main tangible result of the workshop was the
creation of the \textit{Scheme Requests for Implementation} (SRFI)
process, which is now well established in the community.  As a
counterpoint to the Revised Reports, which require that everyone
approves a change, a SRFI is merely a request by an individual author
or a group of authors, and thus only subject to the approval of the
author(s).  The SRFI process sets up a mailing list for each such
proposal, where members of the community can make suggestions to
improve the SRFI; the author can revise the document throughout the
draft period, after which she can finalize or withdraw the SRFI.

While the SRFIs are a valuable vehicle for discussing proposals, and
many are widely adopted by implementations, they do not have a
coherent organizing principle; for some subjects, several competing
SRFIs exist.  Thus, the SRFIs in some cases serve as an incubator for
future standards.  However, they are no substitute for a coherent
language design and set of standard libraries.

At the 2002 Scheme Workshop, attendees formed a Strategy Committee to
discuss a process for producing new revisions of the report.  This
committee drafted a charter for Scheme standardization, which was
confirmed at the 2003 Scheme Workshop.  Subsequently, a Steering
Committee according to the charter was selected, consisting of Alan
Bawden, Guy L.\ Steele Jr., and Mitch Wand.  An editors' committee
charged with producing this report was also formed at the end of 2003,
consisting of Will Clinger, R.\ Kent Dybvig, Marc Feeley, Matthew
Flatt, Richard Kelsey, Manuel Serrano, and Mike Sperber, with Marc
Feeley acting as Editor-in-Chief.  Richard Kelsey resigned from the
committee in April 2005, and was replaced by Anton van Straaten.  Marc
Feeley and Manuel Serrano resigned from the committee in January 2006.
Subsequently, the charter was revised to reduce the size of the
editors' committee to five and to replace the office of
Editor-in-Chief by a Chair and a Project Editor.  R.\ Kent Dybvig
served as Chair, and Mike Sperber served as Project Editor.  Parts of
the report were posted as Scheme Requests for Implementation (SRFIs)
and discussed by the community before being revised and finalized for
the report.  Jacob Matthews and Robby Findler wrote the operational
semantics for the language core.

The \rn{6} editors broke with the unanimous-consent rule; instead,
decisions have been made by simple vote.  While many decisions were
prepared by e-mail discussion, most votes were actually cast in a
series of telephone conferences starting in 2007.

The editors' committee produced the first \rn{6} draft (dubbed
``\rn{5.91}'') in September 2006, at the same time initiating two
public mailing lists (on a newly created \url{r6rs.org} domain) for general
discussion of the standard and for making formal comments, which the
editors are obliged to answer.  Discussion has been lively.  At the
time of writing, about 195 formal comments have been submitted.  130
formal comments have received formal responses from the editors.  Many
formal comments have resulted in changes to the report, leading to a
second draft (``\rn{5.92}'') that came out in January 2007.  At least one
more draft is expected to appear before \rn{6} is finalized.  For
\rn{5.92}, the report has been split into one document describing the
library system and the base library, and another describing the
standard libraries.  A document containing non-normative appendices as
well as a rationale document are expected to complement the final
report.

\section{Case study: records}
\label{sec:records}

``Records'' provide a perfect example of the limits of the
unanimous-consent principle: Until and including \rn{5}, Scheme lacked
a facility for user-defined types disjoint from the built-in ones, as
well as a simple facility for defining record-like aggregate types.  Few
disagree that records are desirable additions, as
well as a crucial prerequisite for Scheme to be taken seriously as a
language for writing applications.  Almost every Scheme system comes
with some kind of ``record'' or ``structure'' facility, some come with
several.  Almost no two of them are alike.

Most record systems conflate several different mechanisms, among them
user-defined types, aggregate types, opacity, and inheritance.  Given
the typical Schemer's desire for clean design, consensus on any given
proposal has traditionally been very difficult.  Moreover, the design of
a syntactic record-type-defining form exhibits more degrees of freedom
than most other special forms in \rn{5}: A particular design is as
much subject to taste as to technical criteria.

Serious discussion of record systems for inclusion in the language
started in 1987 on the \texttt{rrrs-authors} mailing list.  Almost
every design aspect of the record system that will be in \rn{6} had already
been discussed back in 1987 and 1988, in particular separation between
user-defined types and aggregates, opacity, inheritance, sealedness,
and procedural vs.\ syntactic systems.  The discussion culminated in
Pavel Curtis's proposal~\cite{Curtis1991}, a candidate for inclusion
in \rn{4}.  (The proposal described only a procedural system,
correctly anticipating that the syntax of a syntactic layer would be
highly contentious.)  Still, the lack of consensus on the particulars
of the proposal prevented inclusion in the report, first for \rn{4},
and then for \rn{5}.

Kelsey's SRFI~9~\cite{srfi9} described a syntactic form for
defining new record types, and alleviated the situation somewhat, and
is widely implemented.

Given \rn{6}'s guiding principle of enabling the creation and
distribution of substantial programs (see the next section), the
editors considered records a necessary addition to the report.  Still,
the specifics of the design were still hotly debated among the
editors.  Setting down a minimum set of requirements at a physical
meeting in May 2005 did not substantially help to reach consensus.  A
subset of the editors finally drafted a proposal later in 2005 and
presented it for public discussion as SRFI~76~\cite{srfi76}.  While
substantial progress was made on improving the proposal and making it
more palatable for many potential users, it is clear that it will not
satisfy everybody: It is over-engineered and too complex to some while
lacking in features to others.  Even in the current \rn{6} draft,
records are not part of the base language but instead reside in a set
of four separate libraries.  Thus, records are still somewhat ``in
play''; future changes seem likely.

Given the wildly different opinions and tastes on what constitutes a
good record system, it is unlikely that unanimous consent on a record
system would be possible in any sufficiently diverse group of Scheme users.
Breaking with the unanimous-consent tradition was necessary to make
progress.

\section{Guiding principles}
\label{sec:principles}

To help guide the standardization effort, the editors have adopted a
set of principles.  Many of the principles had already been used for
producing the previous revisions of the report.  However, a few new
principles were added:

\begin{itemize}
\item allow programmers to create and distribute substantial programs
  and libraries, e.g., SRFI implementations, that run without
  modification in a variety of Scheme implementations;
  
\item support procedural, syntactic, and data abstraction more fully
  by allowing programs to define hygiene-bending and hygiene-breaking
  syntactic abstractions and new unique datatypes along with
  procedures and hygienic macros in any scope;
  
\item allow programmers to rely on a level of automatic run-time type
  and bounds checking sufficient to ensure type safety; and

\item allow implementations to generate efficient code, without
  requiring programmers to use implementation-specific operators or
  declarations.
\end{itemize}
%
Satisfying the first of these principles in particular would have been
impossible under a unanimous-consent regime, as the records discussion
showed.  Consequently, the design decisions taken for \rn{6} contain
progress (or at least movement) on a large number of issues, but many
issues have been resolved by compromises that probably would not have
been approved by unanimous consent.  Producing a coherent and tasteful
whole is now primarily at the discretion of the individuals on the
editors' committee, and the editors bear the responsibility of
reflecting input from the community in draft revisions of the report.

\section{Language changes}
\label{sec:changes}

At the time of writing, the combined base and standard
libraries reports total about 150 pages---three times the size
of \rn{5}.  (The core report is about 87 pages.)  The main additions
are:
%
\begin{itemize}
\item A library system allows programmers to organize code into parts,
  and package and share those parts.
\item A records system, consisting of a procedural and two syntactic
  layers has been added.
\item An exception system and a condition system for describing
  exceptional situations have been added.
\item The basis for the character and string types is now Unicode.
\item A version of the \texttt{syntax-case} facility for writing
  hygienic macros as code has been added.
\item A number of standard libraries have been added, for purposes
  such as dealing with lists, binary data, sorting, I/O, file-system
  manipulation, specialized arithmetic, hash tables, and enumerations.
\end{itemize}
%
A large number of smaller changes were made, such as the move to case
sensitivity, closure properties of complex-number arithmetic, or
\texttt{letrec*} semantics for internal definitions.  Moreover, many
specifications were tightened to reduce variance between
implementations, and to improve portability.

All in all, the changes between \rn{5} and \rn{6} are significantly
greater than between any other pair of subsequent Revised
Reports.

\section{Conclusion}
\label{sec:conclusion}

Unanimous consent was an excellent rule for the reports up to \rn{5},
leading to a solid foundation for Scheme as a principled and minimal
language.  However, sticking with the unanimous-consent rule would
have prevented most of the progress made with \rn{6}.  Time will tell
whether the changes made are overall for the better: The fact that
decisions are made only by simple votes among five individuals is a
potential problem that can only be alleviated by determined scrutiny
and active involvement of the community at large.  So far, the
community has done an excellent job identifying problems with the
drafts reports and of helping to improve the document in other ways.

\paragraph*{Acknowledgements} I am grateful to my co-editors for the
huge effort they put into \rn{6}, and the patience and equanimity they
have displayed during the effort.  I also thank the members of the Scheme
community who have posted on the various mailing lists associated with
\rn{6} and who are instrumental in making \rn{6} a success.  I also
thank Will Clinger for helping with the history section, and
pointing out numerous mistakes in a draft of this paper.

\bibliographystyle{plain}
\bibliography{rrs}


\end{document}

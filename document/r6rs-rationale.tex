\documentclass[twoside,twocolumn]{algol60}
%\documentclass[twoside]{algol60}

\usepackage{xr}

\pagestyle{headings} 
\showboxdepth=0
\makeindex
% Macros for R^nRS.

\usepackage{makeidx}
\usepackage{url}

% tex2page.sty mucks with in some manner
\let\centerlinesaved=\centerline
\usepackage{tex2page}
\let\centerline=\centerlinesaved

% \let\htmlonly=\iffalse
% \let\endhtmlonly=\fi
% \let\texonly=\iftrue
% \let\endtexonly=\fi

\makeatletter

\texonly
\newcommand{\topnewpage}{\@topnewpage}
\endtexonly

\htmlonly
\newcommand{\topnewpage}[0][]{#1}
\endhtmlonly

\newcommand{\authorsc}[1]{{\scriptsize\scshape #1}}

% Chapters, sections, etc.

\newcommand{\extrapart}[1]{
 % \chapter{#1}
  \chapter*{#1}
  \markboth{#1}{#1}
  \vskip 1ex
  \addcontentsline{toc}{chapter}{#1}}

\newcommand{\clearchaptergroupstar}[1]{
  \texonly
  \clearpage
  \addcontentsline{toc}{chaptergroup}{#1}
  \topnewpage[
    \centerline{\large\bf\uppercase{#1}}
    \bigskip]
    \endtexonly
  }

\newcommand{\clearchapterstar}[1]{
  \clearpage
  \topnewpage[
    \centerline{\large\bf\uppercase{#1}}
    \bigskip]}

\newcommand{\clearextrapart}[1]{
  \clearchapterstar{#1}
  \markboth{#1}{#1}
  \addcontentsline{toc}{chapter}{#1}}

\newcommand{\vest}{}
\newcommand{\dotsfoo}{$\ldots\,$}

\newcommand{\sharpfoo}[1]{{\tt\##1}}
\newcommand{\schfalse}{\sharpfoo{f}}
\newcommand{\schtrue}{\sharpfoo{t}}

\newcommand{\ampfoo}[1]{{\tt\&#1}}

\newcommand{\libfoo}[1]{{\tt(#1)}}

\newcommand{\singlequote}{{\tt'}}  %\char19
\newcommand{\doublequote}{{\tt"}}
\newcommand{\backquote}{{\tt\char18}}
\newcommand{\backwhack}{{\tt\char`\\}}
\newcommand{\comma}{{\tt\char`\@}}
\newcommand{\atsign}{{\tt\char`\@}}
\newcommand{\bang}{{\tt\char`\!}}
\newcommand{\sharpsign}{{\tt\#}}
\newcommand{\verticalbar}{{\tt|}}
\newcommand{\openbracket}{{\tt\char`\[}}
\newcommand{\closedbracket}{{\tt\char`\]}}
\newcommand{\ampersand}{{\tt\char`\&}}

\newcommand{\coerce}{\discretionary{->}{}{->}}

% Knuth's \in sucks big boulders
\def\elem{\hbox{\raise.13ex\hbox{$\scriptstyle\in$}}}

\newcommand{\meta}[1]{{\noindent\hbox{\rm$\langle$#1$\rangle$}}}
\let\hyper=\meta
\newcommand{\hyperi}[1]{\hyper{#1$_1$}}
\newcommand{\hyperii}[1]{\hyper{#1$_2$}}
\newcommand{\hyperiii}[1]{\hyper{#1$_3$}}
\newcommand{\hyperj}[1]{\hyper{#1$_i$}}
\newcommand{\hypern}[1]{\hyper{#1$_n$}}
\texonly
\newcommand{\var}[1]{\noindent\hbox{\textnormal{\textit{#1}}}}
\endtexonly
\htmlonly
\newcommand{\var}[1]{\textnormal{\textit{#1}}}
\endhtmlonly
\newcommand{\vari}[1]{\var{#1$_1$}}
\newcommand{\varii}[1]{\var{#1$_2$}}
\newcommand{\variii}[1]{\var{#1$_3$}}
\newcommand{\variv}[1]{\var{#1$_4$}}
\newcommand{\varj}[1]{\var{#1$_j$}}
\newcommand{\varn}[1]{\var{#1$_n$}}

\newcommand{\vr}[1]{{\noindent\hbox{$#1$\/}}}  % Careful, is \/ always the right thing?
\newcommand{\vri}[1]{\vr{#1_1}}
\newcommand{\vrii}[1]{\vr{#1_2}}
\newcommand{\vriii}[1]{\vr{#1_3}}
\newcommand{\vriv}[1]{\vr{#1_4}}
\newcommand{\vrv}[1]{\vr{#1_5}}
\newcommand{\vrj}[1]{\vr{#1_j}}
\newcommand{\vrn}[1]{\vr{#1_n}}

%%R4%% The excessive use of the code font in the numbers section was
% confusing, somewhat obnoxious, and inconsistent with the rest
% of the report and with parts of the section itself.  I added
% a \tupe no-op, and changed most old uses of \type to \tupe,
% to make it easier to change the fonts back if people object
% to the change.

\newcommand{\type}[1]{{\it#1}}
\newcommand{\tupe}[1]{{#1}}

\newcommand{\defining}[1]{\mainindex{#1}{\em #1}}
\newcommand{\ide}[1]{{\schindex{#1}\frenchspacing\tt{#1}}}

\newcommand{\lambdaexp}{{\cf lambda} expression}

\newcommand{\callcc}{{\tt call-with-current-continuation}}

\newcommand{\mainschindex}[1]{\label{#1}\index{#1@\texttt{#1}}}
\newcommand{\mainindex}[1]{\index{#1}}
\newcommand{\schindex}[1]{\index{#1@\texttt{#1}}}
\newcommand{\sharpindex}[1]{\index{#1@\texttt{\#{}#1}}}
\newcommand{\ampindex}[1]{\index{#1@\texttt{\&{}#1}}}
\newcommand{\libindex}[1]{\index{#1@\texttt{(#1)}}}

\newcommand{\extref}[2]{\texonly\ref{#1}\endtexonly\htmlonly{on ``#2''}\endhtmlonly}

\renewenvironment{theindex}
{\clearpage
\topnewpage[
    \begin{center}
      \large\bf\MakeUppercase{\indexheading}
    \end{center}
    \vskip 1ex \bigskip]
    \markboth{Index}{Index}
    \addcontentsline{toc}{chapter}{\indexheading}
    \parindent\z@
    \texonly\parskip\z@ plus .1pt\endtexonly\relax\let\item\@idxitem
    \indexintro\par\bigskip}
               {\texonly\clearpage\endtexonly}


\newcommand{\domain}[1]{#1}
\newcommand{\nodomain}[1]{}
%\newcommand{\todo}[1]{{\rm$[\![$!!~#1$]\!]$}}
\newcommand{\todo}[1]{}

% \frobq will make quote and backquote look nicer.
\def\frobqcats{%\catcode`\'=13
\catcode`\`=13{}}
{\frobqcats
\gdef\frobqdefs{%\def'{\singlequote}
\def`{\backquote}}}
\def\frobq{\frobqcats\frobqdefs}

% \cf = code font
% Unfortunately, \cf \cf won't work at all, so don't even attempt to
% next constructions which use them...
\newcommand{\cf}{\frenchspacing\frobq\tt}

\texonly
% Same as \obeycr, but doesn't do a \@gobblecr.
{\catcode`\^^M=13 \gdef\myobeycr{\catcode`\^^M=13 \def^^M{\\}}%
\gdef\restorecr{\catcode`\^^M=5 }}
\endtexonly

{\obeyspaces\gdef {\hbox{\hskip0.5em}}}

\gdef\gobblecr{\@gobblecr}

\def\setupcode{\@makeother\^}

% Scheme example environment
% At 11 points, one column, these are about 56 characters wide.
% That's 32 characters to the left of the => and about 20 to the right.

\newcommand{\exception}[1]{{\cf#1} \textnormal{\textit{exception}}}
\newenvironment{schemenoindent}{
  % Commands for scheme examples
  \newcommand{\ev}{\>\>\evalsto}
  \newcommand{\lev}{\\\>\evalsto}
  \newcommand{\unspecified}{{\em{}unspecified}}
  \newcommand{\theunspecified}{{\em{}the unspecified value}}
  \setupcode
  \small \cf \obeyspaces \myobeycr
  \begin{tabbing}%
\qquad\=\hspace*{5em}\=\hspace*{9em}\=\evalsto~\=\kill%   was 16em
\gobblecr}{\unskip\end{tabbing}}

%\newenvironment{scheme}{\begin{schemenoindent}\+\kill}{\end{schemenoindent}}
\newenvironment{scheme}{
  % Commands for scheme examples
  \newcommand{\ev}{\>\>\evalsto}
  \newcommand{\lev}{\\\>\evalsto}
  \renewcommand{\em}{\rmfamily\itshape}
  \newcommand{\unspecified}{{\em{}unspecified}}
  \newcommand{\theunspecified}{{\em{}the unspecified value}}
  \setupcode
  \small \cf \obeyspaces \myobeycr
  \begin{tabbing}%
\qquad\=\hspace*{5em}\=\hspace*{9em}\=\evalsto~\=\+\kill%   was 16em
\gobblecr}{\unskip\end{tabbing}}

\newcommand{\evalsto}{$\Longrightarrow$}

% Rationale

\newenvironment{rationale}{%
\bgroup\small\noindent{\em Rationale:}\space}{%
\egroup}

% Notes

\newenvironment{note}{%
\bgroup\small\noindent{\em Note:}\space}{%
\egroup}

% Names of library modules

\newcommand{\library}[1]{{\tt (#1)}}
\newcommand{\deflibrary}[1]{\library{#1}\libindex{#1}}

% Manual entries

\newenvironment{entry}[1]{
  \vspace{3.1ex plus .5ex minus .3ex}\noindent#1%
\unpenalty\nopagebreak}{\vspace{0ex plus 1ex minus 1ex}}

\newcommand{\exprtype}{syntax}

\newcommand{\unspecifiedreturn}{the unspecified value}

% Primitive prototype
\newcommand{\pproto}[2]{\unskip%
\hbox{\cf\spaceskip=0.5em#1}\hfill\penalty 0%
\hbox{ }\nobreak\hfill\hbox{\rm #2}\break}

% Parenthesized prototype
\newcommand{\proto}[3]{\pproto{(\mainschindex{#1}\hbox{#1}{\it#2\/})}{#3}}

% Variable prototype
\newcommand{\vproto}[2]{\mainschindex{#1}\pproto{#1}{#2}}

% Condition-type prototype
 \newcommand{\ctproto}[1]{\ampindex{#1}\pproto{\ampfoo{#1}}{condition type}}

% Extending an existing definition (\proto without the index entry)
\newcommand{\rproto}[3]{\pproto{(\hbox{#1}{\it#2\/})}{#3}}

% Extending an existing definition, with index entry
\newcommand{\irproto}[3]{\schindex{#1}\rproto{#1}{#2}{#3}}

% Variable prototype
\newcommand{\rvproto}[2]{\pproto{#1}{#2}}

% Grammar environment

\newenvironment{grammar}{
  \def\:{\goesto{}}
  \def\|{$\vert$}
  \cf \myobeycr
  \begin{tabbing}
    %\qquad\quad \= 
    \qquad \= $\vert$ \= \kill
  }{\unskip\end{tabbing}}

%\newcommand{\unsection}{\unskip}
\newcommand{\unsection}{{\vskip -2ex}}

% Commands for grammars
\newcommand{\arbno}[1]{#1\hbox{\rm*}}  
\newcommand{\atleastone}[1]{#1\hbox{$^+$}}

\newcommand{\goesto}{$\longrightarrow$}

\newcommand{\syntax}{{\em Syntax: }}
\newcommand{\semantics}{{\em Semantics: }}
\newcommand{\implresp}{{\em Implementation responsibilities: }}

\newcommand{\rrs}[1]{\textit{Revised$^#1$ Report on the Algorithmic Language Scheme}}

\newcommand{\libindexentry}[1]{#1 (library)}

\makeatother


\newcommand{\rn}[1]{R$^{#1}$RS}
\newcommand{\rsix}{\rn{6}}

\texonly\externaldocument[report:]{r6rs}\endtexonly
\texonly\externaldocument[lib:]{r6rs-lib}\endtexonly
\texonly\externaldocument[app:]{r6rs-app}\endtexonly

\def\headertitle{Revised$^{5.94}$ Scheme Rationale}
\def\integerversion{6}

\begin{document}

\thispagestyle{empty}

\topnewpage[{
\begin{center}   {\huge\bf
        Revised{\Huge$^{\mathbf{\htmlonly\tiny\endhtmlonly{}5.94}}$} Report on the Algorithmic Language \\
                              \vskip 3pt
                              Scheme\\
                                \vskip 1.5ex
                              --- Rationale ---}

\vskip 1ex$$
\begin{tabular}{l@{\extracolsep{.5in}}lll}
\multicolumn{4}{c}{M\authorsc{ICHAEL} S\authorsc{PERBER}}
\\
\multicolumn{4}{c}{W\authorsc{ILLIAM} C\authorsc{LINGER},
  R.\ K\authorsc{ENT} D\authorsc{YBVIG},
  M\authorsc{ATTHEW} F\authorsc{LATT},
  A\authorsc{NTON} \authorsc{VAN} S\authorsc{TRAATEN}}
\\
\multicolumn{4}{c}{(\textit{Editors})} \\[1ex]
\multicolumn{4}{c}{\bf 15 March 2007}
\end{tabular}
$$



\end{center}

\chapter*{Summary}
\medskip

This document describes rationales for some of the design decisions
behind the \textit{Revised$^6$ Report on the Algorithmic Language
  Scheme}.  The focus is on changes made since the last revision on
the report.  However, numerous fundamental design decisions of Scheme
are also explained.  This report also contains some historical notes.
The formal comments submitted for drafts of the report and their
responses, as archived on \url{http://www.r6rs.org/} provides more
valuable background information on many decisions that are reflected
in the report.

This document frequently refers back to the \textit{Revised$^6$ Report
  on the Algorithmic Language Scheme}~\cite{R6RS}, the
\textit{Revised$^6$ Report on the Algorithmic Language Scheme ---
  Libraries ---}~\cite{R6RS-libraries}, and the
\textit{Revised$^6$ Report on the Algorithmic Language Scheme ---
  Non-Normative Appendices ---}~\cite{R6RS-appendices}; references to the report are
identified by designations such as ``report section'' or ``report
chapter'', references to the library report are identified by
designations such as ``library section'' or ``library chapter'', and
references to the appendices are identified by
designations such as ``appendix'' or ``appendix section''.

\bigskip

\begin{center}
{\large \bf
*** DRAFT*** \\
}\end{center}

This is a preliminary draft.  It is intended to reflect the decisions
taken by the editors' committee, but contains many mistakes,
ambiguities and inconsistencies.

}]

\texonly\clearpage\endtexonly

\chapter*{Contents}
\addvspace{3.5pt}                  % don't shrink this gap
\renewcommand{\tocshrink}{-4.0pt}  % value determined experimentally
{%\footnotesize
\tableofcontents
}

\vfill

\texonly\clearpage\endtexonly

%%%%%%%%%%%%%%%%%%%%%%%%%%%%%%%%%%%%%%%%%%%%%%%%%%%%%%%%%%%%%%%%

\chapter{Introduction}

The \textit{Revised$^6$ Report on the Algorithmic Language Scheme}
(\rn{6} for short) is the sixth of the Revised Reports on Scheme.

Prior to \rn{6}, decisions on changes to the Revised Reports were made
at physical meetings, by votes among those present.  The process
started out with a rule of consensus, which required a large majority
for changes to pass.  This rule changed to unanimous consent, which
was in place until \textit{Revised$^6$ Report on the Algorithmic
  Language Scheme} (\rn{5} for short).  \rn{5} itself came out in
1998, six years after the 1992 meeting where most of the changes
between \rn{4} and \rn{5} had been decided.  By that time, a
disconnect between the process for producing Revised Reports and the
current state of the community had happened: Whereas previously, the
Scheme community was well represented by the authors present at the
meetings, many of the original authors were no longer part of the
Scheme community, and the community had diversified far beyond its
original size.

Richard Kelsey, after having co-edited \rn{5}, organized a workshop in
1998 (co-located with ICFP 1998) to discuss future directions to the
Scheme.  Many proposals for future changes to the language were
discussed, but no clear picture emerged as to how a new Revised Report
might be produced. The main tangible result of the workshop was the
creation of the \textit{Scheme Requests for Implementation} (SRFI)
process (\url{http://srfi.schemers.org}), which is now well
established in the community.  As a counterpoint to the Revised
Reports, which require that everyone approves a change, a SRFI is
merely a request by an individual author or a group of authors, and
thus only subject to the approval of the author(s).  The SRFI process
sets up a mailing list for each such proposal, where members of the
community can make suggestions to improve the SRFI; the author can
revise the document throughout the draft period, after which she can
finalize or withdraw.

While the SRFIs are a valuable vehicle for discussing proposals, and
many are widely adopted by implementations, they do not have a
coherent organizing principle; for some subjects, several competing
SRFIs exist.  Thus, the SRFIs in some cases serve as an incubator for
future standards but are no substitute for a coherent
language design and set of standard libraries.

The editors of \rn{6} broke with the unanimous-consent rule; instead,
decisions have been made by simple vote.  While many decisions were
prepared by e-mail discussion, most votes were actually cast in a
series of telephone conferences starting in 2007.


\chapter{Numbers}
\label{numberschapter}

\section{Infinities, NaNs}
\label{infinitiesnansection}

Infinities and NaNs and are artefacts that help
deal with the inexactness of binary floating-point arithmetic.  The
semantics dealing with infinities and NaNs, or the circumstances
leading to their generation are somewhat arbitrary.  However, as most
Scheme implementation use an IEEE~754-conformant
implementation~\cite{IEEE} of flonums and inexact reals, \rn{6} uses
the particular semantics from this standard as the basis for the
treatment of infinities and NaNs in the report.  This is also the
reason why infinities and NaNs are flonums and inexact reals, allowing
Scheme systems to exploit the closure properties arising from their
being part of the standard IEEE-754 floating-point representation.
See section~\ref{closurepropertiessection} for details on closure
properties.

\rn{6} intentionally does not require a Scheme implementation to use
infinities and NaNs as specified in IEEE~754.  Hence, support for them
is optional.

\section{Distinguished -0.0}

FIXME

\chapter{Lexical syntax and read syntax}

\section{Symbol and identifier syntax}

\subsection{Escaped symbol constituents}

While revising the syntax of symbols and identifiers, the editors'
goal was to make symbols subject to write/read invariance, i.e.\ to
allow each symbol to be written out using {\cf put-datum} (section
\ref{lib:put-datum}) or {\cf write} (section~\ref{lib:write}), and
read back in using {\cf get-datum} (section \ref{lib:get-datum}) or
{\cf read} (section \ref{lib:read}), yielding the same symbol.  This
was not the case in \rrs{5}, as symbols could contain arbitrary
characters such as spaces which could not be part of their external
representation.  Moreover, symbols could distinguish case, whereas
their external representation could not.

For representing unusual characters in the symbol syntax, the report
provides the {\cf\backwhack{}x} escape syntax, which allows specifying
an arbitrary Unicode scalar value.  This also has the advantage that
arbitrary symbols can be represented using only ASCII, which allows
referencing them from Scheme programs restricted to ASCII or some
other subset of Unicode.

A popular choice for extending the set of characters that can occur in
symbols is the vertical-bar syntax of Common Lisp.  The vertical-bar
syntax of Common Lisp carries the risk of confusing the syntax of
identifiers with that of consecutive lexemes, and also does not allow
representing arbitrary characters only using ASCII.  Consequently, it
was not adopted for \rn{6}.

\subsection{Case sensitivity}
\label{casesensitivityrationalesection}

The change from case-insensitive syntax in \rn{5} to case-sensitive
syntax is a major change.  Many technical arguments exist in favor of
both case sensitivity and case insensitivity, and any attempt to list
them all here would be incomplete.  What is relevant here is that
switching to case sensitivity will break backwards compatibility, and
might set a precedent for switching a technically more or less
arbitrary decision again in the future.

The editors decided to switch to case sensitivity because they
perceived that a significant majority of the Scheme community favored
the change.  This perception has been strengthened by polls at the
2004 Scheme workshop, on the {\cf plt-scheme} mailing list, and the
{\cf r6rs-discuss} mailing list.

The directives described in appendix~\ref{app:caseinsensitivityappendix}
allow specifying that a code portion (or other syntactic data) was
written under the old assumption of case-insensitivity and therefore
must be case-folded upon reading.

\subsection{Identifiers starting with {\tt ->}}

\rn{6} introduces a special rule in the lexical syntax for
identifiers starting with the characters {\cf ->}.  In \rn{5}, such
identifiers were not valid lexemes.  (In \rn{5}, a lexeme starting
with a {\cf -} character has to be a number.)  However, many existing
Scheme implementations prior to \rn{6} already supported identifiers
starting with {\cf ->}.  (Many readers would classify any lexeme as an
identifier starting with {\cf -} for which {\cf string->number}
returns \schfalse{}.)  As a result, a signficant amount of otherwise
portable Scheme code used identifiers starting with {\cf ->}, which
are a convenient choice for many names.  Therefore, \rn{6} legalizes
these identifiers.  The separate rule is not particularly elegant.
However, designing a more elegant rule that does not overlap with
numbers or other lexeme classes has proved to be surprisingly
difficult.

 
\section{Comments}

While \rn{5} only provided the {\cf;} syntax for comments, the report
now describes three essential kinds: In addition to {\cf;}, {\tt \#|}
and {\tt |\#} delimit block comments, and {\cf\sharpsign;} starts a
``datum comment''.  ({\cf\sharpsign!r6rs} is also a kind of command,
albeit with a specific, fixed purpose.) 

Block comments provide a more convient way of writing multi-line
comments, and are an often-requested and often-implemented syntactic
addition to the language. 

The rationale for {\cf\sharpsign;} is not as readily apparent: It
automatically comments out a single datum, the basic unit of Scheme
syntax, something that the other comment mechanisms cannot do.  As the
specification notes, {\cf \sharpsign\verticalbar} \ldots {\cf
  \verticalbar\sharpsign} cannot generally be used to comment out an
arbitrary datum or set of data. Moreover, while {\cf\sharpsign;} is
probably most useful during development and debugging, it is still
useful to have a standard notation for commenting out a datum,
particularly since programmers sometimes develop and debug a single
piece of code concurrently on multiple systems.

\chapter{Semantic concepts}

\section{Argument aund subform checking}
\label{argumentchecking}

The report requires implementations to check the arguments of
procedures and subforms for syntactic forms for adherence to the
specification.  However, implementations are not required to detect
every violation of a specification.  Specifically, the report allows
the following exceptions:
%
\begin{enumerate}
\item Checking of restrictions that are undecidable is not required,
  such as certain properties of procedures passed as arguments, or
  properties of subexpressions, whose macro expansion may not
  terminate.
\item Checking that an argument is a list where doing so would be
  impractical or expensive is not required.  Specically, procedures that
  invoke another procedure passed as an argument are not required to
  check that a list remains a list after every invocation.
\item Checking that an argument is of the specified types when future
  extensions are explicitly allowed.
\end{enumerate}
%
The second item deserves special attention, as the specific decisions
made for the report are meant to enable ``picky'' implementations that
catch as many violations and unportable assumptions made by programs
as possible, while also enabling practical and efficient
implementations that execute programs quickly.

\section{Safety}

\rn{5} described many situations not specified in the report as ``is
an error'': Portable programs could not cause such situations, but
implementations were free to implement arbitary behavior under this
umbrella.  This included ``crashing'' the running programming, or
somehow compromising the integrity of its execution model, resulting
in random behavior.  This was in sharp contrast to the assumption to
many users that assumed Scheme was a ``safe'' language where each
violation of a restriction of the language standard or the
implementation would at least result in defined behavior, such as
interrupting or aborting the program and/or starting a debugger.

To avoid the problems associated with this arbitrary behavior, all
libraries specified in the report must now be safe, and react to
detected violations of the specification by raising
an exception, which allows the program to detect and react
to the violation itself.

The report allows implementations to provide ``unsafe'' libraries that
may compromise safety.

\section{Proper tail recursion}

While a proper tail recursion has been a cornerstone property of
Scheme since its inception, it is difficult to implement efficiently
on some architectures, specifically those compiling to higher-level
intermediate languages such as C or to certain virtual-machine
architectures such as JVM or CIL.

Nevertheless, abandoning proper tail recursion as a language property
and relegating it to optional optimizations would have far-reaching
consequences: Many programs written with the assumption of proper tail
recursion would no longer work.  Moreover, the lack of proper tail
recursion would prevent the natural expression of certain programming
styles such as Actors-style message-passing systems, self-replacing
servers, or automata written as mutually recursive procedures.
Furthermore, if they did not exist, special ``loop'' constructs would
have to be added to the language to compensate for the lack of another
iteration construct.  Consequently, proper tail recursion remains an
essential and required aspect of the Scheme language.

\chapter{Notation and terminology}

\section{Requirement levels}

\rn{6} distinguishes between different requirement levels, both for
the programmer and for the implementation.  Specifically, the
distinction between ``should'' and ``must'' is important: For example,
``should'' is used for restrictions on argument types that are
undecidable or potentially too expensive to enforce.  The use of
``should'' allows implementations to perform quite extensive checking
of restrictions on arguments (see section~\ref{argumentchecking}), but
also to eschew more expensive checks.

\section{Entry format}

While it is reasonable to require the programmer to adhere to
restrictions on arguments, some of these restrictions are eihter
undecidable or too expensive to always enforce (see
section~\ref{argumentchecking}).  Therefore, some entries have
additional paragraphs labelled ``\textit{Implementation
  responsibilities}'' that distinguishes the responsibilities of the
implementation from those of the implementation.

\chapter{Libraries}

The design of the library system was a challenging process: Many
existing Scheme implementations offer ``module systems'', but they
differ dramatically both in functionality and in the goals they
address.  The library system was designed with the primary
requirement of allowing programmers to write, distribute, and evolve
portable code.  An secondary requirement was to be able to separately
compile libraries in the sense that compiling a library requires only
having compiled its dependencies.  This entailed the following
corollary requirements:
%
\begin{itemize}
\item Composing libraries requires some kind of dependency management.
\item Libraries from different sources may have name conflicts.
  Consequently, some form of name-space management is needed.
\item In Scheme, abstractions provided by code includes macros.
  Hence, it must be possible to export macros from libraries, with all
  the consequences dictated by the referential-transparency property
  of hygienic macros.
\end{itemize}
%
The library system does not address the following goals, which were
considered during the design process:
%
\begin{itemize}
\item Mutually dependent libraries.
\item Separation of library interface from library implementation.
\item Local modules and local imports. 
\end{itemize}
%
This section discusses some aspects of the design of the library
system that have been controversial:

\section{Syntax}

It's not clear that a sequence of forms is more convenient than a
single form for processing and generation. Both syntactic choices have
technical merits and drawbacks. The single-form syntax chosen for
\rn{6} has the advantage of being self-delimiting.

A difference between top-level programs and libraries is that a
program contains only one top-level program but multiple libraries.
As a result, we believe that delimiting the text for a library body is
a common enough need (in streams of various kinds) that it is worth
standardizing the delimiters; parentheses are the obvious choice.

\section{Local import}

Some Scheme implementations feature module systems that allow
importing a module's bindings into a local environment.  While we
agree with the described benefits of local import, a drawback of local
imports is that library dependencies are not readily apparent from a
library header. This drawback seems significant enough to merit
longer-term consideration. Since leaving out local import for now does
not preclude it from being added later, and since the report already
tackles so many library-related issues, the editors decided to omit
local import from the report.

\section{Local modules}

Some Scheme implementations feature module systems that allow defining
``local modules'' in a local environment rather than only at the
library level.  Specifically, this enables syntactic abstraction over
modules.  Local modules are undoubtedly useful, but also complicate
the scoping rules of the language: Whereas the library system only
allows transporting bindings from one library top level to another,
local modules allow transporting bindings into local scopes, with
far-reaching consequences for the rules that may be used to determine
scoping, particularly in the absence of {\cf syntax-case}.

Moreover it is debatable whether top-level {\cf library} is (or should
be) the same construct as local {\cf module}. The job of the library
system is to organize the top-level namespace, and not to serve as a
target for expansion of sophisticated macros. Of course, the broader
roles and syntactic similarity of {\cf module} and {\cf library}
suggest merging the concepts, but merging the concepts further
broadens the role of each.  Such generalization may seem intuitively
right to Scheme programmers, but all attempts by the editors at such
broadening led away from consensus rather than toward it.

\section{Fixed {\tt import} and {\tt export} clauses}

The {\cf import} and {\cf export} clauses of the {\cf library} form
are fixed---they cannot be the result of macro expansion.  Just as
with local import, this decision reflects the desire to make library
dependencies readily apparent from the library header.

\section{Compound library names}

Libary names are compound.  This differs from the treatment of
identifiers in the rest of the language.  Using compound compound
names reflects experience across programming languages that a
structured top-level name space is necessary to avoid collisons.
Embedding a hierachy within a single string or symbol is certainly
possible. However, the editors became convinced by the arguments in
favor of using list data to represent hierarchical structure, rather
than encoding it in a string or symbol.  Versioning, in particular, is
natural in list form, but awkward in encoded form.  Consequently,
despite the syntactic complexity of compound names, and despite the
potential mishandling of the hierarchy by implementations, the editors
chose the list representation.

\section{Versioning}

Libraries and {\cf import} clauses optionally carry versioning
information.  This allows reflecting the development history of a
library, but also significantly increases the complexity of the
library system.  Experience with module systems gathered in other
languages as well as with shared libraries at the operating-system
level consistently indicates that relying only on the name of a module
for identification causes conflicts impossible to rectify in the
absence of versioning information, and thus diminishes the
opportunities for sharing code.  Therefore, versioning is part of the
library system.

\section{Treatment of different versions}

Libraries with the same name but different versions cannot coexist
within the same program.  This prevents combining libraries and
programs that require conflicting versions of the same library.  An
alternative choice would have been to allow the co-existence of
different versions of the same library.  Based on experience with
other shared-library systems (including Windows DLLs and Unix shared
objects), the potential for confusion created by multiple library
instances seems too great. At the same time, it is not clear that this
prior experience applies directly; Windows DLLs are generally not
versioned, and while Linux shared objects are versioned for linking,
inter-object references do not designate a particular source object.
The editors chose the conservative approach disallowing multiple
instances of a library.

\chapter{Top-level programs}

The notion of ``top-level program'' is new in \rn{6} and replaces the
notion of ``Scheme program'' in \rn{5}.  The two are quite different:
While a \rn{6} top-level program is defined to be a complete, textual
entity, an \rn{5} program could evolve by being entered piecemeal into
a running Scheme system.  Many Scheme systems have interactive
command-line environments based on the semantics of \rn{5} programs.
However, the specification of \rn{5} programs was not really
sufficient do describe how to operate an arbitrary Scheme system: The
\rn{5} is ambiguous on some aspects of the semantics such as binding.
Moreover, \rn{5}'s {\cf load} procedure does say how to load source
code into the running system, but the pragmatics of {\cf load} would
often make compiling programs before execution problematic, in
particular with regard to macros.  Furthermore, Scheme implementations
handle treatment of and recovery from errors in wildly different ways.

Tightening the specification of programs from \rn{5} would have been
possible, but could have restricted the design employed by Scheme
implementations in undesirable ways.  Moreover, alternative approaches
to structuring the user interface of a Scheme implementation have
emerged since \rn{5}.  Consequently, \rn{6} makes no attempt of trying
to specify the semantics of programs as in \rn{5}; the design of an
interactive environment is now completely in the hands of the
implementors.  On the other hand, being able to distribute portable
programs as one of the goals of the \rn{6} process.  As a result, the
notion of top-level program was added to the report.

\chapter{Expansion process}

The description of macro expansion is considerable more involved than
it was in \rn{5}: One reason for this that the specification of
expansion in \rn{5} was ambiguous in several important respects.  For
example, \rn{5} does not specify whether {\cf define} is a binding
form.  Also, it was not clear whether definitions of macros had to
precede their uses.  The fact that the set of available bindings may
influence the matching process of macro expansion further complicates
matters.  The specific algorithm \rn{6} describes is one of the
simplest expansion strategies that addresses these questions.  It has
the advantage that it visits every subform of the source code only
once.

The description of the expansion process specifically avoids
specifying the recursive case, where a macro use expands into a
definition whose binding would influence the expansion of the macro
use after the fact, as this might lead to confusing programs.
Implementors are encouraged to detect such cases as syntax violations.

\chapter{Base library}

\section{Library organization}

The libraries of the Scheme standard are organized according to
projected use.  Hence, the \rsixlibrary{base} library exports
procedures and syntactic abstractions that are likely to be useful for
most Scheme programs and libraries.  Conversely, each of the libraries
relegated to the separate report on libraries is likely to be missing
from the imports of a substantial number of programs and libraries.
Naturally, the specific decisions about this organization and the
separation of concerns of the libraries are debatable, and represent a
best attempt of the editors.

A number of secondary criteria were also used in choosing the exports
of the base library.  In particular, macros transformers defined using
the facilities of the base library are guaranteed to be hygienic;
hygiene-breaking transformers are only available through the
\rsixlibrary{syntax-case} library.

Note that \rsixlibrary{base} is not a ``primitive library'' in the
sense that all other libraries of the Scheme standard can be
implemented portably using only its exports.  Moreover, the library
organization is not layered from more primitive to more advanced
libraries, even though some libraries can certainly be implemented in
terms of others.  Organizing the libraries according to such a
layering would have meant imposing unnecessary constraints on
implementations.  Implementations differ in their sets of primitives,
and thus any layering mandated by the report is unlikely to reflect
the layering employed by any specific implementation.  (The
distinction between primitive and derived features was removed from
the report for the same reasons.)

\section{Bodies}

\rn{6} draft treats top-level bodies as a special case.  Allowing
definitions and expressions to be mixed in top-level bodies has ugly
semantics, and introduces a special case, but was allowed as a
concession to convenience when constructing programs rapidly via cut
and paste.

Definitions are not interchangeable with expressions, so definitions
cannot be allowed to appear wherever expressions can appear, and
expressions cannot be allowed to appear wherever definitions can
appear.  Composition of definitions with expressions will therefore be
restricted in some way; the question is not whether there will be
restrictions, but what those restrictions will be.

Historically, top-level definitions in Scheme have had a different
semantics from definitions in bodies.  In a body, definitions serve as
syntactic sugar for the bindings of a {\cf letrec} (or {\cf letrec*}
in \rn{6}) that is implicit at the head of every body.

That semantics can be stretched to cover top-level bodies by
converting expressions to definitions of ignored variables, but does
not easily generalize to allow definitions to be placed anywhere
within expressions.  The editors considered a different generalization
of definition placement, but decided against that generalization in
part because a survey of current Scheme code found surprisingly few
places where the generalization would be useful.

If szch generalization proposed, were adopted, programmers who are
familiar with Java and similar languages might expect definitions to
be allowed in the same kinds of contexts that allow declarations in
Java.  These programmers would have to be reminded that Scheme's
definitions have {\cf letrec*} scope, while Java declarations (inside
a method body) have {\cf let*} scope and cannot be used to define
recursive procedures; that Scheme's begin expressions do not introduce
a new scope, while Java's curly braces do introduce a new scope; that
flow analysis is nontrivial in higher order languages, while Java can
use a trivial flow analysis to reject programs with undefined
variables; that Scheme's macro expander must locate all definitions,
while Java has no macro system; and so on.  Rather than explain how
those facts justify restricting definitions to appear as top-level
forms of a body, it is simpler to explain that definitions are just
syntactic sugar for the bindings of an implicit {\cf letrec*} at the
head of each body, and to explain that the relaxation of that
restriction for top-level bodies is (like several other features of
top-level bodies) an ad hoc special case.

\section{Equivalence predicates}

\subsection{Equivalence of NaNs}

The basic reason why the behavior of {\cf eqv?} is not specified on
NaNs is that the IEEE-754 standard does not say much about how the
bits of a NaN are to be interpreted, and explicitly allows
implementations of that standard to use most of a NaN's bits to encode
implementation-dependent semantics.  The implementors of a Scheme
system should therefore decide how {\cf eqv?} should interpret those
bits.

Arguably, \rn{6} should require
%
\begin{scheme}
(let ((x \hyper{expression})) (eqv? x x))%
\end{scheme}
%
to evaluate to true when \hyper{expression} evaluates to a number;
both \rn{5} and \rn{6} imply this for certain other types, and for
most numbers, but not for NaNs.  Since the IEEE-754 and draft
IEEE-754R~\cite{IEEE754R} both say that the interpretation of a NaN's
payload is left up to implementations, and implementations of Scheme
often do not have much control over the implementation of IEEE
arithmetic, it would be unwise for \rn{6} to insist upon the truth of
%
\begin{scheme}
(let ((x \hyper{expression}))
  (or (not (number? x))
      (eqv? x x)))%
\end{scheme}
even though that expression is likely to evaluate to true in most
systems.  For example, a system with delayed boxing of inexact real
numbers may box the two arguments to {\cf eqv?} separately, the boxing
process may involve a change of precision, and the two separate
changes of precision may result in two different payloads.

When \var{x} and \var{y} are flonums represented in IEEE floating
point or similar, it is reasonable to implement {\cf (eqv? \var{x}
  \var{y})} by a bitwise comparison of the floating point
representations.  \rn{6} should not require this, however, because
%
\begin{enumerate}
\item \rn{6} does not require that flonums be represented by a
  floating point representation,
\item the interpretation of a NaN's payload is explicitly
  implementation-dependent according to both the IEEE-754 standard and
  the current draft of its proposed replacement, IEEE-754R, and
\item the semantics of Scheme should remain independent
  of bit-level representations.
\end{enumerate}
%
For example, IEEE-754, IEEE-754R, and the draft \rn{6} all allow the
external representation {\cf +nan.0} to be read as a NaN whose payload
encodes the input port and position at which {\cf +nan.0} was read.
This is no different from any other external representation such as
{\cf ()}, {\cf \sharpsign()}, or {\cf 324}.  An implementation can
have arbitrarily many bit-level representations of the empty vector,
for example, and some do.  That is why the behavior of the {\cf eq?}
and {\cf eqv?} procedures on vectors cannot be defined by reference to
bit-level representations, and must instead be defined explicitly.

\section{Generic arithmetic}

Numerical computation was long neglected by the Lisp community.  Until
Common Lisp there was no carefully thought out strategy for organizing
numerical computation, and with the exception of the MacLisp system
\cite{Pitman83} little effort was made to execute numerical code
efficiently.  The Scheme reports recognized the excellent work of the
Common Lisp committee and accepted many of their recommendations,
while simplifying and generalizing in some ways consistent with the
purposes of Scheme.

\section{IEEE-754 conformance}

As mentioned in chapter~\ref{numberschapter}, the treatment of
infinities, NaNs and -0.0, if present in a Scheme implementation, are
in line with IEEE-754~\cite{IEEE} and IEEE-754R~\cite{IEEE754R}.
Analogously, the specification of branch cuts for certain
transcendental functions have been changed from \rn{5} to conform to
the IEEE standard.

\section{Domains of numerical predicates}

The domains of the {\cf finite?}, {\cf infinite?}, and {\cf nan?}
procedures could be expanded to include all numbers, or perhaps even
all objects.  However, \rn{6} restricts them to real numbers.
Expanding {\cf nan?} to complex numbers would involve at least some
arbitrariness; not expanding its domain while expanding the domains of
the other two would introduce an irregularity into the domains of
these three procedures, which are likely to be used together; it is
easier for programmers who wish to use these procedures with complex
numbers to express their intent in terms of the real-only versions
than it would be for the editors to guess their intent.

\section{Numerical types}

Scheme's numerical types are the exactness types $\{ \textrm{exact},
\textrm{inexact} \}$, the tower types $\{ \textrm{integer},
\textrm{rational}, \textrm{real}, \textrm{complex}, \textrm{number}
\}$, and the Cartesian product of the exactness types with the tower
types, where $\left< t_1, t2 \right>$; is regarded as a subtype of
both $t_1$ and $t_2$.

These types have an aesthetic symmetry to them, but they are not
equally important.  Judging by the number of \rn{5} procedures whose
domain is restricted to values of some numerical type, the most
important numerical types are the exact integers, the integers, the
rationals, the reals, and the complex numbers.

Many programmers and implementors regard the focus of \rn{5} on those
particular numerical types as something of a mistake.  In practice,
there is reason to believe that the most important numerical types are
the exact integers, the exact rationals, the inexact reals, and the
inexact complex numbers.  This section explores one of the reasons
those four types are so important in practice, and why reals have an
exact zero as their imaginary part in \rn{6} (a change from \rn{5}).

\subsection{Closure Properties}
\label{closurepropertiessection}

Scheme's types are not completely arbitrary.  Each type corresponds to
a set of values that turns up repeatedly as the natural domain or
range of the functions that are computed by Scheme's standard
procedures.  The reason these types turn up so often is that they are
closed under certain sets of operations.

The exact integers, for example, are closed under the integral
operations of addition, subtraction, and multiplication.  The exact
rationals are closed under the rational operations, which consist of
the integral operations plus division (although we must make a special
case for division by zero).  The reals (and inexact reals) are closed
under some (often inexact) interpretation of rational and irrational
operations such as exp and sin, but are not closed under operations
such as {\cf log}, {\cf sqrt}, and {\cf expt}.  The complex (and
inexact complex) numbers are closed under the largest set of
operations.

\subsection{Representation-specific operations}

A naive implementation of Scheme's arithmetic operations is slow
compared to the arithmetic operations of most other languages, mainly
because most operations must perform some sort of case dispatch on the
representation types of their arguments.  The potential for this case
dispatch arises when the type of an operation's argument is
represented by a union of two or more representation types, or because
the operation is required to signal an error when given an argument of
an incorrect type.  (The second reason can be regarded as a special
case of the first.)

To make Scheme's arithmetic more efficient, many implementations
provide sets of operations whose domain is restricted to a single
representation type, and which are not expected to signal an error
when given arguments of incorrect type.

Alternatively, or in addition, several compilers perform some kind of
flow analysis that attempts to infer the representation types of
expressions.  When a single representation type can be inferred for
each argument of an operation, and those types match the types
expected by some representation-specific version of the operation,
then the compiler can substitute the specific version for the more
general version that was specified in the source code.

\subsection{Flow analysis}

Flow analysis is performed by solving the type and interval
constraints that arise from such things as:

\begin{itemize}
\item the types of literal constants, e.g.\ {\cf 2} is an exact integer
  that is known to be within the interval $[2,2]$
  
\item conditional control flow that is predicated on known
  inequalities, e.g. {\cf (if (< i n) \hyperi{expr} \hyperii{expr})}
  
\item conditional control flow that is predicated on known type
  predicates, e.g. {\cf (if (real? x) \hyper{real case} \hyper{unreal case})}
  
\item the closure properties of known operations (for example, {\cf (+
    \vari{flonum} \varii{flonum})} always evaluates to a flonum)
\end{itemize}
  
The purpose of flow analysis (as motivated in this section) is to infer a
single representation type for each argument of an operation.  That
places a premium on predicates and closure properties from which a
single representation type can be inferred.

In practice, the most important single representation types are
fixnum, flonum, and compnum.  (A compnum is a pair of flonums,
representing an inexact complex number.)  These are the representation
types for which a short sequence of machine code can be generated when
the representation type is known, but for which considerably less
efficient code will probably have to be generated when the
representation type cannot be inferred.

The fixnum representation type is not closed under any operations of
\rn{5}, so it is hard for flow analysis to infer the fixnum type from
portable code.  Sometimes the combination of a more general type (e.g.
exact integer) and an interval (e.g.  $[0,n)$, where $n$ is known to
be a fixnum) can imply the fixnum representation type.  Adding
fixnum-specific operations that map fixnums to fixnums (by computing
modulo $2^n$, say) greatly increases the number of fixnum
representation types that a compiler can infer.

The flonum representation type is not closed under operations such as
{\cf sqrt} and {\cf expt}, so flow analysis tends to break down in the
presence of those operations.  This is unfortunate, because those
operations are normally used only with arguments for which the result
is expected to be a flonum.  Adding flonum-specific versions such as
{\cf flsqrt} and {\cf flexpt} improves the effectiveness of flow
analysis.

\rn{5} creates a more insidious problem by defining {\cf (real?
  \var{z})} to be true if and only if {\cf (zero? (imag-part
  \var{z}))} is true.  This means, for example, that {\cf -2.5+0.0i}
is real.  If {\cf -2.5+0.0i} is represented as a compnum, then the
compiler cannot rely on x being a flonum during the \hyper{real case}
of {\cf (if (real? x) \hyper{real case} \hyper{unreal case})}.  This
problem could be fixed by writing all of the arithmetic operations so
that any compnum with a zero imaginary part is converted to a flonum
before it is returned, but that merely creates an analogous problem
for compnum arithmetic, as explained below.  \rn{6} adopted a proposal
by Brad Lucier to fix the problem: {\cf (real? \var{z})} is now true
if and only if {\cf (imag-part \var{z})} is an exact zero.

The compnum representation type is closed under virtually all
operations, provided no operation that accepts two compnums as its
argument ever returns a flonum.  To work around the problem described
in the paragraph above, several implementations automatically convert
compnums with a zero imaginary part to the flonum representation.
This practice virtually destroys the effectiveness of flow analysis
for inferring the compnum representation, so it is not a good
workaround.  To improve the effectiveness of flow analysis, it is
better to change the definition of Scheme's real numbers as described
in the paragraph above.

\section{div and mod}

Given arithmetic on exact integers of arbitrary precision, it is a
trivial matter to derive signed and unsigned integer types of finite
range from it by modular reduction.  For example 32-bit signed
two-complement arithmetic behaves like computing with the residue
classes ``mod $2^{32}$'', where the set $\{-2^{31}, \ldots,
2^{31}-1\}$ has been chosen to represent the residue classes.
Likewise, unsigned 32-bit arithmetic also behaves like computing ``mod
$2^{32}$'', but with a different set of representatives $\{0, \ldots,
2^{32}-1\}$.

Unfortunately, the \rn{5} operations {\cf quotient}, {\cf remainder},
and {\cf modulo} are not ideal for this purpose.  In the following
example, {\cf remainder} fails to transport the additive group
structure of the integers over to the residues modulo 3.
%
\begin{scheme}
(remainder (+ -2 3) 3) \ev 1,
(remainder (+ (remainder -2 3)
              (remainder 3 3))
           3) \ev -2%
\end{scheme}
%
In fact, {\cf modulo} should have been used, producing residues in
$\{0,1,2\}$. For modular reduction with symmetric residues, i.e. in
$\{-1,0,1\}$ in the example, it is necessary to define a more
complicated reduction altogether.

Therefore, {\cf quotient}, {\cf remainder}, and {\cf modulo} have been
replaced in \rn{6} by the {\cf div}, {\cf mod}, {\cf div0}, and {\cf
  mod0} procedures, which are more useful when implementing modular
reduction.  The underlying mathemtical functions $\mathrm{div}$,
$\mathrm{mod}$, $\mathrm{div}_0$, and $\mathrm{mod}_0$ (see report
section~\extref{report:integerdivision}{Integer division}) have been
adapted from the $\mathrm{div}$ and $\mathrm{mod}$ operations from Egner
et al.~\cite{cleaninguptower}.  They differ in the representatives
from the residue classes they return: $\mathrm{div}$ and $\mathrm{mod}$
always compute a non-negative residue, whereas $\mathrm{div}_0$ and
$\mathrm{mod}_0$ compute a residue from a set centered on 0.  The
former can be used, for example, to implement unsigned fixed-width
arithmetic, whereas the latter correspond to two's-complement arithmetic.

These operations differ slightly from the $\mathrm{div}$ and
$\mathrm{mod}$ operations from Egner et al.\ that make both operations
available through a single pair of operations, that distinguish
between the two cases for residues by the sign of the divisor (as well
as returning $0$ for a zero divisor).  Splitting the operations into
two sets of procedures avoids some potential confusion.

The procedures {\cf modulo}, {\cf remainder}, and {\cf quotient} from
\rn{5} can easily be defined in terms of {\cf div} and {\cf mod}.

\section{Characters and strings}

Where \rn{5} specified characters and strings in terms of its own,
limited character set, \rn{6} specifies characters and strings in
terms of Unicode.  The primary goal of the design change were to
improve the portability of Scheme programs that manipulate text, while
preserving a maximum of backward compatibility with \rn{5}.

\rn{6} defines characters to be representations of Unicode scalar
values, and strings to be indexed sequences of characters.  This is a
different representation for Unicode text than the representations
chosen by some other programming languages such as Java or
C\sharpsign{}, which use UTF-16 code units as the basis for the type
of characters.

The representation of Unicode text corresponds to the lowest semantic
level of the Unicode standard: The Unicode standard specifies most
semantic properties in terms of Unicode scalar values.  Thus, Unicode
strings in Scheme allow the straightforward implementation of
semantically sensitive algorithms on strings in terms of these scalar
values.

In contrast, UTF-16 is a specific encoding for Unicode text, and
performing semantic manipulation on UTF-16 representations of text is
awkward.  Choosing UTF-16 as the basis for the string representation
would have meant that a character object potentially carries no
semantic information at all, as surrogates have to be combined
pairwise to yield the corresponding Unicode scalar value.  (As a
result, the APIs of Java for semantic operations on Unicode text often
come in two overloadings, one for character objects and one for
integers that are Unicode scalar values.)

The surrogates cover a numerical range deliberately omitted from the
set of Unicode scalar values.  Hence, surrogates have no
representation as characters---they are merely an artefact of the
design of UTF-16.  Including surrogates in the set of characters
introduces complications similar to the complications of using UTF-16
directly.  In particular, most Unicode consortium standards and
recommendations explicitly prohibit unpaired surrogates, including the
the UTF-8 encoding, the UTF-16 encoding, the UTF-32 encoding, and
recommendations for implementing the ANSI C {\cf wchar\_t} type.  Even
UCS-4, which originally permitted a larger range of values that
includes the surrogate range, has been redefined to match UTF-32
exactly. That is, the original UCS-4 range was shrunk and surrogates
were excluded.

Arguably, a higher-level model for text could be used as the basis for
Scheme's character and string types, such as grapheme clusters.
However, no design satisfying the goals stated above was available
when the report was written.

\section{Multiple values}

FIXME: Why include in the first place

\rn{6} does not specify the semantics of multiple values completely.
In particular, it does not specify what happens when more than one
value (or zero values) are returned to a continuation that implicitly
accepts only one value.  In particular,
%
\begin{scheme}
((lambda (x) x) (values  1 2)) \lev \unspecified%
\end{scheme}
%
Whether an implementation must raise an exception when evaluating such
an expression, or should exhibit some other, non-exceptional behavior
is a contentious issue.  Variations of two different and fundamentally
incompatible positions on this issue exist, each with its own merits:
%
\begin{enumerate}
\item Passing the wrong number of values to a continuation is
typically an error, one that implementations ideally detect and report.

\item There is no such thing as returning the wrong number of values
  to a continuation.  In particular, continuations not created by {\cf
    begin} or {\cf call-with-values} should ignore all but the first
  value, and treat zero values as one unspecified value.
\end{enumerate}
%
\rn{6} allows an implementation to take either position.  Moreover, it
allows an implementation to let {\cf set!}, {\cf vector-set!}, and
other effect-only operators to pass zero values to their
continuations, preventing a program to make obscure use of the return
value.  This causes a potential compatibility problem with \rn{5},
which specifies that such expression return a single unspecified
value, but the benefits of the change were deemed to outweigh the costs.

\section{{\tt call-with-current-continuation} and {\tt dynamic-wind}}

The {\cf call-with-current-continuation} procedure is one of the
distinguishing features of Scheme, and enables the implementation of
many high-level control structures in terms of the base language.  The
{\cf dynamic-wind} procedure was added more recently in \rn{5}.  It
enables implementing a number of abstractions related to
continuations, such as implementing a general dynamic environment, and
making sure that finalization code runs when some dynamic extent
expires.  More generally, the {\cf dynamic-wind} procedure provides a
guarantee that
%
\begin{scheme}
(dynamic-wind \var{before} \var{thunk} \var{after})%
\end{scheme}
%
cannot return unless both \var{before} and \var{after} have been
evaluated, in that order.  They might be evaluated several times, but
always in that order, and these evaluations are never nested.  As this
guarantee is crucial for enabling many of the uses of {\cf
  call-with-current-continuation} and {\cf dynamic-wind}, both are
specified jointly.

\chapter{Unicode}

\section{Case mapping}

The various case-mapping procedures of the \rsixlibrary{unicode}
library all operate in a locale-independent manner.  The Unicode
standard also offers locale-sensitive case operations, not implemented
by the procedures from the \rsixlibrary{unicode} library.  While the
library does not make available the full spectrum of case-related
functionality defined by the Unicode standard, it does provide the
most commonly used procedures.  In particular, this strategy has
allowed providing procedures mostly compatible with those provided by
\rn{5}.  (A minor exception is the case-insensitive procedures for
string comparison.  However, it is unlikely that this affects many
existing programs.)  Providing locale-sensitive operations would have
meant significant novel design effort without significant precedent,
which is why they are not part of \rn{6}.

The case-mapping procedures operating on characters are not sufficient
for implementing case mapping on strings.  For example, the upper-case
version of the German ``\ss{}'' in a string is ``SS''.  As {\cf
  char-upcase} can only return a single character, it must return
\ss{} for \ss.  This limits then usefulness of the procedures
operating on characters, but provides compatibility with \rn{5}
sufficient for many existing applications.  Moroever, it provides
direct access to the corresponding attributes of the Unicode character
database.

\chapter{Bytevectors}

Bytevectors are a representation for binary data, based on
SRFI~74~\cite{srfi74}.  The primary motivation for including them in
\rn{6} was to enable binary I/O.  Positions in bytevectors always
refer to certain bytes or octets.  However, the operations of the
\rsixlibrary{bytevectors} library provide access to binary data in
various byte-aligned formats, such as signed and unsigned integers of
various widths, IEEE floating-point representations, and textual
encodings.  This differs notably from representations for binary data
as homogeneous vectors of numbers.  In settings related to I/O, an
application often needs to access different kinds of entities from a
single binary block.  Providing operations for them on a single
datatype considerably reduces both programming effort and library
size.

Bytevectors can also be used to encode sequences of unboxed numbers.
However, unencapsulated use of bytevectors for this purpose may lead
to aliasing, which may reduce the effectiveness of compiler
optimizations.  However, the records facility provides sealedness and
opacity, which, together with bytevectors, make it possible to
construct a portable reference implementation for new data types that
provide fast and memory-efficient arrays of homogeneous numerical
data.  The design and implementation of those data types is beyond the
scope of the report, but \rn{6} provides the machinery needed to
implement them effectively.

\chapter{List utilities}

The \rsixlibrary{lists} library provides a small number of useful
procedures operating on lists, including several procedures from
\rn{5}.  The goal of the library is to provide only procedures likely
to be useful for many programs.  Consequently, the selection
represented by \rsixlibrary{lists} is less exhaustive than the widely
implemented SRFI~1~\cite{srfi1}.  Several changes were made with
respect to the corresponding procedures SRFI~1 to simplify the
specification, and establish uniform naming conventions.

\chapter{Sorting}

The procedures of the \rsixlibrary{sorting} library provide simple
interfaces to sorting algorithms useful to many programs.  In
particular, {\cf list-sort} and {\cf vector-sort} guarantee stable
sorting using $O(n \lg n)$ calls to the comparison procedure.
Straightforward implementations of merge sort~\cite{algorithms} have
the desired properties.  Note that, at least with merge sort,
stability carries no significant implementation or performance burden.

The choice of ``strictly less than'' for the comparison relation is
consistent with the most common choice of existing Scheme libraries
for sorting.  Moreover, using a boolean comparison procedure instead
one returning three possible values (for less than, equal, and greater
than) would make calling the sorting procedures less convenient, with
no discernable performance advantage.

The specification of the {\cf vector-sort!} procedure is meant to
allow an implementation using {\cf quicksort}, hence the $O(n^2)$
bound on the number of calls to the comparison procedure, and the
omission of the stability requirement.

\chapter{Control structures}

\section{{\tt when} and {\tt unless}}

The {\cf when} and {\cf unless} forms are syntactic sugar for {\cf if}
expressions in one arm.  They provide minor convenience, and are
useful for distinguishing one-armed conditionals from two-armed {\cf
  if} expressions.  Some programmers use {\cf when} and {\cf unless}
to mark conditionals with side effects, avoid one-armed {\cf if}
expressions, and use two-armed {\cf if} expressions primarily for
their values, not their effects.

\section{{\tt case-lambda}}

The {\cf case-lambda} form allows constructing procedures that
distinguish different numbers of arguments.  Using {\cf case-lambda}
makes this considerably easier than deconstructing a list containing
optional arguments explicitly.  Moreover, Scheme implementations might
optimize dispatch on the number of arguments when expressed as {\cf
  case-lambda}, which is considerably harder on code explicitly
deconstructing argument lists.

\chapter{Records}

\section{Syntactic layer}

While the syntactic layer could be expressed portably in terms of the
procedural layer, standardizing a particular surface syntax
facilitates communication via code.

Moreover, the syntactic layer can be implemented more efficiently, and
thus it facilitates the development of efficient portable libraries
that define and use record types.  The syntactic layer is designed to
allow expansion- or compile-time determination of record
characteristics, including field offsets, so that, for example, record
accesses can be be compiled efficiently to simple memory indirects.
This property would be lost if the parent were not generally known
until run time.  Consequently, the {\cf parent} clause only accepts
other record types defined using the syntactic layer.

\section{Constructor mechanism}

Significant demand for the features of the present mechanism,
particularly the general construction mechanism, was expressed during
the review process for SRFI~76~\cite{srfi76}, upon which the present
mechanism is based.

FIXME: more material from the report itself

\section{Sealed record types}

Record types may be sealed.  This feature allows enforcing abstraction
barriers, which is useful in itself, but also allows more efficient
compilation.

In particular, when the implementor of an abstract data type chooses
to represent that ADT by record type, and allows one of the classes
that represent the ADT to be exposed and subclassed, then the ADT is
no longer abstract.  Its implementors must expose enough information
to allow for effective subtyping, and must commit to enough of the
representation to allow those subclasses to continue to work even as
the ADT evolves.

In practice, the ADT cannot evolve freely over time.  Its
representation cannot be changed without breaking most of its
subclasses.  This kind of brittleness has been problematic of many
large programs.  Sealedness allows the construction of record types
that evolve more gracefully over time.

Moreover, making a record type sealed may prevent its accessors and
mutators from becoming polymorphic, making effective flow analysis and
optimization difficult.  This is particularly relevant for
Scheme implementations that use record to implement some of Scheme's
other primitive data types such as pairs.

\section{Opaque record types}

FIXME: I need Kent's or Matthew's help here

\chapter{Conditions and exceptions}

\section{Exceptions}

The goals of the exception mechanism are to help programmers share
code which relies on exception handling, and to provide information on
violations of specifications of procedures and syntactic forms.  This
exception mechanism is en extension of SRFI~34~\cite{srfi34}, which
was primarily designed to meet the first goal.  However, it has proven
suitable for addressing the second goal of dealing with violations as
well.   (More on the second goal below in the discussion of the
condition system.)

For some violations such the use of unsupported NaNs or infinities, as
well as other applications, an exception handler may be able to
repair the cause of the exception, for example by substituting a
suitable object for the NaN or infinity.  Therefore, the exception
mechanism extends SRFI~34 by continuable exceptions, and specifies the
continuation of an exception handler

\section{Conditions}

Conditions are values that communicate information about exceptional
situations between parts of a program. Code that detects an exception
may be in a different part of the program than the code that handles
it. In fact, the former may have been written independently from the
latter.  Consequently, to facilitate effective handling of exceptions,
conditions should communicate as much information as possible as
accurately as possible, and still allow effective handling by code
that did not precisely anticipate the nature of the exception that has
occurred.

The \rsixlibrary{conditions} library provides two mechanisms to
enable this kind of communication:
%
\begin{itemize}
\item subtyping (through record types) among condition types allows
  handling code to determine the general nature of an exception even
  though it does not anticipate its exact nature,
\item compound conditions allow an exceptional situation to be
  described in multiple ways.
\end{itemize}
%
As an example, a networking error that occurs during a file operation
on a remote drive fits two descriptions: A networking error or a
file-system error.  An exception handler might only look for one of
the two.  Compound conditions are a simple solution to this problem.
Moreover, compound conditions also make providing auxiliary
information as part of the condition object, such as an error message,
easier.

The standard condition hierarchy makes an important distinction
between \emph{errors} and \emph{violations}: An error is an
exceptional situation in the environment, which the program cannot
avoid or prevent.  For example, I/O errors are represented by
condition types that are subtypes of {\cf\&error}.  Violations, on the
other hand, are exceptional situations that the program could have
avoided.  Violations are typically programming mistakes.  The
distinction between the two is not always clear, as may be possible
but inordinately difficult or expensive to detect certain violations.
The use of {\cf eval} also blurs the distinction.  Nevertheless, many
cases do allow distinguishing between errors and violations.
Consequently, exception handlers that handle errors are common,
whereas exception handlers that handle violations should be introduced
with great care.


\chapter{I/O}

\section{File names}
\label{filenamesection}

The filenames in most common operating systems, despite their
appearance in most cases, are not text: For example, Unix uses
null-terminated byte sequences, and Windows uses null-terminated
sequences of UTF-16 code units.  On Unix, the textual representation
of a file name depends on the locale, environmental setting.  In both
cases, a file name may be an invalid encoding and thus not correspond
to a string.  An appropriate representation for file names that covers
these cases while still offering convenient access to file-system
names through strings is still an open research problem.  Therefore,
\rn{6} allows specifying file names as strings, but also allows an
implementation to add its own representation for file names.

\section{Binary and textual ports}

The plethora of widely used encodings for texts makes providing
textual I/O significantly more complicated than the simple model
offered by \rn{5}.  In particular, realistic textual I/O should
address encodings such as UTF-16 that include a header word
determining the ``actual'' encoding of the rest of the byte stream,
stateful encodings, and textual formats such as XML, which specify the
encoding in a header line.  Consequently, a library implementing
textual I/O should support specifying an encoding upon opening a port,
but should also support opening a port in ``binary mode'' to determine
the encoding and switch to ``text mode''.

In contrast, arbitrary switching between ``binary mode'' and ``text
mode'' is difficult to support, as it may interfere with efficient
buffering strategies, and because the semantics may be unclear in the
case of stateful encodings.  Consequently, the \rsixlibrary{io ports}
library allows switching from ``binary mode'' to ``text mode'' by
converting a binary port into a textual port, but not the other way
around.  Applications that read from sources that intersperse binary
and textual data should open a binary port and use either {\cf
  bytevector->string} or the procedures from the
\rsixlibrary{bytevectors} library to convert the binary data to text.

The separation of binary and textual ports enables creating ports from
both binary and textual sources and sinks.  It also makes creating
both binary and textual versions of some procedures many procedures
necessary.

\section{Argument conventions}

While the \rsixlibrary{io simple} library provides mostly
\rn{5}-compatible procedures for performing textual I/O, the
\rsixlibrary{io ports} library uses a different convention for
argument ordering.  In particular, the port is always the first
argument.  This enables the use of optional arguments for information
about the data to be read or written, such as the range in a
bytevector.  As this convention is incompatible with the convention of
\rsixlibrary{io simple}, corresponding procedures have different
names.

\chapter{File system}

The \rsixlibrary{files} library provides a minimal set of procedures
useful in many programs: The {\cf file-exists?} procedure allows a
program to detect the presence of a file if it is going to overwrite
it, and {\cf delete-file} allows taking the appropriate action if the
old file is no longer useful.

Standardization of procedures that return or pass to another procedure
the name of a file is more difficult than standardization of {\cf
  file-exists?} and {\cf delete-file}, because strings are either
awkward or in sufficient to represent file names on some platforms,
such as Unix and Windows.  See section~\ref{filenamesection}.

\chapter{Arithmetic}

\section{Fixnums}

FIXME: Mike

\section{Flonums}

FIXME: Mike

\chapter{{\tt syntax-case}}

FIXME

\chapter{Hash tables}

FIXME: Why not SRFI

\chapter{Enumerations}

Many procedures in many libraries accept arguments from a finite set,
or subsets of a finite sets to describe a certain mode of operation,
or several flags to describe a mode of operation.  Examples in the
\rn{6} include the endianness for bytes-object operations, and file
and buffering modes in the I/O library.  Offering a default policy for
dealing with such values fosters portable and readable code, much as
records do for compound values, or multiple values for procedures
computing several values.  Moreover, representations of sets from a
finite set of options should offer the standard set operations, as
they tend to occur in practice.  (One such set operation is the
complement, which makes lists of symbols a less than suitable
representation.)

Different Scheme implementations have taken different approaches to
this problem in the past, which suggests that a default policy does
not merely encode what any sensible programmer would do anyway.  As
possible uses occur quite frequently, this particular aspect of
interface construction has been standardized.


\chapter{Composite library}

FIXME

\chapter{{\tt eval}}

FIXME

\chapter{Mutable pairs}

The presence of mutable pairs causes numerous problems:
%
\begin{itemize}
\item It complicates the specification of higher-procedures that
  operate on lists.
\item It inhibits certain compiler optimizations such as
  deforestation.
\item It complicates reasoning about programs that use lists.
\item It complicates the implementation of procedures that accept
  variable numbers of arguments.  FIXME: Kent, more info?
\end{itemize}
%
However, removing mutable pairs from the language entirely would have
caused significant compatibility problems for existing code.  As a
compromise, the {\cf set-car!} and {\cf set-cdr!} procedures were
moved to a separate library.  This facilitates statically determining
if a program ever mutates pairs, encourages writing programs that do
not mutate pairs, and may help deprecating or removing mutable pairs
in the future.

\chapter{Mutable strings}

The presence of mutable strings problems similar to some of the
problems caused by the presence of mutable pairs.  Hence, the same
reasoning applies for moving the mutation operations into a separate
library.

\chapter{R$^5$RS compatibility}

FIXME: Mike

%%%%%%%%%%%%%%%%%%%%%%%%%%%%%%%%%%%%%%%%%%%%%%%%%%%%%%%%%%%%%%%%

%\newpage                   %  Put bib on it's own page (it's just one)
%\twocolumn[\vspace{-.18in}]%  Last bib item was on a page by itself.
\renewcommand{\bibname}{References}

\bibliographystyle{plain}
\bibliography{abbrevs,rrs}

\vfill\eject


\newcommand{\indexheading}{Alphabetic index of definitions of
  concepts, keywords, and procedures}
\newcommand{\indexintro}{}

\printindex

\end{document}

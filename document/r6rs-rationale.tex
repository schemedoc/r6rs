\documentclass[twoside,twocolumn]{algol60}
%\documentclass[twoside]{algol60}

\pagestyle{headings} 
\showboxdepth=0
% Macros for R^nRS.

\usepackage{makeidx}
\usepackage{url}

% tex2page.sty mucks with in some manner
\let\centerlinesaved=\centerline
\usepackage{tex2page}
\let\centerline=\centerlinesaved

% \let\htmlonly=\iffalse
% \let\endhtmlonly=\fi
% \let\texonly=\iftrue
% \let\endtexonly=\fi

\makeatletter

\texonly
\newcommand{\topnewpage}{\@topnewpage}
\endtexonly

\htmlonly
\newcommand{\topnewpage}[0][]{#1}
\endhtmlonly

\newcommand{\authorsc}[1]{{\scriptsize\scshape #1}}

% Chapters, sections, etc.

\newcommand{\extrapart}[1]{
 % \chapter{#1}
  \chapter*{#1}
  \markboth{#1}{#1}
  \vskip 1ex
  \addcontentsline{toc}{chapter}{#1}}

\newcommand{\clearchaptergroupstar}[1]{
  \texonly
  \clearpage
  \addcontentsline{toc}{chaptergroup}{#1}
  \topnewpage[
    \centerline{\large\bf\uppercase{#1}}
    \bigskip]
    \endtexonly
  }

\newcommand{\clearchapterstar}[1]{
  \clearpage
  \topnewpage[
    \centerline{\large\bf\uppercase{#1}}
    \bigskip]}

\newcommand{\clearextrapart}[1]{
  \clearchapterstar{#1}
  \markboth{#1}{#1}
  \addcontentsline{toc}{chapter}{#1}}

\newcommand{\vest}{}
\newcommand{\dotsfoo}{$\ldots\,$}

\newcommand{\sharpfoo}[1]{{\tt\##1}}
\newcommand{\schfalse}{\sharpfoo{f}}
\newcommand{\schtrue}{\sharpfoo{t}}

\newcommand{\ampfoo}[1]{{\tt\&#1}}

\newcommand{\libfoo}[1]{{\tt(#1)}}

\newcommand{\singlequote}{{\tt'}}  %\char19
\newcommand{\doublequote}{{\tt"}}
\newcommand{\backquote}{{\tt\char18}}
\newcommand{\backwhack}{{\tt\char`\\}}
\newcommand{\comma}{{\tt\char`\@}}
\newcommand{\atsign}{{\tt\char`\@}}
\newcommand{\bang}{{\tt\char`\!}}
\newcommand{\sharpsign}{{\tt\#}}
\newcommand{\verticalbar}{{\tt|}}
\newcommand{\openbracket}{{\tt\char`\[}}
\newcommand{\closedbracket}{{\tt\char`\]}}
\newcommand{\ampersand}{{\tt\char`\&}}

\newcommand{\coerce}{\discretionary{->}{}{->}}

% Knuth's \in sucks big boulders
\def\elem{\hbox{\raise.13ex\hbox{$\scriptstyle\in$}}}

\newcommand{\meta}[1]{{\noindent\hbox{\rm$\langle$#1$\rangle$}}}
\let\hyper=\meta
\newcommand{\hyperi}[1]{\hyper{#1$_1$}}
\newcommand{\hyperii}[1]{\hyper{#1$_2$}}
\newcommand{\hyperiii}[1]{\hyper{#1$_3$}}
\newcommand{\hyperj}[1]{\hyper{#1$_i$}}
\newcommand{\hypern}[1]{\hyper{#1$_n$}}
\texonly
\newcommand{\var}[1]{\noindent\hbox{\textnormal{\textit{#1}}}}
\endtexonly
\htmlonly
\newcommand{\var}[1]{\textnormal{\textit{#1}}}
\endhtmlonly
\newcommand{\vari}[1]{\var{#1$_1$}}
\newcommand{\varii}[1]{\var{#1$_2$}}
\newcommand{\variii}[1]{\var{#1$_3$}}
\newcommand{\variv}[1]{\var{#1$_4$}}
\newcommand{\varj}[1]{\var{#1$_j$}}
\newcommand{\varn}[1]{\var{#1$_n$}}

\newcommand{\vr}[1]{{\noindent\hbox{$#1$\/}}}  % Careful, is \/ always the right thing?
\newcommand{\vri}[1]{\vr{#1_1}}
\newcommand{\vrii}[1]{\vr{#1_2}}
\newcommand{\vriii}[1]{\vr{#1_3}}
\newcommand{\vriv}[1]{\vr{#1_4}}
\newcommand{\vrv}[1]{\vr{#1_5}}
\newcommand{\vrj}[1]{\vr{#1_j}}
\newcommand{\vrn}[1]{\vr{#1_n}}

%%R4%% The excessive use of the code font in the numbers section was
% confusing, somewhat obnoxious, and inconsistent with the rest
% of the report and with parts of the section itself.  I added
% a \tupe no-op, and changed most old uses of \type to \tupe,
% to make it easier to change the fonts back if people object
% to the change.

\newcommand{\type}[1]{{\it#1}}
\newcommand{\tupe}[1]{{#1}}

\newcommand{\defining}[1]{\mainindex{#1}{\em #1}}
\newcommand{\ide}[1]{{\schindex{#1}\frenchspacing\tt{#1}}}

\newcommand{\lambdaexp}{{\cf lambda} expression}

\newcommand{\callcc}{{\tt call-with-current-continuation}}

\newcommand{\mainschindex}[1]{\label{#1}\index{#1@\texttt{#1}}}
\newcommand{\mainindex}[1]{\index{#1}}
\newcommand{\schindex}[1]{\index{#1@\texttt{#1}}}
\newcommand{\sharpindex}[1]{\index{#1@\texttt{\#{}#1}}}
\newcommand{\ampindex}[1]{\index{#1@\texttt{\&{}#1}}}
\newcommand{\libindex}[1]{\index{#1@\texttt{(#1)}}}

\newcommand{\extref}[2]{\texonly\ref{#1}\endtexonly\htmlonly{on ``#2''}\endhtmlonly}

\renewenvironment{theindex}
{\clearpage
\topnewpage[
    \begin{center}
      \large\bf\MakeUppercase{\indexheading}
    \end{center}
    \vskip 1ex \bigskip]
    \markboth{Index}{Index}
    \addcontentsline{toc}{chapter}{\indexheading}
    \parindent\z@
    \texonly\parskip\z@ plus .1pt\endtexonly\relax\let\item\@idxitem
    \indexintro\par\bigskip}
               {\texonly\clearpage\endtexonly}


\newcommand{\domain}[1]{#1}
\newcommand{\nodomain}[1]{}
%\newcommand{\todo}[1]{{\rm$[\![$!!~#1$]\!]$}}
\newcommand{\todo}[1]{}

% \frobq will make quote and backquote look nicer.
\def\frobqcats{%\catcode`\'=13
\catcode`\`=13{}}
{\frobqcats
\gdef\frobqdefs{%\def'{\singlequote}
\def`{\backquote}}}
\def\frobq{\frobqcats\frobqdefs}

% \cf = code font
% Unfortunately, \cf \cf won't work at all, so don't even attempt to
% next constructions which use them...
\newcommand{\cf}{\frenchspacing\frobq\tt}

\texonly
% Same as \obeycr, but doesn't do a \@gobblecr.
{\catcode`\^^M=13 \gdef\myobeycr{\catcode`\^^M=13 \def^^M{\\}}%
\gdef\restorecr{\catcode`\^^M=5 }}
\endtexonly

{\obeyspaces\gdef {\hbox{\hskip0.5em}}}

\gdef\gobblecr{\@gobblecr}

\def\setupcode{\@makeother\^}

% Scheme example environment
% At 11 points, one column, these are about 56 characters wide.
% That's 32 characters to the left of the => and about 20 to the right.

\newcommand{\exception}[1]{{\cf#1} \textnormal{\textit{exception}}}
\newenvironment{schemenoindent}{
  % Commands for scheme examples
  \newcommand{\ev}{\>\>\evalsto}
  \newcommand{\lev}{\\\>\evalsto}
  \newcommand{\unspecified}{{\em{}unspecified}}
  \newcommand{\theunspecified}{{\em{}the unspecified value}}
  \setupcode
  \small \cf \obeyspaces \myobeycr
  \begin{tabbing}%
\qquad\=\hspace*{5em}\=\hspace*{9em}\=\evalsto~\=\kill%   was 16em
\gobblecr}{\unskip\end{tabbing}}

%\newenvironment{scheme}{\begin{schemenoindent}\+\kill}{\end{schemenoindent}}
\newenvironment{scheme}{
  % Commands for scheme examples
  \newcommand{\ev}{\>\>\evalsto}
  \newcommand{\lev}{\\\>\evalsto}
  \renewcommand{\em}{\rmfamily\itshape}
  \newcommand{\unspecified}{{\em{}unspecified}}
  \newcommand{\theunspecified}{{\em{}the unspecified value}}
  \setupcode
  \small \cf \obeyspaces \myobeycr
  \begin{tabbing}%
\qquad\=\hspace*{5em}\=\hspace*{9em}\=\evalsto~\=\+\kill%   was 16em
\gobblecr}{\unskip\end{tabbing}}

\newcommand{\evalsto}{$\Longrightarrow$}

% Rationale

\newenvironment{rationale}{%
\bgroup\small\noindent{\em Rationale:}\space}{%
\egroup}

% Notes

\newenvironment{note}{%
\bgroup\small\noindent{\em Note:}\space}{%
\egroup}

% Names of library modules

\newcommand{\library}[1]{{\tt (#1)}}
\newcommand{\deflibrary}[1]{\library{#1}\libindex{#1}}

% Manual entries

\newenvironment{entry}[1]{
  \vspace{3.1ex plus .5ex minus .3ex}\noindent#1%
\unpenalty\nopagebreak}{\vspace{0ex plus 1ex minus 1ex}}

\newcommand{\exprtype}{syntax}

\newcommand{\unspecifiedreturn}{the unspecified value}

% Primitive prototype
\newcommand{\pproto}[2]{\unskip%
\hbox{\cf\spaceskip=0.5em#1}\hfill\penalty 0%
\hbox{ }\nobreak\hfill\hbox{\rm #2}\break}

% Parenthesized prototype
\newcommand{\proto}[3]{\pproto{(\mainschindex{#1}\hbox{#1}{\it#2\/})}{#3}}

% Variable prototype
\newcommand{\vproto}[2]{\mainschindex{#1}\pproto{#1}{#2}}

% Condition-type prototype
 \newcommand{\ctproto}[1]{\ampindex{#1}\pproto{\ampfoo{#1}}{condition type}}

% Extending an existing definition (\proto without the index entry)
\newcommand{\rproto}[3]{\pproto{(\hbox{#1}{\it#2\/})}{#3}}

% Extending an existing definition, with index entry
\newcommand{\irproto}[3]{\schindex{#1}\rproto{#1}{#2}{#3}}

% Variable prototype
\newcommand{\rvproto}[2]{\pproto{#1}{#2}}

% Grammar environment

\newenvironment{grammar}{
  \def\:{\goesto{}}
  \def\|{$\vert$}
  \cf \myobeycr
  \begin{tabbing}
    %\qquad\quad \= 
    \qquad \= $\vert$ \= \kill
  }{\unskip\end{tabbing}}

%\newcommand{\unsection}{\unskip}
\newcommand{\unsection}{{\vskip -2ex}}

% Commands for grammars
\newcommand{\arbno}[1]{#1\hbox{\rm*}}  
\newcommand{\atleastone}[1]{#1\hbox{$^+$}}

\newcommand{\goesto}{$\longrightarrow$}

\newcommand{\syntax}{{\em Syntax: }}
\newcommand{\semantics}{{\em Semantics: }}
\newcommand{\implresp}{{\em Implementation responsibilities: }}

\newcommand{\rrs}[1]{\textit{Revised$^#1$ Report on the Algorithmic Language Scheme}}

\newcommand{\libindexentry}[1]{#1 (library)}

\makeatother


\newcommand{\rn}[1]{R$^{#1}$RS}
\newcommand{\rsix}{\rn{6}}

\texonly
\externaldocument[report:]{r6rs}
\externaldocument[lib:]{r6rs-lib}
\externaldocument[app:]{r6rs-app}
\endtexonly

\def\headertitle{Revised$^{\rnrsrevision}$ Scheme Rationale}
\def\TZPtitle{Revised^\rnrsrevision{} Report on the Algorithmic Language Scheme - Non-Normative Appendices -}

\begin{document}

\thispagestyle{empty}

\topnewpage[{
\begin{center}   {\huge\bf
        Revised{\Huge$^{\mathbf{\htmlonly\tiny\endhtmlonly{}\rnrsrevision}}$} Report on the Algorithmic Language \\
                              \vskip 3pt
                              Scheme\\
                                \vskip 1.5ex
                              --- Rationale ---}

\vskip 1ex$$
\begin{tabular}{l@{\extracolsep{.5in}}lll}
\multicolumn{4}{c}{M\authorsc{ICHAEL} S\authorsc{PERBER}}
\\
\multicolumn{4}{c}{R.\ K\authorsc{ENT} D\authorsc{YBVIG},
  M\authorsc{ATTHEW} F\authorsc{LATT},
  A\authorsc{NTON} \authorsc{VAN} S\authorsc{TRAATEN}}
\\
\multicolumn{4}{c}{(\textit{Editors})} \\
\multicolumn{4}{c}{
  R\authorsc{ICHARD} K\authorsc{ELSEY}, W\authorsc{ILLIAM} C\authorsc{LINGER},
  J\authorsc{ONATHAN} R\authorsc{EES}} \\
\multicolumn{4}{c}{(\textit{Editors, Revised$^5$ Report on the
    Algorithmic Language Scheme})} \\
\multicolumn{4}{c}{\bf \rnrsrevisiondate}
\end{tabular}
$$



\end{center}

\chapter*{Summary}
\medskip

This document describes rationales for some of the design decisions
behind the \textit{Revised$^6$ Report on the Algorithmic Language
  Scheme}.  The focus is on changes made since the last revision on
the report.  Moreover, numerous fundamental design decisions of Scheme
are also explained.  This report also contains some historical notes.
The formal comments submitted for drafts of the report and their
responses, as archived on \url{http://www.r6rs.org/} provides more
valuable background information on many decisions that are reflected
in the report.

Note that this document rationale is not intended to be an exhaustive
justification for every decision and design aspect of the report.
Instead, it provides information about some of the arguments
considered by the editors' committee when decisions were made, as
background information and as guidelines for future makers of
decisions.  As such, the arguments given here may not be convincing to
every reader, but they convinced the editors at the time the
respective decisions were made.

This document frequently refers back to the \textit{Revised$^6$ Report
  on the Algorithmic Language Scheme}~\cite{R6RS}, the
\textit{Revised$^6$ Report on the Algorithmic Language Scheme ---
  Libraries ---}~\cite{R6RS-libraries}, and the \textit{Revised$^6$
  Report on the Algorithmic Language Scheme --- Non-Normative
  Appendices ---}~\cite{R6RS-appendices}; specific references to the
report are identified by designations such as ``report section'' or
``report chapter'', references to the library report are identified by
designations such as ``library section'' or ``library chapter'', and
references to the appendices are identified by designations such as
``appendix'' or ``appendix section''.  This document frequently refers
to the whole \textit{Revised$^6$ Report on the Algorithmic Language
  Scheme} as ``\rn{6}'', and to the \textit{Revised$^5$ Report
  on the Algorithmic Language Scheme} as ``\rn{5}''.

\medskip

We intend this report to belong to the entire Scheme community, and so
we grant permission to copy it in whole or in part without fee.  In
particular, we encourage implementors of Scheme to use this report as
a starting point for manuals and other documentation, modifying it as
necessary.

\bigskip

% We're done.
}]

\texonly\clearpage\endtexonly

\chapter*{Contents}
\addvspace{3.5pt}                  % don't shrink this gap
\renewcommand{\tocshrink}{-4.0pt}  % value determined experimentally
{%\footnotesize
\tableofcontents
}

\vfill

\texonly\clearpage\endtexonly

%%%%%%%%%%%%%%%%%%%%%%%%%%%%%%%%%%%%%%%%%%%%%%%%%%%%%%%%%%%%%%%%

\chapter{Historical background}

The \textit{Revised$^6$ Report on the Algorithmic Language Scheme}
(\rn{6} for short) is the sixth of the Revised Reports on Scheme.

\vest The first description of Scheme was written by Gerald Jay
Sussman and Guy Lewis Steele Jr.\ in
1975~\cite{Scheme75}.  A revised report by Steele and
Sussman~\cite{Scheme78}
appeared in 1978 and described the evolution
of the language as its MIT implementation was upgraded to support an
innovative compiler~\cite{Rabbit}.  Three distinct projects began in
1981 and 1982 to use variants of Scheme for courses at MIT, Yale, and
Indiana University~\cite{Rees82,MITScheme,Scheme311}\nocite{Scheme84}.  An introductory
computer science textbook using Scheme was published in
1984~\cite{SICP}.  A number of textbooks describing and using Scheme
have been published since~\cite{tspl3}.

\vest As Scheme became more widespread,
local dialects began to diverge until students and researchers
occasionally found it difficult to understand code written at other
sites.
Fifteen representatives of the major implementations of Scheme therefore
met in October 1984 to work toward a better and more widely accepted
standard for Scheme.
Participating in this workshop were Hal Abelson, Norman Adams, David
Bartley, Gary Brooks, William Clinger, Daniel Friedman, Robert Halstead,
Chris Hanson, Christopher Haynes, Eugene Kohlbecker, Don Oxley, Jonathan Rees,
Guillermo Rozas, Gerald Jay Sussman, and Mitchell Wand.
Their report~\cite{RRRS}, edited by Will Clinger,
was published at MIT and Indiana University in the summer of 1985.
Further revision took place in the spring of 1986~\cite{R3RS} (edited
by Jonathan Rees and Will Clinger),
and in the spring of 1988~\cite{R4RS} (also edited by Will Clinger and
Jonathan Rees).  Another revision published in 1998, edited
by Richard Kelsey, Will Clinger and Jonathan Rees,
reflected further revisions agreed upon in a meeting at Xerox PARC in
June 1992~\cite{R5RS}.

Attendees of the Scheme Workshop in Pittsburgh in October 2002 formed
a Strategy Committee to discuss a process for producing new revisions
of the report.  The strategy committee drafted a charter for Scheme
standardization.  This charter, together with a process for selecting
editorial committees for producing new revisions for the report, was
confirmed by the attendees of the Scheme Workshop in Boston in
November 2003.  Subsequently, a Steering Committee according to the
charter was selected, consisting of Alan Bawden, Guy L.\ Steele Jr.,
and Mitch Wand.  An editors' committee charged with producing this report was
also formed at the end of 2003, consisting of Will Clinger,
R.\ Kent Dybvig, Marc Feeley, Matthew Flatt, Richard Kelsey, Manuel
Serrano, and Mike Sperber, with Marc Feeley acting as Editor-in-Chief.
Richard Kelsey resigned from the committee in April 2005, and was
replaced by Anton van Straaten.  
Marc Feeley and Manuel Serrano
resigned from the committee in January 2006.  Subsequently, the charter
was revised to reduce the size of the editors' committee to five and
to replace the office of Editor-in-Chief by a Chair and a Project
Editor~\cite{SchemeCharter2006}.  R.\ Kent Dybvig served as Chair, and
Mike Sperber served as Project Editor.  Will Clinger resigned from the
committee in May 2007.
Parts of the report were posted as Scheme Requests for Implementation
(SRFIs, see \url{http://srfi.schemers.org/})
and discussed by the community before being revised and finalized for
the report~\cite{srfi75,srfi76,srfi77,srfi83,srfi93}.
Jacob Matthews and Robby
Findler wrote the operational semantics for the language core.


\chapter{Requirement levels}

\rn{6} distinguishes between different requirement levels, both for
the programmer and for the implementation.  Specifically, the
distinction between ``should'' and ``must'' is important: For example,
``should'' is used for restrictions on argument types that are
undecidable or potentially too expensive to enforce.  The use of
``should'' allows implementations to perform quite extensive checking
of restrictions on arguments (see section~\ref{argumentchecking}), but
also to eschew more expensive checks.


\chapter{Numbers}
\label{numberschapter}

\section{Infinities, NaNs}
\label{infinitiesnansection}

Infinities and NaNs and are artefacts that help deal with the
inexactness of binary floating-point arithmetic.  The semantics
dealing with infinities and NaNs, or the circumstances leading to
their generation are somewhat arbitrary.  However, as most Scheme
implementation use an IEEE~754-conformant implementation~\cite{IEEE}
of flonums, \rn{6} uses the particular semantics from this standard as
the basis for the treatment of infinities and NaNs in the report.
This is also the reason why infinities and NaNs are flonums and thus
inexact real number objects, allowing Scheme systems to exploit the closure
properties arising from their being part of the standard IEEE-754
floating-point representation.  See
section~\ref{closurepropertiessection} for details on closure
properties.

\rn{6} intentionally does not require a Scheme implementation to use
infinities and NaNs as specified in IEEE~754.  Hence, support for them
is optional.

\section{Distinguished -0.0}
 
A distinguished -0.0 is another artefact of IEEE~754, which can be
used to construct certain branch cuts.  A Scheme implementation is not
required to distinguish -0.0.  If it does, however, the behavior of
the transcendental functions is sensitive to the distinction.  

\chapter{Lexical syntax and datum syntax}

\section{Symbol and identifier syntax}

\subsection{Escaped symbol constituents}

While revising the syntax of symbols and identifiers, the editors'
goal was to make symbols subject to write/read invariance, i.e.\ to
allow each symbol to be written out using {\cf put-datum} (section
\extref{lib:put-datum}{Textual output}) or {\cf write}
(section~\extref{lib:write}{Simple I/O}), and
read back in using {\cf get-datum} (section
\extref{lib:get-datum}{Textual input}) or
{\cf read} (section \extref{lib:read}{Simple I/O})), yielding the same symbol.  This
was not the case in \rrs{5}, as symbols could contain arbitrary
characters such as spaces which could not be part of their external
representation.  Moreover, symbols could distinguish case, whereas
their external representation could not.

For representing unusual characters in the symbol syntax, the report
provides the {\cf\backwhack{}x} escape syntax, which allows specifying
an arbitrary Unicode scalar value.  This also has the advantage that
arbitrary symbols can be represented using only ASCII, which allows
referencing them from Scheme programs restricted to ASCII or some
other subset of Unicode.

Among existing implementations of Scheme , a popular choice for
extending the set of characters that can occur in symbols is the
vertical-bar syntax of Common Lisp.  The vertical-bar syntax of Common
Lisp carries the risk of confusing the syntax of identifiers with that
of consecutive lexemes, and also does not allow representing arbitrary
characters only using ASCII.  Consequently, it was not adopted for
\rn{6}.

\subsection{Case sensitivity}
\label{casesensitivityrationalesection}

The change from case-insensitive syntax in \rn{5} to case-sensitive
syntax is a major change.  Many technical arguments exist in favor of
both case sensitivity and case insensitivity, and any attempt to list
them all here would be incomplete.  
Switching to case sensitivity will break backwards compatibility, and
might set a precedent for switching a technically more or less
arbitrary decision again in the future.

The editors decided to switch to case sensitivity because they
perceived that a significant majority of the Scheme community favored
the change.  This perception has been strengthened by polls at the
2004 Scheme workshop, on the {\cf plt-scheme} mailing list, and the
{\cf r6rs-discuss} mailing list.

The directives described in
appendix~\extref{app:caseinsensitivityappendix}{Optional case
  insensitivity} allow specifying that a code portion (or other
syntactic data) was written under the old assumption of
case-insensitivity and therefore must be case-folded upon reading.

\subsection{Identifiers starting with {\tt ->}}

\rn{6} introduces a special rule in the lexical syntax for
identifiers starting with the characters {\cf ->}.  In \rn{5}, such
identifiers were not valid lexemes.  (In \rn{5}, a lexeme starting
with a {\cf -} character---except for {\cf -} itself---has to be a
representation of a number object.)
However, many existing
Scheme implementations prior to \rn{6} already supported identifiers
starting with {\cf ->}.  (Many readers would classify any lexeme as an
identifier starting with {\cf -} for which {\cf string->number}
returns \schfalse{}.)  As a result, a significant amount of otherwise
portable Scheme code used identifiers starting with {\cf ->}, which
are a convenient choice for many names.  Therefore, \rn{6} legalizes
these identifiers.  The separate production in the grammar is not particularly elegant.
However, designing a more elegant production that does not overlap with
representations of number objects or other lexeme classes has proved to be surprisingly
difficult.

 
\section{Comments}

While \rn{5} only provided the {\cf;} syntax for comments, the report
now describes three essential kinds: In addition to {\cf;}, {\tt \#|}
and {\tt |\#} delimit block comments, and {\cf\sharpsign;} starts a
``datum comment''.  ({\cf\sharpsign!r6rs} is also a kind of command,
albeit with a specific, fixed purpose.) 

Block comments provide a more convenient way of writing multi-line
comments, and are an often-requested and often-implemented syntactic
addition to the language. 

The rationale for {\cf\sharpsign;} is not as readily apparent: It
automatically comments out a single datum, the basic unit of Scheme
syntax, something that the other comment mechanisms cannot do.  {\cf
  \sharpsign\verticalbar} \ldots {\cf \verticalbar\sharpsign} cannot
generally be used to comment out an arbitrary datum or set of data.
Moreover, while {\cf\sharpsign;} is probably most useful during
development and debugging, it is still useful to have a standard
notation for commenting out a datum, particularly since programmers
sometimes develop and debug a single piece of code concurrently on
multiple systems.

\section{Future extensions}

The {\cf\sharpsign} is the prefix of several different kinds of
syntactic data: vectors, bytevectors, syntactic abbreviations related
to quasiquotes and syntax construction, nested comments, characters,
{\cf\sharpsign!r6rs}, and implementation-specific extensions to the
syntax that start with {\cf\sharpsign!}.  In each case, the character
following the {\cf\sharpsign} identifies the class of syntactic datum.
In the case of bytevectors, the syntax anticipates several different
kinds of homogeneous vectors, even though \rn{6} specifies only
one. The {\cf u8} after the {\cf\sharpsign{}v} identifies the
components of the vector as unsigned 8-bit entities or octets.


\chapter{Semantic concepts}

\section{Argument and subform checking}
\label{argumentchecking}

The report requires implementations to check the arguments of
procedures and subforms for syntactic forms for adherence to the
specification.  However, implementations are not required to detect
every violation of a specification.  Specifically, the report allows
the following exceptions:
%
\begin{enumerate}
\item Some restrictions are undecidable.  Hence, checking is not
  required, such as certain properties of procedures passed as
  arguments, or properties of subexpressions, whose macro expansion
  may not terminate.
\item Checking that an argument is a list where doing so would be
  impractical or expensive is not required.  Specifically, procedures that
  invoke another procedure passed as an argument are not required to
  check that a list remains a list after every invocation.
\item With some procedures, future extensions to the arguments they
  accept are explicitly allowed.
\end{enumerate}
%
The second item deserves special attention, as the specific decisions
made for the report are meant to enable ``picky'' implementations that
catch as many violations and unportable assumptions made by programs
as possible, while also enabling practical and efficient
implementations that execute programs quickly.

\section{Safety}

\rn{5} described many situations not specified in the report as ``is
an error'': Portable programs could not cause such situations, but
implementations were free to implement arbitrary behavior under this
umbrella.  This included ``crashing'' the running program, or
somehow compromising the integrity of its execution model, resulting
in random behavior.  This was in sharp contrast to the assumption to
many users that assumed Scheme was a ``safe'' language where each
violation of a restriction of the language standard or the
implementation would at least result in defined behavior, such as
interrupting or aborting the program and/or starting a debugger.

To avoid the problems associated with this arbitrary behavior, all
libraries specified in the report must be safe, and react to
detected violations of the specification by raising
an exception, which allows the program to detect and react
to the violation itself.

The report allows implementations to provide ``unsafe'' libraries that
may compromise safety.

\section{Proper tail recursion}

Intuitively, no space is needed for an active tail call because the
continuation that is used in the tail call has the same semantics as the
continuation passed to the procedure containing the call.  Although an improper
implementation might use a new continuation in the call, a return
to this new continuation would be followed immediately by a return
to the continuation passed to the procedure.  A properly tail-recursive
implementation returns to that continuation directly.

Proper tail recursion was one of the central ideas in Steele and
Sussman's original version of Scheme.  Their first Scheme interpreter
implemented both functions and actors.  Control flow was expressed using
actors, which differed from functions in that they passed their results
on to another actor instead of returning to a caller.  In the terminology
of the report, each actor finished with a tail call to another actor.

Steele and Sussman later observed that in their interpreter the code
for dealing with actors was identical to that for functions and thus
there was no need to include both in the language.

While a proper tail recursion has been a cornerstone property of
Scheme since its inception, it is difficult to implement efficiently
on some architectures, specifically those compiling to higher-level
intermediate languages such as C or to certain virtual-machine
architectures such as JVM or CIL.

Nevertheless, abandoning proper tail recursion as a language property
and relegating it to optional optimizations would have far-reaching
consequences: Many programs written with the assumption of proper tail
recursion would no longer work.  Moreover, the lack of proper tail
recursion would prevent the natural expression of certain programming
styles such as Actors-style message-passing systems, self-replacing
servers, or automata written as mutually recursive procedures.
Furthermore, if they did not exist, special ``loop'' constructs would
have to be added to the language to compensate for the lack of a general
iteration construct.  Consequently, proper tail recursion remains an
essential aspect of the Scheme language.

\chapter{Entry format}

While it is reasonable to require the programmer to adhere to
restrictions on arguments, some of these restrictions are either
undecidable or too expensive to always enforce (see
section~\ref{argumentchecking}).  Therefore, some entries have
an additional paragraph labelled ``\textit{Implementation
  responsibilities}'' that distinguishes the responsibilities of the
implementation from those of the implementation.

\chapter{Libraries}

The design of the library system was a challenging process: Many
existing Scheme implementations offer ``module systems'', but they
differ dramatically both in functionality and in the goals they
address.  The library system was designed with the primary
requirement of allowing programmers to write, distribute, and evolve
portable code.  A secondary requirement was to be able to separately
compile libraries in the sense that compiling a library requires only
having compiled its dependencies.  This entailed the following
corollary requirements:
%
\begin{itemize}
\item Composing libraries requires management of dependencies.
\item Libraries from different sources may have name conflicts.
  Consequently, name-space management is needed.
\item In Scheme, abstractions provided by code includes macros.
  Hence, it should be possible to export macros from libraries, with all
  the consequences dictated by the referential-transparency property
  of hygienic macros.
\end{itemize}
%
The library system does not address the following goals, which were
considered during the design process:
%
\begin{itemize}
\item independent compilation
\item mutually dependent libraries
\item separation of library interface from library implementation
\item local modules and local imports
\end{itemize}
%
This section discusses some aspects of the design of the library
system that have been controversial:

\section{Syntax}

A library definition is a single form, rather than a sequence of forms
where some forms are some kind of header and the remaining forms
contain the actual code.
It is not clear that a sequence of forms is more convenient than a
single form for processing and generation. Both syntactic choices have
technical merits and drawbacks. The single-form syntax chosen for
\rn{6} has the advantage of being self-delimiting.

A difference between top-level programs and libraries is that a
program contains only one top-level program but multiple libraries.
Thus, delimiting the text for a library body will be a
a common need (in streams of various kinds) that it is worth
standardizing the delimiters; parentheses are the obvious choice.

\section{Local import}

Some Scheme implementations feature module systems that allow
importing a module's bindings into a local environment.  While local
import has several benefits, a drawback of local
imports is that library dependencies are not readily apparent from a
library header. This drawback seems significant enough to merit
longer-term consideration. Since leaving out local import for now does
not preclude it from being added later, and since the report already
tackles so many library-related issues, the editors decided to omit
local import from the report.

\section{Local modules}

Some Scheme implementations feature module systems that allow defining
``local modules'' in a local environment rather than only at the
library level.  Specifically, this enables syntactic abstraction over
modules.  Local modules are useful, but also complicate
the scoping rules of the language: Whereas the library system only
allows transporting bindings from one library top level to another,
local modules allow transporting bindings into local scopes, with
far-reaching consequences for the rules that may be used to determine
scoping, particularly in the absence of {\cf syntax-case}.

Moreover it is debatable whether a top-level library should
be the same construct as a local module. The job of the library
system is to organize the top-level namespace, and not to serve as a
target for expansion of sophisticated macros. Of course, the broader
roles and syntactic similarity of {\cf module} and {\cf library}
suggest merging the concepts, but merging the concepts further
broadens the role of each.  Such generalization may seem intuitively
right to Scheme programmers, but all attempts by the editors at such
broadening led away from consensus rather than toward it.

\section{Fixed {\tt import} and {\tt export} clauses}

The {\cf import} and {\cf export} clauses of the {\cf library} form
are fixed---they cannot be the result of macro expansion.  Just as
with local import, this decision reflects the desire to make library
dependencies readily apparent from the library header.

\section{Instantiation and initialization}

Opinions vary on how libraries should be instantiated and
initialized during the expansion and execution of library bodies,
whether library instances should be distinguished across phases,
and whether levels should be declared so that they constrain 
identifier uses to particular phases. This report therefore leaves
considerable latitude to implementations, while attempting to
provide enough guarantees to make portable libraries feasible.

\section{Immutable exports}

The asymmetry in the prohibitions against assignments to explicitly
and implicitly exported variables reflects the fact that the violation
can be determined for implicitly exported variables only when the
importing library is expanded.

\section{Compound library names}

Library names are compound.  This differs from the treatment of
identifiers in the rest of the language.  Using compound names
reflects experience across programming languages that a structured
top-level name space is necessary to avoid collisions.  Embedding a
hierarchy within a single string or symbol is certainly possible.
However, in Scheme, list data is the natural means for represent
hierarchical structure, rather than encoding it in a string or symbol.
The hierarchical structure makes it easy formulating policies for
choosing unique names, or possible storage formats in a file system.
See appendices~\extref{app:librarynamesappendix}{Unique library names}
and \extref{app:filesystemmappingappendix}{Library / file system
  mapping} Consequently, despite the syntactic complexity of compound
names, and despite the potential mishandling of the hierarchy by
implementations, the editors chose the list representation.

\section{Versioning}

Libraries and {\cf import} clauses optionally carry versioning
information.  This allows reflecting the development history of a
library, but also significantly increases the complexity of the
library system.  Experience with module systems gathered in other
languages as well as with shared libraries at the operating-system
level consistently indicates that relying only on the name of a module
for identification causes conflicts impossible to rectify in the
absence of versioning information, and thus diminishes the
opportunities for sharing code.  Therefore, versioning is part of the
library system.

\section{Treatment of different versions}

Libraries with the same name but different versions cannot coexist
within the same program.  This prevents combining libraries and
programs that require conflicting versions of the same library.  An
alternative choice would have been to allow the co-existence of
different versions of the same library.  Based on experience with
other shared-library systems (including Windows DLLs and Unix shared
objects), the potential for confusion created by multiple library
instances seems too great.  Therefore, the reports recommends against
multiple instances of a library.

\chapter{Top-level programs}

The notion of ``top-level program'' is new in \rn{6} and replaces the
notion of ``Scheme program'' in \rn{5}.  The two are quite different:
While a \rn{6} top-level program is defined to be a complete, textual
entity, an \rn{5} program could evolve by being entered piecemeal into
a running Scheme system.  Many Scheme systems have interactive
command-line environments based on the semantics of \rn{5} programs.
However, the specification of \rn{5} programs was not really
sufficient do describe how to operate an arbitrary Scheme system: The
\rn{5} is ambiguous on some aspects of the semantics such as binding.
Moreover, \rn{5}'s {\cf load} procedure does say how to load source
code into the running system, but the pragmatics of {\cf load} would
often make compiling programs before execution problematic, in
particular with regard to macros.  Furthermore, Scheme implementations
handle treatment of and recovery from errors in wildly different ways.

Tightening the specification of programs from \rn{5} would have been
possible, but could have restricted the design employed by Scheme
implementations in undesirable ways.  Moreover, alternative approaches
to structuring the user interface of a Scheme implementation have
emerged since \rn{5}.  Consequently, \rn{6} makes no attempt of trying
to specify the semantics of programs as in \rn{5}; the design of an
interactive environment is now completely in the hands of the
implementors.  On the other hand, being able to distribute portable
programs as one of the goals of the \rn{6} process.  As a result, the
notion of top-level program was added to the report.

By allowing the interleaving of definitions and expressions, top-level 
programs support exploratory and interactive development, without 
imposing unnecessary organizational overhead on code which may not be 
intended for reuse.

\chapter{Primitive syntax}

\section{Unspecified evaluation order}

The order in which the subexpressions of an application are evaluated
is unspecified, as is the order in which certain subexpressions of
some other forms such as {\cf letrec} are evaluated.  While this
causes occasional problems, it encourages programmers to write
programs that do not depend on a specific evaluation order, and thus
may be easier to read.  Moreover, it allows the programmer to express
that the evaluation order really does not matter for the result.  A
secondary consideration is that some compilers are able to generate
better code if they can choose evaluation order.


\chapter{Expansion process}

The description of macro expansion is considerable more involved than
it was in \rn{5}: One reason for this that the specification of
expansion in \rn{5} was ambiguous in several important respects.  For
example, \rn{5} does not specify whether {\cf define} is a binding
form.  Also, it was not clear whether definitions of macros had to
precede their uses.  The fact that the set of available bindings may
influence the matching process of macro expansion further complicates
matters.  The specific algorithm \rn{6} describes is one of the
simplest expansion strategies that addresses these questions.  It has
the advantage that it visits every subform of the source code only
once.

The description of the expansion process specifically avoids
specifying the recursive case, where a macro use expands into a
definition whose binding would influence the expansion of the macro
use after the fact, as this might lead to confusing programs.
Implementations should detect such cases as syntax violations.

\chapter{Base library}

\section{Library organization}

The libraries of the Scheme standard are organized according to
projected use.  Hence, the \rsixlibrary{base} library exports
procedures and syntactic abstractions that are likely to be useful for
most Scheme programs and libraries.  Conversely, each of the libraries
relegated to the separate report on libraries is likely to be missing
from the imports of a substantial number of programs and libraries.
Naturally, the specific decisions about this organization and the
separation of concerns of the libraries are debatable, and represent a
best attempt of the editors.

A number of secondary criteria were also used in choosing the exports
of the base library.  In particular, macros transformers defined using
the facilities of the base library are guaranteed to be hygienic;
hygiene-breaking transformers are only available through the
\rsixlibrary{syntax-case} library.

Note that \rsixlibrary{base} is not a ``primitive library'' in the
sense that all other libraries of the Scheme standard can be
implemented portably using only its exports.  Moreover, the library
organization is not layered from more primitive to more advanced
libraries, even though some libraries can certainly be implemented in
terms of others.  Organizing the libraries according to such a
layering would have meant imposing unnecessary constraints on
implementations.  Implementations differ in their sets of primitives,
and thus any layering mandated by the report is unlikely to reflect
the layering employed by any specific implementation.  (The
distinction between primitive and derived features was removed from
the report for the same reasons.)

\section{Bodies}

In library bodies and local bodies, all definitions must precede all
expressions. \rn{6} treats bodies in top-level programs as a special
case.  Allowing definitions and expressions to be mixed in top-level
programs has ugly semantics, and introduces a special case, but was
allowed as a concession to convenience when constructing programs
rapidly via cut and paste.

Definitions are not interchangeable with expressions, so definitions
cannot be allowed to appear wherever expressions can appear, and
expressions cannot be allowed to appear wherever definitions can
appear.  Composition of definitions with expressions will therefore be
restricted in some way; the question is not whether there will be
restrictions, but what those restrictions will be.

Historically, top-level definitions in Scheme have had a different
semantics from definitions in bodies.  In a body, definitions serve as
syntactic sugar for the bindings of a {\cf letrec} (or {\cf letrec*}
in \rn{6}) that is implicit at the head of every body.

That semantics can be stretched to cover top-level programs by
converting expressions to definitions of ignored variables, but does
not easily generalize to allow definitions to be placed anywhere
within expressions.  Different generalizations of definition placement
are possible, however a survey of current Scheme code found
surprisingly few places where such a generalization would be useful.

If such a generalization proposed, were adopted, programmers who are
familiar with Java and similar languages might expect definitions to
be allowed in the same kinds of contexts that allow declarations in
Java.  These programmers would have to be reminded that Scheme's
definitions have {\cf letrec*} scope, while Java declarations (inside
a method body) have {\cf let*} scope and cannot be used to define
recursive procedures; that Scheme's begin expressions do not introduce
a new scope, while Java's curly braces do introduce a new scope; that
flow analysis is nontrivial in higher order languages, while Java can
use a trivial flow analysis to reject programs with undefined
variables; that Scheme's macro expander must locate all definitions,
while Java has no macro system; and so on.  Rather than explain how
those facts justify restricting definitions to appear as top-level
forms of a body, it is simpler to explain that definitions are just
syntactic sugar for the bindings of an implicit {\cf letrec*} at the
head of each body, and to explain that the relaxation of that
restriction for top-level bodies is (like several other features of
top-level programs) an ad hoc special case.

\section{Export levels}

The {\cf syntax-rules} and {\cf identifier-syntax} forms are
used to create macro transformers and are thus needed only at
expansion time, i.e., meta level $1$.  

The identifiers {\cf unquote}, {\cf unquote-splicing}, {\cf =>}, and
{\cf else} serve as literals in the syntax of one or more
syntactic forms; e.g., {\cf else} serves as a
literal in the syntax of {\cf cond} and {\cf case}.
Bindings of these identifiers are exported from the base library so
that they can be distinguished from other bindings of these identifiers
or renamed on import.
The identifiers {\cf ...}, {\cf \_}, and {\cf set!} serve as
literals in the syntax of {\cf syntax-rules} and
{\cf identifier-syntax} forms and are thus exported along with those
forms with level $1$.

\section{Equivalence predicates}

\subsection{Treatment of procedures}

The above definition of {\cf eqv?} allows implementations latitude in
their treatment of procedures: implementations are free either to
detect or to fail to detect that two procedures are equivalent to each
other, and can decide whether or not to merge representations of
equivalent procedures by using the same pointer or bit pattern to
represent both.  Moreover, they can use implementation techniques such
as inlining and beta reduction that duplicate otherwise equivalent
procedures.

\subsection{Equivalence of NaNs}

The basic reason why the behavior of {\cf eqv?} is not specified on
NaNs is that the IEEE-754 standard does not say much about how the
bits of a NaN are to be interpreted, and explicitly allows
implementations of that standard to use most of a NaN's bits to encode
implementation-dependent semantics.  The implementors of a Scheme
system should therefore decide how {\cf eqv?} should interpret those
bits.

Arguably, \rn{6} should require
%
\begin{scheme}
(let ((x \hyper{expression})) (eqv? x x))%
\end{scheme}
%
to evaluate to true when \hyper{expression} evaluates to a number object;
both \rn{5} and \rn{6} imply this for certain other types, and for
most numbers objects, but not for NaNs.  Since the IEEE-754 and draft
IEEE-754R~\cite{IEEE754R} both say that the interpretation of a NaN's
payload is left up to implementations, and implementations of Scheme
often do not have much control over the implementation of IEEE
arithmetic, it would be unwise for \rn{6} to insist upon the truth of
%
\begin{scheme}
(let ((x \hyper{expression}))
  (or (not (number? x))
      (eqv? x x)))%
\end{scheme}
even though that expression is likely to evaluate to true in most
systems.  For example, a system with delayed boxing of inexact real
number objects might box the two arguments to {\cf eqv?} separately, the boxing
process might involve a change of precision, and the two separate
changes of precision may result in two different payloads.

When \var{x} and \var{y} are flonums represented in IEEE floating
point or similar, it is reasonable to implement {\cf (eqv? \var{x}
  \var{y})} by a bitwise comparison of the floating-point
representations.  \rn{6} should not require this, however, because
%
\begin{enumerate}
\item \rn{6} does not require that flonums be represented by a
  floating-point representation,
\item the interpretation of a NaN's payload is explicitly
  implementation-dependent according to both the IEEE-754 standard and
  the current draft of its proposed replacement, IEEE-754R, and
\item the semantics of Scheme should remain independent
  of bit-level representations.
\end{enumerate}
%
For example, IEEE-754, IEEE-754R, and the draft \rn{6} all allow the
external representation {\cf +nan.0} to be read as a NaN whose payload
encodes the input port and position at which {\cf +nan.0} was read.
This is no different from any other external representation such as
{\cf ()}, {\cf \sharpsign()}, or {\cf 324}.  An implementation can
have arbitrarily many bit-level representations of the empty vector,
for example, and some do.  That is why the behavior of the {\cf eq?}
and {\cf eqv?} procedures on vectors cannot be defined by reference to
bit-level representations, and must instead be defined explicitly.

\subsection{{\tt eq?}}

It is usually possible to implement {\cf eq?}\ much more efficiently
than {\cf eqv?}, for example, as a simple pointer comparison instead
of as some more complicated operation.  One reason is that it may not
be possible to compute {\cf eqv?}\ of two number objects in constant time,
whereas {\cf eq?}\ implemented as pointer comparison will always
finish in constant time.  The {\cf eq?} predicate may be used like
{\cf eqv?}\ in applications using procedures to implement objects with
state since it obeys the same constraints as {\cf eqv?}.

\section{Generic arithmetic}

\subsection{Full numerical tower}

\rn{5} did not require implementations to support the full numeric
tower.  Consequently, it was hard to write portable programs that
perform substantial arithmetic, and it is unnecessarily difficult even
to write programs whose arithmetic is portable between different
implementations in the same category.  The portability problems were
most easily solved by requiring all implementations to support the
full numerical tower.

\subsection{IEEE-754 conformance}

As mentioned in chapter~\ref{numberschapter}, the treatment of
infinities, NaNs and -0.0, if present in a Scheme implementation, are
in line with IEEE-754~\cite{IEEE} and IEEE-754R~\cite{IEEE754R}.
Analogously, the specification of branch cuts for certain
transcendental functions have been changed from \rn{5} to conform to
the IEEE standard.

\subsection{Transcendental functions}

The specification of the transcendental functions follows
Steele~\cite{CLtL}, which in turn cites Penfield~\cite{Penfield81};
refer to these sources for more detailed discussion of branch cuts,
boundary conditions, and implementation of these functions.

\subsection{Domains of numerical predicates}

The domains of the {\cf finite?}, {\cf infinite?}, and {\cf nan?}
procedures could be expanded to include all number objects, or perhaps even
all objects.  However, \rn{6} restricts them to real number objects.
Expanding {\cf nan?} to complex number objects would involve at least some
arbitrariness; not expanding its domain while expanding the domains of
the other two would introduce an irregularity into the domains of
these three procedures, which are likely to be used together; it is
easier for programmers who wish to use these procedures with complex
number objects to express their intent in terms of the real-only versions
than it would be for the editors to guess their intent.

\subsection{Numerical types}

Scheme's numerical types are the exactness types exact and inexact,
the tower types integer, rational, real, complex, and number, and the
Cartesian product of the exactness types with the tower types, where
$\left< t_1, t2 \right>$; is regarded as a subtype of both $t_1$ and
$t_2$.

These types have an aesthetic symmetry to them, but they are not
equally important.  Judging by the number of \rn{5} procedures whose
domain is restricted to values of some numerical type, the most
important numerical types are the exact integer objects, the integer objects, the
rational number objects, the real number objects, and the complex
number objects.

Many programmers and implementors regard the focus of \rn{5} on those
particular numerical types as something of a mistake.  In practice,
there is reason to believe that the most important numerical types are
the exact integer objects, the exact rational number objects, the
inexact real number objects, and the
inexact complex number objects.  This section explores one of the reasons
those four types are so important in practice, and why real number objects have an
exact zero as their imaginary part in \rn{6} (a change from \rn{5}).

\subsection{Closure Properties}
\label{closurepropertiessection}

Scheme's types are not completely arbitrary.  Each type corresponds to
a set of values that turns up repeatedly as the natural domain or
range of the functions that are computed by Scheme's standard
procedures.  The reason these types turn up so often is that they are
closed under certain sets of operations.

The exact integer objects, for example, are closed under the integral
operations of addition, subtraction, and multiplication.  The exact
rational number objects are closed under the rational operations, which consist of
the integral operations plus division (although division by zero is a special
case).  The real number objects (and inexact real number objects) are closed
under some (often inexact) interpretation of rational and irrational
operations such as exp and sin, but are not closed under operations
such as {\cf log}, {\cf sqrt}, and {\cf expt}.  The complex (and
inexact complex) number objects are closed under the largest set of
operations.

\subsubsection{Representation-specific operations}

A naive implementation of Scheme's arithmetic operations is slow
compared to the arithmetic operations of most other languages, mainly
because most operations must perform some sort of case dispatch on the
representation types of their arguments.  The potential for this case
dispatch arises when the type of an operation's argument is
represented by a union of two or more representation types, or because
the operation must raise an exception when given an argument of
an incorrect type.  (The second reason can be regarded as a special
case of the first.)

To make Scheme's arithmetic more efficient, many implementations
provide sets of operations whose domain is restricted to a single
representation type, and which are not expected to raise an exception
when given arguments of incorrect type.

Alternatively, or in addition, several compilers perform some kind of
flow analysis that attempts to infer the representation types of
expressions.  When a single representation type can be inferred for
each argument of an operation, and those types match the types
expected by some representation-specific version of the operation,
then the compiler can substitute the specific version for the more
general version that was specified in the source code.

\subsubsection{Flow analysis}

Flow analysis is performed by solving the type and interval
constraints that arise from such things as:

\begin{itemize}
\item the types of literal constants, e.g.\ {\cf 2} is an exact
  integer object
  that is known to be within the interval $[2,2]$
  
\item conditional control flow that is predicated on known
  inequalities, e.g. {\cf (if (< i n) \hyperi{expression} \hyperii{expression})}
  
\item conditional control flow that is predicated on known type
  predicates, e.g. {\cf (if (real? x) \hyperi{expression} \hyperii{expression})}
  
\item the closure properties of known operations (for example, {\cf (+
    \vari{flonum} \varii{flonum})} always evaluates to a flonum)
\end{itemize}
  
The purpose of flow analysis (as motivated in this section) is to infer a
single representation type for each argument of an operation.  That
places a premium on predicates and closure properties from which a
single representation type can be inferred.

In practice, the most important single representation types are
fixnum, flonum, and compnum.  (A compnum is a pair of flonums,
representing an inexact complex number object.)  These are the representation
types for which a short sequence of machine code can be generated when
the representation type is known, but for which considerably less
efficient code will probably have to be generated when the
representation type cannot be inferred.

The fixnum representation type is not closed under any operation of
\rn{5}, so it is hard for flow analysis to infer the fixnum type from
portable code.  Sometimes the combination of a more general type (e.g.
exact integer object) and an interval (e.g.  $[0,n)$, where $n$ is known to
be a fixnum) can imply the fixnum representation type.  Adding
fixnum-specific operations that map fixnums to fixnums 
greatly increases the number of fixnum
representation types that a compiler can infer.

The flonum representation type is not closed under operations such as
{\cf sqrt} and {\cf expt}, so flow analysis tends to break down in the
presence of those operations.  This is unfortunate, because those
operations are normally used only with arguments for which the result
is expected to be a flonum.  Adding flonum-specific versions such as
{\cf flsqrt} and {\cf flexpt} improves the effectiveness of flow
analysis.

\rn{5} creates a more insidious problem by defining {\cf (real?
  \var{z})} to be true if and only if {\cf (zero? (imag-part
  \var{z}))} is true.  This means, for example, that {\cf -2.5+0.0i}
is real.  If {\cf -2.5+0.0i} is represented as a compnum, then the
compiler cannot rely on x being a flonum during the \hyper{real case}
of {\cf (if (real? x) \hyperi{expression} \hyperii{expression})}.  This
problem could be fixed by writing all of the arithmetic operations so
that any compnum with a zero imaginary part is converted to a flonum
before it is returned, but that merely creates an analogous problem
for compnum arithmetic, as explained below.  \rn{6} adopted a proposal
by Brad Lucier to fix the problem: {\cf (real? \var{z})} is now true
if and only if {\cf (imag-part \var{z})} is an exact zero.

The compnum representation type is closed under virtually all
operations, provided no operation that accepts two compnums as its
argument ever returns a flonum.  To work around the problem described
in the paragraph above, several implementations automatically convert
compnums with a zero imaginary part to the flonum representation.
This practice virtually destroys the effectiveness of flow analysis
for inferring the compnum representation, so it is not a good
workaround.  To improve the effectiveness of flow analysis, it is
better to change the definition of Scheme's real number objects as described
in the paragraph above.

\subsubsection{div and mod}

Given arithmetic on exact integer objects of arbitrary precision, it is a
trivial matter to derive signed and unsigned integer types of finite
range from it by modular reduction.  For example 32-bit signed
two-complement arithmetic behaves like computing with the residue
classes ``mod $2^{32}$'', where the set $\{-2^{31}, \ldots,
2^{31}-1\}$ has been chosen to represent the residue classes.
Likewise, unsigned 32-bit arithmetic also behaves like computing ``mod
$2^{32}$'', but with a different set of representatives $\{0, \ldots,
2^{32}-1\}$.

Unfortunately, the \rn{5} operations {\cf quotient}, {\cf remainder},
and {\cf modulo} are not ideal for this purpose.  In the following
example, {\cf remainder} fails to transport the additive group
structure of the integers over to the residues modulo 3.
%
\begin{scheme}
(remainder (+ -2 3) 3) \ev 1,
(remainder (+ (remainder -2 3)
              (remainder 3 3))
           3) \ev -2%
\end{scheme}
%
In fact, {\cf modulo} should have been used, producing residues in
$\{0,1,2\}$. For modular reduction with symmetric residues, i.e. in
$\{-1,0,1\}$ in the example, it is necessary to define a more
complicated reduction altogether.

Therefore, {\cf quotient}, {\cf remainder}, and {\cf modulo} have been
replaced in \rn{6} by the {\cf div}, {\cf mod}, {\cf div0}, and {\cf
  mod0} procedures, which are more useful when implementing modular
reduction.  The underlying mathematical functions $\mathrm{div}$,
$\mathrm{mod}$, $\mathrm{div}_0$, and $\mathrm{mod}_0$ (see report
section~\extref{report:integerdivision}{Integer division}) have been
adapted from the $\mathrm{div}$ and $\mathrm{mod}$ operations by Egner
et al.~\cite{cleaninguptower}.  They differ in the representatives
from the residue classes they return: $\mathrm{div}$ and $\mathrm{mod}$
always compute a non-negative residue, whereas $\mathrm{div}_0$ and
$\mathrm{mod}_0$ compute a residue from a set centered on 0.  The
former can be used, for example, to implement unsigned fixed-width
arithmetic, whereas the latter correspond to two's-complement arithmetic.

These operations differ slightly from the $\mathrm{div}$ and
$\mathrm{mod}$ operations from Egner et al.  The latter make both operations
available through a single pair of operations that distinguish
between the two cases for residues by the sign of the divisor (as well
as returning $0$ for a zero divisor).  Splitting the operations into
two sets of procedures avoids potential confusion.

The procedures {\cf modulo}, {\cf remainder}, and {\cf quotient} from
\rn{5} can easily be defined in terms of {\cf div} and {\cf mod}.

\subsection{{\tt Numerical predicates}}

The behavior of the numerical type predicates {\cf complex?}, {\cf
  real?}, {\cf rational?}, and {\cf integer} behavior is motivated by
closure properties described in
section~\ref{closurepropertiessection}.  Conversely, the procedures
{\cf real-valued?}, {\cf rational-valued?}, and {\cf integer-valued?}
test whether a given number object can be coerced to the specified type
without loss of numerical accuracy.

\subsection{Notes on individual procedures}

\paragraph{{\tt round}}

The {\cf round} procedure rounds to even for consistency with the
default rounding mode specified by the IEEE floating-point standard.

\paragraph{{\tt sqrt}}

The behavior of {\cf sqrt} is consistent with the IEEE floating-point
standard.

\paragraph{{\tt number->string}}

If \var{z} is an inexact number object represented using binary floating
point, and the radix is 10, then the expression listed in the
specification is normally satisfied by a result containing a decimal
point.  The unspecified case allows for infinities, NaNs, and
representations other than binary floating-point.

\section{Characters and strings}

Where \rn{5} specified characters and strings in terms of its own,
limited character set, \rn{6} specifies characters and strings in
terms of Unicode.  The primary goal of the design change were to
improve the portability of Scheme programs that manipulate text, while
preserving a maximum of backward compatibility with \rn{5}.

\rn{6} defines characters to be representations of Unicode scalar
values, and strings to be indexed sequences of characters.  This is a
different representation for Unicode text than the representations
chosen by some other programming languages such as Java or
C\sharpsign{}, which use UTF-16 code units as the basis for the type
of characters.

The representation of Unicode text corresponds to the lowest semantic
level of the Unicode standard: The Unicode standard specifies most
semantic properties in terms of Unicode scalar values.  Thus, Unicode
strings in Scheme allow the straightforward implementation of
semantically sensitive algorithms on strings in terms of these scalar
values.

In contrast, UTF-16 is a specific encoding for Unicode text, and
performing semantic manipulation on UTF-16 representations of text is
awkward.  Choosing UTF-16 as the basis for the string representation
would have meant that a character object potentially carries no
semantic information at all, as surrogates have to be combined
pairwise to yield the corresponding Unicode scalar value.  (As a
result, the APIs of Java for semantic operations on Unicode text often
come in two overloadings, one for character objects and one for
integers that are Unicode scalar values.)

The surrogates cover a numerical range deliberately omitted from the
set of Unicode scalar values.  Hence, surrogates have no
representation as characters---they are merely an artefact of the
design of UTF-16.  Including surrogates in the set of characters
introduces complications similar to the complications of using UTF-16
directly.  In particular, most Unicode consortium standards and
recommendations explicitly prohibit unpaired surrogates, including the
UTF-8 encoding, the UTF-16 encoding, the UTF-32 encoding, and
recommendations for implementing the ANSI C {\cf wchar\_t} type.  Even
UCS-4, which originally permitted a larger range of values that
includes the surrogate range, has been redefined to match UTF-32
exactly. That is, the original UCS-4 range was shrunk and surrogates
were excluded.

Arguably, a higher-level model for text could be used as the basis for
Scheme's character and string types, such as grapheme clusters.
However, no design satisfying the goals stated above was available
when the report was written.

\section{Symbols}

Symbols have exactly the properties needed to represent
identifiers in programs, and so most implementations
of Scheme use them internally for that purpose.  Symbols are useful
for many other applications; for instance, they may be used the way
enumerated values are used in C and Pascal.

\section{Control features}

\section{{\tt call-with-current-continuation} and {\tt dynamic-wind}}

\vest A common use of {\cf call-with-current-continuation} is for
structured, non-local exits from loops or procedure bodies, but in fact
{\cf call-with-current-continuation} is useful for implementing a
wide variety of advanced control structures.

Most programming languages incorporate one or more special-purpose
escape constructs with names like {\tt exit}, \hbox{{\cf return}}, or
even {\tt goto}.  In 1965, however, Peter Landin~\cite{Landin65}
invented a general purpose escape operator called the J-operator.  John
Reynolds~\cite{Reynolds72} described a simpler but equally powerful
construct in 1972.  The {\cf catch} special form described by Sussman
and Steele in the 1975 report on Scheme is exactly the same as
Reynolds's construct, though its name came from a less general construct
in MacLisp.  Several Scheme implementors noticed that the full power of the
\ide{catch} construct could be provided by a procedure instead of by a
special syntactic construct, and the name
{\cf call-with-current-continuation} was coined in 1982.  This name is
descriptive, but opinions differ on the merits of such a long name, and
some people use the name \ide{call/cc} instead.

The {\cf dynamic-wind} procedure was added more recently in \rn{5}.
It enables implementing a number of abstractions related to
continuations, such as implementing a general dynamic environment, and
making sure that finalization code runs when some dynamic extent
expires.  More generally, the {\cf dynamic-wind} procedure provides a
guarantee that
%
\begin{scheme}
(dynamic-wind \var{before} \var{thunk} \var{after})%
\end{scheme}
%
cannot return unless both \var{before} and \var{after} have been
evaluated, in that order.  They might be evaluated several times, but
always in that order, and these evaluations are never nested.  As this
guarantee is crucial for enabling many of the uses of {\cf
  call-with-current-continuation} and {\cf dynamic-wind}, both are
specified jointly.


\subsection{Multiple values}

Many computations conceptually return several results.  Scheme
expressions implementing such computations can return the results as
several values using the {\cf values} procedure.  Of course, such
expressions could alternatively return the results as a single
compound value, such as a list, vector, or a record.  However, values
in programs usually represent conceptual wholes; in many cases,
multiple results yielded by a computation lack this coherence.
Moreover, this would be inefficient in many implementations, and a
compiler would need to perform significant optimization to remove the
boxing and unboxing inherent in packaging multiple results into a
single values.  Most importantly, the mechanism for multiple values in
Scheme establishes a standard policy for returning several results
from an expression, which makes constructing interfaces and using them
easier.

\rn{6} does not specify the semantics of multiple values completely.
In particular, it does not specify what happens when more than one
value (or zero values) are returned to a continuation that implicitly
accepts only one value.  In particular:
%
\begin{scheme}
((lambda (x) x) (values  1 2)) \lev \unspecified%
\end{scheme}
%
Whether an implementation must raise an exception when evaluating such
an expression, or should exhibit some other, non-exceptional behavior
is a contentious issue.  Variations of two different and fundamentally
incompatible positions on this issue exist, each with its own merits:
%
\begin{enumerate}
\item Passing the wrong number of values to a continuation is
typically an error, one that implementations ideally detect and report.

\item There is no such thing as returning the wrong number of values
  to a continuation.  In particular, continuations not created by {\cf
    begin} or {\cf call-with-values} should ignore all but the first
  value, and treat zero values as one unspecified value.
\end{enumerate}
%
\rn{6} allows an implementation to take either position.  Moreover, it
allows an implementation to let {\cf set!}, {\cf vector-set!}, and
other effect-only operators to pass zero values to their
continuations, preventing a program to make obscure use of the return
value.  This causes a potential compatibility problem with \rn{5},
which specifies that such expression return a single unspecified
value, but the benefits of the change were deemed to outweigh the costs.

\section{Macro transformers}

\subsection{{\tt syntax-rules}}

While the first subform of \hyper{srpattern} of a \hyper{syntax rule}
in a {\cf syntax-rules} form (see report
section~\extref{report:syntaxrulessection}{Macro transformers})
may be an identifier, the
identifier is not involved in the matching and is not considered a
pattern variable or literal identifier.  This is actually important,
as the identifier is most often the keyword used to identify the
macro.  The scope of the keyword is determined by the binding form or
syntax definition that binds it to the associated macro transformer.
If the keyword were a pattern variable or literal identifier, then the
template that follows the pattern would be within its scope regardless
of whether the keyword were bound by {\cf let-syntax}, {\cf
  letrec-syntax}, or {\cf define-syntax}.

\chapter{Formal semantics}

The operational semantics in report
chapter~\extref{report:formalsemanticschapter}{Formal semantics}
replaces the denotational semantics in \rn{5}.  The denotational
semantics in \rn{5} has several problems, most seriously its
incomplete treatment of the unspecific evaluation order of
applications: the denotational semantics suggests that a single
unspecified order is used.  Modelling nondeterminism is generally
difficult with denotational semantics, and an operational semantics
allows specifying the unspecified evaluation order precisely.


\chapter{Unicode}

\section{Case mapping}

The various case-mapping procedures of the \rsixlibrary{unicode}
library all operate in a locale-independent manner.  The Unicode
standard also offers locale-sensitive case operations, not implemented
by the procedures from the \rsixlibrary{unicode} library.  While the
library does not make available the full spectrum of case-related
functionality defined by the Unicode standard, it does provide the
most commonly used procedures.  In particular, this strategy has
allowed providing procedures mostly compatible with those provided by
\rn{5}.  (A minor exception is the case-insensitive procedures for
string comparison.  However, it is unlikely that this affects many
existing programs.)  Providing locale-sensitive operations would have
meant significant novel design effort without significant precedent,
which is why they are not part of \rn{6}.

The case-mapping procedures operating on characters are not sufficient
for implementing case mapping on strings.  For example, the upper-case
version of the German ``\ss{}'' in a string is ``SS''.  As {\cf
  char-upcase} can only return a single character, it must return
\ss{} for \ss.  This limits the usefulness of the procedures
operating on characters, but provides compatibility with \rn{5}
sufficient for many existing applications.  Moreover, it provides
direct access to the corresponding attributes of the Unicode character
database.

\chapter{Bytevectors}

Bytevectors are a representation for binary data, based on
SRFI~74~\cite{srfi74}.  The primary motivation for including them in
\rn{6} was to enable binary I/O.  Positions in bytevectors always
refer to certain bytes or octets.  However, the operations of the
\rsixlibrary{bytevectors} library provide access to binary data in
various byte-aligned formats, such as signed and unsigned integers of
various widths, IEEE floating-point representations, and textual
encodings.  This differs notably from representations for binary data
as homogeneous vectors of numbers.  In settings related to I/O, an
application often needs to access different kinds of entities from a
single binary block.  Providing operations for them on a single
datatype considerably reduces both programming effort and library
size.

Bytevectors can also be used to encode sequences of unboxed number objects.
However, unencapsulated use of bytevectors for this purpose may lead
to aliasing, which may reduce the effectiveness of compiler
optimizations.  However, the records facility provides sealedness and
opacity, which, together with bytevectors, make it possible to
construct a portable implementation for new data types that
provide fast and memory-efficient arrays of homogeneous numerical
data.

\chapter{List utilities}

The \rsixlibrary{lists} library provides a small number of useful
procedures operating on lists, including several procedures from
\rn{5}.  The goal of the library is to provide only procedures likely
to be useful for many programs.  Consequently, the selection
represented by \rsixlibrary{lists} is less exhaustive than the widely
implemented SRFI~1~\cite{srfi1}.  Several changes were made with
respect to the corresponding procedures SRFI~1 to simplify the
specification, and to establish uniform naming conventions.

\section{Notes on individual procedures}

\paragraph{{\tt memp}, {\tt member}, {\tt memv}, and {\tt memq}}

Although they are ordinarily used as predicates, {\cf memp}, {\cf
  member}, {\cf memv}, and {\cf memq}, do not have question marks in
their names because they return useful values rather than just
\schtrue{} or \schfalse{}.

\chapter{Sorting}

The procedures of the \rsixlibrary{sorting} library provide simple
interfaces to sorting algorithms useful to many programs.  In
particular, {\cf list-sort} and {\cf vector-sort} guarantee stable
sorting using $O(n \lg n)$ calls to the comparison procedure.
Straightforward implementations of merge sort~\cite{algorithms} have
the desired properties.  Note that, at least with merge sort,
stability carries no significant implementation or performance burden.

The choice of ``strictly less than'' for the comparison relation is
consistent with the most common choice of existing Scheme libraries
for sorting.  Moreover, using a procedure returning three possible
values (for less than, equal, and greater than) instead of a boolean
comparison procedure would make calling the sorting procedures less
convenient, with no discernible performance advantage.

The specification of the {\cf vector-sort!} procedure is meant to
allow an implementation using quicksort, hence the $O(n^2)$
bound on the number of calls to the comparison procedure, and the
omission of the stability requirement.

\chapter{Control structures}

\section{{\tt when} and {\tt unless}}

The {\cf when} and {\cf unless} forms are syntactic sugar for {\cf if}
expressions in one arm.  They provide minor convenience, and are
useful for distinguishing one-armed conditionals from two-armed {\cf
  if} expressions.  Some programmers use {\cf when} and {\cf unless}
to mark conditionals with side effects, avoid one-armed {\cf if}
expressions, and use two-armed {\cf if} expressions primarily for
their values, not their effects.

\section{{\tt case-lambda}}

The {\cf case-lambda} form allows constructing procedures that
distinguish different numbers of arguments.  Using {\cf case-lambda}
makes this considerably easier than deconstructing a list containing
optional arguments explicitly.  Moreover, Scheme implementations might
optimize dispatch on the number of arguments when expressed as {\cf
  case-lambda}, which is considerably harder for code that explicitly
deconstructs argument lists.

\chapter{Records}


\section{Syntactic layer}

While the syntactic layer could be expressed portably in terms of the
procedural layer, standardizing a particular surface syntax
facilitates communication via code.

Moreover, some implementations are able to implement the syntactic
layer more efficiently than a portable implementation built on top of
the procedural layer, and thus the syntactic layer facilitates the
development of efficient portable libraries that define and use record
types.  The syntactic layer is designed to allow expansion- or
compile-time determination of record characteristics (at least in the
absence of a {\cf parent-rtd} clause), including field offsets, so
that, for example, record accesses can be be compiled efficiently to
simple memory indirects.  This property may be lost if the {\cf
  parent-rtd} clause is present, and the parent is thus not generally
known until run time.

\section{Positional access and field names}

The record and field names passed to {\cf make-record-type-descriptor}
and appearing in the syntactic layer are for informational purposes
only, e.g., for printers and debuggers.  In particular, the accessor
and mutator creation routines do not use names, but rather field
indices, to identify fields.
Thus, field names are not required to be distinct in the procedural or
syntactic layers.  This relieves macros and other code generators from
the need to generate distinct names.

The record and field names are used in the syntactic layer for the
generation of accessor and mutator names, and thus duplicate field names
may lead to accessor and mutator naming conflicts.

\section{Lack of multiple inheritance}

Multiple inheritance was considered but omitted from the records
facility, as it raises a number of semantic issues such as sharing
among common parent types.

\section{Constructor mechanism}

The constructor-descriptor mechanism is an infra\-struc\-ture for
creating specialized constructors, rather than just creating default
constructors that accept the initial values of all the fields as
arguments. This infrastructure achieves full generality while leaving
each level of an inheritance hierarchy in control over its own fields
and allowing child record definitions to be abstracted away from the
actual number and contents of parent fields.

The constructor mechanism allows the initial values of the fields to be specially
computed or to default to constant values. It also allows for
operations to be performed on or with the resulting record, such as
the registration of a record for finalization. Moreover, the
constructor-descriptor mechanism allows the creation of such
initializers in a modular manner, separating the initialization
concerns of the parent types from those of the extensions.

\section{Sealed record types}

Record types may be sealed.  This feature allows enforcing abstraction
barriers, which is useful in itself, but also allows more efficient
compilation.

In particular, when the implementor of an abstract data type chooses
to represent that ADT by record type, and allows one of the classes
that represent the ADT to be exposed and subclassed, then the ADT is
no longer abstract.  Its implementors must expose enough information
to allow for effective subtyping, and must commit to enough of the
representation to allow those subclasses to continue to work even as
the ADT evolves.

In practice, the ADT cannot evolve freely over time.  Its
representation cannot be changed without breaking most of its
subclasses.  This kind of brittleness is often problematic in
large programs.  Sealedness allows the construction of record types
that evolve more gracefully over time.

Moreover, making a record type sealed may prevent its accessors and
mutators from becoming polymorphic, which would make effective flow analysis and
optimization difficult.  This is particularly relevant for
Scheme implementations that use record to implement some of Scheme's
other primitive data types such as pairs.

\chapter{Conditions and exceptions}

\section{Exceptions}

The goals of the exception mechanism are to help programmers share
code which relies on exception handling, and to provide information on
violations of specifications of procedures and syntactic forms.  This
exception mechanism is an extension of SRFI~34~\cite{srfi34}, which
was primarily designed to meet the first goal.  However, it has proven
suitable for addressing the second goal of dealing with violations as
well.   (More on the second goal below in the discussion of the
condition system.)

For some violations such the use of unsupported NaNs or infinities, as
well as other applications, an exception handler may be able to
repair the cause of the exception, for example by substituting a
suitable object for the NaN or infinity.  Therefore, the exception
mechanism extends SRFI~34 by continuable exceptions, and specifies the
continuation of an exception handler

\section{Conditions}

Conditions are values that communicate information about exceptional
situations between parts of a program. Code that detects an exception
may be in a different part of the program than the code that handles
it. In fact, the former may have been written independently from the
latter.  Consequently, to facilitate effective handling of exceptions,
conditions should communicate as much information as possible as
accurately as possible, and still allow effective handling by code
that did not precisely anticipate the nature of the exception that has
occurred.

The \rsixlibrary{conditions} library provides two mechanisms to
enable this kind of communication:
%
\begin{itemize}
\item subtyping (through record types) among condition types allows
  handling code to determine the general nature of an exception even
  though it does not anticipate its exact nature,
\item compound conditions allow an exceptional situation to be
  described in multiple ways.
\end{itemize}
%
As an example, a networking error that occurs during a file operation
on a remote drive fits two descriptions: ``networking error'' and
``file-system error''.  An exception handler might only look for one of
the two.  Compound conditions are a simple solution to this problem.
Moreover, compound conditions also make providing auxiliary
information as part of the condition object, such as an error message,
easier.

The standard condition hierarchy makes an important distinction
between \emph{errors} and \emph{violations}: An error is an
exceptional situation in the environment, which the program cannot
avoid or prevent.  For example, I/O errors are represented by
condition types that are subtypes of {\cf\&error}.  Violations, on the
other hand, are exceptional situations that the program could have
avoided.  Violations are typically programming mistakes.  The
distinction between the two is not always clear, and it may be possible
but inordinately difficult or expensive to detect certain violations.
The use of {\cf eval} also blurs the distinction.  Nevertheless, many
cases do allow distinguishing between errors and violations.
Consequently, exception handlers that handle errors are common,
whereas programmers should introduce exception handlers that handle
violations with great care.


\chapter{I/O}

\section{File names}
\label{filenamesection}

The filenames in most common operating systems, despite their
appearance in most cases, are not text: For example, Unix uses
null-terminated byte sequences, and Windows uses null-terminated
sequences of UTF-16 code units.  On Unix, the textual representation
of a file name depends on the locale, environmental setting.  In both
cases, a file name may be an invalid encoding and thus not correspond
to a string.  An appropriate representation for file names that covers
these cases while still offering convenient access to file-system
names through strings is still an open research problem.  Therefore,
\rn{6} allows specifying file names as strings, but also allows an
implementation to add its own representation for file names.

\section{File options}

The flags specified for {\cf file-options} represent only a common
subset of meaningful options on popular platforms.  The {\cf
  file-options} form does not restrict the \hyper{file-options name}s,
so implementations can extend the file options by platform-specific
flags.

\section{End-of-line styles}

The set of end-of-line styles recognized by the \rsixlibrary{ports}
library is not closed because end-of-line styles other than those
listed might become commonplace in the future.

\section{Error-handling modes}

The set of error-handling modes is not closed because implementations
may support error-handling modes other than those listed.

\section{Binary and textual ports}

The plethora of widely used encodings for texts makes providing
textual I/O significantly more complicated than the simple model
offered by \rn{5}.  In particular, realistic textual I/O should
address encodings such as UTF-16 that include a header word
determining the ``actual'' encoding of the rest of the byte stream,
stateful encodings, and textual formats such as XML, which specify the
encoding in a header line.  Consequently, a library implementing
textual I/O should support specifying an encoding upon opening a port,
but should also support opening a port in ``binary mode'' to determine
the encoding and switch to ``text mode''.

In contrast, arbitrary switching between ``binary mode'' and ``text
mode'' is difficult to support, as it may interfere with efficient
buffering strategies, and because the semantics may be unclear in the
case of stateful encodings.  Consequently, the \rsixlibrary{io ports}
library allows switching from ``binary mode'' to ``text mode'' by
converting a binary port into a textual port, but not the other way
around.  The {\cf transcoded-port} procedure closes the binary port to
preclude interference between the binary port and the textual port
constructed from it.  Applications that read from sources that
intersperse binary and textual data should open a binary port and use
either {\cf bytevector->string} or the procedures from the
\rsixlibrary{bytevectors} library to convert the binary data to text.

The separation of binary and textual ports enables creating ports from
both binary and textual sources and sinks.  It also makes creating
both binary and textual versions of some procedures many procedures
necessary.

\section{File positions}

Transcoded ports do not always support the {\cf port-position} and
{\cf set-port-position!} operations: The position of a transcoded port
may not be well-defined, and may be hard to calculate even when
defined, especially when transcoding is buffered.

\section{Freshness of standard ports}

The ports returned by {\cf standard-input-port}, {\cf
  standard-output-port}, and {\cf standard-error-port} is fresh so it
can be safely closed or converted to a textual port without risking
the usability of an existing port.


\section{Argument conventions}

While the \rsixlibrary{io simple} library provides mostly
\rn{5}-compatible procedures for performing textual I/O, the
\rsixlibrary{io ports} library uses a different convention for
argument ordering.  In particular, the port is always the first
argument.  This enables the use of optional arguments for information
about the data to be read or written, such as the range in a
bytevector.  As this convention is incompatible with the convention of
\rsixlibrary{io simple}, corresponding procedures have different
names.

\chapter{File system}

The \rsixlibrary{files} library provides a minimal set of procedures
useful in many programs: The {\cf file-exists?} procedure allows a
program to detect the presence of a file if it is going to overwrite
it, and {\cf delete-file} allows taking the appropriate action if the
old file is no longer useful.

Standardization of procedures that return or pass to another procedure
the name of a file is more difficult than standardization of {\cf
  file-exists?} and {\cf delete-file}, because strings are either
awkward or in sufficient to represent file names on some platforms,
such as Unix and Windows.  See section~\ref{filenamesection}.

\chapter{Arithmetic}

\section{Fixnums and flonums}

Fixnum and flonum arithmetic is already supported by many systems,
mainly for efficiency. Standardization of fixnum and flonum arithmetic
increases the portability of code that uses it.  Standardizing the
precision of fixnum and flonum arithmetic would make it inefficient on
some systems, which would defeat its purpose.  Therefore, \rn{6}
specifies the syntax and much of the semantics of fixnum and flonum
arithmetic, but makes the precision implementation-dependent.

Existing implementations employ different implementation strategies
for fixnums: Some implement the model specified by \rn{6} (overflows
cause exceptions), some implement modular arithmetic (overflows ``wrap
around''), and others do not handle arithmetic overflows at all.  The
latter model violates the safety requirement of \rn{6}.  In programs
that use fixnums instead of generic arithmetic, overflows are
typically programming mistakes.  The model chosen for \rn{6} has the
advantage that such overflows do not get silently converted into
meaningless number objects, and that the programs gets notified of the
violation through the exception system.

\section{Notes on individual procedures}

\paragraph{{\tt fl+} and {\tt fl*}}

These procedures are restricted to two arguments because their
generalizations to three or more arguments would require
precision proportional to the number of arguments.

\paragraph{{\tt real->flonum}}

This procedure is necessary because not all real number objects are inexact, and
some inexact real number objects may not be flonums.

\paragraph{{\tt flround}}

The {\cf flround} procedure rounds to even for consistency with the default rounding
mode specified by the IEEE floating-point standard.

\paragraph{{\tt flsqrt}}

The behavior of {\cf flsqrt} on $-0.0$ is consistent with the IEEE
floating-point standard.


\chapter{{\tt syntax-case}}

While many syntax transformers are succinctly expressed using the
high-level {\cf syntax-rules} form, others are difficult or impossible
to write, including some that introduce visible bindings for or references
to identifiers that do not appear explicitly in the input form, ones that
maintain state or read from the file system, and ones that construct new
identifiers.
The {\cf syntax-case} system~\cite{syntacticabstraction} 
allows the programmer to write transformers that perform these sorts of
transformations, and arbitrary additional transformations, without
sacrificing the default enforcement of hygiene or the high-level
pattern-based syntax matching and template-based output construction
provided by {\cf syntax-rules} (report
section~\extref{report:syntax-rules}{Macro transformers}).

Because {\cf syntax-case} does not require literals, including quoted
lists or vectors, to be copied or even traversed, it may be able to preserve sharing
and cycles within and among the constants of a program.  It also
allows source-object correlation, i.e., the maintenance of ties
between the original source code and expanded output, allowing
implementations to provide source-level support for debuggers and
other tools.

\chapter{Hashtables}

\section{Caching}

The specification notes that hashtables are allowed to cache the
results of calling the hash function and equivalence function, and
that any hashtable operation may call the hash function more than
once.  Hashtable lookups are often followed by updates, so caching may
improve performance.  Hashtables are free to change their internal
representation at any time, which may result in many calls to the hash
function.

\section{Immutable hashtables}

Hashtable references may be less expensive with immutable hashtables.
Also, the creator of a hashtable may wish to prevent 
modifications, particularly by code outside of the creator's 
control.

\section{Hash functions}

The {\cf make-eq-hashtable} and {\cf make-eqv-hashtable} constructors
are designed to hide their hash function.  This allows implementations
to use the machine address of an object as its hash value, rehashing
parts of the table as necessary if a garbage collector moves
objects to different addresses.

\chapter{Enumerations}

Many procedures in many libraries accept arguments from a finite set,
or subsets of a finite sets to describe a certain mode of operation,
or several flags to describe a mode of operation.  Examples in the
\rn{6} include the endianness for bytes-object operations, and file
and buffering modes in the I/O library.  Offering a default policy for
dealing with such values fosters portable and readable code, much as
records do for compound values, or multiple values for procedures
computing several values.  Moreover, representations of sets from a
finite set of options should offer the standard set operations, as
they tend to occur in practice.  (One such set operation is the
complement, which makes lists of symbols a less than suitable
representation.)

Different Scheme implementations have taken different approaches to
this problem in the past, which suggests that a default policy does
more than merely encode what any sensible programmer would do anyway.  As
possible uses occur quite frequently, this particular aspect of
interface construction has been standardized.


\chapter{Composite library}

The \thersixlibrary{} library is intended as a convenient import for
libraries where fine control over imported bindings is not necessary
or desirable. The \thersixlibrary{} library exports all bindings for
{\cf expand} as well as {\cf run} so that it is convenient for writing
{\cf syntax-case} macros as well as run-time code.

The \rsixlibrary{} library does not include a few select libraries:
%
\begin{itemize}
\item \rsixlibrary{eval}, as its presence may make creating
  self-contained programs more difficult;
\item \rsixlibrary{mutable-pairs}, as its absence from a program may enable compiler
  optimizations, and as mutable pairs might be deprecated in the future;
\item \rsixlibrary{mutable-strings}, for similar reasons as for
  \rsixlibrary{mutable-pairs};
\item \rsixlibrary{r5rs}, as its use is deprecated.
\end{itemize}

\chapter{Mutable pairs}

The presence of mutable pairs causes numerous problems:
%
\begin{itemize}
\item It complicates the specification of higher-procedures that
  operate on lists.
\item It inhibits certain compiler optimizations such as
  deforestation.
\item It complicates reasoning about programs that use lists.
\item It complicates the implementation of procedures that accept
  variable numbers of arguments.
\end{itemize}
%
However, removing mutable pairs from the language entirely would have
caused significant compatibility problems for existing code.  As a
compromise, the {\cf set-car!} and {\cf set-cdr!} procedures were
moved to a separate library.  This facilitates statically determining
if a program ever mutates pairs, encourages writing programs that do
not mutate pairs, and may help deprecating or removing mutable pairs
in the future.

\chapter{Mutable strings}

The presence of mutable strings problems similar to some of the
problems caused by the presence of mutable pairs.  Hence, the same
reasoning applies for moving the mutation operations into a separate
library.

%%%%%%%%%%%%%%%%%%%%%%%%%%%%%%%%%%%%%%%%%%%%%%%%%%%%%%%%%%%%%%%%

%\newpage                   %  Put bib on it's own page (it's just one)
%\twocolumn[\vspace{-.18in}]%  Last bib item was on a page by itself.
\renewcommand{\bibname}{References}

\bibliographystyle{plain}
\bibliography{abbrevs,rrs}

\end{document}

\chapter{Sorting}
\label{sortingchapter}

This chapter describes the \deflibrary{r6rs sorting} library for
sorting lists and vectors.

\begin{entry}{%
\proto{list-sort}{ proc list}{procedure}
\proto{vector-sort}{ proc vector}{procedure}}

\domain{\var{Proc} must be a procedure that accepts any two elements
  of the list or vector.  This procedure should not have any side
  effects.}  \var{Proc} returns a true value when its first argument
is strictly less than its second, and \schfalse{} otherwise.

The {\cf list-sort} and {\cf vector-sort} procedures perform a stable
sort of \var{list} or \var{vector}, without changing \var{list} or
\var{vector} in any way.  The {\cf list-sort} procedure returns a
list, and {\cf vector-sort} returns a vector.  The results may be {\cf
  eq?} to the argument when the argument is already sorted, and the
result of {\cf list-sort} may share structure with a tail of the
original list.  The sorting algorithm performs $O(n \lg n)$ calls to
the predicate where $n$ is the length of \var{list} or \var{vector},
and all arguments passed to the predicate are elements of the list or
vector being sorted, but the pairing of arguments and the sequencing
of calls to the predicate are not specified.

\implresp The implementation must check the restrictions
on \var{proc} to the extent performed by applying it as described.
\end{entry}

%%% Local Variables: 
%%% mode: latex
%%% TeX-master: "r6rs-lib"
%%% End: 

\chapter{Libraries}

The library system presented here is designed to let programmers share
libraries, i.e., code that is intended to be incorporated into larger
programs, and especially into programs that use library code from multiple
sources.  The library system supports macro definitions within libraries,
allows macro exports, and distinguishes the phases in which definitions
and imports are needed.  This chapter defines the notation for
libraries, a semantics for library expansion and execution, and a simple
format for sharing libraries.

Libraries addresses the following specific goals:

\begin{itemize}
\item Separate compilation and analysis; no two libraries have to be
  compiled at the same time (i.e., the meanings of two libraries
  cannot depend on each other cyclically, and compilation of two
  different libraries cannot rely on state shared across
  compilations), and significant program analysis does not require a
  whole program.
\item Independent compilation/analysis of unrelated libraries, where
  ``unrelated'' means that neither depends on the other through a
  transitive closure of imports.
\item Explicit declaration of dependencies, so that the meaning of
  each identifier is clear at compile time, and so that there is no
  ambiguity about whether a library needs to be executed for another
  library's compile time and/or run time.
\item Namespace management, so that different library producers are
  unlikely to define the same top-level name.
\end{itemize}

It does not address the following:

\begin{itemize}
\item Mutually dependent libraries.
\item Separation of library interface from library implementation.
\item Code outside of a library (e.g., {\cf 5} by itself as a program).
\item Local modules and local imports. 
\end{itemize}

\section{Library form}

A library declaration contains the following elements:

\begin{itemize}
\item a name for the library (possibly compound, with versioning),
\item a list of import dependencies, where each dependency specifies the
      following:
\begin{itemize}
\item the imported library's name,
\item the relevant phases, e.g., expand or run time, and
\item the subset of the library's exports to make available within the
      importing library, and the local names to use within the importing
      library for each of the library's exports,
\end{itemize}
\item a list of exports, which name a subset of the library's imports and
      definitions, and
\item a library body, consisting of a sequence of definitions followed
      by a sequence of expressions.
\end{itemize}

\section{Syntax and Semantics}

A library definition is written with the library form:

\begin{scheme}
(library \hyper{library~name}
  (export \hyper{export~spec} \ldots)
  (import \hyper{import~spec} \ldots)
  \hyper{library~body})%
\end{scheme}

The \hyper{library~name} specifies the name of the library, the
{\cf import} form specifies the imported bindings, and the
{\cf export} form specifies the exported bindings.
The \hyper{library~body} specifies the set of definitions, both for local
(unexported) and exported bindings, and the set of initialization
expressions (commands) to be evaluated for their effects.
The exported bindings may be defined within the library or imported into
the library.

An identifier can be imported from two or more libraries only if the
binding exported by each library is the same (i.e., the binding is
defined in one library, and it arrives throgh the imports only by
exporting and re-exporting).  Otherwise, no identifier can be imported
multiple times, defined multiple times, or both defined and imported.

A \hyper{library name} shall be one of the following:

\begin{scheme}
\hyper{identifier}
(\hyperi{identifier} \hyperii{identifier} \ldots \hyper{version})%
\end{scheme}

where \hyper{version} is empty or has the following form:
%
\begin{scheme}
(\hyperi{subversion} \hyperii{subversion} \ldots)%
\end{scheme}

Each \hyper{subversion} shall be an exact nonnegative integer.

As a \hyper{library name}, \hyper{identifier} is shorthand for {\cf
  (\hyper{identifier})}.

Each \hyper{import~spec} specifies a set of bindings to be imported into
the library, the phases in which they are to be available, and the local
names by which they are to be known.  A \hyper{import spec} shall
be one of the following:
%
\begin{scheme}
\hyper{import set}
(for \hyper{import~set} \hyper{import~phase} \ldots)%
\end{scheme}

An \hyper{import phase}  is one of the following:
\begin{scheme}
run
expand
(meta \hyper{level})%
\end{scheme}

where \hyper{level} is an exact nonnegative integer.

As an \hyper{import phase}, {\cf run} is an abbreviation for {\cf
  (meta 0)}, and {\cf expand} is an abbreviation for {\cf (meta 1)}.
Phases are discussed in section~\ref{phasessection}.

An \hyper{import~set} names a set of bindings from another library, and
possibly specifies local names for the imported bindings.  It shall be
one of the following:

\begin{scheme}
\hyper{library~reference}
(only \hyper{import~set} \hyper{identifier} \ldots)
(except \hyper{import~set} \hyper{identifier} \ldots)
(add-prefix \hyper{import~set} \hyper{identifier} \ldots)
(rename \hyper{import~set} (\hyper{identifier} \hyper{identifier} \ldots))%
\end{scheme}

A \hyper{library~reference} identifies a library by its (possibly compound)
name and optionally by its version.  It shall have one the following forms:

\begin{scheme}
\hyper{identifier}
(\hyperi{identifier} \hyperii{identifier} \ldots \hyper{version~reference})%
\end{scheme}

\hyper{Identifier} is shorthand for {\cf (\hyper{identifier})}.
A \hyper{version reference} shall have one of the following forms:

\begin{scheme}
\hyper{empty}
(\hyperi{subversion~reference} \hyperii{subversion~reference} \ldots)%
\end{scheme}

A \hyper{subversion reference} shall have one of the following forms:

\begin{scheme}
\hyper{subversion}
\hyper{subversion~condition}%
\end{scheme}

where a \hyper{subversion condition} shall have one of these forms:

\begin{scheme}
(>= \hyper{subversion})
(<= \hyper{subversion})
(and \hyperi{subversion~condition} \hyperii{subversion~condition} \ldots)
(or \hyperi{subversion~condition} \hyperii{subversion~condition} \ldots)
(not \hyper{subversion~condition})%
\end{scheme}


The sequence of identifiers in the importing library's
\hyper{library~reference} must match the sequence of identifiers in the
imported library's \hyper{library~name}.
The importing library's \hyper{version~reference} specifies a predicate on a
prefix of the imported library's \hyper{version}.
Each integer must match exactly and each condition has the expected meaning.
Everything beyond the prefix specified in the version reference matches
unconditionally.
When more than one library is identified by a library reference, the
choice of libraries is determined in some implementation-dependent manner.

To avoid problems such as incompatible types and replicated state, two
libraries whose library names contain the same sequence of identifiers but
whose versions do not match cannot co-exist in the same program.

By default, all of an imported library's exported bindings are made
visible within an importing library using the names given to the bindings
by the imported library.
The preceise set of bindings to be imported and the names of those
bindings can be adjusted with the {\cf only}, {\cf except},
{\cf add-prefix}, and {\cf rename} forms as described below.

\begin{itemize}
\item The {\cf only} form produces a subset of the bindings from another
\hyper{import~set}, including only the listed
\hyper{identifier}s; if any of the included \hyper{identifier}s is not in
\hyper{import~set}, an exception is raised.
\item The {\cf except} form produces a subset of the bindings from another
\hyper{import~set}, including all but the listed
\hyper{identifier}s; if any of the excluded \hyper{identifier}s is not in
\hyper{import~set}, an exception is raised.
\item The {\cf add-prefix} adds a prefix to each
name from another \hyper{import~set}.
\item The {\cf rename} form, for each pair of identifiers (\hyper{identifier}
\hyper{identifier}), removes a binding from the set from \hyper{import~set},
and adds it back with a different name. 
The first identifier is the original name, and the
second identifier is the new name. 
If the original name is not in \hyper{import~set}, or
if the new name is already in \hyper{import~set}, an exception is raised.
\end{itemize}

An \hyper{export~set} names a set of imported and locally defined bindings to
be exported fpr optionally specified phases, possibly with different
external names.  An \hyper{export spec} shall have one of the
following forms:

\begin{scheme}
\hyper{export~set}
(for (\hyper{export~set} \ldots) \hyper{import~phase} \ldots)%
\end{scheme}

where an \hyper{export set} shall have one of these forms:

\begin{scheme}
\hyper{identifier}
(rename (\hyper{identifier} \hyper{identifier}) \ldots)%
\end{scheme}

In an \hyper{export~set}, an \hyper{identifier} names a single binding defined
within the library or imported, where the external name for the export is
the same as the name of the binding within the library. 
A {\cf rename} set exports the binding named by the first
\hyper{identifier} in each pair, using the second \hyper{identifier} as the
external name.

The \hyper{library~body} of a {\cf library} form contains definitions for
local and exported bindings and initalization expressions to be evaluated
when the library is invoked.

A \hyper{library~body} is like a \hyper{body} (see section~\ref{bodiessection}) except that
\hyper{library~body}s need not include any expressions.  It shall
have the following form:

\begin{scheme}
\hyper{declaration} \ldots \hyper{definition} \ldots \hyper{expression} \ldots%
\end{scheme}

FIXME need to update syntax of {definition} to include {syntax definition}.

Note that {\cf begin} forms occurring in a body are spliced into the
body; see section~\ref{begin}.

The transformer expressions and transformer bindings are created
from left to right, as described in chapter~\ref{expansionchapter}.
The variable-definition right-hand-side expressions are evualated
from left to right, as if in an implicit {\cf letrec*},
and the body expressions are also evaluated from left to right
after the variable-definition right-hand-side expressions.
The location of each exported variable is then initialized to the value
of its local counterpart.
The effect of returning twice to the continuation of the last body
expression is unspecified.

The names {\cf library}, {\cf export}, {\cf import},
{\cf for}, {\cf run}, {\cf expand}, {\cf meta},
{\cf import}, {\cf export}, {\cf only}, {\cf except}, and
{\cf rename} appearing in the library syntax are part of the
syntax and are not reserved, i.e, the same can be used for other
purposes within the library or even exported from or imported 
into a library with different meanings, without affecting their
use in the {\cf library} form.

In the case of any ambiguities that arise from the use of one of
these names as a library name when using the shorthand (single-identifier)
\hyper{library~reference} syntax should be resolved in favor of the interpretation
given to the name by the library syntax.
For example, {\cf (import (for lib expand))} should be taken as
importing library {\cf lib} for {\cf expand}, not as importing
library {\cf (for lib expand)}.
The user can always eliminate such ambiguities by avoiding the shorthand
\hyper{library~reference} syntax when such an ambiguity might arise.

Bindings defined with a library are not visible in code that appears
outside of the library unless they are explicitly exported from the
library. 
An exported macro may, however, \emph{implicitly export} an identifier
defined within or imported into the library.
That is, it may insert a reference to that identifier into the output code
it produces.

All explicitly exported variables are immutable both in the exporting and
importing libraries.
All implicitly exported variables are mutable in the exporting library but
immutable in the importing libraries.
In consequence, any change after the initial assignment to the value of an
implicitly exported variable is not reflected by the references inserted
by an exported macro outside of the exporting library, and exported macros
may not insert assignments to implicitly exported variables outside of the
exporting library.

\section{Import phases}
\label{phasessection}

All bindings imported via a library's {\cf import} form are
\emph{visible} throughout the library's \hyper{library~body}.
An exception may be raised, however, if a binding is used out of its declared
phase(s):

\begin{itemize}
\item Bindings used in run-time code must be imported ``for {\cf run},''
which is equivalent to ``for {\cf (meta 0)}.''
\item Bindings used in the body of a transformer (appearing on the
right-hand-side of a transformer binding) in run-time code must be
imported ``for {\cf expand},'' which is equivalent to
``for {\cf (meta 1)},
\item Bindings used in the body of a transformer appearing within the body of a
transformer in run-time code must be imported ``for {\cf (meta 2)},''
and so on.
\end{itemize}

The valid import phases of an imported binding are determined by the enclosing
{\cf for} form, if any, in the {\cf import} form of the importing
library, in addition to the phase of the identifier in the exporting library.
An \hyper{import~set} without an enclosing {\cf for} is equivalent to
{\cf (for \hyper{import~set} run)}.

Import and export phases are combined by pairwise addition of all phase
combinations.  For example, references to an imported identifier exported
for phases $p_a$ and $p_b$ and imported for phases $q_a$, $q_b$, and $q_c$
are valid at phases $p_a+q_q$, $p_a+q_b$, $p_a+q_c$, $p_b+q_q$, $p_b+q_b$,
$nd p_b+q_c$.

The import phases implicitly determine when information about a
library must be available and also when the various forms contained within
a library must be evaluated.

Every library can be characterized by expand-time information (minimally,
its imported libraries, a list of the exported keywords, a list of the
exported variables, and code to evaluate the transformer expressions) and
run-time information (minimally, code to evaluate the variable definition
right-hand-side expressions, and code to evaluate the body expressions).
The expand-time information must be available to expand references to
any exported binding, and the run-time information must be available to
evaluate references to any exported variable binding.

If any of a library's bindings is imported by another library ``for
{\cf expand}'' (or for any meta level greater than 0), both expand-time and
run-time information for the first library is made available when the second
library is expanded.
If any of a library's bindings is imported by another library ``for
{\cf run},'' the expand-time information for the first library is made available when
the second library is expanded, and the run-time information for the first
library is made available when the run-time information for the second library
is made available.

We must also consider when the code to evaluate a library's transformer
expressions is executed and when the code to evaluate the library's
variable-definition right-hand-side expressions and body expressions is
executed.
We refer to executing the transformer expressions as \emph{visiting}
the library and to executing the variable-definition right-hand-side 
expressions and body expressions as \emph{invoking} the library.
A library must be visited before code that uses its bindings can be
expanded, and it must be invoked before code that uses its bindings can be
executed.
Visiting or invoking a library may also trigger the visiting or
invoking of other libraries.

More precisely, visiting a library at phase $N$ causes the system to:

\begin{itemize}
\item Visit at phase $N$ any library that is imported by this library
      ``for {\cf run}'' and that is not yet visited at phase $N$.
\item Visit at phase $N+M$ any library that is imported by this
      library ``for {\cf (meta \var{M})},'' $M>0$ and that is not yet
      visited at phase $N+M$.
\item Invoke at phase $N+M$ any library that is imported by this
      library ``for {\cf (meta \var{M})},'' $M>0$ and that is not yet
      invoked at phase $N+M$.
\item Evaluate the library's transformer expressions.
\end{itemize}

The order in which imported libraries are visited and invoked is not
defined, but imported libraries must be visited and invoked before the
library's transformer expressions are evaluated.

Similarly, invoking a library at meta phase $N$ causes the system to:

\begin{itemize}
\item Invoke at phase $N$ any library that is imported by this library
      ``for {\cf run}'' and that is not yet invoked at phase $N$.
\item Evaluate the library's variable-definition right-hand-side and body
      expressions.
\end{itemize}

The order in which imported libraries are invoked is not defined, but
imported libraries must be invoked before the library's variable-definition
right-hand-side and body expressions are evaluated.

An implementation is allowed to distinguish visits of a library across
different phases or to treat a visit at any phase as a visit at all
phases.
Similarly, an implementation is allowed to distinguish invocations of a
library across different phases or to treat an invocation at any phase as
an invocation at all phases.
An implementation is further allowed to start each expansion of a
{\cf library} form by removing all library bindings above phase 0.
Thus, a portable library's meaning must not depend on whether the
invocations are distinguished or preserved across phases or {\cf library}
expansions.

\section{Eval}
\label{evalsection}

FIXME: move to appropriate section

\begin{entry}{%
\proto{eval}{ expression environment}{procedure}}

\domain{\var{Expression} shall be a datum value whose external
  representation is a Scheme expression.  \var{Environment} shall be
  an environment specifier.  (See below.)}

Environments can be constructed with the {\cf environment} procedure,
which accepts a set of import specifiers represented as datums.

FIXME: what does it do
\end{entry}

\begin{entry}{%
\proto{environment}{ import-spec \dots}{procedure}}

\domain{\var{Import-spec} shall be a datum representing an
  \hyper{import spec}.}
The {\cf environment} procedure returns an environment corresponding
to \var{import-spec}

\begin{scheme}
(eval '(+ 3 4) (environment '(r6rs base))) \lev 7%
\end{scheme}

An exception with condition type FIXME is raised if the expand-time or
run-time information for a library named in one of the
\var{import-spec}s is not \emph{available} when the call to {\cf
  environment} occurs, in the sense of section~\ref{phasessection}.
\end{entry}

\section{Examples}

FIXME need some eval examples

Hello world:

\begin{schemenoindent}
(library hello
  (import (r6rs))
  (export)
  (display "Hello World")
  (newline))%
\end{schemenoindent}

Examples for various \hyper{import~spec}s and \hyper{export~spec}s:

\begin{scheme}
(library (stack)
  (import (r6rs))
  (export make push! pop! empty!)

  (define (make) (list '()))
  (define (push! s v) (set-car! s (cons v (car s))))
  (define (pop! s) (let ([v (caar s)])
                     (set-car! s (cdar s))
                     v))
  (define (empty! s) (set-car! s '())))

(library (balloons)
  (import (r6rs))
  (export make push pop)

  (define (make w h) (cons w h))
  (define (push b amt)
    (cons (- (car b) amt) (+ (cdr b) amt)))
  (define (pop b) (display "Boom! ") 
                  (display (* (car b) (cdr b))) 
                  (newline)))

(library (party)
  (import (r6rs)
          (only (stack)
                make push! pop!) ; not empty!
          (add-prefix (balloons) balloon:))
  ;; Total exports:
  ;; make, push, push!, make-party, pop!
  (export (rename (balloon:make make)
	          (balloon:push push))
	  push!
	  make-party
	  (rename (party-pop! pop!)))

  ;; Creates a party as a stack of balloons,
  ;; starting with two balloons
  (define (make-party)
    (let ([s (make)]) ; from stack
      (push! s (balloon:make 10 10))
      (push! s (balloon:make 12 9))
      s))
  (define (party-pop! p)
    (balloon:pop (pop! p))))


(library (main)
  (import (r6rs) (party))

  (define p (make-party))
  (pop! p)        ; displays "Boom! 108"
  (push! p (push (make 5 5) 1))
  (pop! p))       ; displays "Boom! 24"%
\end{scheme}

Examples for macros and phases:

\begin{schemenoindent}
(library (id-stuff)
  (import (r6rs))
  (export find-dup)

  (define (find-dup l)
    (and (pair? l)
         (let loop ((rest (cdr l)))
           (cond
            [(null? rest) (find-dup (cdr l))]
            [(bound-identifier=? (car l) (car rest)) 
                                 (car rest)]
            [else (loop (cdr rest))])))))

(library (values-stuff)
  (import (r6rs) (import (for (id-stuff) expand)))
  (export (for mvlet expand run))

  (define-syntax mvlet
    (lambda (stx)
      (syntax-case stx ()
        [(\_ [(id ...) expr] body0 body ...)
         (not (find-dup
                (syntax-object->list
                  (syntax (id ...)))))
         (syntax
           (call-with-values
               (lambda () expr) 
             (lambda (id ...) body0 body ...)))]))))

(library (let-div)
  (import (r6rs) (mvlet))
  (export let-div)

  (define (quotient+remainder n d)
    (let ([q (quotient n d)])
      (values q (- n (* q d)))))
  (define-syntax let-div
    (syntax-rules ()
     [(\_ n d (q r) body0 body ...)
      (mvlet [(q r) (quotient+remainder n d)]
        body0 body ...)])))%
\end{schemenoindent}


%%% Local Variables: 
%%% mode: latex
%%% TeX-master: "r6rs"
%%% End: 

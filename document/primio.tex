\section{Condition types}

In exceptional situations arising from ``I/O errors,'' the procedures
described in the specification below will raise an exception with
condition type {\cf\&i/o}.  Except where explicitly specified, there
is no guarantee that the raised condition object will contain all the
information that would be applicable. It is recommended, however, that
an implementation provide all information about an exceptional
situation in the condition object that is available at the place where
it is detected.

\begin{entry}{%
\ctproto{i/o}
\proto{i/o-error?}{ obj}{procedure}}

This condition type could be defined by
%
\begin{scheme}
(define-condition-type \&i/o \&error
  i/o-error?)
\end{scheme}        

This is a supertype for a set of more specific I/O errors.
\end{entry}   

\begin{entry}{%
\ctproto{i/o-read}
\proto{i/o-read-error?}{ obj}{procedure}}

\begin{scheme}
(define-condition-type \&i/o-read \&i/o
  i/o-read-error?)
\end{scheme}

This condition type describes read errors that occurred during an I/O
operation.
\end{entry}   

\begin{entry}{%
\ctproto{i/o-write}
\proto{i/o-write-error?}{ obj}{procedure}}

This condition type could be defined by
%
\begin{scheme}
(define-condition-type \&i/o-write \&i/o
  i/o-write-error?)
\end{scheme}
This condition type describes write errors that occurred during an I/O
    operation.
  \end{entry}   
  
\begin{entry}{%
\ctproto{i/o-invalid-position}
\proto{i/o-invalid-position-error?}{ obj}{procedure}}

This condition type could be defined by
%
\begin{scheme}
(define-condition-type \&i/o-invalid-position \&i/o
  i/o-invalid-position-error?
  (position i/o-error-position))
\end{scheme}

This condition type describes attempts to set the file position to an
invalid position. The value of the position field is the file position that
the program intended to set. This condition describes a range error, but
not a contract violation.
\end{entry}   

\begin{entry}{%
\ctproto{i/o-filename}
\proto{i/o-filename-error?}{ obj}{procedure}
\proto{i/o-error-filename}{ condition}{procedure}}

This condition type could be defined by
%
\begin{scheme}
(define-condition-type \&i/o-filename \&i/o
  i/o-filename-error?
  (filename i/o-error-filename))
\end{scheme}

This condition type describes an I/O error that occurred during an
operation on a named file. Condition objects belonging to this type
must specify a file name in the {\cf filename} field.
\end{entry}

\begin{entry}{%
\ctproto{i/o-file-protection}
\proto{i/o-file-protection-error?}{ obj}{procedure}}

This condition type could be defined by
%
\begin{scheme}
(define-condition-type \&i/o-file-protection
    \&i/o-filename
  i/o-file-protection-error?)
\end{scheme}

A condition of this type specifies that an operation tried to operate on a
named file with insufficient access rights.
\end{entry}   

\begin{entry}{%
\ctproto{i/o-file-is-read-only}
\proto{i/o-file-is-read-only-error?}{ obj}{procedure}}

This condition type could be defined by
%
\begin{scheme}
(define-condition-type \&i/o-file-is-read-only
    \&i/o-file-protection
  i/o-file-is-read-only-error?)
\end{scheme}

A condition of this type specifies that an operation tried to operate on a
named read-only file under the assumption that it is writeable.
\end{entry}   

\begin{entry}{%
\ctproto{i/o-file-already-exists}
\proto{i/o-file-already-exists-error?}{ obj}{procedure}}

This condition type could be defined by
%
\begin{scheme}
(define-condition-type \&i/o-file-already-exists
    \&i/o-filename
  i/o-file-already-exists-error?)
\end{scheme}
A condition of this type specifies that an operation tried to operate on an
existing named file under the assumption that it does not exist.
\end{entry}   

\begin{entry}{%
\ctproto{i/o-file-exists-not}
\proto{i/o-exists-not-error?}{ obj}{procedure}}

This condition type could be defined by
%
\begin{scheme}
(define-condition-type \&i/o-file-exists-not
    \&i/o-filename
  i/o-file-exists-not-error?)
\end{scheme}

A condition of this type specifies that an operation tried to operate on an
non-existent named file under the assumption that it exists.
\end{entry}   

\section{Primitive I/O}

This section defines the \library{r6rs i/o primitive} library, a
simple, primitive I/O subsystem.  It provides unbuffered I/O, and is
close to what a typical operating system offers. Thus, its interface
is suitable for implementing high-throughput and zero-copy I/O.

The Primitive I/O layer also allows clients to implement custom data
sources and sinks via a simple interface.

The Primitive I/O layer only handles blocking-I/O.

\subsection{Filenames}

Some of the procedures described here accept a filename filename as an
argument. Valid values for such a filename include strings naming a file using
the native notation of the operating system the Scheme implementation happens
to be running on.

It is expected that some implementations will extend this set of values by a more
abstract representation: This is necessary, as the most common operating
systems do not really use strings for representing filenames, but rather byte
or word sequences. Moreover, the string notation is difficult to manipulate and
not very portable.

A \var{filename} parameter name means that the
corresponding argument must be a filename.

\subsection{File options}
\label{fileoptionssection}

\mainindex{file options}
When opening a file, the various procedures in this library accept a
{\cf file-options} object containing a set of flags that specify how
the file is to be opened. A {\cf file-options} object is an enum-set
(see section~\ref{enumerationssection}) over the symbols constituting
valid file options.
A \var{file-options} parameter name means that the
corresponding argument must be a file-options object.

\begin{entry}{%
\proto{file-options}{ \hyper{file-options name} \dotsfoo}{\exprtype}}

Each \hyper{file-options name} must be a symbol.
The {\cf file-options} syntax returns a file-options object with the
specified options set. The following options (all affecting output
only) have predefined meaning:

\begin{itemize}   
\item {\cf create} create file if it does not already exist
\item {\cf exclusive} an exception with condition type
  {\cf\&i/o-file-already-exists} will be exclusive raised if this
  option and {\cf create} are both set and the file already exists
\item {\cf truncate}
  file is truncated
\end{itemize}

\hyper{File-options name}s other than those listed above may be used;
they have implementation-specific and thus potentially unpredictable
meaning.

The file-options object returned by {\cf (file-options)} specifies,
when supplied to an operation opening a file for output, that the file
must exist (otherwise an exception with condition type
{\cf\&i/o-file-exists-not} is raised) and its data is unchanged by the
operation.

\begin{rationale}
  The flags specified above represent only a common subset of
  meaningful options on popular platforms.  The {\cf file-options}
  form does not restrict the \hyper{file-options name}s to allow
  implementations to extend the file options by platform-specific
  flags.
\end{rationale}
\end{entry}   

\subsection{Readers and Writers}

The objects representing input data sources are called readers and
those representing output data sinks are called writers for output.
They are unbuffered and operate purely on binary data.  Although some
reader and writer objects might conceivably have something to do with
files or devices, programmers should never assume it.
A \var{reader} parameter name means that the
corresponding argument must be a reader.
A \var{writer} parameter name means that the
corresponding argument must be a writer.

The \library{r6rs i/o primitive} library has one condition type
specific to readers and writers:

\begin{entry}{%
\ctproto{i/o-reader/writer}
\proto{i/o-reader-writer-error?}{ obj}{procedure}
\proto{i/o-error-reader/writer}{ condition}{procedure}}

This condition type could be defined by

\begin{scheme}
(define-condition-type \&i/o-reader/writer \&i/o
  i/o-reader/writer-error?
  (reader/writer i/o-error-reader/writer))
\end{scheme}

This condition type allows specifying the particular reader or writer
with which an I/O error is associated. Conditions raised by the
Primitive I/O procedures may include an {\cf\&i/o-reader/writer}
condition, but are not required to do so.
\end{entry}

\subsection{I/O buffers}

\begin{entry}{%
\proto{make-i/o-buffer}{ size}{procedure}}

\domain{\var{Size} must be a non-negative exact integer.}
This creates a bytes object of size \var{size} with undefined contents.
Callers of the Primitive I/O procedures are encouraged to use bytes
objects created by {\cf make-i/o-buffer} because they might have
alignment and placement characteristics that {\cf make reader-read!}
and {\cf writer-write!} more efficient.  (These procedures are still
required to work on regular bytes objects, however.)
\end{entry}

\subsection{Readers}

The purpose of reader objects is to represent the output of arbitrary
algorithms in a form susceptible to imperative I/O.

\begin{entry}{%
\proto{reader?}{ obj}{procedures}}
   
Returns \schtrue{} if \var{obj} is a reader, otherwise returns \schfalse.
\end{entry}

\begin{entry}{%
\pproto{(make-simple-reader \var{id} \var{descriptor}}{procedure}}
\mainschindex{make-simple-reader}{\tt\obeyspaces\\
    \var{chunk-size} \var{read!} \var{available} \var{get-position}
    \var{set-position!} \var{end-position} \var{close})}

Returns a reader object. \var{Id} must be a string naming the reader,
provided for informational purposes only. \var{Descriptor} may be any
object; the Primitive I/O system will not use it internally for any
purpose. \var{Descriptor} can be extracted from the reader object via
{\cf reader-descriptor}. Thus, \var{descriptor} can be used to keep the
internal state of certain kinds of readers.
   
\var{Chunk-size} must be a positive exact integer, and represents a recommended
efficient size of the read operations on this reader. This is typically the
block size of the buffers of the operating system. As such, it is only a
hint for clients of the reader---calls to the \var{read!} procedure (see below)
may specify a different read count.
   
The remaining arguments are procedures.  For each such procedure, an
operation exists that will call that procedure.  For example, the {\cf
  reader-read!} operation, when called on a simple reader, will call
its \var{read!} procedure.
It is encouraged that these procedures check that their arguments are
of the appropriate types.  The operations that call them will perform
no checking beyond ensuring that their \var{reader} arguments are
indeed readers, and, if applicable, that the reader supports the operation.
These procedures must raise
exceptions with condition types as specified above when they encounter an
exceptional situation. When they do not, the effects are unspecified.

\var{Get-position}, \var{set-position!}, and \var{end-position} may be
omitted, in which case the corresponding arguments must be \schfalse.
   
\begin{itemize}
\item {\cf (\var{read!} \var{bytes} \var{start} \var{count})}
       
  \var{Start} and \var{count} must be non-negative exact integers.
  This reads up to \var{count} bytes from the reader and writes them
  into \var{bytes}, which must be a bytes object, starting at index
  start. \var{Bytes} must have size at least $\var{start} +
  \var{count}$. This procedure returns the number of bytes read as an
  exact integer. It returns 0 if it encounters an end of file, or if
  count is 0. If count is positive, this procedure blocks until at
  least one byte has been read or it has encountered end of file.
  
  \var{Bytes} may or may not be a bytes object returned by {\cf
    make-i/o-buffer}. It is possible that {\cf reader-read!} operates
  more efficiently if it is, however.
  
  \var{Count} may or may not be the same as the chunk size of the
  reader. It is possible that {\cf reader-read!} operates more
  efficiently if it is, however.

\item {\cf (\var{available})}
       
  This returns an estimate of the total number of bytes left in the
  reader. The return value is either an exact integer, or
  \schfalse{} if no such estimate is possible. There is no guarantee
  that this estimate will have any specific relationship to the true
  number of available bytes.

\item {\cf (\var{get-position})}
       
  This procedure, when present, returns the current position in the 
  reader as an exact integer counting the number of bytes since the
  beginning of the source. (EOFs do not count as bytes.)
  
\item {\cf (\var{set-position!} \var{pos})}
       
  This procedure, when present, moves to position \var{pos} (which
  must be a non-negative exact integer) in the reader.
       
\item {\cf (\var{end-position})}
       
  This procedure, when present, returns the position in the reader
  of the next end of file, without changing the current position.

\item {\cf (\var{close})}
       
  This procedure marks the reader as closed, performs any necessary
  cleanup, and releases the resources associated with the reader.
  Further operations on the reader must raise an exception with
  condition type {\cf\&contract}.
\end{itemize}

\end{entry}

\begin{entry}{%
\proto{reader-id}{ reader}}
   
This returns a string naming the reader, provided for informational
purposes only. For a file reader returned by {\cf open-file-reader} or
{\cf open-file-reader+writer}, this will be a string representation of the file
name.
  
For a reader created by {\cf make-simple-reader}, this returns the value that was
supplied as the \var{id} argument to {\cf make-simple-reader}.
\end{entry}

\begin{entry}{%
\proto{reader-descriptor}{ reader}{procedure}}
   
For a reader created by {\cf make-simple-reader}, this returns the value that was
supplied as the descriptor argument to {\cf make-simple-reader}.
   
For all other readers, this returns an unspecified value.
\end{entry}

\begin{entry}{%
\proto{reader-chunk-size}{ reader}{procedure}}
   
This returns a positive exact integer that represents a recommended
efficient size of the read operations on this reader. This is typically the
block size of the buffers of the operating system. As such, it is only a
hint for clients of the reader---calls to the {\cf reader-read!} procedure (see
below) may specify a different read count.

For a reader created by {\cf make-simple-reader}, this returns the
value that was supplied as the chunk-size argument to {\cf
  make-simple-reader}.
\end{entry}

\begin{entry}{%
\proto{reader-read!}{ reader bytes start count}{procedure}}
   
\domain{\var{Start} and \var{count} must be non-negative exact
  integers.  \var{Bytes} must have at least $\var{start} +
\var{count}$ elements.}
This reads up to \var{count} bytes from the reader and writes them into
\var{bytes} starting at index
\var{start}.  This procedure returns the number of bytes read
as an exact integer. It returns 0 if it encounters an end of file, or
if \var{count} is 0.  This procedure blocks until at least one byte has been
read or it has encountered end of file.
   
\var{Bytes} may or may not be a bytes object returned by {\cf
  make-i/o-buffer}. It is possible that {\cf reader-read!} operates
more efficiently if it is, however.
   
\var{Count} may or may not be the same as the chunk size of the
reader. It is possible that {\cf reader-read!} operates more
efficiently if it is, however.
   
For a reader created by {\cf make-simple-reader}, this tail-calls to
the \var{read!} procedure of reader with the remaining arguments.
\end{entry}

\begin{entry}{%
\proto{reader-available}{ reader}{procedure}}
   
This returns an estimate of the total number of available bytes left in the
stream. The return value is either an exact integer, or \schfalse{} if no such
estimate is possible. There is no guarantee that this estimate will have
any specific relationship to the true number of available bytes.
   
For a reader created by {\cf make-simple-reader}, this tail-calls the \var{available}
procedure of reader.
\end{entry}   

\begin{entry}{%
\proto{reader-has-get-position?}{ reader}{procedure}}
   
This returns \schtrue{} if reader supports the reader-get-position
procedure, and \schfalse{} otherwise.
\end{entry}

\begin{entry}{%
\proto{reader-get-position}{ reader}{procedure}}
   
When {\cf reader-has-get-position} returns \schtrue{} for reader, this
returns the current position in the byte stream as an exact integer
counting the number of bytes since the beginning of the stream.
   
For a reader created by {\cf make-simple-reader}, this tail-calls the
\var{get-position} procedure of reader, if present. If reader does not
have a \var{get-position} procedure, an exception with condition type
{\cf\&contract} is raised.
\end{entry}

\begin{entry}{%
\proto{reader-has-set-position!?}{ reader}{procedure}}
   
This returns \schtrue{} if reader supports the {\cf reader-set-position!} operation, and
\schfalse{} otherwise.
\end{entry}

\begin{entry}{%
\proto{reader-set-position!}{ reader pos}{procedure}}

\domain{\var{Pos} must be a non-negative exact integer.}
When {\cf reader-has-set-position!?} returns \schtrue{} for
\var{reader}, moves to position \var{pos} in the stream.
 
For a reader created by {\cf make-simple-reader}, this tail-calls the
\var{set-position!}  procedure of \var{reader} with the \var{pos}
argument, if present.  If \var{reader} does not support the operation,
an {\cf exception} with condition type {\cf\&contract} is raised.
For readers not created by {\cf make-simple-reader}, the return values
are unspecified.
\end{entry}

\begin{entry}{%
\proto{reader-has-end-position?}{ reader}{procedure}}
   
This returns \schtrue{} if reader supports the {\cf reader-end-position}
operation, and \schfalse{} otherwise.
\end{entry}   

\begin{entry}{%
\proto{reader-end-position}{ reader}{procedure}}
   
When {\cf reader-has-end-position?} returns \schtrue{} for \var{reader},
this returns the position in the byte stream of the next end of file,
without changing the current position.
   
For a reader created by {\cf mmake-simple-reader}, this tail-calls the
\var{end-position} procedure of \var{reader}, if present. If
\var{reader} does not have a \var{end-position} procedure, an
exception with condition type {\cf\&contract} is raised.
\end{entry}

\begin{entry}{%
\proto{reader-close}{ reader}{procedure}}
   
This marks \var{reader} as closed, performs any necessary cleanup, and
releases the resources associated with the reader. Further operations
on the reader must raise an exception with condition type
{\cf\&contract}.
   
For a reader created by {\cf make-simple-reader}, this tail-calls the
\var{close} procedure of reader.  For all other readers, the return
values are unspecified.
\end{entry}

\begin{entry}{%
\proto{open-bytes-reader}{ bytes}{procedure}}

This returns a \defining{bytes reader} that uses a copy of
\var{bytes}, a bytes object, as its contents. This reader supports
the {\cf reader-get-position}, {\cf reader-set-position!}, and \var{reader-end-position}
operations.
\end{entry}

\begin{entry}{%
\proto{open-file-reader}{ filename}{procedure}
\rproto{open-file-reader}{ filename file-options}{procedure}}
   
This returns a reader connected to the file named by \var{filename}.
The \var{file-options} object defaults to {\cf (file-options)} if not
present. It may determine various aspects of the returned reader, see
section~\ref{fileoptionssection}. This reader supports {\cf reader-get-position},
{\cf reader-set-position!}, and {\cf reader-end-position} operations.
\end{entry} 

\begin{entry}{%
\proto{standard-input-reader}{}{procedure}}
   
This returns a reader connected to the standard input. The meaning of
``standard input'' is implementation-dependent.
\end{entry}

\subsection{Writers}

The purpose of writer objects is to represent the input of arbitrary
algorithms in a form susceptible to imperative I/O.

\begin{entry}{%
\proto{writer?}{ obj}{procedure}}
   
Returns \schtrue{} if \var{obj} is a writer, otherwise returns \schfalse.
\end{entry}

\begin{entry}{%
\pproto{(make-simple-writer \var{id} \var{descriptor}}{procedure}}
\mainschindex{make-simple-writer}{\tt\obeyspaces\\
    \var{chunk-size} \var{write!} \var{get-position} \var{set-position!}
    \var{end-position} \var{close})}
   
Returns a writer object. \var{Id} must be a string naming the writer,
provided for informational purposes only. For a file, this will be a
string representation of the file name. \var{Descriptor} may be any
object; the Primitive I/O system will not use it internally for any
purpose. \var{Descriptor} can be extracted from the writer object via
{\cf writer-descriptor}. Thus, descriptor can be used to keep the
internal state of certain kinds of writers.

\var{Chunk-size} must be a positive exact integer, and is the
recommended efficient size of the write operations on this writer. As
such, it is only a hint for clients of the reader---calls to the
\var{write!} procedure (see below) may specify a different write
count.

The remaining arguments are procedures.  For each such procedure, an
operation exists that will call that procedure.  For example, the {\cf
  writer-write!} operation, when called on a simple writer, will call
its \var{write!} procedure.
It is encouraged that these procedures check that their arguments are
of the appropriate types.  The operations that call them will perform
no checking beyond ensuring that their \var{writer} arguments are
indeed writers, and, if applicable, that the writer supports the operation.
These procedures must raise
exceptions with condition types as specified above when they encounter an
exceptional situation. When they do not, the effects are unspecified.

\var{Get-position}, \var{set-position!}, and \var{end-position} may be
omitted, in which case the corresponding arguments must be \schfalse.

\begin{itemize}
\item {\cf (\var{write!} \var{bytes} \var{start} \var{count})}
   
  \var{Start} and \var{count} must be non-negative exact integers.
  This writes up to \var{count} bytes in bytes object \var{bytes}
  starting at index \var{start}. Before it does this, it will block
  until it can write at least one byte. It returns the number of bytes
  actually written as a positive exact integer.
  
  \var{Bytes} may or may not be a bytes object returned by {\cf
    make-i/o-buffer}. It is possible that {\cf writer-write!} operates more
  efficiently if it is, however.
  
  \var{Count} may or may not be the same as the chunk size of the
  reader. It is possible that {\cf writer-write!} operates more
  efficiently if it is, however.

\item {\cf (\var{get-position})}
   
  This procedure, when present, returns the current position in the byte
  stream as an exact integer counting the number of bytes since the
  beginning of the stream.
   
\item {\cf (\var{set-position!} \var{pos})}
   
  This procedure, when present, moves to position \var{pos} (which must be a
  non-negative exact integer) in the stream.
  
\item {\cf (\var{end-position})}
   
    This procedure, when present, returns the byte position of the next end
    of file, without changing the current position.
   
\item {\cf (\var{close})}
  
  This procedure marks the writer as closed, performs any necessary
  cleanup, and releases the resources associated with the writer. Further
  operations on the writer must raise an exception with condition type
  {\cf\&contract}.
\end{itemize}
\end{entry}

\begin{entry}{%
\proto{writer-id}{ writer}{procedure}}
   
This returns string naming the writer, provided for informational
purposes only. For a file writer returned by {\cf open-file-writer}
or {\cf open-file-reader+writer}, this will be a string
representation of the file name.
   
For a writer created by {\cf make-simple-writer}, this returns the
value of the \var{id} field of the argument writer.
\end{entry}

\begin{entry}{%
\proto{writer-descriptor}{ writer}{procedure}}
   
  For a writer created by {\cf make-simple-writer}, this returns the value of the
  \var{descriptor} field of the argument writer.
  
  For all other writers, this returns an unspecified value.
\end{entry}

\begin{entry}{%
\proto{writer-chunk-size}{ writer}{procedure}}
   
This returns a positive exact integer, and is the recommended efficient
size of the write operations on this writer. As such, it is only a hint for
clients of the reader---calls to {\cf writer-write!} (see below) may specify a
different write count.

For a writer created by {\cf make-simple-writer}, this returns the value of the
\var{chunk-size} field of the argument writer.
\end{entry}

\begin{entry}{%
\proto{writer-write!}{ writer bytes start count}{procedure}}
   
\domain{\var{Start} and \var{count} must be non-negative exact
  integers. \var{Bytes} must have at least $\var{start} +
\var{count}$ elements.}
This writes up to \var{count} bytes in bytes object \var{bytes}
starting at index \var{start}. Before it does this, it will block
until it can write at least one byte. It returns the number of bytes
actually written as a positive exact integer.
   
\var{Bytes} may or may not be a bytes object returned by {\cf
  make-i/o-buffer}. It is possible that {\cf writer-write!} operates
more efficiently if it is, however.

\var{Count} may or may not be the same as the chunk size of the reader. It is
possible that {\cf writer-write!} operates more efficiently if it is, however.

For a writer created by {\cf make-simple-writer}, this tail-calls the
\var{write!} procedure of writer with the remaining arguments.  For
all other writers, the return values are unspecified.
\end{entry}   

\begin{entry}{%
\proto{writer-has-get-position?}{ writer}{procedure}}
   
This returns \schtrue{} if writer supports the {\cf
  writer-get-position} operation, and \schfalse{} otherwise.
\end{entry}   

\begin{entry}{%
\proto{writer-get-position}{ writer}{procedure}}
   
When {\cf writer-has-get-position?} returns \schtrue{} for
\var{writer}, this returns the current position in the byte stream as
an exact integer counting the number of bytes since the beginning of
the stream.
  
For a writer created by {\cf make-simple-writer}, this calls the
\var{get-position} procedure of \var{writer}, if present. If \var{writer}
does not have a get-position procedure, an exception with condition
type {\cf\&contract} is raised.
\end{entry}

\begin{entry}{%
\proto{writer-has-set-position!?}{ writer}{procedure}}
   
This returns \schtrue{} if writer supports the \var{writer-set-position!} operation, and
\schfalse{} otherwise.
\end{entry}

\begin{entry}{%
\proto{writer-set-position!}{ writer pos}{procedure}}
   
When {\cf writer-has-set-position!?} returns \schtrue{} for
\var{writer}, this moves to position \var{pos} (which must be a non-negative
exact integer) in the stream.
   
For a writer created by {\cf make-simple-writer}, this calls the
\var{set-position!}  procedure of \var{writer} with the \var{pos}
argument, if present. If \var{writer} does not have a
\var{set-position!} procedure, an exception with condition type
{\cf\&contract} is raised.  For writers not created by {\cf
  make-simple-writer}, the return values are unspecified.
\end{entry}

\begin{entry}{%
\proto{writer-has-end-position?}{ writer}{procedure}}
   
This returns \schtrue{} if \var{writer} supports the {\cf
  writer-end-position} operation, and \schfalse{} otherwise.
\end{entry}

\begin{entry}{%
\proto{writer-end-position}{ writer}{procedure}}
   
When {\cf writer-has-end-position?} returns \schtrue{} for writer,
this returns the byte position of the next end of file, without
changing the current position.
   
For a writer created by {\cf make-simple-writer}, this calls the
\var{end-position} procedure of \var{writer}, if present. If \var{writer}
does not have a \var{end-position} procedure, an exception with condition
type {\cf\&contract} is raised.
\end{entry}

\begin{entry}{%
\proto{writer-close}{ writer}{procedure}}
   
This marks the writer as closed, performs any necessary cleanup, and
releases the resources associated with the writer. Further operations
on the writer must raise an exception with condition type
{\cf\&contract}.
   
For a writer created by {\cf make-simple-writer}, this calls the
\var{close} procedure of \var{writer}.  For all other writers,
the return values are unspecified.
\end{entry}

\begin{entry}{%
\proto{open-bytes-writer}{}{procedure}}
   
This returns a \defining{bytes writer} that can yield everything written to it as
a bytes object. This writer supports the {\cf writer-get-position},
{\cf writer-set-position!}, and {\cf writer-end-position} operations.
operations.
\end{entry}

\begin{entry}{%
\proto{writer-bytes}{ writer}{procedure}}
   
The \var{writer} argument must be a bytes writer.  This procedure
returns a newly allocated bytes object containing the data written to
writer in sequence. Doing this in no way invalidates the writer or
change its store.
\end{entry}

\begin{entry}{%
\proto{open-file-writer}{ filename}{procedure}
\rproto{open-file-writer}{ filename file-options}{procedure}}
   
This returns a writer connected to the file named by \var{filename}.
The \var{file-options} object defaults to {\cf (file-options)} if not
present. It determines various aspects of the returned writer, see
section~\ref{fileoptionssection}.  This writer supports the {\cf writer-get-position},
{\cf writer-set-position!}, and {\cf writer-end-position} operations.
\end{entry}

\begin{entry}{%
\proto{standard-output-writer}{}{procedure}}
   
This returns a writer connected to the standard output. The meaning of
``standard output'' is implementation-dependent.
\end{entry}

\begin{entry}{%
\proto{standard-error-writer}{}{procedure}}
   
This returns a writer connected to the standard error. The meaning of
``standard error'' is implementation-dependent.
\end{entry}   

\subsection{Opening files for reading and writing}

\begin{entry}{%
\proto{open-file-reader+writer}{ filename}{procedure}
\rproto{open-file-reader+writer}{ filename file-options}{procedure}}
   
This returns two values, a reader and a writer connected to the file
named by \var{filename}. The \var{file-options} object defaults to
{\cf (file-options)} if not present. It determines various aspects of
the returned writer and possibly the reader, see
section~\ref{fileoptionssection}. The reader and the writer support
the {\cf reader-get-position}, {\cf reader-set-position!}, {\cf
  reader-end-position}, and the {\cf writer-get-position}, {\cf
  writer-set-position!}, {\cf writer-end-position} operations,
respectively.

\begin{note}
  This procedure enables opening a file for simultaneous input and
  output in environments where it is not possible to call
  open-file-reader and open-file-writer on the same file.
\end{note}     
\end{entry}

\subsection{Examples}

\begin{schemenoindent}
; Algorithmic reader producing an infinite
; stream of blanks:

(define (make-infinite-blanks-reader)
  (make-simple-reader
    "<blanks, blanks, and more blanks>"
    \schfalse{}
    4096
    (lambda (bytes start count)
      (let loop ((index 0))
        (if (>= index count)
            index
            (begin
              (bytes-u8-set! bytes (+ start index) 32)
              (loop (+ 1 index))))))
    (lambda ()
      1000) ; some number
    \schfalse{} \schfalse{} \schfalse{}
    unspecific))
 
; Sample implementation of bytes writer

(define-record-type (buffer
                     make-buffer buffer?)
  (fields (mutable bytes
                   buffer-bytes set-buffer-bytes!)
          (mutable size
                   buffer-set-size set-buffer-size!)))

(define (open-bytes-writer)
  (let ((buffer
         (make-buffer (make-bytes 512) 0))
        (pos 0))

    (define (ensure-open)
      (if (not buffer)
          (raise (condition
                  (\&message
                   (message "bytes writer closed"))
                  (\&contract)
                  (\&i/o-reader/writer
                   (reader/writer writer))))))

    (define writer
      (make-writer
       "<bytes writer>"
       buffer
       3
       (lambda (bytes start count)
         (ensure-open)
         ;; resize buffer if necessary
         (let loop ((length 
                     (bytes-length
                       (buffer-bytes buffer))))
           (cond
            ((> (+ pos count) length)
             (loop (* 2 length)))
            ((> length (bytes-length (car buffer)))
             (let ((new-buffer (make-bytes length)))
               (bytes-copy! (buffer-bytes buffer) 0
                            new-buffer 0
                            (buffer-size buffer))
               (set-buffer-bytes! buffer new-buffer)))))

         (bytes-copy! bytes start
                      (buffer-bytes buffer) pos
                      count)
         (set-buffer-size!
           buffer
           (max (buffer-size buffer) (+ pos count)))
         (set! pos (+ pos count))
         count)
       (lambda ()
         (ensure-open)
         pos)
       (lambda (new-pos)
         (ensure-open)
         (if (<= new-pos (buffer-size buffer))
             (set! pos new-pos)
             (raise
              (condition
               (\&message
                (message "invalid position"))
               (\&i/o-invalid-position
                (position new-pos))))))
       (lambda ()
         (ensure-open)
         (buffer-size buffer))
       (lambda ()
         (set-buffer-bytes! buffer \schfalse{}))))
    writer))

(define (writer-bytes writer)
  (let* ((buffer (writer-descriptor writer))
         (target (make-bytes (buffer-size buffer))))
    (bytes-copy! (buffer-bytes buffer) 0
                 target 0
                 (buffer-size buffer))
    target))
\end{schemenoindent}
                
%%% Local Variables: 
%%% mode: latex
%%% TeX-master: "r6rs"
%%% End: 

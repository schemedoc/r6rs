%\vfill\eject
\chapter{Semantic concepts}
\label{basicchapter}

\section{Variables, syntactic keywords, and regions}
\label{specialformsection}
\label{variablesection}

In a library body,
an identifier\index{identifier} may name a type of syntax, or it may name
a location where a value can be stored.  An identifier that names a type
of syntax is called a {\em syntactic keyword}\mainindex{syntactic keyword}
and is said to be {\em bound} to that syntax.  An identifier that names a
location is called a {\em variable}\mainindex{variable} and is said to be
{\em bound} to that location.  The set of all visible
bindings\mainindex{binding} in effect at some point in a program is
known as the {\em environment} in effect at that point.  The value
stored in the location to which a variable is bound is called the
variable's value.  By abuse of terminology, the variable is sometimes
said to name the value or to be bound to the value.  This is not quite
accurate, but confusion rarely results from this practice.

\todo{Define ``assigned'' and ``unassigned'' perhaps?}

\todo{In programs without side effects, one can safely pretend that the
variables are bound directly to the arguments.  Or:
In programs without \ide{set!}, one can safely pretend that the
variable is bound directly to the value. }

\vest Certain expression types are used to create new kinds of syntax
and bind syntactic keywords to those new syntaxes, while other
expression types create new locations and bind variables to those
locations.  These expression types are called {\em binding constructs}.
\mainindex{binding construct}
The constructs in the base library that bind syntactic keywords are listed in section~\ref{macrosection}.
The most fundamental of the variable binding constructs is the
{\cf lambda} expression, because all other variable binding constructs
can be explained in terms of {\cf lambda} expressions.  The other
variable binding constructs are {\cf let}, {\cf let*}, {\cf letrec*},
{\cf letrec}, {\cf let-values}, {\cf let*-values}, {\cf do}, and {\cf
  case-lambda} expressions (see sections~\ref{lambda}, \ref{letrec}, 
\ref{do}, and \ref{case-lambda}).

%Note: internal definitions not mentioned here.

\vest Like Algol and Pascal, and unlike most other dialects of Lisp
except for Common Lisp, Scheme is a statically scoped language with
block structure.  To each place where an identifier is bound in a program
there corresponds a \defining{region} of the program text within which
the binding is visible.  The region is determined by the particular
binding construct that establishes the binding; if the binding is
established by a {\cf lambda} expression, for example, then its region
is the entire {\cf lambda} expression.  Every mention of an identifier
refers to the binding of the identifier that established the
innermost of the regions containing the use.  If there is no binding of
the identifier whose region contains the use, then the use refers to the
binding for the variable in the top level environment of the library
body or a binding imported from another library.  If any
(FIXME chapters~\ref{modulelibraries}); if there is no
binding for the identifier,
it is said to be \defining{unbound}.\mainindex{bound}

\section{Multiple return values}

A Scheme expression can evaluate to an arbitrary finite number of
values.  These values are passed to the expression's continuation.

Not all continuations accept any number of values: A continuation that
accepts the argument to a procedure call is guaranteed to accept
exactly one value.  The effect of passing such a continuation a
different number of values is unspecified.  The {\cf call-with-values}
described in section~\ref{controlsection} allows creating
continuations that specified numbers of return values.  If a number of
return values is passed to a continuation created by {\cf
  call-with-values} not accepted by its \var{consumer} an exception is
raised.

A number of forms in the base library have sequences of expressions
as subforms that are evaluated sequentially, with the return values of
all but the last expression being discarded.  The continuations
discarding these values accept any number of values.

\section{Exceptional situations}
\label{exceptionalsituationsection}

\mainindex{exceptional situation}A variety of exceptional situations
are distinguished in this report, among them violations of program
syntax, violations of a procedure's specification, violations of
implementation restrictions, and exceptional situations in the
environment.  When an exception is raised, an object is provided that
describes the nature of the exceptional siutation.  The report uses
the condition system described in section~\ref{conditionssection} to
describe exceptional situations, classifying them by condition types.

For most exceptional situations where an exception is raised, the
program cannot continue at the place the situation was detected.  In
that case, the exception handler invoked by the exception must not
return.  In some cases, however, continuing is permissible; the
handler may return.  See~\ref{exceptionssection}.

The above requirements for violations and implementation restrictions
only apply in \textit{safe mode}.  Implementations may not raise
exceptions in those situations in \textit{unsafe mode}.  The
distinction is explained in section~\ref{safeunsafemodesection}.

Implementation restrictions indicate circumstances under which an
implementation is permitted to raise an exception if it is unable to
continue execution of a correct program because of some restriction
imposed by the implementation.

Some possible implementation restrictions
such as the lack of representations for NaNs and infinities (see
section~\ref{infinitiesnanssection}) are covered by this report, and
implementations must raise an exception of the appropriate condition
type if it encounters such a situation.

Implementation restrictions not explicitly covered in this report are
of course discouraged, but implementations are encouraged to report
violations of implementation restrictions.\mainindex{implementation
  restriction} For example, an implementation may raise an exception
with condition type {\cf\&implementation-restriction} if it does not
have enough storage to run a program.


\section{Safety}

The standard libraries whose exports are described by this document
are said to be \defining{safe libraries}.  Libraries that import only
from safe libraries, and do not contain any {\cf (safe 0)} or {\cf
  unsafe} declarations (see section~\ref{declarationssection}, are
also said to be safe libraries.  A script is said to be safe if and
only if its library part is a safe library.

As defined by this document, the Scheme programming language
is safe in the following sense:
If a Scheme script is said to be safe, then its execution
cannot go so badly wrong as to crash or to continue to
execute while behaving in ways that are
inconsistent with the semantics described in this document,
unless said execution first encounters some implementation
restriction or other defect in the implementation of Scheme
that is executing the script.

Violations of an implementation restriction must raise an
exception with condition type {\cf\&implementation-restriction},
as must all
violations and errors that would otherwise threaten system
integrity in ways that might result in execution that is
inconsistent with the semantics described in this document.

The above safety properties are guaranteed only for scripts
and libraries that are said to be safe.  Implementations
may provide access to unsafe libraries, and may interpret
{\cf (safe 0)} and {\cf unsafe} declarations in ways that
cannot guarantee safety.


\section{Storage model}
\label{storagemodel}

Variables and objects such as pairs, vectors, and strings implicitly
denote locations\mainindex{location} or sequences of locations.  A string, for
example, denotes as many locations as there are characters in the string. 
(These locations need not correspond to a full machine word.) A new value may be
stored into one of these locations using the {\tt string-set!} procedure, but
the string continues to denote the same locations as before.

An object fetched from a location, by a variable reference or by
a procedure such as {\cf car}, {\cf vector-ref}, or {\cf string-ref}, is
equivalent in the sense of \ide{eqv?} % and \ide{eq?} ??
(section~\ref{equivalencesection})
to the object last stored in the location before the fetch.

Every location is marked to show whether it is in use.
No variable or object ever refers to a location that is not in use.
Whenever this report speaks of storage being allocated for a variable
or object, what is meant is that an appropriate number of locations are
chosen from the set of locations that are not in use, and the chosen
locations are marked to indicate that they are now in use before the variable
or object is made to denote them.

In many systems it is desirable for constants\index{constant} (i.e. the values of
literal expressions) to reside in read-only-memory.  To express this, it is
convenient to imagine that every object that denotes locations is associated
with a flag telling whether that object is mutable\index{mutable} or
immutable\index{immutable}.  In such systems literal constants and the strings
returned by \ide{symbol->string} are immutable objects, while all objects
created by the other procedures listed in this report are mutable.  An
attempt to store a new value into a location that is denoted by an
immutable object should raise an exception.

\section{Proper tail recursion}
\label{proper tail recursion}

Implementations of Scheme are required to be
{\em properly tail-recursive}\mainindex{proper tail recursion}.
Procedure calls that occur in certain syntactic
contexts defined below are `tail calls'.  A Scheme implementation is
properly tail-recursive if it supports an unbounded number of active
tail calls.  A call is {\em active} if the called procedure may still
return.  Note that this includes calls that may be returned from either
by the current continuation or by continuations captured earlier by
{\cf call-with-current-continuation} that are later invoked.
In the absence of captured continuations, calls could
return at most once and the active calls would be those that had not
yet returned.
A formal definition of proper tail recursion can be found
in~\cite{propertailrecursion}.  The rules for identifying tail calls
in base-library constructs are described in
section~\ref{basetailcontextsection}.

\begin{rationale}

Intuitively, no space is needed for an active tail call because the
continuation that is used in the tail call has the same semantics as the
continuation passed to the procedure containing the call.  Although an improper
implementation might use a new continuation in the call, a return
to this new continuation would be followed immediately by a return
to the continuation passed to the procedure.  A properly tail-recursive
implementation returns to that continuation directly.

Proper tail recursion was one of the central ideas in Steele and
Sussman's original version of Scheme.  Their first Scheme interpreter
implemented both functions and actors.  Control flow was expressed using
actors, which differed from functions in that they passed their results
on to another actor instead of returning to a caller.  In the terminology
of this section, each actor finished with a tail call to another actor.

Steele and Sussman later observed that in their interpreter the code
for dealing with actors was identical to that for functions and thus
there was no need to include both in the language.

\end{rationale}

%%% Local Variables: 
%%% mode: latex
%%% TeX-master: "r6rs"
%%% End: 

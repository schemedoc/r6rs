%\vfill\eject
\chapter{Semantic concepts}
\label{basicchapter}

\section{Programs and libraries}

A Scheme program consists of a \textit{top-level program\index{top-level program}}
together with a set of \textit{libraries\index{library}}, each
of which defines a part of the program connected to the others through
explicitly specified exports and imports.  A library consists of a set
of export and import specifications and a body, which consists of
definitions, and expressions.
A top-level program is similar to a library, but
has no export specifications.
Chapters~\ref{librarychapter} and \ref{programchapter}
describe the syntax and semantics of libraries and top-level programs,
respectively. 
Chapter~\ref{baselibrarychapter} describes a base
library that defines many of the constructs traditionally associated with
Scheme.
A separate report~\cite{R6RS-libraries}
describes the various \textit{standard libraries}\index{standard
  library} provided by a Scheme system.

The division between the base library and the other standard libraries is
based on use, not on construction.  In particular, some facilities
that are typically implemented as ``primitives'' by a compiler or the
run-time system rather than in terms of other standard procedures
 or syntactic forms are not part of the base library, but are defined in
separate libraries.  Examples include the fixnums and flonums libraries,
the exceptions and conditions libraries, and the libraries for
records.

\section{Variables, keywords, and regions}
\label{specialformsection}
\label{variablesection}

Within the body of a library or top-level program,
an identifier\index{identifier} may name a kind of syntax, or it may name
a location where a value can be stored.  An identifier that names a kind
of syntax is called a {\em keyword}\mainindex{keyword}, or {\em syntactic keyword}\mainindex{syntactic keyword},
and is said to be {\em bound} to that kind of syntax (or, in the case of a
syntactic abstraction, a {\em transformer} that translates the syntax into more
primitive forms; see section~\ref{macrosection}).  An identifier that names a
location is called a {\em variable}\mainindex{variable} and is said to be
{\em bound} to that location.  
At each point within a top-level program or a library, a specific, fixed set
of identifiers is bound.  The set of these identifiers, the set of \textit{visible
bindings}\mainindex{binding}, is
known as the {\em environment} in effect at that point.

Certain forms are used to create syntactic abstractions
and to bind keywords to transformers for those new syntactic abstractions, while other
forms create new locations and bind variables to those
locations.  Collectively, these forms are called {\em binding
  constructs}.\mainindex{binding construct}
Some binding constructs take the form of
\textit{definitions}\index{definition}, while others are
expressions.
With the exception of exported library bindings, a binding created
by a definition is visible only within the body in which the
definition appears, e.g., the body of a library, top-level program,
or {\cf lambda} expression.
Exported library bindings are also visible within the bodies of
the libraries and top-level programs that import them (see
chapter~\ref{librarychapter}).

Expressions that bind variables include the {\cf lambda},
{\cf let}, {\cf let*}, {\cf letrec}, {\cf letrec*}, {\cf let-values},
and {\cf let*-values} forms from the base library (see
sections~\ref{lambda}, \ref{letrec}).
Of these, {\cf lambda} is the most fundamental.
Variable definitions appearing within the body of 
such an expression, or within the bodies of a library or top-level
program, are treated as a set of
{\cf letrec*} bindings.
In addition, for library bodies, 
the variables exported from the library can be referenced by
importing libraries and top-level programs.

Expressions that bind keywords include the {\cf
  let-syntax} and {\cf letrec-syntax} forms (see
section~\ref{bindsyntax}).  A {\cf define} form (see section~\ref{define}) is a
definition that creates a variable binding (see 
section~\ref{defines}), and a {\cf define-syntax} form is
a definition that creates a keyword binding (see
section~\ref{syntaxdefinitionsection}).

\vest Scheme is a statically scoped language with
block structure.  To each place in a top-level program or library body where an identifier is bound 
there corresponds a \defining{region} of code within which
the binding is visible.  The region is determined by the particular
binding construct that establishes the binding; if the binding is
established by a {\cf lambda} expression, for example, then its region
is the entire {\cf lambda} expression.  Every mention of an identifier
refers to the binding of the identifier that established the
innermost of the regions containing the use.  If a use of an
identifier appears in a place where none of the surrounding expressions
contains a binding for the identifier, the use may refer to a
binding established by a definition or import at the top of the
enclosing library or top-level program
(see chapter~\ref{librarychapter}).
If there is no binding for the identifier,
it is said to be \defining{unbound}.\mainindex{bound}

\section{Exceptional situations}
\label{exceptionalsituationsection}

\mainindex{exceptional situation}A variety of exceptional situations
are distinguished in this report, among them violations of syntax,
violations of a procedure's specification, violations of
implementation restrictions, and exceptional situations in the
environment.  When an exceptional situation is detected by the
implementation, an \textit{exception is raised}\mainindex{raise},
which means that a special procedure called the \textit{current
  exception handler} is called.  A program can also raise an
exception, and override the current exception handler; see
section~\extref{lib:exceptionssection}.

When an exception is raised, an object is provided that
describes the nature of the exceptional situation.  The report uses
the condition system described in library section~\extref{lib:conditionssection}{Conditions} to
describe exceptional situations, classifying them by condition types.

Some exceptional situations allow continuing the program if the
exception handler takes appropriate action.  The corresponding
exceptions are called \textit{continuable}\index{continuable exception}.
For most of the exceptional situations described in this report,
portable programs cannot rely upon the exception being continuable
at the place where the situation was detected.
For those exceptions, the exception handler that is invoked by the
exception should not return.
In some cases, however, continuing is permissible, and the
handler may return.  See library section~\extref{lib:exceptionssection}{Exceptions}.

Implementations must raise an exception
when they are unable to continue correct execution
of a correct program due to some \defining{implementation restriction}.  For
example, an implementation that does not support infinities
must raise an exception with condition type
{\cf\&implementation-restriction} when it evaluates an expression
whose result would be an infinity.

Some possible implementation restrictions such as the lack of
representations for NaNs and infinities (see
section~\ref{infinitiesnanssection}) are anticipated by this report,
and implementations typically must raise an exception of the
appropriate condition type if they encounter such a situation.

This report uses the phrase ``an exception is raised'' synonymously
with ``an exception must be raised''.
This report uses the phrase ``an exception with condition type \var{t}''
to indicate that the object provided with the
exception is a condition object of the specified type.
The phrase ``a continuable exception is raised'' indicates an
exceptional situation that permits the exception handler to return.

\section{Argument checking}
\label{argumentcheckingsection}

\mainindex{argument checking}
Many procedures specified in this report or as part of a
standard library restrict the arguments they accept.
Typically, a procedure accepts only specific numbers and types of arguments.
Many syntactic forms similarly restrict the values to which one or
more of their subforms can evaluate.
These restrictions imply responsibilities\mainindex{responsibility} for
both the programmer and the implementation.
Specifically, the programmer is responsible for ensuring
that the values indeed adhere to the restrictions described
in the specification.  The implementation must check
that the restrictions in the specification are indeed met, to the
extent that it is reasonable, possible, and necessary to allow the
specified operation to complete successfully.  The implementation's
responsibilities are specified in more detail in
chapter~\ref{entryformatchapter} and throughout the report.

Note that it is not always possible for an implementation to completely check
the restrictions set forth in a specification.  For example, if an
operation is specified to accept a procedure with specific properties,
checking of these properties is undecidable in general.  Similarly,
some operations accept both lists and procedures that are
called by these operations.  Since lists can be mutated by the procedures
through the \rsixlibrary{mutable-pairs} library (see library
chapter~\extref{lib:pairmutationchapter}{Mutable pairs}), an argument that is a list
when the operation starts may become a non-list during the execution of the operation.
Also, the procedure might escape to a different continuation,
preventing the operation from performing more checks.
Requiring the operation to check that the argument is a list after
each call to such a procedure would be impractical.  Furthermore, some
operations that accept lists only need to traverse these lists
partially to perform their function; requiring the implementation to
traverse the remainder of the list to verify that all specified
restrictions have been met might
violate reasonable performance assumptions.  For these reasons, the
programmer's obligations may exceed the checking obligations of the
implementation.

When an implementation detects a violation of a restriction for an
argument, it must raise an exception with condition type
{\cf\&assertion} in a way consistent with the safety of execution as
described in the next section.

\section{Syntax violations}

The subforms of a special form usually need to obey certain syntactic
restrictions.  As forms may be subject to macro expansion, which may
not terminate, the question of whether they obey the specified
restrictions is undecidable in general.

When macro expansion terminates, however, implementations must detect
violations of the syntax.  A \defining{syntax violation} is an error
with respect to the syntax of library bodies, top-level bodies,
or the ``\exprtype'' entries in the
specification of the base library or the standard libraries.
Moreover, attempting to assign to an immutable variable (i.e., the
variables exported by a library; see
section~\ref{importsareimmutablesection}) is also
considered a syntax violation.

If a top-level or library form in a program is not syntactically
correct, then the implementation must raise an exception with
condition type {\cf\&syntax}, and execution of that top-level program
or library must not be allowed to begin.

\section{Safety}
\label{safeunsafemodesection}

The standard libraries whose exports are described by this document
are said to be \defining{safe libraries}.  Libraries and top-level
programs that import only from safe libraries are also said to be safe.

As defined by this document, the Scheme programming language
is safe in the following sense:
The execution of a safe top-level program
cannot go so badly wrong as to crash or to continue to
execute while behaving in ways that are
inconsistent with the semantics described in this document,
unless an exception is raised.

Violations of an implementation restriction must raise an
exception with condition type {\cf\&implementation-\linebreak[0]restriction},
as must all
violations and errors that would otherwise threaten system
integrity in ways that might result in execution that is
inconsistent with the semantics described in this document.

The above safety properties are guaranteed only for top-level programs
and libraries that are said to be safe.  In particular,
implementations may provide access to unsafe libraries in ways that
cannot guarantee safety.

\section{Boolean values}
\label{booleanvaluessection}

Although there is a separate boolean type, any Scheme value can be
used as a boolean value for the purpose of a conditional test.  In a
conditional test, all values count as true in such a test except for
\schfalse{}.  This report uses the word ``true'' to refer to any
Scheme value except \schfalse{}, and the word ``false'' to refer to
\schfalse{}. \mainindex{true} \mainindex{false}

\section{Multiple return values}
\label{multiplereturnvaluessection}

A Scheme expression can evaluate to an arbitrary finite number of
values.  These values are passed to the expression's continuation.

Not all continuations accept any number of values. For example, a continuation that
accepts the argument to a procedure call is guaranteed to accept
exactly one value.  The effect of passing some other number of values
to such a continuation is unspecified.  The {\cf call-with-values}
procedure
described in section~\ref{controlsection} makes it possible to create
continuations that accept specified numbers of return values.
If the number of
return values passed to a continuation created by a call to
{\cf call-with-values} is not accepted by its consumer
that was passed in that call, then an exception is raised.
A more complete description of the number of values accepted by
different continuations and the consequences of passing an unexpected
number of values is given in the description of the {\cf values}
procedure in section~\ref{values}.

A number of forms in the base library have sequences of expressions
as subforms that are evaluated sequentially, with the return values of
all but the last expression being discarded.  The continuations
discarding these values accept any number of values.

\section{Unspecified behavior}

\vest If an expression is said to ``return unspecified values'',
then the expression must evaluate without raising an exception, but
the values returned depend on the implementation; this report
explicitly does not say how many or what values should be returned.
Programmers should not rely on a specific number of return values or
the specific values themselves.
\mainindex{unspecified behavior}\mainindex{unspecified values}


\section{Storage model}
\label{storagemodel}

Variables and objects such as pairs, vectors, bytevectors, strings, hashtables, records implicitly
refer to locations\mainindex{location} or sequences of locations.  A string, for
example, contains as many locations as there are characters in the string. 
(These locations need not correspond to a full machine word.) A new value may be
stored into one of these locations using the {\tt string-set!} procedure, but
the string contains the same locations as before.

An object fetched from a location, by a variable reference or by
a procedure such as {\cf car}, {\cf vector-ref}, or {\cf string-ref}, is
equivalent in the sense of \ide{eqv?} % and \ide{eq?} ??
(section~\ref{equivalencesection})
to the object last stored in the location before the fetch.

Every location is marked to show whether it is in use.
No variable or object ever refers to a location that is not in use.
Whenever this report speaks of storage being allocated for a variable
or object, what is meant is that an appropriate number of locations are
chosen from the set of locations that are not in use, and the chosen
locations are marked to indicate that they are now in use before the variable
or object is made to refer to them.

It is desirable for constants\index{constant} (i.e. the values of
literal expressions) to reside in read-only-memory.  To express this,
it is convenient to imagine that every object that refers to locations
is associated with a flag telling whether that object is
mutable\index{mutable} or immutable\index{immutable}.  Literal
constants, the strings returned by \ide{symbol->string}, records with
no mutable fields, and other values explicitly designated as immutable
are immutable objects, while all objects created by the other
procedures listed in this report are mutable.  An attempt to store a
new value into a location referred to by an immutable object
should raise an exception with condition type {\cf\&assertion}.


\section{Proper tail recursion}
\label{proper tail recursion}

Implementations of Scheme must be
{\em properly tail-recursive}\mainindex{proper tail recursion}.
Procedure calls that occur in certain syntactic
contexts called \textit{tail contexts}\index{tail context} 
are \textit{tail calls}\mainindex{tail call}.  A Scheme implementation is
properly tail-recursive if it supports an unbounded number of active
tail calls.  A call is {\em active} if the called procedure may still
return.  Note that this includes regular returns as well as returns
through continuations captured earlier by
{\cf call-with-current-continuation} that are later invoked.
In the absence of captured continuations, calls could
return at most once and the active calls would be those that had not
yet returned.
A formal definition of proper tail recursion can be found
in Clinger's paper~\cite{propertailrecursion}.  The rules for identifying tail calls
in constructs from the \rsixlibrary{base} library are described in
section~\ref{basetailcontextsection}.

\section{Dynamic extent and dynamic environment}
\label{dynamicenvironmentsection}

For a procedure call, the time between when it is initiated and when
it returns is called its \defining{dynamic extent}.  In Scheme, {\cf
  call-with-current-continuation}
(section~\ref{call-with-current-continuation}) allows reentering a
dynamic extent after its procedure call has returned.  Thus, the
dynamic extent of a call may not be a single, connected time period.

Some operations described in the report acquire information in
addition to their explicit arguments from the \defining{dynamic
  environment}.  For example, {\cf call-with-current-continuation}
accesses an implicit context established
by {\cf dynamic-wind} (section~\ref{dynamic-wind}), and the {\cf
  raise} procedure (library
section~\extref{lib:exceptionssection}{Exceptions}) accesses the
current exception handler.  The operations that modify the dynamic
environment do so dynamically, for the dynamic extent of a call to a
procedure like {\cf dynamic-wind} or {\cf with-exception-handler}.
When such a call returns, the previous dynamic environment is
restored.  The dynamic environment can be thought of as part of the
dynamic extent of a call.  Consequently, it is captured by {\cf
  call-with-current-continuation}, and restored by invoking the escape
procedure it creates.

%%% Local Variables: 
%%% mode: latex
%%% TeX-master: "r6rs"
%%% End: 

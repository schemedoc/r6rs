\section{Condition types}
\label{iocondsection}

The procedures described in this chapter, when they detect an
exceptional situation that arises from an ``I/O errors'', raise an
exception with condition type {\cf\&i/o}.  Except where explicitly
specified, there is no guarantee that the raised condition object
contains all the information that would be applicable.  It is
recommended, however, that an implementation collect all information
that is available about an exceptional situation at the place where it
is detected and place it in the condition object.

The condition types and corresponding predicates and accessors are
exported by both the \rsixlibrary{i/o ports} and \rsixlibrary{i/o
  simple} libraries.  They are also exported by the \rsixlibrary{files}
library described in chapter~\ref{filesystemchapter}.

\begin{entry}{%
\ctproto{i/o}
\proto{make-i/o-error}{}{procedure}
\proto{i/o-error?}{ obj}{procedure}}

This condition type could be defined by
%
\begin{scheme}
(define-condition-type \&i/o \&error
  make-i/o-error i/o-error?)
\end{scheme}        

This is a supertype for a set of more specific I/O errors.
\end{entry}   

\begin{entry}{%
\ctproto{i/o-read}
\proto{make-i/o-read-error}{}{procedure}
\proto{i/o-read-error?}{ obj}{procedure}}

This condition type could be defined by
\begin{scheme}
(define-condition-type \&i/o-read \&i/o
  make-i/o-read-error i/o-read-error?)
\end{scheme}

This condition type describes read errors that occurred during an I/O
operation.
\end{entry}   

\begin{entry}{%
\ctproto{i/o-write}
\proto{make-i/o-write-error}{}{procedure}
\proto{i/o-write-error?}{ obj}{procedure}}

This condition type could be defined by
%
\begin{scheme}
(define-condition-type \&i/o-write \&i/o
  make-i/o-write-error i/o-write-error?)
\end{scheme}
This condition type describes write errors that occurred during an I/O
    operation.
  \end{entry}   
  
\begin{entry}{%
\ctproto{i/o-invalid-position}
\proto{make-i/o-invalid-position-error}{}{procedure}
\proto{i/o-invalid-position-error?}{ obj}{procedure}}

This condition type could be defined by
%
\begin{scheme}
(define-condition-type \&i/o-invalid-position \&i/o
  make-i/o-invalid-position-error
  i/o-invalid-position-error?
  (position i/o-error-position))
\end{scheme}

This condition type describes attempts to set the file position to an
invalid position. The value of the position field is the file position that
the program intended to set. This condition describes a range error, but
not an assertion violation.
\end{entry}   

\begin{entry}{%
\ctproto{i/o-filename}
\proto{make-i/o-filename-error}{ filename}{procedure}
\proto{i/o-filename-error?}{ obj}{procedure}
\proto{i/o-error-filename}{ condition}{procedure}}

This condition type could be defined by
%
\begin{scheme}
(define-condition-type \&i/o-filename \&i/o
  make-i/o-filename-error i/o-filename-error?
  (filename i/o-error-filename))
\end{scheme}

This condition type describes an I/O error that occurred during an
operation on a named file. Condition objects belonging to this type
must specify a file name in the {\cf filename} field.
\end{entry}

\begin{entry}{%
\ctproto{i/o-file-protection}
\proto{make-i/o-file-protection-error}{ filename}{procedure}
\proto{i/o-file-protection-error?}{ obj}{procedure}}

This condition type could be defined by
%
\begin{scheme}
(define-condition-type \&i/o-file-protection
    \&i/o-filename
  make-i/o-file-protection-error
  i/o-file-protection-error?)
\end{scheme}

A condition of this type specifies that an operation tried to operate on a
named file with insufficient access rights.
\end{entry}   

\begin{entry}{%
\ctproto{i/o-file-is-read-only}
\proto{make-i/o-file-is-read-only-error}{ filename}{procedure}
\proto{i/o-file-is-read-only-error?}{ obj}{procedure}}

This condition type could be defined by
%
\begin{scheme}
(define-condition-type \&i/o-file-is-read-only
    \&i/o-file-protection
  make-i/o-file-is-read-only-error
  i/o-file-is-read-only-error?)
\end{scheme}

A condition of this type specifies that an operation tried to operate on a
named read-only file under the assumption that it is writeable.
\end{entry}   

\begin{entry}{%
\ctproto{i/o-file-already-exists}
\proto{make-i/o-file-already-exists-error}{ filename}{procedure}
\proto{i/o-file-already-exists-error?}{ obj}{procedure}}

This condition type could be defined by
%
\begin{scheme}
(define-condition-type \&i/o-file-already-exists
    \&i/o-filename
  make-i/o-file-already-exists-error
  i/o-file-already-exists-error?)
\end{scheme}
A condition of this type specifies that an operation tried to operate on an
existing named file under the assumption that it did not exist.
\end{entry}   

\begin{entry}{%
\ctproto{i/o-file-exists-not}
\proto{make-i/o-exists-not-error}{ filename}{procedure}
\proto{i/o-exists-not-error?}{ obj}{procedure}}

This condition type could be defined by
%
\begin{scheme}
(define-condition-type \&i/o-file-exists-not
    \&i/o-filename
  make-i/o-exists-not-error
  i/o-file-exists-not-error?)
\end{scheme}

A condition of this type specifies that an operation tried to operate on an
non-existent named file under the assumption that it existed.
\end{entry}   

\begin{entry}{%
\ctproto{i/o-port}
\proto{make-i/o-port-error}{ port}{procedure}
\proto{i/o-port-error?}{ obj}{procedure}
\proto{i/o-error-port}{ condition}{procedure}}

This condition type could be defined by
%
\begin{scheme}
(define-condition-type \&i/o-port \&i/o
  make-i/o-port-error i/o-port-error?
  (port i/o-error-port))
\end{scheme}

This condition type specifies the port with which an I/O
error is associated. Except for condition objects provided for
encoding and decoding errors, conditions raised by procedures may
include an {\cf\&i/o-port-error} condition, but are not required to do
so.
\end{entry}

%%% Local Variables: 
%%% mode: latex
%%% TeX-master: "r6rs-lib"
%%% End: 

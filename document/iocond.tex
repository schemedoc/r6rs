\section{Condition types}
\label{iocondsection}

In exceptional situations arising from ``I/O errors'', the procedures
described in this chapter raise an exception with condition type
{\cf\&i/o}.  Except where explicitly specified, there is no guarantee
that the raised condition object contains all the information that
would be applicable.  It is recommended, however, that an
implementation collect all information that is available about an
exceptional situation at the place where it is detected and place it
in the condition object.

The condition types and corresponding predicates and accessors are
exported by both the \library{r6rs i/o primitive} and \library{r6rs
  i/o simple} libraries.

\begin{entry}{%
\ctproto{i/o}
\proto{i/o-error?}{ obj}{procedure}}

This condition type could be defined by
%
\begin{scheme}
(define-condition-type \&i/o \&error
  i/o-error?)
\end{scheme}        

This is a supertype for a set of more specific I/O errors.
\end{entry}   

\begin{entry}{%
\ctproto{i/o-read}
\proto{i/o-read-error?}{ obj}{procedure}}

This condition type could be defined by
\begin{scheme}
(define-condition-type \&i/o-read \&i/o
  i/o-read-error?)
\end{scheme}

This condition type describes read errors that occurred during an I/O
operation.
\end{entry}   

\begin{entry}{%
\ctproto{i/o-write}
\proto{i/o-write-error?}{ obj}{procedure}}

This condition type could be defined by
%
\begin{scheme}
(define-condition-type \&i/o-write \&i/o
  i/o-write-error?)
\end{scheme}
This condition type describes write errors that occurred during an I/O
    operation.
  \end{entry}   
  
\begin{entry}{%
\ctproto{i/o-invalid-position}
\proto{i/o-invalid-position-error?}{ obj}{procedure}}

This condition type could be defined by
%
\begin{scheme}
(define-condition-type \&i/o-invalid-position \&i/o
  i/o-invalid-position-error?
  (position i/o-error-position))
\end{scheme}

This condition type describes attempts to set the file position to an
invalid position. The value of the position field is the file position that
the program intended to set. This condition describes a range error, but
not a contract violation.
\end{entry}   

\begin{entry}{%
\ctproto{i/o-filename}
\proto{i/o-filename-error?}{ obj}{procedure}
\proto{i/o-error-filename}{ condition}{procedure}}

This condition type could be defined by
%
\begin{scheme}
(define-condition-type \&i/o-filename \&i/o
  i/o-filename-error?
  (filename i/o-error-filename))
\end{scheme}

This condition type describes an I/O error that occurred during an
operation on a named file. Condition objects belonging to this type
must specify a file name in the {\cf filename} field.
\end{entry}

\begin{entry}{%
\ctproto{i/o-file-protection}
\proto{i/o-file-protection-error?}{ obj}{procedure}}

This condition type could be defined by
%
\begin{scheme}
(define-condition-type \&i/o-file-protection
    \&i/o-filename
  i/o-file-protection-error?)
\end{scheme}

A condition of this type specifies that an operation tried to operate on a
named file with insufficient access rights.
\end{entry}   

\begin{entry}{%
\ctproto{i/o-file-is-read-only}
\proto{i/o-file-is-read-only-error?}{ obj}{procedure}}

This condition type could be defined by
%
\begin{scheme}
(define-condition-type \&i/o-file-is-read-only
    \&i/o-file-protection
  i/o-file-is-read-only-error?)
\end{scheme}

A condition of this type specifies that an operation tried to operate on a
named read-only file under the assumption that it is writeable.
\end{entry}   

\begin{entry}{%
\ctproto{i/o-file-already-exists}
\proto{i/o-file-already-exists-error?}{ obj}{procedure}}

This condition type could be defined by
%
\begin{scheme}
(define-condition-type \&i/o-file-already-exists
    \&i/o-filename
  i/o-file-already-exists-error?)
\end{scheme}
A condition of this type specifies that an operation tried to operate on an
existing named file under the assumption that it did not exist.
\end{entry}   

\begin{entry}{%
\ctproto{i/o-file-exists-not}
\proto{i/o-exists-not-error?}{ obj}{procedure}}

This condition type could be defined by
%
\begin{scheme}
(define-condition-type \&i/o-file-exists-not
    \&i/o-filename
  i/o-file-exists-not-error?)
\end{scheme}

A condition of this type specifies that an operation tried to operate on an
non-existent named file under the assumption that it existed.
\end{entry}   

%%% Local Variables: 
%%% mode: latex
%%% TeX-master: "r6rs-lib"
%%% End: 

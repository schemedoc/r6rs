\chapter{Expansion Process}
\label{expansionchapter}

Macro uses (see section~\ref{macrosection} are expanded into \textit{core
forms}\mainindex{core form} at the start of evaluation (before compilation or interpretation)
by a syntax \emph{expander}.
(The set of core forms is implementation-dependent, as is the
representation of these forms in the expander's output.)
The expander is invoked once for each top-level form in a program.
If the expander encounters a syntactic abstraction, it invokes
the associated transformer to expand the syntactic abstraction, then
repeats the expansion process for the form returned by the transformer.
If the expander encounters a core form, it recursively
processes the subforms, if any, and reconstructs the form from the
expanded subforms.
Information about identifier bindings is maintained during expansion
to enforce lexical scoping for variables and keywords.

To handle internal definitions, the expander processes the initial
forms in \hyper{body} (see section~\ref{bodiessection}),
\hyper{library body} (see section~\ref{librarybodysection}), or
\hyper{script body} (see chapter~\ref{scriptchapter}) from left to
right.  How the expander processes each form encountered as it does so
depends upon the kind of form.

\begin{description}
\item[macro use]
The expander invokes the associated transformer to transform the macro
use, then recursively performs whichever of these actions are appropriate
for the resulting form.

\item[{\cf define-syntax} form]
The expander expands and evaluates the right-hand-side expression and binds the
keyword to the resulting transformer.

\item[{\cf define} form]
The expander records the fact that the defined identifier is a variable but defers
expansion of the right-hand-side expression until after all of the
definitions have been processed.

\item[{\cf begin} form]
The expander splices the subforms into the list of body forms it is
processing.  (See section~\ref{begin}.)

\item[{\cf let-syntax} or {\cf letrec-syntax} form]
The expander splices the inner body forms into the list of (outer) body forms it is
processing, arranging for the keywords bound by the {\cf let-syntax}
and {\cf letrec-syntax} to be visible only in the inner body forms.

\item[expression, i.e., nondefinition]
The expander completes the expansion of the deferred right-hand-side forms
and the current and remaining expressions in the body, then
constructs a residual {\cf letrec*} form from the defined variables,
expanded right-hand-side expressions, and expanded body expressions.
\end{description}

It is a syntax violation
if the keyword that identifies one of the body forms
as a definition (derived or core) is redefined by the same definition or a
later definition in the same body.
To detect this error, the expander records the identifying keyword for each
macro use, {\cf define-syntax} form, {\cf define}
form, {\cf begin} form, {\cf let-syntax} form, and {\cf letrec-syntax}
form it encounters while processing the definitions and checks each
newly defined identifier ({\cf define} or {\cf define-syntax}
left-hand side) against the recorded keywords, as with
{\cf bound-identifier=?} (section~\ref{identifierpredicatessection}).
For example, the following forms are syntax violation.

\begin{scheme}
(let ()
  (define define 17)
  define)

(let-syntax ([def0 (syntax-rules ()
                     [(\_ x) (define x 0)])])
  (let ()
    (def0 z)
    (define def0 '(def 0))
    (list z def0)))
\end{scheme}

Expansion of each variable definition right-hand side is deferred until
after all of the definitions have been seen so that each keyword and
variable reference within the right-hand side resolves to the local
binding, if any.

Note that this algorithm does not directly reprocess any form.
It requires a single left-to-right pass over the definitions followed by a
single pass (in any order) over the body expressions and deferred
right-hand sides.

For example, in

\begin{scheme}
(lambda (x)
  (define-syntax defun
    (syntax-rules ()
      [(\_ (x . a) e) (define x (lambda a e))]))
  (defun (even? n) (or (= n 0) (odd? (- n 1))))
  (define-syntax odd?
    (syntax-rules () [(\_ n) (not (even? n))]))
  (odd? (if (odd? x) (* x x) x)))
\end{scheme}

The definition of {\cf defun} is encountered first, and the keyword
{\cf defun} is associated with the transformer resulting from
the expansion and evaluation of the corresponding right-hand side.
A use of {\cf defun} is encountered next and expands into a
{\cf define} form.
Expansion of the right-hand side of this define form is deferred.
The definition of {\cf odd?} is next and results in the association
of the keyword {\cf odd?} with the transformer resulting from
expanding and evaluating the corresponding right-hand side.
A use of {\cf odd?} appears next and is expanded; the resulting
call to {\cf not} is recognized as an expression
because {\cf not} is bound as a variable.
At this point, the expander completes the expansion of the current
expression (the {\cf not} call) and the deferred right-hand side of the
{\cf even?} definition;
the uses of {\cf odd?} appearing in these expressions are expanded
using the transformer associated with the keyword {\cf odd?}.
The final output is the equivalent of

\begin{scheme}
(lambda (x)
  (letrec* ([even?
              (lambda (n)
                (or (= n 0)
                    (not (even? (- n 1)))))])
    (not (even? (if (not (even? x)) (* x x) x)))))
\end{scheme}

although the structure of the output is implementation dependent.

%%% Local Variables: 
%%% mode: latex
%%% TeX-master: "r6rs"
%%% End: 

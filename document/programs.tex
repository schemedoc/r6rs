\chapter{Top-level programs}
\label{programchapter}

A \defining{top-level program} specifies an entry point for defining and running
a Scheme program.  A top-level program specifies a set of libraries to import and
code to run.  Through the imported libraries, whether directly or through the
transitive closure of importing, a top-level program defines a complete Scheme
program.

Top-level programs follow the convention of many common platforms of accepting 
a list of string \defining{command-line arguments} that may be used to
pass data to the script.

\section{Top-level program syntax}
\label{programsyntaxsection}

A top-level program is a delimited piece of text, typically a file, that follows
the following syntax:
%
\begin{grammar}
\meta{toplevel program} \: \meta{import form} \meta{toplevel body}
\meta{import form} \: (import \arbno{\meta{import spec}})
\meta{toplevel body} \: \arbno{\meta{toplevel body form}}
\meta{toplevel body form} \: \meta{definition} \| \meta{expression}
\end{grammar}
%
The rules for \meta{toplevel program} specify syntax at the form level.

The \meta{import form} is identical to the import clause in
libraries (see section~\ref{librarysyntaxsection}), 
and specifies a set of libraries to import.  A \meta{toplevel 
 body} is like a \meta{library body} (see
section~\ref{librarybodysection}), except that 
definitions and expressions may occur in any order.  Thus, the syntax
specified by \meta{toplevel body form} refers to the result of macro
expansion.

\begin{rationale}
By allowing the interleaving of definitions and expressions, top-level 
programs support exploratory and interactive development, without 
imposing unnecessary organizational overhead on code which may not be 
intended for reuse.
\end{rationale}

When base-library {\cf begin} forms occur anywhere within a top-level body,
they are spliced into the body; see section~\ref{begin}.
Some or all of the top-level body, including portions wrapped in {\cf begin}
forms, may be specified by a syntactic abstraction
(see section~\ref{macrosection}).

\section{Top-level program semantics}

A top-level program is executed by treating the program similarly to a library, and
invoking it.  The semantics of a top-level body may be roughly explained by
a simple translation into a library body: 
Each \hyper{expression} that appears before a
definition in
the top-level body is converted into a dummy definition 
{\cf (define \hyper{variable} (begin \hyper{expression} (unspecified)))},
where \hyper{variable} is a fresh identifier.
(It is generally impossible to determine which forms are 
definitions and expressions without concurrently expanding the body, so
the actual translation is somewhat more complicated; see
chapter~\ref{expansionchapter}.)

On platforms that support it, a top-level program may access its command-line 
arguments by calling the {\cf command-line} procedure (see library 
section~\extref{lib:command-line}{Command-line access and exit values}).

%%% Local Variables: 
%%% mode: latex
%%% TeX-master: "r6rs"
%%% End: 

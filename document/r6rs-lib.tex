\documentclass[twoside,twocolumn]{algol60}
%\documentclass[twoside]{algol60}

\usepackage{xr}

\pagestyle{headings} 
\showboxdepth=0
\makeindex
% Macros for R^nRS.

\usepackage{makeidx}
\usepackage{url}

% tex2page.sty mucks with in some manner
\let\centerlinesaved=\centerline
\usepackage{tex2page}
\let\centerline=\centerlinesaved

% \let\htmlonly=\iffalse
% \let\endhtmlonly=\fi
% \let\texonly=\iftrue
% \let\endtexonly=\fi

\makeatletter

\texonly
\newcommand{\topnewpage}{\@topnewpage}
\endtexonly

\htmlonly
\newcommand{\topnewpage}[0][]{#1}
\endhtmlonly

\newcommand{\authorsc}[1]{{\scriptsize\scshape #1}}

% Chapters, sections, etc.

\newcommand{\extrapart}[1]{
 % \chapter{#1}
  \chapter*{#1}
  \markboth{#1}{#1}
  \vskip 1ex
  \addcontentsline{toc}{chapter}{#1}}

\newcommand{\clearchaptergroupstar}[1]{
  \texonly
  \clearpage
  \addcontentsline{toc}{chaptergroup}{#1}
  \topnewpage[
    \centerline{\large\bf\uppercase{#1}}
    \bigskip]
    \endtexonly
  }

\newcommand{\clearchapterstar}[1]{
  \clearpage
  \topnewpage[
    \centerline{\large\bf\uppercase{#1}}
    \bigskip]}

\newcommand{\clearextrapart}[1]{
  \clearchapterstar{#1}
  \markboth{#1}{#1}
  \addcontentsline{toc}{chapter}{#1}}

\newcommand{\vest}{}
\newcommand{\dotsfoo}{$\ldots\,$}

\newcommand{\sharpfoo}[1]{{\tt\##1}}
\newcommand{\schfalse}{\sharpfoo{f}}
\newcommand{\schtrue}{\sharpfoo{t}}

\newcommand{\ampfoo}[1]{{\tt\&#1}}

\newcommand{\libfoo}[1]{{\tt(#1)}}

\newcommand{\singlequote}{{\tt'}}  %\char19
\newcommand{\doublequote}{{\tt"}}
\newcommand{\backquote}{{\tt\char18}}
\newcommand{\backwhack}{{\tt\char`\\}}
\newcommand{\comma}{{\tt\char`\@}}
\newcommand{\atsign}{{\tt\char`\@}}
\newcommand{\bang}{{\tt\char`\!}}
\newcommand{\sharpsign}{{\tt\#}}
\newcommand{\verticalbar}{{\tt|}}
\newcommand{\openbracket}{{\tt\char`\[}}
\newcommand{\closedbracket}{{\tt\char`\]}}
\newcommand{\ampersand}{{\tt\char`\&}}

\newcommand{\coerce}{\discretionary{->}{}{->}}

% Knuth's \in sucks big boulders
\def\elem{\hbox{\raise.13ex\hbox{$\scriptstyle\in$}}}

\newcommand{\meta}[1]{{\noindent\hbox{\rm$\langle$#1$\rangle$}}}
\let\hyper=\meta
\newcommand{\hyperi}[1]{\hyper{#1$_1$}}
\newcommand{\hyperii}[1]{\hyper{#1$_2$}}
\newcommand{\hyperiii}[1]{\hyper{#1$_3$}}
\newcommand{\hyperj}[1]{\hyper{#1$_i$}}
\newcommand{\hypern}[1]{\hyper{#1$_n$}}
\texonly
\newcommand{\var}[1]{\noindent\hbox{\textnormal{\textit{#1}}}}
\endtexonly
\htmlonly
\newcommand{\var}[1]{\textnormal{\textit{#1}}}
\endhtmlonly
\newcommand{\vari}[1]{\var{#1$_1$}}
\newcommand{\varii}[1]{\var{#1$_2$}}
\newcommand{\variii}[1]{\var{#1$_3$}}
\newcommand{\variv}[1]{\var{#1$_4$}}
\newcommand{\varj}[1]{\var{#1$_j$}}
\newcommand{\varn}[1]{\var{#1$_n$}}

\newcommand{\vr}[1]{{\noindent\hbox{$#1$\/}}}  % Careful, is \/ always the right thing?
\newcommand{\vri}[1]{\vr{#1_1}}
\newcommand{\vrii}[1]{\vr{#1_2}}
\newcommand{\vriii}[1]{\vr{#1_3}}
\newcommand{\vriv}[1]{\vr{#1_4}}
\newcommand{\vrv}[1]{\vr{#1_5}}
\newcommand{\vrj}[1]{\vr{#1_j}}
\newcommand{\vrn}[1]{\vr{#1_n}}

%%R4%% The excessive use of the code font in the numbers section was
% confusing, somewhat obnoxious, and inconsistent with the rest
% of the report and with parts of the section itself.  I added
% a \tupe no-op, and changed most old uses of \type to \tupe,
% to make it easier to change the fonts back if people object
% to the change.

\newcommand{\type}[1]{{\it#1}}
\newcommand{\tupe}[1]{{#1}}

\newcommand{\defining}[1]{\mainindex{#1}{\em #1}}
\newcommand{\ide}[1]{{\schindex{#1}\frenchspacing\tt{#1}}}

\newcommand{\lambdaexp}{{\cf lambda} expression}

\newcommand{\callcc}{{\tt call-with-current-continuation}}

\newcommand{\mainschindex}[1]{\label{#1}\index{#1@\texttt{#1}}}
\newcommand{\mainindex}[1]{\index{#1}}
\newcommand{\schindex}[1]{\index{#1@\texttt{#1}}}
\newcommand{\sharpindex}[1]{\index{#1@\texttt{\#{}#1}}}
\newcommand{\ampindex}[1]{\index{#1@\texttt{\&{}#1}}}
\newcommand{\libindex}[1]{\index{#1@\texttt{(#1)}}}

\newcommand{\extref}[2]{\texonly\ref{#1}\endtexonly\htmlonly{on ``#2''}\endhtmlonly}

\renewenvironment{theindex}
{\clearpage
\topnewpage[
    \begin{center}
      \large\bf\MakeUppercase{\indexheading}
    \end{center}
    \vskip 1ex \bigskip]
    \markboth{Index}{Index}
    \addcontentsline{toc}{chapter}{\indexheading}
    \parindent\z@
    \texonly\parskip\z@ plus .1pt\endtexonly\relax\let\item\@idxitem
    \indexintro\par\bigskip}
               {\texonly\clearpage\endtexonly}


\newcommand{\domain}[1]{#1}
\newcommand{\nodomain}[1]{}
%\newcommand{\todo}[1]{{\rm$[\![$!!~#1$]\!]$}}
\newcommand{\todo}[1]{}

% \frobq will make quote and backquote look nicer.
\def\frobqcats{%\catcode`\'=13
\catcode`\`=13{}}
{\frobqcats
\gdef\frobqdefs{%\def'{\singlequote}
\def`{\backquote}}}
\def\frobq{\frobqcats\frobqdefs}

% \cf = code font
% Unfortunately, \cf \cf won't work at all, so don't even attempt to
% next constructions which use them...
\newcommand{\cf}{\frenchspacing\frobq\tt}

\texonly
% Same as \obeycr, but doesn't do a \@gobblecr.
{\catcode`\^^M=13 \gdef\myobeycr{\catcode`\^^M=13 \def^^M{\\}}%
\gdef\restorecr{\catcode`\^^M=5 }}
\endtexonly

{\obeyspaces\gdef {\hbox{\hskip0.5em}}}

\gdef\gobblecr{\@gobblecr}

\def\setupcode{\@makeother\^}

% Scheme example environment
% At 11 points, one column, these are about 56 characters wide.
% That's 32 characters to the left of the => and about 20 to the right.

\newcommand{\exception}[1]{{\cf#1} \textnormal{\textit{exception}}}
\newenvironment{schemenoindent}{
  % Commands for scheme examples
  \newcommand{\ev}{\>\>\evalsto}
  \newcommand{\lev}{\\\>\evalsto}
  \newcommand{\unspecified}{{\em{}unspecified}}
  \newcommand{\theunspecified}{{\em{}the unspecified value}}
  \setupcode
  \small \cf \obeyspaces \myobeycr
  \begin{tabbing}%
\qquad\=\hspace*{5em}\=\hspace*{9em}\=\evalsto~\=\kill%   was 16em
\gobblecr}{\unskip\end{tabbing}}

%\newenvironment{scheme}{\begin{schemenoindent}\+\kill}{\end{schemenoindent}}
\newenvironment{scheme}{
  % Commands for scheme examples
  \newcommand{\ev}{\>\>\evalsto}
  \newcommand{\lev}{\\\>\evalsto}
  \renewcommand{\em}{\rmfamily\itshape}
  \newcommand{\unspecified}{{\em{}unspecified}}
  \newcommand{\theunspecified}{{\em{}the unspecified value}}
  \setupcode
  \small \cf \obeyspaces \myobeycr
  \begin{tabbing}%
\qquad\=\hspace*{5em}\=\hspace*{9em}\=\evalsto~\=\+\kill%   was 16em
\gobblecr}{\unskip\end{tabbing}}

\newcommand{\evalsto}{$\Longrightarrow$}

% Rationale

\newenvironment{rationale}{%
\bgroup\small\noindent{\em Rationale:}\space}{%
\egroup}

% Notes

\newenvironment{note}{%
\bgroup\small\noindent{\em Note:}\space}{%
\egroup}

% Names of library modules

\newcommand{\library}[1]{{\tt (#1)}}
\newcommand{\deflibrary}[1]{\library{#1}\libindex{#1}}

% Manual entries

\newenvironment{entry}[1]{
  \vspace{3.1ex plus .5ex minus .3ex}\noindent#1%
\unpenalty\nopagebreak}{\vspace{0ex plus 1ex minus 1ex}}

\newcommand{\exprtype}{syntax}

\newcommand{\unspecifiedreturn}{the unspecified value}

% Primitive prototype
\newcommand{\pproto}[2]{\unskip%
\hbox{\cf\spaceskip=0.5em#1}\hfill\penalty 0%
\hbox{ }\nobreak\hfill\hbox{\rm #2}\break}

% Parenthesized prototype
\newcommand{\proto}[3]{\pproto{(\mainschindex{#1}\hbox{#1}{\it#2\/})}{#3}}

% Variable prototype
\newcommand{\vproto}[2]{\mainschindex{#1}\pproto{#1}{#2}}

% Condition-type prototype
 \newcommand{\ctproto}[1]{\ampindex{#1}\pproto{\ampfoo{#1}}{condition type}}

% Extending an existing definition (\proto without the index entry)
\newcommand{\rproto}[3]{\pproto{(\hbox{#1}{\it#2\/})}{#3}}

% Extending an existing definition, with index entry
\newcommand{\irproto}[3]{\schindex{#1}\rproto{#1}{#2}{#3}}

% Variable prototype
\newcommand{\rvproto}[2]{\pproto{#1}{#2}}

% Grammar environment

\newenvironment{grammar}{
  \def\:{\goesto{}}
  \def\|{$\vert$}
  \cf \myobeycr
  \begin{tabbing}
    %\qquad\quad \= 
    \qquad \= $\vert$ \= \kill
  }{\unskip\end{tabbing}}

%\newcommand{\unsection}{\unskip}
\newcommand{\unsection}{{\vskip -2ex}}

% Commands for grammars
\newcommand{\arbno}[1]{#1\hbox{\rm*}}  
\newcommand{\atleastone}[1]{#1\hbox{$^+$}}

\newcommand{\goesto}{$\longrightarrow$}

\newcommand{\syntax}{{\em Syntax: }}
\newcommand{\semantics}{{\em Semantics: }}
\newcommand{\implresp}{{\em Implementation responsibilities: }}

\newcommand{\rrs}[1]{\textit{Revised$^#1$ Report on the Algorithmic Language Scheme}}

\newcommand{\libindexentry}[1]{#1 (library)}

\makeatother


\texonly\externaldocument[report:]{r6rs}\endtexonly

\def\headertitle{Revised$^{5.93}$ Scheme Libraries}
\def\integerversion{6}

\begin{document}

\thispagestyle{empty}

\topnewpage[{
\begin{center}   {\huge\bf
        Revised{\Huge$^{\mathbf{\htmlonly\tiny\endhtmlonly{}5.93}}$} Report on the Algorithmic Language \\
                              \vskip 3pt
                              Scheme\\
                                \vskip 1.5ex
                              --- Standard Libraries ---}

\vskip 1ex
$$
\begin{tabular}{l@{\extracolsep{.5in}}lll}
\multicolumn{4}{c}{M\authorsc{ICHAEL} S\authorsc{PERBER}}
\\
\multicolumn{4}{c}{R.\ K\authorsc{ENT} D\authorsc{YBVIG},
  M\authorsc{ATTHEW} F\authorsc{LATT},
  A\authorsc{NTON} \authorsc{VAN} S\authorsc{TRAATEN}}
\\
\multicolumn{4}{c}{(\textit{Editors})} \\
\multicolumn{4}{c}{
  R\authorsc{ICHARD} K\authorsc{ELSEY}, W\authorsc{ILLIAM} C\authorsc{LINGER},
  J\authorsc{ONATHAN} R\authorsc{EES}} \\
\multicolumn{4}{c}{(\textit{Editors, Revised$^5$ Report on the
    Algorithmic Language Scheme})} \\
\multicolumn{4}{c}{\bf 15 March 2007}
\end{tabular}
$$



\end{center}

\chapter*{Summary}
\medskip

The report gives a defining description of the standard libraries of the
programming language Scheme.

Chapter~\ref{unicodechapter} describes the library implementing Unicode
semantics for characters and strings, chapter~\ref{bytevectorschapter}
describes the library for handling binary data,
chapter~\ref{listutilities} describes the library containing list utilities
procedures, chapter~\ref{sortingchapter} describes the sorting
library, chapter~\ref{recordschapter} describes the record system,
\ref{exceptionsconditionschapter} describes the libraries for
exceptions and conditions, chapter~\ref{iochapter} describes the I/O
libraries, chapter~\ref{filesystemchapter} describes the file-system library,
chapter~\ref{numberchapter} describes specialized libraries
for dealing with numbers and arithmetic,
chapter~\ref{syntaxcasechapter} describes the {\cf syntax-case}
facility for writing arbitrary macros, chapter~\ref{hashtablechapter}
describes the library for hash tables,
chapter~\ref{enumerationschapter} describes the enumerations library, and
chapter~\ref{misclibchapter} describes various miscellaneous libraries.

Chapter~\ref{complibchapter} describes the composite library
containing most of the forms described in this report.
Chapter~\ref{evalchapter} describes the {\cf eval} facility for
evaluating Scheme expressions represented as data.
Chapter~\ref{pairmutationchapter} describes the operations for
mutating pairs.  Chapter~\ref{r5rscompatchapter} describes a library
with some procedures from the previous
version of this report for backwards compatibility.

The report concludes with a list of references and an
alphabetic index.

This report frequently refers back to the \textit{Revised$^6$ Report
  on the Algorithmic Language Scheme}~\cite{R6RS}; references to the
report are identified by designations such as ``report section'' or
``report chapter''.

\bigskip

\begin{center}
{\large \bf
*** DRAFT*** \\
}\end{center}

This is a preliminary draft.  It is intended to reflect the decisions
taken by the editors' committee, but contains many mistakes,
ambiguities and inconsistencies.

}]

\texonly\clearpage\endtexonly

\chapter*{Contents}
\addvspace{3.5pt}                  % don't shrink this gap
\renewcommand{\tocshrink}{-4.0pt}  % value determined experimentally
{%\footnotesize
\tableofcontents
}

\vfill

\texonly\clearpage\endtexonly

\chapter{Unicode}

The procedures exported by the \deflibrary{r6rs unicode}
library provide access to some aspects
of the Unicode semantics for characters and strings:
category information, case-independent comparisons,
case mappings, and normalization.

Some of the procedures that operate on characters or strings ignore the
difference between upper case and lower case.  The procedures that
ignore case have \hbox{``{\tt -ci}''} (for ``case
insensitive'') embedded in their names.

\section{Characters}

\begin{entry}{%
\proto{char-upcase}{ char}{procedure}
\proto{char-downcase}{ char}{procedure}
\proto{char-titlecase}{ char}{procedure}
\proto{char-foldcase}{ char}{procedure}}

These procedures take a character argument and return a character
result. If the argument is an upper case or title case character, and if
there is a single character that is its lower case form, then
{\cf char-downcase} returns that character. If the argument is a lower case
or title case character, and there is a single character that is
its upper case form, then {\cf char-upcase} returns that character.
If the argument is a lower case
or upper case character, and there is a single character that is
its title case form, then {\cf char-titlecase} returns that character.
Finally, if the character has a case-folded character,
then {\cf char-foldcase} returns that character.
Otherwise the character returned is the same
as the argument.
For Turkic characters \.I ({\tt \#\backwhack{}x130})
and \i{} ({\tt \#\backwhack{}x131}),
{\cf char-foldcase} behaves as the identity function; otherwise 
{\cf char-foldcase} is the
same as {\cf char-downcase} composed with {\cf char-upcase}.

\begin{scheme}
(char-upcase \sharpsign\backwhack{}i) \ev \sharpsign\backwhack{}I
(char-downcase \sharpsign\backwhack{}i) \ev \sharpsign\backwhack{}i
(char-titlecase \sharpsign\backwhack{}i) \ev \sharpsign\backwhack{}I
(char-foldcase \sharpsign\backwhack{}i) \ev \sharpsign\backwhack{}i

(char-upcase \sharpsign\backwhack{}\ss) \ev \sharpsign\backwhack{}\ss
(char-downcase \sharpsign\backwhack{}\ss) \ev \sharpsign\backwhack{}\ss
(char-titlecase \sharpsign\backwhack{}\ss) \ev \sharpsign\backwhack{}\ss
(char-foldcase \sharpsign\backwhack{}\ss) \ev \sharpsign\backwhack{}\ss

(char-upcase \sharpsign\backwhack{}$\Sigma$) \ev \sharpsign\backwhack{}$\Sigma$
(char-downcase \sharpsign\backwhack{}$\Sigma$) \ev \sharpsign\backwhack{}$\sigma$
(char-titlecase \sharpsign\backwhack{}$\Sigma$) \ev \sharpsign\backwhack{}$\Sigma$
(char-foldcase \sharpsign\backwhack{}$\Sigma$) \ev \sharpsign\backwhack{}$\sigma$

(char-upcase \sharpsign\backwhack{}$\varsigma$) \ev \sharpsign\backwhack{}$\Sigma$
(char-downcase \sharpsign\backwhack{}$\varsigma$) \ev \sharpsign\backwhack{}$\varsigma$
(char-titlecase \sharpsign\backwhack{}$\varsigma$) \ev \sharpsign\backwhack{}$\Sigma$
(char-foldcase \sharpsign\backwhack{}$\varsigma$) \ev \sharpsign\backwhack{}$\sigma$
\end{scheme}

\begin{note}
  These procedures are consistent with
  Unicode's locale-independent mappings from scalar values to
  scalar values for upcase, downcase, titlecase, and case-folding
  operations.  These mappings can be extracted from {\cf
    UnicodeData.txt} and {\cf CaseFolding.txt} from the Unicode
  Consortium, ignoring Turkic mappings in the latter.

  Note that these character-based procedures are an incomplete
  approximation to case conversion, even ignoring the user's locale.
  In general, case mappings require the context of a string, both in
  arguments and in result. See {\cf string-upcase} and {\cf
    string-downcase} for more general case-conversion procedures.
\end{note}
\end{entry}


\begin{entry}{%
\proto{char-ci=?}{ \vari{char} \varii{char} \variii{char} \dotsfoo}{procedure}
\proto{char-ci<?}{ \vari{char} \varii{char} \variii{char} \dotsfoo}{procedure}
\proto{char-ci>?}{ \vari{char} \varii{char} \variii{char} \dotsfoo}{procedure}
\proto{char-ci<=?}{ \vari{char} \varii{char} \variii{char} \dotsfoo}{procedure}
\proto{char-ci>=?}{ \vari{char} \varii{char} \variii{char} \dotsfoo}{procedure}}

These procedures are similar to {\cf char=?}\ et cetera, but operate
on the case-folded versions of the characters.

\begin{scheme}
(char-ci<? \sharpsign\backwhack{}z \sharpsign\backwhack{}Z) \ev \schfalse
(char-ci=? \sharpsign\backwhack{}z \sharpsign\backwhack{}Z) \ev \schtrue
(char-ci=? \sharpsign\backwhack{}$\varsigma$ \sharpsign\backwhack{}$\sigma$) \ev \schtrue
\end{scheme}
\end{entry}


\begin{entry}{%
\proto{char-alphabetic?}{ char}{procedure}
\proto{char-numeric?}{ char}{procedure}
\proto{char-whitespace?}{ char}{procedure}
\proto{char-upper-case?}{ letter}{procedure}
\proto{char-lower-case?}{ letter}{procedure}
\proto{char-title-case?}{ letter}{procedure}}

These procedures return \schtrue{} if their arguments are alphabetic,
numeric, whitespace, upper case, lower case, or title case characters,
respectively; otherwise they return \schfalse.

A character is alphabetic if it is a Unicode letter, i.e.\ if it is in
one of the categories Lu, Ll, Lt, Lm, and Lo.  A character is numeric if
it is in categeory Nd.  A characters is whitespace if it is in one of
the space, line, or paragraph separator categories (Zs, Zl or Zp), or
if is Unicode 9 (Horizontal tabulation), Unicode 10 (Line feed),
Unicode 11 (Vertical tabulation), Unicode 12 (Form feed), or Unicode
13 (Carriage return).  A character is upper case if it has the Unicode
``Uppercase'' property, lower case if it has the ``Lowercase''
property, and title case if it is in the Lt general category.

\begin{scheme}
(char-alphabetic? \sharpsign\backwhack{}a) \ev \schtrue{}
(char-numeric? \sharpsign\backwhack{}1) \ev \schtrue{}
(char-whitespace? \sharpsign\backwhack{}space) \ev \schtrue{}
(char-whitespace? \sharpsign\backwhack{}x00A0) \ev \schtrue{}
(char-upper-case? \sharpsign\backwhack{}$\Sigma$) \ev \schtrue{}
(char-lower-case? \sharpsign\backwhack{}$\sigma$) \ev \schtrue{}
(char-lower-case? \sharpsign\backwhack{}x00AA) \ev \schtrue{}
(char-title-case? \sharpsign\backwhack{}I) \ev \schfalse{}
(char-title-case? \sharpsign\backwhack{}x01C5) \ev \schtrue{}
\end{scheme}
\end{entry}

\begin{entry}{%
\proto{char-general-category}{ char}{procedure}}

Returns a symbol representing the
Unicode general category of \var{char}, one of {\cf Lu}, {\cf Ll}, {\cf Lt},
{\cf Lm}, {\cf Lo}, {\cf Mn}, {\cf Mc}, {\cf Me}, {\cf Nd}, {\cf Nl},
{\cf No}, {\cf Ps}, {\cf Pe}, {\cf Pi}, {\cf Pf}, {\cf Pd}, {\cf Pc},
{\cf Po}, {\cf Sc}, {\cf Sm}, {\cf Sk}, {\cf So}, {\cf Zs}, {\cf Zp},
{\cf Zl}, {\cf Cc}, {\cf Cf}, {\cf Cs}, {\cf Co}, or {\cf Cn}.

\begin{scheme}
(char-general-category \#\backwhack{}a) \ev Ll
(char-general-category \#\backwhack{}space) \lev Zs
(char-general-category \#\backwhack{}x10FFFF) \lev Cn  
\end{scheme}
\end{entry}

\section{Strings}

\begin{entry}{%
\proto{string-upcase}{ \var{string}}{procedure}
\proto{string-downcase}{ \var{string}}{procedure}
\proto{string-titlecase}{ \var{string}}{procedure}
\proto{string-foldcase}{ \var{string}}{procedure}}

These procedures take a string argument and return a string
result.  They are defined in terms of Unicode's locale-independent
case mappings from scalar-value sequences to scalar-value sequences.
In particular, the length of the result string can be different than
the length of the input string.

The {\cf string-upcase} procedure converts a string to upper case;
{\cf string-downcase} converts a string to lowercase. The {\cf
  string-foldcase} procedure converts the string to its case-folded
counterpart, using the full case-folding mapping, but without the
special mappings for Turkic languages.  The {\cf string-titlecase}
procedure converts the first character to title case in each
contiguous sequence of cased characters within \var{string}, and it
downcases all other cased characters; for the purposes of detecting
cased-character sequences, case-ignorable characters are ignored
(i.e., they do not interrupt the sequence).

\begin{scheme}
(string-upcase "Hi") \ev "HI"
(string-downcase "Hi") \ev "hi"
(string-foldcase "Hi") \ev "hi"

(string-upcase "Stra\ss{}e") \ev "STRASSE"
(string-downcase "Stra\ss{}e") \ev "stra\ss{}e"
(string-foldcase "Stra\ss{}e") \ev "strasse"
(string-downcase "STRASSE")  \ev "strasse"

(string-downcase "$\Sigma$") \ev "$\sigma$"

; \textrm{Chi Alpha Omicron Sigma}:
(string-upcase "$\mathit{XAO}\Sigma$") \ev "$\mathit{XAO}\Sigma$" 
(string-downcase "$\mathit{XAO}\Sigma$") \ev "$\chi\alpha{}o\varsigma$"
(string-downcase "$\mathit{XAO}\Sigma\Sigma$") \ev "$\chi\alpha{}o\sigma\varsigma$"
(string-downcase "$\mathit{XAO}\Sigma~\Sigma$") \ev "$\chi\alpha{}o\varsigma~\sigma$"
(string-foldcase "$\mathit{XAO}\Sigma\Sigma$") \ev "$\chi\alpha{}o\sigma\sigma$"
(string-upcase "$\chi\alpha{}o\varsigma$") \ev "$\mathit{XAO}\Sigma$"
(string-upcase "$\chi\alpha{}o\sigma$") \ev "$\mathit{XAO}\Sigma$"

(string-titlecase "kNock KNoCK")
\ev "Knock Knock"
(string-titlecase "who's there?")
\ev "Who's There?"
(string-titlecase "r6rs") \ev "R6Rs"
(string-titlecase "R6RS") \ev "R6Rs"
\end{scheme}

\begin{note}
  These mappings can be extracted from {\cf UnicodeData.txt}, {\cf
    SpecialCasing.txt}, {\cf WordBreakProprty.txt} 
    (the ``MidLetter'' property partly defines case-ignorable characters), 
    and {\cf CaseFolding.txt} from the Unicode Consortium.

  Since these procedures are locale-independent, they may not
  be completely appropriate for some locales.
\end{note}

\end{entry}

\begin{entry}{%
\proto{string-ci=?}{ \vari{string} \varii{string} \variii{string} \dotsfoo}{procedure}
\proto{string-ci<?}{ \vari{string} \varii{string} \variii{string} \dotsfoo}{procedure}
\proto{string-ci>?}{ \vari{string} \varii{string} \variii{string} \dotsfoo}{procedure}
\proto{string-ci<=?}{ \vari{string} \varii{string} \variii{string} \dotsfoo}{procedure}
\proto{string-ci>=?}{ \vari{string} \varii{string} \variii{string} \dotsfoo}{procedure}}

These procedures are similar to {\cf string=?}\ et cetera, but 
operate on the case-folded versions of the strings.

\begin{scheme}
(string-ci<? "z" "Z") \ev \schfalse
(string-ci=? "z" "Z") \ev \schtrue
(string-ci=? "Stra\ss{}e" "Strasse") 
\ev \schtrue
(string-ci=? "Stra\ss{}e" "STRASSE")
\ev \schtrue
(string-ci=? "$\mathit{XAO}\Sigma$" "$\chi\alpha{}o\sigma$")
\ev \schtrue
\end{scheme}

\end{entry}

\begin{entry}{
\proto{string-normalize-nfd}{ \var{string}}{procedure}
\proto{string-normalize-nfkd}{ \var{string}}{procedure}
\proto{string-normalize-nfc}{ \var{string}}{procedure}
\proto{string-normalize-nfkc}{ \var{string}}{procedure}}
  
These procedures take a string argument and return a string
result, which is the input string normalized
to Unicode normalization form D, KD, C, or KC, respectively.

\begin{scheme}
(string-normalize-nfd "\backwhack{}xE9;")
\ev "\backwhack{}x65;\backwhack{}x301;"
(string-normalize-nfc "\backwhack{}xE9;")
\ev "\backwhack{}xE9;"
(string-normalize-nfd "\backwhack{}x65;\backwhack{}x301;")
\ev "\backwhack{}x65;\backwhack{}x301;"
(string-normalize-nfc "\backwhack{}x65;\backwhack{}x301;")
\ev "\backwhack{}xE9;"
\end{scheme}
\end{entry}

%%% Local Variables: 
%%% mode: latex
%%% TeX-master: "r6rs"
%%% End: 
 \par
\chapter{Bytevectors}
\label{bytevectorschapter}

Many applications deal with blocks of binary data by accessing
them in various ways---extracting signed or unsigned numbers of
various sizes.  Therefore, the \defrsixlibrary{bytevectors} library
provides a single type for
blocks of binary data with multiple ways to access that data. It deals
with integers and floating-point representations 
in various sizes with specified endianness.

Bytevectors\mainindex{bytevector} are objects of a disjoint
type. Conceptually, a bytevector represents a sequence of 8-bit
bytes.  The description of bytevectors uses the term \defining{byte}
for an exact integer object in the interval $\{-128, \ldots, 127\}$ and the
term \defining{octet} for an exact integer object in the interval $\{0,
\ldots, 255\}$.  A byte corresponds to its two's complement
representation as an octet.

The length of a bytevector is the number of bytes it contains. This
number is fixed. A valid index into a bytevector is an exact,
non-negative integer object less than the length of the bytevector.
The first byte of a bytevector has index 0;
the last byte has an index one less than the length of the bytevector.

Generally, the access procedures come in different flavors according
to the size of the represented integer and the endianness of the
representation.  The procedures also distinguish signed and unsigned
representations.
The signed representations all use two's complement.

Like string literals, literals representing bytevectors do not need to
be quoted:
%
\begin{scheme}
\#vu8(12 23 123) \ev \#vu8(12 23 123)%
\end{scheme}

\section{Endianness}

Many operations described in this chapter accept an
\defining{endianness} argument.  Endianness describes the encoding of
exact integer objects as several contiguous bytes in a bytevector~\cite{IEN137}. 
For this purpose, the binary representation of the integer object is split into
consecutive bytes.  \mainindex{little-endian}The little-endian
encoding places the least significant byte of an integer first, with
the other bytes following in increasing order of significance.
\mainindex{big-endian}The big-endian encoding places the most
significant byte of an integer first, with the other bytes following
in decreasing order of significance. 

This terminology also applies to IEEE-754 numbers: IEEE~754 describes
how to represent a floating-point number as an exact integer object, and
endianness describes how the bytes of such an integer are laid out in
a bytevector.

\begin{note}
  Little- and big-endianness are only the most common kinds of
  endianness.  Some architectures distinguish between the endianness
  at different levels of a binary representation.
\end{note}

\section{General operations}

\begin{entry}{%
\proto{endianness}{ \hyper{endianness symbol}}{\exprtype}}

\domain{The name of \hyper{endianness symbol} must be a symbol describing an
  endianness.  An implementation must support at least the symbols
  {\cf big} and {\cf little}, but may support other endianness
  symbols.}  {\cf (endianness \hyper{endianness symbol})} evaluates to
the symbol named \hyper{endianness symbol}.  Whenever one of the
procedures operating on bytevectors accepts an endianness as an
argument, that argument must be one of these symbols.  It is a syntax
violation for \hyper{endianness symbol} to be anything other than an
endianness symbol supported by the implementation.

\begin{note}
  Implementors should use widely accepted designations
  for endianness symbols other than {\cf big} and {\cf little}.
\end{note}

\begin{note}
  Only the name of \hyper{endianness symbol} is significant.
\end{note}
\end{entry}

\begin{entry}{%
\proto{native-endianness}{}{procedure}}

Returns the endianness symbol associated implementation's preferred
endianness (usually that of the underlying machine architecture).
This may be any \hyper{endianness symbol}, including a symbol other
than {\cf big} and {\cf little}.
\end{entry}   

\begin{entry}{%
\proto{bytevector?}{ obj}{procedure}}
   
Returns \schtrue{} if \var{obj} is a bytevector,
otherwise returns \schfalse{}.
\end{entry}

\begin{entry}{%
\proto{make-bytevector}{ k}{procedure}
\rproto{make-bytevector}{ k fill}{procedure}}
   
Returns a newly allocated bytevector of \var{k} bytes.
   
If the \var{fill} argument is missing, the initial contents of the
returned bytevector are unspecified.
   
If the \var{fill} argument is present, it must be an exact integer
object in
the interval $\{-128, \ldots 255\}$ that specifies the initial value
for the bytes of the bytevector: If \var{fill} is positive, it is
interpreted as an octet; if it is negative, it is interpreted as a byte.
\end{entry}   

\begin{entry}{%
\proto{bytevector-length}{ bytevector}{procedure}}
   
Returns, as an exact integer object, the number of bytes in \var{bytevector}.
\end{entry}

\begin{entry}{%
\proto{bytevector=?}{ \vari{bytevector} \varii{bytevector}}{procedure}}
   
Returns \schtrue{} if \vari{bytevector} and \varii{bytevector} are equal---that
is, if they have the same length and equal bytes at all valid indices.
It returns \schfalse{} otherwise.
\end{entry}

\begin{entry}{%
\proto{bytevector-fill!}{ bytevector fill}}

\domain{The \var{fill} argument is as in the description of the {\cf
    make-bytevector} procedure.}
The {\cf bytevector-fill!} procedure stores \var{fill} in every element of \var{bytevector}
and returns \unspecifiedreturn.  Analogous to {\cf vector-fill!}.
\end{entry}

\begin{entry}{%
\pproto{(bytevector-copy! \var{source} \var{source-start}}{procedure}}
\mainschindex{bytevector-copy!}{\tt\obeyspaces\\
     \var{target} \var{target-start} \var{k})}

\domain{\var{Source} and \var{target} must be bytevectors.
  \var{Source-start}, \var{target-start},
  and \var{k} must be non-negative exact integer objects that satisfy
  
  \begin{displaymath}
    \begin{array}{rcccccl}
      0 & \leq & \var{source-start} & \leq & \var{source-start} + \var{k} & \leq & l_{\var{source}}
      \\
      0 & \leq & \var{target-start} & \leq & \var{target-start} + \var{k} & \leq & l_{\var{target}}
    \end{array}
  \end{displaymath}
  %
  where $l_{\var{source}}$ is the length of \var{source} and
  $l_{\var{target}}$ is the length of \var{target}.}
   
   
  The {\cf bytevector-copy!} procedure copies the bytes from \var{source} at indices 
  \begin{displaymath}
     \var{source-start}, \ldots \var{source-start} + \var{k} - 1
  \end{displaymath}
  to consecutive indices in \var{target} starting at \var{target-index}.
   
  This must work even if the memory regions for the source and the target
  overlap, i.e., the bytes at the target location after the copy must be
  equal to the bytes at the source location before the copy.
   
  This returns \unspecifiedreturn.
\begin{scheme}
(let ((b (u8-list->bytevector '(1 2 3 4 5 6 7 8))))
  (bytevector-copy! b 0 b 3 4)
  (bytevector->u8-list b)) \ev (1 2 3 1 2 3 4 8)%
\end{scheme}
\end{entry}

\begin{entry}{%
\proto{bytevector-copy}{ bytevector}{procedure}}
   
Returns a newly allocated copy of \var{bytevector}.
\end{entry}

\section{Operations on bytes and octets}

\begin{entry}{%
\proto{bytevector-u8-ref}{ bytevector k}{procedure}
\proto{bytevector-s8-ref}{ bytevector k}{procedure}}
   
\domain{\var{K} must be a valid index of \var{bytevector}.}
   
The {\cf bytevector-u8-ref} procedure returns the byte at index \var{k} of \var{bytevector},
as an octet.
   
The {\cf bytevector-s8-ref} procedure returns the byte at index \var{k} of \var{bytevector},
as a (signed) byte.

\begin{scheme}
(let ((b1 (make-bytevector 16 -127))
      (b2 (make-bytevector 16 255)))
  (list
    (bytevector-s8-ref b1 0)
    (bytevector-u8-ref b1 0)
    (bytevector-s8-ref b2 0)
    (bytevector-u8-ref b2 0))) \lev (-127 129 -1 255)%
\end{scheme}
\end{entry}   

\begin{entry}{%
\proto{bytevector-u8-set!}{ bytevector k octet}{procedure}
\proto{bytevector-s8-set!}{ bytevector k byte}{procedure}}
   
\domain{\var{K} must be a valid index of \var{bytevector}.}
   
The {\cf bytevector-u8-set!} procedure stores \var{octet} in element \var{k} of
\var{bytevector}.
   
The {\cf bytevector-s8-set!} procedure stores the two's-complement representation of
\var{byte} in element \var{k} of \var{bytevector}.
   
Both procedures return \unspecifiedreturn.

\begin{scheme}
(let ((b (make-bytevector 16 -127)))

  (bytevector-s8-set! b 0 -126)
  (bytevector-u8-set! b 1 246)

  (list
    (bytevector-s8-ref b 0)
    (bytevector-u8-ref b 0)
    (bytevector-s8-ref b 1)
    (bytevector-u8-ref b 1))) \lev (-126 130 -10 246)%
\end{scheme}
\end{entry}

\begin{entry}{%
\proto{bytevector->u8-list}{ bytevector}{procedure}
\proto{u8-list->bytevector}{ list}{procedure}}
   
\domain{\var{List} must be a list of octets.}

The {\cf bytevector->u8-list} procedure returns a newly allocated list of the octets of
\var{bytevector} in the same order.

The {\cf u8-list->bytevector} procedure returns a newly allocated bytevector whose
elements are the elements of list \var{list}, in
the same order.  It is analogous to {\cf list->vector}.
\end{entry}

\section{Operations on integers of arbitrary size}

\begin{entry}{%
\proto{bytevector-uint-ref}{ bytevector k endianness size}{procedure}
\proto{bytevector-sint-ref}{ bytevector k endianness size}{procedure}
\proto{bytevector-uint-set!}{ bytevector k n endianness size}{procedure}
\proto{bytevector-sint-set!}{ bytevector k n endianness size}{procedure}}
   
\domain{\var{Size} must be a positive exact integer object. $\var{K}, \ldots,
  \var{k} + \var{size} - 1$ must be valid indices of \var{bytevector}.}
   
The {\cf bytevector-uint-ref} procedure retrieves the exact integer object corresponding to the
unsigned representation of size \var{size} and specified by \var{endianness}
at indices $\var{k}, \ldots, \var{k} + \var{size} - 1$.
   
The {\cf bytevector-sint-ref} procedure retrieves the exact integer object corresponding to the two's-complement representation of size \var{size} and specified by \var{endianness} at
indices $\var{k}, \ldots, \var{k} + \var{size} - 1$.
   
\domain{For {\cf bytevector-uint-set!}, \var{n} must be an exact
  integer object in the interval $\{0, \ldots, 256^{\mathit{size}}-1\}$.}

The {\cf bytevector-uint-set!} procedure stores the unsigned representation of size \var{size}
and specified by \var{endianness} into \var{bytevector} at indices
$\var{k}, \ldots, \var{k} + \var{size} - 1$.
   
\domain{For {\cf bytevector-sint-set!}, \var{n} must be an exact
  integer object in
  the interval $\{-256^{\mathit{size}}/2, \ldots,
  256^{\mathit{size}}/2-1\}$.}
{\cf bytevector-sint-set!} stores the two's-complement
representation of size \var{size} and specified by \var{endianness}
into \var{bytevector} at indices $\var{k}, \ldots, \var{k} + \var{size} - 1$.
   
The \ldots{\cf -set!} procedures return \unspecifiedreturn.

\begin{scheme}
(define b (make-bytevector 16 -127))

(bytevector-uint-set! b 0 (- (expt 2 128) 3)
                     (endianness little) 16)

(bytevector-uint-ref b 0 (endianness little) 16)\lev
    \#xfffffffffffffffffffffffffffffffd

(bytevector-sint-ref b 0 (endianness little) 16)\lev -3

(bytevector->u8-list b)\lev (253 255 255 255 255 255 255 255
               255 255 255 255 255 255 255 255)

(bytevector-uint-set! b 0 (- (expt 2 128) 3)
                 (endianness big) 16)

(bytevector-uint-ref b 0 (endianness big) 16) \lev
    \#xfffffffffffffffffffffffffffffffd

(bytevector-sint-ref b 0 (endianness big) 16) \lev -3

(bytevector->u8-list b) \lev (255 255 255 255 255 255 255 255
               255 255 255 255 255 255 255 253))%
\end{scheme}
\end{entry}

\begin{entry}{%
\proto{bytevector->uint-list}{ bytevector endianness size}{procedure}
\proto{bytevector->sint-list}{ bytevector endianness size}{procedure}
\proto{uint-list->bytevector}{ list endianness size}{procedure}
\proto{sint-list->bytevector}{ list endianness size}{procedure}}
   
\domain{\var{Size} must be a positive exact integer object.  For {\cf
    uint-list->bytevector}, \var{list} must be a list of exact
  integer objects in the interval $\{0, \ldots, 256^{\mathit{size}}-1\}$.  For
  {\cf sint-list->bytevector}, \var{list} must be a list of exact
  integer objects in the interval $\{-256^{\mathit{size}}/2, \ldots,
  256^{\mathit{size}}/2-1\}$.  The length of \var{bytevector} or,
  respectively, of \var{list} must be divisible by \var{size}.}
   
These procedures convert between lists of integer objects and their consecutive
representations according to \var{size} and \var{endianness} in the
\var{bytevector} objects in the same way as {\cf bytevector->u8-list} and {\cf
  u8-list->bytevector} do for one-byte representations.

\begin{scheme}
(let ((b (u8-list->bytevector '(1 2 3 255 1 2 1 2))))
  (bytevector->sint-list b (endianness little) 2)) \lev (513 -253 513 513)

(let ((b (u8-list->bytevector '(1 2 3 255 1 2 1 2))))
  (bytevector->uint-list b (endianness little) 2)) \lev (513 65283 513 513)%
\end{scheme}
\end{entry}

\section{Operations on 16-bit integers}

\begin{entry}{%
\proto{bytevector-u16-ref}{ bytevector k endianness}{procedure}
\proto{bytevector-s16-ref}{ bytevector k endianness}{procedure}
\proto{bytevector-u16-native-ref}{ bytevector k}{procedure}
\proto{bytevector-s16-native-ref}{ bytevector k}{procedure}
\proto{bytevector-u16-set!}{ bytevector k n endianness}{procedure}
\proto{bytevector-s16-set!}{ bytevector k n endianness}{procedure}
\proto{bytevector-u16-native-set!}{ bytevector k n}{procedure}
\proto{bytevector-s16-native-set!}{ bytevector k n}{procedure}}
   
\domain{\var{K} must be a valid index of \var{bytevector}; so must
  $\var{k} + 1$. For {\cf bytevector-u16-set!} and {\cf
    bytevector-\hp{}u16-native-\hp{}set!}, \var{n} must be an exact integer object in
  the interval $\{0, \ldots, 2^{16}-1\}$.  For {\cf bytevector-s16-set!}
  and {\cf bytevector-s16-native-set!}, \var{n} must be an exact
  integer object in the interval $\{-2^{15}, \ldots, 2^{15}-1\}$.}
   
These retrieve and set two-byte representations of numbers at indices
\var{k} and $\var{k}+1$, according to the endianness specified by
\var{endianness}. The procedures with {\cf u16} in their names deal with the
unsigned representation; those with {\cf s16} in their names deal
with the two's-complement representation.

The procedures with {\cf native} in their names employ the native
endianness, and work only at aligned indices:
\var{k} must be a multiple of 2.
   
The \ldots{\cf -set!} procedures return \unspecifiedreturn.

\begin{scheme}
(define b
  (u8-list->bytevector
    '(255 255 255 255 255 255 255 255
      255 255 255 255 255 255 255 253)))

(bytevector-u16-ref b 14 (endianness little)) \lev 65023
(bytevector-s16-ref b 14 (endianness little)) \lev -513
(bytevector-u16-ref b 14 (endianness big)) \lev 65533
(bytevector-s16-ref b 14 (endianness big)) \lev -3

(bytevector-u16-set! b 0 12345 (endianness little))
(bytevector-u16-ref b 0 (endianness little)) \lev 12345

(bytevector-u16-native-set! b 0 12345)
(bytevector-u16-native-ref b 0) \ev 12345

(bytevector-u16-ref b 0 (endianness little)) \lev \unspecified%
\end{scheme}
\end{entry}

\section{Operations on 32-bit integers}

\begin{entry}{%
\proto{bytevector-u32-ref}{ bytevector k endianness}{procedure}
\proto{bytevector-s32-ref}{ bytevector k endianness}{procedure}
\proto{bytevector-u32-native-ref}{ bytevector k}{procedure}
\proto{bytevector-s32-native-ref}{ bytevector k}{procedure}
\proto{bytevector-u32-set!}{ bytevector k n endianness}{procedure}
\proto{bytevector-s32-set!}{ bytevector k n endianness}{procedure}
\proto{bytevector-u32-native-set!}{ bytevector k n}{procedure}
\proto{bytevector-s32-native-set!}{ bytevector k n}{procedure}}
   
\domain{$\var{K}, \ldots, \var{k}+ 3$ must be valid indices of
  \var{bytevector}.
  For {\cf bytevector-u32-set!} and {\cf
    bytevector-\hp{}u32-\hp{}native-\hp{}set!}, \var{n} must be an exact integer
  object in
  the interval $\{0, \ldots, 2^{32}-1\}$.  For {\cf bytevector-s32-set!}
  and {\cf bytevector-s32-native-set!}, \var{n} must be an exact
  integer object in the interval $\{-2^{31}, \ldots, 2^{32}-1\}$.}
   
These retrieve and set four-byte representations of numbers at indices $\var{k},
\ldots, \var{k}+ 3$, according to the endianness specified by \var{endianness}. The
procedures with {\cf u32} in their names deal with the unsigned representation;
those with {\cf s32} with the two's-complement representation.
   
The procedures with {\cf native} in their names employ the native endianness, and
work only at aligned indices: \var{k} must be a multiple of 4.
   
The \ldots{\cf{}-set!} procedures return \unspecifiedreturn.

\begin{scheme}
(define b
  (u8-list->bytevector
    '(255 255 255 255 255 255 255 255
      255 255 255 255 255 255 255 253)))

(bytevector-u32-ref b 12 (endianness little)) \lev 4261412863
(bytevector-s32-ref b 12 (endianness little)) \lev -33554433
(bytevector-u32-ref b 12 (endianness big)) \lev 4294967293
(bytevector-s32-ref b 12 (endianness big)) \lev -3%
\end{scheme}
\end{entry}

\section{Operations on 64-bit integers}

\begin{entry}{%
\proto{bytevector-u64-ref}{ bytevector k endianness}{procedure}
\proto{bytevector-s64-ref}{ bytevector k endianness}{procedure}
\proto{bytevector-u64-native-ref}{ bytevector k}{procedure}
\proto{bytevector-s64-native-ref}{ bytevector k}{procedure}
\proto{bytevector-u64-set!}{ bytevector k n endianness}{procedure}
\proto{bytevector-s64-set!}{ bytevector k n endianness}{procedure}
\proto{bytevector-u64-native-set!}{ bytevector k n}{procedure}
\proto{bytevector-s64-native-set!}{ bytevector k n}{procedure}}
 
\domain{$\var{K}, \ldots, \var{k}+ 7$ must be valid indices of
  \var{bytevector}.
  For {\cf bytevector-u64-set!} and {\cf
    bytevector-\hp{}u64-\hp{}native-\hp{}set!}, \var{n} must be an exact integer
  object in
  the interval $\{0, \ldots, 2^{64}-1\}$.  For {\cf bytevector-s64-set!}
  and {\cf bytevector-s64-native-set!}, \var{n} must be an exact
  integer object in the interval $\{-2^{63}, \ldots, 2^{64}-1\}$.}
   
These retrieve and set eight-byte representations of numbers at
indices $\var{k}, \ldots, \var{k}+ 7$, according to the endianness
specified by \var{endianness}. The procedures with {\cf u64} in their names deal
with the unsigned representation; those with {\cf s64} with the
two's-complement representation.
   
The procedures with {\cf native} in their names employ the native endianness, and
work only at aligned indices: \var{k} must be a multiple of 8.
   
The \ldots{\cf{}-set!} procedures return \unspecifiedreturn.

\begin{scheme}
(define b
  (u8-list->bytevector
    '(255 255 255 255 255 255 255 255
      255 255 255 255 255 255 255 253)))

(bytevector-u64-ref b 8 (endianness little)) \lev 18302628885633695743
(bytevector-s64-ref b 8 (endianness little)) \lev -144115188075855873
(bytevector-u64-ref b 8 (endianness big)) \lev 18446744073709551613
(bytevector-s64-ref b 8 (endianness big)) \lev -3%
\end{scheme}
\end{entry}

\section{Operations on IEEE-754 representations}

\begin{entry}{%
\proto{bytevector-ieee-single-native-ref}{ bytevector k}{procedure}
\proto{bytevector-ieee-single-ref}{ bytevector k endianness}{procedure}}

\domain{$\var{K}, \ldots, \var{k}+3$ must be valid indices of
  \var{bytevector}.  For {\cf bytevector-ieee-single-native-ref}, \var{k} must
  be a multiple of $4$.}

These procedures return the inexact real number object that best
represents the IEEE-754 single-precision number represented by the
four bytes beginning at index
\var{k}.
\end{entry}

\begin{entry}{%
\proto{bytevector-ieee-double-native-ref}{ bytevector k}{procedure}
\proto{bytevector-ieee-double-ref}{ bytevector k endianness}{procedure}}

\domain{$\var{K}, \ldots, \var{k}+7$ must be valid indices of
  \var{bytevector}.  For {\cf bytevector-ieee-double-native-ref}, \var{k} must
  be a multiple of $8$.}

These procedures return the inexact real number object that best
represents the IEEE-754 double-precision number represented by the
eight bytes beginning at index \var{k}.
\end{entry}

\begin{entry}{%
\proto{bytevector-ieee-single-native-set!}{ bytevector k x}{procedure}
\pproto{(bytevector-ieee-single-set! \var{bytevector}}{procedure}}
{\tt\obeyspaces\\
     \var{k} \var{x} \var{endianness})}

\domain{$\var{K}, \ldots, \var{k}+3$ must be valid indices of
  \var{bytevector}.  For {\cf bytevector-ieee-single-native-set!}, \var{k} must
  be a multiple of $4$.}

These procedures store an IEEE-754 single-precision representation of \var{x} into
elements \var{k} through $\var{k}+3$ of \var{bytevector}, and return
\unspecifiedreturn.
\end{entry}

\begin{entry}{%
\proto{bytevector-ieee-double-native-set!}{ bytevector k x}{procedure}
\pproto{(bytevector-ieee-double-set! \var{bytevector}}{procedure}}
{\tt\obeyspaces\\
     \var{k} \var{x} \var{endianness})}

\domain{$\var{K}, \ldots, \var{k}+7$ must be valid indices of
  \var{bytevector}.  For {\cf bytevector-ieee-double-native-set!}, \var{k} must
  be a multiple of $8$.}

These procedures store an IEEE-754 double-precision representation of \var{x} into
elements \var{k} through $\var{k}+7$ of \var{bytevector}, and return
\unspecifiedreturn.
\end{entry}

\section{Operations on strings}

This section describes procedures that convert between strings and
bytevectors containing Unicode encodings of those strings.  When
decoding bytevectors, encoding errors are handled as with the {\cf
  replace} semantics of textual I/O (see
section~\ref{transcoderssection}): If an invalid or incomplete
character encoding is encountered, then the replacement character
U+FFFD is appended to the string being generated, an appropriate
number of bytes are ignored, and decoding continues with the following
bytes.

\begin{entry}{%
\proto{string->utf8}{ string}{procedure}}

Returns a newly allocated (unless empty) bytevector that
contains the UTF-8 encoding of the given string.
\end{entry}

\begin{entry}{%
\proto{string->utf16}{ string}{procedure}
\rproto{string->utf16}{ string endianness}{procedure}}

\domain{If \var{endianness} is specified, it must be the symbol {\cf
    big} or the symbol {\cf little}.}  The {\cf string->utf16}
  procedure returns a newly allocated (unless empty) bytevector that
contains the UTF-16BE or UTF-16LE encoding of the given string (with
no byte-order mark).  If endianness is not specified or is {\cf big},
then UTF-16BE is used.  If endianness is {\cf little}, then UTF-16LE
is used.
\end{entry}

\begin{entry}{%
\proto{string->utf32}{ string}{procedure}
\rproto{string->utf32}{ string endianness}{procedure}}

\domain{If \var{endianness} is specified, it must be the symbol {\cf
    big} or the symbol {\cf little}.}  The {\cf string->utf32}
procedure returns
a newly allocated (unless empty) bytevector that contains the UTF-32BE
or UTF-32LE encoding of the given string (with no byte mark).  If
endianness is not specified or is {\cf big}, then UTF-32BE is used.
If endianness is {\cf little}, then UTF-32LE is used.
\end{entry}

\begin{entry}{%
\proto{utf8->string}{ bytevector}{procedure}}

Returns a newly allocated (unless empty) string whose character
sequence is encoded by the given bytevector.
\end{entry}

\begin{entry}{%
\proto{utf16->string}{ bytevector endianness}{procedure}
\pproto{(utf16->string \var{bytevector}}{procedure}}
{\tt\obeyspaces\\
    \var{endianness} \var{endianness-mandatory})}

\domain{\var{Endianness} must be the symbol {\cf big} or
  the symbol {\cf little}.} The {\cf utf16->string} procedure returns
a newly allocated (unless empty) string whose character sequence is
encoded by the given bytevector.  \var{Bytevector} is decoded
according to UTF-16BE or UTF-16LE: If \var{endianness-mandatory?} is
absent or \schfalse, {\cf utf16->string} determines the endianness
according to a UTF-16 BOM at the beginning of \var{bytevector} if a
BOM is present; in this case, the BOM is not decoded as a character.
Also in this case, if no UTF-16 BOM is present, \var{endianness}
specifies the endianness of the encoding.  If
\var{endianness-mandatory?} is a true value, \var{endianness}
specifies the endianness of the encoding, and any UTF-16 BOM in the
encoding is decoded as a regular character.

\begin{note}
  A UTF-16 BOM is either a sequence of bytes \sharpsign{}xFE,
  \sharpsign{}xFF specifying {\cf big} and UTF-16BE, or \sharpsign{}xFF,
  \sharpsign{}xFE specifying {\cf little} and UTF-16LE.
\end{note}
\end{entry}

\begin{entry}{%
\proto{utf32->string}{ bytevector endianness}{procedure}
\pproto{(utf32->string \var{bytevector}}{procedure}}
{\tt\obeyspaces\\
    \var{endianness} \var{endianness-mandatory})}

\domain{\var{Endianness} must be the symbol {\cf big} or
  the symbol {\cf little}.} The {\cf utf32->string} procedure returns
a newly allocated (unless empty) string whose character sequence is
encoded by the given bytevector.  \var{Bytevector} is decoded
according to UTF-32BE or UTF-32LE: If \var{endianness-mandatory?} is
absent or \schfalse, {\cf utf32->string} determines the endianness
according to a UTF-32 BOM at the beginning of \var{bytevector} if a
BOM is present; in this case, the BOM is not decoded as a character.
Also in this case, if no UTF-32 BOM is present, \var{endianness}
specifies the endianness of the encoding.  If
\var{endianness-mandatory?} is a true value, \var{endianness}
specifies the endianness of the encoding, and any UTF-32 BOM in the
encoding is decoded as a regular character.

\begin{note}
  A UTF-32 BOM is either a sequence of bytes \sharpsign{}x00,
  \sharpsign{}x00, \sharpsign{}xFE, \sharpsign{}xFF specifying {\cf
    big} and UTF-32BE, or \sharpsign{}xFF, \sharpsign{}xFE,
  \sharpsign{}x00, \sharpsign{}x00, specifying {\cf little} and
  UTF-32LE.
\end{note}
\end{entry}

%%% Local Variables: 
%%% mode: latex
%%% TeX-master: "r6rs-lib"
%%% End: 
 \par
\chapter{List utilities}
\label{listutilities}

This chapter describes the \deflibrary{r6rs lists} library.

\begin{entry}{%
\proto{find}{ proc list}{procedure}}

\domain{\var{Proc} must be a procedure; it must take a single argument
  if \var{list} is non-empty.}  The {\cf find} procedure applies
\var{proc} to the elements of \var{list} in order.  If
\var{proc} returns a true value for an element, {\cf find}
immediately returns that element.  If \var{proc} returns
\schfalse{} for all elements of the list, it returns \schfalse{}.

\begin{scheme}
(find even? '(3 1 4 1 5 9)) \ev 4
(find even? '(3 1 5 1 5 9)) \ev \schfalse{}
\end{scheme}
  
\end{entry}

\begin{entry}{%
\proto{forall}{ proc \vari{l} \varii{l} \dotsfoo{} \varn{l}}{procedure}
\proto{exists}{ proc \vari{l} \varii{l} \dotsfoo{} \varn{l}}{procedure}}

\domain{Each \var{l} must be the
  empty list or a chain of pairs according to the conditions specified
  below.  \var{Proc} must be a procedure; it must take as many
  arguments as there are \var{l}s if \vari{l} is non-empty.}

For natural numbers $i = 0, 1, \ldots$, the {\cf forall} procedure
successively applies \var{proc} to arguments $x_i^1 \ldots x_i^n$,
where $x_i^j$ is the $i$th element of \varj{l}, until \schfalse{} is
returned.  If \var{proc} returns true values for all but the last
element of \vari{l}, \var{forall} performs a tail call of \var{proc}
on the $k$th elements, where $k$ is the length of \vari{l}---in this
case, the \var{l}s must all be lists of length $k$.  If \var{proc}
returns \schfalse{} on any set of elements, {\cf forall} returns
\schfalse{} after the first such application of \var{proc} without
further traversing the \var{l}s.  If the \var{l}s are all empty, {\cf
  forall} returns \schtrue.

For natural numbers $i = 0, 1, \ldots$, the {\cf exists} procedure
applies \var{proc} successively to arguments $x_i^1 \ldots x_i^n$,
where $x_i^j$ is the $i$th element of \varj{l}, until a true value is
returned.  If \var{proc} returns \schfalse{} for all but the last
elements of the \var{l}s, \var{exists} performs a tail call of
\var{proc} on the $k$th elements, where $k$ is the length of
\vari{l}---in this case, the \var{l}s must all be lists of length $k$.
If \var{proc} returns a true value on any set of elements, {\cf
  exists} returns that value after the first such application of
\var{proc} without further traversing the \var{l}s.  If the \var{l}s
are all empty, {\cf exists} returns \schfalse.

\begin{scheme}
(forall even? '(3 1 4 1 5 9)) \lev \schfalse{}
(forall even? '(3 1 4 1 5 9 . 2)) \lev \schfalse{}
(forall even? '(2 4 14)) \ev \schtrue{}
(forall even? '(2 4 14 . 9)) \lev \exception{\cf\&contract}
(forall (lambda (n) (and (even? n) n)) '(2 4 14)) \lev 14
(forall < '(1 2 3) '(2 3 4)) \ev \schtrue{}
(forall < '(1 2 4) '(2 3 4)) \ev \schfalse{}

(exists even? '(3 1 4 1 5 9)) \lev \schtrue{}
(exists even? '(3 1 1 5 9)) \ev \schfalse{}
(exists even? '(3 1 1 5 9 . 2)) \lev \exception{\cf\&contract}
(exists (lambda (n) (and (even? n) n)) '(2 1 4 14)) \lev 2
(exists < '(1 2 4) '(2 3 4)) \ev \schtrue{}
(exists > '(1 2 3) '(2 3 4)) \ev \schfalse{}
\end{scheme}
\end{entry}

\begin{entry}{%
\proto{filter}{ proc list}{procedure}
\proto{partition}{ proc list}{procedure}
}

\domain{\var{Proc} must be a procedure; it must take a single argument
  if \var{list} is non-empty.}  The {\cf filter} procedure successively applies
\var{proc} to the elements of \var{list} and returns a list of
the values of \var{list} for which \var{proc} returned a true
value.  The {\cf partition} procedure also successively applies \var{proc} to
the elements of \var{list}, but returns two values, the first one a
list of the values of \var{list} for which \var{proc} returned a
true value, and the second a list of the values of \var{list} for
which \var{proc} returned \schfalse.

\begin{scheme}
(filter even? '(3 1 4 1 5 9 2 6)) \lev (4 2 6)

(partition even? '(3 1 4 1 5 9 2 6)) \lev (4 2 6) (3 1 1 5 9) ; two values
\end{scheme}

\end{entry}

\begin{entry}{%
\proto{fold-left}{ combine nil \vari{list} \varii{list} \dotsfoo \varn{list}}{procedure}}

\domain{If
  more than one \var{list} is given, then they must all be the same
  length.  \var{Combine} must be a
  procedure; if the \var{list}s are non-empty, it must take one more
  argument than there are {\it list}s.}
The {\cf fold-left} procedure iterates the \var{combine} procedure over an
accumulator value and the values of the {\it list}s from left to
right, starting with an accumulator value of \var{nil}.  More
specifically, {\cf fold-left} returns \var{nil} if the {\it list}s are
empty.  If they are not empty, \var{combine} is first applied to
\var{nil} and the respective first elements of the {\it list}s in
order.  The result becomes the new accumulator value, and \var{combine}
is applied to new accumulator value and the respective next elements
of the {\it list}.  This step is repeated until the end of the list is
reached; then the accumulator value is returned.

\begin{scheme}
(fold-left + 0 '(1 2 3 4 5)) \ev 15

(fold-left (lambda (a e) (cons e a)) '()
           '(1 2 3 4 5)) \lev (5 4 3 2 1)

(fold-left (lambda (count x)
             (if (odd? x) (+ count 1) count))
           0
           '(3 1 4 1 5 9 2 6 5 3)) \lev 7

(fold-left (lambda (max-len s)
             (max max-len (string-length s)))
           0
           '("longest" "long" "longer")) \lev 7

(fold-left cons '(q) '(a b c)) \lev ((((q) . a) . b) . c)

(fold-left + 0 '(1 2 3) '(4 5 6)) \lev 21
\end{scheme}
\end{entry}


\begin{entry}{%
\proto{fold-right}{ combine nil \vari{list} \varii{list} \dotsfoo \varn{list}}{procedure}}

\domain{ If
  more than one \var{list} is given, then they must all be the same
  length.  \var{Combine} must be a
  procedure; if the \var{list}s are non-empty, it must take one more
  argument than there are {\it list}s.}
The {\cf fold-right} procedure iterates the \var{combine} procedure over
the values of the {\it list}s from right to left and an accumulator
value, starting with an accumulator value of \var{nil}.  More
specifically, {\cf fold-right} returns \var{nil} if the {\it list}s
are empty.  If they are not empty, \var{combine} is first applied to the
respective last elements of the {\it list}s in order and \var{nil}.
The result becomes the new accumulator value, and \var{combine} is
applied to the respective previous elements of the {\it list} and the
new accumulator value.  This step is repeated until the beginning of the
list is reached; then the accumulator value is returned.

\begin{scheme}
(fold-right + 0 '(1 2 3 4 5)) \ev 15

(fold-right cons '() '(1 2 3 4 5)) \lev (1 2 3 4 5)

(fold-right (lambda (x l)
              (if (odd? x) (cons x l) l))
            '()
            '(3 1 4 1 5 9 2 6 5))
\ev (3 1 1 5 9 5)

(fold-right cons '(q) '(a b c)) \lev (a b c q)

(fold-right + 0 '(1 2 3) '(4 5 6)) \lev 21
\end{scheme}
\end{entry}

\begin{entry}{%
\proto{remp}{ proc list}{procedure}
\proto{remove}{ obj list}{procedure}
\proto{remv}{ obj list}{procedure}
\proto{remq}{ obj list}{procedure}}

\domain{\var{Proc} must be a procedure; it must take a single argument
  if \var{list} is non-empty.}
Each of these procedures returns a list of the elements of \var{list}
that do not satisfy a given condition.  The {\cf remp} procedure successively
applies \var{proc} to the elements of \var{list} and returns a
list of the values of \var{list} for which \var{proc} returned
\schfalse.  The {\cf remove}, {\cf remv}, and {\cf remq} procedures return a list of
the elements that are not \var{obj}.  The {\cf remq} procedure uses {\cf eq?}\ to
compare \var{obj} with the elements of \var{list}, while {\cf remv}
uses {\cf eqv?}\ and {\cf remove} uses {\cf equal?}.

\begin{scheme}
(remp even? '(3 1 4 1 5 9 2 6 5)) \lev (3 1 1 5 9 5)

(remove 1 '(3 1 4 1 5 9 2 6 5)) \lev (3 4 5 9 2 6 5)

(remv 1 '(3 1 4 1 5 9 2 6 5)) \lev (3 4 5 9 2 6 5)

(remq 'foo '(bar foo baz)) \ev (bar baz)
\end{scheme}
\end{entry}

\begin{entry}{%
\proto{memp}{ proc l}{procedure}
\proto{member}{ obj l}{procedure}
\proto{memv}{ obj l}{procedure}
\proto{memq}{ obj l}{procedure}
}

\domain{\var{Proc} must be a procedure; it must take a single argument
  if \var{l} is non-empty.
  \var{l} must be the empty list or a chain of pairs of size according
to the conditions stated below.}

These procedures return the first sublist of \var{l} whose car
satisfies a given condition, where the subchains of \var{l} are the
chains of pairs returned by {\tt (list-tail \var{l} \var{k})} for
\var{k} less than the length of \var{l}.  The {\cf memp} procedure applies
\var{proc} to the cars of the sublists of \var{l} until it
finds one for which \var{proc} returns a true value without traversing
\var{l} further.  The {\cf
  member}, {\cf memv}, and {\cf memq} procedures look for the first occurrence of
\var{obj}.  If \var{l} does not contain an element satisfying the
condition, then \schfalse{} (not the empty list) is returned; in that
case, \var{l} must be a list.  The {\cf
  member} procedure uses {\cf equal?}\ to compare \var{obj} with the elements of
\var{l}, while {\cf memv} uses {\cf eqv?}\ and {\cf memq} uses
{\cf eq?}.

\begin{scheme}
(memp even? '(3 1 4 1 5 9 2 6 5)) \lev (4 1 5 9 2 6 5)

(memq 'a '(a b c))              \ev  (a b c)
(memq 'b '(a b c))              \ev  (b c)
(memq 'a '(b c d))              \ev  \schfalse
(memq (list 'a) '(b (a) c))     \ev  \schfalse
(member (list 'a)
        '(b (a) c))             \ev  ((a) c)
(memq 101 '(100 101 102))       \ev  \unspecified
(memv 101 '(100 101 102))       \ev  (101 102)%
\end{scheme} 
\begin{rationale}
  Although they are ordinarily used as predicates, {\cf memp}, {\cf
    member}, {\cf memv}, {\cf memq}, do not have question marks in
  their names because they return useful values rather than just
  \schtrue{} or \schfalse{}.
\end{rationale}
\end{entry}

\begin{entry}{%
\proto{assp}{ proc al}{procedure}
\proto{assoc}{ obj al}{procedure}
\proto{assv}{ obj al}{procedure}
\proto{assq}{ obj al}{procedure}}

\domain{\var{Al} (for ``association list'') must be the empty list or
  a chain of pairs of size according to
  the conditions specified below, where each car contains a pair.
  \var{Proc} must be a procedure; it
  must take a single argument if \var{al} is non-empty.}

These procedures find the first pair in \var{al}
whose car field satisfies a given condition, and returns that pair
without traversing \var{al} further.
If no pair in \var{al} satisfies the condition, then \schfalse{}
is returned; in that case, \var{al} must be a
list.  The {\cf assp} procedure successively applies
\var{proc} to the car fields of \var{al} and looks for a pair
for which it returns a true value.  The {\cf assoc}, {\cf assv}, and {\cf
  assq} procedures look for a pair that has \var{obj} as its car.  The
{\cf assoc} procedure uses 
{\cf equal?}\ to compare \var{obj} with the car fields of the pairs in
\var{al}, while {\cf assv} uses {\cf eqv?}\ and {\cf assq} uses
{\cf eq?}.


\begin{scheme}
(define d '((3 a) (1 b) (4 c)))

(assp even? d) \ev (4 c)
(assp odd? d) \ev (3 a)

(define e '((a 1) (b 2) (c 3)))
(assq 'a e)     \ev  (a 1)
(assq 'b e)     \ev  (b 2)
(assq 'd e)     \ev  \schfalse
(assq (list 'a) '(((a)) ((b)) ((c))))
                \ev  \schfalse
(assoc (list 'a) '(((a)) ((b)) ((c))))   
                           \ev  ((a))
(assq 5 '((2 3) (5 7) (11 13)))    
                           \ev  \unspecified
(assv 5 '((2 3) (5 7) (11 13)))    
                           \ev  (5 7)%
\end{scheme}

\end{entry}

%%% Local Variables: 
%%% mode: latex
%%% TeX-master: "r6rs"
%%% End: 

    \par
\chapter{Sorting}
\label{sortingchapter}

This chapter describes the \defrsixlibrary{sorting} library for
sorting lists and vectors.

\begin{entry}{%
\proto{list-sort}{ proc list}{procedure}
\proto{vector-sort}{ proc vector}{procedure}}

\domain{\var{Proc} should accept any two elements
  of \var{list} or \var{vector}, and should not have any side
  effects.}  \var{Proc} should return a true value when its first argument
is strictly less than its second, and \schfalse{} otherwise.

The {\cf list-sort} and {\cf vector-sort} procedures perform a stable
sort of \var{list} or \var{vector} in ascending order according to
\var{proc}, without changing \var{list} or
\var{vector} in any way.  The {\cf list-sort} procedure returns a
list, and {\cf vector-sort} returns a vector.  The results may be {\cf
  eq?} to the argument when the argument is already sorted, and the
result of {\cf list-sort} may share structure with a tail of the
original list.  The sorting algorithm performs $O(n \lg n)$ calls to
\var{proc} where $n$ is the length of \var{list} or \var{vector},
and all arguments passed to \var{proc} are elements of the list or
vector being sorted, but the pairing of arguments and the sequencing
of calls to \var{proc} are not specified.
If multiple returns occur from {\cf list-sort} or {\cf vector-sort}, the return
values returned by earlier returns are not mutated.

\begin{scheme}
(list-sort < '(3 5 2 1)) \ev (1 2 3 5)
(vector-sort < '\sharpsign(3 5 2 1)) \ev \sharpsign(1 2 3 5)%
\end{scheme}

\implresp The implementation must check the restrictions
on \var{proc} to the extent performed by applying it as described.
An
implementation may check whether \var{proc} is an appropriate argument
before applying it.
\end{entry}

\begin{entry}{%
\proto{vector-sort!}{ proc vector}{procedure}}

\domain{\var{Proc} should accept any two elements
  of the vector, and should not have any side
  effects.  \var{Proc} should return a true value when its first
argument is strictly less than its second, and \schfalse{} otherwise.}
The {\cf vector-sort!} procedure destructively sorts \var{vector} in
ascending order according to \var{proc}.  The sorting algorithm
performs $O(n^2)$ calls to \var{proc} where $n$ is the length of
\var{vector}, and all arguments passed to \var{proc} are elements
of the vector being sorted, but the pairing of arguments and the
sequencing of calls to \var{proc} are not specified.  The sorting
algorithm may be unstable.  The procedure returns \unspecifiedreturn.

\begin{scheme}
(define v (vector 3 5 2 1))
(vector-sort! v) \ev \theunspecified
v \ev \sharpsign(1 2 3 5)
\end{scheme}
\implresp The implementation must check the restrictions
on \var{proc} to the extent performed by applying it as described.
An
implementation may check whether \var{proc} is an appropriate argument
before applying it.
\end{entry}

%%% Local Variables: 
%%% mode: latex
%%% TeX-master: "r6rs-lib"
%%% End: 
    \par
\chapter{Records}
\label{recordschapter}
\mainindex{record}
This section describes abstractions for creating new data types
representing records---data structures with named fields. The record
mechanism comes in four libraries:

\begin{itemize}
\item the \library{r6rs records procedural} library,
  a procedural layer for creating and manipulating record types and record
  instances,
\item the \library{r6rs records explicit} library,
  an explicit-naming syntactic layer for defining record types and
  explicitly named bindings for various procedures to manipulate the record
  type,
\item the \library{r6rs records implicit} library,
  an implicit-naming syntactic layer that extends the explicit-naming
  syntactic layer, allowing the names of the defined procedures to be
  determined implicitly from the names of the record type and fields, and
\item the \library{r6rs records inspection} library,
  a set of inspection procedures.
\end{itemize}
% 
The procedural layer allows programs to construct new record types
and the associated procedures for creating and manipulating records
dynamically.
It is particularly useful for writing interpreters that construct
host-compatible record types.  It may also serve as a target for expansion
of the syntactic layers.

The explicit-naming syntactic layer provides a basic syntactic interface
whereby a single record definition serves as a shorthand for the definition of
several record creation and manipulation routines: a construction procedure, a
predicate, field accessors, and field mutators. As the name suggests, the
explicit-naming syntactic layer requires the programmer to name each of these
procedures explicitly.

The implicit-naming syntactic layer extends the explicit-naming syntactic layer
by allowing the names for the construction procedure, predicate, accessors, and
mutators to be determined automatically from the name of the record and names
of the fields. This establishes a standard naming convention and allows
record-type definitions to be more succinct, with the downside that the
procedure
definitions cannot easily be located via a simple search for the 
procedure name.
The programmer may override some or all of the default names by specifying them
explicitly, as in the explicit-naming syntactic layer.

The two syntactic layers are designed to be fully compatible; the
implicit-naming layer is simply a conservative extension of the
explicit-naming layer.  The design makes both explicit-naming and
implicit-naming definitions reasonably natural while allowing a seamless
transition between explicit and implicit naming.

Each of these layers permits record types to be extended via single
inheritance, allowing record types to model hierarchies that occur in
applications like algebraic data types as well as single-inheritance class
systems.

Each of the layers also supports generative and nongenerative record types.

The inspection procedures allow programs to obtain from a record instance a
descriptor for the type and from there obtain access to the fields of the
record instance. This allows the creation of portable printers and inspectors.
A program may prevent access to a record's type and thereby protect the
information stored in the record from the inspection mechanism by declaring the
type opaque. Thus, opacity as presented here can be used to enforce abstraction
barriers.

This section uses the \var{rtd} and \var{constructor-descriptor}
parameter names for arguments that must be record-type descriptors
and constructor descriptors, respectively (see
section~\ref{recordsproceduralsection}).

\section{Procedural layer}
\label{recordsproceduralsection}

The procedural layer is provided by the \deflibrary{r6rs records
  procedural} library.

\begin{entry}{%
\pproto{(make-record-type-descriptor \var{name}}{procedure}}
\mainschindex{make-record-type-descriptor}{\tt\obeyspaces\\
        \var{parent} \var{uid} \var{sealed?} \var{opaque?} \var{fields})}
   
Returns a \defining{record-type descriptor}, or \defining{rtd},
representing a record type distinct from all built-in types and
other record types.

The \var{name} argument must be a symbol naming the record type; it is
intended purely for informational purposes and may be used for printing by
the underlying Scheme system.

The \var{parent} argument must be either \schfalse{} or an rtd. If it is an
rtd, the returned record type, \var{t}, extends the record type
\var{p} represented by \var{parent}. Each record of type \var{t} is also a
record of type \var{p}, and all operations applicable to a record of
type \var{p} are also applicable to a record of type \var{t}, except for
inspection operations if \var{t} is opaque but \var{p} is not. An exception with
condition type {\cf\&contract} is raised if \var{parent} is sealed (see below).
   
The extension relationship is transitive in the sense that a type extends
its parent's parent, if any, and so on.
   
The \var{uid} argument must be either \schfalse{} or a symbol.
If \var{uid} is a symbol, the record-creation operation is
\emph{nongenerative} i.e., a new record type is created only
if no previous call to {\cf make-record-type-descriptor}
was made with the \var{uid}.
If \var{uid} is \schfalse{}, the record-creation operation is
\emph{generative}, i.e., a new record type is created even if
a previous call to {\cf make-record-type-descriptor} was
made with the same arguments.

If {\cf make-record-type-descriptor} is
called twice with the same \var{uid} symbol, the parent
arguments in the two calls must be {\cf eqv?}, the \var{fields}
arguments {\cf equal?}, the \var{sealed?} arguments boolean-equivalent
(both false or both non-false), and the \var{opaque?} arguments
boolean-equivalent.
If these conditions are not met, an exception with condition type
{\cf\&contract} is raised when the second call occurs.
If they are met, the second call returns, without creating a new
record type, the same record-type descriptor
(in the sense of {\cf eqv?}) as the first call.

\begin{note}   
  Users are encouraged to use symbol names
  constructed using the UUID namespace (for example, using the
  record-type name as a prefix) for the uid argument.
\end{note}

The \var{sealed?} flag must be a boolean. If true, the returned record type
is sealed, i.e., it cannot be extended.

The \var{opaque?} flag must be a boolean. If true, the record type
is opaque.
If passed an instance of the record type,
{\cf record?} returns
\schfalse{} and {\cf record-rtd} (see ``Inspection'' below) raises
an exception with condition type {\cf\&contract}.
The record type is also opaque if an opaque parent is
supplied.  If \var{opaque?} is false and an opaque parent is not
supplied, the record is not opaque.

The \var{fields} argument must be a list of field specifiers. Each
field specifier must be a list of the form {\cf (mutable \var{name})}
or a list of the form {\cf (immutable \var{name})}.
Each name must be a symbol and names the corresponding field of the record
type; the names need not be distinct.  A field identified as mutable may
be modified, whereas an attempt to obtain a mutator for a field identified
as immutable raises an exception with condition type {\cf\&contract}.
Where field order is relevant, e.g., for record construction and field
access, the fields are considered to be ordered as specified, although
no particular order is required for the actual representation of a
record instance.

The specified fields are added to the parent fields, if any, to determine
the complete set of fields of the returned record type.

A record type is considered immutable if each of its complete set of
fields is immutable, and is mutable otherwise.

A generative record-type descriptor created by a call to {\cf
  make-record-type-descriptor} is not {\cf eqv?} to any record-type
descriptor (generative or nongenerative) created by another call to
{\cf make-record-type-descriptor}. A generative record-type descriptor
is {\cf eqv?}  only to itself, i.e., {\tt (eqv?~\vri{rtd} \vrii{rtd})} iff
{\tt (eq?~\vri{rtd} \vrii{rtd})}.
Also, two nongenerative record-type descriptors are {\cf eqv?} iff they were
created by calls to {\cf make-record-type-descriptor} with the same
uid arguments.

\begin{rationale}
  The record and field names passed to
  {\cf make-record-type-descriptor} and appearing in the explicit-naming
  syntactic layer are for informational purposes only, e.g., for
  printers and debuggers.
  In particular, the accessor and mutator creation routines do not use
  names, but rather field indices, to identify fields.
  
  Thus, field names are not required to be distinct in the procedural or
  implicit-naming syntactic layers.
  This relieves macros and other code generators from the need to
  generate distinct names.

  The record and field names are used in the implicit-naming syntactic
  layer for the generation of accessor and mutator names, and duplicate
  field names may lead to accessor and mutator naming conflicts.
\end{rationale}

\begin{rationale}
  Sealing a record type can help to enforce abstraction barriers by preventing
  extensions that may expose implementation details of the parent type.
  Type extensions also make monomorphic code polymorphic and
  difficult to change the parent class at a later time, and also
  prevent effective predictions of types by a compiler or human
  reader.
\end{rationale}

\begin{rationale}
  Multiple inheritance was considered but omitted from the records
  facility, as it raises a number of semantic issues such as
  sharing among common parent types.
\end{rationale}
\end{entry}

\begin{entry}{%
\proto{record-type-descriptor?}{ obj}{procedure}}
   
Returns \schtrue{} if the argument is a record-type descriptor,
\schfalse{} otherwise.
\end{entry}

\begin{entry}{%
\pproto{(make-record-constructor-descriptor \var{rtd}}{procedure}}
\mainschindex{make-record-constructor-descriptor}{\tt\obeyspaces\\
        \var{parent-constructor-descriptor} \var{protocol})}

Returns a \defining{record-constructor descriptor} (or
\defining{constructor descriptor} for short) that can be used to
create record constructors (via {\cf record-constructor}; see below)
or other constructor descriptors.  \var{rtd} must be a record-type
descriptor.  \defining{protocol} must be a procedure or \schfalse.
If it is \schfalse, a default \var{protocol} procedure is supplied.
If \var{protocol} is a procedure, it is called by {\cf record-constructor}
with a single argument \var{p} and must return a procedure that creates
and returns an instance of the record type using \var{p} as described
below.

If \var{rtd} is not an extension of another record type, then
\var{parent-constructor-descriptor} must be \schfalse.
In this case, \var{protocol}'s argument \var{p} is a procedure \var{new}
that expects one parameter for every field of \var{rtd} and returns a
record instance with the fields of \var{rtd} initialized to its arguments.
The procedure returned by \var{protocol} may take any number of arguments
but must call \var{new} with the number of arguments it expects and return
the resulting record instance, as shown in the simple example below.

\begin{scheme}
(lambda (\var{new})
  (lambda (v1 \ldots)
    (\var{new} v1 \ldots)))
\end{scheme}

Here, the call to \var{new} returns a record whose fields
are simply initialized with the arguments {\tt v1 \ldots}.
The expression above is equivalent to
{\cf (lambda (\var{new}) \var{new})}.

If \var{rtd} is an extension of another record type \var{parent-rtd},
\var{parent-constructor-descriptor} must be a constructor descriptor
of \var{parent-rtd} or \schfalse.
If \var{parent-constructor-descriptor} is \schfalse, a default
constructor descriptor is supplied.
In this case, \var{p} is a procedure that accepts the same number
of arguments as the constructor of \var{parent-constructor-descriptor}
and returns a procedure \var{new}, which, as above,
expects one parameter for every field of \var{rtd} (not including parent
fields) and returns a record instance with the fields of \var{rtd}
initialized to its arguments and the fields of \var{parent-rtd} and
its parents initialized by the constructor of
\var{parent-constructor-descriptor}.
A simple \var{protocol} in this case might be written as follows.

\begin{scheme}
(lambda (\var{p})
  (lambda (x1 \ldots v1 \ldots)
    (let ((\var{new} (\var{p} x \ldots)))
      (\var{new} v1 \ldots))))
\end{scheme}

This passes some number of arguments {\tt x1 \ldots} to \var{p} for the
constructor of \var{parent-constructor-descriptor} and calls \var{new}
with {\tt v1 \ldots} to initialize the child fields.

The constructor descriptors for a record type form a chain of
protocols exactly parallel to the chain of record-type parents. Each
constructor descriptor in the chain determines the field values for the
associated record type.
Child record constructors need not know the number or contents of parent
fields, only the number of arguments required by the parent constructor.

\var{protocol} may be \schfalse, specifying a default, only
if \var{rtd} is not an extension of another record
type, or, if it is, if the parent constructor-descriptor
encapsulates a default protocol. In the first case, the
default \var{protocol} procedure is equivalent to the following:

\begin{scheme}
(lambda (p)
  (lambda field-values
    (apply p field-values)))
\end{scheme}

or, simply, {\cf (lambda (\var{p}) \var{p})}.

In the second case, the default \var{protocol} procedure returns a
constructor that accepts one argument for each of the record type's
complete set of 
fields (including those of the parent record type, the parent's parent 
record type, etc.) and returns a record with the fields initialized to
those arguments, with the field values for the parent coming before
those of the extension in the argument list.

Even if \var{rtd} extends another record type,
\var{parent-constructor-descriptor} may also be \schfalse, in which case a
constructor with default protocol is supplied.

\begin{rationale}
  The constructor-descriptor mechanism is an infra\-struc\-ture for
  creating specialized constructors, rather than just creating default
  constructors that accept the initial values of all the fields as
  arguments. This infrastructure achieves full generality while
  leaving each level of an inheritance hierarchy in control over its
  own fields and allowing child record definitions to be abstracted
  away from the actual number and contents of parent fields.

  The design allows the initial values of the fields to be specially
  computed or to default to constant values. It also allows for
  operations to be performed on or with the resulting record, such as
  the registration of a widget record for finalization. Moreover, the
  constructor-descriptor mechanism allows the creation of such
  initializers in a modular manner, separating the initialization
  concerns of the parent types from those of the extensions.
  
  The mechanism described here achieves complete generality without
  cluttering the syntactic layer, sacrificing a bit of
  notational convenience in special cases.
\end{rationale}

\end{entry}

\begin{entry}{%
\proto{record-constructor}{ constructor-descriptor}{procedure}}
   
Calls the \var{protocol} of \var{constructor-descriptor} (as described for
{\cf make-record-constructor-descriptor}) and returns the resulting
construction procedure \var{constructor} for instances of the record type
associated with \var{constructor-descriptor}.

Two values created by \var{constructor} are equal according to {\cf
  equal?} iff they are {\cf eqv?}, provided their record type is not
used to implement any of the types explicitly mentioned in the
definition of {\cf equal?}.

For any \var{constructor} returned by {\cf record-constructor},
the following holds:

\begin{scheme}
(let ((r (\var{constructor} v \ldots)))
  (eqv? r r))                \ev \schtrue
\end{scheme}

For mutable records, but not necessarily for immutable ones, the following
hold.
(A record of an mutable record type is mutable;
a record of an immutable record type is immutable.)

\begin{scheme}
(let ((r (\var{constructor} v \ldots)))
  (eq? r r))                 \ev \schtrue

(let ((f (lambda () (\var{constructor} v \ldots))))
  (eq? (f) (f)))             \ev \schfalse
\end{scheme}
\end{entry}

\begin{entry}{%
\proto{record-predicate}{ rtd}{procedure}}
   
Returns a procedure that, given an object \var{obj}, returns
a boolean that is \schtrue{}
iff \var{obj} is a record of the type represented by
\var{rtd}.
\end{entry}

\begin{entry}{%
\proto{record-accessor}{ rtd k}{procedure}}

\domain{\var{K} must be a valid field index  of \var{rtd}.}
The {\cf record-accessor} procedure returns a one-argument procedure that, given a
record of the type represented by \var{rtd}, returns the value of the
selected field of that record.

The field selected is the one corresponding the the \var{k}th element
(0-based) of the \var{fields} argument to the invocation of {\cf
  make-record-type-descriptor} that created \var{rtd}. Note that
\var{k} cannot be used to specify a field of any type \var{rtd} extends.

If the accessor procedure is given something other than
a record of the type represented by \var{rtd}, an exception with
condition type {\cf\&contract} is raised.  Records
of the type represented by \var{rtd} include records of extensions of
the type represented by \var{rtd}.
\end{entry}

\begin{entry}{%
\proto{record-mutator}{ rtd k}{procedure}}
   
\domain{\var{K} must be a valid field index  of \var{rtd}.}
The {\cf record-mutator} procedure returns a two-argument procedure that, given a
record \var{r} of the type represented by \var{rtd} and an object
\var{obj}, stores \var{obj} within the field of \var{r} specified by
\var{k}. The \var{k} argument is as in {\cf record-accessor}. If
\var{k} specifies an immutable field, an exception with condition type
{\cf\&contract} is raised.
The mutator returns the unspecified value.
\end{entry}

\begin{scheme}
(define :point
  (make-record-type-descriptor
    'point \schfalse{}
    \schfalse{} \schfalse{} \schfalse{} 
    '((mutable x) (mutable y))))

(define make-point
  (record-constructor
    (make-record-constructor-descriptor :point
      \schfalse{} \schfalse{})))

(define point? (record-predicate :point))
(define point-x (record-accessor :point 0))
(define point-y (record-accessor :point 1))
(define point-x-set! (record-mutator :point 0))
(define point-y-set! (record-mutator :point 1))

(define p1 (make-point 1 2))
(point? p1) \ev \schtrue{}
(point-x p1) \ev 1
(point-y p1) \ev 2
(point-x-set! p1 5) \ev \theunspecified
(point-x p1) \ev 5

(define :point2
  (make-record-type-descriptor
    'point2 :point 
    \schfalse{} \schfalse{} \schfalse{} '((mutable x) (mutable y))))

(define make-point2
  (record-constructor
    (make-record-constructor-descriptor :point2
      \schfalse{} \schfalse{})))
(define point2? (record-predicate :point2))
(define point2-xx (record-accessor :point2 0))
(define point2-yy (record-accessor :point2 1))

(define p2 (make-point2 1 2 3 4))
(point? p2) \ev \schtrue{}
(point-x p2) \ev 1
(point-y p2) \ev 2
(point2-xx p2) \ev 3
(point2-yy p2) \ev 4
\end{scheme}

\section{Explicit-naming syntactic layer}
\label{recordsexplicitnamingsection}

The explicit-naming syntactic layer is provided by the
\deflibrary{r6rs records explicit} library.

The record-type-defining form {\cf define-record-type} is a definition and
can appear anywhere any other \hyper{definition} can appear.

\begin{entry}{%
\proto{define-record-type}{ \hyper{name spec} \arbno{\hyper{record clause}}}{\exprtype}}

A {\cf define-record-type} form defines a record type along with
associated constructor descriptor and constructor, predicate, field
accessors, and field mutators. The {\cf define-record-type} form expands into
a set of definitions in the environment where {\cf define-record-type}
appears; hence, it is possible to refer to the bindings (except for
that of the record type itself) recursively.

The \hyper{name spec} specifies the names of the record type,
construction procedure, and predicate. It must take the following
form.

\begin{scheme}
(\hyper{record name} \hyper{constructor name} \hyper{predicate name})
\end{scheme}

\hyper{Record name}, \hyper{constructor name}, and \hyper{predicate
  name} must all be identifiers.

\hyper{Record name}, taken as a symbol, becomes the name of the record
type.  Additionally, it is bound by this definition to an expand-time
or run-time description of the record type for use as parent name in
syntactic record-type definitions that extend this definition. It may
also be used as a handle to gain access to the underlying record-type
descriptor and constructor descriptor (see {\cf
  record-type-descriptor} and {\cf record-constructor-descriptor}
below).

\hyper{Constructor name} is defined by this definition to be a
constructor for the defined record type, with a protocol specified by
the {\cf protocol} clause, or, in its absence, using a default protocol. For
details, see the description of the {\cf protocol} clause below.

\hyper{Predicate name} is defined by this definition to a predicate
for the defined record type.

Each \hyper{record clause} must take one of the following forms; it is
a syntax violation if multiple \hyper{record clause}s of the same kind appear in a
{\cf define-record-type} form.

\begin{itemize}
\item {\tt (fields \arbno{\hyper{field-spec}})}
   
  where each \hyper{field-spec} has one of the following forms
  
\begin{scheme}
(immutable \hyper{field name} \hyper{accessor name})
(mutable \hyper{field name}
         \hyper{accessor name} \hyper{mutator name})
\end{scheme}

  \hyper{Field name}, \hyper{accessor name}, and \hyper{mutator name}
  must all be identifiers. The first form declares an immutable field
  called \hyper{field name}, with the corresponding accessor named \hyper{accessor
  name}. The second form declares a mutable field called \hyper{field name},
  with the corresponding accessor named \hyper{accessor name}, and with the
  corresponding mutator named \hyper{mutator name}.
   
  The \hyper{field name}s become, as symbols, the names of the fields of the
  record type being created, in the same order. They are not used in any
  other way.
   
\item {\tt (parent \hyper{parent name})}
   
  This specifies that the record type is to have parent type
  \hyper{parent name}, where \hyper{parent name} is the \hyper{record
      name} of a record type previously defined using {\cf
      define-record-type}. The absence of a {\cf parent} clause implies a
    record type with no parent type.
   
\item {\tt (protocol \hyper{expression})}
   
  \hyper{Expression} is evaluated in the same environment as the
  define-record-type form, and must evaluate to a protocol appropriate
  for the record type being defined (see the description of
  {\cf make-record-constructor-descriptor}). The protocol is used to
  create a record-constructor descriptor where, if the record type
  being defined has a parent, the parent-type constructor descriptor
  is the one associated with the parent type specified in the {\cf
    parent} clause.
   
  If no {\cf protocol} clause is specified, a constructor descriptor
  is still created using a default protocol. The rules for this are
  the same as for {\cf make-record-constructor-descriptor}: the clause
  can be absent only if the record type defined has no parent type, or
  if the parent definition does not specify a protocol.
   
\item {\tt (sealed \schtrue)}\\
  {\tt (sealed \schfalse)}
   
  If this option is specified with operand \schtrue,
  the defined record type is sealed.
  Otherwise, the defined record type is not sealed.
   
\item {\tt (opaque \schtrue)}\\
  {\tt (opaque \schfalse)}
   
  If this option is specified with operand \schtrue, or if an opaque
  parent record type is specified, the defined record type is opaque.
  Otherwise, the defined record type is not opaque.
   
\item {\tt (nongenerative \hyper{uid})}
   
  This specifies that the record type is nongenerative with uid
  \hyper{uid}, which must be an \hyper{identifier}.
  If two record-type definitions specify the same \var{uid}, then
  the implied arguments to {\cf make-record-type-descriptor}
  must be equivalent as described under {\cf make-record-type-descriptor}.
  If this condition is not met, it is either considered a syntax violation or
  an exception with condition type {\cf\&contract} is raised.
  If the condition is met, a single record type is generated for both
  definitions.

  In the absence of a {\cf nongenerative} clause, a new record type is
  generated every time a {\cf define-record-type} form is evaluated:

\begin{scheme}
(let ((f (lambda (x)
           (define-record-type r \ldots)
           (if x r? (make-r \ldots)))))
  ((f \schtrue) (f \schfalse))) \ev \schfalse{}
\end{scheme}
\end{itemize}

All bindings created by {\cf define-record-type} (for the record type,
the construction procedure, the predicate, the accessors, and the
mutators) must have names that are pairwise distinct.
\end{entry}

\begin{entry}{%
\proto{record-type-descriptor}{ \hyper{record name}}{\exprtype}}
   
Evaluates to the record-type descriptor associated with the type
specified by \hyper{record-name}.
   
Note that {\cf record-type-descriptor} works on both opaque and non-opaque record
types.
\end{entry}

\begin{entry}{%
\proto{record-constructor-descriptor}{ \hyper{record name}}{\exprtype}}
   
Evaluates to the record-constructor descriptor associated with
\hyper{record name}.
\end{entry}

Explicit-naming syntactic-layer examples:

\begin{scheme}
(define-record-type (point3 make-point3 point3?)
  (fields (immutable x point3-x)
          (mutable y point3-y set-point3-y!))
  (nongenerative
    point3-4893d957-e00b-11d9-817f-00111175eb9e))

(define-record-type (cpoint make-cpoint cpoint?)
  (parent point3)
  (protocol
   (lambda (p)
     (lambda (x y c) 
       ((p x y) (color->rgb c)))))
  (fields
    (mutable rgb cpoint-rgb cpoint-rgb-set!)))

(define (color->rgb c)
  (cons 'rgb c))

(define p3-1 (make-point3 1 2))
(define p3-2 (make-cpoint 3 4 'red))

(point3? p3-1) \ev \schtrue{}
(point3? p3-2) \ev \schtrue{}
(point3? (vector)) \ev \schfalse{}
(point3? (cons 'a 'b)) \ev \schfalse{}
(cpoint? p3-1) \ev \schfalse{}
(cpoint? p3-2) \ev \schtrue{}
(point3-x p3-1) \ev 1
(point3-y p3-1) \ev 2
(point3-x p3-2) \ev 3
(point3-y p3-2) \ev 4
(cpoint-rgb p3-2) \ev '(rgb . red)

(set-point3-y! p3-1 17)
(point3-y p3-1) \ev 17)

(record-rtd p3-1) \lev (record-type-descriptor point3)

(define-record-type (ex1 make-ex1 ex1?)
  (protocol (lambda (new) (lambda a (new a))))
  (fields (immutable f ex1-f)))

(define ex1-i1 (make-ex1 1 2 3))
(ex1-f ex1-i1) \ev '(1 2 3)

(define-record-type (ex2 make-ex2 ex2?)
  (protocol
    (lambda (new) (lambda (a . b) (new a b))))
  (fields (immutable a ex2-a)
          (immutable b ex2-b)))

(define ex2-i1 (make-ex2 1 2 3))
(ex2-a ex2-i1) \ev 1
(ex2-b ex2-i1) \ev '(2 3)

(define-record-type (unit-vector
                     make-unit-vector
                     unit-vector?)
  (protocol
   (lambda (new)
     (lambda (x y z)
       (let ((length (sqrt (+ (* x x) (* y y) (* z z)))))
         (new  (/ x length)
               (/ y length)
               (/ z length))))))
  (fields (immutable x unit-vector-x)
          (immutable y unit-vector-y)
          (immutable z unit-vector-z)))
\end{scheme}

\section{Implicit-naming syntactic layer}

The implicit-naming syntactic layer is provided by the
\deflibrary{r6rs records implicit} library.

The {\cf define-record-type} form of the implicit-naming syntactic
layer is a conservative extension of the {\cf define-record-type} form
of the explicit-naming layer: a {\cf define-record-type} form that
conforms to the syntax of the explicit-naming layer also conforms to
the syntax of the implicit-naming layer, and any definition in the
implicit-naming layer can be understood by its translation into the
explicit-naming layer.

This means that a record type defined by the {\cf define-record-type}
form of either layer can be used by the other.

The implicit-naming syntactic layer extends the explicit-naming layer
in two ways. First, \hyper{name-spec} may be a single identifier
representing just the record name. In this case, the name of the
construction procedure is generated by prefixing the record name with
{\tt make-}, and the predicate name is generated by adding a question
mark ({\tt ?}) to the end of the record name. For example, if the
record name is {\tt frob}, the name of the construction procedure is
{\tt make-frob}, and the predicate name is {\tt frob?}.

Second, the syntax of \hyper{field-spec} is extended to allow the
accessor and mutator names to be omitted. That is, \hyper{field-spec}
can take one of the following forms as well as the forms described in
the preceding section.

\begin{scheme}
(immutable \hyper{field name})
(mutable \hyper{field name})%
\end{scheme}

If \hyper{field-spec} takes one of these forms, the accessor name
is generated by appending the record name and field name with a hyphen
separator, and the mutator name (for a mutable field) is generated by
adding a {\tt -set!} suffix to the accessor name. For example, if the
record name is {\tt frob} and the field name is {\tt widget}, the
accessor name is {\tt frob-widget} and the mutator name is
{\tt frob-widget-set!}.

Any definition that takes advantage of implicit naming can be
rewritten trivially to a definition that conforms to the syntax of the
explicit-naming layer merely by specifying the names explicitly. For
example, the implicit-naming layer record definition:

\begin{scheme}
(define-record-type frob
  (fields (mutable widget))
  (protocol
    (lambda (c) (c (make-widget n)))))
\end{scheme}

is equivalent to the following explicit-naming layer record definition.

\begin{scheme}
(define-record-type (frob make-frob frob?)
  (fields (mutable widget
                   frob-widget frob-widget-set!))
  (protocol
    (lambda (c) (c (make-widget n)))))
\end{scheme}

With the implicit-naming layer, one can choose to specify just some of
the names explicitly; for example, the following overrides the choice
of accessor and mutator names for the widget field.

\begin{scheme}
(define-record-type frob
  (fields (mutable widget getwid setwid!))
  (protocol
    (lambda (c) (c (make-widget n)))))
\end{scheme}

\begin{scheme}
(define *ex3-instance* \schfalse{})

(define-record-type ex3
  (parent cpoint)
  (protocol
   (lambda (p)
     (lambda (x y t)
       (let ((r ((p x y 'red) t)))
         (set! *ex3-instance* r)
         r))))
  (fields 
   (mutable thickness))
  (sealed \schtrue{}) (opaque \schtrue{}))

(define ex3-i1 (make-ex3 1 2 17))
(ex3? ex3-i1) \ev \schtrue{}
(cpoint-rgb ex3-i1) \ev '(rgb . red)
(ex3-thickness ex3-i1) \ev 17
(ex3-thickness-set! ex3-i1 18)
(ex3-thickness ex3-i1) \ev 18
*ex3-instance* \ev ex3-i1

(record? ex3-i1) \ev \schfalse{}
\end{scheme}

\begin{entry}{%
\rproto{record-type-descriptor}{ \hyper{record name}}{\exprtype}}

This is the same as {\cf record-type-descriptor} from the
\library{r6rs records explicit} library.
\end{entry}

\begin{entry}{%
\rproto{record-constructor-descriptor}{ \hyper{record name}}{\exprtype}}
   
This is the same as {\cf record-constructor-descriptor} from the
\library{r6rs records explicit} library.
\end{entry}

\section{Inspection}

The implicit-naming syntactic layer is provided by the
\deflibrary{r6rs records inspection} library.

A set of procedures are provided for inspecting records and their
record-type descriptors. These procedures are designed to allow the
writing of portable printers and inspectors.

Note that {\cf record?} and {\cf record-rtd} treat records of opaque
record types as if they were not records. On the other hand, the
inspection procedures that operate on record-type descriptors
themselves are not affected by opacity. In other words, opacity
controls whether a program can obtain an rtd from an instance. If the
program has access to the original rtd via {\cf
  make-record-type-descriptor} or {\cf record-type-descriptor} it can
still make use of the inspection procedures.

Any of the standard types mentioned in this report may or may not be
implemented as a non-opaque record type.  Consequently, {\cf record?},
when applied to an object of one of these types, may return
\schtrue{}.  In this case, inspection is possible for these objects.

\begin{entry}{%
\proto{record?}{ obj}{procedure}}
   
Returns \schtrue{} if \var{obj} is a record, and its record type is
not opaque. Returns \schfalse{} otherwise.  
\end{entry}

\begin{entry}{%
\proto{record-rtd}{ record}{procedure}}
   
Returns the rtd representing the type of \var{record} if the type is not
opaque. The rtd of the most precise type is returned; that is, the
type \var{t} such that \var{record} is of type \var{t} but not of any
type that extends \var{t}.  If the type is opaque, an exception is
raised with condition type {\cf\&contract}.
\end{entry}

\begin{entry}{%
\proto{record-type-name}{ rtd}{procedure}}
   
Returns the name of the record-type descriptor \var{rtd}.
\end{entry}   

\begin{entry}{%
\proto{record-type-parent}{ rtd}{procedure}}
   
Returns the parent of the record-type descriptor \var{rtd}, or
\schfalse{} if it has none.
\end{entry}

\begin{entry}{%
\proto{record-type-uid}{ rtd}{procedure}}
   
Returns the uid of the record-type descriptor rtd, or \schfalse{} if it has none.
(An implementation may assign a generated uid to a record type even if the
type is generative, so the return of a uid does not necessarily imply that
the type is nongenerative.)
\end{entry}

\begin{entry}{%
\proto{record-type-generative?}{ rtd}{procedure}}
   
Returns \schtrue{} if \var{rtd} is generative, and \schfalse{} if not.
\end{entry}

\begin{entry}{%
\proto{record-type-sealed?}{ rtd}{procedure}}

Returns a boolean value indicating whether the record-type descriptor is
sealed.
\end{entry}

\begin{entry}{%
\proto{record-type-opaque?}{ rtd}{procedure}}
   
Returns a boolean value indicating whether the record-type descriptor is
opaque.
\end{entry}

\begin{entry}{%
\proto{record-type-field-names}{ rtd}{procedure}}
   
Returns a list of symbols naming the fields of the type represented by rtd
(not including the fields of parent types) where the fields are ordered as
described under {\cf make-record-type-descriptor}.
\end{entry}

\begin{entry}{%
\proto{record-field-mutable?}{ rtd k}{procedure}}
   
Returns a boolean value indicating whether the field specified by
\var{k} of the type represented by \var{rtd} is mutable, where \var{k}
is as in {\cf record-accessor}.
\end{entry}

%%% Local Variables: 
%%% mode: latex
%%% TeX-master: "r6rs"
%%% End: 
 \par
\chapter{Exceptions and conditions}
\label{exceptionsconditionschapter}

Scheme allows programs to deal with exceptional situations using two
cooperating facilities: The exception system for raising and handling
exceptional situations, and the condition system for describing these
situations.

The exception system allows the program, when it detects an
exceptional situation, to pass control to an exception handler, and
for dynamically establishing such exception handlers.  Exception
handlers are always invoked with an object describing the exceptional
situation.  Scheme's condition system provides a standardized taxonomy
of such descriptive objects, as well as a facility for extending the
taxonomy.

\section{Exceptions}
\label{exceptionssection}
\mainindex{exceptions}

This section describes Scheme's exception-handling and
exception-raising constructs provided by the \deflibrary{r6rs
  exceptions} library.

\begin{note}
  This specification follows SRFI~34~\cite{srfi34}.
\end{note}

Exception handlers are one-argument procedures that determine the
action the program takes when an exceptional situation is signalled.
The system implicitly maintains a current exception handler.

\mainindex{current exception handler}The program raises an exception
by invoking the current exception handler, passing to it an object
encapsulating information about the exception. Any procedure accepting
one argument may serve as an exception handler and any object may be
used to represent an exception.

The system maintains the current exception handler as part of the
\defining{dynamic environment} of the program, the context for {\tt
  dynamic-wind}. The dynamic environment can be thought of as that
part of a continuation that does not specify the destination of any
returned values. It includes the {\tt dynamic-wind} context and the
current exception handler.

When a program begins its execution, the current
exception handler is expected to handle all {\cf\&serious}
conditions by interrupting execution, reporting that an
exception has been raised, and displaying information
about the condition object that was provided.  The handler
may then exit, or may provide a choice of other options.
Moreover, the exception handler is expected to return when
passed any other non-{\cf\&serious} condition.
Interpretation of these expectations necessarily depends
upon the nature of the system in which programs are executed,
but the intent is that users perceive the raising of an
exception as a controlled escape from the situation that
raised the exception, not as a crash.

\begin{entry}{%
\proto{with-exception-handler}{ \var{handler} \var{thunk}}{procedure}}

\domain{\var{Handler} must be a
procedure that accepts one argument.}  The {\cf
with-exception-handler} procedure returns the result(s) of invoking
\var{thunk}.  \var{Handler} is installed as the current
exception handler for the dynamic extent (as determined by {\tt
  dynamic-wind}) of the invocation of \var{thunk}.

\implresp The implementation must check the restrictions on
\var{handler} to the extent performed by applying it as described,
when it is called as a result of a call to {\cf raise} or {\cf
  raise-continuable}.
\end{entry}

\begin{entry}{%
\pproto{(guard (\hyper{variable} \hyperi{clause} \hyperii{clause} \dotsfoo)  \hyper{body})}{\exprtype}}
\mainschindex{guard}

\syntax
Each \hyper{clause} must have the same form as a {\tt cond} clause.
(See report section~\extref{report:cond}{Derived conditionals}.)

\semantics 
Evaluating a {\tt guard} form evaluates \hyper{body} with an exception
handler that binds the raised object to \hyper{variable} and within the scope of
that binding evaluates the clauses as if they were the clauses of a
{\tt cond} expression. That implicit {\tt cond} expression is evaluated with the
continuation and dynamic environment of the {\tt guard} expression. If every
\hyper{clause}'s \hyper{test} evaluates to false and there is no else clause, then
{\cf raise} is re-invoked on the raised object within the dynamic
environment of the original call to raise except that the current
exception handler is that of the {\tt guard} expression.  
\end{entry}

\begin{entry}{%
\proto{raise}{ \var{obj}}{procedure}}

Raises a non-continuable exception by invoking the current exception
handler on \var{obj}. The handler is called with a continuation whose
dynamic environment is that of the call to {\tt raise}, except that
the current exception handler is the one that was in place when the handler being
called was installed.  The continuation of the handler raises a non-continuable
exception with condition type {\tt \&non-continuable}.
\end{entry}

\begin{entry}{%
\proto{raise-continuable}{ \var{obj}}{procedure}}

Raises a continuable exception by invoking the current exception
handler on \var{obj}. The handler is called with a continuation that
is equivalent to the continuation of the call to {\tt
  raise-continuable} with these two exceptions: (1) the current
exception handler is the one that was in place 
when the handler being called was installed, and
(2) if the handler being called returns, then it will again become the
current exception handler.  If the handler returns, the values it
returns become the values returned by the call to
{\tt raise-continuable}.
\end{entry}

\begin{scheme}
(guard (con
         ((error? con)
          (if (message-condition? con)
              (display (condition-message con))
              (display "an error has occurred")
              'error))
         ((violation? con)
          (if (message-condition? con)
              (display (condition-message con))
              (display "the program has a bug"))
          'violation))
  (raise
    (condition
      (\&error)
      (\&message (message "I am an error")))))
   {\it prints:} I am an error
   \ev error%

(guard (con
         ((error? con)
          (if (message-condition? con)
              (display (condition-message con))
              (display "an error has occurred")
              'error)))
  (raise
    (condition
      (\&violation)
      (\&message (message "I am an error")))))
  \ev \exception{\&violation}

(guard (con
         ((error? con)
          (display "error opening file")
          \schfalse))
  (call-with-input-file "foo.scm" read))
   {\it prints:} error opening file
   \ev \schfalse{}

(with-exception-handler
  (lambda (con)
    (cond
      ((not (warning? con))
       (raise con))
      ((message-condition? con)
       (display (condition-message con)))
      (else
       (display "a warning has been issued")))
    42)
  (lambda ()
    (+ (raise-continuable
         (condition
           (\&warning)
           (\&message
             (message "should be a number"))))
       23)))
   {\it prints:} should be a number
   \ev 65
\end{scheme}

\section{Conditions}
\label{conditionssection}

The section describes Scheme \deflibrary{r6rs
  conditions} library for creating and inspecting
condition types and values. A condition value encapsulates information
about an exceptional situation\mainindex{exceptional situation}, or
\defining{exception}. Scheme also defines a
number of basic condition types.

\begin{note}
  This specification is similar to, but not identical with
  SRFI~35~\cite{srfi35}.
\end{note}

Scheme conditions provides two mechanisms to enable communication
about exceptional situation: subtyping among condition types allows
handling code to determine the general nature of an exception even
though it does not anticipate its exact nature, and compound
conditions allow an exceptional situation to be described in multiple
ways.

\begin{rationale}
Conditions are values that communicate information about exceptional
situations between parts of a program. Code that detects an exception
may be in a different part of the program than the code that handles
it. In fact, the former may have been written independently from the
latter. Consequently, to facilitate effective handling of exceptions,
conditions should communicate as much information as possible as
accurately as possible, and still allow effective handling by code
that did not precisely anticipate the nature of the exception that
occurred.
\end{rationale}

\subsection{Condition objects}

Conditions are objects with named fields. Each condition belongs to
one or more condition types. Each condition type specifies a set of
field names. A condition belonging to a condition type includes a
value for each of the type's field names. These values can be
extracted from the condition by using the appropriate field name.

The condition system distinguishes between \textit{simple
  conditions}\index{simple condition} and \textit{compound
  conditions}\index{compound condition}.  A compound condition
consists of an ordered set of simple conditions.  Thus, every
condition can be viewed as an ordered set of simple component
conditions: If it is simple, the set consists of the condition itself;
if it is compound, it consists of the simple conditions that compose
it.

There is a tree of condition types with the distinguished {\tt
  \&condition} as its root. All other condition types have a parent
condition type.

\begin{entry}{%
\proto{make-condition-type}{ \var{id} \var{parent} \var{field-names}}{procedure}}

Returns a new condition type. \var{Id} must
be a symbol that serves as a symbolic name for the condition type.
\var{Parent} must itself be a condition type. \var{Field-names} must
be a list of symbols. It identifies the fields of the conditions
associated with the condition type.

\var{Field-names} must be disjoint from the field names of
\var{parent} and its ancestors. 
\end{entry}

\begin{entry}{%
\proto{condition-type?}{ \var{thing}}{procedure}}

Returns \schtrue{} if \var{thing} is a condition type, and \schfalse{}
otherwise.
\end{entry}

\begin{entry}{%
\proto{make-condition}{ \var{type} \var{alist}}{procedure}}

Returns a simple condition value belonging to condition
type \var{type}. \var{Alist} must be an association list mapping
field names to arbitrary values.  There must be a pair in the
association list for each field of \var{type} and its direct and indirect
supertypes. The {\cf make-condition} procedure returns the condition
value, which fields and values as indicated by \var{alist}.
\end{entry}

\begin{entry}{%
\proto{condition?}{ \var{obj}}{procedure}}

Returns \schtrue{} if \var{obj} is a condition object, and \schfalse{}
otherwise.
\end{entry}

\begin{entry}{%
\proto{condition-has-type?}{ \var{condition} \var{condition-type}}{procedure}}

The {\cf condition-has-type?} procedure tests if condition condition belongs to
condition type \var{condition-type}. It returns \schtrue{} if any of
condition's types includes \var{condition-type}, either directly or as
an ancestor, and \schfalse{} otherwise.
\end{entry}

\begin{entry}{%
\proto{condition->list}{ \var{condition}}{procedure}}

Returns a list of the component conditions of \var{condition}.
\end{entry}

\begin{entry}{%
\proto{condition-ref}{ \var{condition} \var{type} \var{field-name}}{procedure}}

\domain{\var{Condition} must be a condition, \var{type} a condition
  type, and \var{field-name} a symbol naming a field of type or its
  direct or indirect supertypes.  Moreover, condition must be a simple
  condition of type \var{type}, or a compound condition containing a
  simple condition of type \var{type}.}  The {\cf condition-ref}
procedure returns the value of the field named by \var{field-name} in
the first component condition of \var{condition} that has type \var{type}.
\end{entry}

\begin{entry}{%
\proto{make-compound-condition}{ \vari{condition} \dotsfoo}{procedure}}

Returns a compound condition consisting of the component conditions of
\vari{condition}, \dotsfoo, in that order.
\end{entry}

\begin{entry}{%
\pproto{(define-condition-type \hyper{condition-type}}{\exprtype}}
{\tt\obeyspaces\\
    \hyper{supertype}\\
  \hyper{predicate}\\
  \hyperi{field-spec} \dotsfoo)}
\mainschindex{define-condition-type}

\syntax \hyper{Condition-type},
\hyper{supertypes}, and \hyper{predicate} must all be identifiers.
Each \hyper{field-spec} must be of the form
%
\begin{scheme}
(\hyper{field} \hyper{accessor})%
\end{scheme}
%
where both \hyper{field} and \hyper{accessor} must be identifiers.

\semantics
The {\cf define-condition-type} form expands into a set of
definitions:

\begin{itemize}
\item \hyper{Condition-type}, which is bound to some value describing a new condition type.
\hyper{Supertype} must be the name of a previously defined condition
type.

\item \hyper{Predicate} is bound to a
predicate that identifies conditions associated with that type, or
with any of its subtypes.

\item Each \hyper{accessor} is bound to a
procedure which extracts the value of the named field from a condition
associated with this condition type.  
\end{itemize}
\end{entry}

\begin{entry}{%
\proto{condition}{ \hyperi{type-field-binding} \dotsfoo}{\exprtype}}

Returns a condition value. Each \hyper{type-field-binding} must be of
the form
%
\begin{scheme}
(\hyper{condition-type} \hyperi{field-binding} \dotsfoo)%
\end{scheme}
%
Each \hyper{field-binding}
must be of the form
%
\begin{scheme}
(\hyper{field} \hyper{expression})  %
\end{scheme}
%
where \hyper{field} is a field identifier from the definition of
\hyper{condition-type}.

The condition returned by condition is created by a call of the form
%
\begin{schemenoindent}
(make-compound-condition
  (make-condition \hyper{condition-type}
                  (list (cons '\hyper{field-name} \hyper{expression})
                        \dotsfoo))
  \dotsfoo)
\end{schemenoindent}
%
with the condition types retaining their order from the condition
form.
\end{entry}

\begin{entry}{%
\ctproto{condition}}

This is the root of the entire condition type hierarchy. It has no
fields.
\end{entry}

\begin{scheme}
(define-condition-type \&c \&condition
  c?
  (x c-x))

(define-condition-type \&c1 \&c
  c1?
  (a c1-a))

(define-condition-type \&c2 \&c
  c2?
  (b c2-b))%
\end{scheme}

\begin{scheme}
(define v1
  (make-condition \&c1
    (list (cons 'x "V1")
          (cons 'a "a1"))))

(c? v1)        \ev \schtrue
(c1? v1)       \ev \schtrue
(c2? v1)       \ev \schfalse
(c-x v1)       \ev "V1"
(c1-a v1)      \ev "a1"%
\end{scheme}

\begin{scheme}
(define v2 (condition (\&c2
                        (x "V2")
                        (b "b2"))))

(c? v2)        \ev \schtrue
(c1? v2)       \ev \schfalse
(c2? v2)       \ev \schtrue
(c-x v2)       \ev "V2"
(c2-b v2)      \ev "b2"%
\end{scheme}

\begin{scheme}
(define v3 (condition (\&c1
                       (x "V3/1")
                       (a "a3"))
                      (\&c2
                       (x "V3/2")
                       (b "b3"))))

(c? v3)        \ev \schtrue
(c1? v3)       \ev \schtrue
(c2? v3)       \ev \schtrue
(c-x v3)       \ev "V3/1"
(c1-a v3)      \ev "a3"
(c2-b v3)      \ev "b3"%
\end{scheme}

\begin{scheme}
(define v4 (make-compound-condition v1 v2))

(c? v4)        \ev \schtrue
(c1? v4)       \ev \schtrue
(c2? v4)       \ev \schtrue
(c-x v4)       \ev "V1"
(c1-a v4)      \ev "a1"
(c2-b v4)      \ev "b2"%
\end{scheme}

\begin{scheme}
(define v5 (make-compound-condition v2 v3))

(c? v5)        \ev \schtrue
(c1? v5)       \ev \schtrue
(c2? v5)       \ev \schtrue
(c-x v5)       \ev "V2"
(c1-a v5)      \ev "a3"
(c2-b v5)      \ev "b2"%
\end{scheme}

\section{Standard condition types}

\begin{entry}{%
\ctproto{message}
\proto{message-condition?}{ obj}{procedure}
\proto{condition-message}{ condition}{procedure}}

This condition type could be defined by
%
\begin{scheme}
(define-condition-type \&message \&condition
  message-condition?
  (message condition-message))%
\end{scheme}
%
It carries a message further describing the nature of the condition to
humans.  
\end{entry}

\begin{entry}{%
\ctproto{warning}
\proto{warning?}{ obj}{procedure}}

This condition type could be defined by
%
\begin{scheme}
(define-condition-type \&warning \&condition
  warning?)%
\end{scheme}
%
This type describes conditions that do not, in
principle, prohibit immediate continued execution of the program, but
may interfere with the program's execution later.
\end{entry}

\begin{entry}{%
\ctproto{serious}
\proto{serious-condition?}{ obj}{procedure}}

This condition type could be defined by
%
\begin{scheme}
(define-condition-type \&serious \&condition
  serious-condition?)%
\end{scheme}

This type describes conditions serious enough that they cannot safely
be ignored. This condition type is primarily intended as a supertype
of other condition types. 
\end{entry}

\begin{entry}{%
\ctproto{error}
\proto{error?}{ obj}{procedure}}

This condition type could be defined by
%
\begin{scheme}
(define-condition-type \&error \&serious
  error?)%
\end{scheme}
%
This type describes errors, typically caused by something that
has gone wrong in the interaction of the program with the external
world or the user.
\end{entry}

\begin{entry}{%
\ctproto{violation}
\proto{violation?}{ obj}{procedure}}

This condition type could be defined by
%
\begin{scheme}
(define-condition-type \&violation \&serious
  violation?)%
\end{scheme}
%
This type describes violations of the language standard or a
library standard, typically caused by a programming error.
\end{entry}  

\begin{entry}{%
\ctproto{non-continuable}
\proto{non-continuable?}{ obj}{procedure}}

This condition type could be defined by
%
\begin{scheme}
(define-condition-type \&non-continuable \&violation
  non-continuable?)%
\end{scheme}
%
This type denotes that an exception handler invoked via
\texttt{raise} has returned.
\end{entry}

\begin{entry}{%
\ctproto{implementation-restriction}
\proto{implementation-restriction?}{ obj}{procedure}}

This condition type could be defined by
%
\begin{scheme}
(define-condition-type \&implementation-restriction
    \&violation
  implementation-restriction?)%
\end{scheme}
%
This type describes a violation of an implementation restriction
allowed by the specification, such as the absence of representations
for NaNs and infinities.  (See section~\ref{flonumssection}.)
\end{entry}

\begin{entry}{%
\ctproto{lexical}
\proto{lexical-violation?}{ obj}{procedure}}

This condition type could be defined by
%
\begin{scheme}
(define-condition-type \&lexical \&violation
  lexical-violation?)%
\end{scheme}
%
This type describes syntax violations at the level of the read syntax.
\end{entry}

\begin{entry}{%
\ctproto{syntax}
\proto{syntax-violation?}{ obj}{procedure}}

This condition type could be defined by
%
\begin{scheme}
(define-condition-type \&syntax \&violation
  syntax-violation?
  (form syntax-violation-form)
  (subform syntax-violation-subform))%
\end{scheme}

This type describes syntax violations.
The {\cf form} field contains the erroneous syntax object or a
datum representing the code of the erroneous form.  The {\cf
  subform} field may contain an optional syntax object or
datum within the erroneous form that more precisely locates the
violation.  It can be \schfalse{} to indicate the absence of more precise
information.
\end{entry}

\begin{entry}{%
\ctproto{undefined}
\proto{undefined-violation?}{ obj}{procedure}}

This condition type could be defined by
%
\begin{scheme}
(define-condition-type \&undefined \&violation
  undefined-violation?)%
\end{scheme}
% 
This type describes unbound identifiers in the program.
\end{entry}

\begin{entry}{%
\ctproto{assertion}
\proto{assertion-violation?}{ obj}{procedure}}

This condition type could be defined by
%
\begin{scheme}
(define-condition-type \&assertion \&violation
  assertion-violation?)%
\end{scheme}
% 
This type describes an invalid call to a procedure, either passing an
invalid number of arguments, or passing an argument of the wrong type.
\end{entry}

\begin{entry}{%
\ctproto{irritants}
\proto{irritants-condition?}{ obj}{procedure}
\proto{condition-irritants}{ condition}{procedure}}

This condition type could be defined by
%
\begin{scheme}
(define-condition-type \&irritants \&condition
  irritants-condition?
  (irritants condition-irritants))%
\end{scheme}
%
The {\cf irritants} field should contain a list of objects.  This
condition provides additional information about a condition, typically
the argument list of a procedure that detected an exception.
Conditions of this type are created by the {\cf error} and {\cf
  assertion-violation} procedures of report
section~\extref{report:errorviolation}{Errors and violations}.
\end{entry}
 
\begin{entry}{%
\ctproto{who}
\proto{who-condition?}{ obj}{procedure}
\proto{condition-who}{ condition}{procedure}}

This condition type could be defined by
%
\begin{scheme}
(define-condition-type \&who \&condition
  who-condition?
  (who condition-who))%
\end{scheme}
%
The {\cf who} field should contain a symbol or string identifying the
entity reporting the exception.
Conditions of this type are created by the {\cf error} and {\cf
  assertion-violation} procedures (report
section~\extref{report:errorviolation}{Errors and violations}), and
the {\cf syntax-violation} procedure
(section~\extref{syntax-violation}{Syntax violations}).
\end{entry}



%%% Local Variables: 
%%% mode: latex
%%% TeX-master: "r6rs-lib"
%%% End: 

     \par
\chapter{I/O}
\label{iochapter}

\section{Condition types}

In exceptional situations arising from ``I/O errors,'' the procedures
described in the specification below will raise an exception with
condition type {\cf\&i/o}.  Except where explicitly specified, there
is no guarantee that the raised condition object will contain all the
information that would be applicable. It is recommended, however, that
an implementation provide all information about an exceptional
situation in the condition object that is available at the place where
it is detected.

\begin{entry}{%
\ctproto{i/o}
\proto{i/o-error?}{ obj}{procedure}}

This condition type could be defined by
%
\begin{scheme}
(define-condition-type \&i/o \&error
  i/o-error?)
\end{scheme}        

This is a supertype for a set of more specific I/O errors.
\end{entry}   

\begin{entry}{%
\ctproto{i/o-read}
\proto{i/o-read-error?}{ obj}{procedure}}

\begin{scheme}
(define-condition-type \&i/o-read \&i/o
  i/o-read-error?)
\end{scheme}

This condition type describes read errors that occurred during an I/O
operation.
\end{entry}   

\begin{entry}{%
\ctproto{i/o-write}
\proto{i/o-write-error?}{ obj}{procedure}}

This condition type could be defined by
%
\begin{scheme}
(define-condition-type \&i/o-write \&i/o
  i/o-write-error?)
\end{scheme}
This condition type describes write errors that occurred during an I/O
    operation.
  \end{entry}   
  
\begin{entry}{%
\ctproto{i/o-invalid-position}
\proto{i/o-invalid-position-error?}{ obj}{procedure}}

This condition type could be defined by
%
\begin{scheme}
(define-condition-type \&i/o-invalid-position \&i/o
  i/o-invalid-position-error?
  (position i/o-error-position))
\end{scheme}

This condition type describes attempts to set the file position to an
invalid position. The value of the position field is the file position that
the program intended to set. This condition describes a range error, but
not a contract violation.
\end{entry}   

\begin{entry}{%
\ctproto{i/o-filename}
\proto{i/o-filename-error?}{ obj}{procedure}
\proto{i/o-error-filename}{ condition}{procedure}}

This condition type could be defined by
%
\begin{scheme}
(define-condition-type \&i/o-filename \&i/o
  i/o-filename-error?
  (filename i/o-error-filename))
\end{scheme}

This condition type describes an I/O error that occurred during an
operation on a named file. Condition objects belonging to this type
must specify a file name in the {\cf filename} field.
\end{entry}

\begin{entry}{%
\ctproto{i/o-file-protection}
\proto{i/o-file-protection-error?}{ obj}{procedure}}

This condition type could be defined by
%
\begin{scheme}
(define-condition-type \&i/o-file-protection
    \&i/o-filename
  i/o-file-protection-error?)
\end{scheme}

A condition of this type specifies that an operation tried to operate on a
named file with insufficient access rights.
\end{entry}   

\begin{entry}{%
\ctproto{i/o-file-is-read-only}
\proto{i/o-file-is-read-only-error?}{ obj}{procedure}}

This condition type could be defined by
%
\begin{scheme}
(define-condition-type \&i/o-file-is-read-only
    \&i/o-file-protection
  i/o-file-is-read-only-error?)
\end{scheme}

A condition of this type specifies that an operation tried to operate on a
named read-only file under the assumption that it is writeable.
\end{entry}   

\begin{entry}{%
\ctproto{i/o-file-already-exists}
\proto{i/o-file-already-exists-error?}{ obj}{procedure}}

This condition type could be defined by
%
\begin{scheme}
(define-condition-type \&i/o-file-already-exists
    \&i/o-filename
  i/o-file-already-exists-error?)
\end{scheme}
A condition of this type specifies that an operation tried to operate on an
existing named file under the assumption that it does not exist.
\end{entry}   

\begin{entry}{%
\ctproto{i/o-file-exists-not}
\proto{i/o-exists-not-error?}{ obj}{procedure}}

This condition type could be defined by
%
\begin{scheme}
(define-condition-type \&i/o-file-exists-not
    \&i/o-filename
  i/o-file-exists-not-error?)
\end{scheme}

A condition of this type specifies that an operation tried to operate on an
non-existent named file under the assumption that it exists.
\end{entry}   

\begin{entry}{%
\ctproto{i/o-operation-not-available}
\proto{i/o-operation-not-available-violation?}{ obj}{procedure}}

This condition type could be defined by
FIXME: should be a subtype of {\cf\&defect}
%
\begin{scheme}
(define-condition-type \&i/o-operation-not-available
    \&violation
  i/o-operation-not-available-violation?)
\end{scheme}

This condition type indicates that the program tried to perform an I/O
operation that was not available.
\end{entry}   
  
\begin{entry}{%
\ctproto{i/o-closed}
\proto{i/o-closed-violation?}{ obj}{procedure}}

This condition type could be defined by
%
\begin{scheme}
(define-condition-type \&i/o-closed \&violation
  i/o-closed-violation?)
\end{scheme}

A condition of this type specifies that an operation tried to operate
on a closed I/O object under the assumption that it is open.
\end{entry}

\section{Primitive I/O}

FIXME: parameter type conventions

This section defines the \library{r6rs i/o primitive} library, a
simple, primitive I/O subsystem.  It provides unbuffered I/O, and is
close to what a typical operating system offers. Thus, its interface
is suitable for implementing high-throughput and zero-copy I/O.

The Primitive I/O layer also allows clients to implement custom data
sources and sinks via a simple interface.

The Primitive I/O layer only handles blocking-I/O.

\subsection{Filenames}

Some of the procedures described here accept a filename filename as an
argument. Valid values for such a filename include strings naming a file using
the native notation of the operating system the Scheme implementation happens
to be running on.

It is expected that a future SRFI will extend this set of values by a more
abstract representation: This is necessary, as the most common operating
systems do not really use strings for representing filenames, but rather byte
or word sequences. Moreover, the string notation is difficult to manipulate and
not very portable.

\subsection{File options}
\label{fileoptionssection}

When opening a file, the various procedures in this library accept a
{\cf file-options} object containing a set of flags that specify how
the file is to be opened. A {\cf file-options} object is an enum-set
(see section~\ref{enumerationssection}) over the symbols constituting
valid file options.

\begin{entry}{%
\proto{file-options}{ \hyper{file-options-name} \dotsfoo}{\exprtype}}
   
The {\cf file-options} syntax returns a file-options object with the
specified options set. The following options (all affecting output
only) may be used:

\begin{itemize}   
\item {\cf create} create file if it does not already exist
\item {\cf exclusive} an exception with condition type
  {\cf\&i/o-file-already-exists} will be exclusive raised if this
  option and {\cf create} are both set and the file already exists
\item {\cf truncate}
  file is truncated
\end{itemize}

FIXME: everything else is ignored

The file-options object returned by {\cf (file-options)} specifies,
when supplied to an operation opening a file for output, that the file
must exist (otherwise an exception with condition type
{\cf\&i/o-file-exists-not} is raised) and its data is unchanged by the
operation.
\end{entry}   

\subsection{Readers and Writers}

The objects representing input data sources are called readers and
those representing output data sinks are called writers for output.
They are unbuffered and operate purely on binary data.  Although some
reader and writer objects might conceivably have something to do with
files or devices, programmers should never assume it.

The \library{r6rs i/o primitive} library has one condition type
specific to readers and writers:

\begin{entry}{%
\ctproto{i/o-reader/writer}
\proto{i/o-reader-writer-error?}{ obj}{procedure}
\proto{i/o-error-reader/writer}{ condition}{procedure}}

This condition type could be defined by

\begin{scheme}
(define-condition-type \&i/o-reader/writer \&i/o
  i/o-reader/writer-error?
  (reader/writer i/o-error-reader/writer))
\end{scheme}

This condition type allows specifying the particular reader or writer
with which an I/O error is associated. Conditions raised by the
Primitive I/O procedures may include an {\cf\&i/o-reader/writer}
condition, but are not required to do so.
\end{entry}

\subsection{I/O buffers}

\begin{entry}{%
\proto{make-i/o-buffer}{ size}{procedure}}
   
This creates a bytes object of size size with undefined contents.
Callers of the Primitive I/O procedures are encouraged to use bytes
objects created by {\cf make-i/o-buffer} because they might have
alignment and placement characteristics that {\cf make reader-read!}
and {\cf writer-write!} more efficient.  (These procedures are still
required to work on regular bytes objects, however.)
\end{entry}

\subsection{Readers}

The purpose of reader objects is to represent the output of arbitrary
algorithms in a form susceptible to imperative I/O.

FIXME: be precise about argument-type checking

\begin{entry}{%
\proto{reader?}{ obj}{procedures}}
   
Returns \schtrue{} if \var{obj} is a reader, otherwise returns \schfalse.
\end{entry}

\begin{entry}{%
\pproto{(make-simple-reader \var{id} \var{descriptor}}{procedure}}
\mainschindex{make-simple-reader}{\tt\obeyspaces\\
    \var{chunk-size} \var{read!} \var{available} \var{get-position}
    \var{set-position!} \var{end-position} \var{close})}
   
Returns a reader object. \var{Id} is a string naming the reader,
provided for informational purposes only. \var{Descriptor} may be any
object; the Primitive I/O system will not use it internally for any
purpose. \var{Descriptor} can be extracted from the reader object via
{\cf reader-descriptor}. Thus, descriptor can be used to keep the
internal state of certain kinds of readers.
   
\var{Chunk-size} must be a positive exact integer, and represents a recommended
efficient size of the read operations on this reader. This is typically the
block size of the buffers of the operating system. As such, it is only a
hint for clients of the reader---calls to the \var{read!} procedure (see below)
may specify a different read count.
   
The remaining arguments are procedures. These procedures must raise
exceptions with condition types as specified above when they encounter an
exceptional situation. When they do not, the effects are unspecified.

\var{Get-position}, \var{set-position!}, and \var{end-position} may be
omitted, in which case the corresponding arguments must be \schfalse.
   
\begin{itemize}
\item {\cf (\var{read!} \var{bytes} \var{start} \var{count})}
       
  \var{Start} and \var{count} must be non-negative exact integers.
  This reads up to \var{count} bytes from the reader and writes them
  into \var{bytes}, which must be a bytes object, starting at index
  start. \var{Bytes} must have size at least $\var{start} +
  \var{count}$. This procedure returns the number of bytes read as an
  exact integer. It returns 0 if it encounters an end of file, or if
  count is 0. If count is positive, this procedure blocks until at
  least one byte has been read or it has encountered end of file.
  
  \var{Bytes} may or may not be a bytes object returned by {\cf
    make-i/o-buffer}. It is possible that {\cf reader-read!} operates
  more efficiently if it is, however.
  
  \var{Count} may or may not be the same as the chunk size of the
  reader. It is possible that {\cf reader-read!} operates more
  efficiently if it is, however.

\item {\cf (\var{available})}
       
  This returns an estimate of the total number of bytes left in the
  reader. The return value is either an exact integer, or
  \schfalse{} if no such estimate is possible. There is no guarantee
  that this estimate will have any specific relationship to the true
  number of available bytes.

\item {\cf (\var{get-position})}
       
  This procedure, when present, returns the current position in the 
  reader as an exact integer counting the number of bytes since the
  beginning of the source. (EOFs do not count as bytes.)
  
\item {\cf (\var{set-position!} \var{pos})}
       
  This procedure, when present, moves to position \var{pos} (which
  must be a non-negative exact integer) in the reader.
       
  The procedure must return a single value, which is ignored.
       
\item {\cf (\var{end-position})}
       
  This procedure, when present, returns the position in the reader
  of the next end of file, without changing the current position.

\item {\cf (\var{close})}
       
  This procedure marks the reader as closed, performs any necessary
  cleanup, and releases the resources associated with the reader.
  Further operations on the reader must raise an exception with
  condition type {\cf\&i/o-closed}.
       
  The procedure must return a single value, which is ignored.
\end{itemize}

\end{entry}

\begin{entry}{%
\proto{reader-id}{ reader}}
   
This returns a string naming the reader, provided for informational
purposes only. For a file reader returned by {\cf open-file-reader} or
{\cf open-file-reader+writer}, this will be a string representation of the file
name.
  
For a reader created by {\cf make-simple-reader}, this returns the value that was
supplied as the \var{id} argument to {\cf make-simple-reader}.
\end{entry}

\begin{entry}{%
\proto{reader-descriptor}{ reader}{procedure}}
   
For a reader created by {\cf make-simple-reader}, this returns the value that was
supplied as the descriptor argument to {\cf make-simple-reader}.
   
For all other readers, this returns an unspecified value.
\end{entry}

\begin{entry}{%
\proto{reader-chunk-size}{ reader}{procedure}}
   
This returns a positive exact integer that represents a recommended
efficient size of the read operations on this reader. This is typically the
block size of the buffers of the operating system. As such, it is only a
hint for clients of the reader---calls to the {\cf reader-read!} procedure (see
below) may specify a different read count.

For a reader created by {\cf make-simple-reader}, this returns the
value that was supplied as the chunk-size argument to {\cf
  make-simple-reader}.
\end{entry}

\begin{entry}{%
\proto{reader-read!}{ reader bytes start count}{procedure}}
   
\var{Start} and \var{count} must be non-negative exact integers. This
reads up to \var{count} bytes from the reader and writes them into
\var{bytes}, which must be a bytes object, starting at index
\var{start}. \var{Bytes} must have at least $\var{start} +
\var{count}$ elements. This procedure returns the number of bytes read
as an exact integer. It returns 0 if it encounters an end of file, or
if \var{count} is 0.  This procedure blocks until at least one byte has been
read or it has encountered end of file.
   
\var{Bytes} may or may not be a bytes object returned by {\cf
  make-i/o-buffer}. It is possible that {\cf reader-read!} operates
more efficiently if it is, however.
   
\var{Count} may or may not be the same as the chunk size of the
reader. It is possible that {\cf reader-read!} operates more
efficiently if it is, however.
   
For a reader created by {\cf make-simple-reader}, this calls the
\var{read!} procedure of reader with the remaining arguments.

FIXME: return value
\end{entry}

\begin{entry}{%
\proto{reader-available}{ reader}{procedure}}
   
This returns an estimate of the total number of available bytes left in the
stream. The return value is either an exact integer, or \schfalse{} if no such
estimate is possible. There is no guarantee that this estimate will have
any specific relationship to the true number of available bytes.
   
For a reader created by {\cf make-simple-reader}, this calls the \var{available}
procedure of reader.
\end{entry}   

\begin{entry}{%
\proto{reader-has-get-position?}{ reader}{procedure}}
   
This returns \schtrue{} if reader supports the reader-get-position
procedure, and \schfalse{} otherwise.
\end{entry}

\begin{entry}{%
\proto{reader-get-position}{ reader}{procedure}}
   
When {\cf reader-has-get-position} returns \schtrue{} for reader, this
returns the current position in the byte stream as an exact integer
counting the number of bytes since the beginning of the stream.
   
For a reader created by {\cf make-simple-reader}, this calls the
\var{get-position} procedure of reader, if present. If reader does not
have a \var{get-position} procedure, an exception with condition type
{\cf\&i/o-operation-not-available} is raised.
\end{entry}

\begin{entry}{%
\proto{reader-has-set-position!?}{ reader}{procedure}}
   
This returns \schtrue{} if reader supports the {\cf reader-set-position!} operation, and
\schfalse{} otherwise.
\end{entry}

\begin{entry}{%
\proto{reader-set-position!}{ reader pos}{procedure}}

When {\cf reader-has-set-position!?} returns \schtrue{} for
\var{reader}, moves to position \var{pos} (which must be a non-negative
exact integer) in the stream.
 
For a reader created by {\cf make-simple-reader}, this calls the
\var{set-position!}  procedure of \var{reader} with the \var{pos}
argument, if present. If \var{reader} does not have a set-position!
procedure, an {\cf exception} with condition type
{\cf\&i/o-operation-not-available} is raised.
\end{entry}

\begin{entry}{%
\proto{reader-has-end-position?}{ reader}{procedure}}
   
This returns \schtrue{} if reader supports the {\cf reader-end-position}
operation, and \schfalse{} otherwise.
\end{entry}   

\begin{entry}{%
\proto{reader-end-position}{ reader}{procedure}}
   
When {\cf reader-has-end-position?} returns \schtrue{} for \var{reader},
this returns the position in the byte stream of the next end of file,
without changing the current position.
   
For a reader created by {\cf mmake-simple-reader}, this calls the
\var{end-position} procedure of \var{reader}, if present. If
\var{reader} does not have a \var{end-position} procedure, an
exception with condition type {\cf\&i/o-operation-not-available} is raised.
\end{entry}

\begin{entry}{%
\proto{reader-close}{ reader}{procedure}}
   
This marks \var{reader} as closed, performs any necessary cleanup, and
releases the resources associated with the reader. Further operations
on the reader must raise an exception with condition type
{\cf\&i/o-closed}.
   
For a reader created by {\cf make-simple-reader}, this calls the
\var{close} procedure of reader.
\end{entry}

\begin{entry}{%
\proto{open-bytes-reader}{ bytes}{procedure}}
   
This returns a \defining{bytes reader} that uses a copy of
\var{bytes}, a bytes object, as its contents. This reader has
\var{get-position}, \var{set-position!}, and \var{end-position}
operations.
\end{entry}

\begin{entry}{%
\proto{open-file-reader}{ filename}{procedure}
\rproto{open-file-reader}{ filename file-options}{procedure}}
   
This returns a reader connected to the file named by \var{filename}.
The \var{file-options} object defaults to {\cf (file-options)} if not
present. It may determine various aspects of the returned reader, see
section~\ref{fileoptionssection}. This reader has \var{get-position},
\var{set-position!}, and \var{end-position} operations.
\end{entry} 

\begin{entry}{%
\proto{standard-input-reader}{}{procedure}}
   
This returns a reader connected to the standard input. The meaning of
``standard input'' is implementation-dependent.
\end{entry}

\subsection{Writers}

The purpose of writer objects is to represent the input of arbitrary
algorithms in a form susceptible to imperative I/O.

\begin{entry}{%
\proto{writer?}{ obj}{procedure}}
   
Returns \schtrue{} if \var{obj} is a writer, otherwise returns \schfalse.
\end{entry}

\begin{entry}{%
\pproto{(make-simple-writer \var{id} \var{descriptor}}{procedure}}
\mainschindex{make-simple-writer}{\tt\obeyspaces\\
    \var{chunk-size} \var{write!} \var{get-position} \var{set-position!}
    \var{end-position} \var{close})}
   
Returns a writer object. \var{Id} is a string naming the writer,
provided for informational purposes only. For a file, this will be a
string representation of the file name. \var{Descriptor} may be any
object; the Primitive I/O system will not use it internally for any
purpose. \var{Descriptor} can be extracted from the writer object via
{\cf writer-descriptor}. Thus, descriptor can be used to keep the
internal state of certain kinds of writers.

\var{Chunk-size} must be a positive exact integer, and is the
recommended efficient size of the write operations on this writer. As
such, it is only a hint for clients of the reader---calls to the
\var{write!} procedure (see below) may specify a different write
count.

The remaining arguments are procedures. These procedures must raise
exceptions with condition types as specified above when they encounter
an exceptional situation. When they do not, the effects are
unspecified.

\var{Get-position}, \var{set-position!}, and \var{end-position} may be
omitted, in which case the corresponding arguments must be \schfalse.

\begin{itemize}
\item {\cf (\var{write!} \var{bytes} \var{start} \var{count})}
   
  \var{Start} and \var{count} must be non-negative exact integers.
  This writes up to \var{count} bytes in bytes object \var{bytes}
  starting at index \var{start}. Before it does this, it will block
  until it can write at least one byte. It returns the number of bytes
  actually written as a positive exact integer.
  
  \var{Bytes} may or may not be a bytes object returned by {\cf
    make-i/o-buffer}. It is possible that {\cf writer-write!} operates more
  efficiently if it is, however.
  
  \var{Count} may or may not be the same as the chunk size of the
  reader. It is possible that {\cf writer-write!} operates more
  efficiently if it is, however.

\item {\cf (\var{get-position})}
   
  This procedure, when present, returns the current position in the byte
  stream as an exact integer counting the number of bytes since the
  beginning of the stream.
   
\item {\cf (\var{set-position!} \var{pos})}
   
  This procedure, when present, moves to position \var{pos} (which must be a
  non-negative exact integer) in the stream.
  
  The procedure must return a single return value, which is ignored.

\item {\cf (\var{end-position})}
   
    This procedure, when present, returns the byte position of the next end
    of file, without changing the current position.
   
\item {\cf (\var{close})}
  
  This procedure marks the writer as closed, performs any necessary
  cleanup, and releases the resources associated with the writer. Further
  operations on the writer must raise an exception with condition type
  {\cf\&i/o-closed}.
   
  The procedure must return a single value, which is ignored.
\end{itemize}
\end{entry}

\begin{entry}{%
\proto{writer-id}{ writer}{procedure}}
   
This returns string naming the writer, provided for informational
purposes only. For a file writer returned by {\cf open-file-writer}
or {\cf open-file-reader+writer}, this will be a string
representation of the file name.
   
For a writer created by {\cf make-simple-writer}, this returns the
value of the \var{id} field of the argument writer.
\end{entry}

\begin{entry}{%
\proto{writer-descriptor}{ writer}{procedure}}
   
  For a writer created by {\cf make-simple-writer}, this returns the value of the
  \var{descriptor} field of the argument writer.
  
  For all other writers, this returns an unspecified value.
\end{entry}

\begin{entry}{%
\proto{writer-chunk-size}{ writer}{procedure}}
   
This returns a positive exact integer, and is the recommended efficient
size of the write operations on this writer. As such, it is only a hint for
clients of the reader---calls to {\cf writer-write!} (see below) may specify a
different write count.

For a writer created by {\cf make-simple-writer}, this returns the value of the
\var{chunk-size} field of the argument writer.
\end{entry}

\begin{entry}{%
\proto{writer-write!}{ writer bytes start count}{procedure}}
   
\var{Start} and \var{count} must be non-negative exact integers.
This writes up to \var{count} bytes in bytes object \var{bytes}
starting at index \var{start}. Before it does this, it will block
until it can write at least one byte. It returns the number of bytes
actually written as a positive exact integer.
   
\var{Bytes} may or may not be a bytes object returned by {\cf
  make-i/o-buffer}. It is possible that {\cf writer-write!} operates
more efficiently if it is, however.

\var{Count} may or may not be the same as the chunk size of the reader. It is
possible that {\cf writer-write!} operates more efficiently if it is, however.

For a writer created by {\cf make-simple-writer}, this calls the
\var{write!} procedure of writer with the remaining arguments.
\end{entry}   

\begin{entry}{%
\proto{writer-has-get-position?}{ writer}{procedure}}
   
This returns \schtrue{} if writer supports the {\cf
  writer-get-position} operation, and \schfalse{} otherwise.
\end{entry}   

\begin{entry}{%
\proto{writer-get-position}{ writer}{procedure}}
   
When {\cf writer-has-get-position?} returns \schtrue{} for
\var{writer}, this returns the current position in the byte stream as
an exact integer counting the number of bytes since the beginning of
the stream.
  
For a writer created by {\cf make-simple-writer}, this calls the
\var{get-position} procedure of \var{writer}, if present. If \var{writer}
does not have a get-position procedure, an exception with condition
type {\cf\&i/o-operation-not-available} is raised.
\end{entry}

\begin{entry}{%
\proto{writer-has-set-position!?}{ writer}{procedure}}
   
This returns \schtrue{} if writer supports the \var{writer-set-position!} operation, and
\schfalse{} otherwise.
\end{entry}

\begin{entry}{%
\proto{writer-set-position!}{ writer pos}{procedure}}
   
When {\cf writer-has-set-position!?} returns \schtrue{} for
\var{writer}, this moves to position \var{pos} (which must be a non-negative
exact integer) in the stream.
   
For a writer created by {\cf make-simple-writer}, this calls the
\var{set-position!}  procedure of \var{writer} with the \var{pos}
argument, if present. If \var{writer} does not have a
\var{set-position!} procedure, an exception with condition type
{\cf\&i/o-operation-not-available} is raised.
\end{entry}

\begin{entry}{%
\proto{writer-has-end-position?}{ writer}{procedure}}
   
This returns \schtrue{} if \var{writer} supports the {\cf
  writer-end-position} operation, and \schfalse{} otherwise.
\end{entry}

\begin{entry}{%
\proto{writer-end-position}{ writer}{procedure}}
   
When {\cf writer-has-end-position?} returns \schtrue{} for writer,
this returns the byte position of the next end of file, without
changing the current position.
   
For a writer created by {\cf make-simple-writer}, this calls the
\var{end-position} procedure of \var{writer}, if present. If \var{writer}
does not have a \var{end-position} procedure, an exception with condition
type {\cf\&i/o-operation-not-available} is raised.
\end{entry}

\begin{entry}{%
\proto{writer-close}{ writer}{procedure}}
   
This marks the writer as closed, performs any necessary cleanup, and
releases the resources associated with the writer. Further operations
on the writer must raise an exception with condition type
{\cf\&i/o-closed}.
   
For a writer created by {\cf make-simple-writer}, this calls the
\var{close} procedure of \var{writer}.
\end{entry}

\begin{entry}{%
\proto{open-bytes-writer}{}{procedure}}
   
This returns a \defining{bytes writer} that can yield everything written to it as
a bytes object. This writer supports the {\cf writer-get-position},
{\cf writer-set-position!}, and {\cf writer-end-position} operations.
operations.
\end{entry}

\begin{entry}{%
\proto{writer-bytes}{ writer}{procedure}}
   
The \var{writer} argument must be a bytes writer.  This procedure
returns a newly allocated bytes object containing the data written to
writer in sequence. Doing this in no way invalidates the writer or
change its store.
\end{entry}

\begin{entry}{%
\proto{open-file-writer}{ filename}{procedure}
\rproto{open-file-writer}{ filename file-options}{procedure}}
   
This returns a writer connected to the file named by \var{filename}.
The \var{file-options} object defaults to {\cf (file-options)} if not
present. It determines various aspects of the returned writer, see
section~\ref{fileoptionssection}.  This writer has \var{get-position},
\var{set-position!}, and \var{end-position} operations.
\end{entry}

\begin{entry}{%
\proto{standard-output-writer}{}{procedure}}
   
This returns a writer connected to the standard output. The meaning of
``standard output'' is implementation-dependent.
\end{entry}

\begin{entry}{%
\proto{standard-error-writer}{}{procedure}}
   
This returns a writer connected to the standard error. The meaning of
``standard error'' is implementation-dependent.
\end{entry}   

\subsection{Opening files for reading and writing}

\begin{entry}{%
\proto{open-file-reader+writer}{ filename}{procedure}
\rproto{open-file-reader+writer}{ filename file-options}{procedure}}
   
This returns two values, a reader and a writer connected to the file
named by \var{filename}. The \var{file-options} object defaults to
{\cf (file-options)} if not present. It determines various aspects of
the returned writer and possibly the reader, see
section~\ref{fileoptionssection}. The reader and the writer support
the {\cf reader-get-position}, {\cf reader-set-position!}, {\cf
  reader-end-position}, and the {\cf writer-get-position}, {\cf
  writer-set-position!}, {\cf writer-end-position} operations,
respectively.

\begin{note}
  This procedure enables opening a file for simultaneous input and
  output in environments where it is not possible to call
  open-file-reader and open-file-writer on the same file.
\end{note}     
\end{entry}

\subsection{Examples}

\begin{schemenoindent}
; This customized reader reads from a list of bytes
; objects. A null bytes object yields EOF.

(define (open-bytess-reader bs)
  (let* ((pos 0))
    (make-simple-reader
     "<byte vectors>"
     bs
     5
     (lambda (bytes start count)
       (cond
        ((null? bs)
         0)
        (else
         (let* ((b (car bs))
                (size (bytes-length b))
                (real-count (min count (- size pos))))
           (bytes-copy! b pos
                           bytes start
                           real-count)
           (set! pos (+ pos real-count))
           (if (= pos size)
               (begin
                 (set! bs (cdr bs))
                 (set! pos 0)))
           real-count))))
     ;; pretty rough ...
     (lambda ()
       (if (null? bs)
           0
           (- (bytes-length (car bs)) pos)))
     \schfalse{} \schfalse{} \schfalse{}
     (lambda ()
       (set! bs \schfalse{})))))

; Algorithmic reader producing an infinite
; stream of blanks:

(define (make-infinite-blanks-reader)
  (make-simple-reader
    "<blanks, blanks, and more blanks>"
    \schfalse{}
    4096
    (lambda (bytes start count)
      (let loop ((index 0))
        (if (>= index count)
            index
            (begin
              (bytes-u8-set! bytes (+ start index) 32)
              (loop (+ 1 index))))))
    (lambda ()
      1000) ; some number
    \schfalse{} \schfalse{} \schfalse{}
    unspecific))

FIXME: examples for writers
\end{schemenoindent}
                    
%%% Local Variables: 
%%% mode: latex
%%% TeX-master: "r6rs"
%%% End: 

\section{Port I/O}
\label{portsiosection}

The \deflibrary{r6rs i/o ports} library defines an I/O layer for
conventional, imperative buffered input and output.
A \defining{port} represents a buffered access object
for a data sink or source or both simultaneously.
The library allows ports to be created from arbitrary data sources
and sinks.

The \library{r6rs i/o ports} library distinguishes between \textit{input
  ports\mainindex{input port}} and \textit{output
  ports\mainindex{output ports}}.  An input port is a source for data,
whereas an output port is a sink for data.  A port may be both an
input port and an output port; such a port typically provides
simultaneous read and write access to a file or other data.

The \library{r6rs i/o ports} library also distinguishes between
\textit{binary ports\mainindex{binary port}}, which are sources
or sinks for uninterpreted bytes, and
\textit{textual ports\mainindex{textual ports}}, which are sources
or sinks for characters and strings.

This section uses \var{input-port}, \var{output-port},
\var{binary-port}, \var{textual-port},
\var{binary-input-port}, \var{textual-input-port},
\var{binary-output-port}, \var{textual-output-port},
and \var{port} as
parameter names for arguments that must be input ports (or combined
input/output ports), output ports (or combined input/output ports),
binary ports, textual ports, binary input ports, textual input ports,
binary output ports, textual output ports, or any kind of port,
respectively.

\subsection{File names}
\label{filenamesection}

Some of the procedures described in this chapter accept a file name as an
argument. Valid values for such a file name include strings that name a file
using the native notation of filesystem paths on an implementation's
underlying operating system, and may include implementation-dependent
values as well.

\begin{rationale}
Implementation-dependent file names may provide a more
abstract and/or more general representation. Indeed, most operating
systems do not use strings for representing file names, but rather byte
or word sequences.
Furthermore the string notation is not fully portable across operating
systems, and is difficult to manipulate.
\end{rationale}

A \var{filename} parameter name means that the
corresponding argument must be a file name.

\subsection{File options}
\label{fileoptionssection}

\mainindex{file options}
When opening a file, the various procedures in this library accept a
{\cf file-options} object that encapsulates flags to specify how
the file is to be opened. A {\cf file-options} object is an enum-set
(see chapter~\ref{enumerationschapter}) over the symbols constituting
valid file options.
A \var{file-options} parameter name means that the
corresponding argument must be a file-options object.

\begin{entry}{%
\proto{file-options}{ \hyper{file-options name} \dotsfoo}{\exprtype}}

Each \hyper{file-options name} must be an \meta{identifier}.
The {\cf file-options} syntax returns a file-options object that 
encapsulates the
specified options. The following options, which affect output
only, have standard meanings:

\begin{itemize}   
\item {\cf create} create file if it does not already exist
\item {\cf exclusive} an exception with condition type
  {\cf\&i/o-file-already-exists} is raised if this
  option and {\cf create} are both set and the file already exists
\item {\cf truncate}
  file is truncated to zero length
\end{itemize}

\meta{Identifiers}s
other than those listed above may be used as \hyper{file-options name}s;
they have implementation-specific meaning, if any.

When supplied to an operation that opens a file for output,
the file-options object returned by {\cf (file-options)} specifies
that the file
must already exist (otherwise an exception with condition type
{\cf\&i/o-file-exists-not} is raised) and its data are unchanged by the
operation;
note that this default seldom coincides with the desired semantics.

\begin{rationale}
  The flags specified above represent only a common subset of
  meaningful options on popular platforms.  The {\cf file-options}
  form does not restrict the \hyper{file-options name}s, so 
  implementations can extend the file options by platform-specific
  flags.
\end{rationale}
\end{entry}   

\subsection{Buffer modes}

Each output port has an associated buffer mode that defines when an
output operation flushes the buffer associated with the output
port. The possible buffer modes are the symbols {\cf none} for no buffering,
{\cf line} for flushing upon line feeds and line separators (U+2028), and
{\cf block} for arbitrary buffering.  This section uses the parameter name
\var{buffer-mode} for arguments that must be buffer-mode symbols.

\begin{entry}{%
\proto{buffer-mode}{ \hyper{name}}{\exprtype}}
   
\hyper{Name} must be one of the \meta{identifier}s {\cf none}, {\cf line}, or
{\cf block}. The result is the corresponding symbol, denoting the
associated buffer mode.

It is a syntax violation if \hyper{name} is not one of the valid
identifiers.
\end{entry}

\begin{entry}{%
\proto{buffer-mode?}{ obj}{procedure}}
   
Returns \schtrue{} if the argument is a valid buffer-mode symbol,
\schfalse{} otherwise.
\end{entry}

\subsection{Transcoders}

Several different Unicode encoding schemes describe standard ways to
encode characters and strings as byte sequences and to decode those
sequences~\cite{Unicode}.
Within this document, a \defining{codec} is an immutable Scheme
object that represents a Unicode or similar encoding scheme.

A \defining{transcoder} is an immutable Scheme object that combines
a codec with an end-of-line convention and a method for handling
decoding errors.
Each transcoder represents some specific bidirectional (but not
necessarily lossless), possibly stateful translation between byte
sequences and Unicode characters and strings.
Every transcoder can operate in the input direction (bytes to characters)
or in the output direction (characters to bytes),
but the composition of those directions need not be identity (and
often isn't).  The composition of two transcoders is not defined.
A \var{transcoder} parameter name means that the corresponding
argument must be a transcoder.

Every port has a single transcoder associated with it.

A binary port is defined as a port whose associated transcoder
is the \defining{binary transcoder}, which is a special
pseudo-transcoder whose input and output directions translate
no byte sequences to characters, and no character sequences to
bytes.

A textual port is defined as a port whose associated transcoder
is not the binary transcoder.

\begin{entry}{%
\proto{latin-1-codec}{}{procedure}
\proto{utf-8-codec}{}{procedure}
\proto{utf-16-codec}{}{procedure}
\proto{utf-32-codec}{}{procedure}}

These are predefined codecs for the ISO 8859-1, UTF-8,
UTF-16, and UTF-32 encoding schemes \cite{Unicode}.

A call to any of these procedures returns a value that is equal in the
sense of {\cf eqv?} to the result of any other call to the same
procedure.
\end{entry}

\begin{entry}{%
\proto{eol-style}{ name}{\exprtype}}

If \var{name} is one of the \meta{identifier}s {\cf lf}, {\cf cr},
{\cf crlf}, or {\cf ls}, then the form evaluates to the corresponding
symbol.  If \var{name} is not one of these identifiers, effect and
result are implementation-dependent: The result may be an
eol-style symbol acceptable as an \var{eol-mode}
argument to {\cf make-transcoder}.  Otherwise, an exception is raised.

\begin{rationale}
  End-of-line styles other than those listed might become commonplace
  in the future.
\end{rationale}
\end{entry}

\begin{entry}{%
\proto{native-eol-style}{}{procedure}}

Returns the default end-of-line style of the underlying platform, e.g.
{\cf lf} on Unix and {\cf crlf} on Windows.
\end{entry}

\begin{entry}{%
\ctproto{i/o-decoding}
\proto{i/o-decoding-error?}{ obj}{procedure}}

\begin{scheme}
(define-condition-type \&i/o-decoding \&i/o-port
  i/o-decoding-error?
  (transcoder i/o-decoding-error-transcoder))
\end{scheme}

An exception with this type is raised when one of the operations for
textual input from a port encounters a sequence of bytes that cannot
be translated into a character or string by the input direction of the
port's transcoder.  The {\cf transcoder} field contains the port's
transcoder.

Exceptions of this type raised by the operations described in this
section are continuable.
When such an exception is raised, the port's position is at
the beginning of the invalid encoding.
If the exception handler returns, it must
return a character or string representing the decoded text starting at
the port's current position, and the exception handler must update the 
port's position to point past the error.
\end{entry}

\begin{entry}{% 
\ctproto{i/o-encoding}
\proto{i/o-encoding-error?}{ obj}{procedure}
\proto{i/o-encoding-error-char}{ condition}{procedure}
\proto{i/o-encoding-error-transcoder}{ condition}{procedure}}

This condition type could be defined by
%
\begin{scheme}
(define-condition-type \&i/o-encoding \&i/o-port
  i/o-encoding-error?
  (char i/o-encoding-error-char)
  (transcoder i/o-encoding-error-transcoder))
\end{scheme}

An exception with this type is raised when one of the operations for
textual output to a port encounters a character that cannot be
translated into bytes by the output direction of the port's transcoder.
The {\cf char} field of the
condition object contains the character that could not be encoded,
and the {\cf transcoder} field contains the transcoder associated
with the port.

Exceptions of this type raised by the operations described in this
section are continuable.  The handler, if it returns, is expected to
output to the port an appropriate encoding for the character that
caused the error.  The operation that raised the exception 
continues after that character.
\end{entry}

\begin{entry}{%
\proto{error-handling-mode}{ name}{\exprtype}}

If \var{name} is one of the \meta{identifier}s {\cf ignore}, {\cf
  raise}, or {\cf replace}, then the result is the corresponding
symbol.  If \var{name} is not one of these identifiers, effect and
result are implementation-dependent: The result may be an
error-handling-mode symbol acceptable as a \var{handling-mode}
argument to {\cf make-transcoder}.  Otherwise, an exception is raised.

\begin{rationale}
  Implementations may support error-handling modes other than those
  listed.
\end{rationale}

The error-handling mode of a transcoder specifies the behavior
of textual I/O operations in the presence of encoding or decoding
errors.

If a textual input operation encounters an invalid or incomplete
character encoding, and the error-handling mode is {\cf ignore},
then an appropriate number of bytes of the
invalid encoding are ignored and decoding continues with the
following bytes.
If the error-handling mode is {\cf replace}, then the replacement
character U+FFFD is injected into the data stream, an appropriate
number of bytes are ignored, and decoding
continues with the following bytes.
If the error-handling mode is {\cf raise}, then a continuable
exception with condition type {\cf\&i/o-decoding} is raised;
see the description of
{\cf\&i/o-decoding} for details
on how to handle such an exception.

If a textual output operation encounters a character it cannot encode,
and the error-handling mode is {\cf ignore}, then the character is
ignored and encoding continues with the next character.
If the error-handling mode is {\cf replace}, a codec-specific
replacement character is emitted by the transcoder, and encoding
continues with the next character.
The replacement character is U+FFFD for transcoders whose codec
is one of the Unicode encodings, but is the {\cf ?}
character for the Latin-1 encoding.
If the error-handling mode is {\cf raise}, an
exception with condition type {\cf\&i/o-encoding} is raised;
see the description of
{\cf\&i/o-decoding} for details
on how to handle such an exception.
\end{entry}

\begin{entry}{%
\proto{make-transcoder}{ codec}{procedure}
\rproto{make-transcoder}{ codec eol-style}{procedure}
\rproto{make-transcoder}{ codec eol-style handling-mode}{procedure}}

\domain{\var{Codec} must be a codec; \var{eol-style}, if present, an
  eol-style symbol; and \var{handling-mode}, if present, an
  error-handling-mode symbol.}  \var{eol-style} may be omitted, in
which case it defaults to the native end-of-line style of the
underlying platform.  \var{handling-mode} may be omitted, in which
case it defaults to {\cf raise}.  The result is a transcoder with the
behavior specified by its arguments.

A transcoder returned by {\cf make-transcoder} is equal in the sense
of {\cf eqv?} (but not necessarily in the sense of {\cf eq?})
to another transcoder returned by {\cf
  make-transcoder} if and only if the \var{code}, \var{eol-style},
and \var{handling-mode} arguments are equal in the sense of {\cf
  eqv?}.
\end{entry}

\begin{entry}{%
\proto{binary-transcoder}{}{procedure}}

Returns the binary transcoder, which is equal in the sense of
{\cf eqv?} (but not necessarily in the sense of {\cf eq?}) to
the result of any call to {\cf binary-transcoder}, and
is not equal to the result of any call to {\cf make-transcoder}.
\end{entry}

\begin{entry}{%
\proto{transcoder-codec}{ transcoder}{procedure}
\proto{transcoder-eol-style}{ transcoder}{procedure}
\proto{transcoder-error-handling-mode}{ transcoder}{procedure}}

These are accessors for transcoder objects; when applied to a
transcoder returned by {\cf make-transcoder}, they return the
\var{codec}, \var{eol-style}, \var{handling-mode} arguments.
When applied to the binary transcoder, they return \schfalse{}.
\end{entry}

\begin{entry}{%
\proto{bytevector->string}{ bytevector transcoder}{procedure}}

Returns the string that results from transcoding the
\var{bytevector} according to the input direction of
the \var{transcoder}.
\end{entry}

\begin{entry}{%
\proto{string->bytevector}{ string transcoder}{procedure}}

Returns the bytevector that results from transcoding the
\var{string} according to the output direction of
the \var{transcoder}.
\end{entry}

\subsection{Input and output ports}

The operations described in this section are common to input and
output ports, both binary and textual.
Every port is associated with a transcoder;
the transcoder of a binary port is the binary transcoder, and
the transcoder of a textual port is not the binary transcoder.
A port may also have an associated \defining{position} that
specifies a particular place within its data sink or source as a byte
count from the beginning of the sink or source, and may also provide
operations for inspecting and setting that place.
(Ends of file do not count as bytes.)

\begin{entry}{%
\proto{port?}{ obj}{procedure}}
   
Returns \schtrue{} if the argument is a port, and returns \schfalse{}
otherwise.
\end{entry}

\begin{entry}{%
\proto{port-transcoder}{ port}{procedure}}

Returns the transcoder associated with \var{port}.
\end{entry}

\begin{entry}{%
\proto{binary-port?}{ port}{procedure}}

Returns \schtrue{} if the transcoder associated with the \var{port}
is the binary transducer, and returns \schfalse{} otherwise.
\end{entry}

\begin{entry}{%
\proto{transcoded-port}{ binary-port transcoder}{procedure}}

\domain{\var{Transcoder} must not be the binary transcoder.}
The {\cf transcoded-port} procedure
returns a new textual port with the specified \var{transcoder}.
Otherwise the new textual port's state is largely the same as
that of the \var{binary-port}.
If the \var{binary-port} is an input port, then the new textual
port will be an input port and
will transcode the bytes that have not yet been read from
the \var{binary-port}.
If the \var{binary-port} is an output port, then the new textual
port will be an output port and
will transcode output characters into bytes that are
written to the byte sink represented by the \var{binary-port}.

As a side effect, however, the {\cf transcoded-port} procedure
closes \var{binary-port} in
a special way that allows the new textual port to continue to
use the byte source or sink represented by the \var{binary-port},
even though the \var{binary-port} itself is closed and cannot
be used by the input and output operations described in this
chapter.

\begin{rationale}
Closing the \var{binary-port} precludes interference between
the \var{binary-port} and the textual port constructed from it.
\end{rationale}
\end{entry}

\begin{entry}{%
\proto{port-has-port-position?}{ port}{procedure}
\proto{port-position}{ port}{procedure}}

The {\cf port-has-port-position?} procedure returns \schtrue{} if the
port supports the {\cf port-position} operation, and \schfalse{}
otherwise.

The {\cf port-position} procedure
returns the exact non-negative integer index of the position at which the
next byte would be read from or written to the port.
This procedure raises an exception with condition type {\cf\&assertion}
if the port does not support the operation.
\end{entry}   

\begin{entry}{%
\proto{port-has-set-port-position!?}{ port}{procedure}
\proto{set-port-position!}{ port pos}{procedure}}

\domain{\var{Pos} must be a non-negative exact integer.}
   
The {\cf port-has-set-port-position?} procedure returns \schtrue{} if the port
supports the {\cf set-port-position!} operation, and \schfalse{}
otherwise.
   
The {\cf set-port-position!} procedure sets the current byte position
of the port to \var{pos}.  If \var{port} is an output or combined
input and output port, this first flushes \var{port}.  (See {\cf
  flush-output-port}, section~\ref{flush-output-port}.)
This procedure raises an exception with condition type {\cf\&assertion}
if the port does not support the operation.
\end{entry}

\begin{entry}{%
\proto{close-port}{ port}{procedure}}
   
Closes the port, rendering the port incapable of delivering or
accepting data. If \var{port} is an output port, it is flushed before
being closed.  This has no effect if the port has already been closed.
A closed port is still a port. The unspecified value is returned.
\end{entry}

\begin{entry}{%
\proto{call-with-port}{ port proc}{procedure}}
   
\domain{\var{Proc} must be a procedure that accepts a single
  argument.}  The {\cf call-with-port} procedure
calls \var{proc} with \var{port} as an argument. If
\var{proc} returns, then the \var{port} is closed automatically and
the values returned by \var{proc} are returned. If \var{proc} does not
return, then the port is not closed automatically, except perhaps when it is
possible to prove that the port will never again be used for a
{\cf lookahead}, {\cf get}, or {\cf put} operation.
\end{entry}

\subsection{Input ports}

An input port allows reading an infinite sequence of bytes
or characters punctuated
by end of file objects. An input port connected to a finite data
source ends in an infinite sequence of end of file objects.

It is unspecified whether a character encoding consisting of several
bytes may have an end of file between the bytes.  If, for example,
{\cf get-char} raises an {\cf\&i/o-decoding} exception because the
character encoding at the port's position is incomplete up to the next
end of file, a subsequent call to {\cf get-char} may successfully
decode a character if bytes completing the encoding are available
after the end of file.

\begin{entry}{%
\proto{input-port?}{ obj}{procedure}}

Returns \schtrue{} if the argument is an input port (or a combined input
and output port), and returns \schfalse{} otherwise.
\end{entry}

\begin{entry}{%
\proto{port-eof?}{ input-port}{procedure}}
   
Returns \schtrue{}
if the {\cf lookahead-u8} procedure (if \var{input-port} is a binary port)
or the {\cf lookahead-char} procedure (if \var{input-port} is a textual port)
would return
the end-of-file object, and returns \schfalse{} otherwise.
\end{entry}

\begin{entry}{%
\proto{open-file-input-port}{ filename}{procedure}
\rproto{open-file-input-port}{ filename file-options}{procedure}
\rproto{open-file-input-port}{ filename file-options transcoder}{procedure}}
   
Returns an input port for the named file. The \var{file-options} and
\var{transcoder} arguments are optional.

The \var{file-options} argument, which may determine
various aspects of the returned port (see section~\ref{fileoptionssection}),
defaults to {\cf (file-options)}.

If \var{transcoder} is specified, it becomes the transcoder associated
with the returned port.
If no \var{transcoder} is specified, then an implementation-dependent
(and possibly locale-dependent) transcoder is associated with the port.

If the binary transcoder is passed as an explicit argument,
then the port will be a binary port and will support the
{\cf port-position} and {\cf set-port-position!}  operations.
Otherwise the port will be a textual port, and whether it supports
the {\cf port-position} and {\cf set-port-position!} operations
will be implementation-dependent (and possibly transcoder-dependent).

\begin{rationale}
  The byte position of a complexly transcoded port may not be
  well-defined, and may be hard to calculate even when defined,
  especially when transcoding is buffered.
\end{rationale}
\end{entry}

\begin{entry}{%
\proto{open-bytevector-input-port}{ bytevector}{procedure}
\rproto{open-bytevector-input-port}{ bytevector transcoder}{procedure}}
   
Returns an input port whose bytes are drawn from the
\var{bytevector}.
If \var{transcoder} is specified, it becomes the transcoder associated
with the returned port.

If no \var{transcoder} argument is given, or
if the binary transcoder is passed as an explicit argument,
then the port will be a binary port and will support the
{\cf port-position} and {\cf set-port-position!}  operations.
Otherwise the port will be a textual port, and whether it supports
the {\cf port-position} and {\cf set-port-position!} operations
will be implementation-dependent (and possibly transcoder-dependent).

If \var{bytevector} is modified after {\cf open-\linebreak[0]bytevector-\linebreak[0]input-\linebreak[0]port}
has been called, the effect on the returned
port is unspecified.
\end{entry}

\begin{entry}{%
\proto{open-string-input-port}{ string}{procedure}}

Returns a textual input port whose characters are drawn from
\var{string}.  The port's transcoder is implementation-dependent,
but is not the binary transcoder.
Whether the port supports
the {\cf port-position} and {\cf set-port-position!} operations
is implementation-dependent.

If \var{string} is modified after {\cf open-string-input-port}
has been called, the effect on the returned port is unspecified.
\end{entry}

\begin{entry}{%
\proto{standard-input-port}{}{procedure}}
   
Returns an input port connected to standard input, possibly a fresh
one on each call.
Whether the port is binary or textual is implementation-dependent,
as is whether the port supports
the {\cf port-position} and {\cf set-port-position!} operations.

\begin{note}
  Implementations are encouraged to return a textual port when
  appropriate, and
  to associate an appropriate transcoder with the port.
\end{note}
\end{entry}

\begin{entry}{%
\pproto{(make-custom-binary-input-port \var{id} \var{read!}}{procedure}}
\mainschindex{make-custom-binary-input-port}{\tt\obeyspaces\\
  \var{get-position} \var{set-position!} \var{close})}

Returns a newly created binary input port whose byte source is
an arbitrary algorithm represented by the \var{read!} procedure.
\var{Id} must be a string naming the new port,
provided for informational purposes only.
\var{Read!} must be a procedure, and should behave as specified
below; it will be called by operations that perform binary input.

Each of the remaining arguments may be \schfalse{}; if any of
those arguments is not \schfalse{}, it must be a procedure and
should behave as specified below.
   
\begin{itemize}
\item {\cf (\var{read!} \var{bytevector} \var{start} \var{count})}
       
  \var{Start} will be a non-negative exact integer,
  \var{count} will be a positive exact integer,
  and \var{bytevector} will be a bytevector whose length is at least
  $\var{start} + \var{count}$.
  The \var{read!} procedure obtains up to \var{count} bytes
  from the byte source and writes those bytes
  into \var{bytevector} starting at index \var{start}.
  The \var{read!} procedure must return the number of bytes
  that it writes, as an exact integer.
  To indicate an end of file condition, the \var{read!}
  should write no bytes and return 0.

\item {\cf (\var{get-position})}
       
  The \var{get-position} procedure (if supplied)
  returns the current position of the input port.
  If not supplied, the custom port will not support
  the {\cf port-position} operation.
  
\item {\cf (\var{set-position!} \var{k})}
       
  The \var{set-position!} procedure (if supplied) sets the position
  of the input port to \var{k}.
  If not supplied, the custom port will not support
  the {\cf set-port-position!} operation.
       
\item {\cf (\var{close})}
       
  The \var{close} procedure (if supplied) performs any
  actions that are necessary when the input port is closed.
\end{itemize}

\end{entry}

\subsection{Binary input}

\begin{entry}{%
\proto{get-u8}{ binary-input-port}{procedure}}
   
Reads from \var{binary-input-port}, blocking as necessary, until data are
available from \var{binary-input-port} or until an end of file is reached.
If a byte becomes available, {\cf get-u8} returns the byte as an octet, and
updates \var{binary-input-port} to point just past that byte. If no input
byte is seen before an end of file is reached, then the end-of-file
object is returned.
\end{entry}

\begin{entry}{%
\proto{lookahead-u8}{ binary-input-port}{procedure}}
   
The {\cf lookahead-u8} procedure is like {\cf get-u8}, but it does not 
update \var{binary-input-port} to point past the byte.
\end{entry}

\begin{entry}{%
\proto{get-bytevector-n}{ binary-input-port k}{procedure}}
   
Reads from \var{binary-input-port}, blocking as necessary, until \var{k}
bytes are available from \var{binary-input-port} or until an end of file is
reached. If \var{k} or more bytes are available before an end
of file, {\cf get-bytevector-n} returns a bytevector of size \var{k}.
If fewer bytes are available before an end of file, {\cf get-bytevector-n}
returns a bytevector
containing those bytes. In either case, the input port is updated to
point just past the bytes read.  If an end of file is reached before
any bytes are available, {\cf get-bytevector-n} returns the end-of-file object.
\end{entry}

\begin{entry}{%
\pproto{(get-bytevector-n! \var{binary-input-port}}{procedure}}
\mainschindex{get-bytevector-n!}{\tt\obeyspaces\\
    \var{bytevector} \var{start} \var{count})}

\domain{\var{Count} must be an exact, non-negative integer, specifying
  the number of bytes to be read. \var{bytevector} must be a bytevector
  with at
  least $\var{start} + \var{count}$ elements.}
   
The {\cf get-bytevector-n!} procedure reads from \var{binary-input-port},
blocking as necessary, until
\var{count} bytes are available from \var{binary-input-port} or until
an end of file is
reached. If \var{count} or more bytes are available before an end of file,
they are written into \var{bytevector} starting at index \var{start}, and
the result is \var{count}. If fewer bytes are available before
the next end of file, the available bytes are written into \var{bytevector}
starting at index \var{start}, and the result is the number of bytes actually
read. In either case, the input port is updated to point just past the
data read. If an end of file is reached before any bytes
are available, {\cf get-bytevector-n!} returns the end-of-file object.
\end{entry}

\begin{entry}{%
\proto{get-bytevector-some}{ binary-input-port}{procedure}}
   
Reads from \var{binary-input-port}, blocking as necessary, until data are
available from \var{binary-input-port} or until an end of file is reached.
If data become available,
{\cf get-bytevector-some} returns a freshly allocated
bytevector of non-zero size containing the available data, and it updates
\var{binary-input-port} to point just past that data.
If no input bytes are seen before an end
of file is reached, then the end-of-file object is returned.
\end{entry}

\begin{entry}{%
\proto{get-bytevector-all}{ binary-input-port}{procedure}}
   
Attempts to read all data until the next end of file, blocking as
necessary. If one or more bytes are read, {\cf get-bytevector-all} returns
a bytevector
containing all bytes up to the next end of file.  Otherwise, {\cf
  get-bytevector-all} returns the end-of-file object.
Note that  {\cf get-bytevector-all}
may block indefinitely, waiting to see an end of
file, even though some bytes are available.
\end{entry}

\subsection{Textual input}

\begin{entry}{%
\proto{get-char}{ textual-input-port}{procedure}}
   
Reads from \var{textual-input-port}, blocking as necessary, until the
complete encoding for a character is available from \var{textual-input-port},
or until the available input data cannot be the prefix of any valid encoding,
or until an end of file is reached.

If a complete character is available before the next end of file, {\cf
  get-char} returns that character, and updates the input port to
point past the data that encoded that character. If an end of file is
reached before any data are read, then {\cf get-char} returns the
end-of-file object.
\end{entry}

\begin{entry}{%
\proto{lookahead-char}{ textual-input-port}{procedure}}
  
The {\cf lookahead-char} procedure is like {\cf get-char}, but it does not 
update \var{textual-input-port} to point past the data
that encode the character.

\begin{note}
  With some of the standard transcoders
  described in this document, up to four bytes of lookahead are
  required. Nonstandard transcoders may require even more lookahead.
\end{note}
\end{entry}

\begin{entry}{%
\proto{get-string-n}{ textual-input-port k}{procedure}}
   
Reads from \var{textual-input-port}, blocking as necessary, until the
encodings of \var{k} characters (including invalid encodings, if they
don't raise an exception) are available, or until an end of
file is reached.
   
If \var{k} or more characters are read before end of file, {\cf
  get-string-n} returns a string consisting of those \var{k}
characters. If fewer characters are available before an end of file,
but one or more characters can be read,
{\cf get-string-n} returns a string containing
those characters. In either case, the input port is updated to point
just past the data read. If no data can be read before an 
end of file, then the end-of-file object is returned.
\end{entry}

\begin{entry}{%
\proto{get-string-n!}{ textual-input-port string start count}{procedure}}

\domain{\var{Start} and \var{count} must be an exact, non-negative
  integer, specifying the number of characters to be read.
  \var{string} must be a string with at least $\var{start} +
  \var{count}$ characters.}

Reads from \var{textual-input-port} in the same manner as {\cf
  get-string-n}.  If \var{count} or more characters are available
before an end of file, they are written into string
starting at index \var{start}, and \var{count} is returned. If fewer
characters are available before an end of file, but one
or more can be read, then those characters are written into string
starting at index \var{start}, and the number of characters actually read is
returned. If no characters can be read before an end of file,
then the end-of-file object is returned.
\end{entry}   

\begin{entry}{%
\proto{get-string-all}{ textual-input-port}{procedure}}
   
Reads from \var{textual-input-port} until an end of file, decoding
characters in the same manner as {\cf get-string-n} and {\cf get-string-n!}.
   
If data are available before the end of file, a string
containing all the text decoded from that data are returned. If no data
precede the end of file, the end-of-file object file object is
returned.
\end{entry}

\begin{entry}{%
\proto{get-line}{ textual-input-port}{procedure}}
   
Reads from \var{textual-input-port} up to and including the next
end-of-line encoding or line separator character (U+2028) or 
end of file, decoding characters in the same manner as {\cf
  get-string-n} and {\cf get-string-n!}.
   
If an end-of-line encoding or line separator is read, then a string
containing all of the text up to (but not including) the end-of-line
encoding is returned, and the port is updated to point just past the
end-of-line encoding or line separator. If an end of file is
encountered before any end-of-line encoding is read, but some data
have been read and decoded as characters, then a string containing
those characters is returned. If an end of file is encountered before
any data are read, then the end-of-file object is
returned.
\end{entry}

\begin{entry}{%
\proto{get-datum}{ textual-input-port}{procedure}}
 
Reads an external representation from \var{textual-input-port} and returns the
datum it represents.  The {\cf get-datum} procedure returns the next
datum that can be parsed from the given \var{textual-input-port}, updating
\var{textual-input-port} to point exactly past the end of the external
representation of the object.

Any \meta{interlexeme space}
(see report section~\ref{report:lexicalsyntaxsection}) in
the input is first skipped.  If an end of file occurs after the
\meta{interlexeme space}, the end of file object (see
report section~\ref{report:eofsection}) is returned.

If a character inconsistent with an external representation is
encountered in the input, an exception with condition types
{\cf\&lexical} and {\cf\&i/o-read} is raised.
Also, if the end of file is encountered
after the beginning of an external representation, but the external
representation is incomplete and therefore cannot be parsed, an exception
with condition types {\cf\&lexical} and {\cf\&i/o-read} is raised.
\end{entry}

\subsection{Output ports}

An output port is a sink to which bytes or characters are written.
These data may control
external devices, or may produce files and other objects that may
subsequently be opened for input.

\begin{entry}{%
\proto{output-port?}{ obj}{procedure}}
   
Returns \schtrue{} if the argument is an output port (or a
combined input and output port), and returns \schfalse{} otherwise.
\end{entry}   

\begin{entry}{%
\proto{flush-output-port}{ output-port}{procedure}}
   
Flushes any output from the buffer of \var{output-port} to the
underlying file, device, or object. The unspecified value is returned.
\end{entry}

\begin{entry}{%
\proto{output-port-buffer-mode}{ output-port}{procedure}}
   
Returns the symbol that represents the buffer-mode of
\var{output-port}.
\end{entry}

\begin{entry}{%
\proto{open-file-output-port}{ filename}{procedure}
\rproto{open-file-output-port}{ filename file-options}{procedure}
\pproto{(open-file-output-port \var{filename}}{procedure}{\tt\obeyspaces%
\hspace*{2em}\var{file-options} \var{buffer-mode})}\\
\pproto{(open-file-output-port \var{filename}}{procedure}{\tt\obeyspaces%
\hspace*{2em}\var{file-options} \var{buffer-mode} \var{transcoder})}}

Returns an output port for the named file.

The \var{file-options} argument, which may determine
various aspects of the returned port (see section~\ref{fileoptionssection}),
defaults to {\cf (file-options)}.

The \var{buffer-mode} argument, if supplied,
must be one of the symbols that name a buffer mode.
The \var{buffer-mode} argument defaults to {\cf block}.

If \var{transcoder} is specified, it becomes the transcoder associated
with the returned port.
If no \var{transcoder} is specified, then an implementation-dependent
(and possibly locale-dependent) transcoder is associated with the port.

If the binary transcoder is passed as an explicit argument,
then the port will be a binary port and will support the
{\cf port-position} and {\cf set-port-position!}  operations.
Otherwise the port will be a textual port, and whether it supports
the {\cf port-position} and {\cf set-port-position!} operations
will be implementation-dependent (and possibly transcoder-dependent).

\begin{rationale}
  The byte position of a complexly transcoded port may not be
  well-defined, and may be hard to calculate even when defined,
  especially when transcoding is buffered.
\end{rationale}
\end{entry}   

\begin{entry}{%
\proto{open-bytevector-output-port}{}{procedure}
\rproto{open-bytevector-output-port}{ transcoder}{procedure}}

Returns an output port that accumulates a bytevector from the
output written to it.  If \var{transcoder} is specified, it becomes
the transcoder associated with the returned port.

If no \var{transcoder} argument is given, or
if the binary transcoder is passed as an explicit argument,
then the port will be a binary port and will support the
{\cf port-position} and {\cf set-port-position!}  operations.
Otherwise the port will be a textual port, and whether it supports
the {\cf port-position} and {\cf set-port-position!} operations
will be implementation-dependent (and possibly transcoder-dependent).
\end{entry}

\begin{entry}{%
\proto{call-with-bytevector-output-port}{ proc}{procedure}
\rproto{call-with-bytevector-output-port}{ proc transcoder}{procedure}}

\domain{\var{Proc} must be a procedure accepting one argument.}
Creates an output port that accumulates a bytevector from the output
written to it, and calls \var{proc} with that output port as an
argument. When \var{proc} returns for the first time, the port is
closed and the bytevector associated with the port is returned.  If
\var{transcoder} is specified, it becomes the transcoder associated
with the port.

The transcoder associated with the output port is determined
as for a call to {\cf open-bytevector-output-port}.
\end{entry}

\begin{entry}{%
\proto{get-output-bytevector}{ output-port}{procedure}}

\domain{\var{Output-port} must be a port that accumulates a bytevector
  from the output written to it, such as the ports created by
  {\cf open-bytevector-output-port} and {\cf call-with-bytevector-output-port}.}
Returns a newly created bytevector containing the output that has
been accumulated by \var{output-port} so far.
If the returned bytevector is modified,
the effect on \var{output-port} is unspecified.
\end{entry}

\begin{entry}{%
\proto{clear-bytevector-output-port!}{ output-port}{procedure}}

\domain{\var{Output-port} must be a port that accumulates a bytevector
  from the output written to it, such as the ports created by
{\cf open-bytevector-output-port} and {\cf call-with-bytevector-output-port}.}
The {\cf clear-bytevector-output-port!} procedure removes all bytes that
have been accumulated by \var{output-port} so far.
\end{entry}

\begin{entry}{%
\proto{open-string-output-port}{ proc}{procedure}}

Returns a textual output port that accumulates a string from the
characters written to it.
The port's transcoder is implementation-dependent,
but is not the binary transcoder.
Whether the port supports
the {\cf port-position} and {\cf set-port-position!} operations
is implementation-dependent.
\end{entry}

\begin{entry}{%
\proto{call-with-string-output-port}{ proc}{procedure}}

\domain{\var{Proc} must be a procedure accepting one argument.}
Creates a textual output port that accumulates a string from the
characters written to it, and calls \var{proc} with that output port
as an argument. When \var{proc} returns for the first time, the port is
closed and a newly created string consisting of the accumulated
characters is returned.  If
\var{transcoder} is specified, it becomes the transcoder associated
with the port.

The transcoder associated with the textual output port is
implementation-dependent, but is not the binary transcoder.
Whether the port supports
the {\cf port-position} and {\cf set-port-position!} operations
is implementation-dependent.
\end{entry}

\begin{entry}{%
\proto{get-output-string}{ textual-output-port}{procedure}}

\domain{\var{Textual-output-port} must be a textual output port
that accumulates characters, such as the ports created by
  {\cf open-string-output-port} and {\cf call-with-string-output-port}.}
Returns a newly created string consisting of the characters that
have been accumulated by the \var{textual-output-port} so far.

If the returned string is modified, 
the effect on \var{textual-output-port} is unspecified.
\end{entry}

\begin{entry}{%
\proto{clear-string-output-port!}{ textual-output-port}{procedure}}

\domain{\var{Textual-output-port} must be a port that accumulates a string
 from the characters written to it, such as the ports created by
{\cf open-string-output-port} and {\cf call-with-string-output-port}.}
The {\cf clear-string-output-port!} procedure removes all characters
that have been accumulated by \var{textual-output-port} so far.
\end{entry}

\begin{entry}{%
\proto{standard-output-port}{}{procedure}
\proto{standard-error-port}{}{procedure}}
   
Returns an output port connected to the standard output or standard error,
respectively, possibly a fresh one on each call.
Whether the port is binary or textual is implementation-dependent,
as is whether the port supports
the {\cf port-position} and {\cf set-port-position!} operations.

\begin{note}
  Implementations are encouraged to return a textual port when
  appropriate, and
  to associate an appropriate transcoder with the port.
\end{note}
\end{entry}

\begin{entry}{%
\pproto{(make-custom-binary-output-port \var{id}}{procedure}}
\mainschindex{make-custom-binary-output-port}{\tt\obeyspaces\\
  \var{write!} \var{get-position} \var{set-position!} \var{close})}

Returns a newly created binary output port whose byte sink is
an arbitrary algorithm represented by the \var{write!} procedure.
\var{Id} must be a string naming the new port,
provided for informational purposes only.
\var{Write!} must be a procedure, and should behave as specified
below; it will be called by operations that perform binary output.

Each of the remaining arguments may be \schfalse{}; if any of
those arguments is not \schfalse{}, it must be a procedure and
should behave as specified in the description of
{\cf make-custom-binary-input-port}.
   
\begin{itemize}
\item {\cf (\var{write!} \var{bytevector} \var{start} \var{count})}
       
  \var{Start} and \var{count} will be non-negative exact integers,
  and \var{bytevector} will be a bytevector whose length is at least
  $\var{start} + \var{count}$.
  The \var{write!} procedure reads up to \var{count} bytes
  from \var{bytevector} starting at index \var{start}, and forward
  them to the byte sink.
  If \var{count} is 0, then the \var{write!} procedure should
  have the effect of passing an end-of-file object to the byte sink.
  In any case, the \var{write!} procedure must return the number of
  bytes that it reads, as an exact integer.
\end{itemize}

\end{entry}

\subsection{Binary output}

\begin{entry}{%
\proto{put-u8}{ binary-output-port octet}{procedure}}

Writes \var{octet} to the output port and returns the unspecified value.
\end{entry}

\begin{entry}{%
\proto{put-bytevector}{ binary-output-port bytevector}{procedure}
\rproto{put-bytevector}{ binary-output-port bytevector start}{procedure}
\pproto{(put-bytevector \var{binary-output-port}}{procedure}}
{\tt\obeyspaces\\
     \var{bytevector} \var{start} \var{count})}
   
\domain{\var{Start} and \var{count} must be non-negative exact
  integers that default to 0 and $\texttt{(bytevector-length \var{bytevector})}
  - \var{start}$, respectively. \var{bytevector} must have a size of at
  least $\var{start} + \var{count}$.}  The {\cf put-bytevector} procedure writes
\var{count} bytes of the bytevector \var{bytevector}, starting at index
\var{start}, to the output port. The unspecified value is returned.
\end{entry}

\subsection{Textual output}

\begin{entry}{%
\proto{put-char}{ textual-output-port char}{procedure}}
   
Writes \var{char} to the port. The
unspecified value is returned.
\end{entry}

\begin{entry}{%
\proto{put-string}{ textual-output-port string}{procedure}}
   
Writes the characters of \var{string} to the port.
The unspecified value is returned.
\end{entry}   

\begin{entry}{%
\proto{put-string-n}{ textual-output-port string}{procedure}
\rproto{put-string-n}{ textual-output-port string start}{procedure}
\rproto{put-string-n}{ textual-output-port string start count}{procedure}}
   
\domain{\var{Start} and \var{count} must be non-negative exact
  integers.  \var{String} must have a length of at least $\var{start} +
  \var{count}$.}  \var{Start} defaults to 0.  \var{Count} defaults to
$\texttt{(string-length \var{string})} - \var{start}$. Writes the
\var{count} characters of \var{string}, starting at
index \var{start}, to the port. The unspecified value is returned.
\end{entry}


\begin{entry}{%
\proto{put-datum}{ textual-output-port datum}{procedure}}

\domain{\var{Datum} should be a datum value.}
The {\cf put-datum} procedure writes an external representation of
\var{datum} to \var{textual-output-port}.
The specific external representation is implementation-dependent.

\begin{note}
  The {\cf put-datum} procedure merely writes the external
  representation, but no trailing delimiter.  If {\cf put-datum} is
  used to write several subsequent external representations to an
  output port, care must be taken to delimit them properly so they can
  be read back in by subsequent calls to {\cf get-datum}.
\end{note}
\end{entry}


\subsection{Input/output ports}

\begin{entry}{%
\proto{open-file-input/output-port}{ filename}{procedure}
\rproto{open-file-input/output-port}{ filename file-options}{procedure}
\pproto{(open-file-input/output-port \var{filename}}{procedure}{\tt\obeyspaces%
\hspace*{2em}\var{file-options} \var{buffer-mode})}\\
\pproto{(open-file-input/output-port \var{filename}}{procedure}{\tt\obeyspaces%
\hspace*{2em}\var{file-options} \var{buffer-mode} \var{transcoder})}}
   
Returns a single port that is both an input port and an
output port for the named file.
The optional arguments default as described in the specification
of {\cf open-file-output-port}.
If the input/output port supports {\cf port-position} and/or
{\cf set-port-position!}, then the same port position is used
for both input and output.
\end{entry}

\begin{entry}{%
\pproto{(make-custom-binary-input/output-port}{procedure}}
\mainschindex{make-custom-binary-input/output-port}{\tt\obeyspaces\\
  \var{id} \var{read!} \var{write!} \var{get-position} \var{set-position!} \var{close})}

Returns a newly created binary input/output port whose
byte source and sink are
arbitrary algorithms represented by the \var{read!} and \var{write!}
procedures.
\var{Id} must be a string naming the new port,
provided for informational purposes only.
\var{Read!} and \var{write!} must be procedures,
and should behave as specified for the
{\cf make-custom-binary-input-port} and
{\cf make-custom-binary-output-port} procedures.

Each of the remaining arguments may be \schfalse{}; if any of
those arguments is not \schfalse{}, it must be a procedure and
should behave as specified in the description of
{\cf make-custom-binary-input-port}.
\end{entry}


%%% Local Variables: 
%%% mode: latex
%%% TeX-master: "r6rs-lib"
%%% End: 

\section{Writing and reading external representations}
\label{datumiosection}

These procedures convert datums to their external representations and
vice versa.  See section~\ref{readsyntaxchapter}.

FIXME: separate library

\begin{entry}{%
\proto{put-datum}{ output-port datum}{procedure}
\rproto{put-datum}{ output-port datum transcoder}{procedure}}

Writes an external representation of \var{datum} to \var{output-port}.
Which external representation is chosen is implementation-dependent.

\begin{note}
  The {\cf put-datum} procedure merely writes the external
  reprentation.  If {\cf put-datum} is used to write several
  subsequent external representations to an output port, care must be
  taken to delimit them properly so they can be read back in by
  subsequent calls to {\cf get-datum}.
\end{note}
\end{entry}

\begin{entry}{%
\proto{get-datum}{ input-port}{procedure}
\rproto{get-datum}{ input-port transcoder}{procedure}}
 
Reads an external representation from \var{input-port} and returns the
datum it represents.  The {\cf get-datum} procedure returns the next
datum parsable from the given \var{input-port}, updating
\var{input-port} to point exactly past the end of the external
representation of the object.

Any \meta{intertoken space} (see section~\ref{lexicalsyntaxsection}) in
the input is first skipped.  If an end of file occurs after the
\meta{intertoken space}, the end of file object (see
section~\ref{eofsection}) is returned, the port remains open, and
further calls to {\cf get-datum} will also return the end of file
object.

If a character inconsistent with an external representation is
encountered in the input, an exception with condition type
{\cf\&lexical} is raised.  Also, if the end of file is encountered
after the beginning of an external representation, but the external
representation is incomplete and therefore not parsable, an exception
with condition type {\cf\&lexical} is raised.
\end{entry}

%%% Local Variables: 
%%% mode: latex
%%% TeX-master: "r6rs"
%%% End: 

\section{Simple I/O}

This section describes the \library{r6rs i/o simple} library, which
provides a somewhat more convenient interface for performing textual
I/O on ports.  This library implements most of the 
I/O procedures of the previous version of this report~\cite{R5RS}.

\begin{entry}{%
\proto{call-with-input-file}{ filename proc}{procedure}
\proto{call-with-output-file}{ filename proc}{procedure}}

\domain{\var{Proc} must be a procedure accepting a single argument.}
These procedures open the file named by \var{filename} for input or
for output, with no specified file options, and call \var{proc} with
the obtained port as an argument.  If \var{proc} returns, then the
port is closed automatically and the values returned by \var{proc} are
returned. If \var{proc} does not return, then the port will not be
closed automatically, unless it is possible to prove that the port
will never again be used for an I/O operation.
\end{entry}

\begin{entry}{%
\rproto{input-port?}{ obj}{procedure}
\rproto{output-port?}{ obj}{procedure}}

These are the same as the {\cf input-port?} and {\cf output-port?}
procedures in the \library{r6rs i/o ports} library.
\end{entry}

\begin{entry}{%
\proto{current-input-port}{}{procedure}
\proto{current-output-port}{}{procedure}}

These return default ports for a input and output.  Normally, these
default ports are associated with standard input and standard output,
respectively, but can be dynamically re-assigned using the {\cf
  with-input-from-file} and {\cf with-output-to-file} procedures
described below.
\end{entry}

\begin{entry}{%
\proto{with-input-from-file}{ filename thunk}{procedure}
\proto{with-output-to-file}{ filename thunk}{procedure}}

\domain{\var{Thunk} must a procedure that takes no arguments.}  The
file is opened for input or output using empty file options, and
\var{thunk} is called with no arguments.  During the dynamic extent of
the call to \var{thunk}, the obtained port is made the value returned
by {\cf current-input-port} or {\cf current-output-port} procedures;
the previous default values are reinstated when the dynamic extent is
exited.  When \var{thunk} returns, the port is closed automatically,
and the previous values for {\cf current-input-port}.  The values
returned by \var{thunk} are returned.  If an escape procedure is used
to escape back into the call to \var{thunk} after \var{thunk} is
returned, the behavior is unspecified.
\end{entry}

\begin{entry}{%
\proto{open-input-file}{ filename}{procedure}}

This opens \var{filename} for input, with empty file options, and returns
the obtained port.
\end{entry}

\begin{entry}{%
\proto{open-output-file}{ filename}{procedure}}

This opens \var{filename} for output, with empty file options, and
returns the obtained port.
\end{entry}

\begin{entry}{%
\proto{close-input-port}{ input-port}{procedure}
\proto{close-output-port}{ output-port}{procedure}}

This closes \var{input-port} or \var{output-port}, respectively.
\end{entry}

\begin{entry}{%
\proto{read-char}{}{procedure}
\rproto{read-char}{ input-port}{procedure}}

This reads from \var{input-port} using the transcoder assocated with
it, blocking as necessary, until the complete encoding for a character
is available from input-port, or the bytes that are available cannot
be the prefix of any valid encoding, or an end of file is reached.

If a complete character is available before the next end of file, {\cf
  read-char} returns that character, and updates the input port to
point past the bytes that encoded that character. If an end of file is
reached before any bytes are read, then {\cf read-char} returns the
end-of-file object.

If \var{input-port} is omitted, it defaults to the value returned by
{\cf current-input-port}.
\end{entry}

\begin{entry}{%
\rproto{peek-char}{}{procedure}
\proto{peek-char}{ input-port}{procedure}}
   
This is the same as {\cf read-char}, but does not consume any data
from the port.
\end{entry}

\begin{entry}{%
\rproto{read}{}{procedure}
\proto{read}{ input-port}{procedure}}

Reads an external representation from \var{input-port} using the
transcoder associated with \var{input-port} and returns the datum it
represents.  The {\cf read} procedure operates in the same way as 
{\cf get-datum}, see section~\ref{get-datum}.

If \var{input-port} is omitted, it defaults to the value returned by
{\cf current-input-port}.
\end{entry}

\begin{entry}{%
\proto{write-char}{ char}{procedure}
\rproto{write-char}{ char output-port}{procedure}}

Writes an encoding of the character \var{char} to the port using the
transcoder associated with \var{output-port}. The unspecified value is
returned.

If \var{output-port} is omitted, it defaults to the value returned by
{\cf current-output-port}.
\end{entry}

\begin{entry}{%
\proto{newline}{}{procedure}
\rproto{newline}{ output-port}{procedure}}

This is equivalent to using {\cf write-char} to write {\cf \#linefeed}
to \var{output-port} using the transcoder associated with \var{output-port}.

If \var{output-port} is omitted, it defaults to the value returned by
{\cf current-output-port}.
\end{entry}

\begin{entry}{%
\proto{display}{ obj}{procedure}
\rproto{display}{ obj output-port}{procedure}}

Writes a representation of \var{obj} to the given \var{port} using the
transcoder associated with \var{output-port}.  Strings that appear in
the written representation are not enclosed in doublequotes, and no
characters are escaped within those strings.  Character objects appear
in the representation as if written by {\cf write-char} instead of by
{\cf write}.  {\cf Display} returns the unspecified value.  The
\var{output-port} argument may be omitted, in which case it defaults
to the value returned by {\cf current-output-port}.
\end{entry}

\begin{entry}{%
\proto{write}{obj }{procedure}
\rproto{write}{ obj output-port}{procedure}}

Writes the external representation of \var{obj} to \var{output-port}
using the transcoder associated with \var{input-port}.  The {\cf write}
procedure operates in the same way as {\cf put-datum}, see
section~\ref{put-datum}.

If \var{output-port} is omitted, it defaults to the value returned by
{\cf current-output-port}.
\end{entry}


%%% Local Variables: 
%%% mode: latex
%%% TeX-master: "r6rs"
%%% End: 


%%% Local Variables: 
%%% mode: latex
%%% TeX-master: "r6rs"
%%% End: 
      \par
\chapter{File system}
\label{filesystemchapter}

This chapter describes the \defrsixlibrary{files} library for
operations on the file system.  This library, in addition to the
procedures described here, also exports the I/O condition types
described in section~\ref{iocondsection}.

\begin{entry}{%
\proto{file-exists?}{ filename}{procedure}}

\domain{\var{Filename} must be a filename (see
  section~\ref{filenamesection}).}  The {\cf file-exists?} procedure
returns \schtrue{} if the named file exists at the time the procedure
is called, \schfalse{} otherwise.
\end{entry}

\begin{entry}{%
\proto{delete-file}{ filename}{procedure}}

\domain{\var{Filename} must be a filename (see
  section~\ref{filenamesection}).}  The {\cf delete-file} procedure
deletes the named file if it exists and can be deleted, and returns
\unspecifiedreturn.  If the file does not exist or cannot be deleted,
an exception with condition type {\cf\&i/o-filename} is raised.
\end{entry}


%%% Local Variables: 
%%% mode: latex
%%% TeX-master: "r6rs-lib"
%%% End: 
   \par
\chapter{Arithmetic}
\label{numberchapter}
\index{number}

This chapter describes Scheme's libraries for more specialized
numerical operations: fixnum and flonum arithmetic, as well as bitwise
operations on exact integer objects.  

\section{Bitwise operations}

A number of procedures operate on the binary two's-complement
representations of exact integer objects: Bit positions within an
exact integer object are counted from the right, i.e.\ bit 0 is the
least significant bit.  Some procedures allow extracting \defining{bit
  fields}, i.e., number objects representing subsequences of the
binary representation of an exact integer object.  Bit fields are
always positive, and always defined using a finite number of bits.

\section{Fixnums}
\label{fixnumssection}

Every implementation must define its fixnum range as a closed
interval
%
\begin{displaymath}
[-2^{w-1}, 2^{w-1} - 1]
\end{displaymath}
%
such that $w$ is a (mathematical) integer $w \geq 24$.  Every
mathematical integer within an implementation's fixnum range must
correspond to an exact integer object that is representable within the
implementation.
A fixnum is an exact integer object whose value lies within this
fixnum range.

This section describes the \defrsixlibrary{arithmetic fixnums} library,
which defines various operations on fixnums.
Fixnum operations perform integer arithmetic on their fixnum
arguments, but raise an exception with condition type
{\cf\&implementation-restriction} if the result is not a fixnum.

This section uses \var{fx}, \vari{fx}, \varii{fx}, etc., as parameter
names for arguments that must be fixnums.

\begin{entry}{%
\rproto{fixnum?}{ obj}{procedure}}

Returns \schtrue{} if \var{obj} is an exact
integer object within the fixnum range, \schfalse{} otherwise.
\end{entry}

\begin{entry}{%
\rproto{fixnum-width}{}{procedure}
\rproto{least-fixnum}{}{procedure}
\rproto{greatest-fixnum}{}{procedure}}

These procedures return $w$,
$-2^{w-1}$ and $2^{w-1} - 1$: the
width, minimum and the maximum value of the fixnum range, respectively.
\end{entry}

\begin{entry}{%
\proto{fx=?}{ \vari{fx} \varii{fx} \variii{fx} \dotsfoo}{procedure}
\proto{fx>?}{ \vari{fx} \varii{fx} \variii{fx} \dotsfoo}{procedure}
\proto{fx<?}{ \vari{fx} \varii{fx} \variii{fx} \dotsfoo}{procedure}
\proto{fx>=?}{ \vari{fx} \varii{fx} \variii{fx} \dotsfoo}{procedure}
\proto{fx<=?}{ \vari{fx} \varii{fx} \variii{fx} \dotsfoo}{procedure}}

These procedures return \schtrue{} if their arguments are (respectively):
equal, monotonically increasing, monotonically decreasing,
monotonically nondecreasing, or monotonically nonincreasing,
\schfalse{} otherwise.
\end{entry}

\begin{entry}{%
\proto{fxzero?}{ fx}{procedure}
\proto{fxpositive?}{ fx}{procedure}
\proto{fxnegative?}{ fx}{procedure}
\proto{fxodd?}{ fx}{procedure}
\proto{fxeven?}{ fx}{procedure}}

These numerical predicates test a fixnum for a particular property,
returning \schtrue{} or \schfalse{}.  The five properties tested by
these procedures are: whether the number object is zero, greater than zero,
less than zero, odd, or even.
\end{entry}

\begin{entry}{%
\proto{fxmax}{ \vari{fx} \varii{fx} \dotsfoo}{procedure}
\proto{fxmin}{ \vari{fx} \varii{fx} \dotsfoo}{procedure}}

These procedures return the maximum or minimum of their arguments.
\end{entry}

\begin{entry}{%
\proto{fx+}{ \vari{fx} \varii{fx}}{procedure}
\proto{fx*}{ \vari{fx} \varii{fx}}{procedure}}

These procedures return the sum or product of their arguments,
provided that sum or product is a fixnum.  An exception with condition
type {\cf\&implementation-restriction} is raised if
that sum or product is not a fixnum.
\end{entry}

\begin{entry}{%
\proto{fx-}{ \vari{fx} \varii{fx}}{procedure}
\rproto{fx-}{ fx}{procedure}}

With two arguments, this procedure returns the difference
$\vari{fx}-\varii{fx}$, provided that difference is a fixnum.

With one argument, this procedure returns the additive
inverse of its argument, provided that integer object is a
fixnum.

An exception with condition type {\cf\&implementation-restriction} is raised if the
mathematically correct result of this procedure is not a fixnum.

\begin{scheme}
(fx- (least-fixnum))  \xev  \exception{\&assertion}%
\end{scheme}
\end{entry}

\begin{entry}{%
\proto{fxdiv-and-mod}{ \vari{fx} \varii{fx}}{procedure}
\proto{fxdiv}{ \vari{fx} \varii{fx}}{procedure}
\proto{fxmod}{ \vari{fx} \varii{fx}}{procedure}
\proto{fxdiv0-and-mod0}{ \vari{fx} \varii{fx}}{procedure}
\proto{fxdiv0}{ \vari{fx} \varii{fx}}{procedure}
\proto{fxmod0}{ \vari{fx} \varii{fx}}{procedure}}

\domain{\varii{Fx} must be nonzero.}
These procedures implement number-theoretic integer division and
return the results of the corresponding mathematical operations
specified in report section~\extref{report:integerdivision}{Integer division}.

\begin{scheme}
(fxdiv \vari{fx} \varii{fx})         \ev \(\vari{fx}~\mathrm{div}~\varii{fx}\)
(fxmod \vari{fx} \varii{fx})         \ev \(\vari{fx}~\mathrm{mod}~\varii{fx}\)
(fxdiv-and-mod \vari{fx} \varii{fx})     \lev \(\vari{fx}~\mathrm{div}~\varii{fx}, \vari{fx}~\mathrm{mod}~\varii{fx}\)\\\>\>; \textrm{two return values}
(fxdiv0 \vari{fx} \varii{fx})        \ev \(\vari{fx}~\mathrm{div}\sb{0}~\varii{fx}\)
(fxmod0 \vari{fx} \varii{fx})        \ev \(\vari{fx}~\mathrm{mod}\sb{0}~\varii{fx}\)
(fxdiv0-and-mod0 \vari{fx} \varii{fx})   \lev \(\vari{fx} \vari{fx}~\mathrm{div}\sb{0}~\varii{fx}, \vari{fx}~\mathrm{mod}\sb{0}~\varii{fx}\)\\\>\>; \textrm{two return values}%
\end{scheme}
\end{entry}

\begin{entry}{%
\proto{fx+/carry}{ \vari{fx} \varii{fx} \variii{fx}}{procedure}}

Returns the two fixnum results of the following computation:
%
\begin{scheme}
(let* ((s (+ \vari{fx} \varii{fx} \variii{fx}))
       (s0 (mod0 s (expt 2 (fixnum-width))))
       (s1 (div0 s (expt 2 (fixnum-width)))))
  (values s0 s1))%
\end{scheme}
\end{entry}

\begin{entry}{%
\proto{fx-/carry}{ \vari{fx} \varii{fx} \variii{fx}}{procedure}}

Returns the two fixnum results of the following computation:
%
\begin{scheme}
(let* ((d (- \vari{fx} \varii{fx} \variii{fx}))
       (d0 (mod0 d (expt 2 (fixnum-width))))
       (d1 (div0 d (expt 2 (fixnum-width)))))
  (values d0 d1))%
\end{scheme}
\end{entry}

\begin{entry}{%
\proto{fx*/carry}{ \vari{fx} \varii{fx} \variii{fx}}{procedure}}

Returns the two fixnum results of the following computation:
\begin{scheme}
(let* ((s (+ (* \vari{fx} \varii{fx}) \variii{fx}))
       (s0 (mod0 s (expt 2 (fixnum-width))))
       (s1 (div0 s (expt 2 (fixnum-width)))))
  (values s0 s1))%
\end{scheme}
\end{entry}

\begin{entry}{%
\proto{fxnot}{ \var{fx}}{procedure}}

Returns the unique fixnum that is congruent
mod $2^w$ to the one's-complement of \var{fx}.
\end{entry}

\begin{entry}{%
\proto{fxand}{ \vari{fx} \dotsfoo}{procedure}
\proto{fxior}{ \vari{fx} \dotsfoo}{procedure}
\proto{fxxor}{ \vari{fx} \dotsfoo}{procedure}}

These procedures return the fixnum that is the bit-wise ``and'',
``inclusive or'', or ``exclusive or'' of the two's complement
representations of their arguments.  If they are passed only one
argument, they return that argument.  If they are passed no arguments,
they return the fixnum (either $-1$ or $0$) that acts as identity for the
operation.
\end{entry}

\begin{entry}{%
\proto{fxif}{ \vari{fx} \varii{fx} \variii{fx}}{procedure}}

Returns the fixnum that is the bit-wise ``if'' of the two's complement
representations of its arguments, i.e.\ for each bit, if it is 1 in
\vari{fx}, the corresponding bit in \varii{fx} becomes the value of
the corresponding bit in the result, and if it is 0, the corresponding
bit in \variii{fx} becomes the corresponding bit in the value of the
result.  This is the fixnum result of the following computation:
\begin{scheme}
(fxior (fxand \vari{fx} \varii{fx})
       (fxand (fxnot \vari{fx}) \variii{fx}))%
\end{scheme}
\end{entry}

\begin{entry}{%
\proto{fxbit-count}{ \var{fx}}{procedure}}

If \var{fx} is non-negative, this procedure returns the
number of 1 bits in the two's complement representation of \var{fx}.
Otherwise it returns the result of the following computation:
%
\begin{scheme}
(fxnot (fxbit-count (fxnot \var{ei})))%
\end{scheme}
\end{entry}

\begin{entry}{%
\proto{fxlength}{ \var{fx}}{procedure}}

Returns the number of bits needed to represent \var{fx} if it is
positive, and the number of bits needed to represent {\cf (fxnot
  \var{fx})} if it is negative, which is the fixnum result of the
following computation:
\begin{scheme}
(do ((result 0 (+ result 1))
     (bits (if (fxnegative? \var{fx})
               (fxnot \var{fx})
               \var{fx})
           (fxarithmetic-shift-right bits 1)))
    ((fxzero? bits)
     result))%
\end{scheme}
\end{entry}

\begin{entry}{%
\proto{fxfirst-bit-set}{ \var{fx}}{procedure}}

Returns the index of the least significant $1$ bit in
the two's complement representation of \var{fx}.  If 
\var{fx} is $0$, then $-1$ is returned.
%
\begin{scheme}
(fxfirst-bit-set 0)        \ev  -1
(fxfirst-bit-set 1)        \ev  0
(fxfirst-bit-set -4)       \ev  2%
\end{scheme}
\end{entry}

\begin{entry}{%
\proto{fxbit-set?}{ \vari{fx} \varii{fx}}{procedure}}

\domain{\varii{Fx} must be non-negative and less than {\cf
    (fixnum-width)}.}  The {\cf fxbit-set?} procedure returns
\schtrue{} if the \varii{fx}th bit is 1 in the two's complement
representation of \vari{fx}, and \schfalse{} otherwise.  This is the
fixnum result of the following computation:
%
\begin{scheme}
(not
  (fxzero?
    (fxand \vari{fx}
           (fxarithmetic-shift-left 1 \varii{fx}))))%
\end{scheme}
%
\end{entry}

\begin{entry}{%
\proto{fxcopy-bit}{ \vari{fx} \varii{fx} \variii{fx}}{procedure}}

\domain{\varii{Fx} must be non-negative and less than {\cf
  (fixnum-width)}. \variii{Fx} must be 0 or
1.}  The {\cf fxcopy-bit} procedure returns the result of replacing
the \varii{fx}th bit of \vari{fx} by \variii{fx}, which is
the result of the following computation:
\begin{scheme}
(let* ((mask (fxarithmetic-shift-left 1 \varii{fx})))
  (fxif mask
        (fxarithmetic-shift-left \variii{fx} \varii{fx})
        \vari{fx}))%
\end{scheme}
%
\end{entry}

\begin{entry}{%
\proto{fxbit-field}{ \vari{fx} \varii{fx} \variii{fx}}{procedure}}

\domain{\varii{Fx} and \variii{fx} must be non-negative and less than
  {\cf (fixnum-width)}.  Moreover, \varii{fx} must be less than or
  equal to \variii{fx}.}  The {\cf fxbit-field} procedure returns the
number represented by the bits at the positions from \varii{fx} (inclusive) to
$\variii{fx}$ (exclusive), which is
the fixnum result of the following computation:
%
\begin{scheme}
(let* ((mask (fxnot
              (fxarithmetic-shift-left -1 \variii{fx}))))
  (fxarithmetic-shift-right (fxand \vari{fx} mask)
                            \varii{fx}))%
\end{scheme}
%
\end{entry}

\begin{entry}{%
\proto{fxcopy-bit-field}{ \vari{fx} \varii{fx} \variii{fx} \variv{fx}}{procedure}}

\domain{\varii{Fx} and \variii{fx} must be non-negative and less than
  {\cf (fixnum-width)}.  Moreover, \varii{fx} must be less than or
  equal to \variii{fx}.}  The {\cf fxcopy-bit-field} procedure returns
the result of replacing in \vari{fx} the bits at positions from
\varii{fx} (inclusive) to $\variii{fx}$ (exclusive) by the bits in
\variv{fx} from position 0 (inclusive) to position
$\variii{fx}-\varii{fx}$ (exclusive), which
is the fixnum result of the following computation:
\begin{scheme}
(let* ((to    \vari{fx})
       (start \varii{fx})
       (end   \variii{fx})
       (from  \variv{fx})
       (mask1 (fxarithmetic-shift-left -1 start))
       (mask2 (fxnot
               (fxarithmetic-shift-left -1 end)))
       (mask (fxand mask1 mask2)))
  (fxif mask
        (fxarithmetic-shift-left from start)
        to))%
\end{scheme}

\begin{scheme}
(fxcopy-bit-field \sharpsign{}b0000001 2 5 \sharpsign{}b1111000) \lev 1
(fxcopy-bit-field \sharpsign{}b0000001 2 5 \sharpsign{}b0001111) \lev 29
(fxcopy-bit-field \sharpsign{}b0001111 2 5 \sharpsign{}b0001111) \lev 31%
\end{scheme}
\end{entry}

\begin{entry}{%
\proto{fxarithmetic-shift}{ \vari{fx} \varii{fx}}{procedure}}

\domain{The absolute value of \varii{fx} must be less than 
{\cf (fixnum-width)}.}  If
%
\begin{scheme}
(floor (* \vari{fx} (expt 2 \varii{fx})))%
\end{scheme}
%
is a fixnum, then that fixnum is returned.  Otherwise an exception
with condition type {\cf\&implementation-\hp{}restriction} is
raised.
\end{entry}

\begin{entry}{%
\proto{fxarithmetic-shift-left}{ \vari{fx} \varii{fx}}{procedure}
\proto{fxarithmetic-shift-right}{ \vari{fx} \varii{fx}}{procedure}}

\domain{\varii{Fx} must be non-negative, and less than {\cf
    (fixnum-width)}.}
  The {\cf fxarithmetic-shift-left} procedure behaves the same as {\cf
  fxarithmetic-shift}, and {\cf (fxarithmetic-shift-right \vari{fx}
  \varii{fx})} behaves the same as {\cf (fxarithmetic-shift \vari{fx}
  (fx- \varii{fx}))}.
\end{entry}

\begin{entry}{%
\proto{fxrotate-bit-field}{ \vari{fx} \varii{fx} \variii{fx} \variv{fx}}{procedure}}

\domain{\varii{Fx}, \variii{fx}, and \variv{fx} must be non-negative
  and less than {\cf (fixnum-width)}.  \varii{Fx} must be less than or
  equal to \variii{fx}. \variv{Fx} must be less than the difference
between \variii{fx} and \varii{fx}.}  The {\cf fxrotate-bit-field}
procedure returns the result of cyclically permuting in \vari{fx} the
bits at positions from \varii{fx} (inclusive) to \variii{fx}
(exclusive) by \variv{fx} bits
towards the more significant bits, which is the result of the
following computation:
\begin{scheme}
(let* ((n     \vari{fx})
       (start \varii{fx})
       (end   \variii{fx})
       (count \variv{fx})
       (width (fx- end start)))
  (if (fxpositive? width)
      (let* ((count (fxmod count width))
             (field0
               (fxbit-field n start end))
             (field1
               (fxarithmetic-shift-left
                 field0 count))
             (field2
               (fxarithmetic-shift-right
                 field0 (fx- width count)))
             (field (fxior field1 field2)))
        (fxcopy-bit-field n start end field))
      n))%
\end{scheme}

\end{entry}

\begin{entry}{%
\proto{fxreverse-bit-field}{ \vari{fx} \varii{fx} \variii{fx}}{procedure}}

\domain{\varii{Fx} and \variii{fx} must be non-negative and less than
  {\cf (fixnum-width)}.  Moreover, \varii{fx} must be less than or
  equal to \variii{fx}.}  The {\cf fxreverse-bit-field} procedure
returns
the fixnum obtained from \vari{fx} by reversing the
order of the bits at positions from \varii{fx} (inclusive) to
\variii{fx} (exclusive).
\begin{scheme}
(fxreverse-bit-field \sharpsign{}b1010010 1 4)    \lev  88 ; \sharpsign{}b1011000%
\end{scheme}

\end{entry}

\section{Flonums}
\label{flonumssection}

This section describes the \defrsixlibrary{arithmetic flonums} library.

This section uses \var{fl}, \vari{fl}, \varii{fl}, etc., as
parameter names for arguments that must be flonums, and \var{ifl}
as a name for arguments that 
must be integer-valued flonums, i.e., flonums for which the
{\cf integer-valued?} predicate returns true.

\begin{entry}{%
\proto{flonum?}{ obj}{procedure}}

Returns \schtrue{} if \var{obj} is a flonum, \schfalse{} otherwise.
\end{entry}

\begin{entry}{%
\proto{real->flonum}{ x}{procedure}}

Returns the best flonum representation of
\var{x}.

The value returned is a flonum that is numerically closest to the
argument.

\begin{note}
  If flonums are represented in binary floating point, then
  implementations should break ties by preferring
  the floating-point representation whose least significant bit is
  zero.
\end{note}
\end{entry}

\begin{entry}{%
\proto{fl=?}{ \vari{fl} \varii{fl} \variii{fl} \dotsfoo}{procedure}
\proto{fl<?}{ \vari{fl} \varii{fl} \variii{fl} \dotsfoo}{procedure}
\proto{fl<=?}{ \vari{fl} \varii{fl} \variii{fl} \dotsfoo}{procedure}
\proto{fl>?}{ \vari{fl} \varii{fl} \variii{fl} \dotsfoo}{procedure}
\proto{fl>=?}{ \vari{fl} \varii{fl} \variii{fl} \dotsfoo}{procedure}}

These procedures return \schtrue{} if their arguments are (respectively):
equal, monotonically increasing, monotonically decreasing,
monotonically nondecreasing, or monotonically nonincreasing,
\schfalse{} otherwise.  These
predicates must be transitive.

\begin{scheme}
(fl=? +inf.0 +inf.0)           \ev  \schtrue{}
(fl=? -inf.0 +inf.0)           \ev  \schfalse{}
(fl=? -inf.0 -inf.0)           \ev  \schtrue{}
(fl=? 0.0 -0.0)                \ev  \schtrue{}
(fl<? 0.0 -0.0)                \ev  \schfalse{}
(fl=? +nan.0 \var{fl})               \ev  \schfalse{}
(fl<? +nan.0 \var{fl})               \ev  \schfalse{}%
\end{scheme}
\end{entry}

\begin{entry}{%
\proto{flinteger?}{ fl}{procedure}
\proto{flzero?}{ fl}{procedure}
\proto{flpositive?}{ fl}{procedure}
\proto{flnegative?}{ fl}{procedure}
\proto{flodd?}{ ifl}{procedure}
\proto{fleven?}{ ifl}{procedure}
\proto{flfinite?}{ fl}{procedure}
\proto{flinfinite?}{ fl}{procedure}
\proto{flnan?}{ fl}{procedure}}

These numerical predicates test a flonum for a particular property,
returning \schtrue{} or \schfalse{}.
The {\cf flinteger?} procedure tests whether the number object is an integer,
{\cf flzero?} tests whether
it is {\cf fl=?} to zero, {\cf flpositive?} tests whether it is greater
than zero, {\cf flnegative?} tests whether it is less
than zero, {\cf flodd?} tests whether it is odd, 
{\cf fleven?} tests whether it is even,
{\cf flfinite?} tests whether it is not an infinity and not a NaN,
{\cf flinfinite?} tests whether it is an infinity, and
{\cf flnan?} tests whether it is a NaN.

\begin{scheme}
(flnegative? -0.0)   \ev \schfalse{}
(flfinite? +inf.0)   \ev \schfalse{}
(flfinite? 5.0)      \ev \schtrue{}
(flinfinite? 5.0)    \ev \schfalse{}
(flinfinite? +inf.0) \ev \schtrue{}%
\end{scheme}

\begin{note}
{\cf (flnegative? -0.0)} must return \schfalse{},
else it would lose the correspondence with
{\cf (fl< -0.0 0.0)}, which is \schfalse{}
according to IEEE 754~\cite{IEEE}.
\end{note}
\end{entry}

\begin{entry}{%
\proto{flmax}{ \vari{fl} \varii{fl} \dotsfoo}{procedure}
\proto{flmin}{ \vari{fl} \varii{fl} \dotsfoo}{procedure}}

These procedures return the maximum or minimum of their arguments.
They always return a NaN when one or more of the arguments is a NaN.
\end{entry}

\begin{entry}{%
\proto{fl+}{ \vari{fl} \dotsfoo}{procedure}
\proto{fl*}{ \vari{fl} \dotsfoo}{procedure}}

These procedures return the flonum sum or product of their flonum
arguments.  In general, they should return the flonum that best
approximates the mathematical sum or product.  (For implementations
that represent flonums using IEEE binary floating point, the
meaning of ``best'' is defined by the IEEE standards.)

\begin{scheme}
(fl+ +inf.0 -inf.0)      \ev  +nan.0
(fl+ +nan.0 \var{fl})          \ev  +nan.0
(fl* +nan.0 \var{fl})          \ev  +nan.0%
\end{scheme}
\end{entry}

\begin{entry}{%
\proto{fl-}{ \vari{fl} \varii{fl} \dotsfoo}{procedure}
\rproto{fl-}{ fl}{procedure}
\proto{fl/}{ \vari{fl} \varii{fl} \dotsfoo}{procedure}
\rproto{fl/}{ fl}{procedure}}

With two or more arguments, these procedures return the flonum
difference or quotient of their flonum arguments, associating to the
left.  With one argument, however, they return the additive or
multiplicative flonum inverse of their argument.  In general, they
should return the flonum that best approximates the mathematical
difference or quotient.  (For implementations that represent flonums
using IEEE binary floating point, the meaning of ``best'' is
reasonably well-defined by the IEEE standards.)

\begin{scheme}
(fl- +inf.0 +inf.0)      \ev  +nan.0%
\end{scheme}

For undefined quotients, {\cf fl/} behaves as specified by the
IEEE standards:

\begin{scheme}
(fl/ 1.0 0.0)  \ev +inf.0
(fl/ -1.0 0.0) \ev -inf.0
(fl/ 0.0 0.0)  \ev +nan.0%
\end{scheme}
\end{entry}

\begin{entry}{%
\proto{flabs}{ fl}{procedure}}

Returns the absolute value of \var{fl}.
\end{entry}

\begin{entry}{%
\proto{fldiv-and-mod}{ \vari{fl} \varii{fl}}{procedure}
\proto{fldiv}{ \vari{fl} \varii{fl}}{procedure}
\proto{flmod}{ \vari{fl} \varii{fl}}{procedure}
\proto{fldiv0-and-mod0}{ \vari{fl} \varii{fl}}{procedure}
\proto{fldiv0}{ \vari{fl} \varii{fl}}{procedure}
\proto{flmod0}{ \vari{fl} \varii{fl}}{procedure}}

These procedures implement number-theoretic integer division and
return the results of the corresponding mathematical operations
specified in report section~\extref{report:integerdivision}{Integer division}.  For zero divisors, these
procedures may return a NaN or some unspecified flonum.

\begin{scheme}
(fldiv \vari{fl} \varii{fl})         \ev \(\vari{fl}~\mathrm{div}~\varii{fl}\)
(flmod \vari{fl} \varii{fl})         \ev \(\vari{fl}~\mathrm{mod}~\varii{fl}\)
(fldiv-and-mod \vari{fl} \varii{fl})     \lev \(\vari{fl}~\mathrm{div}~\varii{fl}, \vari{fl}~\mathrm{mod}~\varii{fl}\)\\\>\>; \textrm{two return values}
(fldiv0 \vari{fl} \varii{fl})        \ev \(\vari{fl}~\mathrm{div}_0~\varii{fl}\)
(flmod0 \vari{fl} \varii{fl})        \ev \(\vari{fl}~\mathrm{mod}_0~\varii{fl}\)
(fldiv0-and-mod0 \vari{fl} \varii{fl})   \lev \(\vari{fl}~\mathrm{div}_0~\varii{fl}, \vari{fl}~\mathrm{mod}_0~\varii{fl}\)\\\>\>; \textrm{two return values}%
\end{scheme}

\end{entry}

\begin{entry}{%
\proto{flnumerator}{ fl}{procedure}
\proto{fldenominator}{ fl}{procedure}}

These procedures return the numerator or denominator of \var{fl}
as a flonum; the result is computed as if \var{fl} was represented as
a fraction in lowest terms.  The denominator is always positive.  The
denominator of 0.0 is defined to be 1.0.
%
\begin{scheme}
(flnumerator +inf.0)           \ev  +inf.0
(flnumerator -inf.0)           \ev  -inf.0
(fldenominator +inf.0)         \ev  1.0
(fldenominator -inf.0)         \ev  1.0
(flnumerator 0.75)             \ev  3.0 ; \textrm{probably}
(fldenominator 0.75)           \ev  4.0 ; \textrm{probably}%
\end{scheme}

Implementations should implement following behavior:

\begin{scheme}
(flnumerator -0.0)             \ev -0.0%
\end{scheme}
\end{entry}

\begin{entry}{%
\proto{flfloor}{ fl}{procedure}
\proto{flceiling}{ fl}{procedure}
\proto{fltruncate}{ fl}{procedure}
\proto{flround}{ fl}{procedure}}

These procedures return integral flonums for flonum arguments that are
not infinities or NaNs.  For such arguments, {\cf flfloor} returns the
largest integral flonum not larger than \var{fl}.  The {\cf flceiling}
procedure
returns the smallest integral flonum not smaller than \var{fl}.
The {\cf fltruncate} procedure returns the integral flonum closest to \var{fl} whose
absolute value is not larger than the absolute value of \var{fl}.
The {\cf flround} procedure returns the closest integral flonum to \var{fl},
rounding to even when \var{fl} represents a number halfway between two integers.

Although infinities and NaNs are not integer objects, these procedures return
an infinity when given an infinity as an argument, and a NaN when
given a NaN:

\begin{scheme}
(flfloor +inf.0)                       \ev  +inf.0
(flceiling -inf.0)                     \ev  -inf.0
(fltruncate +nan.0)                    \ev  +nan.0%
\end{scheme}
\end{entry}

\begin{entry}{%
\proto{flexp}{ fl}{procedure}
\proto{fllog}{ fl}{procedure}
\rproto{fllog}{ \vari{fl} \varii{fl}}{procedure}
\proto{flsin}{ fl}{procedure}
\proto{flcos}{ fl}{procedure}
\proto{fltan}{ fl}{procedure}
\proto{flasin}{ fl}{procedure}
\proto{flacos}{ fl}{procedure}
\proto{flatan}{ fl}{procedure}
\rproto{flatan}{ \vari{fl} \varii{fl}}{procedure}}

These procedures compute the usual transcendental functions.  
The {\cf flexp} procedure computes the base-$e$ exponential of \var{fl}.
The {\cf fllog} procedure with a single argument computes the natural logarithm of
\var{fl} (not the base ten logarithm); {\cf (fllog \vari{fl}
  \varii{fl})} computes the base-\varii{fl} logarithm of \vari{fl}.
The {\cf flasin}, {\cf flacos}, and {\cf flatan} procedures compute arcsine,
arccosine, and arctangent, respectively.  {\cf (flatan \vari{fl}
  \varii{fl})} computes the arc tangent of \vari{fl}/\varii{fl}.

See report
section~\extref{report:transcendentalfunctions}{Transcendental functions} for the underlying
mathematical operations.  In the event that these operations do not
yield a real result for the given arguments, the result may be a NaN,
or may be some unspecified flonum.

Implementations that use IEEE binary floating-point arithmetic 
should follow the relevant standards for these procedures.

\begin{scheme}
(flexp +inf.0)                \ev +inf.0
(flexp -inf.0)                \ev 0.0
(fllog +inf.0)                \ev +inf.0
(fllog 0.0)                   \ev -inf.0
(fllog -0.0)                  \ev \unspecified\\\>; \textrm{if -0.0 is distinguished}
(fllog -inf.0)                \ev +nan.0
(flatan -inf.0)               \lev -1.5707963267948965\\\>; \textrm{approximately}
(flatan +inf.0)               \lev 1.5707963267948965\\\>; \textrm{approximately}%
\end{scheme}
\end{entry}

\begin{entry}{%
\proto{flsqrt}{ fl}{procedure}}

Returns the principal square root of \var{fl}. For $-0.0$,
{\cf flsqrt} should return $-0.0$; for other negative arguments,
the result may be a NaN or some unspecified flonum.

\begin{scheme}
(flsqrt +inf.0)               \ev  +inf.0
(flsqrt -0.0)                 \ev  -0.0%
\end{scheme}
\end{entry}

\begin{entry}{%
\proto{flexpt}{ \vari{fl} \varii{fl}}{procedure}}

\domain{Either \vari{fl} should be non-negative, or, if \vari{fl} is
  negative, \varii{fl} should be an integer object.}
The {\cf flexpt} procedure returns \vari{fl} raised to the power \varii{fl}.  If \vari{fl} is
negative and \varii{fl} is not an integer object, the result may be a
NaN, or may be some unspecified flonum.  If \vari{fl} is zero, then
the result is zero.
\end{entry}

\begin{entry}{%
\ctproto{no-infinities}
\proto{make-no-infinities-violation}{ obj}{procedure}
\proto{no-infinities-violation?}{ obj}{procedure}
\ctproto{no-nans}
\proto{make-no-nans-violation}{ obj}{procedure}
\proto{no-nans-violation?}{ obj}{procedure}}

These condition types could be defined by the following code:

\begin{scheme}
(define-condition-type \&no-infinities
    \&implementation-restriction
  make-no-infinities-violation
  no-infinities-violation?)

(define-condition-type \&no-nans
    \&implementation-restriction
  make-no-nans-violation no-nans-violation?)%
\end{scheme}

These types describe that a program has executed an arithmetic
operations that is specified to return an infinity or a NaN,
respectively, on a Scheme implementation that is not able to represent
the infinity or NaN.  (See report section~\extref{report:infinitiesnanssection}{Representability of infinities and NaNs}.)
\end{entry}

\begin{entry}{%
\proto{fixnum->flonum}{ fx}{procedure}}

Returns a flonum that is numerically closest to \var{fx}.

\begin{note}
The result of this procedure may not be
numerically equal to \var{fx}, because the fixnum precision
may be greater than the flonum precision.
\end{note}
\end{entry}

\section{Exact bitwise arithmetic}
\label{exactsection}

This section describes the \defrsixlibrary{arithmetic bitwise}
library.  The exact bitwise arithmetic provides generic operations on
exact integer objects.  This section uses \var{ei}, \vari{ei}, \varii{ei}, etc.,
as parameter names that must be exact integer objects.


\begin{entry}{%
\proto{bitwise-not}{ ei}{procedure}}

Returns the exact integer object whose two's complement representation is the
one's complement of the two's complement representation of \var{ei}.
\end{entry}

\begin{entry}{%
\proto{bitwise-and}{ \vari{ei} \dotsfoo}{procedure}
\proto{bitwise-ior}{ \vari{ei} \dotsfoo}{procedure}
\proto{bitwise-xor}{ \vari{ei} \dotsfoo}{procedure}}

These procedures return the exact integer object that is the bit-wise
``and'', ``inclusive or'', or ``exclusive or'' of the two's complement
representations of their arguments.  If they are passed only one
argument, they return that argument.  If they are passed no arguments,
they return the integer object (either $-1$ or $0$) that acts as identity for
the operation.
\end{entry}

\begin{entry}{%
\proto{bitwise-if}{ \vari{ei} \varii{ei} \variii{ei}}{procedure}}

Returns the exact integer object that is the bit-wise ``if'' of the two's complement
representations of its arguments, i.e.\ for each bit, if it is 1 in
\vari{ei}, the corresponding bit in \varii{ei} becomes the value of
the corresponding bit in the result, and if it is 0, the corresponding
bit in \variii{ei} becomes the corresponding bit in the value of the
result.
This is the result of the following computation:
\begin{scheme}
(bitwise-ior (bitwise-and \vari{ei} \varii{ei})
             (bitwise-and (bitwise-not \vari{ei}) \variii{ei}))%
\end{scheme}
\end{entry}

\begin{entry}{%
\proto{bitwise-bit-count}{ ei}{procedure}}
 
If \var{ei} is non-negative, this procedure returns the number of
1 bits in the two's complement representation of \var{ei}.
Otherwise it returns the result of the following computation:
%
\begin{scheme}
(bitwise-not (bitwise-bit-count (bitwise-not \var{ei})))%
\end{scheme}
\end{entry}

\begin{entry}{%
\proto{bitwise-length}{ ei}{procedure}}

Returns the number of bits needed to represent \var{ei} if it is
positive, and the number of bits needed to represent {\cf (bitwise-not
  \var{ei})} if it is negative, which is the exact integer object that
is the result of the following computation:
\begin{scheme}
(do ((result 0 (+ result 1))
     (bits (if (negative? \var{ei})
               (bitwise-not \var{ei})
               \var{ei})
           (bitwise-arithmetic-shift bits -1)))
    ((zero? bits)
     result))%
\end{scheme}
\end{entry}

\begin{entry}{%
\proto{bitwise-first-bit-set}{ ei}{procedure}}

Returns the index of the least significant $1$
bit in the two's complement representation of \var{ei}.
If \var{ei} is $0$, then $-1$ is returned.
\begin{scheme}
(bitwise-first-bit-set 0)        \ev  -1
(bitwise-first-bit-set 1)        \ev  0
(bitwise-first-bit-set -4)       \ev  2%
\end{scheme}
\end{entry}

\begin{entry}{%
\proto{bitwise-bit-set?}{ \vari{ei} \varii{ei}}{procedure}}

\domain{\varii{Ei} must be non-negative.}
The {\cf bitwise-bit-set?} procedure returns
\schtrue{} if the \varii{ei}th bit is 1 in the two's complement
representation of \vari{ei}, and \schfalse{}
otherwise.  This is the result of the following computation:
\begin{scheme}
(not (zero?
       (bitwise-and
         (bitwise-arithmetic-shift-left 1 \varii{ei})
         \vari{ei})))%
\end{scheme}
\end{entry}

\begin{entry}{%
\proto{bitwise-copy-bit}{ \vari{ei} \varii{ei} \variii{ei}}{procedure}}

\domain{\varii{Ei} must be non-negative, and \variii{ei}
must be either $0$ or $1$.}
The {\cf bitwise-copy-bit} procedure returns the result of replacing
the \varii{ei}th bit of \vari{ei} by \variii{ei}, which is
the result of the following computation:
\begin{scheme}
(let* ((mask (bitwise-arithmetic-shift-left 1 \varii{ei})))
  (bitwise-if mask
            (bitwise-arithmetic-shift-left \variii{ei} \varii{ei})
            \vari{ei}))%
\end{scheme}
\end{entry}

\begin{entry}{%
\proto{bitwise-bit-field}{ \vari{ei} \varii{ei} \variii{ei}}{procedure}}

\domain{\varii{Ei} and \variii{ei} must be non-negative, and
  \varii{ei} must be less than or equal to \variii{ei}.}
The {\cf bitwise-bit-field} procedure returns the
number represented by the bits at the positions from \varii{ei}
(inclusive) to $\variii{ei}$ (exclusive), which is
the result of the following computation:
%
\begin{scheme}
(let ((mask
       (bitwise-not
        (bitwise-arithmetic-shift-left -1 \variii{ei}))))
  (bitwise-arithmetic-shift-right
    (bitwise-and \vari{ei} mask)
    \varii{ei}))%
\end{scheme}
\end{entry}

\begin{entry}{%
\proto{bitwise-copy-bit-field}{ \vari{ei} \varii{ei} \variii{ei} \variv{ei}}{procedure}}

\domain{\varii{Ei} and \variii{ei} must be non-negative,
and \varii{ei} must be less than or equal to \variii{ei}.}
The {\cf bitwise-copy-bit-field} procedure returns
the result of replacing in \vari{ei} the bits at positions from
\varii{ei} (inclusive) to $\variii{ei}$ (exclusive) by the bits in
\variv{ei} from position 0 (inclusive) to position
$\variii{ei}-\varii{ei}$ (exclusive), which
is the result of the following computation:
%
\begin{scheme}
(let* ((to    \vari{ei})
       (start \varii{ei})
       (end   \variii{ei})
       (from  \variv{ei})
       (mask1
         (bitwise-arithmetic-shift-left -1 start))
       (mask2
         (bitwise-not
           (bitwise-arithmetic-shift-left -1 end)))
       (mask (bitwise-and mask1 mask2)))
  (bitwise-if mask
              (bitwise-arithmetic-shift-left from
                                             start)
              to))%
\end{scheme}
\end{entry}

\begin{entry} {%
\proto{bitwise-arithmetic-shift}{ \vari{ei} \varii{ei}}{procedure}}

Returns the result of the following computation:
%
\begin{scheme}
(floor (* \vari{ei} (expt 2 \varii{ei})))%
\end{scheme}

Examples:
%
\begin{scheme}
(bitwise-arithmetic-shift -6 -1) \lev -3
(bitwise-arithmetic-shift -5 -1) \lev -3
(bitwise-arithmetic-shift -4 -1) \lev -2
(bitwise-arithmetic-shift -3 -1) \lev -2
(bitwise-arithmetic-shift -2 -1) \lev -1
(bitwise-arithmetic-shift -1 -1) \lev -1%
\end{scheme}
\end{entry}

\begin{entry}{%
\proto{bitwise-arithmetic-shift-left}{ \vari{ei} \varii{ei}}{procedure}
\proto{bitwise-arithmetic-shift-right}{ \vari{ei} \varii{ei}}{procedure}}

\domain{\varii{Ei} must be non-negative.}  The {\cf
  bitwise-\hp{}arithmetic-\hp{}shift-\hp{}left} procedure returns the same result as {\cf
  bitwise-arithmetic-shift}, and
\begin{scheme}
(bitwise-arithmetic-shift-right \vari{ei} \varii{ei})%
\end{scheme}
returns the same result as 
\begin{scheme}
(bitwise-arithmetic-shift \vari{ei} (- \varii{ei}))\textrm{.}%
\end{scheme}
\end{entry}

\begin{entry}{%
\proto{bitwise-rotate-bit-field}{ \vari{ei} \varii{ei} \variii{ei} \variv{ei}}{procedure}}

\domain{\varii{Ei}, \variii{ei}, \variv{ei} must be non-negative, 
\varii{ei} must be less than or equal to \variii{ei}, and
\variv{ei} must be non-negative.}
The {\cf bitwise-rotate-bit-field} procedure returns the result of cyclically permuting in \vari{ei} the
bits at positions from \varii{ei} (inclusive) to \variii{ei} (exclusive) by \variv{ei} bits
towards the more significant bits, which is the result of the
following computation:
%
\begin{scheme}
(let* ((n     \vari{ei})
       (start \varii{ei})
       (end   \variii{ei})
       (count \variv{ei})
       (width (- end start)))
  (if (positive? width)
      (let* ((count (mod count width))
             (field0
               (bitwise-bit-field n start end))
             (field1 (bitwise-arithmetic-shift-left
                       field0 count))
             (field2 (bitwise-arithmetic-shift-right
                       field0
                       (- width count)))
             (field (bitwise-ior field1 field2)))
        (bitwise-copy-bit-field n start end field))
      n))%
\end{scheme}
\end{entry}

\begin{entry}{%
\proto{bitwise-reverse-bit-field}{ \vari{ei} \varii{ei} \variii{ei}}{procedure}}

\domain{\varii{Ei} and \variii{ei} must be non-negative, and
  \varii{ei} must be less than or equal to \variii{ei}.}  The {\cf bitwise-reverse-bit-field} procedure returns
the result obtained from \vari{ei} by reversing the
order of the bits at positions from \varii{ei} (inclusive) to
\variii{ei} (exclusive).
\begin{scheme}
(bitwise-reverse-bit-field \sharpsign{}b1010010 1 4)   \lev  88 ; \sharpsign{}b1011000%
\end{scheme}
\end{entry}

%%% Local Variables: 
%%% mode: latex
%%% TeX-master: "r6rs-lib"
%%% End: 
   \par
\chapter{{\tt syntax-case}}
\label{syntaxcasechapter}

% should include algebra for marks and substitutions
% ---see Waddell's dissertation or POPL '99 module paper
% but don't want to rule out van Tonder's shallow-binding approach

The \defrsixlibrary{syntax-case} library
provides
support for writing low-level macros
in a high-level style, with automatic syntax checking, input
destructuring, output restructuring, maintenance of lexical scoping
and referential transparency (hygiene), and support for controlled
identifier capture.

\section{Hygiene}
\label{hygienesection}

% hygiene condition for macro expansion
% (Kohlbecker, E.E., Friedman, D.P., Felleisen, M., Duba, B. 'Hygienic macro expansion' (1986))
% "Generated identifiers that become binding instances in the completely
% expanded program must only bind variables that are generated at the same
% transcription step."

Barendregt's \emph{hygiene condition}~\cite{barendregt} for the
lambda-calculus is an informal notion that requires the free variables of
an expression $N$ that is to be substituted into another expression $M$ not to
be captured by bindings in $M$ when such capture is not intended.
Kohlbecker, et al~\cite{hygienic} propose a corresponding
\emph{hygiene condition for macro expansion} that applies in all situations
where capturing is not explicit:
``Generated identifiers that become binding instances in
the completely expanded program must only bind variables that
are generated at the same transcription step''.
In the terminology of this document, the ``generated identifiers'' are
those introduced by a transformer rather than those present in the form
passed to the transformer, and a ``macro transcription step'' corresponds
to a single call by the expander to a transformer.
Also, the hygiene condition applies to all introduced bindings rather than
to introduced variable bindings alone.

This leaves open what happens to an introduced identifier that appears
outside the scope of a binding introduced by the same call.
Such an identifier refers to the lexical binding in effect where it
appears (within a {\cf syntax} \hyper{template};
see section~\ref{syntaxcasesection}) inside the transformer body or one of
the helpers it calls.
This is essentially the referential transparency property described
by Clinger and Rees~\cite{macrosthatwork}.

Thus, the hygiene condition can be restated as follows:

\begin{quotation}
\noindent
A binding for an identifier introduced into the output of a transformer
call from the expander must capture only references to the identifier
introduced into the output of the same transformer call.
A reference to an identifier introduced into the output of a transformer
refers to the closest enclosing binding for the introduced identifier or,
if it appears outside of any enclosing binding for the introduced
identifier, the closest enclosing lexical binding where the identifier
appears (within a {\cf syntax} \hyper{template})
inside the transformer body or one of the helpers it calls.
\end{quotation}

Explicit captures are handled via {\cf datum\coerce{}syntax}; see
section~\ref{conversionssection}.

Operationally, the expander can maintain hygiene with the help of
\emph{marks\mainindex{mark}} and \emph{substitutions\mainindex{substitution}}.
Marks are applied selectively by the expander to the output of each
transformer it invokes, and substitutions are applied to the portions
of each binding form that are supposed to be within the scope of the bound
identifiers.
Marks are used to distinguish like-named identifiers that are
introduced at different times (either present in the source or introduced
into the output of a particular transformer call), and substitutions are
used to map identifiers to their expand-time values.

Each time the expander encounters a macro use, it applies an
\defining{antimark} to the input form, invokes the associated transformer,
then applies a fresh mark to the output.
Marks and antimarks cancel, so the portions of the input that appear in
the output are effectively left unmarked, while the portions of the output
that are introduced are marked with the fresh mark.

Each time the expander encounters a binding form it creates a set of
substitutions, each mapping one of the (possibly marked) bound identifiers
to information about the binding.
(For a {\cf lambda} expression, the expander might map each bound
identifier to a representation of the formal parameter in the output of
the expander.
For a {\cf let-syntax} form, the expander might map each bound
identifier to the associated transformer.)
These substitutions are applied to the portions of the input form in
which the binding is supposed to be visible.

Marks and substitutions together form a \defining{wrap} that is layered on the
form being processed by the expander and pushed down toward the leaves as
necessary.
A wrapped form is referred to as a \defining{wrapped syntax object}.
Ultimately, the wrap may rest on a leaf that represents an identifier, in
which case the wrapped syntax object is referred to more precisely
as an \emph{identifier}.
An identifier contains a name along with the wrap.
(Names are typically represented by symbols.)

When a substitution is created to map an identifier to an expand-time
value, the substitution records the name of the identifier and
the set of marks that have been applied to that identifier, along
with the associated expand-time value.
The expander resolves identifier references by looking for the latest
matching substitution to be applied to the identifier, i.e., the outermost
substitution in the wrap whose name and marks match the name and
marks recorded in the substitution.
The name matches if it is the same name (if using symbols, then by
{\cf eq?}), and the marks match if the marks recorded with the
substitution are the same as those that appear \emph{below} the
substitution in the wrap, i.e., those that were applied \emph{before} the
substitution.
Marks applied after a substitution, i.e., appear over the substitution in
the wrap, are not relevant and are ignored.

An algebra that defines how marks and substitutions work more precisely is
given in section~2.4 of Oscar Waddell's PhD thesis~\cite{Waddellphd}.

\section{Syntax objects}
\label{syntaxobjectssection}

A \defining{syntax object} is a representation of a Scheme form that contains
contextual information about the form in addition to its structure.
This contextual information is used by the expander to maintain
lexical scoping and may also be used by an implementation to maintain
source-object correlation.

Syntax objects may be wrapped or unwrapped.
A wrapped syntax object (section~\ref{hygienesection}) consists of a
\textit{wrap} (section~\ref{hygienesection}) and some internal representation
of a Scheme form.
(The internal representation is unspecified, but is typically a 
datum value or datum value annotated with source information.)
A wrapped syntax object representing an identifier is itself referred to as
an identifier; thus, the term \textit{identifier\mainindex{identifier}} may refer either to
the syntactic entity (symbol, variable, or keyword) or to the
concrete representation of the syntactic entity as a syntax object.
Wrapped syntax objects may or may not be distinct from other types of values,
but syntax objects representing identifiers are distinct
from other types of values.

An unwrapped syntax object is one that is unwrapped, fully or partially,
i.e., whose outer layers consist of lists and vectors and whose leaves are
either wrapped syntax objects or nonsymbol values.

The term syntax object is used in this document to refer to
a syntax object that is either wrapped or unwrapped.
More formally, a syntax object is:

\begin{itemize}
\item a pair of syntax objects,
\item a vector of syntax objects,
\item a nonpair, nonvector, nonsymbol value, or
\item a wrapped syntax object.
\end{itemize}

The distinction between the terms ``syntax object'' and ``wrapped syntax
object'' is important.
For example, when invoked by the expander, a transformer
(section~\ref{transformerssection}) must accept a wrapped syntax object but
may return any syntax object, including an unwrapped syntax object.

\section{Transformers}
\label{transformerssection}

In {\cf define-syntax} (report
section~\extref{report:define-syntax}{Syntax definitions}), {\cf
  let-syntax}, and {\cf letrec-syntax} forms (report
section~\extref{report:let-syntax}{Binding constructs for syntactic
  keywords}), a binding for a syntactic keyword must be an expression
that evaluates to a \defining{transformer}\index{macro transformer}.  (This is only the user's
responsibility; the implementation must check this only if evaluation
of a transformer expression actually terminates.  See the respective
specifications.)

A transformer is a \defining{transformation procedure} or a
\defining{variable transformer}.
A transformation procedure is a procedure that must accept one
argument, a wrapped syntax object (section~\ref{syntaxobjectssection})
representing the input, and return a \defining{syntax object}
(section~\ref{syntaxobjectssection}) representing the output.
The transformer is called by the expander whenever a reference to
a keyword with which it has been associated is found.
If the keyword appears in the car of a list-structured
input form, the transformer receives the entire list-structured
form, and its output replaces the entire form.
Except with variable transformers (see below),
if the keyword is found in any other definition or expression
context, the transformer receives a wrapped syntax object representing
just the keyword reference, and its output replaces just the reference.
Except with variable transformers, an exception with condition
type {\cf\&syntax} is raised if the keyword appears on the left-hand side
of a {\cf set!} expression.

\begin{entry}{%
\proto{make-variable-transformer}{ proc}{procedure}}

\domain{\var{Proc} should accept one argument,
a wrapped syntax object, and return a syntax object.}

The {\cf make-variable-transformer} procedure creates a
\defining{variable transformer}.
A variable transformer is like an ordinary transformer except
that, if a keyword associated with a variable transformer appears on
the left-hand side of a {\cf set!} expression, an exception is
not raised.
Instead, \var{proc} is called with a
wrapped syntax object representing the entire {\cf set!} expression as
its argument, and its return value replaces the entire {\cf set!}
expression.

\implresp The implementation must check the restrictions on \var{proc}
only to the extent performed by applying it as described.
An
implementation may check whether \var{proc} is an appropriate argument
before applying it.
\end{entry}

\section{Parsing input and producing output}
\label{syntaxcasesection}

Transformers can destructure their input with {\cf syntax-case} and rebuild
their output with {\cf syntax}.

\begin{entry}{%
\pproto{(syntax-case \hyper{expression} (\hyper{literal} \dots)}{\exprtype}
\mainschindex{syntax-case}{\tt\obeyspaces%
\hspace{2em}\hyper{syntax-case clause} \dots)}\\
\litprotonoindex{\_}
\litprotonoindex{...}}\schindex{\_}\schindex{...}

\syntax Each \hyper{literal} must be an identifier.
Each \hyper{syntax-case clause} must take one of the following two forms.

\begin{scheme}
(\hyper{pattern} \hyper{output expression})
(\hyper{pattern} \hyper{fender} \hyper{output expression})%
\end{scheme}

\hyper{Fender} and \hyper{output expression} must be
\hyper{expression}s.

A \hyper{pattern} is an identifier, constant, or one of the following.

\begin{schemenoindent}
(\hyper{pattern} \ldots)
(\hyper{pattern} \hyper{pattern} \ldots . \hyper{pattern})
(\hyper{pattern} \ldots \hyper{pattern} \hyper{ellipsis} \hyper{pattern} \ldots)
(\hyper{pattern} \ldots \hyper{pattern} \hyper{ellipsis} \hyper{pattern} \ldots . \hyper{pattern})
\#(\hyper{pattern} \ldots)
\#(\hyper{pattern} \ldots \hyper{pattern} \hyper{ellipsis} \hyper{pattern} \ldots)%
\end{schemenoindent}

An \hyper{ellipsis} is the identifier ``{\cf ...}'' (three periods).\schindex{...}

An identifier appearing within a \hyper{pattern} may be an underscore
(~{\cf \_}~), a literal identifier listed in the list of literals
{\cf (\hyper{literal} \dots)}, or an ellipsis (~{\cf ...}~).
All other identifiers appearing within a \hyper{pattern} are
\textit{pattern variables\mainindex{pattern variable}}.
It is a syntax violation if an ellipsis or underscore appears in {\cf (\hyper{literal} \dots)}.

{\cf \_} and {\cf ...} are the same as in the \rsixlibrary{base} library.

Pattern variables match arbitrary input subforms and
are used to refer to elements of the input.
It is a syntax violation if the same pattern variable appears more than once in a
\hyper{pattern}.

Underscores also match arbitrary input subforms but are not pattern variables
and so cannot be used to refer to those elements.
Multiple underscores may appear in a \hyper{pattern}.

A literal identifier matches an input subform if and only if the input
subform is an identifier and either both its occurrence in the input
expression and its occurrence in the list of literals have the same
lexical binding, or the two identifiers have the same name and both have
no lexical binding.

A subpattern followed by an ellipsis can match zero or more elements of
the input.

More formally, an input form $F$ matches a pattern $P$ if and only if
one of the following holds:

\begin{itemize}
\item $P$ is an underscore (~{\cf \_}~).

\item $P$ is a pattern variable.

\item $P$ is a literal identifier
and $F$ is an equivalent identifier in the
sense of {\cf free-identifier=?}
(section~\ref{identifierpredicatessection}).

\item $P$ is of the form
{\cf ($P_1$ \dots{} $P_n$)}
and $F$ is a list of $n$ elements that match $P_1$ through
$P_n$.

\item $P$ is of the form
{\cf ($P_1$ \dots{} $P_n$ . $P_x$)}
and $F$ is a list or improper list of $n$ or more elements
whose first $n$ elements match $P_1$ through $P_n$
and
whose $n$th cdr matches $P_x$.

\item $P$ is of the form
{\cf ($P_1$ \dots{} $P_k$ $P_e$ \hyper{ellipsis} $P_{m+1}$ \dots{} $P_n$)},
where \hyper{ellipsis} is the identifier {\cf ...}
and $F$ is a proper list of $n$
elements whose first $k$ elements match $P_1$ through $P_k$,
whose next $m-k$ elements each match $P_e$,
and
whose remaining $n-m$ elements match $P_{m+1}$ through $P_n$.

\item $P$ is of the form
{\cf ($P_1$ \dots{} $P_k$ $P_e$ \hyper{ellipsis} $P_{m+1}$ \dots{} $P_n$ . $P_x$)},
where \hyper{ellipsis} is the identifier {\cf ...}
and $F$ is a list or improper list of $n$
elements whose first $k$ elements match $P_1$ through $P_k$,
whose next $m-k$ elements each match $P_e$,
whose next $n-m$ elements match $P_{m+1}$ through $P_n$,
and 
whose $n$th and final cdr matches $P_x$.

\item $P$ is of the form
{\cf \#($P_1$ \dots{} $P_n$)}
and $F$ is a vector of $n$ elements that match $P_1$ through
$P_n$.

\item $P$ is of the form
{\cf \#($P_1$ \dots{} $P_k$ $P_e$ \hyper{ellipsis} $P_{m+1}$ \dots{} $P_n$)},
where \hyper{ellipsis} is the identifier {\cf ...}
and $F$ is a vector of $n$ or more elements
whose first $k$ elements match $P_1$ through $P_k$,
whose next $m-k$ elements each match $P_e$,
and
whose remaining $n-m$ elements match $P_{m+1}$ through $P_n$.

\item $P$ is a pattern datum (any nonlist, nonvector, nonsymbol
datum) and $F$ is equal to $P$ in the sense of the
{\cf equal?} procedure.
\end{itemize}

\semantics
{\cf syntax-case} first evaluates \hyper{expression}.
It then attempts to match
the \hyper{pattern} from the first \hyper{syntax-case clause} against the resulting value,
which is unwrapped as necessary to perform the match.
If the pattern matches the value and no
\hyper{fender} is present,
\hyper{output expression} is evaluated and its value returned as the
value of the {\cf syntax-case} expression.
If the pattern does not match the value, {\cf syntax-case} tries
the second \hyper{syntax-case clause}, then the third, and so on.
It is a syntax violation if the value does not match any of the patterns.

If the optional \hyper{fender} is present, it serves as an additional
constraint on acceptance of a clause.
If the \hyper{pattern} of a given \hyper{syntax-case clause} matches the input value,
the corresponding \hyper{fender} is evaluated.
If \hyper{fender} evaluates to a true value, the clause is accepted;
otherwise, the clause is rejected as if the pattern had failed to match
the value.
Fenders are logically a part of the matching process, i.e., they
specify additional matching constraints beyond the basic structure of
the input.

Pattern variables contained within a clause's
\hyper{pattern} are bound to the corresponding pieces of the input
value within the clause's \hyper{fender} (if present) and
\hyper{output expression}.
Pattern variables can be referenced only within {\cf syntax}
expressions (see below).
Pattern variables occupy the same name space as program variables and
keywords.

If the {\cf syntax-case} form is in tail context, the \hyper{output
  expression}s are also in tail position.
\end{entry}

\begin{entry}{%
\proto{syntax}{ \hyper{template}}{\exprtype}}

\begin{note}
{\cf \#'\hyper{template}} is equivalent to {\cf (syntax
  \hyper{template})}.
\end{note}

A {\cf syntax} expression is similar to a {\cf quote} expression
except that (1) the values of pattern variables appearing within
\hyper{template} are inserted into \hyper{template}, (2) contextual
information associated both with the input and with the template is
retained in the output to support lexical scoping, and (3) the value
of a {\cf syntax} expression is a syntax object.

A \hyper{template} is a pattern variable, an identifier that
is not a pattern
variable, a pattern datum, or one of the following.

\begin{scheme}
(\hyper{subtemplate} \ldots)
(\hyper{subtemplate} \ldots . \hyper{template})
\#(\hyper{subtemplate} \ldots)%
\end{scheme}

A \hyper{subtemplate} is a \hyper{template} followed by zero or more ellipses.

The value of a {\cf syntax} form is a copy of \hyper{template} in which
the pattern variables appearing within the template are replaced with
the input subforms to which they are bound.
Pattern data and identifiers that are not pattern variables
or ellipses are copied directly into the output.
A subtemplate followed by an ellipsis expands
into zero or more occurrences of the subtemplate.
Pattern variables that occur in subpatterns followed by one or more
ellipses may occur only in subtemplates that are
followed by (at least) as many ellipses.
These pattern variables are replaced in the output by the input
subforms to which they are bound, distributed as specified.
If a pattern variable is followed by more ellipses in the subtemplate
than in the associated subpattern, the input form is replicated as
necessary.
The subtemplate must contain at least one pattern variable from a
subpattern followed by an ellipsis, and for at least one such pattern
variable, the subtemplate must be followed by exactly as many ellipses as
the subpattern in which the pattern variable appears.
(Otherwise, the expander would not be able to determine how many times the
subform should be repeated in the output.)
It is a syntax violation if the constraints of this paragraph are not met.

A template of the form
{\cf (\hyper{ellipsis} \hyper{template})} is identical to \hyper{template}, except that
ellipses within the template have no special meaning.
That is, any ellipses contained within \hyper{template} are
treated as ordinary identifiers.
In particular, the template {\cf (... ...)} produces a single
ellipsis.
This allows macro uses to expand into forms containing
ellipses.

\label{wrappingrules}
The output produced by {\cf syntax} is wrapped or unwrapped according to
the following rules.

\begin{itemize}
\item the copy of {\cf (\hyperi{t} .  \hyperii{t})} is a pair if \hyperi{t}
      or \hyperii{t} contain any pattern variables,
\item the copy of {\cf (\hyper{t} \hyper{ellipsis})} is a list if \hyper{t}
      contains any pattern variables,
\item the copy of {\cf \#(\hyperi{t} ... \hypern{t})} is a vector if any of
      \hyperi{t},~\dots,~\hypern{t} contain any pattern variables, and
\item the copy of any portion of \hyper{t} not containing any pattern variables
      is a wrapped syntax object.
\end{itemize}

The input subforms inserted in place of the pattern variables are wrapped
if and only if the corresponding input subforms are wrapped.
\end{entry}

The following definitions of {\cf or} illustrate {\cf syntax-case}
and {\cf syntax}.
The second is equivalent to the first but uses the {\cf \#'}
prefix instead of the full {\cf syntax} form.

\begin{schemenoindent}
(define-syntax or
  (lambda (x)
    (syntax-case x ()
      [(\_) (syntax \schfalse{})]
      [(\_ e) (syntax e)]
      [(\_ e1 e2 e3 ...)
       (syntax (let ([t e1])
                 (if t t (or e2 e3 ...))))])))

(define-syntax or
  (lambda (x)
    (syntax-case x ()
      [(\_) \#'\schfalse{}]
      [(\_ e) \#'e]
      [(\_ e1 e2 e3 ...)
       \#'(let ([t e1])
           (if t t (or e2 e3 ...)))])))
\end{schemenoindent}

The examples below define \emph{identifier macros\mainindex{identifier
  macro}}, macro uses
supporting keyword references that do not necessarily appear in the first
position of a list-structured form.
The second example uses {\cf make-variable-transformer} to handle the case
where the keyword appears on the left-hand side of a
{\cf set!} expression.

\begin{scheme}
(define p (cons 4 5))
(define-syntax p.car
  (lambda (x)
    (syntax-case x ()
      [(\_ . rest) \#'((car p) . rest)]
      [\_  \#'(car p)])))
p.car \ev 4
(set! p.car 15) \ev \exception{\&syntax}

(define p (cons 4 5))
(define-syntax p.car
  (make-variable-transformer
    (lambda (x)
      (syntax-case x (set!)
        [(set! \_ e) \#'(set-car! p e)]
        [(\_ . rest) \#'((car p) . rest)]
        [\_  \#'(car p)]))))
(set! p.car 15)
p.car           \ev 15
p               \ev (15 5)%
\end{scheme}

\section{Identifier predicates}
\label{identifierpredicatessection}

\begin{entry}{%
\proto{identifier?}{ obj}{procedure}}

Returns \schtrue{} if \var{obj} is an identifier, i.e., a
syntax object representing an identifier, and \schfalse{} otherwise.

The {\cf identifier?} procedure is often used within a fender to verify
that certain subforms of an input form are identifiers, as in the
definition of {\cf rec}, which creates self-contained
recursive objects, below.

\begin{scheme}
(define-syntax rec
  (lambda (x)
    (syntax-case x ()
      [(\_ x e)
       (identifier? \#'x)
       \#'(letrec ([x e]) x)])))

(map (rec fact
       (lambda (n)
         (if (= n 0)                 
             1
             (* n (fact (- n 1))))))
     '(1 2 3 4 5)) \lev (1 2 6 24 120)
 
(rec 5 (lambda (x) x)) \ev \exception{\&syntax}%
\end{scheme}
\end{entry}

The procedures {\cf bound-identifier=?} and {\cf free-identifier=?}
each take two identifier arguments and return \schtrue{} if their
arguments are equivalent and \schfalse{} otherwise.
These predicates are used to compare identifiers according to their
\emph{intended use} as free references or bound identifiers in a given
context.

\begin{entry}{%
\proto{bound-identifier=?}{ \vari{id} \varii{id}}{procedure}}

\domain{\vari{Id} and \varii{id} must be identifiers.}
The procedure {\cf bound-identifier=?} returns \schtrue{} if and only if a
binding for one would capture a reference to the other in the output of
the transformer, assuming that the reference appears within the scope of
the binding.
In general, two identifiers are {\cf bound-identifier=?} only if
both are present in the original program or both are introduced by the
same transformer application
(perhaps implicitly---see {\cf datum\coerce{}syntax}).
Operationally, two identifiers are
considered equivalent by {\cf bound-identifier=?} if and only if they
have the same name and same marks (section~\ref{hygienesection}).

The {\cf bound-identifier=?} procedure can be used for detecting
duplicate identifiers in a binding construct or for other
preprocessing of a binding construct that requires detecting instances
of the bound identifiers.
\end{entry}

\begin{entry}{%
\proto{free-identifier=?}{ \vari{id} \varii{id}}{procedure}}

\domain{\vari{Id} and \varii{id} must be identifiers.}
The {\cf free-identifier=?} procedure returns \schtrue{} if and
only if the two identifiers would resolve to the same binding if both were
to appear in the output of a transformer outside of any bindings inserted
by the transformer.
(If neither of two like-named identifiers resolves to a binding, i.e., both
are unbound, they are considered to resolve to the same binding.)
Operationally, two identifiers are considered equivalent by
{\cf free-identifier=?} if and only the topmost matching
substitution for each maps to the same binding (section~\ref{hygienesection})
or the identifiers have the same name and no matching substitution.

{\cf syntax-case} and {\cf syntax-rules} use
{\cf free-identifier=?} to compare identifiers listed in the literals
list against input identifiers.

The following definition of unnamed {\cf let}
uses {\cf bound-identifier=?} to detect duplicate identifiers.

\begin{schemenoindent}
(define-syntax let
  (lambda (x)
    (define unique-ids?
      (lambda (ls)
        (or (null? ls)
            (and (let notmem?
                        ([x (car ls)] [ls (cdr ls)])
                   (or (null? ls)
                       (and (not (bound-identifier=?
                                   x (car ls)))
                            (notmem? x (cdr ls)))))
                 (unique-ids? (cdr ls))))))
    (syntax-case x ()
      [(\_ ((i v) ...) e1 e2 ...)
       (unique-ids? \#'(i ...))
       \#'((lambda (i ...) e1 e2 ...) v ...)])))
\end{schemenoindent}

The argument {\cf \#'(i ...)} to {\cf unique-ids?} is guaranteed
to be a list by the rules given in the description of {\cf syntax}
above.

With this definition of {\cf let}:

\begin{scheme}
(let ([a 3] [a 4]) (+ a a)) \lev \exception{\&syntax}%
\end{scheme}

However,

\begin{scheme}
(let-syntax
  ([dolet (lambda (x)
            (syntax-case x ()
              [(\_ b)
               \#'(let ([a 3] [b 4]) (+ a b))]))])
  (dolet a)) \lev 7%
\end{scheme}

since the identifier {\cf a} introduced by {\cf dolet}
and the identifier {\cf a} extracted from the input form are not
{\cf bound-identifier=?}.

The following definition of {\cf case} is equivalent to the one in
section~\ref{syntaxcasesection}.
Rather than including {\cf else} in the literals list as before,
this version explicitly tests for {\cf else} using
{\cf free-identifier=?}.

\begin{schemenoindent}
(define-syntax case
  (lambda (x)
    (syntax-case x ()
      [(\_ e0 [(k ...) e1 e2 ...] ...
              [else-key else-e1 else-e2 ...])
       (and (identifier? \#'else-key)
            (free-identifier=? \#'else-key \#'else))
       \#'(let ([t e0])
           (cond
             [(memv t '(k ...)) e1 e2 ...]
             ...
             [else else-e1 else-e2 ...]))]
      [(\_ e0 [(ka ...) e1a e2a ...]
              [(kb ...) e1b e2b ...] ...)
       \#'(let ([t e0])
           (cond
             [(memv t '(ka ...)) e1a e2a ...]
             [(memv t '(kb ...)) e1b e2b ...]
             ...))])))
\end{schemenoindent}

With either definition of {\cf case}, {\cf else} is not
recognized as an auxiliary
keyword if an enclosing lexical binding for {\cf else} exists.
For example,

\begin{scheme}
(let ([else \schfalse{}])
  (case 0 [else (write "oops")])) \lev \exception{\&syntax}%
\end{scheme}

since {\cf else} is bound
lexically and is
therefore not the same {\cf else} that appears in the definition of
{\cf case}.
\end{entry}

\section{Syntax-object and datum conversions}
\label{conversionssection}

\begin{entry}{%
\proto{syntax->datum}{ syntax-object}{procedure}}

The procedure {\cf syntax\coerce{}datum}
strips all syntactic information from a syntax
object and returns the corresponding Scheme datum.
\end{entry}

Identifiers stripped in this manner are converted to their symbolic
names, which can then be compared with {\cf eq?}.
Thus, a predicate {\cf symbolic-identifier=?} might be defined as follows.

\begin{scheme}
(define symbolic-identifier=?
  (lambda (x y)
    (eq? (syntax->datum x)
         (syntax->datum y))))%
\end{scheme}

% not be true with import alias and rename
%Two identifiers that are {\cf bound-identifier=?} or
%{\cf free-identifier=?} are {\cf symbolic-identifier=?}; in order to
%refer to the same binding, two identifiers must have the same name.
%The converse is not always true, since two identifiers may have
%the same name but different bindings.

\begin{entry}{%
\proto{datum->syntax}{ template-id datum}{procedure}}
\end{entry}

\domain{\var{Template-id} must be a
template identifier and \var{datum} should be a datum value.}
The {\cf datum->syntax} procedure returns a syntax object representation of \var{datum} that
contains the same contextual information as
\var{template-id}, with the effect that the
syntax object behaves
as if it were introduced into the code when
\var{template-id} was introduced.

The {\cf datum\coerce{}syntax} procedure allows a transformer to ``bend'' lexical
scoping rules by creating \textit{implicit
  identifiers\mainindex{implicit identifier}}
that behave as if they were present in the input form,
thus permitting the definition of macros
that introduce visible bindings for or references to
identifiers that do not appear explicitly in the input form.
For example, the following defines a {\cf loop} expression that
uses this controlled form of identifier capture to
bind the variable {\cf break} to an escape procedure
within the loop body.
(The derived {\cf with-syntax} form is like {\cf let} but binds
pattern variables---see section~\ref{derivedsection}.)

\begin{scheme}
(define-syntax loop
  (lambda (x)
    (syntax-case x ()
      [(k e ...)
       (with-syntax
           ([break (datum->syntax \#'k 'break)])
         \#'(call-with-current-continuation
             (lambda (break)
               (let f () e ... (f)))))])))

(let ((n 3) (ls '()))
  (loop
    (if (= n 0) (break ls))
    (set! ls (cons 'a ls))
    (set! n (- n 1)))) \lev (a a a)%
\end{scheme}

Were {\cf loop} to be defined as

\begin{scheme}
(define-syntax loop
  (lambda (x)
    (syntax-case x ()
      [(\_ e ...)
       \#'(call-with-current-continuation
           (lambda (break)
             (let f () e ... (f))))])))%
\end{scheme}

the variable {\cf break} would not be visible in {\cf e \dots}.

The datum argument \var{datum} may also represent an arbitrary
Scheme form, as demonstrated by the following definition of
{\cf include}.

\begin{scheme}
(define-syntax include
  (lambda (x)
    (define read-file
      (lambda (fn k)
        (let ([p (open-file-input-port fn)])
          (let f ([x (get-datum p)])
            (if (eof-object? x)
                (begin (close-port p) '())
                (cons (datum->syntax k x)
                      (f (get-datum p))))))))
    (syntax-case x ()
      [(k filename)
       (let ([fn (syntax->datum \#'filename)])
         (with-syntax ([(exp ...)
                        (read-file fn \#'k)])
           \#'(begin exp ...)))])))%
\end{scheme}

{\cf (include "filename")} expands into a {\cf begin} expression
containing the forms found in the file named by
{\cf "filename"}.
For example, if the file {\cf flib.ss} contains
{\cf (define f (lambda (x) (g (* x x))))}, and the file
{\cf glib.ss} contains
{\cf (define g (lambda (x) (+ x x)))},
the expression

\begin{scheme}
(let ()
  (include "flib.ss")
  (include "glib.ss")
  (f 5))%
\end{scheme}

evaluates to {\cf 50}.

The definition of {\cf include} uses {\cf datum\coerce{}syntax} to convert
the objects read from the file into syntax objects in the proper
lexical context, so that identifier references and definitions within
those expressions are scoped where the {\cf include} form appears.

Using {\cf datum\coerce{}syntax}, it is even possible to break hygiene
entirely and write macros in the style of old Lisp macros.
The {\cf lisp-transformer} procedure defined below creates a transformer
that converts its input into a datum, calls the programmer's procedure on
this datum, and converts the result back into a syntax object that is
scoped at top level (or, more accurately, wherever
{\cf lisp-transformer} is defined).

\begin{scheme}
(define lisp-transformer
  (lambda (p)
    (lambda (x)
      (datum\coerce{}syntax \#'lisp-transformer
        (p (syntax\coerce{}datum x))))))%
\end{scheme}

\section{Generating lists of temporaries}
\label{generatingtemporariessection}

Transformers can introduce a fixed number of identifiers into their
output simply by naming each identifier.
In some cases, however, the number of identifiers to be introduced depends
upon some characteristic of the input expression.
A straightforward definition of {\cf letrec}, for example,
requires as many
temporary identifiers as there are binding pairs in the
input expression.
The procedure {\cf generate-temporaries} is used to construct
lists of temporary identifiers.

\begin{entry}{%
\proto{generate-temporaries}{ l}{procedure}}

\domain{\var{L} must be be a list or syntax object representing a list-structured
form; its contents are not important.}
The number of temporaries generated is the number of elements in \var{l}.
Each temporary is guaranteed to be unique, i.e., different from all other
identifiers.

A definition of {\cf letrec} equivalent to the one using
{\cf syntax-rules} given in report
appendix~\extref{report:derivedformsappendix}{Sample definitions for
derived forms} is shown below.

\begin{schemenoindent}
(define-syntax letrec
  (lambda (x)
    (syntax-case x ()
      ((\_ ((i e) ...) b1 b2 ...)
       (with-syntax
           (((t ...) (generate-temporaries \#'(i ...))))
         \#'(let ((i <undefined>) ...)
             (let ((t e) ...)
               (set! i t) ...
               (let () b1 b2 ...))))))))
\end{schemenoindent}

This version uses {\cf generate-temporaries} instead of recursively defined
helper to generate the necessary temporaries.
\end{entry}

\section{Derived forms and procedures}
\label{derivedsection}

The forms and procedures described in this section are \emph{derived},
i.e., they can defined in terms of the forms and procedures described
in earlier sections of this document.

\begin{entry}{%
\pproto{(with-syntax ((\hyper{pattern} \hyper{expression}) \dotsfoo) \hyper{body})}{\exprtype}}
\mainschindex{with-syntax}

The derived {\cf with-syntax} form is used to bind pattern variables,
just as {\cf let} is used to bind variables.
This allows a transformer to construct its output in separate
pieces, then put the pieces together.

Each \hyper{pattern} is identical in form to a {\cf syntax-case} pattern.
The value of each \hyper{expression} is computed and destructured according
to the corresponding \hyper{pattern}, and pattern variables within
the \hyper{pattern} are bound as with {\cf syntax-case} to the
corresponding portions of the value within \hyper{body}.

The {\cf with-syntax} form may be defined in terms of {\cf syntax-case} as
follows.

\begin{scheme}
(define-syntax with-syntax
  (lambda (x)
    (syntax-case x ()
      ((\_ ((p e0) ...) e1 e2 ...)
       (syntax (syntax-case (list e0 ...) ()
                 ((p ...) (let () e1 e2 ...))))))))%
\end{scheme}

The following definition of {\cf cond} demonstrates the use of
{\cf with-syntax} to support transformers that employ recursion
internally to construct their output.
It handles all {\cf cond} clause variations and takes care to produce
one-armed {\cf if} expressions where appropriate.

\begin{schemenoindent}
(define-syntax cond
  (lambda (x)
    (syntax-case x ()
      [(\_ c1 c2 ...)
       (let f ([c1 \#'c1] [c2* \#'(c2 ...)])
         (syntax-case c2* ()
           [()
            (syntax-case c1 (else =>)
              [(else e1 e2 ...) \#'(begin e1 e2 ...)]
              [(e0) \#'e0]
              [(e0 => e1)
               \#'(let ([t e0]) (if t (e1 t)))]
              [(e0 e1 e2 ...)
               \#'(if e0 (begin e1 e2 ...))])]
           [(c2 c3 ...)
            (with-syntax ([rest (f \#'c2 \#'(c3 ...))])
              (syntax-case c1 (=>)
                [(e0) \#'(let ([t e0]) (if t t rest))]
                [(e0 => e1)
                 \#'(let ([t e0]) (if t (e1 t) rest))]
                [(e0 e1 e2 ...)
                 \#'(if e0 
                        (begin e1 e2 ...)
                        rest)]))]))])))
\end{schemenoindent}
\end{entry}

\begin{entry}{%
\proto{quasisyntax}{ \hyper{template}}{\exprtype}
\litproto{unsyntax}
\litproto{unsyntax-splicing}}

The {\cf quasisyntax} form is similar to {\cf syntax}, but it allows parts
of the quoted text to be evaluated, in a manner similar to the operation
of {\cf quasiquote} (report section~\extref{report:quasiquotesection}{Quasiquotation}).

Within a {\cf quasisyntax} \var{template}, subforms of
{\cf unsyntax} and {\cf unsyntax-splicing} forms are evaluated,
and everything else is treated as ordinary template material, as
with {\cf syntax}.
The value of each {\cf unsyntax} subform is inserted into the output
in place of the {\cf unsyntax} form, while the value of each
{\cf unsyntax-splicing} subform is spliced into the surrounding list
or vector structure.
Uses of {\cf unsyntax} and {\cf unsyntax-splicing} are valid only within
{\cf quasisyntax} expressions.

A {\cf quasisyntax} expression may be nested, with each {\cf quasisyntax}
introducing a new level of syntax quotation and each {\cf unsyntax} or
{\cf unsyntax-splicing} taking away a level of quotation.
An expression nested within $n$ {\cf quasisyntax} expressions must
be within $n$ {\cf unsyntax} or {\cf unsyntax-splicing} expressions to
be evaluated.

As noted in report section~\extref{report:abbreviationsection}{Abbreviations},
{\cf \#`\hyper{template}} is equivalent to {\cf (quasisyntax
  \hyper{template})}, {\cf \#,\hyper{template}} is equivalent to {\cf (unsyntax
  \hyper{template})}, and {\cf \#,@\hyper{template}} is equivalent to {\cf (unsyntax-splicing
  \hyper{template})}.

The {\cf quasisyntax} keyword can be used in place of {\cf with-syntax} in many
cases.
For example, the definition of {\cf case} shown under the description
of {\cf with-syntax} above can be rewritten using {\cf quasisyntax}
as follows.

\begin{schemenoindent}
(define-syntax case
  (lambda (x)
    (syntax-case x ()
      [(\_ e c1 c2 ...)
       \#`(let ([t e])
           \#,(let f ([c1 \#'c1] [cmore \#'(c2 ...)])
               (if (null? cmore)
                   (syntax-case c1 (else)
                     [(else e1 e2 ...)
                      \#'(begin e1 e2 ...)]
                     [((k ...) e1 e2 ...)
                      \#'(if (memv t '(k ...))
                            (begin e1 e2 ...))])
                   (syntax-case c1 ()
                     [((k ...) e1 e2 ...)
                      \#`(if (memv t '(k ...))
                            (begin e1 e2 ...)
                            \#,(f (car cmore)
                                  (cdr cmore)))]))))])))
\end{schemenoindent}
                          
Uses of {\cf unsyntax} and {\cf unsyntax-splicing} with zero or more than
one subform are valid only in splicing (list or vector) contexts.
{\cf (unsyntax \var{template} \dots)} is equivalent to
{\cf (unsyntax \var{template}) \dots}, and
{\cf (unsyntax-splicing \var{template} \dots)} is equivalent to
{\cf (unsyntax-splicing \var{template}) \dots}.
These forms are primarily useful as intermediate forms in the output
of the {\cf quasisyntax} expander.

\begin{note}
Uses of {\cf unsyntax} and {\cf unsyntax-splicing} with 
zero or more than one subform enable certain 
idioms~\cite{bawdenquasiquote}, such as {\cf \#,@\#,@}, which has the
effect of a doubly indirect splicing when used within a doubly nested
and doubly evaluated {\cf quasisyntax} expression, as with the
nested {\cf quasiquote} examples shown in
section~\extref{report:quasiquotesection}{Quasiquotation}.
\end{note}
\end{entry}

\begin{note}
Any {\cf syntax-rules} form can be expressed with
{\cf syntax-case} by making the {\cf lambda} expression and
{\cf syntax} expressions explicit, and
{\cf syntax-rules} may be defined in terms of {\cf syntax-case}
as follows.

\begin{scheme}
(define-syntax syntax-rules
  (lambda (x)
    (syntax-case x ()
      [(\_ (k ...) [(\_ . p) f ... t] ...)
       \#'(lambda (x)
           (syntax-case x (k ...)
             [(\_ . p) f ... \#'t] ...))])))%
\end{scheme}

A more robust implementation would verify that the literals
{\cf \hyper{literal} \dots} are all identifiers, that the first position
of each pattern is an identifier, and that at most one fender
is present in each clause.
\end{note}

\begin{note}
The {\cf identifier-syntax} form of the base library (see
report section~\extref{report:identifier-syntax}{Macro transformers}) may be defined in terms of {\cf
  syntax-case}, {\cf syntax}, and {\cf make-variable-transformer} as
follows.

\begin{schemenoindent}
(define-syntax identifier-syntax
  (syntax-rules (set!)
    [(\_ e)
     (lambda (x)
       (syntax-case x ()
         [id (identifier? \#'id) \#'e]
         [(\_ x (... ...)) \#'(e x (... ...))]))]
    [(\_ (id exp1) ((set! var val) exp2))
     (and (identifier? \#'id) (identifier? \#'var))
     (make-variable-transformer
       (lambda (x)
         (syntax-case x (set!)
           [(set! var val) \#'exp2]
           [(id x (... ...)) \#'(exp1 x (... ...))]
           [id (identifier? \#'id) \#'exp1])))]))
\end{schemenoindent}
\end{note}

\section{Syntax violations}

\begin{entry}{%
\proto{syntax-violation}{ who message form}{procedure}
\rproto{syntax-violation}{ who message form subform}{procedure}}

\domain{\var{Who} must be \schfalse{} or a string or a symbol.
  \var{Message} must be a string.
  \var{Form} must be a syntax object or a datum value.
  \var{Subform} must be a syntax object or a datum value.}
The {\cf syntax-violation} procedure raises an exception, reporting 
a syntax violation.  
The \var{who} argument should describe the macro transformer that
detected the exception.  The \var{message} argument should describe
the violation.
The \var{form} argument is the erroneous source syntax
object or a datum value representing a form. The optional \var{subform} argument is a syntax
object or datum value representing a form that more precisely locates the
violation.

If \var{who} is \schfalse{}, {\cf syntax-violation} attempts to
infer an appropriate value for the condition object (see below) as
follows:  When \var{form} is either an identifier or a
list-structured syntax object containing an identifier as its first element, then
the inferred value is the identifier's symbol.
Otherwise, no value for \var{who} is provided as part of the
condition object.

The condition object provided with the exception (see
chapter~\ref{exceptionsconditionschapter}) has the following condition types:
%
\begin{itemize}
\item If \var{who} is not \schfalse{} or can be inferred, the condition has condition type
  {\cf \&who}, with \var{who} as the value of the {\cf who} field.  In
  that case, \var{who} should identify the procedure or entity that
  detected the exception.  If it is \schfalse, the condition does not
  have condition type {\cf \&who}.
\item The condition has condition type {\cf \&message}, with
  \var{message} as the value of the {\cf message} field.
\item The condition has condition type {\cf \&syntax} 
  with \var{form} as the value of the {\cf form} field,
  and \var{subform} as the value of the {\cf subform} field.
  If \var{subform} is not provided, the value of the {\cf subform}
  field is \schfalse.
\end{itemize}
\end{entry}

%%% Local Variables: 
%%% mode: latex
%%% TeX-master: "r6rs-lib"
%%% End: 
 \par
\chapter{Hash tables}
\label{hashtablechapter}

The \deflibrary{r6rs hash-tables} library provides hash tables.
A \defining{hash table} is a data structure that associates keys with values.
Any object can be used as a key, provided a \defining{hash function}
and a suitable \defining{equivalence function} is available.  A hash function is a
procedure that maps
keys to integers, and must be compatible with the equivalence function,
which is a procedure that accepts two keys and returns true if they
are equivalent, otherwise returns \schfalse{}.
Standard hash tables for arbitrary objects based on the {\cf eq?} and 
{\cf eqv?} predicates (see section~\ref{equivalencesection}) are provided.  
Also, standard hash functions for several types are provided.

This section uses the \var{hash-table} parameter name for arguments
that must be hash tables, and the \var{key} parameter name for
arguments that must be hash-table keys.

\section{Constructors}

\mainindex{hash table}

\begin{entry}{%
\proto{make-eq-hash-table}{}{procedure}
\rproto{make-eq-hash-table}{ \var{k}}{procedure}}

Returns a newly allocated mutable hash table that accepts
arbitrary objects as keys,
and compares those keys with {\cf eq?}. If an argument is given, the initial 
capacity of the hash table is set to approximately \var{k} elements.

\end{entry}

\begin{entry}{%
\proto{make-eqv-hash-table}{}{procedure}
\rproto{make-eqv-hash-table}{ \var{k}}{procedure}}

Returns a newly allocated mutable hash table that accepts
arbitrary objects as keys,
and compares those keys with {\cf eqv?}.
If an argument is given, the initial 
capacity of the hash table is set to approximately \var{k} elements.

\end{entry}

\begin{entry}{%
\proto{make-hash-table}{ \var{hash-function} \var{equiv}}{procedure}
\rproto{make-hash-table}{ \var{hash-function} \var{equiv} \var{k}}{procedure}}

\domain{\var{Hash-function} and \var{equiv} must be procedures.
\var{Hash-function} will be called by other procedures described in
this chapter with a key as argument, and must return a 
non-negative exact integer.
\var{Equiv} will be called by other procedures described in
this chapter with two keys as arguments.}
The {\cf make-hash-table} procedure returns a newly allocated mutable
hash table using \var{hash-function} 
as the hash function and \var{equiv} as the equivalence function used to 
compare keys.
If a third argument is given, the 
initial capacity of the hash table is set to approximately \var{k} elements.

Both the hash function \var{hash-function} and the equivalence
function \var{equiv} should behave like pure functions
on the domain of keys.  For example, the {\cf string-hash}
and {\cf string=?} procedures are permissible only if all
keys are strings and the contents of those strings are never
changed so long as any of them continue to serve as a key in
the hash table.  Furthermore any pair of values for which
the equivalence function \var{equiv} returns true should
be hashed to the same exact integers by 
\var{hash-function}.

\begin{note}
Hash tables are allowed to cache the results of calling the
hash function and equivalence function, so programs cannot
rely on the hash function being called for every lookup or
update.  Furthermore any hash-table operation may call the
hash function more than once.
\end{note}

\begin{rationale}
Hash-table lookups are often followed by updates, so caching
may improve performance.  Hash tables are free to change
their internal representation at any time, which may result
in many calls to the hash function.
\end{rationale}

\end{entry}

\section{Procedures}

\begin{entry}{%
\proto{hash-table?}{ \var{hash-table}}{procedure}}

Returns \schtrue{} if \var{hash-table} is a hash table,
otherwise returns \schfalse{}.
\end{entry}

\begin{entry}{\proto{hash-table-size}{ \var{hash-table}}{procedure}}

Returns the number of keys contained in \var{hash-table} as an exact integer.
\end{entry}

\begin{entry}{%
\proto{hash-table-ref}{ \var{hash-table} \var{key} \var{default}}{procedure}}

Returns the value in \var{hash-table} associated with \var{key}.
If \var{hash-table} does not contain an association for \var{key},
then \var{default} is returned.
\end{entry}

\begin{entry}{\proto{hash-table-set!}{ \var{hash-table} \var{key} \var{obj}}{procedure}}

Changes \var{hash-table} to associate \var{key} with \var{obj},
adding a new association or replacing any existing association for \var{key},
and returns the unspecified value.
\end{entry}

\begin{entry}{\proto{hash-table-delete!}{ \var{hash-table} \var{key}}{procedure}}

Removes any association for \var{key} within \var{hash-table}, and
returns the unspecified value.
\end{entry}

\begin{entry}{\proto{hash-table-contains?}{ \var{hash-table} \var{key}}{procedure}}

Returns \schtrue{} if \var{hash-table} contains an association
for \var{key}, otherwise returns \schfalse{}.
\end{entry}

\begin{entry}{%
\proto{hash-table-update!}{ \var{hash-table} \var{key} \var{proc} \var{default}}{procedure}}

\domain{\var{Proc} must be a procedure that takes a single argument.}
The {\cf hash-table-update!} procedure applies \var{proc} to the value in \var{hash-table}
associated with \var{key}, 
or to \var{default} if \var{hash-table} does not contain an
association for \var{key}.
The \var{hash-table} is then changed to associate \var{key}
with the result of \var{proc}.

The behavior of {\cf hash-table-update!} is equivalent to the
following code, but may be implemented 
more efficiently in cases where the implementation can
avoid multiple lookups of the same key:
\begin{scheme}
(hash-table-set!
 hash-table key
 (proc (hash-table-ref
        hash-table key default)))
\end{scheme}
\end{entry}

\begin{entry}{\proto{hash-table-fold}{ \var{proc} \var{hash-table} \var{init}}{procedure}}

\domain{\var{Proc} must be a procedure that takes three arguments.}
For every association in \var{hash-table}, {\cf hash-table-fold} calls
\var{proc} with the association
key, the association value, and an accumulated value as arguments.
The accumulated value is \var{init} for the first
invocation of \var{proc}, and for subsequent
invocations of \var{proc}, it is the return value
of the previous invocation of \var{proc}. The order
of the calls to \var{proc} is indeterminate. The
return value of {\cf hash-table-fold} is the value of
the last invocation of \var{proc}. If any side
effect is performed on the hash table while a
{\cf hash-table-fold} operation is in progress, then the
behavior of {\cf hash-table-fold} is unspecified.

\end{entry}

\begin{entry}{%
\proto{hash-table-copy}{ \var{hash-table}}{procedure}
\rproto{hash-table-copy}{ \var{hash-table} \var{immutable}}{procedure}}

Returns a copy of \var{hash-table}.  If the
\var{immutable} argument is provided and is true, the returned hash table is immutable;
otherwise it is mutable.

\begin{rationale}
Hash table references may be less expensive with immutable hash tables.
Also, a library may choose to export a hash table which
cannot be changed by clients.
\end{rationale}

\end{entry}
\begin{entry}{%
\proto{hash-table-clear!}{ \var{hash-table}}{procedure}
\rproto{hash-table-clear!}{ \var{hash-table} \var{k}}{procedure}}

Removes all associations from \var{hash-table} and returns the unspecified value.

If a second argument is given, the current
capacity of the hash table is reset to approximately \var{k} elements.
\end{entry}

\begin{entry}{\proto{hash-table-keys}{ \var{hash-table}}{procedure}}

Returns a list of all keys in \var{hash-table}.
The order of the list is unspecified.
Equivalent to:
\begin{scheme}
(hash-table-fold (lambda (k v a) (cons k a)) 
                 hash-table
                 '())
\end{scheme}
\end{entry}

\begin{entry}{\proto{hash-table-values}{ \var{hash-table}}{procedure}}

Returns a list of all values in \var{hash-table}.
The order of the list is unspecified.
Equivalent to:
\begin{scheme}
(hash-table-fold (lambda (k v a) (cons v a)) 
                 hash-table
                 '())
\end{scheme}
\end{entry}

\section{Inspection}

\begin{entry}{\proto{hash-table-equivalence-function}{ \var{hash-table}}{procedure}}

Returns the equivalence function used by
\var{hash-table} to compare keys.  For hash tables
created with {\cf make-eq-hash-table} and {\cf make-eqv-hash-table},
returns {\cf eq?} and {\cf eqv?} respectively.
\end{entry}

\begin{entry}{\proto{hash-table-hash-function}{ \var{hash-table}}{procedure}}

Returns the hash function used by \var{hash-table}.
For hash tables created by {\cf make-eq-hash-table} 
or {\cf make-eqv-hash-table}, \schfalse{} is returned.

\begin{rationale}
The {\cf make-eq-hash-table} and {\cf make-eqv-hash-table} constructors
are designed to hide their hash function.  This allows implementations
to use the machine address of an object as its hash value, rehashing
parts of the table as necessary whenever the garbage collector moves
objects to a different address.
\end{rationale}
\end{entry}

\begin{entry}{\proto{hash-table-mutable?}{ \var{hash-table}}{procedure}}

Returns \schtrue{} if \var{hash-table} is mutable, otherwise returns \schfalse{}.
\end{entry}

\section{Hash functions}

The {\cf equal-hash}, {\cf string-hash}, and {\cf string-ci-hash}
procedures of this section are acceptable as hash functions only
if the keys on which they are called do not suffer side effects
while the hash table remains in use.

\begin{entry}{\proto{equal-hash}{ \var{obj}}{procedure}}

Returns an integer hash value for \var{obj}, based on
its structure and current contents.  This hash function is suitable
for use with {\cf equal?} as an equivalence function.
\end{entry}

\begin{entry}{\proto{string-hash}{ \var{string}}{procedure}}

Returns an integer hash value for \var{string}, based on
its current contents.
This hash function is suitable
for use with {\cf string=?} as an equivalence function.
\end{entry}

\begin{entry}{\proto{string-ci-hash}{ \var{string}}{procedure}}

Returns an integer hash value for \var{string} based on
its current contents, ignoring case.
This hash function is suitable
for use with {\cf string-ci=?} as an equivalence function.
\end{entry}

\begin{entry}{\proto{symbol-hash}{ \var{symbol}}{procedure}}

Returns an integer hash value for \var{symbol}.
\end{entry}

%%% Local Variables: 
%%% mode: latex
%%% TeX-master: "r6rs"
%%% End: 
 \par
\section{Enumerations}
\label{enumerationssection}

This section describes the \deflibrary{r6rs enum} library for dealing with enumerated values
\mainschindex{enumeration}and sets of enumerated values.  Enumerated
values are represented by ordinary symbols, while finite sets of
enumerated values form a separate type, known as the
\defining{enumeration sets}.
The enumeration sets are further partitioned into sets that
share the same \defining{universe} and \defining{enumeration type}.
These universes and enumeration types are created by the
{\cf make-enumeration} procedure.  Each call to that procedure
creates a new enumeration type.

This library interprets each enumeration set with respect to
its specific universe of symbols and enumeration type.
This facilitates efficient implementation of enumeration sets
and enables the complement operation.

In the definition of the following procedures, let \var{enum-set}
range over the enumeration sets, which are defined as the subsets
of the universes that can be defined using {\cf make-enumeration}.

\begin{entry}{%
\proto{make-enumeration}{ list}{procedure}}

{\cf make-enumeration} takes an arbitrary list of symbols,
creates a new enumeration type whose universe consists of
those symbols (in canonical order of their first appearance
in the list) and returns that universe as an enumeration
set whose universe is itself and whose enumeration type is
the newly created enumeration type.
\end{entry}

\begin{entry}{%
\proto{enum-set-universe}{ enum-set}{procedure}}

{\cf enum-set-universe} returns the set of all symbols that comprise
the universe of its argument.
\end{entry}

\begin{entry}{%
\proto{enum-set-indexer}{ enum-set}{procedure}}

{\cf enum-set-indexer} returns a unary procedure that, given a symbol
that is in the universe of \var{enum-set}, returns its 0-origin index
within the canonical ordering of the symbols in the universe; given a
value not in the universe, the unary procedure returns \schfalse.

\begin{scheme}
(let* ((e (make-enumeration '(red green blue)))
       (i (enum-set-indexer e)))
  (list (i 'red) (i 'green) (i 'blue) (i 'yellow))) \lev (0 1 2 \schfalse)
\end{scheme}

{\cf enum-set-indexer} could be defined as follows (using the
{\cf memq} procedure from the \library{r6rs lists} library):

\begin{scheme}
(define (enum-set-indexer set)
  (let* ((symbols (enum-set->list
                    (enum-set-universe set)))
         (cardinality (length symbols)))
    (lambda (x)
      (let ((probe (memq x symbols)))
        (if probe
            (- cardinality (length probe))
            \schfalse)))))
\end{scheme}
\end{entry}

\begin{entry}{%
\proto{enum-set-constructor}{ enum-set}{procedure}}

{\cf enum-set-constructor} returns a unary procedure that, given a
list of symbols that belong to the universe of \var{enum-set}, returns
a subset of that universe that contains exactly the symbols in the
list.  If any value in the list is not a symbol that belongs to the
universe, then the unary procedure raises an exception with
condition type {\cf\&contract}.
\end{entry}

\begin{entry}{%
\proto{enum-set->list}{ enum-set}{procedure}}

{\cf enum-set->list} returns a list of the symbols that belong to its
argument, in the canonical order of the universe of \var{enum-set}.

\begin{scheme}
(let* ((e (make-enumeration '(red green blue)))
       (c (enum-set-constructor e)))
  (enum-set->list (c '(blue red)))) \lev (red blue)
\end{scheme}
\end{entry}

\begin{entry}{%
\proto{enum-set-member?}{ symbol enum-set}{procedure}
\proto{enum-set-subset?}{ \vari{enum-set} \varii{enum-set}}{procedure}
\proto{enum-set=?}{ \vari{enum-set} \varii{enum-set}}{procedure}}

{\cf enum-set-member?} returns \schtrue{} if its first argument is an
element of its second argument, \schfalse{} otherwise.

{\cf enum-set-subset?} returns \schtrue{} if the universe of
\vari{enum-set} is a subset of the universe of \varii{enum-set}
(considered as sets of symbols) and every element of \vari{enum-set}
is a member of its second.  It returns \schfalse{} otherwise.

{\cf enum-set=?} returns \schtrue{} if \vari{enum-set}  is a
subset of \varii{enum-set} and vice versa, as determined by the
{\cf enum-set-subset?} procedure.  This implies that the universes of
the two sets are equal as sets of symbols, but does not imply
that they are equal as enumeration types.  Otherwise, \schfalse{} is
returned.

\begin{scheme}
(let* ((e (make-enumeration '(red green blue)))
       (c (enum-set-constructor e)))
  (list
   (enum-set-member? 'blue (c '(red blue)))
   (enum-set-member? 'green (c '(red blue)))
   (enum-set-subset? (c '(red blue)) e)
   (enum-set-subset? (c '(red blue)) (c '(blue red)))
   (enum-set-subset? (c '(red blue)) (c '(red)))
   (enum-set=? (c '(red blue)) (c '(blue red)))))
\ev (\schtrue{} \schfalse{} \schtrue{} \schtrue{} \schfalse{} \schtrue{})
\end{scheme}
\end{entry}

\begin{entry}{%
\proto{enum-set-union}{ \vari{enum-set} \varii{enum-set}}{procedure}
\proto{enum-set-intersection}{ \vari{enum-set} \varii{enum-set}}{procedure}
\proto{enum-set-difference}{ \vari{enum-set} \varii{enum-set}}{procedure}}


\domain{\vari{enum-set} and \varii{enum-set} shall be enumeration sets 
  that have the same enumeration type.  If their enumeration types
  differ, a {\cf\&contract} violation is raised.}

{\cf enum-set-union} returns the union of \vari{enum-set} and \varii{enum-set}.
{\cf enum-set-intersection} returns the intersection of \vari{enum-set} and \varii{enum-set}.
{\cf enum-set-difference} returns the difference of \vari{enum-set}
and \varii{enum-set}.

\begin{scheme}
(let* ((e (make-enumeration '(red green blue)))
       (c (enum-set-constructor e)))
  (list (enum-set->list
         (enum-set-union (c '(blue)) (c '(red))))
        (enum-set->list
         (enum-set-intersection (c '(red green))
                                (c '(red blue))))
        (enum-set->list
         (enum-set-difference (c '(red green))
                              (c '(red blue))))))
\lev ((red blue) (red) (green))
\end{scheme}
\end{entry}

\begin{entry}{%
\proto{enum-set-complement}{ enum-set}{procedure}}

{\cf enum-set-complement} takes an enumeration set and returns its
complement with respect to its universe.


\begin{scheme}
(let* ((e (make-enumeration '(red green blue)))
       (c (enum-set-constructor e)))
  (enum-set->list
   (enum-set-complement (c '(red)))))
\ev (green blue)
\end{scheme}
\end{entry}

\begin{entry}{%
\proto{enum-set-projection}{ \vari{enum-set} \varii{enum-set}}{procedure}}

{\cf enum-set-projection} projects \vari{enum-set} into the universe
of \varii{enum-set}, dropping any elements of \vari{enum-set} that do
not belong to the universe of \varii{enum-set}.  (If \vari{enum-set}
is a subset of the universe of its second, then no elements are
dropped, and the injection is returned.)

\begin{scheme}
(let ((e1 (make-enumeration
           '(red green blue black)))
      (e2 (make-enumeration
           '(red black white))))
  (enum-set->list
   (enum-set-projection e1 e2))))
\ev (red black)
\end{scheme}
\end{entry}

\begin{entry}{}
\pproto{(define-enumeration \hyper{type-name}}{\exprtype}
\mainschindex{define-enumeration}{\tt\obeyspaces%
  (\hyper{symbol} \dotsfoo)\\
  \hyper{constructor-syntax})}

The {\cf define-enumeration} form defines an enumeration type and
provides two macros for constructing its members and sets of its
members.

A {\cf define-enumeration} form is a definition and can appear
anywhere any other \hyper{definition} can appear.

\hyper{type-name} is an identifier that will be bound to a macro;
\hyper{symbol}~\dotsfoo{} are the symbols that will comprise the
universe of the enumeration (in order).

{\cf (\hyper{type-name} \hyper{symbol})} checks at macro-expansion
time whether \hyper{symbol} is in the universe associated with
\hyper{type-name}.  If it is, then {\cf (\hyper{type-name}
  \hyper{symbol})} is equivalent to {\cf \hyper{symbol}}.  
It is a syntax violation if it is not.

\hyper{constructor-syntax} is an identifier that will be bound to a
macro that, given any finite sequence of the symbols in the universe,
possibly with duplicates, expands into an expression that evaluates
to the enumeration set of those symbols.

{\cf (\hyper{constructor-syntax} \hyper{symbol}~\dotsfoo{})} checks at
macro-expansion time whether every \hyper{symbol}~\dotsfoo{} is in the
universe associated with \hyper{type-name}.  It is a syntax violation
if one or more is not.
Otherwise
\begin{scheme}
(\hyper{constructor-syntax} \hyper{symbol}~\dotsfoo{})
\end{scheme}
%
is equivalent to
%
\begin{scheme}
((enum-set-constructor (\hyper{constructor-syntax}))
 (list '\hyper{symbol}~\dotsfoo{}))\rm.
\end{scheme}

\begin{scheme}
(define-enumeration color
  (black white purple maroon)
  color-set)

(color black)                      \ev black
(color purpel)                     \ev \exception{\&syntax}
(enum-set->list (color-set))       \ev ()
(enum-set->list
 (color-set maroon white))         \ev (white maroon)
\end{scheme}
\end{entry}

%%% Local Variables: 
%%% mode: latex
%%% TeX-master: "r6rs"
%%% End: 

    \par
\chapter{Miscellaneous libraries}
\label{misclibchapter}

\section{{\tt when} and {\tt unless}}

This section describes the \deflibrary{r6rs when-unless} library.

\begin{entry}{%
\proto{when}{ \hyper{test} \hyperi{expression} \hyperii{expression} \dotsfoo}{\exprtype}
\proto{unless}{ \hyper{test} \hyperi{expression} \hyperii{expression} \dotsfoo}{\exprtype}}

\syntax \hyper{Test} must be an expression.
\semantics A {\cf when} expression is evaluated by evaluating the
\hyper{test} expression.  If \hyper{test} evaluates to a true value,
the remaining \hyper{expression}s are evaluated in order, and the
result(s) of the last \hyper{expression} is(are) returned as the
result(s) of the entire {\cf when} expression.  Otherwise, the {\cf
	  when} expression evaluates to the unspecified value.  An {\cf unless}
expression is evaluated by evaluating the \hyper{test} expression.
If \hyper{test} evaluates to false, the remaining
\hyper{expression}s are evaluated in order, and the result(s) of the
last \hyper{expression} is(are) returned as the result(s) of the
entire {\cf unless} expression.  Otherwise, the {\cf unless} expression
evaluates to the unspecified value.

\begin{scheme}
(when (> 3 2) 'greater) \ev greater
(when (< 3 2) 'greater) \ev \theunspecified
(unless (> 3 2) 'less) \ev \theunspecified
(unless (< 3 2) 'less) \ev less
\end{scheme}

The {\cf when} and {\cf unless} expressions are derived forms.  They
could be defined in terms of base library forms by the following macros:

\begin{scheme}
(define-syntax \ide{when}
  (syntax-rules ()
    ((when test result1 result2 ...)
     (if test
         (begin result1 result2 ...)))))

(define-syntax \ide{unless}
  (syntax-rules ()
    ((unless test result1 result2 ...)
     (if (not test)
         (begin result1 result2 ...)))))
\end{scheme}

\end{entry}

\section{{\tt case-lambda}}

This section describes the \deflibrary{r6rs case-lambda} library.

\begin{entry}{%
\proto{case-lambda}{ \hyperi{clause} \hyperii{clause} \dotsfoo}{\exprtype}}
    
\syntax
Each \hyper{clause} should be of the form
%
\begin{scheme}
(\hyper{formals} \hyper{body})%
\end{scheme}

\hyper{Formals} must be as in a {\cf lambda} form
(report section~\ref{report:lambda}), \hyper{body} must be a body according to
report section~\ref{report:bodiessection}.

\semantics A {\cf case-lambda} expression evaluates to a procedure.
This procedure, when applied, tries to match its arguments to the
\hyper{clause}s in order.  The arguments match a clause if one the
following conditions is fulfilled:
%
\begin{itemize}
\item \hyper{Formals} has the form {\tt (\hyper{variable} \dotsfoo)}
and the number of arguments is the same as the number of formal
parameters in \hyper{formals}.
\item \hyper{Formals} has the form\\ {\tt
(\hyperi{variable} \dotsfoo \hypern{variable} . \hyper{variable$_{n+1}$)}
}\\
and the number of arguments is at least $n$.
\item \hyper{Formals} has the form {\tt \hyper{variable}}.
\end{itemize}
%
For the first clause matched by the arguments, the variables of the
\hyper{formals} are bound to fresh locations containing the
argument values in the same arrangement as with {\cf lambda}.

If the arguments match none of the clauses, an exception with condition 
type {\cf\&contract} is raised.

\begin{scheme}
(define foo
  (case-lambda 
   (() 'zero)
   ((x) (list 'one x))
   ((x y) (list 'two x y))
   ((a b c d . e) (list 'four a b c d e))
   (rest (list 'rest rest))))

(foo) \ev zero
(foo 1) \ev (one 1)
(foo 1 2) \ev (two 1 2)
(foo 1 2 3) \ev (rest (1 2 3))
(foo 1 2 3 4) \ev (four 1 2 3 4 ())
\end{scheme}

A sample definition of {\cf case-lambda} in terms of simpler forms is in
appendix~\ref{libderivedformsappendix}.
\end{entry}

\section{Delayed evaluation}

This section describes the \deflibrary{r6rs promises} library.

\begin{entry}{%
\proto{delay}{ \hyper{expression}}{\exprtype}}

\todo{Fix.}

The {\cf delay} construct is used together with the procedure \ide{force} to
implement \defining{lazy evaluation} or \defining{call by need}.
{\tt(delay~\hyper{expression})} returns an object called a
\defining{promise} which at some point in the future may be asked (by
the {\cf force} procedure) \todo{Bartley's white lie; OK?} to evaluate
\hyper{expression}, and deliver the resulting value.
The effect of \hyper{expression} returning multiple values
is unspecified.

\end{entry}

\begin{entry}{%
\proto{force}{ promise}{procedure}}

{\var{Promise} must be a promise.}

Forces the value of \var{promise}.  If no value has been computed for
the promise, then a value is computed and returned.  The value of the
promise is cached (or ``memoized'') so that if it is forced a second
time, the previously computed value is returned.
% without any recomputation.
% [As pointed out by Marc Feeley, the "without any recomputation"
% isn't necessarily true. --Will]

\begin{scheme}
(force (delay (+ 1 2)))   \ev  3
(let ((p (delay (+ 1 2))))
  (list (force p) (force p)))  
                               \ev  (3 3)

(define a-stream
  (letrec ((next
            (lambda (n)
              (cons n (delay (next (+ n 1)))))))
    (next 0)))
(define head car)
(define tail
  (lambda (stream) (force (cdr stream))))

(head (tail (tail a-stream)))  
                               \ev  2%
\end{scheme}

Promises are mainly intended for programs written in
functional style.  The following examples should not be considered to
illustrate good programming style, but they illustrate the property that
only one value is computed for a promise, no matter how many times it is
forced.
% the value of a promise is computed at most once.
% [As pointed out by Marc Feeley, it may be computed more than once,
% but as I observed we can at least insist that only one value be
% used! -- Will]

\begin{scheme}
(define count 0)
(define p
  (delay (begin (set! count (+ count 1))
                (if (> count x)
                    count
                    (force p)))))
(define x 5)
p                     \ev  {\it{}a promise}
(force p)             \ev  6
p                     \ev  {\it{}a promise, still}
(begin (set! x 10)
       (force p))     \ev  6%
\end{scheme}

Here is a possible implementation of {\cf delay} and {\cf force}.
Promises are implemented here as procedures of no arguments,
and {\cf force} simply calls its argument:

\begin{scheme}
(define force
  (lambda (object)
    (object)))%
\end{scheme}

The expression

\begin{scheme}
(delay \hyper{expression})%
\end{scheme}

has the same meaning as the procedure call

\begin{scheme}
(make-promise (lambda () \hyper{expression}))\rm
\end{scheme}

as follows

\begin{scheme}
(define-syntax delay
  (syntax-rules ()
    ((delay expression)
     (make-promise (lambda () expression))))),%
\end{scheme}

where {\cf make-promise} is defined as follows:

% \begin{scheme}
% (define make-promise
%   (lambda (proc)
%     (let ((already-run? \schfalse) (result \schfalse))
%       (lambda ()
%         (cond ((not already-run?)
%                (set! result (proc))
%                (set! already-run? \schtrue)))
%         result))))%
% \end{scheme}

\begin{scheme}
(define make-promise
  (lambda (proc)
    (let ((result-ready? \schfalse)
          (result \schfalse))
      (lambda ()
        (if result-ready?
            result
            (let ((x (proc)))
              (if result-ready?
                  result
                  (begin (set! result-ready? \schtrue)
                         (set! result x)
                         result))))))))%
\end{scheme}

\begin{rationale}
A promise may refer to its own value, as in the last example above.
Forcing such a promise may cause the promise to be forced a second time
before the value of the first force has been computed.
This complicates the definition of {\cf make-promise}.
\end{rationale}

Various extensions to this semantics of {\cf delay} and {\cf force}
are supported in some implementations:

\begin{itemize}
\item Calling {\cf force} on an object that is not a promise may simply
return the object.

\item It may be the case that there is no means by which a promise can be
operationally distinguished from its forced value.  That is, expressions
like the following may evaluate to either \schtrue{} or to \schfalse{},
depending on the implementation:

\begin{scheme}
(eqv? (delay 1) 1)          \ev  \unspecified
(pair? (delay (cons 1 2)))  \ev  \unspecified%
\end{scheme}

\item Some implementations may implement ``implicit forcing'', where
the value of a promise is forced by primitive procedures like \cf{cdr}
and \cf{+}:

\begin{scheme}
(+ (delay (* 3 7)) 13)  \ev  34%
\end{scheme}
\end{itemize}
\end{entry}

\section{Command-line access}
\label{scriptlibsection}

The procedure described in this section is exported by the
\deflibrary{r6rs programs} library.

\begin{entry}{%
\proto{command-line}{}{procedure}}

When a script is being executed, this returns a list of strings with
at least one element.  The first element is an implementation-specific
name for the running script.  The following elements are command-line
arguments according to the operating system's conventions.
\end{entry}

%%% Local Variables: 
%%% mode: latex
%%% TeX-master: "r6rs-lib"
%%% End: 


%%% Local Variables: 
%%% mode: latex
%%% TeX-master: "r6rs-lib"
%%% End: 
 \par
\chapter{Composite library}
\label{complibchapter}

The \deflibrary{r6rs} library is a composite of most of the libraries
described in this report.  The only exceptions are:
%
\begin{itemize}
\item \library{r6rs mutable-pairs} (chapter~\ref{pairmutationchapter})
\item \library{r6rs mutable-strings} (chapter~\ref{stringmutationchapter})
\item \library{r6rs eval} (chapter~\ref{eval})
\item \library{r6rs r5rs} (chapter~\ref{r5rscompatchapter})
\end{itemize}
%
The library exports all procedures and syntactic forms provided by the
component libraries.

All of the bindings exported by \library{r6rs} are exported for both {\cf run}
and {\cf expand}; see report section~\extref{report:phasessection}{Import and export levels}.

%%% Local Variables: 
%%% mode: latex
%%% TeX-master: "r6rs-lib"
%%% End: 
 \par
\chapter{\tt{eval}}
\label{evalchapter}

The \library{r6rs eval} library allows a program to create Scheme
expressions as data at run time and evaluate them.

\begin{entry}{%
\proto{eval}{ expression environment-specifier}{procedure}}

Evaluates \var{expression} in the specified environment and returns its value.
\var{Expression} must be a valid Scheme expression represented as a
datum value, and \var{environment-specifier} must be a 
\defining{library specifier}, which can be created using the {\cf
  environment} procedure described below.

If the first argument to {\cf eval} is determined not to be a syntactically correct
expression, then {\cf eval} must raise an exception with condition
type {\cf \&syntax}.  Specifically, if the first argument to {\cf
  eval} is a definition or a splicing {\cf begin} form containing a
definition, it must raise an exception with condition type {\cf
  \&syntax}.
\end{entry}

\begin{entry}{%
\proto{environment}{ import-spec \dots}{procedure}}

\domain{\var{Import-spec} must be a datum representing an
  \hyper{import spec} (see report
  section~\extref{report:librarysyntaxsection}{Library form}).}
The {\cf environment} procedure returns an environment corresponding
to \var{import-spec}

The bindings of the environment represented by the specifier are
immutable: If {\cf eval} is applied to an expression that is
determined to contain an
assignment to one of the variables of the environment, then {\cf eval} must
raise an exception with a condition type {\cf\&assertion}.

\begin{scheme}
(library (foo)
  (export)
  (import (r6rs))
  (write
    (eval '(let ((x 3)) x)
          (environment '(r6rs))))) \\\> {\it writes} 3

(library (foo)
  (export)
  (import (r6rs))
  (write
    (eval
      '(eval:car (eval:cons 2 4))
      (environment
        '(prefix (only (r6rs) car cdr cons null?)
                 eval:))))) \\\> {\it writes} 2
\end{scheme}
\end{entry}

%%% Local Variables: 
%%% mode: latex
%%% TeX-master: "r6rs-lib"
%%% End: 
    \par
\chapter{Mutable pairs}
\label{pairmutationchapter}

The procedures provided by the \deflibrary{r6rs mutable-pairs} library allow assigning values to
the car and cdr fields of pairs.  In programs that use them, the
criteria for determining the validity of list arguments are more
complex than with only immutable lists.
Section~\ref{mutablelistargumentsection} spells out the definitions
needed for clarifying the specifications of procedures accepting
lists.  Section~\ref{proceduresmutablelistargumentssection} clarifies
the specifications of the procedures of this report using these
definitions.

\section{Procedures}

\begin{entry}{%
\proto{set-car!}{ pair obj}{procedure}}

\nodomain{\var{Pair} must be a pair.}  
Stores \var{obj} in the car field of \var{pair}.
Returns the unspecified value.

\begin{scheme}
(define (f) (list 'not-a-constant-list))
(define (g) '(constant-list))
(set-car! (f) 3)             \ev  \theunspecified
(set-car! (g) 3)             \ev  \unspecified%
          ; should raise \exception{\&contract}
\end{scheme}

Passing an immutable pair to {\cf set-car!} should cause an exception
with condition type {\cf\&contract} to be raised.
\end{entry}


\begin{entry}{%
\proto{set-cdr!}{ pair obj}{procedure}}

\nodomain{\var{Pair} must be a pair.}
Stores \var{obj} in the cdr field of \var{pair}.
Returns the unspecified value.

Passing an immutable pair to {\cf set-cdr!} should cause an exception
with condition type {\cf\&contract} to be raised.

\begin{scheme}
(let ((x (list 'a 'b 'c 'a))
      (y (list 'a 'b 'c 'a 'b 'c 'a)))
  (set-cdr! (list-tail x 2) x)
  (set-cdr! (list-tail y 5) y)
  (list
   (equal? x x)
   (equal? x y)
   (equal? (list x y 'a) (list y x 'b)))) \lev  (\schtrue{} \schtrue{} \schfalse{})
\end{scheme}
\end{entry}

\section{Mutable list arguments}
\label{mutablelistargumentsection}

Through the {\cf set-car!} and {\cf set-cdr!} procedures, lists are
mutable in Scheme, so a pair that is the head of a list at one moment
may not always be the head of a list:
%
\begin{scheme}
(define x (list 'a 'b 'c))
(define y x)
y                       \ev  (a b c)
(list? y)               \ev  \schtrue
(set-cdr! x 4)          \ev  \theunspecified
x                       \ev  (a . 4)
(eqv? x y)              \ev  \schtrue
y                       \ev  (a . 4)
(list? y)               \ev  \schfalse
(set-cdr! x x)          \ev  \theunspecified
(list? x)               \ev  \schfalse%
\end{scheme}

Any procedures defined in this
report specified as accepting a list argument must check if that
argument indeed appears to be a list.  This checking is complicated by
the fact that many procedures accepting lists such as {\cf map} or
{\cf filter} call arbitrary procedures that are passed as arguments.
These argument procedures may mutate the list while it is being
traversed.  Moreover, in the presence of concurrent evaluation,
whether a pair is the head of a list is not computable in general.

Consequently, procedures like {\cf length} are only required to check
that a list argument is a \textit{plausible list\index{plausible
    list}}.  Informally, a plausible list is an object that appears as
a list during a sequential traversal, where that traversal must also
detect a cycle.  In particular, a plausible immutable list is always a list.
A more formal definition follows:

Plausible lists are defined with respect to the time interval between
the time an argument is passed to the specified procedure and the
first return of a value to that procedure's continuation.

The times are in any global time that satisfies the axioms proposed in
chapter 2 of MIT AI TR-633~\cite{AITR633}.

\begin{note}
In most implementations,
the definitions above are believed to be invariant under
transformations of global time that are allowed by those axioms.
\end{note}

A \textit{plausible list up to $n$ between times $t_0$ and $t_n$} is a
Scheme value $x$ such that
%
\begin{enumerate}
\item $x$ is a pair, and $n$ is $0$; or
\item $x$ is the empty list, and $n$ is $0$; or
\item $x$ is a pair $p$, $n > 0$, and there exists some time
  $t_1$ in $(t_0,t_n]$ such that taking the cdr of $p$ at
  time $t_1$ yields a plausible list up to $n-1$ between
  times $t_1$ and $t_n$.
\end{enumerate}

A \textit{plausible list up to and including $n$} is a plausible list
up to $n$ where case 2 does not apply either directly or through
recursion.

A \textit{plausible list of length $n$ between times $t_0$ and $t_n$}
is a Scheme value $x$ such that

\begin{enumerate}
\item $x$ is the empty list, and $n$ is $0$; or
\item $x$ is a pair $p$, $n > 0$, and there exists some time
  $t_1$ in $(t_0,t_n]$ such that taking the cdr of $p$ at
  time $t_1$ yields a plausible list of length $n-1$
  between times $t_1$ and $t_n$.
\end{enumerate}

A \textit{plausible prefix of length $n$ between times $t_0$ and
  $t_n$} is a sequence of Scheme values $x_0,\ldots,x_n$ and strictly
increasing times $t_1,\ldots,t_n$ such that $x_0$ through $x_{n-1}$
are pairs, $x_n$ is either the empty list or a pair, and taking the
cdr of a pair $x_{i-1}$ at time $t_i$ yields $x_i$.

\textit{A plausible alist up to $n$ between times $t_0$ and $t_n$} is
a plausible list up to $n$ between $t_0$ and $t_n$ such that, for all
possible choices of the times $t_1$ and pairs $p$ mentioned in part
(3) above, there exists a time $t_2$ such that $t_1 < t_2 < t_n$ and
the car of $p$ at time $t_2$ is a pair.

\textit{A plausible alist of length $n$} is defined similarly, as is
\textit{a plausible alist prefix of length $n$}.

\textit{A plausible list (alist) between times $t_0$ and $t_n$} is a
plausible list (alist) of some length $n$ between those times.

\section{Procedures with list arguments}
\label{proceduresmutablelistargumentssection}

This section contains clarifications to the domains of the procedures of the base
library and the list-utilities library.

\subsection{Base-library procedures}

These are clarifications to the domains of the procedures of the base
library described in sections~\ref{listsection} and \ref{controlsection}:

\begin{entry}{%
\irproto{list?}{ obj}{procedure}}

Returns \schtrue{} if \var{obj} is a plausible list, otherwise returns
\schfalse{}.
\end{entry}

\begin{entry}{%
\irproto{list-tail}{ l \vr{k}}{procedure}}

\domain{\var{L} must be a plausible list up to \var{k}.}
\end{entry}

\begin{entry}{%
\irproto{list-ref}{ l \vr{k}}{procedure}}

\domain{\var{L} must be a plausible list up to and including \var{k}.}
\end{entry}

\begin{entry}{%
\irproto{map}{ proc \vari{list} \varii{list} \dotsfoo}{procedure}}

\domain{The \var{list}s must be plausible lists, and \var{proc} must be a
procedure taking as many arguments as there are {\it list}s
and returning a single value.  If more
than one \var{list} is given, a natural number $n$ must exist such
that all \varj{list} are plausible lists of length $n$.}
\end{entry}

\begin{entry}{%
\irproto{list->string}{ list}{procedure}}

\domain{\var{List} must be a plausible list where, for every
  natural number $n$ and for every plausible prefix $x_i$ of that argument
  of length $n$, there exists a time $t$ with $t_i < t <
  t_r$, where $t_r$ is the time of first return from {\cf
    list\coerce{}string}, for which the car of $x_i$ is a character.}
\end{entry}

\begin{entry}{%
\irproto{apply}{ proc \vari{arg} $\ldots$ args}{procedure}}
\domain{\var{Proc} must be a procedure and \var{args} must be a
  plausible list.}
\end{entry}

\subsection{List utilities}

These are clarifications to the domains of the procedures of the list-utilities
library described in chapter~\ref{listutilities}:

\begin{entry}{%
\irproto{forall}{ proc \vari{l} \varii{l} \dotsfoo}{procedure}
\irproto{exists}{ proc \vari{l} \varii{l} \dotsfoo}{procedure}}

\domain{For the \var{l}s, there must exist a natural number $n$
  such that all \var{l}s are plausible lists of length $n$, they
  must be values of according the conditions specified below.
  \var{Proc} must be a procedure taking as many arguments as there are
  \var{l}s.}

If the \var{l}s are not as specified, then a natural number $n$ must
exist where each \varj{l} is the first Scheme value of a plausible
prefix of length $n$ such that the car of the last value $x_n$ of that
prefix satisfies the given condition (i.e.\ the procedure returns a
false value with {\cf forall} or a true value with {\cf exists}) at
some time after $t_n$ and before the procedure returns.
\end{entry}

\begin{entry}{%
\irproto{memp}{ proc l}{procedure}
\irproto{member}{ obj l}{procedure}
\irproto{memv}{ obj l}{procedure}
\irproto{memq}{ obj l}{procedure}
}

\domain{\var{Proc} must be a procedure taking a single argument.
  \var{L} must be a plausible list or a value according to the
  conditions specified below.}

If \var{l} is not a plausible list, then it must be such that a
natural number $n$ exists where \var{l} is the first Scheme value of a
plausible prefix of length $n$ such that the car of the last value $x_n$ of that
prefix satisfies the given condition at some time after $t_n$ and before
the procedure returns.
\end{entry}

\begin{entry}{%
\irproto{assp}{ proc al}{procedure}
\irproto{assoc}{ obj al}{procedure}
\irproto{assv}{ obj al}{procedure}
\irproto{assq}{ obj al}{procedure}}

\domain{\var{Al} (for ``association list'') must be a a plausible
  alist or a value according to the conditions specified below.
  \var{Proc} must be a procedure taking a single argument.}

If \var{al} is not a plausible alist, then a natural number $n$ must
exist such that \var{al} is the first Scheme value of a plausible
prefix of length $n$ such that every Scheme value $x_1$ through $x_n$
of that prefix is a pair, and $x_n$ has a pair as its car at some time
after $t_n$, and at some time after that the car of that pair is the
first argument, all before the procedure returns.
\end{entry}

%%% Local Variables: 
%%% mode: latex
%%% TeX-master: "r6rs"
%%% End: 
  \par
\chapter{R$^5$RS compatibility}
\label{r5rscompatchapter}

The features described in this chapter are exported from the
\defrsixlibrary{r5rs} library and provide some functionality of the
preceding revision of this report~\cite{R5RS} that was omitted from
the main part of the current report.

\begin{entry}{%
\proto{exact->inexact}{ z}{procedure}
\proto{inexact->exact}{ z}{procedure}}

These are the same as the {\cf inexact} and {\cf exact}
procedures; see report section~\extref{report:inexact}{Generic conversions}.
\end{entry}

\begin{entry}{%
\proto{quotient}{ \vari{n} \varii{n}}{procedure}
\proto{remainder}{ \vari{n} \varii{n}}{procedure}
\proto{modulo}{ \vari{n} \varii{n}}{procedure}}

These procedures implement number-theoretic (integer)
division.  \varii{N} must be non-zero.  All three procedures
return integer objects.  If \vari{n}/\varii{n} is an integer object:
\begin{scheme}
    (quotient \vari{n} \varii{n})   \ev \vari{n}/\varii{n}
    (remainder \vari{n} \varii{n})  \ev 0
    (modulo \vari{n} \varii{n})     \ev 0
\end{scheme}
If \vari{n}/\varii{n} is not an integer object:
\begin{scheme}
    (quotient \vari{n} \varii{n})   \ev \var{n$_q$}
    (remainder \vari{n} \varii{n})  \ev \var{n$_r$}
    (modulo \vari{n} \varii{n})     \ev \var{n$_m$}
\end{scheme}
where \var{n$_q$} is $\vari{n}/\varii{n}$ rounded towards zero,
$0 < |\var{n$_r$}| < |\varii{n}|$, $0 < |\var{n$_m$}| < |\varii{n}|$,
\var{n$_r$} and \var{n$_m$} differ from \vari{n} by a multiple of \varii{n},
\var{n$_r$} has the same sign as \vari{n}, and
\var{n$_m$} has the same sign as \varii{n}.

Consequently, for integer objects \vari{n} and \varii{n} with
\varii{n} not equal to 0,
\begin{scheme}
     (= \vari{n} (+ (* \varii{n} (quotient \vari{n} \varii{n}))
           (remainder \vari{n} \varii{n})))
                                 \ev  \schtrue%
\end{scheme}
provided all number object involved in that computation are exact.

\begin{scheme}
(modulo 13 4)           \ev  1
(remainder 13 4)        \ev  1

(modulo -13 4)          \ev  3
(remainder -13 4)       \ev  -1

(modulo 13 -4)          \ev  -3
(remainder 13 -4)       \ev  1

(modulo -13 -4)         \ev  -1
(remainder -13 -4)      \ev  -1

(remainder -13 -4.0)    \ev  -1.0%
\end{scheme}

\begin{note}
  These procedures could be defined in terms of {\cf div} and {\cf
    mod} (see report section~\extref{report:div}{Arithmetic operations}) as follows (without checking of the
  argument types):
\begin{scheme}
(define (sign n)
  (cond
    ((negative? n) -1)
    ((positive? n) 1)
    (else 0)))

(define (quotient n1 n2)
  (* (sign n1) (sign n2) (div (abs n1) (abs n2))))

(define (remainder n1 n2)
  (* (sign n1) (mod (abs n1) (abs n2))))

(define (modulo n1 n2)
  (* (sign n2) (mod (* (sign n2) n1) (abs n2))))
\end{scheme}
\end{note}
\end{entry}

\begin{entry}{%
\proto{delay}{ \hyper{expression}}{\exprtype}}

The {\cf delay} construct is used together with the procedure \ide{force} to
implement \defining{lazy evaluation} or \defining{call by need}.
{\tt(delay~\hyper{expression})} returns an object called a
\defining{promise} which at some point in the future may be asked (by
the {\cf force} procedure) to evaluate
\hyper{expression}, and deliver the resulting value.
The effect of \hyper{expression} returning multiple values
is unspecified.

\end{entry}

\begin{entry}{%
\proto{force}{ promise}{procedure}}

{\var{Promise} must be a promise.}

Forces the value of \var{promise}.  If no value has been computed for
the promise, then a value is computed and returned.  The value of the
promise is cached (or ``memoized'') so that if it is forced a second
time, the previously computed value is returned.

\begin{scheme}
(force (delay (+ 1 2)))   \ev  3
(let ((p (delay (+ 1 2))))
  (list (force p) (force p)))  
                               \ev  (3 3)

(define a-stream
  (letrec ((next
            (lambda (n)
              (cons n (delay (next (+ n 1)))))))
    (next 0)))
(define head car)
(define tail
  (lambda (stream) (force (cdr stream))))

(head (tail (tail a-stream)))  
                               \ev  2%
\end{scheme}

Promises are mainly intended for programs written in
functional style.  The following examples should not be considered to
illustrate good programming style, but they illustrate the property that
only one value is computed for a promise, no matter how many times it is
forced.

\begin{scheme}
(define count 0)
(define p
  (delay (begin (set! count (+ count 1))
                (if (> count x)
                    count
                    (force p)))))
(define x 5)
p                     \ev  {\it{}a promise}
(force p)             \ev  6
p                     \ev  {\it{}a promise, still}
(begin (set! x 10)
       (force p))     \ev  6%
\end{scheme}

Here is a possible implementation of {\cf delay} and {\cf force}.
Promises are implemented here as procedures of no arguments,
and {\cf force} simply calls its argument:

\begin{scheme}
(define force
  (lambda (object)
    (object)))%
\end{scheme}

The expression

\begin{scheme}
(delay \hyper{expression})%
\end{scheme}

has the same meaning as the procedure call

\begin{scheme}
(make-promise (lambda () \hyper{expression}))%
\end{scheme}

as follows

\begin{scheme}
(define-syntax delay
  (syntax-rules ()
    ((delay expression)
     (make-promise (lambda () expression))))),%
\end{scheme}

where {\cf make-promise} is defined as follows:

\begin{scheme}
(define make-promise
  (lambda (proc)
    (let ((result-ready? \schfalse)
          (result \schfalse))
      (lambda ()
        (if result-ready?
            result
            (let ((x (proc)))
              (if result-ready?
                  result
                  (begin (set! result-ready? \schtrue)
                         (set! result x)
                         result))))))))%
\end{scheme}
\end{entry}

\begin{entry}{%
\proto{null-environment}{ n}{procedure}}

\domain{\var{N} must be the exact integer object 5.}  The {\cf
  null-environment} procedure returns an
environment specifier suitable for use with {\cf eval} (see
chapter~\ref{evalchapter}) representing an environment that is empty except
for the (syntactic) bindings for all keywords described in
the previous revision of this report~\cite{R5RS}.
\end{entry}

\begin{entry}{%
\proto{scheme-report-environment}{ n}{procedure}}

\domain{\var{N} must be the exact integer object 5.}  The {\cf scheme-report-environment} procedure returns
an environment specifier for an environment that is empty except for
the bindings for the identifiers described in the previous
revision of this report~\cite{R5RS}, omitting {\cf load}, {\cf
  interaction-environment}, {\cf
  transcript-on}, {\cf transcript-off}, and {\cf char-ready?}.  The
bindings have as values the procedures of the same names described in
this report.
\end{entry}


%%% Local Variables: 
%%% mode: latex
%%% TeX-master: "r6rs-lib"
%%% End: 
 \par
\chapter{Sample definitions for derived forms}
\label{libderivedformsappendix}

\subsubsection*{{\tt case-lambda}}

The {\cf case-lambda} keyword (see section~\ref{case-lambda})
could be defined in terms of base library by the following macros:
%
\begin{scheme}
(define-syntax \ide{case-lambda}
  (syntax-rules ()
    ((case-lambda
      (formals-0 body0-0 body1-0 ...)
      (formals-1 body0-1 body1-1 ...)
      ...)
     (lambda args
       (let ((l (length args)))
         (case-lambda-helper
          l args
          (formals-0 body0-0 body1-0 ...)
          (formals-1 body0-1 body1-1 ...) ...))))))

(define-syntax \ide{case-lambda-helper}
  (syntax-rules ()
    ((case-lambda-helper
      l args
      ((formal ...) body ...)
      clause ...)
     (if (= l (length '(formal ...)))
         (apply (lambda (formal ...) body ...)
                args)
         (case-lambda-helper l args clause ...)))
    ((case-lambda-helper
      l args
      ((formal . formals-rest) body ...)
      clause ...)
     (case-lambda-helper-dotted
       l args
       (body ...)
       (formal . formals-rest)
       formals-rest 1
       clause ...))
    ((case-lambda-helper
      l args
      (formal body ...))
     (let ((formal args))
       body ...))))

(define-syntax \ide{case-lambda-helper-dotted}
  (syntax-rules ()
    ((case-lambda-helper-dotted
      l args
      (body ...)
      formals
      (formal . formals-rest) k
      clause ...)
     (case-lambda-helper-dotted
      l args
      (body ...)
      formals
      formals-rest (+ 1 k)
      clause ...))
    ((case-lambda-helper-dotted
      l args
      (body ...)
      formals
      rest-formal k
      clause ...)
     (if (>= l k)
         (apply (lambda formals body ...) args)
         (case-lambda-helper
          l args clause ...)))))
\end{scheme}

%%% Local Variables: 
%%% mode: latex
%%% TeX-master: "r6rs-lib"
%%% End: 
 \par
%\newpage                   %  Put bib on it's own page (it's just one)
%\twocolumn[\vspace{-.18in}]%  Last bib item was on a page by itself.
\renewcommand{\bibname}{References}

\bibliographystyle{plain}
\bibliography{abbrevs,rrs}

\vfill\eject


\newcommand{\indexheading}{Alphabetic index of definitions of
  concepts, keywords, and procedures}
\newcommand{\indexintro}{}

\printindex

\end{document}

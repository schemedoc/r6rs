\chapter{Records}
\label{recordschapter}
\mainindex{record}
This section describes abstractions for creating new data types
representing records.

A record is a compound data structure with a fixed number of
components, called \textit{fields}\mainindex{field}.  Each record has
an associated type specified by a \defining{record-type descriptor},
which is an object that specifies the fields of the record and various
other properties that all records of that type share.  Record objects
are created by a \defining{record constructor}, a procedure that
creates a fresh record object and initializes its fields to values.
Records of different types can be distinguished from each other and
from other types of objects by \textit{record predicates}. A record predicate
returns \schtrue{} when passed a record of the type specified by the
record-type descriptor and \schfalse{} otherwise.
An \defining{accessor} extracts from a record
the component associated with a field, and a \defining{mutator}
changes the component to a different value.

Record types can be extended via single inheritance, allowing record
types to model hierarchies that occur in applications like algebraic
data types as well as single-inheritance class systems.  If a record
type \var{t} extends another record type \var{p}, each record of type
\var{t} is also a record of type \var{p}, and the predicate,
accessors, and mutators applicable to a record of type \var{p} are
also applicable to a record of type \var{t}.  The extension
relationship is transitive in the sense that a type extends its
parent's parent, if any, and so on.  A record type that does not
extend another record type is called a \defining{base record type}.

A
record type can be \defining{sealed} to prevent it from being
extended.  Moreover, a record type can be \defining{nongenerative},
i.e., it is globally identified by a ``uid'', and new, compatible
definitions of a nongenerative record type with the same uid as a
previous always yield the same record type.

The record mechanism spans three libraries:

\begin{itemize}
\item the \rsixlibrary{records syntactic} library,
  a syntactic layer for defining a record type and
  associated constructor, predicate, accessor, and mutators,
\item the \rsixlibrary{records procedural} library,
  a procedural layer for creating and manipulating record types and creating
  constructors, predicates, accessors, and mutators;
\item the \rsixlibrary{records inspection} library,
  a set of inspection procedures.
\end{itemize}
% 
The inspection procedures allow programs to obtain from a record
instance a descriptor for the type and from there obtain access to the
fields of the record instance. This facility allows the creation of
portable printers and inspectors.  A program may prevent access to a
record's type---and thereby protect the information stored in the
record from the inspection mechanism---by declaring the type opaque.
Thus, opacity as presented here can be used to enforce abstraction
barriers.

Any of the standard types mentioned in this report may or may not be
implemented as an opaque record type.  Thus, it may be possible to use
inspection on objects of the standard types.

The procedural layer is particularly useful for writing interpreters
that construct host-compatible record types.  It may also serve as a
target for expansion of the syntactic layers.  The
record operations provided through the procedural layer may, however, be
less efficient than the operations provided through the
syntactic layer, which is designed to allow expand-time determination
of record-instance sizes and field offsets.  Therefore, alternative implementations
of syntactic record-type definition should, when possible, expand into
the syntactic layer rather than the procedural layer.

The syntactic layer is used more commonly and therefore described
first.  This chapter uses the \var{rtd} and
\var{constructor-descriptor} parameter names for arguments that must
be record-type descriptors and constructor descriptors, respectively
(see section~\ref{recordsproceduralsection}).


\section{Mutability and equivalence of records}
\label{recordsequivalencesection}

The fields of a record type are designated \textit{mutable} or
\textit{immutable}.  Correspondingly, a record type with no mutable
field is called \textit{immutable}\mainindex{immutable record
  type}\mainindex{mutable record type}, and all records of that type
are immutable objects.  All other record types are \textit{mutable},
and so are their records.

Each call to a record constructor returns a new record with a fresh
location (see report section~\extref{report:storagemodel}{Storage
  model}).  Consequently, for two records \vari{obj} and \varii{obj},
the return value of {\cf (eqv? \vari{obj} \varii{obj})}, as well as the
return value of {\cf (eq? \vari{obj} \varii{obj})}, adheres to
the following criteria (see report
section~\extref{report:eqv?}{Equivalence predicates}):

\begin{itemize}
\item If \vari{obj} and \varii{obj} have different record types (i.e.,
  their record-type descriptors are not {\cf eqv?}), {\cf eqv?}
  returns \schfalse.
\item If \vari{obj} and \varii{obj} are both records of the
  same record type, and are the results of two separate calls to
  record constructors, then {\cf eqv?} returns \schfalse.
\item If \vari{obj} and \varii{obj} are both the result of a single call to a
  record constructor, then {\cf eqv?} returns \schtrue.
\item If \vari{obj} and \varii{obj} are both records of the same
  record type, where applying an accessor to both yields results
  for which {\cf eqv?} returns \schfalse, then {\cf eqv?} returns \schfalse.
\end{itemize}


\section{Syntactic layer}
\label{recordssyntacticsection}

The syntactic layer is provided by the \defrsixlibrary{records
  syntactic} library.  Some details of the specification are
explained in terms of the specification of the procedural layer below.

The record-type-defining form {\cf define-record-type} is a definition and
can appear anywhere any other \hyper{definition} can appear.

\begin{entry}{%
\proto{define-record-type}{ \hyper{name spec} \arbno{\hyper{record clause}}}{\exprtype}
\litproto{fields}
\litproto{mutable}
\litproto{immutable}
\litproto{parent}
\litproto{protocol}
\litproto{sealed}
\litproto{opaque}
\litproto{nongenerative}
\litproto{parent-rtd}}

A {\cf define-record-type} form defines a record type along with
associated constructor descriptor and constructor, predicate, field
accessors, and field mutators. The {\cf define-record-type} form expands into
a set of definitions in the environment where {\cf define-record-type}
appears; hence, it is possible to refer to the bindings (except for
that of the record type itself) recursively.

The \hyper{name spec} specifies the names of the record type,
constructor, and predicate. It must take one of the following
forms:

\begin{scheme}
(\hyper{record name} \hyper{constructor name} \hyper{predicate name})
\hyper{record name}%
\end{scheme}

\hyper{Record name}, \hyper{constructor name}, and \hyper{predicate
  name} must all be identifiers.

\hyper{Record name}, taken as a symbol, becomes the name of the record
type. (See the description of {\cf make-record-type-descriptor}
below.)
Additionally, it is bound by this definition to an expand-time or run-time
representation of the record type and can be used as parent name in
syntactic record-type definitions that extend this definition. It can
also be used as a handle to gain access to the underlying record-type
descriptor and constructor descriptor (see {\cf
  record-type-descriptor} and {\cf record-constructor-descriptor}
below).

\hyper{Constructor name} is defined by this definition to be a
constructor for the defined record type, with a protocol specified by
the {\cf protocol} clause, or, in its absence, using a default protocol. For
details, see the description of the {\cf protocol} clause below.

\hyper{Predicate name} is defined by this definition to a predicate
for the defined record type.

The second form of \hyper{name spec} is an abbreviation for the first
form, where the name of the constructor is generated by prefixing the
record name with {\tt make-}, and the predicate name is generated by
adding a question mark ({\tt ?}) to the end of the record name. For
example, if the record name is {\tt frob}, the name of the
constructor is {\tt make-frob}, and the predicate name is
{\tt frob?}.

Each \hyper{record clause} must take one of the following forms; it is
a syntax violation if multiple \hyper{record clause}s of the same kind appear in a
{\cf define-record-type} form.

{\cf (fields \arbno{\hyper{field spec}})}

Each \hyper{field spec} has one of the following forms
  
\begin{scheme}
(immutable \hyper{field name} \hyper{accessor name})
(mutable \hyper{field name}
         \hyper{accessor name} \hyper{mutator name})
(immutable \hyper{field name})
(mutable \hyper{field name})
\hyper{field name}%
\end{scheme}

\hyper{Field name}, \hyper{accessor name}, and \hyper{mutator name}
must all be identifiers. The first form declares an immutable field
called \hyper{field name}, with the corresponding accessor named
\hyper{accessor name}. The second form declares a mutable field called
\hyper{field name}, with the corresponding accessor named
\hyper{accessor name}, and with the corresponding mutator named
\hyper{mutator name}.

If \hyper{field spec} takes the third or fourth form, the accessor
name is generated by appending the record name and field name with a
hyphen separator, and the mutator name (for a mutable field) is
generated by adding a {\tt -set!} suffix to the accessor name. For
example, if the record name is {\tt frob} and the field name is {\tt
  widget}, the accessor name is {\tt frob-widget} and the mutator name
is {\tt frob-widget-set!}.

If \hyper{field spec} is just a \hyper{field name} form, it is an
abbreviation for {\cf (immutable \hyper{field name})}.

The \hyper{field name}s become, as symbols, the names of the fields in
the record-type descriptor being created, in the same order.

The {\cf fields} clause may be absent; this is equivalent to an empty
{\cf fields} clause.

{\cf (parent \hyper{parent name})}
   
Specifies that the record type is to have parent type \hyper{parent
  name}, where \hyper{parent name} is the \hyper{record name} of a
record type previously defined using {\cf define-record-type}. 
The record-type definition
associated with \hyper{parent name} must not be sealed.

{\cf (protocol \hyper{expression})}
   
\hyper{Expression} is evaluated in the same environment as the
{\cf define-record-type} form, and must evaluate to a protocol appropriate
for the record type being defined.

The protocol is used to create a record-constructor descriptor as
described below.  If no {\cf protocol} clause is specified, a
constructor descriptor is still created using a default protocol.  The
clause can be absent only if the record type being defined has no parent
type, or if the parent definition does not specify a protocol.

{\cf (sealed \schtrue)}\\
{\cf (sealed \schfalse)}
   
If this option is specified with operand \schtrue, the defined record
type is sealed, i.e., no extensions of the record type can be created.
If this option is specified with operand \schfalse, or is absent, the
defined record type is not sealed.

{\cf (opaque \schtrue)}\\
{\cf (opaque \schfalse)}
   
If this option is specified with operand \schtrue, or if an opaque
parent record type is specified, the defined record type is opaque.
Otherwise, the defined record type is not opaque.  See the
specification of {\cf record-rtd} below for details.
   
{\cf (nongenerative \hyper{uid})}\\
{\cf (nongenerative)}
   
This specifies that the record type is nongenerative with uid
\hyper{uid}, which must be an \hyper{identifier}.
If \hyper{uid} is absent, a unique uid is generated at macro-expansion time.
If two record-type definitions specify the same \var{uid}, then
the record-type definitions should be equivalent, i.e.,
the implied arguments to {\cf make-record-type-descriptor}
must be equivalent as described under {\cf
  make-record-type-descriptor}.  See section~\ref{make-record-type-descriptor}.
If this condition is not met, it is either considered a syntax violation or
an exception with condition type {\cf\&assertion} is raised.
If the condition is met, a single record type is generated for both
definitions.

In the absence of a {\cf nongenerative} clause, a new record type is
generated every time a {\cf define-record-type} form is evaluated:

\begin{scheme}
(let ((f (lambda (x)
           (define-record-type r \ldots)
           (if x r? (make-r \ldots)))))
  ((f \schtrue) (f \schfalse))) \ev \schfalse{}
\end{scheme}

{\cf (parent-rtd \hyper{parent rtd} \hyper{parent cd})}

Specifies that the record type is to have its parent type specified by
\hyper{parent rtd}, which should be an expression evaluating to a
record-type descriptor or \schfalse, and \hyper{parent cd}, which should be an
expression evaluating to a constructor descriptor (see below) or
\schfalse.

If \hyper{parent rtd} evaluates to \schfalse, then if \hyper{parent cd}
evaluates to a value, that value must be \schfalse.

If \hyper{parent rtd} evaluates to a record-type descriptor, the
record type
must not be sealed.  Moreover, a record-type definition must not have
both a {\cf parent} and a {\cf parent-rtd} clause.

\begin{note}
  The syntactic layer is designed to allow record-instance sizes and field
  offsets to be determined at expand time, i.e., by a macro definition of
  {\cf define-record-type}, as long as the parent (if any) is known.
  Implementations that take advantage of this may generate less
  efficient constructor, accessor, and mutator code when the
  {\cf parent-rtd} clause is used, since the type of the parent is
  generally not known until run time.
  The {\cf parent} clause should therefore be used instead when possible.
\end{note}

All bindings created by {\cf define-record-type} (for the record type,
the constructor, the predicate, the accessors, and the
mutators) must have names that are pairwise distinct.

If
no {\cf parent} clause is present, no {\cf parent-rtd} clause
is present, or a {\cf parent-rtd} clause is present but
\hyper{parent rtd} evaluates to \schfalse,
the record type is a base type.  

The constructor created by a {\cf define-record-type} form is a
procedure as follows:
%
\begin{itemize}
\item If the record type is a base type and no {\cf protocol} clause is present,
  the constructor accepts as many arguments as there are fields, in
  the same order as they appear in the {\cf fields} clause, and
  returns a record object with the fields initialized to the
  corresponding arguments.
\item If the record type is a base type and a {\cf protocol} clause is present,
  the protocol expression, if it evaluates to a value,
  must evaluate to a procedure that accepts a
  single argument.  The protocol procedure is called once during the
  evaluation of the {\cf define-record-type} form with a
  procedure \var{p} as its argument.  It should return a procedure,
  which will become the constructor bound to \hyper{constructor name}.
  The procedure \var{p} accepts as many arguments as there are fields,
  in the same order as they appear in the {\cf fields} clause, and
  returns a record object with the fields initialized to the
  corresponding arguments.

  The constructor returned by the protocol procedure can accept an
  arbitrary number of arguments, and should call \var{p} once to
  construct a record object, and return that record object.

  For example, the following protocol expression for a record-type
  definition with three fields creates a constructor that accepts 
  values for all fields, and initialized them in the reverse order of
  the arguments:
%
\begin{scheme} 
(lambda (p)
  (lambda (v1 v2 v3)
    (p v3 v2 v1)))%
 \end{scheme}

\item If the record type is not a base type and a {\cf protocol}
  clause is present, then the protocol procedure is called once with a procedure
  \var{n} as its argument.  As in the previous case, the protocol
  procedure should return a procedure, which will become the
  constructor bound to \hyper{constructor name}.  However, \var{n} is
  different from \var{p} in the previous case: It accepts arguments
  corresponding to the arguments of the constructor of the parent
  type.  It then returns a procedure \var{p} that accepts as many
  arguments as there are (additional) fields in this type, in the same order as in
  the {\cf fields} clause, and returns a record object with the fields
  of the parent record types initialized according to their
  constructors and the arguments to \var{n}, and the fields of
  this record type initialized to its arguments of \var{p}.

  The constructor returned by the protocol procedure can accept an
  arbitrary number of arguments, and should call \var{n} once to
  construct the procedure \var{p}, and call \var{p} once to create the
  record object, and finally return that record object.

  For example, the following protocol expression assumes that the
  constructor of the parent type takes three arguments:
\begin{scheme}
(lambda (n)
  (lambda (v1 v2 v3 x1 x2 x3 x4)
    (let ((p (n v1 v2 v3)))
      (p x1 x2 x3 x4))))%
\end{scheme}
The resulting constructor accepts seven arguments, and initializes
the fields of the parent types according to the constructor of the
parent type, with {\cf v1}, {\cf v2}, and {\cf v3} as arguments.  It
also initializes the fields of this record type to the values of {\cf
  x1}, \ldots, {\cf x4}.

\item If there is a {\cf parent} clause, but no {\cf protocol} clause,
  then the parent type must not have a
  {\cf protocol} clause itself.
  Similarly, if there is a {\cf parent-rtd} clause whose \hyper{parent rtd}
  evaluates to a record-type descriptor, but no {\cf protocol} clause,
  then the \hyper{parent cd} expression, if it evaluates to a value,
  must evaluate to \schfalse.
  The constructor bound to
  \hyper{constructor name} is a procedure that accepts arguments corresponding to the 
  parent types' constructor first, and then one argument for each field in the same
  order as in the {\cf fields} clause. The constructor
  returns a record object with the fields initialized to the corresponding
  arguments.
\end{itemize}

A protocol may perform other actions consistent with the requirements
described above, including mutation of the new record or other side
effects, before returning the record.
\end{entry}

Any definition that takes advantage of implicit naming for the
constructor, predicate, accessor, and mutator names can be rewritten
trivially to a definition that specifies all names explicitly. For
example, the implicit-naming record definition:

\begin{scheme}
(define-record-type frob
  (fields (mutable widget))
  (protocol
    (lambda (p)
      (lambda (n) (p (make-widget n))))))%
\end{scheme}

is equivalent to the following explicit-naming record definition.

\begin{scheme}
(define-record-type (frob make-frob frob?)
  (fields (mutable widget
                   frob-widget
                   frob-widget-set!))
  (protocol
    (lambda (p)
      (lambda (n) (p (make-widget n))))))%
\end{scheme}

Also, the implicit-naming record definition:
 
\begin{scheme}
(define-record-type point (fields x y))%
\end{scheme}

is equivalent to the following explicit-naming record
definition:

\begin{scheme}
(define-record-type (point make-point point?)
  (fields 
    (immutable x point-x)
    (immutable y point-y)))%
\end{scheme}

With implicit naming, it is still possible to specify some of
the names explicitly; for example, the following overrides the choice
of accessor and mutator names for the widget field.

\begin{scheme}
(define-record-type frob
  (fields (mutable widget getwid setwid!))
  (protocol
    (lambda (p)
      (lambda (n) (p (make-widget n))))))%
\end{scheme}

\begin{entry}{%
\proto{record-type-descriptor}{ \hyper{record name}}{\exprtype}}
   
Evaluates to the record-type descriptor (see below) associated with the type
specified by \hyper{record name}.

\begin{note}   
The {\cf record-type-descriptor} procedure works on both opaque and non-opaque record
types.
\end{note}
\end{entry}

\begin{entry}{%
\proto{record-constructor-descriptor}{ \hyper{record name}}{\exprtype}}
   
Evaluates to the record-constructor descriptor (see below) associated with
\hyper{record name}.
\end{entry}

The following example uses the {\cf record?} procedure from the 
\rsixlibrary{records inspection} library (section
\ref{recordinspectionsection}):

\begin{scheme}
(define-record-type (point make-point point?)
  (fields (immutable x point-x)
          (mutable y point-y set-point-y!))
  (nongenerative
    point-4893d957-e00b-11d9-817f-00111175eb9e))

(define-record-type (cpoint make-cpoint cpoint?)
  (parent point)
  (protocol
   (lambda (n)
     (lambda (x y c) 
       ((n x y) (color->rgb c)))))
  (fields
    (mutable rgb cpoint-rgb cpoint-rgb-set!)))

(define (color->rgb c)
  (cons 'rgb c))

(define p1 (make-point 1 2))
(define p2 (make-cpoint 3 4 'red))

(point? p1) \ev \schtrue{}
(point? p2) \ev \schtrue{}
(point? (vector)) \ev \schfalse{}
(point? (cons 'a 'b)) \ev \schfalse{}
(cpoint? p1) \ev \schfalse{}
(cpoint? p2) \ev \schtrue{}
(point-x p1) \ev 1
(point-y p1) \ev 2
(point-x p2) \ev 3
(point-y p2) \ev 4
(cpoint-rgb p2) \ev (rgb . red)

(set-point-y! p1 17) \ev \unspecified
(point-y p1) \ev 17)

(record-rtd p1) \lev (record-type-descriptor point)

(define-record-type (ex1 make-ex1 ex1?)
  (protocol (lambda (p) (lambda a (p a))))
  (fields (immutable f ex1-f)))

(define ex1-i1 (make-ex1 1 2 3))
(ex1-f ex1-i1) \ev (1 2 3)

(define-record-type (ex2 make-ex2 ex2?)
  (protocol
    (lambda (p) (lambda (a . b) (p a b))))
  (fields (immutable a ex2-a)
          (immutable b ex2-b)))

(define ex2-i1 (make-ex2 1 2 3))
(ex2-a ex2-i1) \ev 1
(ex2-b ex2-i1) \ev (2 3)

(define-record-type (unit-vector
                     make-unit-vector
                     unit-vector?)
  (protocol
   (lambda (p)
     (lambda (x y z)
       (let ((length 
               (sqrt (+ (* x x)
                        (* y y)
                        (* z z)))))
         (p (/ x length)
            (/ y length)
            (/ z length))))))
  (fields (immutable x unit-vector-x)
          (immutable y unit-vector-y)
          (immutable z unit-vector-z)))

(define *ex3-instance* \schfalse{})

(define-record-type ex3
  (parent cpoint)
  (protocol
   (lambda (n)
     (lambda (x y t)
       (let ((r ((n x y 'red) t)))
         (set! *ex3-instance* r)
         r))))
  (fields 
   (mutable thickness))
  (sealed \schtrue{}) (opaque \schtrue{}))

(define ex3-i1 (make-ex3 1 2 17))
(ex3? ex3-i1) \ev \schtrue{}
(cpoint-rgb ex3-i1) \ev (rgb . red)
(ex3-thickness ex3-i1) \ev 17
(ex3-thickness-set! ex3-i1 18) \lev \unspecified
(ex3-thickness ex3-i1) \ev 18
{}*ex3-instance* \ev ex3-i1

(record? ex3-i1) \ev \schfalse{}%
\end{scheme}


\section{Procedural layer}
\label{recordsproceduralsection}

The procedural layer is provided by the \defrsixlibrary{records
  procedural} library.

\begin{entry}{%
\pproto{(make-record-type-descriptor \var{name}}{procedure}}
\mainschindex{make-record-type-descriptor}{\tt\obeyspaces\\
        \var{parent} \var{uid} \var{sealed?} \var{opaque?} \var{fields})}
   
Returns a \defining{record-type descriptor}, or \defining{rtd},
representing a record type distinct from all built-in types and
other record types.

The \var{name} argument must be a symbol. It names the record type,
and is intended purely for informational purposes and may be used for printing by
the underlying Scheme system.

The \var{parent} argument must be either \schfalse{} or an rtd. If it is an
rtd, the returned record type, \var{t}, extends the record type
\var{p} represented by \var{parent}.  An exception with
condition type {\cf\&assertion} is raised if \var{parent} is sealed (see below).
   
The \var{uid} argument must be either \schfalse{} or a symbol.
If \var{uid} is a symbol, the record-creation operation is
\emph{nongenerative} i.e., a new record type is created only
if no previous call to {\cf make-record-type-descriptor}
was made with the \var{uid}.
If \var{uid} is \schfalse{}, the record-creation operation is
\emph{generative}, i.e., a new record type is created even if
a previous call to {\cf make-record-type-descriptor} was
made with the same arguments.

If {\cf make-record-type-descriptor} is
called twice with the same \var{uid} symbol, the parent
arguments in the two calls must be {\cf eqv?}, the \var{fields}
arguments {\cf equal?}, the \var{sealed?} arguments boolean-equivalent
(both \schfalse{} or both true), and the \var{opaque?} arguments
boolean-equivalent if the parents are not opaque.
If these conditions are not met, an exception with condition type
{\cf\&assertion} is raised when the second call occurs.
If they are met, the second call returns, without creating a new
record type, the same record-type descriptor
(in the sense of {\cf eqv?}) as the first call.

\begin{note}   
  Users are encouraged to use symbol names
  constructed using the UUID namespace~\cite{RFC4122} (for example, using the
  record-type name as a prefix) for the uid argument.
\end{note}

The \var{sealed?} flag must be a boolean. If true, the returned record type
is sealed, i.e., it cannot be extended.

The \var{opaque?} flag must be a boolean. If true, the record type
is opaque.
If passed an instance of the record type,
{\cf record?} returns
\schfalse{}.  Moreover, if {\cf record-rtd} (see ``Inspection'' below)
is called with an instance of the record type, 
an exception with condition type {\cf\&assertion} is raised.
The record type is also opaque if an opaque parent is
supplied.  If \var{opaque?} is \schfalse{} and an opaque parent is not
supplied, the record is not opaque.

The \var{fields} argument must be a vector of field specifiers. Each
field specifier must be a list of the form {\cf (mutable \var{name})}
or a list of the form {\cf (immutable \var{name})}.
Each name must be a symbol and names the corresponding field of the record
type; the names need not be distinct.  A field identified as mutable may
be modified, whereas, when a program attempts to obtain a mutator for a field identified
as immutable, an exception with condition type {\cf\&assertion} is raised.
Where field order is relevant, e.g., for record construction and field
access, the fields are considered to be ordered as specified, although
no particular order is required for the actual representation of a
record instance.

The specified fields are added to the parent fields, if any, to determine
the complete set of fields of the returned record type.
If \var{fields} is modified after {\cf make-record-type-descriptor}
has been called, the effect on the returned
rtd is unspecified.

A generative record-type descriptor created by a call to {\cf
  make-record-type-descriptor} is not {\cf eqv?} to any record-type
descriptor (generative or nongenerative) created by another call to
{\cf make-record-type-descriptor}. A generative record-type descriptor
is {\cf eqv?}  only to itself, i.e., {\cf (eqv? \vari{rtd} \varii{rtd})} iff
{\cf (eq? \vari{rtd} \varii{rtd})}.
Also, two nongenerative record-type descriptors are {\cf eqv?} iff they were
created by calls to {\cf make-record-type-descriptor} with the same
uid arguments.
\end{entry}

\begin{entry}{%
\proto{record-type-descriptor?}{ obj}{procedure}}
   
Returns \schtrue{} if the argument is a record-type descriptor,
\schfalse{} otherwise.
\end{entry}

\begin{entry}{%
\pproto{(make-record-constructor-descriptor \var{rtd}}{procedure}}
\mainschindex{make-record-constructor-descriptor}{\tt\obeyspaces\\
        \var{parent-constructor-descriptor} \var{protocol})}

Returns a \defining{record-constructor descriptor} (or
\defining{constructor descriptor} for short) that specifies a \defining{record
constructor} (or \textit{constructor} for short),
that can be used to construct record values of the type
specified by \var{rtd}, and which can be obtained
via {\cf record-constructor}.   A constructor descriptor can
also be used to create other constructor descriptors for subtypes of
its own record type.  \var{Rtd} must be a record-type
descriptor.  \var{Protocol}\mainindex{protocol} must be a procedure or \schfalse.
If it is \schfalse, a default \var{protocol} procedure is supplied.

If \var{protocol} is a procedure, it is handled analogously to the
protocol expression in a {\cf define-record-type} form.

If \var{rtd} is a base record type
\var{parent-constructor-descriptor} must be \schfalse.  In this case,
\var{protocol} is called by {\cf record-constructor} with a single
argument \var{p}.  \var{P} is a procedure that expects one argument
for every field of \var{rtd} and returns a record with the fields of
\var{rtd} initialized to these arguments.  The procedure returned by
\var{protocol} should call \var{p} once with the number of arguments
\var{p} expects and return the resulting record as shown in the simple
example below:
%
\begin{scheme}
(lambda (p)
  (lambda (v1 v2 v3)
    (p v1 v2 v3)))%
\end{scheme}
%
Here, the call to {\cf p} returns a record whose fields are
initialized with the values of {\cf v1}, {\cf v2}, and {\cf v3}.  The
expression above is equivalent to {\cf (lambda (p) p)}.  Note that the
procedure returned by \var{protocol} is otherwise unconstrained;
specifically, it can take any number of arguments.

\medskip

If \var{rtd} is an extension of another record type \var{parent-rtd}
and \var{protocol} is a procedure, \var{parent-constructor-descriptor}
must be a constructor descriptor of \var{parent-rtd} or \schfalse.  If
\var{parent-constructor-descriptor} is a constructor descriptor,
\var{protocol} is called by {\cf record-constructor} with a single
argument \var{n}, which is a procedure that accepts the same number of
arguments as the constructor of \var{parent-constructor-descriptor}
and returns a procedure \var{p} that, when called, constructs the
record itself. The \var{p} procedure expects one argument for every
field of \var{rtd} (not including parent fields) and returns a record
with the fields of \var{rtd} initialized to these arguments, and the
fields of \var{parent-rtd} and its parents initialized as specified by
\var{parent-constructor-descriptor}.

The procedure returned by \var{protocol} should call \var{n} once with
the number of arguments \var{n} expects, call the procedure \var{p} it
returns once with the number of arguments \var{p} expects and return the
resulting record.  A simple \var{protocol} in this case might be
written as follows:
%
\begin{scheme}
(lambda (n)
  (lambda (v1 v2 v3 x1 x2 x3 x4)
    (let ((p (n v1 v2 v3)))
      (p x1 x2 x3 x4))))%
\end{scheme}
%
This passes arguments {\cf v1}, {\cf v2}, {\cf v3} to \var{n} for 
\var{parent-constructor-descriptor} and calls {\cf p}
with {\cf x1}, \ldots, {\cf x4} to initialize the fields of \var{rtd} itself.

Thus, the constructor descriptors for a record type form a sequence of
protocols parallel to the sequence of record-type parents. Each
constructor descriptor in the chain determines the field values for the
associated record type.
Child record constructors need not know the number or contents of parent
fields, only the number of arguments accepted by the parent constructor.

\var{Protocol} may be \schfalse, specifying a default constructor that
accepts one argument for each field of \var{rtd} (including the
fields of its parent type, if any).  Specifically, if \var{rtd} is a
base type, the default \var{protocol} procedure behaves as if it were
{\cf (lambda (p) p)}.  If \var{rtd} is an extension of another type,
then \var{parent-constructor-descriptor} must be either \schfalse{} or
itself specify a default constructor, and the default
\var{protocol} procedure behaves as if it were:
%
\begin{scheme}
(lambda (n)
  (lambda (\vari{v} \ldots \varj{v} \vari{x} \ldots \vark{x})
    (let ((p (n \vari{v} \ldots \varj{v})))
      (p \vari{x} \ldots \vark{x}))))%
\end{scheme}
%
The resulting constructor accepts one argument for each of the record
type's complete set of fields (including those of the parent record
type, the parent's parent record type, etc.) and returns a record with
the fields initialized to those arguments, with the field values for
the parent coming before those of the extension in the argument list.
(In the example, $j$ is the complete number of fields of the parent
type, and $k$ is the number of fields of \var{rtd} itself.)

If \var{rtd} is an extension of another record type and
\var{parent-constructor-descriptor} is \schfalse{},
\var{parent-constructor-descriptor} is treated as if it were a
constructor descriptor for the parent rtd of \var{rtd} with a default
protocol.

\implresp If \var{protocol} is a procedure, the implementation must
check the restrictions on it to the extent performed by applying it as
described when the constructor is called.
An
implementation may check whether \var{protocol} is an appropriate argument
before applying it.

\begin{scheme}
(define rtd1
  (make-record-type-descriptor
   'rtd1 \schfalse{} \schfalse{} \schfalse{} \schfalse{}
   '\sharpsign((immutable x1) (immutable x2))))

(define rtd2
  (make-record-type-descriptor
   'rtd2 rtd1 \schfalse{} \schfalse{} \schfalse{}
   '\sharpsign((immutable x3) (immutable x4))))

(define rtd3
  (make-record-type-descriptor
   'rtd3 rtd2 \schfalse{} \schfalse{} \schfalse{}
   '\sharpsign((immutable x5) (immutable x6))))

(define protocol1
  (lambda (p)
    (lambda (a b c)
      (p (+ a b) (+ b c)))))

(define protocol2
  (lambda (n)
    (lambda (a b c d e f)
      (let ((p (n a b c)))
        (p (+ d e) (+ e f))))))

(define protocol3
  (lambda (n)
    (lambda (a b c d e f g h i)
      (let ((p (n a b c d e f)))
        (p (+ g h) (+ h i))))))

(define cd1
  (make-record-constructor-descriptor
    rtd1 \schfalse{} protocol1))

(define cd2
  (make-record-constructor-descriptor
    rtd2 cd1 protocol2))

(define cd3
  (make-record-constructor-descriptor
    rtd3 cd2 protocol3))

(define make-rtd1 (record-constructor cd1))

(define make-rtd2 (record-constructor cd2))

(define make-rtd3 (record-constructor cd3))

(make-rtd3 1 2 3 4 5 6 7 8 9)\lev
  \(\langle\)\textrm{record with fields initialized to 3, 5, 9, 11, 15, 17}\(\rangle\)
\end{scheme}
\end{entry}

\begin{entry}{%
\proto{record-constructor}{ constructor-descriptor}{procedure}}
   
Calls the \var{protocol} of \var{constructor-descriptor} (as described for
{\cf make-record-constructor-descriptor}) and returns the resulting
constructor \var{constructor} for records of the record type
associated with \var{constructor-descriptor}.
\end{entry}

\begin{entry}{%
\proto{record-predicate}{ rtd}{procedure}}
   
Returns a procedure that, given an object \var{obj}, returns
\schtrue{}
if \var{obj} is a record of the type represented by
\var{rtd}, and \schfalse{} otherwise.
\end{entry}

\begin{entry}{%
\proto{record-accessor}{ rtd k}{procedure}}

\domain{\var{K} must be a valid field index of \var{rtd}.}  The {\cf
  record-accessor} procedure returns a one-argument procedure whose
argument must be a record of the type represented by \var{rtd}.  This
procedure returns the value of the selected field of that record.

The field selected corresponds to the \var{k}th element
(0-based) of the \var{fields} argument to the invocation of {\cf
  make-record-type-descriptor} that created \var{rtd}. Note that
\var{k} cannot be used to specify a field of any type \var{rtd} extends.
\end{entry}

\begin{entry}{%
\proto{record-mutator}{ rtd k}{procedure}}
   
\domain{\var{K} must be a valid field index of \var{rtd}.}  The {\cf
  record-mutator} procedure returns a two-argument procedure whose
arguments must be a record record \var{r} of the type represented by
\var{rtd} and an object \var{obj}.  This procedure stores \var{obj}
within the field of \var{r} specified by \var{k}. The \var{k} argument
is as in {\cf record-accessor}. If \var{k} specifies an immutable
field, an exception with condition type {\cf\&assertion} is raised.
The mutator returns \unspecifiedreturn.
\end{entry}

\begin{scheme}
(define :point
  (make-record-type-descriptor
    'point \schfalse{}
    \schfalse{} \schfalse{} \schfalse{} 
    '\sharpsign((mutable x) (mutable y))))

(define :point-cd
  (make-record-constructor-descriptor :point \schfalse{} \schfalse{}))

(define make-point (record-constructor :point-cd))

(define point? (record-predicate :point))
(define point-x (record-accessor :point 0))
(define point-y (record-accessor :point 1))
(define point-x-set! (record-mutator :point 0))
(define point-y-set! (record-mutator :point 1))

(define p1 (make-point 1 2))
(point? p1) \ev \schtrue{}
(point-x p1) \ev 1
(point-y p1) \ev 2
(point-x-set! p1 5) \ev \theunspecified
(point-x p1) \ev 5

(define :point2
  (make-record-type-descriptor
    'point2 :point 
    \schfalse{} \schfalse{} \schfalse{} '\sharpsign((mutable x) (mutable y))))

(define make-point2
  (record-constructor
    (make-record-constructor-descriptor :point2
      \schfalse{} \schfalse{})))
(define point2? (record-predicate :point2))
(define point2-xx (record-accessor :point2 0))
(define point2-yy (record-accessor :point2 1))

(define p2 (make-point2 1 2 3 4))
(point? p2) \ev \schtrue{}
(point-x p2) \ev 1
(point-y p2) \ev 2
(point2-xx p2) \ev 3
(point2-yy p2) \ev 4

(define :point-cd/abs
  (make-record-constructor-descriptor
   :point \schfalse{}
   (lambda (new)
     (lambda (x y)
       (new (abs x) (abs y))))))

(define make-point/abs
  (record-constructor :point-cd/abs))

(point-x (make-point/abs -1 -2)) \lev 1
(point-y (make-point/abs -1 -2)) \lev 2

(define :cpoint
  (make-record-type-descriptor
   'cpoint :point
   \schfalse{} \schfalse{} \schfalse{}
   '\sharpsign((mutable rgb))))

(define make-cpoint
  (record-constructor
   (make-record-constructor-descriptor
    :cpoint :point-cd
    (lambda (p)
      (lambda (x y c)
	((p x y) (color->rgb c)))))))

(define make-cpoint/abs
  (record-constructor
   (make-record-constructor-descriptor
    :cpoint :point-cd/abs
    (lambda (p)
      (lambda (x y c)
	((p x y) (color->rgb c)))))))

(define cpoint-rgb
  (record-accessor :cpoint 0))

(define (color->rgb c)
  (cons 'rgb c))

(cpoint-rgb (make-cpoint -1 -3 'red)) \lev (rgb . red)
(point-x (make-cpoint -1 -3 'red)) \lev -1
(point-x (make-cpoint/abs -1 -3 'red)) \lev 1%
\end{scheme}

\section{Inspection}
\label{recordinspectionsection}

The \defrsixlibrary{records inspection} library
provides procedures for inspecting records and their
record-type descriptors. These procedures are designed to allow the
writing of portable printers and inspectors.

On the one hand, {\cf record?} and {\cf record-rtd} treat records of opaque
record types as if they were not records. On the other hand, the
inspection procedures that operate on record-type descriptors
themselves are not affected by opacity. In other words, opacity
controls whether a program can obtain an rtd from a record. If the
program has access to the original rtd via {\cf
  make-record-type-descriptor} or {\cf record-type-descriptor}, it can
still make use of the inspection procedures.

\begin{entry}{%
\proto{record?}{ obj}{procedure}}
   
Returns \schtrue{} if \var{obj} is a record, and its record type is
not opaque, and returns \schfalse{} otherwise.  
\end{entry}

\begin{entry}{%
\proto{record-rtd}{ record}{procedure}}
   
Returns the rtd representing the type of \var{record} if the type is not
opaque. The rtd of the most precise type is returned; that is, the
type \var{t} such that \var{record} is of type \var{t} but not of any
type that extends \var{t}.  If the type is opaque, an exception is
raised with condition type {\cf\&assertion}.
\end{entry}

\begin{entry}{%
\proto{record-type-name}{ rtd}{procedure}}
   
Returns the name of the record-type descriptor \var{rtd}.
\end{entry}   

\begin{entry}{%
\proto{record-type-parent}{ rtd}{procedure}}
   
Returns the parent of the record-type descriptor \var{rtd}, or
\schfalse{} if it has none.
\end{entry}

\begin{entry}{%
\proto{record-type-uid}{ rtd}{procedure}}
   
Returns the uid of the record-type descriptor rtd, or \schfalse{} if it has none.
(An implementation may assign a generated uid to a record type even if the
type is generative, so the return of a uid does not necessarily imply that
the type is nongenerative.)
\end{entry}

\begin{entry}{%
\proto{record-type-generative?}{ rtd}{procedure}}
   
Returns \schtrue{} if \var{rtd} is generative, and \schfalse{} if not.
\end{entry}

\begin{entry}{%
\proto{record-type-sealed?}{ rtd}{procedure}}

Returns \schtrue{} if the record-type descriptor is
sealed, and \schfalse{} if not.
\end{entry}

\begin{entry}{%
\proto{record-type-opaque?}{ rtd}{procedure}}
   
Returns \schtrue{} if the the record-type descriptor is
opaque, and \schfalse{} if not.
\end{entry}

\begin{entry}{%
\proto{record-type-field-names}{ rtd}{procedure}}
   
Returns a vector of symbols naming the fields of the type represented by \var{rtd}
(not including the fields of parent types) where the fields are ordered as
described under {\cf make-record-type-descriptor}.  The returned
vector may be immutable.
If the returned vector is modified, the effect on 
\var{rtd} is unspecified.
\end{entry}

\begin{entry}{%
\proto{record-field-mutable?}{ rtd k}{procedure}}
   
Returns \schtrue{} if the field specified by
\var{k} of the type represented by \var{rtd} is mutable, and
\schfalse{} if not.  \var{K} is as in {\cf record-accessor}.
\end{entry}

%%% Local Variables: 
%%% mode: latex
%%% TeX-master: "r6rs-lib"
%%% End: 

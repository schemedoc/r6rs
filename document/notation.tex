\chapter{Notation and terminology}

\section{Error situations and unspecified behavior}

\mainindex{error}
When speaking of an error situation, this report uses the phrase ``an
error is signalled'' to indicate that implementations must detect and
report the error.  If such wording does not appear in the discussion of
an error, then implementations are not required to detect or report the
error, though they are encouraged to do so.  An error situation that
implementations are not required to detect is usually referred to simply
as ``an error.''

\vest For example, it is an error for a procedure to be passed an argument that
the procedure is not explicitly specified to handle, even though such
domain errors are seldom mentioned in this report.  Implementations may
extend a procedure's domain of definition to include such arguments.

\vest This report uses the phrase ``may report a violation of an
implementation restriction'' to indicate circumstances under which an
implementation is permitted to report that it is unable to continue
execution of a correct program because of some restriction imposed by the
implementation.  Implementation restrictions are of course discouraged,
but implementations are encouraged to report violations of implementation
restrictions.\mainindex{implementation restriction}

\vest For example, an implementation may report a violation of an
implementation restriction if it does not have enough storage to run a
program.

\vest If the value of an expression is said to be ``unspecified,'' then
the expression must evaluate to some object without signalling an error,
but the value depends on the implementation; this report explicitly does
not say what value should be returned. \mainindex{unspecified}

\todo{Talk about unspecified behavior vs. unspecified values.}

\todo{Look at KMP's situations paper.}


\section{Entry format}

The chapters describing bindings in the core language and the standard
libraries are organized
into entries.  Each entry describes one language feature or a group of
related features, where a feature is either a syntactic construct or a
built-in procedure.  An entry begins with one or more header lines of the form

\noindent\pproto{\var{template}}{\var{category}}\unpenalty

If \var{category} is ``\exprtype'', the entry describes an expression
type, and the template gives the syntax of the expression type.
Components of expressions are designated by syntactic variables, which
are written using angle brackets, for example, \hyper{expression},
\hyper{variable}.  Syntactic variables should be understood to denote segments of
program text; for example, \hyper{expression} stands for any string of
characters which is a syntactically valid expression.  The notation
\begin{tabbing}
\qquad \hyperi{thing} $\ldots$
\end{tabbing}
indicates zero or more occurrences of a \hyper{thing}, and
\begin{tabbing}
\qquad \hyperi{thing} \hyperii{thing} $\ldots$
\end{tabbing}
indicates one or more occurrences of a \hyper{thing}.

If \var{category} is ``procedure'', then the entry describes a procedure, and
the header line gives a template for a call to the procedure.  Argument
names in the template are \var{italicized}.  Thus the header line

\noindent\pproto{(vector-ref \var{vector} \var{k})}{procedure}\unpenalty

indicates that the built-in procedure {\tt vector-ref} takes
two arguments, a vector \var{vector} and an exact non-negative integer
\var{k} (see below).  The header lines

\noindent%
\pproto{(make-vector \var{k})}{procedure}
\pproto{(make-vector \var{k} \var{fill})}{procedure}\unpenalty

indicate that the {\tt make-vector} procedure must be defined to take
either one or two arguments.

\label{typeconventions}
It is an error for an operation to be presented with an argument that it
is not specified to handle.  For succinctness, we follow the convention
that if an argument name is also the name of a type listed in
section~\ref{disjointness}, then that argument must be of the named type.
For example, the header line for {\tt vector-ref} given above dictates that the
first argument to {\tt vector-ref} must be a vector.  The following naming
conventions also imply type restrictions:
\newcommand{\foo}[1]{\vr{#1}, \vri{#1}, $\ldots$ \vrj{#1}, $\ldots$}
$$
\begin{tabular}{ll}
\var{obj}&any object\\
\foo{list}&list (see section~\ref{listsection})\\
\foo{z}&complex number\\
\foo{x}&real number\\
\foo{y}&real number\\
\foo{q}&rational number\\
\foo{n}&integer\\
\foo{k}&exact non-negative integer\\
\end{tabular}
$$

\todo{Provide an example entry??}


\section{Evaluation examples}

The symbol ``\evalsto'' used in program examples should be read
``evaluates to.''  For example,

\begin{scheme}
(* 5 8)      \ev  40%
\end{scheme}

means that the expression {\tt(* 5 8)} evaluates to the object {\tt 40}.
Or, more precisely:  the expression given by the sequence of characters
``{\tt(* 5 8)}'' evaluates, in the initial environment, to an object
that may be represented externally by the sequence of characters ``{\tt
40}''.  See section~\ref{externalreps} for a discussion of external
representations of objects.

\section{Naming conventions}

By convention, the names of procedures that always return a boolean
value usually end
in ``\ide{?}''.  Such procedures are called predicates.

By convention, the names of procedures that store values into previously
allocated locations (see section~\ref{storagemodel}) usually end in
``\ide{!}''.
Such procedures are called mutation procedures.
By convention, the value returned by a mutation procedure is
\unspecifiedreturn (see section~\ref{unspecifiedvalue}).

By convention, ``\ide{->}'' appears within the names of procedures that
take an object of one type and return an analogous object of another type.
For example, {\cf list->vector} takes a list and returns a vector whose
elements are the same as those of the list.


	
\todo{Terms that need defining: thunk, command (what else?).}

%%% Local Variables: 
%%% mode: latex
%%% TeX-master: "r6rs"
%%% End: 

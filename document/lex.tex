% Lexical structure

%%\vfill\eject
\chapter{Lexical conventions and read syntax}

This section gives an informal account of some of the lexical
conventions used in writing Scheme programs.  For a formal syntax of
Scheme, see section~\ref{BNF}.

\vest Upper and lower case forms of a letter are distinguished, except
in literals specifying hexadecimal numbers.  For example, {\cf Foo} is
an identifier different from {\cf FOO}.  However, {\tt\#x1AB} is the
same number as {\tt\#X1ab}.

\section{Identifiers}
\label{syntaxsection}

Most identifiers\mainindex{identifier} allowed by other programming
languages are also acceptable to Scheme.  In particular,
a sequence of letters, digits, and ``extended alphabetic
characters'' that begins with a character that cannot begin a number is
an identifier.  In addition, \ide{+}, \ide{-}, and \ide{...} are identifiers. 
Here are some examples of identifiers:

\begin{scheme}
lambda                   q
list->vector             soup
{+}                        V17a
<=?                      a34kTMNs
the-word-recursion-has-many-meanings%
\end{scheme}

Extended alphabetic characters may be used within identifiers as if
they were letters.  The following are extended alphabetic characters:

\begin{scheme}
!\ \$ \% \verb"&" * + - . / :\ < = > ? @ \verb"^" \verb"_" \verb"~" %
\end{scheme}

FIXME: complete, once the formal syntax is done

See section~\ref{extendedalphas} for a formal syntax of identifiers.

\vest Identifiers have two uses within Scheme programs:
\begin{itemize}
\item Any identifier may be used as a variable\index{variable}
 or as a syntactic keyword\index{syntactic keyword}
(see sections~\ref{variablesection} and~\ref{macrosection}).

\item When an identifier appears as a literal or within a literal
(see section~\ref{quote}), it is being used to denote a {\em symbol}
(see section~\ref{symbolsection}).

\end{itemize}

%\label{keywordsection}
%The following identifiers are syntactic keywords, and should not be used
%as variables:
%
%\begin{scheme}
%=>           do            or
%and          else          quasiquote
%begin        if            quote
%case         lambda        set!
%cond         let           unquote
%define       let*          unquote-splicing
%delay        letrec%
%\end{scheme}
%
%Some implementations allow all identifiers, including syntactic
%keywords, to be used as variables.  This is a compatible extension to
%the language, but ambiguities in the language result when the
%restriction is relaxed, and the ways in which these ambiguities are
%resolved vary between implementations.

\vest The rules for writing a symbol are exactly the same as the rules for
writing an identifier; see sections~\ref{syntaxsection}
and~\ref{identifiersyntax}.

\section{Whitespace and comments}

\defining{Whitespace} characters are spaces and newlines.
(Implementations typically provide additional whitespace characters such
as tab or page break.)  Whitespace is used for improved readability and
as necessary to separate tokens from each other, a token being an
indivisible lexical unit such as an identifier or number, but is
otherwise insignificant.  Whitespace may occur between any two tokens,
but not within a token.  Whitespace may also occur inside a string,
where it is significant.

A semicolon ({\tt;}) indicates the start of a
comment.\mainindex{comment}\mainschindex{;}  The comment continues to the
end of the line on which the semicolon appears.  Comments are invisible
to Scheme, but the end of the line is visible as whitespace.  This
prevents a comment from appearing in the middle of an identifier or
number.

Another way to indicate a comment is to prefix a \hyper{datum} (cf.\
Section~\ref{datumsyntax}) with {\tt \#;}, possibly with whitespace in
between.  The comment consists of the comment prefix {\tt \#;} and the
\hyper{datum}.  (This notation is useful for ``commenting out''
sections of code.)

Moreover, block comments may be indicated with properly nested {\tt
  \#|} and {\tt |\#} pairs.

\begin{scheme}
\#|
   The FACT procedure computes the factorial
   of a non-negative integer.
|\#
(define fact
  (lambda (n)
    ;; base case
    (if (= n 0)
        \#;(= n 1)
        1       ; identity of *
        (* n (fact (- n 1))))))%
\end{scheme}

\section{Other notations}

\todo{Rewrite?}

For a description of the notations used for numbers, see
section~\ref{numbernotations}.

\begin{description}{}{}

\item[{\tt.\ + -}]
These are used in numbers, and may also occur anywhere in an identifier
except as the first character.  A delimited plus or minus sign by itself
is also an identifier.
A delimited period (not occurring within a number or identifier) is used
in the notation for pairs (section~\ref{listsection}), and to indicate a
rest-parameter in a  formal parameter list (section~\ref{lambda}).
A delimited sequence of three successive periods is also an identifier.

\item[{\tt( ) \openbracket{} \closedbracket{}}]
Parentheses and brackets are used for grouping and to notate lists
(section~\ref{listsection}).  Brackets are equivalent to parentheses,
with the restriction that an open parenthesis cannot be closed by
a closed bracket and vice versa.

\item[\singlequote]
The single quote character is used to indicate literal data (section~\ref{quote}).

\item[\backquote]
The backquote character is used to indicate almost-constant
data (section~\ref{quasiquote}).

\item[\tt, ,@]
The character comma and the sequence comma at-sign are used in conjunction
with backquote (section~\ref{quasiquote}).

\item[\tt"]
The double quote character is used to delimit strings (section~\ref{stringsection}).

\item[\backwhack]
Backslash is used in the syntax for character constants
(section~\ref{charactersection}) and as an escape character within string
constants (section~\ref{stringsection}).

% A box used because \verb is not allowed in command arguments.
\setbox0\hbox{\tt \verb"{" \verb"}" \verb"|"}
\item[\copy0]
Curly braces and vertical bar
are reserved for possible future extensions to the language.

\item[\sharpsign] Sharp sign is used for a variety of purposes depending on
the character that immediately follows it:

\item[\schtrue{} \schfalse{}]
These are the boolean constants (section~\ref{booleansection}).

\item[\sharpsign\backwhack]
This introduces a character constant (section~\ref{charactersection}).

\item[\sharpsign\tt(]
This introduces a vector constant (section~\ref{vectorsection}).  Vector constants
are terminated by~{\tt)}~.

\item[{\tt\#e \#i \#b \#o \#d \#x}]
These are used in the notation for numbers (section~\ref{numbernotations}).

\end{description}

\section{Read syntax}

\subsection{Abbreviations}\unsection
\label{quotesection}

\begin{entry}{%
\pproto{\singlequote\hyper{datum}}{}
\pproto{\backquote\hyper{datum}}{}
\pproto{,\hyper{datum}}{}
\pproto{,@\hyper{datum}}{}
}

\mainschindex{'}\singlequote\hyper{datum} is an abbreviation
for {\cf (quote \hyper{datum})}.
\mainschindex{`}\backquote\hyper{datum} is an abbreviation
for {\cf (quasiquote \hyper{datum})}.
\mainschindex{,}{\tt ,}\hyper{datum} is an abbreviation
for {\cf (unquote \hyper{datum})}.
\mainschindex{,@}{\tt ,@}\hyper{datum} is an abbreviation
for {\cf (unquote-splicing \hyper{datum})}.
\end{entry}

\subsection{Constant literals}

\begin{entry}{%
\pproto{\hyper{constant}}{\exprtype}}

Numerical constants, string constants, character constants, and
boolean constants evaluate ``to themselves''; they need not be quoted.
\end{entry}

\subsection{Boolean literals}

The standard boolean objects for true and false are written as
\schtrue{} and \schfalse.\sharpindex{t}\sharpindex{f}  The character
after a boolean literal must be a delimiter character such as a
space or parenthesis.

\subsubsection{Character literals}

Character literals are written using the notation
\sharpsign\backwhack\hyper{character} or
\sharpsign\backwhack\hyper{character name} or
\sharpsign\backwhack{}x\atleastone{\hyper{digit 16}}, where the latter
specifies the scalar value of a character with a hexadecimal number of
no more than eight digits.

For example:

\begin{scheme}
\#\backwhack{}a          ; lower case letter
\#\backwhack{}A          ; upper case letter
\#\backwhack{}(          ; left parenthesis
\#\backwhack{}           ; the space character
\#\backwhack{}nul        ; Unicode 0
\#\backwhack{}alarm      ; Unicode 7
\#\backwhack{}backspace  ; Unicode 8
\#\backwhack{}tab        ; Unicode 9
\#\backwhack{}linefeed   ; Unicode 10
\#\backwhack{}vtab       ; Unicode 11
\#\backwhack{}page       ; Unicode 12
\#\backwhack{}return     ; Unicode 13
\#\backwhack{}esc        ; Unicode 27
\#\backwhack{}space      ; Unicode 32;
             ; the preferred way to write a space
\#\backwhack{}delete     ; Unicode 127

\#\backwhack{}xFF        ; Unicode 255
\#\backwhack{}x03BB      ; Unicode 955
\#\backwhack{}x00006587  ; Unicode 25991
\#\backwhack{}\(\lambda\)          ; Unicode 955

\#\backwhack{}x0001z     ; parse error
\#\backwhack{}\(\lambda\)x         ; parse error
\#\backwhack{}alarmx     ; parse error
\#\backwhack{}alarm x    ; Unicode 7 followed by x
\#\backwhack{}Alarm      ; parse error
\#\backwhack{}alert      ; parse error
\#\backwhack{}xA         ; Unicode 10
\#\backwhack{}xFF        ; Unicode 255
\#\backwhack{}xff        ; Unicode 255
\#\backwhack{}x ff       ; Unicode 120
             ; followed by another datum, ff
\#\backwhack{}x(ff)      ; Unicode 120 
             ; followed by another datum,
             ; a parenthesized ff
\#\backwhack{}(x)        ; parse error
\#\backwhack{}(x         ; parse error
\#\backwhack{}((x)       ; Unicode 40
             ; followed by another datum,
             ; parenthesized x
\#\backwhack{}x00110000  ; parse error (out of range)
\#\backwhack{}x000000001 ; parse error (too many digits)
\#\backwhack{}xD800      ; parse error (in excluded range)
\end{scheme}

Case is significant in \sharpsign\backwhack\hyper{character}, and in in
\sharpsign\backwhack{\rm$\langle$character name$\rangle$}, % \hyper doesn't allow a linebreak
but not in \sharpsign\backwhack{}x\atleastone{\hyper{digit 16}}.  
The character after a character literal
must be a delimiter character such as a
space or parenthesis.  This rule resolves various ambiguous cases, for
example, the sequence of characters ``{\tt\sharpsign\backwhack space}''
could be taken to be either a representation of the space character or a
representation of the character ``{\tt\sharpsign\backwhack s}'' followed
by a representation of the symbol ``{\tt pace}.''

\subsubsection{String literals}

\vest String literals are written as sequences of characters enclosed within doublequotes
({\cf "}).  Within a string literal, various escape
sequences\mainindex{escape sequence} denote characters other than
themselves.  Escape sequences always start with a backslash (\backwhack{}):

\begin{itemize}
\item{\tt \backwhack{}a} : alarm, Unicode 7
\item{\tt \backwhack{}b} : backspace, Unicode 8 
\item{\tt \backwhack{}t} : tab, Unicode 9 
\item{\tt \backwhack{}n} : linefeed, Unicode 10 
\item{\tt \backwhack{}v} : vertical tab, Unicode 11 
\item{\tt \backwhack{}f} : formfeed, Unicode 12 
\item{\tt \backwhack{}r} : return, Unicode 13 
\item{\tt \backwhack{}}\verb|"| : doublequote, Unicode 34 
\item{\tt \backwhack{}\backwhack{}} : backslash, Unicode 92 
\item{\tt \backwhack{}\hyper{linefeed}\hyper{intraline whitespace}} : nothing
\item{\tt \backwhack{}\ } : space, Unicode 32(useful for terminating the
  previous escape sequence before continuing with whitespace)
\item{\tt \backwhack{}x\atleastone{\hyper{digit 16}}} : (note the
  terminating semi-colon) where no more than eight \hyper{digit 16}s
  are provided, and the sequence of \hyper{digit 16}s forms a
  hexadecimal number between 0 and \sharpsign{}xFFFF excluding the
  range $\left[\sharpsign{}x\textrm{D800},
    \sharpsign{}x\textrm{DFFF}\right]$.
\end{itemize}

These escape sequences are case-sensitive, except that \hyper{digit
  16} can be an uppercase or lowercase hexadecimal digit.

Any other character in a string after a backslash is an error. Any
character outside of an escape sequence and not a doublequote stands
for itself in the string literal. For example the single-character
string {\tt "$\lambda$"} (double quote, a lower case lambda, double
quote) denotes the same string literal as {\tt "\backwhack{}x03bb;"}.

Examples:

\begin{center}
  \begin{tabular}{ll}
    {\tt "abc"}    & ; Unicode sequence 97, 98, 99\\
    {\tt "\backwhack{}x41;bc"} & ; ``Abc'', Unicode sequence 65, 98, 99\\
    {\tt "\backwhack{}x41; bc"} & ; ``A bc'', Unicode sequence 65, 32, 98, 99\\
    {\tt "\backwhack{}x41bc;"} & ; Unicode sequence 16828\\
    {\tt "\backwhack{}x41"} & ; parse error\\
    {\tt "\backwhack{}x;"} & ; parse error\\
    {\tt "\backwhack{}x41bx;"} & ; parse error\\
    {\tt "\backwhack{}x00000041;"} & ; ``A'', Unicode sequence 65\\
    {\tt "\backwhack{}x0010FFFF;"} &  ; Unicode sequence \#x10FFFF\\
    {\tt "\backwhack{}x00110000;"}&  ; parse error (out of range)\\
    {\tt "\backwhack{}x000000001;"}& ; parse error (too many digits)\\
    {\tt "\backwhack{}xD800;"}&      ; parse error (in excluded range)
  \end{tabular}
\end{center}

\subsection{Number literals}
\label{numbernotations}

The syntax of the written representations for numbers is described formally in
section~\ref{numbersyntax}.  Note that case is not significant in numerical
constants.

A number may be written in binary, octal, decimal, or
hexadecimal by the use of a radix prefix.  The radix prefixes are {\cf
\#b}\sharpindex{b} (binary), {\cf \#o}\sharpindex{o} (octal), {\cf
\#d}\sharpindex{d} (decimal), and {\cf \#x}\sharpindex{x} (hexadecimal).  With
no radix prefix, a number is assumed to be expressed in decimal.

A
numerical constant may be specified to be either exact or
inexact by a prefix.  The prefixes are {\cf \#e}\sharpindex{e}
for exact, and {\cf \#i}\sharpindex{i} for inexact.  An exactness
prefix may appear before or after any radix prefix that is used.  If
the written representation of a number has no exactness prefix, the
constant may be either inexact or exact.  It is
inexact if it contains a decimal point, an
exponent, or a ``\sharpsign'' character in the place of a digit,
otherwise it is exact.

In systems with inexact numbers
of varying precisions it may be useful to specify
the precision of a constant.  For this purpose, numerical constants
may be written with an exponent marker that indicates the
desired precision of the inexact
representation.  The letters {\cf s}, {\cf f},
{\cf d}, and {\cf l} specify the use of \var{short}, \var{single},
\var{double}, and \var{long} precision, respectively.  (When fewer
than four internal
inexact
representations exist, the four size
specifications are mapped onto those available.  For example, an
implementation with two internal representations may map short and
single together and long and double together.)  In addition, the
exponent marker {\cf e} specifies the default precision for the
implementation.  The default precision has at least as much precision
as \var{double}, but
implementations may wish to allow this default to be set by the user.

\begin{scheme}
3.14159265358979F0
       {\rm Round to single ---} 3.141593
0.6L0
       {\rm Extend to long ---} .600000000000000%
\end{scheme}

If \var{x} is an external representation of an inexact real number
according to R5RS, and \var{p} is a sequence of 1 or more decimal
digits, then {\cf \var{x}|\var{p}} is an external representation that
denotes the best binary floating point approximation to \var{x} using
a \var{p}-bit significand.  For example, {\cf 1.1|53} is an external
representation for the best approximation to 1.1 in IEEE double
precision.

If \var{x} is an external representation of an inexact real number
according to R5RS, then \var{x} by itself should be regarded as
equivalent to {\cf \var{x}|53}

The literals {\cf +inf.0} and {\cf -inf.0} represent positive and
negative infinity, respectively.  The {\cf +nan.0} and {\cf -nan.0}
literals represent the NaN that is the result of {\cf (/ 0.0 0.0)}
and may represent other NaNs as well.

Implementations that use binary floating point representations
of real numbers should represent {\cf \var{x}|\var{p}}
using a \var{p}-bit significand if practical, or by a greater
precision if a \var{p}-bit significand is not practical, or
by the largest available precision if \var{p} or more bits
of significand is not practical within the implementation.

\begin{note}
The precision of a significand should not be confused with the
number of bits used to represent the significand.  In the IEEE
floating point standards, for example, the significand's most
significant bit is implicit in single and double precision but
is explicit in extended precision.  Whether that bit is implicit
or explicit does not affect the mathematical precision.
In implementations that use binary floating point, the default
precision can be calculated by calling the following procedure:

\begin{scheme}
(define (precision)
  (do ((n 0 (+ n 1))
       (x 1.0 (/ x 2.0)))
    ((= 1.0 (+ 1.0 x)) n)))
\end{scheme}
\end{note}      

\begin{note}
When the underlying floating-point representation is IEEE double
precision, the {\cf |\var{p}} suffix should not be
omitted for all cases:
Denormalized numbers have diminished precision, and therefore should
carry a {\cf |\var{p}} suffix with the actual width of the
significand.
\end{note}

If a \meta{decimal 10} does not contain a non-empty \meta{mantissa
  width} and does not contain one of the exponent markers {\cf s},
{\cf f}, {\cf d}, or {\cf l}, but does contain a decimal point or the
exponent marker {\cf e}, then it is an external representation for a
flonum.  Furthermore {\cf inf.0}, {\cf +inf.0}, {\cf -inf.0}, {\cf
  nan.0}, {\cf +nan.0}, and {\cf -nan.0} are external representations
for flonums.  Some or all of the other external representations for
inexact reals may also represent flonums, but that is not required by
this SRFI.

If a \meta{decimal 10} contains a non-empty \meta{mantissa width} or
one of the exponent markers {\cf s}, {\cf f}, {\cf d}, or {\cf l},
then it represents an inexact number, but does not necessarily
represent a flonum.

\subsection{Vectors}

Vectors are written using the notation {\tt\#(\var{obj} \dotsfoo)}.
For example, a vector of length 3 containing the number zero in element
0, the list {\cf(2 2 2 2)} in element 1, and the string {\cf "Anna"} in
element 2 can be written as following:

\begin{scheme}
\#(0 (2 2 2 2) "Anna")%
\end{scheme}

Note that this is the external representation of a vector, not an
expression evaluating to a vector.

\subsection{Pairs and lists}

The most general notation (external representation) for Scheme pairs is
the ``dotted'' notation \hbox{\cf (\vari{c} .\ \varii{c})} where
\vari{c} is the value of the car field and \varii{c} is the value of the
cdr field.  For example {\cf (4 .\ 5)} is a pair whose car is 4 and whose
cdr is 5.  Note that {\cf (4 .\ 5)} is the external representation of a
pair, not an expression that evaluates to a pair.

A more streamlined notation can be used for lists: the elements of the
list are simply enclosed in parentheses and separated by spaces.  The
empty list\index{empty list} is written {\tt()} .  For example,

\begin{scheme}
(a b c d e)%
\end{scheme}

and

\begin{scheme}
(a . (b . (c . (d . (e . ())))))%
\end{scheme}

are equivalent notations for a list of symbols.

%%% Local Variables: 
%%% mode: latex
%%% TeX-master: "r6rs"
%%% End: 

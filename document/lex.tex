% Lexical structure
\hyphenation{white-space}
%%\vfill\eject
\chapter{Lexical syntax and datum syntax}
\label{readsyntaxchapter}

The syntax of Scheme code is organized in three levels:
%
\begin{enumerate}
\item the \textit{lexical syntax} that describes how a program text is split
  into a sequence of lexemes,
\item the \textit{datum syntax}, formulated in terms of the lexical
  syntax, that structures the lexeme sequence as a sequence of
  \textit{syntactic data\mainindex{datum}\mainindex{syntactic
      datum}}, where a syntactic datum is
    a recursively structured entity,
\item the \textit{program syntax} formulated in terms of the read
  syntax, imposing further structure and assigning meaning to
  syntactic data.
\end{enumerate}
%
Syntactic data (also called \textit{external
  representations\index{external representation}}) double
as a notation for data, and Scheme's \rsixlibrary{io ports} library
(library section~\extref{lib:portsiosection}{Port I/O})
provides the {\cf get-datum} and {\cf put-datum} procedures
for reading and writing syntactic data, converting between their
textual representation and the corresponding values. 
Each syntactic datum uniquely determines a corresponding \defining{datum value}.
A syntactic datum can be used in a program to obtain the corresponding
datum value using {\cf quote} (see section~\ref{quote}).

Scheme source code consists of syntactic data and (non-significant) comments.
Syntactic data in Scheme source code are called
\textit{forms}\mainindex{form}.
Consequently, Scheme's syntax has the property that any sequence of
characters that is a form is also a syntactic datum representing
some object.  This can lead to confusion, since it may not be obvious
out of context whether a given sequence of characters is intended to
be a representation of data or the text of a program.
It is also a source of power, since it
facilitates writing programs such as interpreters or compilers that
treat programs as data (or vice versa).  A form nested inside another form is
called a \defining{subform}.

A datum value may have several different external representations.
For example, both ``{\tt \#e28.000}'' and
``{\tt\#x1c}'' are syntactic data representing the exact integer
object 28, and the syntactic data ``{\tt(8 13)}'', ``{\tt( 08 13 )}'', ``{\tt(8 .\
  (13 .\ ()))}''
all represent a list containing the exact integer objects 8 and 13. 
Syntactic data that represent equal objects (in the sense of {\cf
  equal?}; see section~\ref{equal?}) are always equivalent 
as forms of a program.

Because of the close correspondence between syntactic data and datum
values, this report sometimes uses the term \defining{datum} for
either a syntactic datum or a datum value when the exact meaning
is apparent from the context.

An implementation must not extend the lexical or datum syntax in
any way, with one exception: it need not treat the syntax
{\cf \sharpsign{}!\meta{identifier}}, for any \meta{identifier} (see
section~\ref{identifiersection}) that is not {\cf r6rs}, as a syntax
violation, and it may use specific {\cf \sharpsign{}!}-prefixed
identifiers as flags indicating that subsequent input contains extensions
to the standard lexical or datum syntax. 
The syntax {\cf \sharpsign{}!r6rs} may be used to signify that
the input afterward is written with the lexical syntax and
datum syntax described by
this report.
{\cf \sharpsign{}!r6rs} is otherwise treated as a comment; see section~\ref{whitespaceandcomments}.

This chapter contains overviews and formal accounts of the lexical
syntax and the datum syntax.

\section{Notation}
\label{BNF}

The formal syntax for Scheme is written in an extended BNF.
Non-terminals are written using angle brackets.  Case is insignificant
for non-terminal names.

All spaces in the grammar are for legibility.
\meta{Empty} stands for the empty string.

The following extensions to BNF are used to make the description more
concise:  \arbno{\meta{thing}} means zero or more occurrences of
\meta{thing}, and \atleastone{\meta{thing}} means at least one
\meta{thing}.

Some non-terminal names refer to the Unicode scalar values of the same
name: \meta{character tabulation} (U+0009), \meta{linefeed} (U+000A),
\meta{carriage return} (U+000D), \meta{line tabulation} (U+000B),
\meta{form feed} (U+000C), \meta{carriage return} (U+000D),
\meta{space} (U+0020), \meta{next line} (U+0085), \meta{line
  separator} (U+2028), and \meta{paragraph separator} (U+2029).

\section{Lexical syntax}
\label{lexicalsyntaxsection}

The lexical syntax determines how a character sequence is split into a
sequence of lexemes\index{lexeme}, omitting non-significant portions
such as comments and whitespace.  The character sequence is assumed to
be text according to the Unicode standard~\cite{Unicode}.  Some of
the lexemes, such as number representations, identifiers, strings etc., of the lexical
syntax are syntactic data in the datum syntax, and thus represent data.
Besides the formal account of the syntax, this section also describes
what datum values are represented by these syntactic data.

The lexical syntax, in the description of comments, contains
a forward reference to \meta{datum}, which is described as part of the
datum syntax.  Being comments, however, these \meta{datum}s do not play
a significant role in the syntax.

Case is significant except in boolean data, number representations, and
hexadecimal numbers specifying Unicode scalar values.  For example, {\cf \#x1A}
and {\cf \#X1a} are equivalent.  The identifier {\cf Foo} is, however,
distinct from the identifier {\cf FOO}.

\subsection{Formal account}
\label{lexicalgrammarsection}

\meta{Interlexeme space} may occur on either side of any lexeme, but not
within a lexeme.

\vest Identifiers, number representations, characters, booleans, and dot must be terminated
by a \meta{delimiter} (e.g., parenthesis, space, or comment) or by the
end of the input.

The following two characters are reserved for future extensions to the
language: {\tt \verb"{" \verb"}"}

\begin{grammar}%
\meta{lexeme} \: \meta{identifier} \| \meta{boolean} \| \meta{number}\index{identifier}
\>  \| \meta{character} \| \meta{string}
\>  \| ( \| ) \| \openbracket{} \| \closedbracket{} \| \sharpsign( \| \sharpsign{}vu8( | \singlequote{} \| \backquote{} \| , \| ,@ \| {\bf.}
\>  \| \sharpsign\singlequote{} \| \sharpsign\backquote{} \| \sharpsign, \| \sharpsign,@
\meta{delimiter} \: ( \| ) \| \openbracket{} \| \closedbracket{} \| " \| ; \| \sharpsign{}
\>  \| \meta{whitespace}
\meta{whitespace} \: \meta{character tabulation}
\> \| \meta{linefeed} \| \meta{line tabulation} \| \meta{form feed}
\> \| \meta{carriage return} \| \meta{next line}
\> \| \meta{any character whose category is Zs, Zl, or Zp}
\meta{line ending} \: \meta{linefeed} \| \meta{carriage return}
\> \| \meta{carriage return} \meta{linefeed} \| \meta{next line}
\> \| \meta{carriage return} \meta{next line} \| \meta{line separator}
\meta{comment} \: ; \= $\langle$\rm all subsequent characters up to a
                    \>\ \rm \meta{line ending} or \meta{paragraph separator}$\rangle$\index{comment}
\qquad \= \| \meta{nested comment}
\> \| \#; \meta{interlexeme space} \meta{datum}
\> \| \#!r6rs
\meta{nested comment} \: \#| \= \meta{comment text}
\> \arbno{\meta{comment cont}} |\#
\meta{comment text} \: \= $\langle$\rm character sequence not containing
\>\ \rm {\tt \#|} or {\tt |\#}$\rangle$
\meta{comment cont} \: \meta{nested comment} \meta{comment text}
\meta{atmosphere} \: \meta{whitespace} \| \meta{comment}
\meta{interlexeme space} \: \arbno{\meta{atmosphere}}%
\end{grammar}

\label{extendedalphas}
\label{identifiersyntax}

% This is a kludge, but \multicolumn doesn't work in tabbing environments.
\setbox0\hbox{\cf\meta{variable} \goesto{} $\langle$}

\begin{grammar}%
\meta{identifier} \: \meta{initial} \arbno{\meta{subsequent}}
 \>  \| \meta{peculiar identifier}
\meta{initial} \: \meta{constituent} \| \meta{special initial}
 \> \| \meta{inline hex escape}
\meta{letter} \:  a \| b \| c \| ... \| z
\> \| A \| B \| C \| ... \| Z
\meta{constituent} \: \meta{letter}
 \> \| $\langle${\rm any character whose Unicode scalar value is greater than}
 \> \quad {\rm 127, and whose category is Lu, Ll, Lt, Lm, Lo, Mn,}
 \> \quad {\rm Nl, No, Pd, Pc, Po, Sc, Sm, Sk, So, or Co}$\rangle$
\meta{special initial} \: ! \| \$ \| \% \| \verb"&" \| * \| / \| : \| < \| =
 \>  \| > \| ? \| \verb"^" \| \verb"_" \| \verb"~"
\meta{subsequent} \: \meta{initial} \| \meta{digit}
 \>  \| \meta{any character whose category is Nd, Mc, or Me}
 \>  \| \meta{special subsequent}
\meta{digit} \: 0 \| 1 \| 2 \| 3 \| 4 \| 5 \| 6 \| 7 \| 8 \| 9
\meta{hex digit} \: \meta{digit}
 \> \| a \| A \| b \| B \| c \| C \| d \| D \| e \| E \| f \| F
\meta{special subsequent} \: + \| - \| .\ \| @
\meta{inline hex escape} \: \backwhack{}x\meta{hex scalar value};
\meta{hex scalar value} \: \atleastone{\meta{hex digit}}
\meta{peculiar identifier} \: + \| - \| ... \| -> \arbno{\meta{subsequent}}
\meta{boolean} \: \schtrue{} \| \#T \| \schfalse{} \| \#F
\meta{character} \: \#\backwhack{}\meta{any character}
 \>  \| \#\backwhack{}\meta{character name}
 \>  \| \#\backwhack{}x\meta{hex scalar value}
\meta{character name} \: nul \| alarm \| backspace \| tab
\> \| linefeed \| newline \| vtab \| page \| return
\> \| esc \| space \| delete
\meta{string} \: " \arbno{\meta{string element}} "
\meta{string element} \: \meta{any character other than \doublequote{} or \backwhack}
 \> \| \backwhack{}a \| \backwhack{}b \| \backwhack{}t \| \backwhack{}n \| \backwhack{}v \| \backwhack{}f \| \backwhack{}r
 \>  \| \backwhack\doublequote{} \| \backwhack\backwhack 
 \>  \| \backwhack\meta{line ending} \| \backwhack\meta{space}
 \>  \| \meta{inline hex escape}%
\end{grammar}

A \meta{hex scalar value} represents a Unicode scalar value
between 0 and \sharpsign{}x10FFFF, excluding the range
$\left[\sharpsign{}x\textrm{D800}, \sharpsign{}x\textrm{DFFF}\right]$.

\label{numbersyntax}%
The rules for \meta{num $R$}, \meta{complex $R$}, \meta{real
$R$}, \meta{ureal $R$}, \meta{uinteger $R$}, and \meta{prefix $R$} below
should be replicated for \hbox{$R = 2, 8, 10,$}
and $16$.  There are no rules for \meta{decimal $2$}, \meta{decimal
$8$}, and \meta{decimal $16$}, which means that number representations containing
decimal points or exponents must be in decimal radix.

\begin{grammar}%
\meta{number} \: \meta{num $2$} \| \meta{num $8$}
   \>  \| \meta{num $10$} \| \meta{num $16$}
\meta{num $R$} \: \meta{prefix $R$} \meta{complex $R$}
\meta{complex $R$} \: %
         \meta{real $R$} %
      \| \meta{real $R$} @ \meta{real $R$}
   \> \| \meta{real $R$} + \meta{ureal $R$} i %
      \| \meta{real $R$} - \meta{ureal $R$} i
   \> \| \meta{real $R$} + \meta{naninf} i %
      \| \meta{real $R$} - \meta{naninf} i
   \> \| \meta{real $R$} + i %
      \| \meta{real $R$} - i
   \> \| + \meta{ureal $R$} i %
      \| - \meta{ureal $R$} i 
   \> \| + \meta{naninf} i %
      \| - \meta{naninf} i
   \> \| + i %
      \| - i
\meta{real $R$} \: \meta{sign} \meta{ureal $R$}
  \> \| + \meta{naninf} \| - \meta{naninf}
\meta{naninf} \: nan.0 \| inf.0
\meta{ureal $R$} \: %
         \meta{uinteger $R$}
   \> \| \meta{uinteger $R$} / \meta{uinteger $R$}
   \> \| \meta{decimal $R$} \meta{mantissa width}
\meta{decimal $10$} \: %
         \meta{uinteger $10$} \meta{suffix}
   \> \| . \atleastone{\meta{digit $10$}} \meta{suffix}
   \> \| \atleastone{\meta{digit $10$}} . \arbno{\meta{digit $10$}} \meta{suffix}
   \> \| \atleastone{\meta{digit $10$}} . \meta{suffix}
\meta{uinteger $R$} \: \atleastone{\meta{digit $R$}}
\meta{prefix $R$} \: %
         \meta{radix $R$} \meta{exactness}
   \> \| \meta{exactness} \meta{radix $R$}
\end{grammar}

\begin{grammar}%
\meta{suffix} \: \meta{empty} 
   \> \| \meta{exponent marker} \meta{sign} \atleastone{\meta{digit $10$}}
\meta{exponent marker} \: e \| E \| s \| S \| f \| F
   \> \| d \| D \| l \| L
\meta{mantissa width} \: \meta{empty}
   \> \| | \atleastone{\meta{digit 10}}
\meta{sign} \: \meta{empty}  \| + \|  -
\meta{exactness} \: \meta{empty}
   \> \| \#i\sharpindex{i} \| \#I \| \#e\sharpindex{e} \| \#E
\meta{radix 2} \: \#b\sharpindex{b} \| \#B
\meta{radix 8} \: \#o\sharpindex{o} \| \#O
\meta{radix 10} \: \meta{empty} \| \#d \| \#D
\meta{radix 16} \: \#x\sharpindex{x} \| \#X
\meta{digit 2} \: 0 \| 1
\meta{digit 8} \: 0 \| 1 \| 2 \| 3 \| 4 \| 5 \| 6 \| 7
\meta{digit 10} \: \meta{digit}
\meta{digit 16} \: \meta{hex digit}
\end{grammar}

\subsection{Line endings}
\label{lineendings}

Line endings are significant in Scheme in single-line comments (see
section~\ref{whitespaceandcomments}) and within string literals.  In
Scheme source code, any of the line endings in \meta{line ending}
marks the end of a line.  Moreover, the two-character line endings
\meta{carriage return} \meta{linefeed} and \meta{carriage return}
\meta{next line} each count as a single line ending.

In a string literal, a line ending not preceded by a {\cf\backwhack}
stands for a linefeed character, which is the standard line-ending
character of Scheme.

\subsection{Whitespace and comments}
\label{whitespaceandcomments}

\defining{Whitespace} characters are spaces, linefeeds,
carriage returns, character tabulations, form feeds, line tabulations,
and any other character whose category is Zs, Zl, or Zp.
Whitespace is used for improved readability and
as necessary to separate lexemes from each other.  Whitespace may
occur between any two lexemes,
but not within a lexeme.  Whitespace may also occur inside a string,
where it is significant.

The lexical syntax includes several comment forms. In all cases,
comments are invisible to Scheme, except that they act as delimiters,
so, for example, a comment cannot appear in the middle of an
identifier or number representation.

A semicolon ({\tt;}) indicates the start of a line
comment.\mainindex{comment}\mainschindex{;} The comment continues to
the end of the line on which the semicolon appears.

Another way to indicate a comment is to prefix a \hyper{datum}
(cf.\ section~\ref{datumsyntax}) with {\tt \#;}\sharpindex{;}, possibly with
\meta{interlexeme space} before the \hyper{datum}.  The comment consists of
the comment prefix {\tt \#;} and the \hyper{datum} together.  This
notation is useful for ``commenting out'' sections of code.

Block comments may be indicated with properly nested {\tt
  \#|}\index{#"|@\texttt{\sharpsign\verticalbar}}\index{"|#@\texttt{\verticalbar\sharpsign}}
and {\tt |\#} pairs.

\begin{scheme}
\#|
   The FACT procedure computes the factorial
   of a non-negative integer.
|\#
(define fact
  (lambda (n)
    ;; base case
    (if (= n 0)
        \#;(= n 1)
        1       ; identity of *
        (* n (fact (- n 1))))))%
\end{scheme}

The lexeme {\cf \sharpsign{}!r6rs}, which signifies that the program text
that follows is written with the lexical and datum syntax described in this
report, is also otherwise treated as a comment.

\subsection{Identifiers}
\label{identifiersection}

Most identifiers\mainindex{identifier} allowed by other programming
languages are also acceptable to Scheme.  In general,
a sequence of letters, digits, and ``extended alphabetic
characters'' is
an identifier when it begins with a character that cannot begin a
number representation.
In addition, \ide{+}, \ide{-}, and \ide{...} are identifiers, as is
a sequence of letters, digits, and extended alphabetic
characters that begins with the two-character sequence \ide{->}.
Here are some examples of identifiers:

\begin{scheme}
lambda         q                soup
list->vector   {+}                V17a
<=             a34kTMNs         ->-
the-word-recursion-has-many-meanings%
\end{scheme}

Extended alphabetic characters may be used within identifiers as if
they were letters.  The following are extended alphabetic characters:

\begin{scheme}
!\ \$ \% \verb"&" * + - . / :\ < = > ? @ \verb"^" \verb"_" \verb"~" %
\end{scheme}

Moreover, all characters whose Unicode scalar values are greater than 127 and
whose Unicode category is Lu, Ll, Lt, Lm, Lo, Mn, Mc, Me, Nd, Nl, No, Pd,
Pc, Po, Sc, Sm, Sk, So, or Co can be used within identifiers.
In addition, any character can be used within an identifier
when specified via an \meta{inline hex escape}.  For example, the
identifier \verb|H\x65;llo| is the same as the identifier
\verb|Hello|, and the identifier \verb|\x3BB;| is the same as the
identifier $\lambda$.

Any identifier may be used as a variable\index{variable} or as a
syntactic keyword\index{syntactic keyword} (see
sections~\ref{variablesection} and~\ref{macrosection}) in a Scheme
program.
Any identifier may also be used as a syntactic datum, in which case it
represents a \textit{symbol}\index{symbol} (see section~\ref{symbolsection}).

\subsection{Booleans}

The standard boolean objects for true and false are written as
\schtrue{} and \schfalse.\sharpindex{t}\sharpindex{f}

\subsection{Characters}

Characters are written using the notation
\sharpsign\backwhack\hyper{character}\index{#\@\texttt{\sharpsign\backwhack}} or
\sharpsign\backwhack\hyper{character name} or
\sharpsign\backwhack{}x\meta{hex scalar value}.

For example:

\texonly
\newcommand{\extab}{\>}
\begin{tabbing}
{\cf\#\backwhack{}x0000000000}\=\kill
\endtexonly
\htmlonly
\newcommand{\extab}{&}
\begin{tabular}{ll}
\endhtmlonly
{\cf\#\backwhack{}a}          \extab \textrm{lower case letter a}\\
{\cf\#\backwhack{}A}          \extab \textrm{upper case letter A}\\
{\cf\#\backwhack{}(}          \extab \textrm{left parenthesis}\\
{\cf\#\backwhack{}}           \extab \textrm{space character}\\
{\cf\#\backwhack{}nul}        \extab \textrm{U+0000}\\
{\cf\#\backwhack{}alarm}      \extab \textrm{U+0007}\\
{\cf\#\backwhack{}backspace}  \extab \textrm{U+0008}\\
{\cf\#\backwhack{}tab}        \extab \textrm{U+0009}\\
{\cf\#\backwhack{}linefeed}   \extab \textrm{U+000A}\\
{\cf\#\backwhack{}newline}   \extab \textrm{U+000A}\\
{\cf\#\backwhack{}vtab}       \extab \textrm{U+000B}\\
{\cf\#\backwhack{}page}       \extab \textrm{U+000C}\\
{\cf\#\backwhack{}return}     \extab \textrm{U+000D}\\
{\cf\#\backwhack{}esc}        \extab \textrm{U+001B}\\
{\cf\#\backwhack{}space}      \extab \textrm{U+0020}\\
 \extab preferred way to write a space\\
{\cf\#\backwhack{}delete}     \extab \textrm{U+007F}\\[1ex]
{\cf\#\backwhack{}xFF}        \extab \textrm{U+00FF}\\
{\cf\#\backwhack{}x03BB}      \extab \textrm{U+03BB}\\
{\cf\#\backwhack{}x00006587}  \extab \textrm{U+6587}\\
{\cf\#\backwhack{}\(\lambda\)} \extab \textrm{U+03BB}\\[1ex]
{\cf\#\backwhack{}x0001z}     \extab \exception{\&lexical}\\
{\cf\#\backwhack{}\(\lambda\)x}         \extab \exception{\&lexical}\\
{\cf\#\backwhack{}alarmx}     \extab \exception{\&lexical}\\
{\cf\#\backwhack{}alarm x}    \extab \textrm{U+0007}\\
 \extab followed by {\cf{}x}\\
{\cf\#\backwhack{}Alarm}      \extab \exception{\&lexical}\\
{\cf\#\backwhack{}alert}      \extab \exception{\&lexical}\\
{\cf\#\backwhack{}xA}         \extab \textrm{U+000A}\\
{\cf\#\backwhack{}xFF}        \extab \textrm{U+00FF}\\
{\cf\#\backwhack{}xff}        \extab \textrm{U+00FF}\\
{\cf\#\backwhack{}x ff}       \extab \textrm{U+0078}\\
 \extab followed by another datum, {\cf{}ff}\\
{\cf\#\backwhack{}x(ff)}      \extab \textrm{U+0078}\\
 \extab followed by another datum,\\
 \extab a parenthesized {\cf{}ff}\\
{\cf\#\backwhack{}(x)}        \extab \exception{\&lexical}\\
{\cf\#\backwhack{}(x}         \extab \exception{\&lexical}\\
{\cf\#\backwhack{}((x)}       \extab \textrm{U+0028}\\
 \extab followed by another datum,\\
 \extab parenthesized {\cf{}x}\\
{\cf\#\backwhack{}x00110000}  \extab \exception{\&lexical}\\
 \extab out of range\\
{\cf\#\backwhack{}x000000001} \extab \textrm{U+0001}  \\
{\cf\#\backwhack{}xD800}      \extab \exception{\&lexical}\\
 \extab in excluded range
\htmlonly
\end{tabular}
\endhtmlonly
\texonly
\end{tabbing}
\endtexonly

(The notation \exception{\&lexical} means that the line in question is
a lexical syntax violation.)

Case is significant in \sharpsign\backwhack\hyper{character}, and in
\sharpsign\backwhack{\rm$\langle$character name$\rangle$}, % \hyper doesn't allow a linebreak
but not in \sharpsign\backwhack{}x\meta{hex scalar value}.  
A \meta{character} must be followed by a \meta{delimiter} or by the end of the input.
This rule resolves various ambiguous cases involving named characters,
requiring, for
example, the sequence of characters ``{\tt\sharpsign\backwhack space}''
to be interpreted as the space character rather than as
the character ``{\tt\sharpsign\backwhack s}'' followed
by the identifier ``{\tt pace}''.

\begin{note}
  The {\cf\sharpsign\backwhack{}newline} notation is retained for
  backward compatibility.  Its use is deprecated;
  {\cf\sharpsign\backwhack{}linefeed} should be used instead.
\end{note}

\subsection{Strings}

\vest String are written as sequences of characters enclosed within doublequotes
({\cf "}).  Within a string literal, various escape
sequences\mainindex{escape sequence} represent characters other than
themselves.  Escape sequences always start with a backslash (\backwhack{}):

\begin{itemize}
\item{\tt \backwhack{}a} : alarm, U+0007
\item{\tt \backwhack{}b} : backspace, U+0008 
\item{\tt \backwhack{}t} : character tabulation, U+0009 
\item{\tt \backwhack{}n} : linefeed, U+000A 
\item{\tt \backwhack{}v} : line tabulation, U+000B 
\item{\tt \backwhack{}f} : formfeed, U+000C 
\item{\tt \backwhack{}r} : return, U+000D 
\item{\tt \backwhack{}}\verb|"| : doublequote, U+0022 
\item{\tt \backwhack{}\backwhack{}} : backslash, U+005C 
\item{\tt \backwhack{}\hyper{line ending}} : nothing
\item{\tt \backwhack{}\hyper{space}} : space, U+0020 (useful for terminating the
  previous escape sequence before continuing with whitespace)
\item{\tt \backwhack{}x\meta{hex scalar value};} : specified character (note the
  terminating semi-colon).
\end{itemize}

These escape sequences are case-sensitive, except that the alphabetic
digits of a \meta{hex scalar value} can be uppercase or lowercase.

Any other character in a string after a backslash is a syntax violation. Except
for a line ending, any
character outside of an escape sequence and not a doublequote stands
for itself in the string literal. For example the single-character
string literal {\tt "$\lambda$"} (doublequote, a lower case lambda, doublequote)
represents the same string as {\tt "\backwhack{}x03bb;"}.
A line ending that does not follow a backslash stands for a linefeed character.

Examples:

\texonly
\begin{tabbing}
{\cf "\backwhack{}x0000000000;"} \=\kill
\endtexonly
\htmlonly
\begin{tabular}{ll}
\endhtmlonly
{\cf "abc"} \extab  \textrm{U+0061, U+0062, U+0063}\\
{\cf "\backwhack{}x41;bc"} \extab  {\cf "Abc"} ; \textrm{U+0041, U+0062, U+0063}\\
{\cf "\backwhack{}x41; bc"} \extab {\cf "A bc"}\\
 \extab U+0041, U+0020, U+0062, U+0063\\
{\cf "\backwhack{}x41bc;"} \extab  \textrm{U+41BC}\\
{\cf "\backwhack{}x41"} \extab \exception{\&lexical}\\
{\cf "\backwhack{}x;"} \extab \exception{\&lexical}\\
{\cf "\backwhack{}x41bx;"} \extab \exception{\&lexical}\\
{\cf "\backwhack{}x00000041;"} \extab  {\cf "A"} ; \textrm{U+0041}\\
{\cf "\backwhack{}x0010FFFF;"} \extab \textrm{U+10FFFF}\\
{\cf "\backwhack{}x00110000;"} \extab  \exception{\&lexical}\\
 \extab out of range\\
{\cf "\backwhack{}x000000001;"} \extab \textrm{U+0001}\\
{\cf "\backwhack{}xD800;"} \extab \exception{\&lexical}\\
 \extab in excluded range\\
{\cf "A}\\
{\cf bc"} \extab \textrm{U+0041, U+000A, U+0062, U+0063}\\
 \extab if no space occurs after the {\cf{}A}
\htmlonly
\end{tabular}
\endhtmlonly
\texonly
\end{tabbing}
\endtexonly
  
\subsection{Numbers}
\label{numbernotations}

The syntax of written representations for number objects is described
formally by the \meta{number} rule in the formal grammar.
Case is not significant in numerical constants.

A number representation may be written in binary, octal, decimal, or
hexadecimal by the use of a radix prefix.  The radix prefixes are {\cf
\#b}\sharpindex{b} (binary), {\cf \#o}\sharpindex{o} (octal), {\cf
\#d}\sharpindex{d} (decimal), and {\cf \#x}\sharpindex{x} (hexadecimal).  With
no radix prefix, a number representation is assumed to be expressed in decimal.

A
numerical constant may be specified to be either exact or
inexact by a prefix.  The prefixes are {\cf \#e}\sharpindex{e}
for exact, and {\cf \#i}\sharpindex{i} for inexact.  An exactness
prefix may appear before or after any radix prefix that is used.  If
the written representation of a number has no exactness prefix, the
constant is
inexact if it contains a decimal point, an
exponent, or
a nonempty mantissa width;
otherwise it is exact.

In systems with inexact number objects
of varying precisions, it may be useful to specify
the precision of a constant.  For this purpose, numerical constants
may be written with an exponent marker that indicates the
desired precision of the inexact
representation.  The letters {\cf s}, {\cf f},
{\cf d}, and {\cf l} specify the use of \var{short}, \var{single},
\var{double}, and \var{long} precision, respectively.  (When fewer
than four internal
inexact
representations exist, the four size
specifications are mapped onto those available.  For example, an
implementation with two internal representations may map short and
single together and long and double together.)  In addition, the
exponent marker {\cf e} specifies the default precision for the
implementation.  The default precision has at least as much precision
as \var{double}, but
implementations may wish to allow this default to be set by the user.

\begin{scheme}
3.1415926535898F0 
       {\rm{}Round to single, perhaps} 3.141593
0.6L0
       {\rm{}Extend to long, perhaps} .600000000000000%
\end{scheme}

A number representation with nonempty mantissa width,
{\cf \var{x}|\var{p}}, represents the best binary
floating-point approximation of \var{x} using a \var{p}-bit significand. 
For example, {\cf 1.1|53} is a
representation of the best approximation of 1.1 in IEEE double
precision.
If \var{x} is an external representation of an inexact real number object
that contains no vertical bar, it should be treated as if specified
with a mantissa width of {\cf 53}.

Implementations that use binary floating-point representations
of real number objects should represent {\cf \var{x}|\var{p}}
using a \var{p}-bit significand if practical, or by a greater
precision if a \var{p}-bit significand is not practical, or
by the largest available precision if \var{p} or more bits
of significand are not practical within the implementation.

\begin{note}
The precision of a significand should not be confused with the
number of bits used to represent the significand.  In the IEEE
floating-point standards, for example, the significand's most
significant bit is implicit in single and double precision but
is explicit in extended precision.  Whether that bit is implicit
or explicit does not affect the mathematical precision.
In implementations that use binary floating point, the default
precision can be calculated by calling the following procedure:

\begin{scheme}
(define (precision)
  (do ((n 0 (+ n 1))
       (x 1.0 (/ x 2.0)))
    ((= 1.0 (+ 1.0 x)) n)))
\end{scheme}
\end{note}      

\begin{note}
When the underlying floating-point representation is IEEE double
precision, the {\cf |\var{p}} suffix should not always be omitted:
Denormalized floating-point representations have diminished precision,
and therefore their external representation should
carry a {\cf |\var{p}} suffix with the actual width of the
significand.
\end{note}

The literals {\cf +inf.0} and {\cf -inf.0} represent positive and
negative infinity, respectively.  The {\cf +nan.0}
literal represents the NaN that is the result of {\cf (/ 0.0 0.0)},
and may represent other NaNs as well.

If \var{x} is an external representation of an inexact real number
object and
contains no vertical bar and no exponent marker
other than {\cf e}, the inexact real number object it represents is a flonum
(see library section~\extref{lib:flonumssection}{Flonums}).
Some or all of the other external representations of
inexact real number objects may also represent flonums, but that is not required by
this report.

\section{Datum syntax}
\label{datumsyntaxsection}

The datum syntax describes the syntax of
syntactic data\mainindex{syntactic datum} in terms of a sequence of
\meta{lexeme}s, as defined in the lexical syntax.

Syntactic data include the lexeme data described in the
previous section as well as the following constructs for forming
compound data:
%
\begin{itemize}
\item pairs and lists, enclosed by \verb|( )| or \verb|[ ]| (see
  section~\ref{pairlistsyntax})
\item vectors (see section~\ref{vectorsyntax})
\item bytevectors (see section~\ref{bytevectorsyntax})
\end{itemize}

\subsection{Formal account}
\label{datumsyntax}

The following grammar describes the syntax of syntactic data in terms
of various kinds of lexemes defined in the grammar in
section~\ref{lexicalsyntaxsection}:

\begin{grammar}%
\meta{datum} \: \meta{lexeme datum}
\>  \| \meta{compound datum}
\meta{lexeme datum} \: \meta{boolean} \| \meta{number}
\>  \| \meta{character} \| \meta{string} \|  \meta{symbol}
\meta{symbol} \: \meta{identifier}
\meta{compound datum} \: \meta{list} \| \meta{vector} \| \meta{bytevector}
\meta{list} \: (\arbno{\meta{datum}}) \| [\arbno{\meta{datum}}]
\>    \| (\atleastone{\meta{datum}} .\ \meta{datum}) \| [\atleastone{\meta{datum}} .\ \meta{datum}]
\>    \| \meta{abbreviation}
\meta{abbreviation} \: \meta{abbrev prefix} \meta{datum}
\meta{abbrev prefix} \: ' \| ` \| , \| ,@ \| \#' | \#` | \#, | \#,@
\meta{vector} \: \#(\arbno{\meta{datum}})
\meta{bytevector} \: \#vu8(\arbno{\meta{u8}})
\meta{u8} \: $\langle${\rm any \meta{number} representing an exact}
 \>\>\quad\quad {\rm integer in $\{0, \ldots, 255\}$}$\rangle$%
\end{grammar}

\subsection{Pairs and lists}
\label{pairlistsyntax}

List and pair data, representing pairs and lists of values
(see section~\ref{listsection}) are written using parentheses or brackets.
Matching pairs of brackets that occur in the rules of \meta{list} are
equivalent to matching pairs of parentheses.

The most general notation for Scheme pairs as syntactic data is
the ``dotted'' notation \hbox{\cf (\hyperi{datum} .\ \hyperii{datum})} where
\hyperi{datum} is the representation of the value of the car field and
\hyperii{datum} is the representation of the value of the
cdr field.  For example {\cf (4 .\ 5)} is a pair whose car is 4 and whose
cdr is 5.

A more streamlined notation can be used for lists: the elements of the
list are simply enclosed in parentheses and separated by spaces.  The
empty list\index{empty list} is written {\tt()} .  For example,

\begin{scheme}
(a b c d e)%
\end{scheme}

and

\begin{scheme}
(a . (b . (c . (d . (e . ())))))%
\end{scheme}

are equivalent notations for a list of symbols.

The general rule is that, if a dot is followed by an open parenthesis,
the dot, open parenthesis, and matching closing parenthesis
can be omitted in the external representation.

The sequence of characters ``{\cf (4 .\ 5)}'' is the external representation of a
pair, not an expression that evaluates to a pair.
Similarly, the sequence of characters ``{\tt(+ 2 6)}'' is {\em not} an
external representation of the integer 8, even though it {\em is} an
expression (in the language of the \rsixlibrary{base} library)
evaluating to the integer 8; rather, it is a
syntactic datum representing a three-element list, the elements of which
are the symbol {\tt +} and the integers 2 and 6.

\subsection{Vectors}
\label{vectorsyntax}

Vector data, representing vectors of values (see
section~\ref{vectorsection}), are written using the notation
{\tt\#(\hyper{datum} \dotsfoo)}.  For example, a vector of length 3
containing the number zero in element 0, the list {\cf(2 2 2 2)} in
element 1, and the string {\cf "Anna"} in element 2 can be written as
following:

\begin{scheme}
\#(0 (2 2 2 2) "Anna")%
\end{scheme}

This is the external representation of a vector, not a
base-library expression that evaluates to a vector.

\subsection{Bytevectors}
\label{bytevectorsyntax}

Bytevector data, representing bytevectors (see
library chapter~\extref{lib:bytevectorschapter}{Bytevectors}), are written using the notation
{\tt\#vu8(\hyper{u8} \dotsfoo)}, where the \hyper{u8}s represent the octets of
the bytevector.  For example, a bytevector of length 3 containing the
octets 2, 24, and 123 can be written as follows:

\begin{scheme}
\#vu8(2 24 123)%
\end{scheme}

This is the external representation of a bytevector, and also an
expression that evaluates to a bytevector.

\subsection{Abbreviations}\unsection
\label{abbreviationsection}

\begin{entry}{%
\pproto{\singlequote\hyper{datum}}{}
\pproto{\backquote\hyper{datum}}{}
\pproto{,\hyper{datum}}{}
\pproto{,\atsign\hyper{datum}}{}
\pproto{\#'\hyper{datum}}{}
\pproto{\#\backquote\hyper{datum}}{}
\pproto{\#,\hyper{datum}}{}
\pproto{\#,@\hyper{datum}}{}
}

Each of these is an abbreviation:
\\\quad\schindex{'}\singlequote\hyper{datum}
for {\cf (quote \hyper{datum})},
\\\quad\schindex{`}\backquote\hyper{datum}
for {\cf (quasiquote \hyper{datum})},
\\\quad\schindex{,}{\cf,}\hyper{datum}
for {\cf (unquote \hyper{datum})},
\\\quad\index{,@\texttt{,\atsign}}{\cf,}\atsign\hyper{datum}
for {\cf (unquote-splicing \hyper{datum})},
\\\quad\sharpindex{'}{\cf\#'}\hyper{datum}
for {\cf (syntax \hyper{datum})},
\\\quad\sharpindex{`}{\cf\#`}\hyper{datum}
for {\cf (quasisyntax \hyper{datum})},
\\\quad\sharpindex{,}{\cf\#,}\hyper{datum}
for {\cf (unsyntax \hyper{datum})}, and
\\\quad\index{#,@\texttt{\#,\atsign}}{\cf\#,@}\hyper{datum}
for {\cf (unsyntax-splicing \hyper{datum})}.
\end{entry}

%%% Local Variables: 
%%% mode: latex
%%% TeX-master: "r6rs"
%%% End: 

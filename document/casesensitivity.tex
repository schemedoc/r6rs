\chapter{Optional case insensitivity}
\label{caseinsensitivityappendix}

The report specifies that the lexical syntax of symbols is
case-sensitive.  This is a major change from the previous revision of
the report~\cite{R5RS}, which breaks compatibility.  Code written
under the old assumption of case-insensitivity may break because of
this change.  Implementations may support such code by implementing
the following directives:

\begin{entry}{%
{\cf{}\#!fold-case}\sharpbangindex{fold-case}\\
{\cf{}\#!no-fold-case}\sharpbangindex{no-fold-case}}

These directives may appear in the read syntax everywhere a
\hyper{datum} occurs, and affects the reading of subsequent symbols.
The {\cf{}\#!fold-case} directive means the name of a each subsequent
symbol is the result of case-folding (see library
section~\extref{lib:string-foldcase}{``Unicode''}) the name that appears in the
source text.  The {\cf{}\#!no-fold-case} directive means the name of a each
subsequent symbol is precisely the name that appears in the source text.

The region where each of these directives has effect depends on the
context: If the directive occurs at top level, the region extends
until the next such directive at top level, or until the end of the
source text if no such directive follows.  If the directive occurs in
the immediate context of a \hyper{list} or \hyper{vector} datum (i.e.\
in a place that would make it an element of the list or vector if it
were a datum), the region of its effect extends until the next such
directive in the same context, or until the end of the context if no
such directive follows. A directive occurring inside the region of
effect of another directive takes precedence.
\end{entry}

%%% Local Variables: 
%%% mode: latex
%%% TeX-master: "r6rs-app"
%%% End: 

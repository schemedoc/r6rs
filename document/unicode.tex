\chapter{Unicode}
\label{unicodechapter}

The procedures exported by the \deflibrary{r6rs unicode}
library provide access to some aspects
of the Unicode semantics for characters and strings:
category information, case-independent comparisons,
case mappings, and normalization~\cite{Unicode}.

Some of the procedures that operate on characters or strings ignore the
difference between upper case and lower case.  The procedures that
ignore case have \hbox{``{\tt -ci}''} (for ``case
insensitive'') embedded in their names.

\section{Characters}

\begin{entry}{%
\proto{char-upcase}{ char}{procedure}
\proto{char-downcase}{ char}{procedure}
\proto{char-titlecase}{ char}{procedure}
\proto{char-foldcase}{ char}{procedure}}

These procedures take a character argument and return a character
result. If the argument is an upper case or title case character, and if
there is a single character that is its lower case form, then
{\cf char-downcase} returns that character. If the argument is a lower case
or title case character, and there is a single character that is
its upper case form, then {\cf char-upcase} returns that character.
If the argument is a lower case
or upper case character, and there is a single character that is
its title case form, then {\cf char-titlecase} returns that character.
Finally, if the character has a case-folded character,
then {\cf char-foldcase} returns that character.
Otherwise the character returned is the same
as the argument.
For Turkic characters \.I ({\tt \#\backwhack{}x130})
and \i{} ({\tt \#\backwhack{}x131}),
{\cf char-foldcase} behaves as the identity function; otherwise 
{\cf char-foldcase} is the
same as {\cf char-downcase} composed with {\cf char-upcase}.

\begin{scheme}
(char-upcase \sharpsign\backwhack{}i) \ev \sharpsign\backwhack{}I
(char-downcase \sharpsign\backwhack{}i) \ev \sharpsign\backwhack{}i
(char-titlecase \sharpsign\backwhack{}i) \ev \sharpsign\backwhack{}I
(char-foldcase \sharpsign\backwhack{}i) \ev \sharpsign\backwhack{}i

(char-upcase \sharpsign\backwhack{}\ss) \ev \sharpsign\backwhack{}\ss
(char-downcase \sharpsign\backwhack{}\ss) \ev \sharpsign\backwhack{}\ss
(char-titlecase \sharpsign\backwhack{}\ss) \ev \sharpsign\backwhack{}\ss
(char-foldcase \sharpsign\backwhack{}\ss) \ev \sharpsign\backwhack{}\ss

(char-upcase \sharpsign\backwhack{}$\Sigma$) \ev \sharpsign\backwhack{}$\Sigma$
(char-downcase \sharpsign\backwhack{}$\Sigma$) \ev \sharpsign\backwhack{}$\sigma$
(char-titlecase \sharpsign\backwhack{}$\Sigma$) \ev \sharpsign\backwhack{}$\Sigma$
(char-foldcase \sharpsign\backwhack{}$\Sigma$) \ev \sharpsign\backwhack{}$\sigma$

(char-upcase \sharpsign\backwhack{}$\varsigma$) \ev \sharpsign\backwhack{}$\Sigma$
(char-downcase \sharpsign\backwhack{}$\varsigma$) \ev \sharpsign\backwhack{}$\varsigma$
(char-titlecase \sharpsign\backwhack{}$\varsigma$) \ev \sharpsign\backwhack{}$\Sigma$
(char-foldcase \sharpsign\backwhack{}$\varsigma$) \ev \sharpsign\backwhack{}$\sigma$
\end{scheme}

\begin{note}
  These procedures are consistent with
  Unicode's locale-independent mappings from scalar values to
  scalar values for upcase, downcase, titlecase, and case-folding
  operations.  These mappings can be extracted from {\cf
    UnicodeData.txt} and {\cf CaseFolding.txt} from the Unicode
  Consortium, ignoring Turkic mappings in the latter.

  Note that these character-based procedures are an incomplete
  approximation to case conversion, even ignoring the user's locale.
  In general, case mappings require the context of a string, both in
  arguments and in result. The {\cf string-upcase}, {\cf
    string-downcase}, {\cf string-titlecase}, and {\cf
    string-foldcase} procedures (section~\ref{string-upcase})
  perform more general case conversion.
\end{note}
\end{entry}


\begin{entry}{%
\proto{char-ci=?}{ \vari{char} \varii{char} \variii{char} \dotsfoo}{procedure}
\proto{char-ci<?}{ \vari{char} \varii{char} \variii{char} \dotsfoo}{procedure}
\proto{char-ci>?}{ \vari{char} \varii{char} \variii{char} \dotsfoo}{procedure}
\proto{char-ci<=?}{ \vari{char} \varii{char} \variii{char} \dotsfoo}{procedure}
\proto{char-ci>=?}{ \vari{char} \varii{char} \variii{char} \dotsfoo}{procedure}}

These procedures are similar to {\cf char=?}\ et cetera, but operate
on the case-folded versions of the characters.

\begin{scheme}
(char-ci<? \sharpsign\backwhack{}z \sharpsign\backwhack{}Z) \ev \schfalse
(char-ci=? \sharpsign\backwhack{}z \sharpsign\backwhack{}Z) \ev \schtrue
(char-ci=? \sharpsign\backwhack{}$\varsigma$ \sharpsign\backwhack{}$\sigma$) \ev \schtrue
\end{scheme}
\end{entry}


\begin{entry}{%
\proto{char-alphabetic?}{ char}{procedure}
\proto{char-numeric?}{ char}{procedure}
\proto{char-whitespace?}{ char}{procedure}
\proto{char-upper-case?}{ letter}{procedure}
\proto{char-lower-case?}{ letter}{procedure}
\proto{char-title-case?}{ letter}{procedure}}

These procedures return \schtrue{} if their arguments are alphabetic,
numeric, whitespace, upper case, lower case, or title case characters,
respectively; otherwise they return \schfalse.

A character is alphabetic if it is a Unicode letter, i.e.\ if it is in
one of the categories Lu, Ll, Lt, Lm, and Lo.  A character is numeric if
it is in category Nd.  A character is whitespace if it is in one of
the space, line, or paragraph separator categories (Zs, Zl or Zp), or
if is U+0009 (Horizontal tabulation), U+000A (Line feed),
U+000B (Vertical tabulation), U+000C (Form feed), or U+000D
(Carriage return).  A character is upper case if it has the Unicode
``Uppercase'' property, lower case if it has the ``Lowercase''
property, and title case if it is in the Lt general category.

\begin{scheme}
(char-alphabetic? \sharpsign\backwhack{}a) \ev \schtrue{}
(char-numeric? \sharpsign\backwhack{}1) \ev \schtrue{}
(char-whitespace? \sharpsign\backwhack{}space) \ev \schtrue{}
(char-whitespace? \sharpsign\backwhack{}x00A0) \ev \schtrue{}
(char-upper-case? \sharpsign\backwhack{}$\Sigma$) \ev \schtrue{}
(char-lower-case? \sharpsign\backwhack{}$\sigma$) \ev \schtrue{}
(char-lower-case? \sharpsign\backwhack{}x00AA) \ev \schtrue{}
(char-title-case? \sharpsign\backwhack{}I) \ev \schfalse{}
(char-title-case? \sharpsign\backwhack{}x01C5) \ev \schtrue{}
\end{scheme}
\end{entry}

\begin{entry}{%
\proto{char-general-category}{ char}{procedure}}

Returns a symbol representing the
Unicode general category of \var{char}, one of {\cf Lu}, {\cf Ll}, {\cf Lt},
{\cf Lm}, {\cf Lo}, {\cf Mn}, {\cf Mc}, {\cf Me}, {\cf Nd}, {\cf Nl},
{\cf No}, {\cf Ps}, {\cf Pe}, {\cf Pi}, {\cf Pf}, {\cf Pd}, {\cf Pc},
{\cf Po}, {\cf Sc}, {\cf Sm}, {\cf Sk}, {\cf So}, {\cf Zs}, {\cf Zp},
{\cf Zl}, {\cf Cc}, {\cf Cf}, {\cf Cs}, {\cf Co}, or {\cf Cn}.

\begin{scheme}
(char-general-category \#\backwhack{}a) \ev Ll
(char-general-category \#\backwhack{}space) \lev Zs
(char-general-category \#\backwhack{}x10FFFF) \lev Cn  
\end{scheme}
\end{entry}

\section{Strings}

\begin{entry}{%
\proto{string-upcase}{ \var{string}}{procedure}
\proto{string-downcase}{ \var{string}}{procedure}
\proto{string-titlecase}{ \var{string}}{procedure}
\proto{string-foldcase}{ \var{string}}{procedure}}

These procedures take a string argument and return a string
result.  They are defined in terms of Unicode's locale-independent
case mappings from Unicode-scalar-value sequences to scalar-value sequences.
In particular, the length of the result string can be different from
the length of the input string.

The {\cf string-upcase} procedure converts a string to upper case;
{\cf string-downcase} converts a string to lowercase. The {\cf
  string-foldcase} procedure converts the string to its case-folded
counterpart, using the full case-folding mapping, but without the
special mappings for Turkic languages.  The {\cf string-titlecase}
procedure converts the first character to title case in each
contiguous sequence of cased characters within \var{string}, and it
downcases all other cased characters; for the purposes of detecting
cased-character sequences, case-ignorable characters are ignored
(i.e., they do not interrupt the sequence).

\begin{scheme}
(string-upcase "Hi") \ev "HI"
(string-downcase "Hi") \ev "hi"
(string-foldcase "Hi") \ev "hi"

(string-upcase "Stra\ss{}e") \ev "STRASSE"
(string-downcase "Stra\ss{}e") \ev "stra\ss{}e"
(string-foldcase "Stra\ss{}e") \ev "strasse"
(string-downcase "STRASSE")  \ev "strasse"

(string-downcase "$\Sigma$") \ev "$\sigma$"

; \textrm{Chi Alpha Omicron Sigma}:
(string-upcase "$\mathit{XAO}\Sigma$") \ev "$\mathit{XAO}\Sigma$" 
(string-downcase "$\mathit{XAO}\Sigma$") \ev "$\chi\alpha{}o\varsigma$"
(string-downcase "$\mathit{XAO}\Sigma\Sigma$") \ev "$\chi\alpha{}o\sigma\varsigma$"
(string-downcase "$\mathit{XAO}\Sigma~\Sigma$") \ev "$\chi\alpha{}o\varsigma~\sigma$"
(string-foldcase "$\mathit{XAO}\Sigma\Sigma$") \ev "$\chi\alpha{}o\sigma\sigma$"
(string-upcase "$\chi\alpha{}o\varsigma$") \ev "$\mathit{XAO}\Sigma$"
(string-upcase "$\chi\alpha{}o\sigma$") \ev "$\mathit{XAO}\Sigma$"

(string-titlecase "kNock KNoCK")
\ev "Knock Knock"
(string-titlecase "who's there?")
\ev "Who's There?"
(string-titlecase "r6rs") \ev "R6Rs"
(string-titlecase "R6RS") \ev "R6Rs"
\end{scheme}

\begin{note}
  The case mappings needed for implementing these procedures
  can be extracted from {\cf UnicodeData.txt}, {\cf
    SpecialCasing.txt}, {\cf WordBreakProperty.txt} 
  (the ``MidLetter'' property partly defines case-ignorable characters), 
  and {\cf CaseFolding.txt} from the Unicode Consortium.

  Since these procedures are locale-independent, they may not
  be completely appropriate for some locales.
\end{note}

\end{entry}

\begin{entry}{%
\proto{string-ci=?}{ \vari{string} \varii{string} \variii{string} \dotsfoo}{procedure}
\proto{string-ci<?}{ \vari{string} \varii{string} \variii{string} \dotsfoo}{procedure}
\proto{string-ci>?}{ \vari{string} \varii{string} \variii{string} \dotsfoo}{procedure}
\proto{string-ci<=?}{ \vari{string} \varii{string} \variii{string} \dotsfoo}{procedure}
\proto{string-ci>=?}{ \vari{string} \varii{string} \variii{string} \dotsfoo}{procedure}}

These procedures are similar to {\cf string=?}\ et cetera, but 
operate on the case-folded versions of the strings.

\begin{scheme}
(string-ci<? "z" "Z") \ev \schfalse
(string-ci=? "z" "Z") \ev \schtrue
(string-ci=? "Stra\ss{}e" "Strasse") 
\ev \schtrue
(string-ci=? "Stra\ss{}e" "STRASSE")
\ev \schtrue
(string-ci=? "$\mathit{XAO}\Sigma$" "$\chi\alpha{}o\sigma$")
\ev \schtrue
\end{scheme}

\end{entry}

\begin{entry}{
\proto{string-normalize-nfd}{ \var{string}}{procedure}
\proto{string-normalize-nfkd}{ \var{string}}{procedure}
\proto{string-normalize-nfc}{ \var{string}}{procedure}
\proto{string-normalize-nfkc}{ \var{string}}{procedure}}
  
These procedures take a string argument and return a string
result, which is the input string normalized
to Unicode normalization form D, KD, C, or KC, respectively.

\begin{scheme}
(string-normalize-nfd "\backwhack{}xE9;")
\ev "\backwhack{}x65;\backwhack{}x301;"
(string-normalize-nfc "\backwhack{}xE9;")
\ev "\backwhack{}xE9;"
(string-normalize-nfd "\backwhack{}x65;\backwhack{}x301;")
\ev "\backwhack{}x65;\backwhack{}x301;"
(string-normalize-nfc "\backwhack{}x65;\backwhack{}x301;")
\ev "\backwhack{}xE9;"
\end{scheme}
\end{entry}

%%% Local Variables: 
%%% mode: latex
%%% TeX-master: "r6rs"
%%% End: 

\chapter{Primitive syntax}

After the {\cf import} form within a {\cf library} form or a top-level
program, the forms
that constitute a the body of the library or the top-level program
depend on the libraries that are
imported. In particular, imported syntactic keywords determine most
of the available forms and whether each form is a 
definition or expression. A few form types are
always available independent of imported libraries, however,
including constant literals, variable references, procedure calls,
 and macro uses.

\subsection{Primitive expression types}
\label{primitiveexpressionsection}

The entries in this section all describe expressions, which may occur
in the place of \hyper{expression} syntactic variables.  See
also section~\ref{expressionsection}.

\subsubsection*{Constant literals}\unsection

\begin{entry}{%
\pproto{\hyper{number}}{\exprtype}
\pproto{\hyper{boolean}}{\exprtype}
\pproto{\hyper{character}}{\exprtype}
\pproto{\hyper{string}}{\exprtype}
\pproto{\hyper{bytevector}}{\exprtype}}\mainindex{literal}

An expression consisting of a number representation, a boolean, a character, a
string, or a bytevector, evaluates ``to itself.

\begin{scheme}
145932     \ev  145932
\schtrue   \ev  \schtrue
"abc"      \ev  "abc"
\#vu8(2 24 123) \ev \#vu8(2 24 123)%
\end{scheme}

As noted in section~\ref{storagemodel}, the value of a literal
expression is immutable.
\end{entry}

\subsubsection*{Variable references}\unsection
\begin{entry}{%
\pproto{\hyper{variable}}{\exprtype}}

An expression consisting of a variable\index{variable}
(section~\ref{variablesection}) is a variable reference.  The value of
the variable reference is the value stored in the location to which the
variable is bound.  It is a syntax violation to reference
an unbound\index{unbound} variable.

\begin{scheme}
; These examples assume the base library
; has been imported.
(define x 28)
x   \ev  28%
\end{scheme}
\end{entry}

\subsubsection*{Procedure calls}\unsection

\begin{entry}{%
\pproto{(\hyper{operator} \hyperi{operand} \dotsfoo)}{\exprtype}}

A procedure call is written by simply enclosing in parentheses
expressions for the procedure to be called and the arguments to be
passed to it.  A form in an expression context is a procedure
call if \hyper{operator} is not an identifier bound as a syntactic keyword.

When a procedure call is evaluated, the operator and operand
expressions are evaluated (in an unspecified order) and the resulting
procedure is passed the resulting
arguments.\mainindex{call}\mainindex{procedure call}
\begin{scheme}%
; These examples assume the base library
; has been imported.
(+ 3 4)                          \ev  7
((if \schfalse + *) 3 4)         \ev  12%
\end{scheme}

If the value of \hyper{operator} is not a procedure, an exception with
condition type {\cf\&assertion} is raised.  Also, if \hyper{operator}
does not accept as many arguments as there are \hyper{operand}s, an
exception with condition type {\cf\&assertion} is raised.

\begin{note} In contrast to other dialects of Lisp, the order of
evaluation is unspecified, and the operator expression and the operand
expressions are always evaluated with the same evaluation rules.
\end{note}

\begin{note}
Although the order of evaluation is otherwise unspecified, the effect of
any concurrent evaluation of the operator and operand expressions is
constrained to be consistent with some sequential order of evaluation.
The order of evaluation may be chosen differently for each procedure call.
\end{note}

\begin{note} In many dialects of Lisp, the form {\tt
()} is a legitimate expression.  In Scheme, expressions written as
list/pair forms must have at
least one subexpression, so {\tt ()} is not a syntactically valid
expression.
\end{note}

\todo{Freeman:
I think an explanation as to why evaluation order is not specified
should be included.  It should not include any reference to parallel
evaluation.  Does any existing compiler generate better code because
the evaluation order is unspecified?  Clinger: yes: T3, MacScheme v2,
probably MIT Scheme and Chez Scheme.  But that's not the main reason
for leaving the order unspecified.}

\end{entry}

\subsection{Macros}
\label{macrosection}

Libraries and top-level programs can define and use new kinds of derived expressions and
definitions called {\em syntactic abstractions} or
{\em macros}.\mainindex{syntactic abstraction}\mainindex{macro}
A syntactic abstraction is created by binding a keyword to a
{\em macro transformer} or, simply, {\em transformer}.
\index{macro transformer}\index{transformer}
The transformer determines
how a use of the macro is transcribed into a more primitive
form.

Most macro uses have the form:
\begin{scheme}
(\hyper{keyword} \hyper{datum} \dotsfoo)%
\end{scheme}%
where \hyper{keyword} is an identifier that uniquely determines the
type of form.  This identifier is called the {\em syntactic
keyword}\index{syntactic keyword}, or simply {\em
keyword}\index{keyword}, of the macro\index{macro keyword}.
The number of \hyper{datum}s and the syntax
of each depends on the syntactic abstraction.

Macro uses can also take the form of improper lists, singleton
identifiers, or {\cf set!} forms, where the second subform of the
{\cf set!} is the keyword (see section~\ref{identifier-syntax})
library section~\extref{lib:make-variable-transformer}{{\cf make-variable-transformer}}):
\begin{scheme}
(\hyper{keyword} \hyper{datum} \dotsfoo . \hyper{datum})
\hyper{keyword}
(set! \hyper{keyword} \hyper{datum})%
\end{scheme}

The {\cf define-syntax}, {\cf let-syntax} and {\cf letrec-syntax}
forms, described in sections~\ref{define-syntax} and \ref{let-syntax},
create bindings for keywords, associate them with macro transformers,
and control the scope within which they are visible.

The {\cf syntax-rules} and {\cf identifier-syntax} forms, described in
section~\ref{syntaxrulessection}, create transformers via a pattern
language.  Moreover, the {\cf syntax-case} form, described in library
chapter~\extref{lib:syntaxcasechapter}{{\cf syntax-case}}, 
creates transformers via a pattern language that permits the use of
arbitrary Scheme code.

Keywords occupy the same name space as variables.
That is, within the same
scope, an identifier can be bound as a variable or keyword, or neither, but
not both, and local bindings of either kind may shadow other bindings of
either kind.

Macros defined using {\cf syntax-rules} and {\cf identifier-syntax}
are ``hygienic'' and ``referentially transparent'' and thus preserve
Scheme's lexical scoping~\cite{Kohlbecker86,
  hygienic,Bawden88,macrosthatwork,syntacticabstraction}:
\mainindex{hygienic} \mainindex{referentially transparent}

\begin{itemize}
\item If a macro transformer inserts a binding for an identifier
(variable or keyword), the identifier is in effect renamed
throughout its scope to avoid conflicts with other identifiers.

\item If a macro transformer inserts a free reference to an
identifier, the reference refers to the binding that was visible
where the transformer was specified, regardless of any local
bindings that may surround the use of the macro.
\end{itemize}

Macros defined using the {\cf syntax-case} facility are also
hygienic unless {\cf datum\coerce{}syntax}
(see library section~\extref{lib:conversionssection}{Syntax-object and datum conversions}) is used.

%%% Local Variables: 
%%% mode: latex
%%% TeX-master: "r6rs"
%%% End: 

\chapter{List utilities}
\label{listutilities}

This chapter describes the \deflibrary{r6rs lists} library.

\begin{entry}{%
\proto{find}{ proc list}{procedure}}

\domain{\var{Proc} must be a procedure that takes a single argument
  and returns a single value.  \var{Proc} must not mutate \var{list}.}
The {\cf find} procedure applies \var{proc} to the elements of
\var{list} in order.  If \var{proc} returns a true value for an
element, {\cf find} immediately returns that element.  If \var{proc}
returns \schfalse{} for all elements of the list, it returns
\schfalse{}.  \var{Proc} is always called in same dynamic environment
as {\cf find} itself.

\begin{scheme}
(find even? '(3 1 4 1 5 9)) \ev 4
(find even? '(3 1 5 1 5 9)) \ev \schfalse{}
\end{scheme}

\implresp The implementation must check that \var{list} is a chain of
pairs up to the found element, or that it is indeed a list if no
element is found.  It should not check that it is a chain of pairs
beyond the found element.  The implementation must check the restrictions on
\var{proc} to the extent performed by applying it as described.
\end{entry}

\begin{entry}{%
\proto{for-all}{ proc \vari{list} \varii{list} \dotsfoo{} \varn{list}}{procedure}
\proto{exists}{ proc \vari{list} \varii{list} \dotsfoo{} \varn{list}}{procedure}}

\domain{The \var{list}s must all have the same length, and
  \var{proc} must be a procedure that accepts $n$ arguments and
  returns a single value.
  \var{Proc} must not mutate the \var{list} arguments.}

For natural numbers $i = 0, 1, \ldots$, the {\cf for-all} procedure
successively applies \var{proc} to arguments $x_i^1 \ldots x_i^n$,
where $x_i^j$ is the $i$th element of \varj{list}, until \schfalse{} is
returned.  If \var{proc} returns true values for all but the last
element of \vari{list}, \var{for-all} performs a tail call of \var{proc}
on the $k$th elements, where $k$ is the length of \vari{list}.  If \var{proc}
returns \schfalse{} on any set of elements, {\cf for-all} returns
\schfalse{} after the first such application of \var{proc}.
If the \var{list}s are all empty, {\cf
  for-all} returns \schtrue.

For natural numbers $i = 0, 1, \ldots$, the {\cf exists} procedure
applies \var{proc} successively to arguments $x_i^1 \ldots x_i^n$,
where $x_i^j$ is the $i$th element of \varj{list}, until a true value is
returned.  If \var{proc} returns \schfalse{} for all but the last
elements of the \var{list}s, \var{exists} performs a tail call of
\var{proc} on the $k$th elements, where $k$ is the length of
\vari{list}---in this case, the \var{list}s must all be lists of length $k$.
If \var{proc} returns a true value on any set of elements, {\cf
  exists} returns that value after the first such application of
\var{proc}.  If the \var{list}s
are all empty, {\cf exists} returns \schfalse.

\var{Proc} is always called in same dynamic environment 
as {\cf for-all} or, respectively, {\cf exists} itself.

\begin{scheme}
(for-all even? '(3 1 4 1 5 9)) \lev \schfalse{}
(for-all even? '(3 1 4 1 5 9 . 2)) \lev \schfalse{}
(for-all even? '(2 4 14)) \ev \schtrue{}
(for-all even? '(2 4 14 . 9)) \lev \exception{\cf\&assertion}
(for-all (lambda (n) (and (even? n) n)) '(2 4 14)) \lev 14
(for-all < '(1 2 3) '(2 3 4)) \ev \schtrue{}
(for-all < '(1 2 4) '(2 3 4)) \ev \schfalse{}

(exists even? '(3 1 4 1 5 9)) \lev \schtrue{}
(exists even? '(3 1 1 5 9)) \ev \schfalse{}
(exists even? '(3 1 1 5 9 . 2)) \lev \exception{\cf\&assertion}
(exists (lambda (n) (and (even? n) n)) '(2 1 4 14)) \lev 2
(exists < '(1 2 4) '(2 3 4)) \ev \schtrue{}
(exists > '(1 2 3) '(2 3 4)) \ev \schfalse{}
\end{scheme}

\implresp The implementation must check that the \var{list}s are
chains of pairs to the extent necessary to determine the return value.
If this requires traversing the lists entirely, the implementation
should check that the \var{list}s all have the same length.  If not,
It should not check that the \var{list}s are chains of pairs beyond
the traversal.  The implementation must check the restrictions on
\var{proc} to the extent performed by applying it as described.
\end{entry}

\begin{entry}{%
\proto{filter}{ proc list}{procedure}
\proto{partition}{ proc list}{procedure}
}

\domain{\var{Proc} must be a procedure that accepts a single argument
  and returns a single value. \var{Proc} must not mutate \var{list}.}
The {\cf filter} procedure successively applies
\var{proc} to the elements of \var{list} and returns a list of
the values of \var{list} for which \var{proc} returned a true
value.  The {\cf partition} procedure also successively applies \var{proc} to
the elements of \var{list}, but returns two values, the first one a
list of the values of \var{list} for which \var{proc} returned a
true value, and the second a list of the values of \var{list} for
which \var{proc} returned \schfalse.
\var{Proc} is always called in same dynamic environment 
as {\cf filter} or, respectively, {\cf partition} itself.

\begin{scheme}
(filter even? '(3 1 4 1 5 9 2 6)) \lev (4 2 6)

(partition even? '(3 1 4 1 5 9 2 6)) \lev (4 2 6) (3 1 1 5 9) ; two values
\end{scheme}

\implresp The implementation must check the restrictions on \var{proc}
to the extent performed by applying it as described.
\end{entry}

\begin{entry}{%
\proto{fold-left}{ combine nil \vari{list} \varii{list} \dotsfoo \varn{list}}{procedure}}

\domain{The \var{list}s must all have the same
  length.  \var{Combine} must be a
  procedure that takes one more
  argument than there are \var{list}s and returns a single value.
  \var{Combine} must not mutate the \var{list} arguments.}
The {\cf fold-left} procedure iterates the \var{combine} procedure over an
accumulator value and the values of the \var{list}s from left to
right, starting with an accumulator value of \var{nil}.  More
specifically, {\cf fold-left} returns \var{nil} if the \var{list}s are
empty.  If they are not empty, \var{combine} is first applied to
\var{nil} and the respective first elements of the \var{list}s in
order.  The result becomes the new accumulator value, and \var{combine}
is applied to new accumulator value and the respective next elements
of the \var{list}.  This step is repeated until the end of the list is
reached; then the accumulator value is returned.
\var{Combine} is always called in same dynamic environment 
as {\cf fold-left} itself.

\begin{scheme}
(fold-left + 0 '(1 2 3 4 5)) \ev 15

(fold-left (lambda (a e) (cons e a)) '()
           '(1 2 3 4 5)) \lev (5 4 3 2 1)

(fold-left (lambda (count x)
             (if (odd? x) (+ count 1) count))
           0
           '(3 1 4 1 5 9 2 6 5 3)) \lev 7

(fold-left (lambda (max-len s)
             (max max-len (string-length s)))
           0
           '("longest" "long" "longer")) \lev 7

(fold-left cons '(q) '(a b c)) \lev ((((q) . a) . b) . c)

(fold-left + 0 '(1 2 3) '(4 5 6)) \lev 21
\end{scheme}

\implresp The implementation should check that the \var{list}s all
have the same length.  The implementation must check the restrictions
on \var{proc} to the extent performed by applying it as described.
\end{entry}


\begin{entry}{%
\proto{fold-right}{ combine nil \vari{list} \varii{list} \dotsfoo \varn{list}}{procedure}}

\domain{The \var{list}s must all have the same
  length.  \var{Combine} must be a
  procedure that takes one more
  argument than there are \var{list}s and returns a single value.
  \var{Combine} must not mutate the \var{list} arguments.}
The {\cf fold-right} procedure iterates the \var{combine} procedure over
the values of the \var{list}s from right to left and an accumulator
value, starting with an accumulator value of \var{nil}.  More
specifically, {\cf fold-right} returns \var{nil} if the \var{list}s
are empty.  If they are not empty, \var{combine} is first applied to the
respective last elements of the \var{list}s in order and \var{nil}.
The result becomes the new accumulator value, and \var{combine} is
applied to the respective previous elements of the \var{list} and the
new accumulator value.  This step is repeated until the beginning of the
list is reached; then the accumulator value is returned.
\var{Proc} is always called in same dynamic environment 
as {\cf fold-right} itself.

\begin{scheme}
(fold-right + 0 '(1 2 3 4 5)) \ev 15

(fold-right cons '() '(1 2 3 4 5)) \lev (1 2 3 4 5)

(fold-right (lambda (x l)
              (if (odd? x) (cons x l) l))
            '()
            '(3 1 4 1 5 9 2 6 5))
\ev (3 1 1 5 9 5)

(fold-right cons '(q) '(a b c)) \lev (a b c q)

(fold-right + 0 '(1 2 3) '(4 5 6)) \lev 21
\end{scheme}

\implresp The implementation should check that the \var{list}s all
have the same length.  The implementation must check the restrictions
on \var{proc} to the extent performed by applying it as described.
\end{entry}

\begin{entry}{%
\proto{remp}{ proc list}{procedure}
\proto{remove}{ obj list}{procedure}
\proto{remv}{ obj list}{procedure}
\proto{remq}{ obj list}{procedure}}

\domain{\var{Proc} must be a procedure that takes a single argument
   and returns a single value.
   \var{Proc} must not mutate the \var{list} arguments.}
Each of these procedures returns a list of the elements of \var{list}
that do not satisfy a given condition.  The {\cf remp} procedure successively
applies \var{proc} to the elements of \var{list} and returns a
list of the values of \var{list} for which \var{proc} returned
\schfalse.  \var{Proc} is always called in same dynamic environment 
as {\cf remp} itself.
The {\cf remove}, {\cf remv}, and {\cf remq} procedures return a list of
the elements that are not \var{obj}.  The {\cf remq} procedure uses {\cf eq?}\ to
compare \var{obj} with the elements of \var{list}, while {\cf remv}
uses {\cf eqv?}\ and {\cf remove} uses {\cf equal?}.

\begin{scheme}
(remp even? '(3 1 4 1 5 9 2 6 5)) \lev (3 1 1 5 9 5)

(remove 1 '(3 1 4 1 5 9 2 6 5)) \lev (3 4 5 9 2 6 5)

(remv 1 '(3 1 4 1 5 9 2 6 5)) \lev (3 4 5 9 2 6 5)

(remq 'foo '(bar foo baz)) \ev (bar baz)
\end{scheme}

\implresp The implementation must check the restrictions on \var{proc}
to the extent performed by applying it as described.
\end{entry}

\begin{entry}{%
\proto{memp}{ proc list}{procedure}
\proto{member}{ obj list}{procedure}
\proto{memv}{ obj list}{procedure}
\proto{memq}{ obj list}{procedure}
}

\domain{\var{Proc} must be a procedure that takes a single argument
  and returns a single value.  \var{Proc} must not mutate \var{list}.}

These procedures return the first sublist of \var{list} whose car
satisfies a given condition, where the sublists of \var{lists} are the
lists returned by {\tt (list-tail \var{list} \var{k})} for
\var{k} less than the length of \var{list}.  The {\cf memp} procedure applies
\var{proc} to the cars of the sublists of \var{list} until it
finds one for which \var{proc} returns a true value without traversing
\var{list} further.  \var{Proc} is always called in same dynamic environment 
as {\cf memp} itself.  The {\cf
  member}, {\cf memv}, and {\cf memq} procedures look for the first occurrence of
\var{obj}.  If \var{list} does not contain an element satisfying the
condition, then \schfalse{} (not the empty list) is returned.  The {\cf
  member} procedure uses {\cf equal?}\ to compare \var{obj} with the elements of
\var{list}, while {\cf memv} uses {\cf eqv?}\ and {\cf memq} uses
{\cf eq?}.

\begin{scheme}
(memp even? '(3 1 4 1 5 9 2 6 5)) \lev (4 1 5 9 2 6 5)

(memq 'a '(a b c))              \ev  (a b c)
(memq 'b '(a b c))              \ev  (b c)
(memq 'a '(b c d))              \ev  \schfalse
(memq (list 'a) '(b (a) c))     \ev  \schfalse
(member (list 'a)
        '(b (a) c))             \ev  ((a) c)
(memq 101 '(100 101 102))       \ev  \unspecified
(memv 101 '(100 101 102))       \ev  (101 102)%
\end{scheme} 

\implresp The implementation must check that \var{list} is a chain of
pairs up to the found element, or that it is indeed a list if no
element is found.  It should not check that it is a chain of pairs
beyond the found element.  The implementation must check the restrictions on
\var{proc} to the extent performed by applying it as described.

\begin{rationale}
  Although they are ordinarily used as predicates, {\cf memp}, {\cf
    member}, {\cf memv}, {\cf memq}, do not have question marks in
  their names because they return useful values rather than just
  \schtrue{} or \schfalse{}.
\end{rationale}
\end{entry}

\begin{entry}{%
\proto{assp}{ proc alist}{procedure}
\proto{assoc}{ obj alist}{procedure}
\proto{assv}{ obj alist}{procedure}
\proto{assq}{ obj alist}{procedure}}

\domain{\var{Alist} (for ``association list'') must be lists of pairs.
  \var{Proc} must be a procedure that takes a single argument
  and returns a single value.  \var{Proc} must not mutate \var{list}.}

These procedures find the first pair in \var{alist}
whose car field satisfies a given condition, and returns that pair
without traversing \var{alist} further.
If no pair in \var{alist} satisfies the condition, then \schfalse{}
is returned.  The {\cf assp} procedure successively applies
\var{proc} to the car fields of \var{al} and looks for a pair
for which it returns a true value.
\var{Proc} is always called in same dynamic environment 
as {\cf assp} itself.  The {\cf assoc}, {\cf assv}, and {\cf
  assq} procedures look for a pair that has \var{obj} as its car.  The
{\cf assoc} procedure uses 
{\cf equal?}\ to compare \var{obj} with the car fields of the pairs in
\var{al}, while {\cf assv} uses {\cf eqv?}\ and {\cf assq} uses
{\cf eq?}.

\implresp The implementation must check that \var{alist} is a chain of
pairs containing pairs up to the found pair, or that it is indeed a
list of pairs if no element is found.  It should not check that it is
a chain of pairs beyond the found element.  The implementation must
check the restrictions on \var{proc} to the extent performed by
applying it as described.

\begin{scheme}
(define d '((3 a) (1 b) (4 c)))

(assp even? d) \ev (4 c)
(assp odd? d) \ev (3 a)

(define e '((a 1) (b 2) (c 3)))
(assq 'a e)     \ev  (a 1)
(assq 'b e)     \ev  (b 2)
(assq 'd e)     \ev  \schfalse
(assq (list 'a) '(((a)) ((b)) ((c))))
                \ev  \schfalse
(assoc (list 'a) '(((a)) ((b)) ((c))))   
                           \ev  ((a))
(assq 5 '((2 3) (5 7) (11 13)))    
                           \ev  \unspecified
(assv 5 '((2 3) (5 7) (11 13)))    
                           \ev  (5 7)%
\end{scheme}

\end{entry}

%%% Local Variables: 
%%% mode: latex
%%% TeX-master: "r6rs-lib"
%%% End: 


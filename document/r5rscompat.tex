\section{R$^5$RS compatibility}

The procedures described in this section are exported from the
\library{r6rs r5rs} library, and provide procedures described in the
previous revision of this report~\cite{R5RS}, but omitted from this
revision.

\begin{entry}{%
\proto{exact->inexact}{ z}{procedure}
\proto{inexact->exact}{ z}{procedure}}

These are the same as the {\cf ->inexact} and {\cf ->exact}
procedures, see section~\ref{->inexact}.
\end{entry}

\begin{entry}{%
\proto{quotient}{ \vari{n} \varii{n}}{procedure}
\proto{remainder}{ \vari{n} \varii{n}}{procedure}
\proto{modulo}{ \vari{n} \varii{n}}{procedure}}

These procedures implement number-theoretic (integer)
division.  \vrii{n} must be non-zero.  All three procedures
return integers.  If \vri{n}/\vrii{n} is an integer:
\begin{scheme}
    (quotient \vri{n} \vrii{n})   \ev \vri{n}/\vrii{n}
    (remainder \vri{n} \vrii{n})  \ev 0
    (modulo \vri{n} \vrii{n})     \ev 0
\end{scheme}
If \vri{n}/\vrii{n} is not an integer:
\begin{scheme}
    (quotient \vri{n} \vrii{n})   \ev \vr{n_q}
    (remainder \vri{n} \vrii{n})  \ev \vr{n_r}
    (modulo \vri{n} \vrii{n})     \ev \vr{n_m}
\end{scheme}
where \vr{n_q} is $\vri{n}/\vrii{n}$ rounded towards zero,
$0 < |\vr{n_r}| < |\vrii{n}|$, $0 < |\vr{n_m}| < |\vrii{n}|$,
\vr{n_r} and \vr{n_m} differ from \vri{n} by a multiple of \vrii{n},
\vr{n_r} has the same sign as \vri{n}, and
\vr{n_m} has the same sign as \vrii{n}.

Consequently, for integers \vri{n} and \vrii{n} with
\vrii{n} not equal to 0,
\begin{scheme}
     (= \vri{n} (+ (* \vrii{n} (quotient \vri{n} \vrii{n}))
           (remainder \vri{n} \vrii{n})))
                                 \ev  \schtrue%
\end{scheme}
provided all numbers involved in that computation are exact.

\begin{scheme}
(modulo 13 4)           \ev  1
(remainder 13 4)        \ev  1

(modulo -13 4)          \ev  3
(remainder -13 4)       \ev  -1

(modulo 13 -4)          \ev  -3
(remainder 13 -4)       \ev  1

(modulo -13 -4)         \ev  -1
(remainder -13 -4)      \ev  -1

(remainder -13 -4.0)    \ev  -1.0  ; inexact%
\end{scheme}

\begin{note}
  These procedures could be defined in terms of {\cf div} and {\cf
    mod} (see section~\ref{div}) as follows (without checking of the
  argument types):
\begin{scheme}
(define (sign n)
  (cond
    ((negative? n) -1)
    ((positive? n) 0)
    (else 0)))

(define (quotient  n1 n2)
  (* (sign n1) (sign n2) (div (abs n1) (abs n2))))

(define (remainder n1 n2)
  (* (sign n1) (mod (abs n1) (abs n2))))

(define (modulo n1 n2)
  (* (sign n2) (mod (* (sign n2) n1) (abs n2))))
\end{scheme}
\end{note}
\end{entry}

\begin{entry}{%
\proto{null-environment}{ n}{procedure}}

\domain{\var{N} must be the exact integer 5.}  This returns an
environment specifier suitable for use with {\cf eval} (see
section~\ref{eval}) representing an environment that is empty except
for the (syntactic) bindings for all syntactic keywords described in
the previous revision of this report~\cite{R5RS}.
\end{entry}

\begin{entry}{%
\proto{scheme-report-environment}{ n}{procedure}}

\domain{\var{N} must be the exact integer 5.}  This procedure returns
an environment specifier for an environment that is empty except for
the bindings for the standard procedures described in the previous
revision of this report~\cite{R5RS}, except for {\cf load}, {\cf
  transcript-on}, {\cf transcript-off}, and {\cf char-ready?}.  The
bindings have as values the procedures of the same names described in
this report.
\end{entry}


%%% Local Variables: 
%%% mode: latex
%%% TeX-master: "r6rs"
%%% End: 

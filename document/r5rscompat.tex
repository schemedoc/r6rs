\chapter{R$^5$RS compatibility}
\label{r5rscompatchapter}

The features described in this chapter are exported from the
\defrsixlibrary{r5rs} library and provide some functionality of the
preceding revision of this report~\cite{R5RS} that was omitted from
the main part of the current report.

\begin{entry}{%
\proto{exact->inexact}{ z}{procedure}
\proto{inexact->exact}{ z}{procedure}}

These are the same as the {\cf inexact} and {\cf exact}
procedures; see report section~\extref{report:inexact}{Generic conversions}.
\end{entry}

\begin{entry}{%
\proto{quotient}{ \vari{n} \varii{n}}{procedure}
\proto{remainder}{ \vari{n} \varii{n}}{procedure}
\proto{modulo}{ \vari{n} \varii{n}}{procedure}}

These procedures implement number-theoretic (integer)
division.  \varii{N} must be non-zero.  All three procedures
return integer objects.  If \vari{n}/\varii{n} is an integer object:
\begin{scheme}
    (quotient \vari{n} \varii{n})   \ev \vari{n}/\varii{n}
    (remainder \vari{n} \varii{n})  \ev 0
    (modulo \vari{n} \varii{n})     \ev 0
\end{scheme}
If \vari{n}/\varii{n} is not an integer object:
\begin{scheme}
    (quotient \vari{n} \varii{n})   \ev \var{n$_q$}
    (remainder \vari{n} \varii{n})  \ev \var{n$_r$}
    (modulo \vari{n} \varii{n})     \ev \var{n$_m$}
\end{scheme}
where \var{n$_q$} is $\vari{n}/\varii{n}$ rounded towards zero,
$0 < |\var{n$_r$}| < |\varii{n}|$, $0 < |\var{n$_m$}| < |\varii{n}|$,
\var{n$_r$} and \var{n$_m$} differ from \vari{n} by a multiple of \varii{n},
\var{n$_r$} has the same sign as \vari{n}, and
\var{n$_m$} has the same sign as \varii{n}.

Consequently, for integer objects \vari{n} and \varii{n} with
\varii{n} not equal to 0,
\begin{scheme}
     (= \vari{n} (+ (* \varii{n} (quotient \vari{n} \varii{n}))
           (remainder \vari{n} \varii{n})))
                                 \ev  \schtrue%
\end{scheme}
provided all number object involved in that computation are exact.

\begin{scheme}
(modulo 13 4)           \ev  1
(remainder 13 4)        \ev  1

(modulo -13 4)          \ev  3
(remainder -13 4)       \ev  -1

(modulo 13 -4)          \ev  -3
(remainder 13 -4)       \ev  1

(modulo -13 -4)         \ev  -1
(remainder -13 -4)      \ev  -1

(remainder -13 -4.0)    \ev  -1.0%
\end{scheme}

\begin{note}
  These procedures could be defined in terms of {\cf div} and {\cf
    mod} (see report section~\extref{report:div}{Arithmetic operations}) as follows (without checking of the
  argument types):
\begin{scheme}
(define (sign n)
  (cond
    ((negative? n) -1)
    ((positive? n) 1)
    (else 0)))

(define (quotient n1 n2)
  (* (sign n1) (sign n2) (div (abs n1) (abs n2))))

(define (remainder n1 n2)
  (* (sign n1) (mod (abs n1) (abs n2))))

(define (modulo n1 n2)
  (* (sign n2) (mod (* (sign n2) n1) (abs n2))))
\end{scheme}
\end{note}
\end{entry}

\begin{entry}{%
\proto{delay}{ \hyper{expression}}{\exprtype}}

The {\cf delay} construct is used together with the procedure \ide{force} to
implement \defining{lazy evaluation} or \defining{call by need}.
{\tt(delay~\hyper{expression})} returns an object called a
\defining{promise} which at some point in the future may be asked (by
the {\cf force} procedure) to evaluate
\hyper{expression}, and deliver the resulting value.
The effect of \hyper{expression} returning multiple values
is unspecified.

\end{entry}

\begin{entry}{%
\proto{force}{ promise}{procedure}}

{\var{Promise} must be a promise.}
The {\cf force} procedure forces the value of \var{promise}.  If no value has been computed for
the promise, then a value is computed and returned.  The value of the
promise is cached (or ``memoized'') so that if it is forced a second
time, the previously computed value is returned.

\begin{scheme}
(force (delay (+ 1 2)))   \ev  3
(let ((p (delay (+ 1 2))))
  (list (force p) (force p)))  
                               \ev  (3 3)

(define a-stream
  (letrec ((next
            (lambda (n)
              (cons n (delay (next (+ n 1)))))))
    (next 0)))
(define head car)
(define tail
  (lambda (stream) (force (cdr stream))))

(head (tail (tail a-stream)))  
                               \ev  2%
\end{scheme}

Promises are mainly intended for programs written in
functional style.  The following examples should not be considered to
illustrate good programming style, but they illustrate the property that
only one value is computed for a promise, no matter how many times it is
forced.

\begin{scheme}
(define count 0)
(define p
  (delay (begin (set! count (+ count 1))
                (if (> count x)
                    count
                    (force p)))))
(define x 5)
p                     \ev  {\it{}a promise}
(force p)             \ev  6
p                     \ev  {\it{}a promise, still}
(begin (set! x 10)
       (force p))     \ev  6%
\end{scheme}

Here is a possible implementation of {\cf delay} and {\cf force}.
Promises are implemented here as procedures of no arguments,
and {\cf force} simply calls its argument:

\begin{scheme}
(define force
  (lambda (object)
    (object)))%
\end{scheme}

The expression

\begin{scheme}
(delay \hyper{expression})%
\end{scheme}

has the same meaning as the procedure call

\begin{scheme}
(make-promise (lambda () \hyper{expression}))%
\end{scheme}

as follows

\begin{scheme}
(define-syntax delay
  (syntax-rules ()
    ((delay expression)
     (make-promise (lambda () expression))))),%
\end{scheme}

where {\cf make-promise} is defined as follows:

\begin{scheme}
(define make-promise
  (lambda (proc)
    (let ((result-ready? \schfalse)
          (result \schfalse))
      (lambda ()
        (if result-ready?
            result
            (let ((x (proc)))
              (if result-ready?
                  result
                  (begin (set! result-ready? \schtrue)
                         (set! result x)
                         result))))))))%
\end{scheme}
\end{entry}

\begin{entry}{%
\proto{null-environment}{ n}{procedure}}

\domain{\var{N} must be the exact integer object 5.}  The {\cf
  null-environment} procedure returns an
environment specifier suitable for use with {\cf eval} (see
chapter~\ref{evalchapter}) representing an environment that is empty except
for the (syntactic) bindings for all keywords described in
the previous revision of this report~\cite{R5RS}.
\end{entry}

\begin{entry}{%
\proto{scheme-report-environment}{ n}{procedure}}

\domain{\var{N} must be the exact integer object 5.}  The {\cf scheme-report-environment} procedure returns
an environment specifier for an environment that is empty except for
the bindings for the identifiers described in the previous
revision of this report~\cite{R5RS}, omitting {\cf load}, {\cf
  interaction-environment}, {\cf
  transcript-on}, {\cf transcript-off}, and {\cf char-ready?}.  The
bindings have as values the procedures of the same names described in
this report.
\end{entry}


%%% Local Variables: 
%%% mode: latex
%%% TeX-master: "r6rs-lib"
%%% End: 

\chapter{Hash tables}
\label{hashtablechapter}

The \deflibrary{r6rs hash-tables} library provides hash tables.
A \defining{hash table} is a data structure that associates keys with values.
Any object can be used as a key, provided a \defining{hash function}
and a suitable equivalence predicate is available.  A hash function is a
procedure that maps
keys to integers, and must be compatible with the equivalence predicate,
which is a procedure that accepts two keys and returns \schtrue{} if they
are equivalent, otherwise returns \schfalse{}.
Standard hash tables for arbitrary objects based on the {\cf eq?} and 
{\cf eqv?} predicates (see section~\ref{equivalencesection}) are provided.  
Also, standard hash functions for several types are provided.

This section uses the \var{hash-table} parameter name for arguments
that must be hash tables, and the \var{key} parameter name for
arguments that must be hash-table keys.

\section{Constructors}

\mainindex{hash table}

\begin{entry}{%
\proto{make-eq-hash-table}{}{procedure}
\rproto{make-eq-hash-table}{ \var{k}}{procedure}}

Returns a newly allocated mutable hash table that accepts
arbitrary objects as keys,
and compares those keys with {\cf eq?}. If an argument is given, the initial 
capacity of the hash table is set to approximately \var{k} elements.

\end{entry}

\begin{entry}{%
\proto{make-eqv-hash-table}{}{procedure}
\rproto{make-eqv-hash-table}{ \var{k}}{procedure}}

Returns a newly allocated mutable hash table that accepts
arbitrary objects as keys,
and compares those keys with {\cf eqv?}.
If an argument is given, the initial 
capacity of the hash table is set to approximately \var{k} elements.

\end{entry}

\begin{entry}{%
\proto{make-hash-table}{ \var{hash-function} \var{equiv?}}{procedure}
\rproto{make-hash-table}{ \var{hash-function} \var{equiv?} \var{k}}{procedure}}

\domain{\var{Hash-function} and \var{equiv?} must be procedures.
\var{Hash-function} will be called by other procedures described in
this chapter with a key as argument, and must return a 
non-negative exact integer.
\var{Equiv?} will be called by other procedures described in
this chapter with two keys as arguments, and must return a 
boolean.}
The {\cf make-hash-table} procedure returns a newly allocated mutable
hash table using \var{hash-function} 
as the hash function and \var{equiv?} as the equivalence predicate used to 
compare keys.
If a third argument is given, the 
initial capacity of the hash table is set to approximately \var{k} elements.

Both the hash function \var{hash-function} and the equivalence
predicate \var{equiv?} should behave like pure functions
on the domain of keys.  For example, the {\cf string-hash}
and {\cf string=?} procedures are permissible only if all
keys are strings and the contents of those strings are never
changed so long as any of them continue to serve as a key in
the hash table.  Furthermore any pair of values for which
the equivalence predicate \var{equiv?} returns true should
be hashed to the same exact integers by 
\var{hash-function}.

\begin{note}
Hash tables are allowed to cache the results of calling the
hash function and equivalence predicate, so programs cannot
rely on the hash function being called for every lookup or
update.  Furthermore any hash table operation may call the
hash function more than once.
\end{note}

\begin{rationale}
Hash table lookups are often followed by updates, so caching
may improve performance.  Hash tables are free to change
their internal representation at any time, which may result
in many calls to the hash function.
\end{rationale}

\end{entry}

\section{Procedures}

\begin{entry}{%
\proto{hash-table?}{ \var{hash-table}}{procedure}}

Returns \schtrue{} if \var{hash-table} is a hash table,
otherwise returns \schfalse.
\end{entry}

\begin{entry}{\proto{hash-table-size}{ \var{hash-table}}{procedure}}

Returns the number of keys contained in \var{hash-table} as an exact integer.
\end{entry}

\begin{entry}{%
\proto{hash-table-ref}{ \var{hash-table} \var{key} \var{default}}{procedure}}

Returns the value in \var{hash-table} associat
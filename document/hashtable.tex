\chapter{Hash tables}
\label{hashtablechapter}

The \deflibrary{r6rs hash-tables} library provides hash tables.
A \defining{hash table} is a data structure that associates keys with values.
Any object can be used as a key, provided a \defining{hash function}
and an equality-checking operation is available.  A hash function is a
procedure that maps
keys to integers, and must be compatible with the equality operation,
which is a procedure that accepts two keys and returns whether they
are equivalent.
Standard hash tables for arbitrary objects based on the {\cf eq?} and 
{\cf eqv?} equalities (see section~\ref{equivalencesection}) are provided.  
Also, standard hash functions for several types are provided.

This section uses the \var{hash-table} parameter name for arguments
that must be hash tables, and the \var{key} parameter name for
arguments that must be hash-table keys.

\section{Constructors}

\mainindex{hash table}

\begin{entry}{%
\proto{make-eq-hash-table}{}{procedure}
\rproto{make-eq-hash-table}{ \var{k}}{procedure}}

Returns a newly allocated mutable hash table that accepts
arbitrary objects as keys,
and compares those keys with {\cf eq?}. If an argument is given, the initial 
capacity of the hash table is set to approximately \var{k} elements.

\end{entry}

\begin{entry}{%
\proto{make-eqv-hash-table}{}{procedure}
\rproto{make-eqv-hash-table}{ \var{k}}{procedure}}

Returns a newly allocated mutable hash table that accepts
arbitrary objects as keys,
and compares those keys with {\cf eqv?}.
If an argument is given, the initial 
capacity of the hash table is set to approximately \var{k} elements.

\end{entry}

\begin{entry}{%
\proto{make-hash-table}{ \var{hash-function} \var{comparator}}{procedure}
\rproto{make-hash-table}{ \var{hash-function} \var{comparator} \var{k}}{procedure}}

\domain{\var{Hash-function} and \var{comparator} must be procedures.
\var{Hash-function} will be called by other procedures described in
this chapter with a key as argument, and must returns a 
non-negative exact integer.
\var{Comparator} will be called by other procedures described in
this chapter with two keys as arguments, and must return a 
boolean.}
The {\cf make-hash-table} procedure returns a newly allocated mutable
hash table using \var{hash-function} 
as the hash function and \var{comparator} as the procedure used to 
compare keys.
If a third argument is given, the 
initial capacity of the hash table is set to approximately \var{k} elements.

Both the hash function \vari{hash-function} and the comparison
predicate \var{comparator} should behave like pure functions
on the domain of keys.  For example, the {\cf string-hash}
and {\cf string=?} procedures are permissible only if all
keys are strings and the contents of those strings are never
changed so long as any of them continue to serve as a key in
the hash table.  Furthermore any pair of values for which
the comparison predicate \var{comparator} returns true should
be hashed to the same exact integers by 
\var{hash-function}.

\begin{note}
Hash tables are allowed to cache the results of calling the
hash function and comparison predicate, so programs cannot
rely on the hash function being called for every lookup or
update.  Furthermore any hash table operation may call the
hash function more than once.
\end{note}

\begin{rationale}
Hash table lookups are often followed by updates, so caching
may improve performance.  Hash tables are free to change
their internal representation at any time, which may result
in many calls to the hash function.
\end{rationale}

\end{entry}

\section{Procedures}

\begin{entry}{%
\proto{hash-table?}{ \var{hash-table}}{procedure}}

Returns \schtrue{} if \var{hash-table} is a hash table,
otherwise returns \schfalse.
\end{entry}

\begin{entry}{\proto{hash-table-size}{ \var{hash-table}}{procedure}}

Returns the number of keys contained in \var{hash-table} as an exact integer.
\end{entry}

\begin{entry}{%
\proto{hash-table-ref}{ \var{hash-table} \var{key} \var{default}}{procedure}}

Returns the value in \var{hash-table} associated with \var{key}.
If \var{hash-table} does not contain an association for \var{key},
then \var{default} is returned.
\end{entry}

\begin{entry}{\proto{hash-table-set!}{ \var{hash-table} \var{key} \var{obj}}{procedure}}

Changes \var{hash-table} to associate \var{key} with \var{obj},
adding a new association or replacing any existing association for \var{key}.
Returns the unspecified value.
\end{entry}

\begin{entry}{\proto{hash-table-delete!}{ \var{hash-table} \var{key}}{procedure}}

Removes any association for \var{key} within \var{hash-table}.
Returns the unspecified value.
\end{entry}

\begin{entry}{\proto{hash-table-contains?}{ \var{hash-table} \var{key}}{procedure}}

Returns \schtrue{} if \var{hash-table} contains an association
for \var{key}, otherwise returns \schfalse{}.
\end{entry}

\begin{entry}{%
\proto{hash-table-update!}{ \var{hash-table} \var{key} \var{proc} \var{default}}{procedure}}

\domain{\var{Proc} must be a procedure that takes a single argument.}
The {\cf hash-table-update!} procedure applies \var{proc} to the value in \var{hash-table}
associated with \var{key}, 
or to \var{default} if \var{hash-table} does not contain an
association for \var{key}.
The \var{hash-table} is then changed to associate \var{key}
with the result of \var{proc}.

The behavior of {\cf hash-table-update!} is equivalent to the
following code, but may be implemented 
more efficiently in cases where the implementation can
avoid looking up the same key twice:
\begin{scheme}
(hash-table-set!
 hash-table key
 (procedure (hash-table-ref
             hash-table key default)))
\end{scheme}
\end{entry}

\begin{entry}{\proto{hash-table-fold}{ \var{proc} \var{hash-table} \var{init}}{procedure}}

\domain{\var{Proc} must be a procedure that takes three arguments.}
For every association in \var{hash-table}, calls
\var{proc} with the association
key, the association value, and an accumulated value as arguments.
The accumulated value is \var{init} for the first
invocation of \var{proc}, and for subsequent
invocations of \var{proc}, it is the return value
of the previous invocation of \var{proc}. The order
of the calls to \var{proc} is indeterminate. The
return value of {\cf hash-table-fold} is the value of
the last invocation of \var{proc}. If any side
effect is performed on the hash table while a
{\cf hash-table-fold} operation is in progress, then the
behavior of {\cf hash-table-fold} is unspecified.

\end{entry}

\begin{entry}{%
\proto{hash-table-copy}{ \var{hash-table}}{procedure}
\rproto{hash-table-copy}{ \var{hash-table} \var{immutable}}{procedure}}

Returns a copy of \var{hash-table}.  If the
\var{immutable} argument is provided and is a true
value, the returned hash table is immutable;
otherwise it is mutable.

\begin{rationale}
Hash table references may be less expensive with immutable hash tables.
Also, a library may choose to export a hash table which
cannot be changed by clients.
\end{rationale}

\end{entry}
\begin{entry}{%
\proto{hash-table-clear!}{ \var{hash-table}}{procedure}
\rproto{hash-table-clear!}{ \var{hash-table} \var{k}}{procedure}}

Removes all associations from \var{hash-table}.  Returns the unspecified value.

If a second argument is given, the current
capacity of the hash table is reset to approximately \var{k} elements.
\end{entry}

\begin{entry}{\proto{hash-table-keys}{ \var{hash-table}}{procedure}}

Returns a list of all keys in \var{hash-table}.
The order of the list is unspecified.
Equivalent to:
\begin{scheme}
(hash-table-fold (lambda (k v a) (cons k a)) 
                 hash-table
                 '())
\end{scheme}
\end{entry}

\begin{entry}{\proto{hash-table-values}{ \var{hash-table}}{procedure}}

Returns a list of all values in \var{hash-table}.
The order of the list is unspecified.
Equivalent to:
\begin{scheme}
(hash-table-fold (lambda (k v a) (cons v a)) 
                 hash-table
                 '())
\end{scheme}
\end{entry}

\section{Inspection}

\begin{entry}{\proto{hash-table-equivalence-predicate}{ \var{hash-table}}{procedure}}

Returns the equivalence predicate used by
\var{hash-table} to compare keys.  For hash tables
created with {\cf make-eq-hash-table} and {\cf make-eqv-hash-table},
returns {\cf eq?} and {\cf eqv?} respectively.
\end{entry}

\begin{entry}{\proto{hash-table-hash-function}{ \var{hash-table}}{procedure}}

Returns the hash function used by \var{hash-table}.
For hash tables created by {\cf make-eq-hash-table} 
or {\cf make-eqv-hash-table}, \schfalse{} is returned.

\begin{rationale}
The {\cf make-eq-hash-table} and {\cf make-eqv-hash-table} constructors
are designed to hide their hash function.  This allows implementations
to use the machine address of an object as its hash value, rehashing
parts of the table as necessary whenever the garbage collector moves
objects to a different address.
\end{rationale}
\end{entry}

\begin{entry}{\proto{hash-table-mutable?}{ \var{hash-table}}{procedure}}

Returns \schtrue{} if \var{hash-table} is mutable, \schfalse{} otherwise.
\end{entry}

\section{Hash functions}

The {\cf equal-hash}, {\cf string-hash}, and {\cf string-ci-hash}
procedures of this section are acceptable as hash functions only
if the keys on which they are called do not suffer side effects
while the hash table remains in use.

\begin{entry}{\proto{equal-hash}{ \var{obj}}{procedure}}

Returns an integer hash value for \var{obj}, based on
its structure and current contents.
\end{entry}

\begin{entry}{\proto{string-hash}{ \var{string}}{procedure}}

Returns an integer hash value for \var{string}, based on
its current contents.
\end{entry}

\begin{entry}{\proto{string-ci-hash}{ \var{string}}{procedure}}

Returns an integer hash value for \var{string} based on
its current contents, ignoring case.
\end{entry}

\begin{entry}{\proto{symbol-hash}{ \var{symbol}}{procedure}}

Returns an integer hash value for \var{symbol}.
\end{entry}

%%% Local Variables: 
%%% mode: latex
%%% TeX-master: "r6rs"
%%% End: 

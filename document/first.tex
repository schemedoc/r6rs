% First page

\thispagestyle{empty}

% \todo{"another" report?}

\topnewpage[{
\begin{center}   {\huge\bf
        Revised{\Huge$^{\mathbf{5.91}}$} Report on the Algorithmic Language \\
                              \vskip 3pt
                                Scheme}

\vskip 1ex
$$
\begin{tabular}{l@{\extracolsep{.5in}}lll}
\multicolumn{4}{c}{M\authorsc{ICHAEL} S\authorsc{PERBER}}
\\
\multicolumn{4}{c}{W\authorsc{ILLIAM} C\authorsc{LINGER},
  R.\ K\authorsc{ENT} D\authorsc{YBVIG},
  M\authorsc{ATTHEW} F\authorsc{LATT},
  A\authorsc{NTON} \authorsc{VAN} S\authorsc{TRAATEN}}
\\
\multicolumn{4}{c}{
  R\authorsc{ICHARD} K\authorsc{ELSEY},
  J\authorsc{ONATHAN} R\authorsc{EES}
}
\\
\multicolumn{4}{c}{({\itshape Editors\/})} \\
H. A\authorsc{BELSON}     &
R. B. F\authorsc{INDLER}  &
C. T. H\authorsc{AYNES}   &
K. M. P\authorsc{ITMAN}   \\
N. I. A\authorsc{DAMS IV} &
D. P. F\authorsc{RIEDMAN} &
E. K\authorsc{OHLBECKER}  &
G. J. R\authorsc{OZAS}    \\
D. H. B\authorsc{ARTLEY}  &
R. H\authorsc{ALSTEAD}    &
J. M\authorsc{ATTHEWS}    &
G. L. S\authorsc{TEELE} J\authorsc{R}. \\
G. B\authorsc{ROOKS}         &
C. H\authorsc{ANSON}         &
D. O\authorsc{XLEY}          &
G. J. S\authorsc{USSMAN}  \\
\multicolumn{4}{c}{M. W\authorsc{AND}}\\[1ex]
\multicolumn{4}{c}{\bf 5 September 2006}
\end{tabular}
$$
\end{center}

\chapter*{Summary}
\medskip

The report gives a defining description of the programming language
Scheme.  Scheme is a statically scoped and properly tail-recursive
dialect of the Lisp programming language invented by Guy Lewis
Steele~Jr.\ and Gerald Jay~Sussman.  It was designed to have an
exceptionally clear and simple semantics and few different ways to
form expressions.  A wide variety of programming paradigms, including
imperative, functional, and message passing styles, find convenient
expression in Scheme.

The introduction offers a brief history of the language and of
the report.  It also gives a short introduction to the basic concepts
of the language.

Chapter~\ref{numbertypeschapter} explains Scheme's number types.
Chapter~\ref{readsyntaxchapter} defines the read syntax of Scheme
programs.  Chapter~\ref{basicchapter} presents the fundamental
semantic ideas of the language.  Chapter~\ref{terminologychapter}
defines notational conventions used in the rest of the report.
Chapters~\ref{librarychapter} and \ref{scriptchapter} describe
libraries and scripts, the basic organizational units of Scheme
programs.  Chapter~\ref{expansionchapter} explains the expansion
process for Scheme code.

Chapter~\ref{baselibrarychapter} explains the Scheme base library which
contains the fundamental forms useful to programmers.

The next set of chapters describe libraries that provide specific
functionality:
Unicode semantics for characters and strings,
binary data,
list utility procedures,
a record system,
exceptions and conditions,
I/O,
specialized libraries for dealing with numbers and arithmetic,
the {\cf syntax-case} facility for writing arbitrary macros,
hash tables,
enumerations,
and various miscellaneous libraries.

Chapter~\ref{complibchapter} describes the composite library
containing most of the forms described in this report.
Chapter~\ref{evalchapter} describes the {\cf eval} facility for
evaluating Scheme expressions represented as data.
Chapter~\ref{pairmutationchapter} describes the operations for
mutating pairs.

Appendix~\ref{formalsemanticschapter} provides a formal semantics for a
core of Scheme.  Appendix~\ref{derivedformsappendix} contains
definitions for some of the derived forms described in the report.

\vest The report concludes with a list of references and an
alphabetic index.

\bigskip

\begin{center}
{\large \bf
*** DRAFT*** \\
}\end{center}

This is a preliminary draft.  It is intended to reflect the decisions
taken by the editors' comittee, but contains many mistakes,
ambiguities and inconsistencies.

}]

\clearpage

\chapter*{Contents}
\addvspace{3.5pt}                  % don't shrink this gap
\renewcommand{\tocshrink}{-3.5pt}  % value determined experimentally
{\footnotesize
\tableofcontents
}

\vfill
\eject

%%% Local Variables: 
%%% mode: latex
%%% TeX-master: "r6rs"
%%% End: 

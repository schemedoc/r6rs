\chapter{{\tt syntax-case}}
\label{syntaxcasechapter}


%TODO:
%  - algebra for marks and substitutions
%    see Waddell's dissertation or POPL '99 module paper
%  - spell out the connection between syntax and lists somewhere
%  - spell out scoping rules, something like:
%    - the body of a library imported "for run" is at meta-level 0
%    - a transformer appearing in a meta-level n expression is at
%      meta-level n+1
%    - an introduced identifier (any identifier other than a pattern
%      variable) in a meta-level n+1 syntax template is at meta-level n
%    - meta-level n identifiers are bound only by meta-level n bindings
%    - in (let () (define-syntax a T) (define x e) ---), occurrences
%      of x outside of a template in T do not refer to the x
%      bound by (define x e), while occurrences of x inside of a template
%      in T do refer x bound by (define x e).
%  - work fresh-syntax into main SRFI body

%SRFI notes (see also Change Log section):
% - we need a test suite
%   (Per Bothner <per@bothner.com>)
% - would like a shorter convenience form for syntax-case definitions
%   (Per Bothner <per@bothner.com>)
% - consider (syntax-lambda (stx) ---) form or similar to abstract away from
%   single-argument procedure
%   (Jorgen Schaefer <forcer@forcix.cx>)
% - mixed feelings on identifier macros
%   (various in ``Various comments'' thread)
% - prefer that fresh-syntax be the default
%   (Andre van Tonder <andre@now.het.brown.edu>)
% - would like quasisyntax to be added if can't be defined as a
%   syntax-case macro
%   (Andre van Tonder <andre@now.het.brown.edu>)
% - please spell out in more detail how the expander deals with ellipses
%   in output templates.  See case example in +srfi93, message 54.
%   (Arthur A. Gleckler <arthur@zurich.csail.mit.edu>)
% - "Let me mention that I like the rules for unwrapping."
%   (Andre van Tonder <andre@now.het.brown.edu>)
% - would like Oscar's algebra updated for r6rs and included
%   (Andre van Tonder <andre@now.het.brown.edu>)

The \library{r6rs syntax-case} library
FIXME: need to state default phase
provides
support for writing low-level macros
in a high-level style, with automatic syntax checking, input
destructuring, output restructuring, maintenance of lexical scoping
and referential transparency (hygiene), and support for controlled
identifier capture, with constant expansion overhead.  Because it does
not require literals, including quoted lists or vectors, to be copied
or even traversed, it preserves sharing and cycles within and among
the constants of a program.  It also supports source-object
correlation, i.e., the maintenance of ties between the original source
code and expanded output, allowing implementations to provide
source-level support for debuggers and other tools.

\begin{rationale}
While many syntactic abstractions are succinctly expressed using the
high-level {\cf syntax-rules} form, others are difficult or impossible
to write, including some that introduce visible bindings for or references
to identifiers that do not appear explicitly in the input form, ones that
maintain state or read from the file system, and ones that construct new
identifiers.
The {\cf syntax-case} system~\cite{syntacticabstraction} described here
allows the programmer to write transformers that perform these sorts of
transformations, and arbitrary additional transformations, without
sacrificing the default enforcement of hygiene or the high-level
pattern-based syntax matching and template-based output construction
provided by {\r6rs} {\cf syntax-rules}.
\end{rationale}

A syntactic abstraction typically takes the form
{\cf (\hyper{keyword} \hyper{subform} \dots)}, where \hyper{keyword} is the
identifier that names the syntactic abstraction.
The syntax of each \hyper{subform} varies from one syntactic abstraction to
another.
Syntactic abstractions can also take the form of improper lists (or even
singleton identifiers; see section~\ref{transformerssection}), although this is
less common.

New syntactic abstractions are defined by associating keywords with
\emph{transformers}.
Keyword bindings are created using {\cf define-syntax},
{\cf let-syntax}, or {\cf letrec-syntax}.
Transformers are created using {\cf syntax-rules} or
{\cf syntax-case} and {\cf syntax}, which allow transformations to
be specified via pattern matching and template reconstruction.

\section{Hygiene}
\label{hygienesection}

% hygiene condition for macro expansion
% (Kohlbecker, E.E., Friedman, D.P., Felleisen, M., Duba, B. 'Hygienic macro expansion' (1986))
% "Generated identifiers that become binding instances in the completely
% expanded program must only bind variables that are generated at the same
% transcription step."

Barendregt's \emph{hygiene condition}~\cite{barendregt} for the
lambda-calculus is an informal notion that requires the free variables of
an expression $N$ that is to be substituted into another expression $M$ not to
be captured by bindings in $M$ when such capture is not intended.
Kohlbecker, et al~\cite{hygienic} propose a corresponding
\emph{hygiene condition for macro expansion} that applies in all situations
where capturing is not explicit:
``Generated identifiers that become binding instances in
the completely expanded program must only bind variables that
are generated at the same transcription step.''
In the terminology of this document, the ``generated identifiers'' are
those introduced by a transformer rather than those present in the form
passed to the transformer, and a ``macro transcription step'' corresponds
to a single call by the expander to a transformer.
Also, the hygiene condition applies to all introduced bindings rather than
to introduced variable bindings alone.

This leaves open what happens to an introduced identifier that appears
outside the scope of a binding introduced by the same call.
Such an identifier refers to the lexical binding in effect where it
appears (within a {\cf syntax} \hyper{template};
see section~\ref{syntaxcasesection}) inside the transformer body or one of
the helpers it calls.
This is essentially the referential transparency property described
by Clinger and Rees~\cite{macrosthatwork}.

Thus, the hygiene condition can be restated as follows:

\begin{quotation}
\noindent
A binding for an identifier introduced into the output of a transformer
call from the expander must capture only references to the identifier
introduced into the output of the same transformer call.
A reference to an identifier introduced into the output of a transformer
refers to the closest enclosing binding for the introduced identifier or,
if it appears outside of any enclosing binding for the introduced
identifier, the closest enclosing lexical binding where the identifier
appears (within a {\cf syntax} \hyper{template})
inside the transformer body or one of the helpers it calls.
\end{quotation}

Explicit captures are handled via {\cf datum\coerce{}syntax}; see
section~\ref{conversionssection}.

Operationally, the expander can maintain hygiene with the help of
\emph{marks\mainindex{mark}} and \emph{substitutions\mainindex/substitution}.
Marks are applied selectively by the expander to the output of each
transformer it invokes, and substitutions are applied to the portions
of each binding form that are supposed to be within the scope of the bound
identifiers.
Marks are used to distinguish like-named identifiers that are
introduced at different times (either present in the source or introduced
into the output of a particular transformer call), and substitutions are
used to map identifiers to their expand-time values.

Each time the expander encounters a syntactic abstraction, it applies an
\defining{antimark} to the input form, invokes the associated transformer,
then applies a fresh mark to the output.
Marks and antimarks cancel, so the portions of the input that appear in
the output are effectively left unmarked, while the portions of the output
that are introduced are marked with the fresh mark.

Each time the expander encounters a binding form it creates a set of
substitutions, each mapping one of the (possibly marked) bound identifiers
to information about the binding.
(For a {\cf lambda} expression, the expander might map each bound
identifier to a representation of the formal parameter in the output of
the expander.
For a {\cf let-syntax} form, the expander might map each bound
identifier to the associated transformer.)
These substitutions are applied to the portions of the input form in
which the binding is supposed to be visible.

Marks and substitutions together form a \defining{wrap} that is layered on the
form being processed by the expander and pushed down toward the leaves as
necessary.
A wrapped form is referred to as a \defining{wrapped syntax object}.
FIXME: syntax object hasn't been defined
Ultimately, the wrap may rest on a leaf that represents an identifier, in
which case the wrapped syntax object is referred to more precisely
as an \emph{identifier}.
An identifier contains a name along with the wrap.
(Names are typically represented by symbols.)

When a substitution is created to map an identifier to an expand-time
value, the substitution records the name of the identifier and
the set of marks that have been applied to that identifier, along
with the associated expand-time value.
The expander resolves identifier references by looking for the latest
matching substitution to be applied to the identifier, i.e., the outermost
substitution in the wrap whose name and marks match the name and
marks recorded in the substitution.
The name matches if it is the same name (if using symbols, then by
{\cf eq?}), and the marks match if the marks recorded with the
substitution are the same as those that appear \emph{below} the
substitution in the wrap, i.e., those that were applied \emph{before} the
substitution.
Marks applied after a substitution, i.e., appear over the substitution in
the wrap, are not relevant and are ignored.

FIXME
An algebra that defines how marks and substitutions work more precisely is
given in section~2.4 of Oscar Waddell's PhD thesis~\cite{Waddellphd}.

\section{Transformers}
\label{transformerssection}

In {\cf define-syntax} (section~\ref{define-syntax}), {\cf
  let-syntax}, and {\cf letrec-syntax} forms
(section~\ref{let-syntax}), a binding for a syntactic keyword must be
a \hyper{transformer spec}.  A \hyper{transformer spec} can be an
expression that evaluates to a transformer.

A transformer is a \defining{transformation procedure} or a
\defining{variable transformer}.
A transformation procedure is a procedure that must accept one
argument, a wrapped syntax object (section~\ref{syntaxobjectssection})
representing the input, and return a \defining{syntax object}
(section~\ref{syntaxobjectssection}) representing the output.
The procedure is called by the expander whenever a reference to
a keyword with which it has been associated is found.
If the keyword appears in the first position of a list-structured
input form, the transformer receives the entire list-structured
form, and its output replaces the entire form.
If the keyword is found in any other definition or expression context, the
transformer receives a wrapped syntax object representing just the keyword
reference, and its output replaces just the reference.
An exception with condition type {\cf\&syntax} is raised if the keyword appears on the
left-hand side of a {\cf set!} expression.

Variable transformers are similar.
If a keyword associated with a variable transformer appears on
the left-hand side of a {\cf set!} expression, however, no 
exception is raised.
Instead, the transformer receives a wrapped syntax object representing the
entire {\cf set!} expression as its argument, and its output
replaces the entire {\cf set!} expression.
A variable transformer is created by passing a transformation
procedure to {\cf make-variable-transformer}

\begin{entry}{%
\proto{make-variable-transformer}{ proc}{procedure}}

\domain{\var{Proc} must be a procedure  FIXME  What should it do?.}

{\cf make-variable-transformer} returns a transformer procedure that
allows the expander to recognize that it is a variable transformer.
\end{entry}

\section{Syntax objects}
\label{syntaxobjectssection}

A syntax object is a representation of a Scheme form that contains
contextual information about the form in addition to its structure.
This contextual information is used by the expander to maintain
lexical scoping and may also be used by an implementation to maintain
source-object correlation.

Syntax objects may be wrapped or unwrapped.
A wrapped syntax object (section~\ref{hygienesection}), consists of a
\textit{wrap} (section~\ref{hygienesection}) and some internal representation
of a Scheme form.
(The internal representation is unspecified, but is typically a 
datum value or datum value annotated with source information.)
A wrapped syntax object representing an identifier is itself referred to as
an identifier; thus, the term \textit{identifier\mainindex{identifier}} may refer either to
the syntactic entity (symbol, variable, or keyword) or to the
concrete representation of the syntactic entity as a syntax object.
Wrapped syntax objects are distinct from other types of values.

An unwrapped syntax object is one that is unwrapped, fully or partially,
i.e., whose outer layers consist of lists and vectors and whose leaves are
either wrapped syntax objects or nonsymbol values.

The term \defining{syntax object} is used in this document to refer to
a syntax object that is either wrapped or unwrapped.
More formally, a syntax object is:

\begin{itemize}
\item a pair or list of syntax objects,
\item a vector of syntax objects,
\item a nonlist, nonvector, nonsymbol value, or
\item a wrapped syntax object.
\end{itemize}

The distinction between the terms ``syntax object'' and ``wrapped syntax
object'' is important.
For example, when invoked by the expander, a transformer
(section~\ref{transformerssection}) must accept a wrapped syntax object but
may return any syntax object, including an unwrapped syntax object.

\section{Parsing input and producing output}
\label{syntaxcasesection}

Transformers destructure their input with {\cf syntax-case} and rebuild
their output with {\cf syntax}.

\begin{entry}{%
\pproto{(syntax-case \hyper{expression} (\hyper{literal} \dots) \hyper{clause} \dots)}{\exprtype}}
\mainschindex{syntax-case}

\syntax Each \hyper{literal} must be an identifier.
Each \hyper{clause} must take one of the following two forms.

\begin{scheme}
(\hyper{pattern} \hyper{output expression})
(\hyper{pattern} \hyper{fender} \hyper{output expression})
\end{scheme}

\hyper{Fender} must and \hyper{output expression} must be
\hyper{expression}s.

A \hyper{pattern} is an identifier, constant, or one of the following.

\begin{schemenoindent}
(\hyper{pattern} \ldots)
(\hyper{pattern} \hyper{pattern} \ldots . \hyper{pattern})
(\hyper{pattern} \ldots \hyper{pattern} \hyper{ellipsis} \hyper{pattern} \ldots)
(\hyper{pattern} \ldots \hyper{pattern} \hyper{ellipsis} \hyper{pattern} \ldots . \hyper{pattern})
\#(\hyper{pattern} \ldots)
\#(\hyper{pattern} \ldots \hyper{pattern} \hyper{ellipsis} \hyper{pattern} \ldots)%
\end{schemenoindent}

An \hyper{ellipsis} is the identifier ``{\cf ...}'' (which cannot be used as
an identifier in either a template or a pattern).\schindex{...}

An identifier appearing within a \hyper{pattern} may be an underscore
(~{\cf \_}~), a literal identifier listed in the list of literals
{\cf (\hyper{literal} \dots)}, or an ellipsis (~{\cf ...}~).
All other identifiers appearing within a \hyper{pattern} are
\textit{pattern variables\mainindex{pattern variable}}.
An ellipsis or underscore may not appear in {\cf (\hyper{literal} \dots)}.

Pattern variables match arbitrary input subforms and
are used to refer to elements of the input.
The same pattern variable may not appear more than once in a
\hyper{pattern}.

Underscores also match arbitrary input subforms but are not pattern variables
and so cannot be used to refer to those elements.
Multiple underscores may appear in a \hyper{pattern}.

A literal identifier matches an input subform if and only if the input
subform is an identifier and either both its occurrence in the input
expression and its occurrence in the list of literals have the same
lexical binding, or the two identifiers have the same name and both have
no lexical binding.

A subpattern followed by an ellipsis can match zero or more elements of
the input.

More formally, an input form $F$ matches a pattern $P$ if and only if

\begin{itemize}
\item $P$ is an underscore (~{\cf \_}~),

\item $P$ is a pattern variable,

\item $P$ is a literal identifier
and $F$ is an equivalent identifier in the
sense of {\cf free-identifier=?}
(section~\ref{identifierpredicatessection}),

\item $P$ is of the form
{\cf ($P_1$ \dots $P_n$)}
and $F$ is a list of $n$ elements that match $P_1$ through
$P_n$,

\item $P$ is of the form
{\cf ($P_1$ \dots $P_n$ . $P_x$)}
and $F$ is a list or improper list of $n$ or more elements
whose first $n$ elements match $P_1$ through $P_n$
and
whose $n$th cdr matches $P_x$,

\item $P$ is of the form
{\cf ($P_1$ \dots $P_k$ $P_e$ \hyper{ellipsis} $P_{m+1}$ \dots $P_n$)},
where \hyper{ellipsis} is the identifier {\cf ...}
and $F$ is a proper list of $n$
elements whose first $k$ elements match $P_1$ through $P_k$,
whose next $m-k$ elements each match $P_e$,
and
whose remaining $n-m$ elements match $P_{m+1}$ through $P_n$,

\item $P$ is of the form
{\cf ($P_1$ \dots $P_k$ $P_e$ \hyper{ellipsis} $P_{m+1}$ \dots $P_n$ . $P_x$)},
where \hyper{ellipsis} is the identifier {\cf ...}
and $F$ is a list or improper list of $n$
elements whose first $k$ elements match $P_1$ through $P_k$,
whose next $m-k$ elements each match $P_e$,
whose next $n-m$ elements match $P_{m+1}$ through $P_n$,
and 
whose $n$th and final cdr matches $P_x$,

\item $P$ is of the form
{\cf \#($P_1$ \dots $P_n$)}
and $F$ is a vector of $n$ elements that match $P_1$ through
$P_n$,

\item $P$ is of the form
{\cf \#($P_1$ \dots $P_k$ $P_e$ \hyper{ellipsis} $P_{m+1}$ \dots $P_n$)},
where \hyper{ellipsis} is the identifier {\cf ...}
and $F$ is a vector of $n$ or more elements
whose first $k$ elements match $P_1$ through $P_k$,
whose next $m-k$ elements each match $P_e$,
and
whose remaining $n-m$ elements match $P_{m+1}$ through $P_n$,
or

\item $P$ is a pattern datum (any nonlist, nonvector, nonsymbol
object) and $F$ is equal to $P$ in the sense of the
{\cf equal?} procedure.
\end{itemize}

\semantics
{\cf syntax-case} first evaluates \hyper{expression}.
It then attempts to match
the \hyper{pattern} from the first \hyper{clause} against the resulting value,
which is unwrapped as necessary to perform the match.
If the pattern matches the value and no
\hyper{fender} is present,
\hyper{output expression} is evaluated and its value returned as the
value of the {\cf syntax-case} expression.
If the pattern does not match the value, {\cf syntax-case} tries
the second \hyper{clause}, then the third, and so on.
An exception with condition type {\cf\&syntax} is raised if the value does not match any of the patterns.

If the optional \hyper{fender} is present, it serves as an additional
constraint on acceptance of a clause.
If the \hyper{pattern} of a given \hyper{clause} matches the input value,
the corresponding \hyper{fender} is evaluated.
If \hyper{fender} evaluates to a true value, the clause is accepted;
otherwise, the clause is rejected as if the pattern had failed to match
the value.
Fenders are logically a part of the matching process, i.e., they
specify additional matching constraints beyond the basic structure of
the input.

Pattern variables contained within a clause's
\hyper{pattern} are bound to the corresponding pieces of the input
value within the clause's \hyper{fender} (if present) and
\hyper{output expression}.
Pattern variables can be referenced only within {\cf syntax}
expressions (see below).
Pattern variables occupy the same name space as program variables and
keywords.
\end{entry}

\begin{entry}{%
\proto{syntax}{ \hyper{template}}{\exprtype}}

\begin{note}
{\cf \#'\hyper{template}} is equivalent to {\cf (syntax
  \hyper{template})}.
\end{note}

A {\cf syntax} expression is similar to a {\cf quote} expression
except that (1) the values of pattern variables appearing within
\hyper{template} are inserted into \hyper{template}, (2) contextual
information associated both with the input and with the template is
retained in the output to support lexical scoping, and (3) the value
of a {\cf syntax} expression is a syntax object.

A \hyper{template} is a pattern variable, an identifier that
is not a pattern
variable, a pattern datum, or one of the following.

\begin{scheme}
(\hyper{subtemplate} \ldots)
(\hyper{subtemplate} \ldots . \hyper{template})
\#(\hyper{subtemplate} \ldots)
\end{scheme}

A \hyper{subtemplate} is a \hyper{template} followed by zero or more ellipses.

The value of a \hyper{syntax} form is a copy of \hyper{template} in which
the pattern variables appearing within the template are replaced with
the input subforms to which they are bound.
Pattern data and identifiers that are not pattern variables
are copied directly into the output.
A subtemplate followed by an ellipsis expands
into zero or more occurrences of the subtemplate.
The subtemplate
must contain at least one pattern variable from a subpattern
followed by an ellipsis.
(Otherwise, the expander would not be able to determine how many times the
subform should be repeated in the output.)
Pattern variables that occur in subpatterns followed by one or more
ellipses may occur only in subtemplates that are
followed by (at least) as many ellipses.
These pattern variables are replaced in the output by the input
subforms to which they are bound, distributed as specified.
If a pattern variable is followed by more ellipses in the template
than in the associated pattern, the input form is replicated as
necessary.

A template of the form
{\cf (\hyper{ellipsis} \hyper{template})} is identical to \hyper{template}, except that
ellipses within the template have no special meaning.
That is, any ellipses contained within \hyper{template} are
treated as ordinary identifiers.
In particular, the template {\cf (... ...)} produces a single
ellipsis, {\cf ...}.
This allows syntactic abstractions to expand into forms containing
ellipses.

\label{wrappingrules}
The output produced by {\cf syntax} is wrapped or unwrapped according to
the following rules.

\begin{itemize}
\item the copy of {\cf (\hyperi{t} .  \hyperii{t})} is a pair if \hyperi{t}
      or \hyperii{t} contain any pattern variables,
\item the copy of {\cf (\hyper{t} \hyper{ellipsis})} is a list if \hyper{t}
      contains any pattern variables,
\item the copy of {\cf \#(\hyperi{t} ... \hypern{t})} is a vector if any of
      \hyperi{t},~\dots,~\hypern{t} contain any pattern variables, and
\item the copy of any portion of \hyper{t} not containing any pattern variables
      is a wrapped syntax object.
\end{itemize}

The input subforms inserted in place of the pattern variables are wrapped
if and only if the corresponding input subforms are wrapped.
\end{entry}

The following definitions of {\cf or} illustrate {\cf syntax-case}
and {\cf syntax}.
The second is equivalent to the first but uses the the {\cf \#'}
prefix instead of the full {\cf syntax} form.

\begin{schemenoindent}
(define-syntax or
  (lambda (x)
    (syntax-case x ()
      [(\_) (syntax \schfalse{})]
      [(\_ e) (syntax e)]
      [(\_ e1 e2 e3 ...)
       (syntax (let ([t e1])
                 (if t t (or e2 e3 ...))))])))

(define-syntax or
  (lambda (x)
    (syntax-case x ()
      [(\_) \#'\schfalse{}]
      [(\_ e) \#'e]
      [(\_ e1 e2 e3 ...)
       \#'(let ([t e1])
           (if t t (or e2 e3 ...)))])))

(define-syntax case
  (lambda (x)
    (syntax-case x (else)
      [(\_ e0 [(k ...) e1 e2 ...] ...
              [else else-e1 else-e2 ...])
       \#'(let ([t e0])
           (cond
             [(memv t '(k ...)) e1 e2 ...]
             ...
             [else else-e1 else-e2 ...]))]
      [(\_ e0 [(ka ...) e1a e2a ...]
              [(kb ...) e1b e2b ...] ...)
       \#'(let ([t e0])
           (cond
             [(memv t '(ka ...)) e1a e2a ...]
             [(memv t '(kb ...)) e1b e2b ...]
             ...))])))
\end{schemenoindent}

The examples below define \emph{identifier macros\mainindex{identifier
  macro}}, syntactic abstractions
supporting keyword references that do not necessarily appear in the first
position of a list-structured form.
The second example uses {\cf make-variable-transformer} to handle the case
where the keyword appears on the left-hand side of a
{\cf set!} expression.

\begin{scheme}
(define p (cons 4 5))
(define-syntax p.car
  (lambda (x)
    (syntax-case x ()
      [(\_ . rest) \#'((car p) . rest)]
      [\_  \#'(car p)])))
p.car \ev 4
(set! p.car 15) \ev \exception{\&syntax}

(define p (cons 4 5))
(define-syntax p.car
  (make-variable-transformer
    (lambda (x)
      (syntax-case x (set!)
        [(set! \_ e) \#'(set-car! p e)]
        [(\_ . rest) \#'((car p) . rest)]
        [\_  \#'(car p)]))))
(set! p.car 15)
p.car           \ev 15
p               \ev (15 5)
\end{scheme}

A derived {\cf identifier-syntax} form that simplifies the definition
of identifier macros is described in section~\ref{derivedsection}.

\section{Identifier predicates}
\label{identifierpredicatessection}

\begin{entry}{%
\proto{identifier?}{ obj}{procedure}}

Returns \schtrue{} if \var{obj} is an identifier, i.e., a
syntax object representing an identifier, and \schfalse{} otherwise.

The {\cf identifier?} procedure is often used within a fender to verify
that certain subforms of an input form are identifiers, as in the
definition of {\cf rec}, which creates self-contained
recursive objects, below.

\begin{scheme}
(define-syntax rec
  (lambda (x)
    (syntax-case x ()
      [(\_ x e)
       (identifier? \#'x)
       \#'(letrec ([x e]) x)])))

(map (rec fact
       (lambda (n)
         (if (= n 0)                 
             1
             (* n (fact (- n 1))))))
     '(1 2 3 4 5)) \lev (1 2 6 24 120)
 
(rec 5 (lambda (x) x)) \ev \exception{\&syntax}
\end{scheme}
\end{entry}

The procedures {\cf bound-identifier=?} and {\cf free-identifier=?}
each take two identifier arguments and return \schtrue{} if their
arguments are equivalent and \schfalse{} otherwise.
These predicates are used to compare identifiers according to their
\emph{intended use} as free references or bound identifiers in a given
context.

\begin{entry}{%
\proto{bound-identifier=?}{ \vari{id} \varii{id}}{procedure}}

\domain{\vari{Id} and \varii{id} must be identifiers.}
The procedure {\cf bound-identifier=?} returns true if and only if a
binding for one would capture a reference to the other in the output of
the transformer, assuming that the reference appears within the scope of
the binding.
In general, two identifiers are {\cf bound-identifier=?} only if
both are present in the original program or both are introduced by the
same transformer application
(perhaps implicitly---see {\cf datum\coerce{}syntax}).
Operationally, two identifiers are
considered equivalent by {\cf bound-identifier=?} if and only if they
have the same name and same marks (section~\ref{hygienesection}).

{\cf bound-identifier=?} can be used for detecting
duplicate identifiers in a binding construct or for other
preprocessing of a binding construct that requires detecting instances
of the bound identifiers.
\end{entry}

\begin{entry}{%
\proto{free-identifier=?}{ \vari{id} \varii{id}}{procedure}}

\domain{\vari{Id} and \varii{id} must be identifiers.}
The procedure {\cf free-identifier=?} returns \schtrue{} if and
only if the two identifiers would resolve to the same binding if both were
to appear in the output of a transformer outside of any bindings inserted
by the transformer.
(If neither of two like-named identifiers resolves to a binding, i.e., both
are unbound, they are considered to resolve to the same binding.)
Operationally, two identifiers are considered equivalent by
{\cf free-identifier=?} if and only the topmost matching
substitution for each maps to the same binding (section~\ref{hygienesection})
or the identifiers have the same name and no matching substitution.

{\cf syntax-case} and {\cf syntax-rules} use
{\cf free-identifier=?} to compare identifiers listed in the literals
list against input identifiers.

The following definition of unnamed {\cf let}
uses {\cf bound-identifier=?} to detect duplicate identifiers.

\begin{schemenoindent}
(define-syntax let
  (lambda (x)
    (define unique-ids?
      (lambda (ls)
        (or (null? ls)
            (and (let notmem?
                        ([x (car ls)] [ls (cdr ls)])
                   (or (null? ls)
                       (and (not (bound-identifier=?
                                   x (car ls)))
                            (notmem? x (cdr ls)))))
                 (unique-ids? (cdr ls))))))
    (syntax-case x ()
      [(\_ ((i v) ...) e1 e2 ...)
       (unique-ids? \#'(i ...))
       \#'((lambda (i ...) e1 e2 ...) v ...)])))
\end{schemenoindent}

The argument {\cf \#'(i ...)} to {\cf unique-ids?} is guaranteed
to be a list by the rules given in the description of {\cf syntax}
above.

With this definition of {\cf let}:

\begin{scheme}
(let ([a 3] [a 4]) (+ a a)) \lev \exception{\&syntax-violation}
\end{scheme}

However,

\begin{scheme}
(let-syntax
  ([dolet (lambda (x)
            (syntax-case x ()
              [(\_ b)
               \#'(let ([a 3] [b 4]) (+ a b))]))])
  (dolet a)) \lev 7
\end{scheme}

since the identifier {\cf a} introduced by {\cf dolet}
and the identifier {\cf a} extracted from the input form are not
{\cf bound-identifier=?}.

The following definition of {\cf case} is equivalent to the one in
section~\ref{syntaxcasesection}.
Rather than including {\cf else} in the literals list as before,
this version explicitly tests for {\cf else} using
{\cf free-identifier=?}.

\begin{schemenoindent}
(define-syntax case
  (lambda (x)
    (syntax-case x ()
      [(\_ e0 [(k ...) e1 e2 ...] ...
              [else-key else-e1 else-e2 ...])
       (and (identifier? \#'else-key)
            (free-identifier=? \#'else-key \#'else))
       \#'(let ([t e0])
           (cond
             [(memv t '(k ...)) e1 e2 ...]
             ...
             [else else-e1 else-e2 ...]))]
      [(\_ e0 [(ka ...) e1a e2a ...]
              [(kb ...) e1b e2b ...] ...)
       \#'(let ([t e0])
           (cond
             [(memv t '(ka ...)) e1a e2a ...]
             [(memv t '(kb ...)) e1b e2b ...]
             ...))])))
\end{schemenoindent}

With either definition of {\cf case}, {\cf else} is not
recognized as an auxiliary
keyword if an enclosing lexical binding for {\cf else} exists.
For example,

\begin{scheme}
(let ([else \schfalse{}])
  (case 0 [else (write "oops")])) \lev \exception{\&syntax-violation}
\end{scheme}

since {\cf else} is bound
lexically and is
therefore not the same {\cf else} that appears in the definition of
{\cf case}.
\end{entry}

\section{Syntax-object and datum conversions}
\label{conversionssection}

\begin{entry}{%
\proto{syntax->datum}{ syntax-object}{procedure}}

The procedure {\cf syntax\coerce{}datum}
strips all syntactic information from a syntax
object and returns the corresponding Scheme datum.
\end{entry}

Identifiers stripped in this manner are converted to their symbolic
names, which can then be compared with {\cf eq?}.
Thus, a predicate {\cf symbolic-identifier=?} might be defined as follows.

\begin{scheme}
(define symbolic-identifier=?
  (lambda (x y)
    (eq? (syntax->datum x)
         (syntax->datum y))))
\end{scheme}

% not be true with import alias and rename
%Two identifiers that are {\cf bound-identifier=?} or
%{\cf free-identifier=?} are {\cf symbolic-identifier=?}; in order to
%refer to the same binding, two identifiers must have the same name.
%The converse is not always true, since two identifiers may have
%the same name but different bindings.

\begin{entry}{%
\proto{datum->syntax}{ template-id datum}{procedure}}
\end{entry}

\domain{\var{Template-id} must be a
template identifier and \var{datum} must be a datum value.}
This returns a syntax object representation of \var{datum} that
contains the same contextual information as
\var{template-id}, with the effect that the
syntax object behaves
as if it were introduced into the code when
\var{template-id} was introduced.

The {\cf datum\coerce{}syntax} procedure allows a transformer to ``bend'' lexical
scoping rules by creating \textit{implicit
  identifiers\mainindex{implicit identifier}}
that behave as if they were present in the input form,
thus permitting the definition of syntactic
abstractions that introduce visible bindings for or references to
identifiers that do not appear explicitly in the input form.
For example, the following defines a {\cf loop} expression that
uses this controlled form of identifier capture to
bind the variable {\cf break} to an escape procedure
within the loop body.
(The derived {\cf with-syntax} form is like {\cf let} but binds
pattern variables---see section~\ref{derivedsection}.)

\begin{scheme}
(define-syntax loop
  (lambda (x)
    (syntax-case x ()
      [(k e ...)
       (with-syntax
           ([break (datum->syntax \#'k 'break)])
         \#'(call-with-current-continuation
             (lambda (break)
               (let f () e ... (f)))))])))

(let ((n 3) (ls '()))
  (loop
    (if (= n 0) (break ls))
    (set! ls (cons 'a ls))
    (set! n (- n 1)))) \lev (a a a)
\end{scheme}

Were {\cf loop} to be defined as

\begin{scheme}
(define-syntax loop
  (lambda (x)
    (syntax-case x ()
      [(\_ e ...)
       \#'(call-with-current-continuation
           (lambda (break)
             (let f () e ... (f))))])))
\end{scheme}

the variable {\cf break} would not be visible in {\cf e \dots}.

The datum argument \var{datum} may also represent an arbitrary
Scheme form, as demonstrated by the following definition of
{\cf include}, an expand-time version of {\cf load}.

\begin{scheme}
(define-syntax include
  (lambda (x)
    (define read-file
      (lambda (fn k)
        (let ([p (open-file-input-port fn)])
          (let f ([x (get-datum p)])
            (if (eof-object? x)
                (begin (close-port p) '())
                (cons (datum->syntax k x)
                      (f (get-datum p))))))))
    (syntax-case x ()
      [(k filename)
       (let ([fn (syntax->datum \#'filename)])
         (with-syntax ([(exp ...)
                        (read-file fn \#'k)])
           \#'(begin exp ...)))])))
\end{scheme}

{\cf (include "filename")} expands into a {\cf begin} expression
containing the forms found in the file named by
{\cf "filename"}.
For example, if the file {\cf flib.ss} contains
{\cf (define f (lambda (x) (g (* x x))))}, and the file
{\cf glib.ss} contains
{\cf (define g (lambda (x) (+ x x)))},
the expression

\begin{scheme}
(let ()
  (include "flib.ss")
  (include "glib.ss")
  (f 5))
\end{scheme}

evaluates to {\cf 50}.

The definition of {\cf include} uses {\cf datum\coerce{}syntax} to convert
the objects read from the file into syntax objects in the proper
lexical context, so that identifier references and definitions within
those expressions are scoped where the {\cf include} form appears.

Using {\cf datum\coerce{}syntax}, it is even possible to break hygiene
entirely and write macros in the style of old Lisp macros.
The {\cf lisp-transformer} procedure defined below creates a transformer
that converts its input into a datum, calls the programmer's procedure on
this datum, and converts the result back into a syntax object that is
scoped at top level (or, more accurately, wherever
{\cf lisp-transformer} is defined).

\begin{scheme}
(define lisp-transformer
  (lambda (p)
    (lambda (x)
      (datum\coerce{}syntax \#'lisp-transformer
        (p (syntax\coerce{}datum x))))))
\end{scheme}

Using {\cf lisp-transformer}, defining a basic version of Common Lisp's
{\cf defmacro} is a straightforward exercise.

\section{Generating lists of temporaries}
\label{generatingtemporariessection}

Transformers can introduce a fixed number of identifiers into their
output simply by naming each identifier.
In some cases, however, the number of identifiers to be introduced depends
upon some characteristic of the input expression.
A straightforward definition of {\cf letrec}, for example,
requires as many
temporary identifiers as there are binding pairs in the
input expression.
The procedure {\cf generate-temporaries} is used to construct
lists of temporary identifiers.

\begin{entry}{%
\proto{generate-temporaries}{ l}{procedure}}

\domain{\var{L} must be be a list or syntax object representing a list-structured
form; its contents are not important.}
The number of temporaries generated is the number of elements in \var{l}.
Each temporary is guaranteed to be unique, i.e., different from all other
identifiers.

A definition of {\cf letrec} that
uses {\cf generate-temporaries} is shown below.

\begin{schemenoindent}\label{defn:letrec}
(define-syntax letrec
  (lambda (x)
    (syntax-case x ()
      ((\_ ((i v) ...) e1 e2 ...)
       (with-syntax
            (((t ...)
              (generate-temporaries (syntax (i ...)))))
          (syntax (let ((i \schfalse{}) ...)
                    (let ((t v) ...)
                      (set! i t) ...
                      (let () e1 e2 ...)))))))))
\end{schemenoindent}

Any transformer that uses {\cf generate-temporaries} in this fashion can
be rewritten to avoid using it, albeit with a loss of clarity.
The trick is to use a recursively defined intermediate form that
generates one temporary per expansion step and completes the
expansion after enough temporaries have been generated.
\end{entry}

\section{Derived forms and procedures}
\label{derivedsection}

The forms and procedures described in this section are \emph{derived},
i.e., they can defined in terms of the forms and procedures described
in earlier sections of this document.

The {\cf syntax-rules} form (section~\ref{syntax-rules}) could be
implemented as a derived form, with the following extensions:

\begin{itemize}
 \item Patterns are generalized slightly to allow a fixed number of   
   subpatterns to appear after an ellipsis, e.g.,
   {\cf (\hyperi{p} ... \hyperii{p} \hyperiii{p})}.

 \item Underscores (~{\cf \_}~) may appear within the pattern
   and match any input, but are not pattern variables and so
   are not bound in the output \hyper{template}.

 \item The first position of a syntax-rules pattern may be any
   identifier, including an underscore, i.e., it need not be the name
   of the macro being defined.
   This position is always ignored.

 \item An optional fender may appear between the pattern and
   template of any clause and has the same meaning as a
   {\cf syntax-case} fender.
\end{itemize}

\begin{entry}{%
\rproto{syntax-rules}{ (\hyper{literal} \dotsfoo) \hyper{clause} \dotsfoo}{\exprtype}}

\syntax
Each \hyper{literal} must be an identifier.
Each \hyper{clause} must take one of the following two forms.

\begin{scheme}
(\hyper{pattern} \hyper{template})
(\hyper{pattern} \hyper{fender} \hyper{template})
\end{scheme}

Each \hyper{pattern} and \hyper{fender} are as in {\cf syntax-case},
and each \hyper{template}
is as in {\cf syntax}.
(See section~\ref{syntaxcasesection}.)

\semantics
Any {\cf syntax-rules} form can be expressed with
{\cf syntax-case} by making the {\cf lambda} expression and
{\cf syntax} expressions explicit, and
{\cf syntax-rules} may be defined in terms of {\cf syntax-case}
as follows.

\begin{scheme}
(define-syntax syntax-rules
  (lambda (x)
    (syntax-case x ()
      [(\_ (k ...) [(\_ . p) f ... t] ...)
       \#'(lambda (x)
           (syntax-case x (k ...)
             [(\_ . p) f ... \#'t] ...))])))
\end{scheme}

A more robust implementation would verify that the literals
{\cf \hyper{literal} \dots} are all identifiers, that the first position
of each pattern is an identifier, and that at most one fender
is present in each clause.

Since the {\cf lambda} and {\cf syntax} expressions are implicit in
a {\cf syntax-rules} form, definitions expressed with
{\cf syntax-rules} are shorter than the equivalent definitions
expressed with {\cf syntax-case}.
The choice of which to use when either suffices is a matter of taste, but
some transformers that can be written easily with {\cf syntax-case}
cannot be written easily or at all with {\cf syntax-rules}.

The definition of {\cf or} below is like the ones given in
section~\ref{syntaxcasesection}, except
that {\cf syntax-rules} is used in place of {\cf syntax-case}
and {\cf syntax}.

\begin{scheme}
(define-syntax or
  (syntax-rules ()
    [(\_) \schfalse{}]
    [(\_ e) e]
    [(\_ e1 e2 e3 ...)
     (let ([t e1])
       (if t t (or e2 e3 ...)))]))
\end{scheme}
\end{entry}

The definitions of {\cf p.car} in section~\ref{syntaxcasesection}
demonstrated how identifier macros might be written using
{\cf syntax-case}.
Many identifier macros can be defined more succinctly using
the derived {\cf identifier-syntax} form.

\begin{entry}{%
\proto{identifier-syntax}{ \hyper{template}}{\exprtype}
\pproto{(identifier-syntax (\hyperi{id} \hyperi{template})}{\exprtype}}\\
{\tt\obeyspaces
                   ((set! \hyperii{id} \hyper{pattern})
                    \hyperii{template}))}

\syntax The \hyper{id}s must be identifiers.

\semantics
When a keyword is bound to a transformer produced by the first form of
{\cf identifier-syntax}, references to the keyword within the scope
of the binding are replaced by \hyper{template}.

\begin{scheme}
(define p (cons 4 5))
(define-syntax p.car (identifier-syntax (car p)))
p.car \ev 4
(set! p.car 15) \ev \exception{\&syntax}
\end{scheme}

The second, more general, form of {\cf identifier-syntax} permits
the transformer to determine what happens when {\cf set!} is used.

\begin{scheme}
(define p (cons 4 5))
(define-syntax p.car
  (identifier-syntax
    [\_ (car p)]
    [(set! \_ e) (set-car! p e)]))
(set! p.car 15)
p.car           \ev 15
p               \ev (15 5)
\end{scheme}

The {\cf identifier-syntax} form may be defined in terms of {\cf syntax-case},
{\cf syntax}, and {\cf make-variable-transformer} as follows.

\begin{schemenoindent}
(define-syntax identifier-syntax
  (syntax-rules (set!)
    [(\_ e)
     (lambda (x)
       (syntax-case x ()
         [id (identifier? \#'id) \#'e]
         [(\_ x (... ...)) \#'(e x (... ...))]))]
    [(\_ (id exp1) ((set! var val) exp2))
     (and (identifier? \#'id) (identifier? \#'var))
     (make-variable-transformer
       (lambda (x)
         (syntax-case x (set!)
           [(set! var val) \#'exp2]
           [(id x (... ...)) \#'(exp1 x (... ...))]
           [id (identifier? \#'id) \#'exp1])))]))
\end{schemenoindent}
\end{entry}

FIXME: Kent: I replaced expr1 expr2 etc. by body
 
\begin{entry}{%
\pproto{(with-syntax ((\hyper{pattern} \hyper{expression}) \dotsfoo) \hyper{body})}{\exprtype}}
\mainschindex{with-syntax}

The derived {\cf with-syntax} form is used to bind pattern variables,
just as {\cf let} is used to bind variables.
This allows a transformer to construct its output in separate
pieces, then put the pieces together.

Each \hyper{pattern} is identical in form to a {\cf syntax-case} pattern.
The value of each \hyper{expression} is computed and destructured according
to the corresponding \hyper{pattern}, and pattern variables within
the \hyper{pattern} are bound as with {\cf syntax-case} to the
corresponding portions of the value within \hyper{body}.

The {\cf with-syntax} form may be defined in terms of {\cf syntax-case} as
follows.

\begin{scheme}
(define-syntax with-syntax
  (lambda (x)
    (syntax-case x ()
      ((\_ ((p e0) ...) e1 e2 ...)
       (syntax (syntax-case (list e0 ...) ()
                 ((p ...) (begin e1 e2 ...))))))))
\end{scheme}

The following definition of {\cf cond} demonstrates the use of
{\cf with-syntax} to support transformers that employ recursion
internally to construct their output.
It handles all {\cf cond} clause variations and takes care to produce
one-armed {\cf if} expressions where appropriate.

\begin{schemenoindent}
(define-syntax cond
  (lambda (x)
    (syntax-case x ()
      [(\_ c1 c2 ...)
       (let f ([c1 \#'c1] [c2* \#'(c2 ...)])
         (syntax-case c2* ()
           [()
            (syntax-case c1 (else =>)
              [(else e1 e2 ...) \#'(begin e1 e2 ...)]
              [(e0) \#'(let ([t e0]) (if t t))]
              [(e0 => e1)
               \#'(let ([t e0]) (if t (e1 t)))]
              [(e0 e1 e2 ...)
               \#'(if e0 (begin e1 e2 ...))])]
           [(c2 c3 ...)
            (with-syntax ([rest (f \#'c2 \#'(c3 ...))])
              (syntax-case c1 (=>)
                [(e0) \#'(let ([t e0]) (if t t rest))]
                [(e0 => e1)
                 \#'(let ([t e0]) (if t (e1 t) rest))]
                [(e0 e1 e2 ...)
                 \#'(if e0 
                        (begin e1 e2 ...)
                        rest)]))]))])))
\end{schemenoindent}
\end{entry}

\section{Syntax violations}

\begin{entry}{%
\proto{syntax-violation}{ who message form}{procedure}
\rproto{syntax-violation}{ who message form subform}{procedure}}

\domain{\var{Name} must be \schfalse{} or a string or a symbol.
  \var{Message} must be a string.
  \var{Form} must be a syntax object or a datum value.
  \var{Subform} must be a syntax object or a datum value.}
The {\cf syntax-violation} procedure raises an exception, reporting 
a syntax violation.  
The \var{who} argument should describe the macro transformer that
detected the exception.  The \var{message} argument should describe
the violation.
The \var{form} argument is the erroneous source syntax
object or a datum value representing a form. The optional \var{subform} argument is a syntax
object or datum value representing a form that more precisely locates the
violation.

If \var{who} is \schfalse{}, {\cf syntax-violation} will attempt to
infer an appropriate value for the condition object (see below) as
follows:  When \var{form} is either an identifier or a
list-structured syntax object containing an identifier as its first element, then
the inferred value is the identifier's symbol.
Otherwise, no value for \var{who} will be provided as part of the
condition object.

The condition object provided with the exception (see
chapter~\ref{exceptionsconditionschapter}) has the following condition types:
%
\begin{itemize}
\item If \var{who} is not \schfalse or can be inferred, the condition has condition type
  {\cf \&who}, with \var{who} as the value of the {\cf who} field.  In
  that case, \var{who} should identify the procedure or entity that
  detected the exception.  If it is \schfalse, the condition does not
  have condition type {\cf \&who}.
\item The condition has condition type {\cf \&message}, with
  \var{message} as the value of the {\cf message} field.
\item The condition has condition type {\cf \&syntax-violation} 
  with \var{form} as the value of the {\cf form} field,
  and \var{subform} as the value of the {\cf subform} field.
  If \var{subform} is not provided, the value of the {\cf subform}
  field is \schfalse.
\end{itemize}
\end{entry}

\section{{\tt quasisyntax}}

\begin{entry}{%
\proto{quasisyntax}{ \hyper{template}}{\exprtype}}

The {\cf quasisyntax} form is similar to {\cf syntax}, but it allows parts
of the quoted text to be evaluated, in a manner similar to the operation
of {\cf quasiquote} (section~\ref{quasiquotesection}).

Within a {\cf quasisyntax} \var{template}, subforms of
{\cf unsyntax} and {\cf unsyntax-splicing} forms are evaluated,
and everything else is treated as ordinary template material, as
with {\cf syntax}.
The value of each {\cf unsyntax} subform is inserted into the output
in place of the {\cf unsyntax} form, while the value of each
{\cf unsyntax-splicing} subform is spliced into the surrounding list
or vector structure.
Uses of {\cf unsyntax} and {\cf unsyntax-splicing} are valid only within
{\cf quasisyntax} expressions.

A {\cf quasisyntax} expression may be nested, with each {\cf quasisyntax}
introducing a new level of syntax quotation and each {\cf unsyntax} or
{\cf unsyntax-splicing} taking away a level of quotation.
An expression nested within $n$ {\cf quasisyntax} expressions must
be within $n$ {\cf unsyntax} or {\cf unsyntax-splicing} expressions to
be evaluated.

The {\cf quasisyntax} keyword can be used in place of {\cf with-syntax} in many
cases.
For example, the definition of {\cf case} shown under the description
of {\cf with-syntax} above can be rewritten using {\cf quasisyntax}
as follows.

\begin{schemenoindent}
(define-syntax case
  (lambda (x)
    (syntax-case x ()
      [(\_ e c1 c2 ...)
       \#`(let ([t e])
           \#,(let f ([c1 \#'c1] [cmore \#'(c2 ...)])
               (if (null? cmore)
                   (syntax-case c1 (else)
                     [(else e1 e2 ...)
                      \#'(begin e1 e2 ...)]
                     [((k ...) e1 e2 ...)
                      \#'(if (memv t '(k ...))
                             (begin e1 e2 ...))])
                   (syntax-case c1 ()
                     [((k ...) e1 e2 ...)
                      \#`(if (memv t '(k ...))
                            (begin e1 e2 ...)
                            \#,(f (car cmore)
                                  (cdr cmore)))]))))])))
\end{schemenoindent}
                          
Uses of {\cf unsyntax} and {\cf unsyntax-splicing} with zero or more than
one subform are valid only in splicing (list or vector) contexts.
{\cf (unsyntax \var{template} \dots)} is equivalent to
{\cf (unsyntax \var{template}) \dots}, and
{\cf (unsyntax-splicing \var{template} \dots)} is equivalent to
{\cf (unsyntax-splicing \var{template}) \dots}.
These forms are primarily useful as intermediate forms in the output
of the {\cf quasisyntax} expander.

\begin{note}
Uses of {\cf unsyntax} and {\cf unsyntax-splicing} with 
zero or more than one subform enable certain 
idioms~\cite{bawdenquasiquote}, such as {\cf \#,@\#,@}, which has the
effect of a doubly indirect splicing when used within a doubly nested
and doubly evaluated {\cf quasisyntax} expression, as with the
nested {\cf quasiquote} examples shown in
section~\ref{quasiquotesection}.  FIXME: we don't have any yet
\end{note}
\end{entry}

%%% Local Variables: 
%%% mode: latex
%%% TeX-master: "r6rs"
%%% End: 

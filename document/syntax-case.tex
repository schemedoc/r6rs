\documentclass{monograph}

%TODO:
%  - algebra for marks and substitutions
%    see Waddell's dissertation or POPL '99 module paper
%  - spell out the connection between syntax and lists somewhere
%  - spell out scoping rules, something like:
%    - the body of a library imported "for run" is at meta-level 0
%    - a transformer appearing in a meta-level n expression is at
%      meta-level n+1
%    - an introduced identifier (any identifier other than a pattern
%      variable) in a meta-level n+1 syntax template is at meta-level n
%    - meta-level n identifiers are bound only by meta-level n bindings
%    - in (let () (define-syntax a T) (define x e) ---), occurrences
%      of x outside of a template in T do not refer to the x
%      bound by (define x e), while occurrences of x inside of a template
%      in T do refer x bound by (define x e).
%  - work fresh-syntax into main SRFI body

%SRFI notes (see also Change Log section):
% - we need a test suite
%   (Per Bothner <per@bothner.com>)
% - would like a shorter convenience form for syntax-case definitions
%   (Per Bothner <per@bothner.com>)
% - consider (syntax-lambda (stx) ---) form or similar to abstract away from
%   single-argument procedure
%   (Jorgen Schaefer <forcer@forcix.cx>)
% - mixed feelings on identifier macros
%   (various in ``Various comments'' thread)
% - prefer that fresh-syntax be the default
%   (Andre van Tonder <andre@now.het.brown.edu>)
% - would like quasisyntax to be added if can't be defined as a
%   syntax-case macro
%   (Andre van Tonder <andre@now.het.brown.edu>)
% - please spell out in more detail how the expander deals with ellipses
%   in output templates.  See case example in +srfi93, message 54.
%   (Arthur A. Gleckler <arthur@zurich.csail.mit.edu>)
% - "Let me mention that I like the rules for unwrapping."
%   (Andre van Tonder <andre@now.het.brown.edu>)
% - would like Oscar's algebra updated for r6rs and included
%   (Andre van Tonder <andre@now.het.brown.edu>)

\usepackage{scheme}
% \usepackage{grammar}

\def\r#1rs{R#1RS}
\iflatex
\input{fullpage.sty}
\fi

\ifhtml
\headerstuff{\raw{
<style type="text/css">
<!--
 a:link, a:active, a:visited {color:blue}
 a:hover {color:white; background:blue}
 a.plain:link, a.plain:active, a.plain:visited {color:blue; text-decoration:none}
 a.plain:hover {color:white; text-decoration:none; background:blue}
 table.indent {margin-left: 20px}
 h1 { font-size: 1.75em }
 h2 { font-size: 1.25em }
 h3 { font-size: 1.12em }
 h4 { font-size: 1em }
-->
</style>
}}
\documenttitle{R6RS Syntax-Case Macros}
\fi

\iflatex
\pagestyle{plain}
\fi

\ifhtml
\renewcommand{\sectionstar}[1]{\raw{\raw{<h1>}}#1\raw{\raw{</h1>}}}
\renewcommand{\subsectionstar}[1]{\raw{\raw{<h2>}}#1\raw{\raw{</h2>}}}
\renewcommand{\subsubsectionstar}[1]{\raw{\raw{<h3>}}#1\raw{\raw{</h3>}}}
\fi

\begin{document}

\iflatex
% block paragraphs
\schemeindent=0pt
\parskip=4pt
\parindent=0pt
\fi

\sectionstar{Title}

R6RS Syntax-Case Macros

\sectionstar{Authors}

Kent Dybvig

\sectionstar{Status}

This SRFI is being submitted by a member of the Scheme Language Editor's
Committee as part of the {\r6rs} Scheme standardization process.  The purpose
of such ``{\r6rs} SRFIs'' is to inform the Scheme community of features and
design ideas under consideration by the editors and to allow the community
to give the editors some direct feedback that will be considered during
the design process.

At the end of the discussion period, this SRFI will be withdrawn.  When
the {\r6rs} specification is finalized, the SRFI may be revised to conform to
the {\r6rs} specification and then resubmitted with the intent to finalize
it.  This procedure aims to avoid the situation where this SRFI is
inconsistent with {\r6rs}.  An inconsistency between {\r6rs} and this SRFI could
confuse some users.  Moreover it could pose implementation problems for
{\r6rs} compliant Scheme systems that aim to support this SRFI.  Note that
departures from the SRFI specification by the Scheme Language Editor's
Committee may occur due to other design constraints, such as design
consistency with other features that are not under discussion as SRFIs.

\ifhtml
\sectionstar{Table of Contents}
\tableofcontents
\fi

\section{Abstract}

The syntactic abstraction system described here extends the {\r5rs} macro
system with support for writing low-level macros in a high-level style,
with automatic syntax checking, input destructuring, output restructuring,
maintenance of lexical scoping and referential transparency (hygiene), and
support for controlled identifier capture, with constant expansion
overhead.
Because it does not require literals, including quoted lists or vectors,
to be copied or even traversed, it preserves sharing and cycles within and
among the constants of a program.
It also supports source-object correlation, i.e., the maintenance of ties
between the original source code and expanded output, allowing
implementations to provide source-level support for debuggers and other
tools.

\section{Rationale\label{sec:rationale}}

While many syntactic abstractions are succinctly expressed using the
high-level \scheme{syntax-rules} form, others are difficult or impossible
to write, including some that introduce visible bindings for or references
to identifiers that do not appear explicitly in the input form, ones that
maintain state or read from the file system, and ones that construct new
identifiers.
The \scheme{syntax-case} system~\cite{Dybvig:syntactic} described here
allows the programmer to write transformers that perform these sorts of
transformations, and arbitrary additional transformations, without
sacrificing the default enforcement of hygiene or the high-level
pattern-based syntax matching and template-based output construction
provided by {\r6rs} \scheme{syntax-rules}.

\section{Specification\label{sec:specification}}

A syntactic abstraction typically takes the form
\scheme{(\var{keyword} \var{subform} \dots)}, where \var{keyword} is the
identifier that names the syntactic abstraction.
The syntax of each \var{subform} varies from one syntactic abstraction to
another.
Syntactic abstractions can also take the form of improper lists (or even
singleton identifiers; see Section~\ref{sec:transformers}), although this is
less common.

New syntactic abstractions are defined by associating keywords with
\emph{transformers}.
Keyword bindings are created using \scheme{define-syntax},
\scheme{let-syntax}, or \scheme{letrec-syntax}.
Transformers are created using \scheme{syntax-rules} or
\scheme{syntax-case} and \scheme{syntax}, which allow transformations to
be specified via pattern matching and template reconstruction.

\subsection{Expansion Process\label{sec:expansion}}

Syntactic abstractions are expanded into core
forms at the start of evaluation (before compilation or interpretation)
by a syntax \emph{expander}.
(The set of core forms is implementation-dependent, as is the
representation of these forms in the expander's output.)
The expander is invoked once for each top-level form in a program.
If the expander encounters a syntactic abstraction, it invokes
the associated transformer to expand the syntactic abstraction, then
repeats the expansion process for the form returned by the transformer.
If the expander encounters a core form, it recursively
processes the subforms, if any, and reconstructs the form from the
expanded subforms.
Information about identifier bindings is maintained during expansion
to enforce lexical scoping for variables and keywords.

To handle internal definitions, the expander
processes the initial forms in a \scheme{library} or \scheme{lambda} body
from left to right.
How the expander processes each form encountered as it does so depends
upon the kind of form.

\begin{description}
\item[syntactic abstraction:]
The expander invokes the associated transformer to transform the syntactic abstraction,
then recursively performs whichever of these actions are appropriate
for the resulting form.

\item[\scheme{define-syntax} form:]
The expander expands and evaluates the right-hand-side expression and binds the
keyword to the resulting transformer.

\item[\scheme{define} form:]
The expander records the fact that the defined identifier is a variable but defers
expansion of the right-hand-side expression until after all of the
definitions have been processed.

\item[\scheme{begin} form:]
The expander splices the subforms into the list of body forms it is processing.

\item[\scheme{let-syntax} or \scheme{letrec-syntax} form:]
The expander splices the inner body forms into the list of (outer) body forms it is
processing, arranging for the keywords bound by the \scheme{let-syntax}
and \scheme{letrec-syntax} to be visible only in the inner body forms.

\item[expression, i.e., nondefinition:]
The expander completes the expansion of the deferred right-hand-side forms
and the current and remaining expressions in the body, then
constructs a residual \scheme{letrec*} form from the defined variables,
expanded right-hand-side expressions, and expanded body expressions.
\end{description}

An error is signaled if the keyword that identifies one of the body forms
as a definition (derived or core) is redefined by the same definition or a
later definition in the same body.
To catch this error, the expander records the identifying keyword for each
syntactic abstraction, \scheme{define-syntax} form, \scheme{define}
form, \scheme{begin} form, \scheme{let-syntax} form, and \scheme{letrec-syntax}
form it encounters while processing the definitions and checks each
newly defined identifier (\scheme{define} or \scheme{define-syntax}
left-hand side) against the recorded keywords, as with
\scheme{bound-identifier=?} (Section~\ref{sec:identifierpredicates}).
For example, the following forms result in an error.

\schemedisplay
(let ()
  (define define 17)
  define)

(let-syntax ([def0 (syntax-rules ()
                     [(_ x) (define x 0)])])
  (let ()
    (def0 z)
    (define def0 '(def 0))
    (list z def0)))
\endschemedisplay

Expansion of each variable definition right-hand side is deferred until
after all of the definitions have been seen so that each keyword and
variable reference within the right-hand side resolves to the local
binding, if any.

Note that this algorithm does not directly reprocess any form.
It requires a single left-to-right pass over the definitions followed by a
single pass (in any order) over the body expressions and deferred
right-hand sides.

For example, in

\schemedisplay
(lambda (x)
  (define-syntax defun
    (syntax-rules () [(_ (x . a) e) (define x (lambda a e))]))
  (defun (even? n) (or (= n 0) (odd? (- n 1))))
  (define-syntax odd? (syntax-rules () [(_ n) (not (even? n))]))
  (odd? (if (odd? x) (* x x) x)))
\endschemedisplay

The definition of \scheme{defun} is encountered first, and the keyword
\scheme{defun} is associated with the transformer resulting from
the expansion and evaluation of the corresponding right-hand side.
A use of \scheme{defun} is encountered next and expands into a
\scheme{define} form.
Expansion of the right-hand side of this define form is deferred.
The definition of \scheme{odd?} is next and results in the association
of the keyword \scheme{odd?} with the transformer resulting from
expanding and evaluating the corresponding right-hand side.
A use of \scheme{odd?} appears next and is expanded; the resulting
call to \scheme{not} is recognized as an expression
because \scheme{not} is bound as a variable.
At this point, the expander completes the expansion of the current
expression (the \scheme{not} call) and the deferred right-hand side of the
\scheme{even?} definition;
the uses of \scheme{odd?} appearing in these expressions are expanded
using the transformer associated with the keyword \scheme{odd?}.
The final output is the equivalent of

\schemedisplay
(lambda (x)
  (letrec* ([even? (lambda (n)
                     (or (= n 0) (not (even? (- n 1)))))])
    (not (even? (if (not (even? x)) (* x x) x)))))
\endschemedisplay

although the structure of the output is implementation dependent.

\subsection{Maintaining Hygiene\label{sec:hygiene}}

% hygiene condition for macro expansion
% (Kohlbecker, E.E., Friedman, D.P., Felleisen, M., Duba, B. 'Hygienic macro expansion' (1986))
% "Generated identifiers that become binding instances in the completely
% expanded program must only bind variables that are generated at the same
% transcription step."

Barendregt's \emph{hygiene condition}~\cite{Barendregt:introduction} for the
lambda-calculus is an informal notion that requires the free variables of
an expression $N$ that is to be substituted into another expression $M$ not to
be captured by bindings in $M$ when such capture is not intended.
Kohlbecker, et al~\cite{Kohlbecker:hygienic} propose a corresponding
\emph{hygiene condition for macro expansion} that applies in all situations
where capturing is not explicit:
``Generated identifiers that become binding instances in
the completely expanded program must only bind variables that
are generated at the same transcription step.''
In the terminology of this document, the ``generated identifiers'' are
those introduced by a transformer rather than those present in the form
passed to the transformer, and a ``macro transcription step'' corresponds
to a single call by the expander to a transformer.
Also, the hygiene condition applies to all introduced bindings rather than
to introduced variable bindings alone.

This leaves open what happens to an introduced identifier that appears
outside the scope of a binding introduced by the same call.
Such an identifier refers to the lexical binding in effect where it
appears (within a \scheme{syntax} \var{template};
see Section~\ref{sec:syntaxcase}) inside the transformer body or one of
the helpers it calls.
This is essentially the referential transparency property described
by Clinger and Rees~\cite{Clinger:macros}.

Thus, the hygiene condition can be restated as follows:

\begin{quotation}
\noindent
A binding for an identifier introduced into the output of a transformer
call from the expander must capture only references to the identifier
introduced into the output of the same transformer call.
A reference to an identifier introduced into the output of a transformer
refers to the closest enclosing binding for the introduced identifier or,
if it appears outside of any enclosing binding for the introduced
identifier, the closest enclosing lexical binding where the identifier
appears (within a \scheme{syntax} \var{template})
inside the transformer body or one of the helpers it calls.
\end{quotation}

Explicit captures are handled via \scheme{datum->syntax}; see
Section~\ref{sec:conversions}.

Operationally, the expander can maintain hygiene with the help of
\emph{marks} and \emph{substitutions}.
Marks are applied selectively by the expander to the output of each
transformer it invokes, and substitutions are applied to the portions
of each binding form that are supposed to be within the scope of the bound
identifiers.
Marks are used to distinguish like-named identifiers that are
introduced at different times (either present in the source or introduced
into the output of a particular transformer call), and substitutions are
used to map identifiers to their expand-time values.

Each time the expander encounters a syntactic abstraction, it applies an
\emph{antimark} to the input form, invokes the associated transformer,
then applies a fresh mark to the output.
Marks and antimarks cancel, so the portions of the input that appear in
the output are effectively left unmarked, while the portions of the output
that are introduced are marked with the fresh mark.

Each time the expander encounters a binding form it creates a set of
substitutions, each mapping one of the (possibly marked) bound identifiers
to information about the binding.
(For a \scheme{lambda} expression, the expander might map each bound
identifier to a representation of the formal parameter in the output of
the expander.
For a \scheme{let-syntax} form, the expander might map each bound
identifier to the associated transformer.)
These substitutions are applied to the portions of the input form in
which the binding is supposed to be visible.

Marks and substitutions together form a \emph{wrap} that is layered on the
form being processed by the expander and pushed down toward the leaves as
necessary.
A wrapped form is referred to as a \emph{wrapped syntax object}.
Ultimately, the wrap may rest on a leaf that represents an identifier, in
which case the wrapped syntax object is referred to more precisely
as an \emph{identifier}.
An identifier contains a name along with the wrap.
(Names are typically represented by symbols.)

When a substitution is created to map an identifier to an expand-time
value, the substitution records the name of the identifier and
the set of marks that have been applied to that identifier, along
with the associated expand-time value.
The expander resolves identifier references by looking for the latest
matching substitution to be applied to the identifier, i.e., the outermost
substitution in the wrap whose name and marks match the name and
marks recorded in the substitution.
The name matches if it is the same name (if using symbols, then by
\scheme{eq?}), and the marks match if the marks recorded with the
substitution are the same as those that appear \emph{below} the
substitution in the wrap, i.e., those that were applied \emph{before} the
substitution.
Marks applied after a substitution, i.e., appear over the substitution in
the wrap, are not relevant and are ignored.

An algebra that defines how marks and substitutions work more precisely is
given in Section~2.4 of Oscar Waddell's PhD thesis~\cite{Waddell:phd}.

\subsection{Keyword Bindings\label{sec:bindings}}

Keyword bindings may be established with \scheme{define-syntax},
\scheme{let-syntax}, or \scheme{letrec-syntax}.

A \scheme{define-syntax} form is a \var{definition} and may appear
anywhere other definitions may appear.
The syntax

\schemedisplay
(define-syntax \var{keyword} \var{transformer-expr})
\endschemedisplay

binds \var{keyword} to the result of evaluating, at expansion time,
the expression \var{transformer-expr}, which must evaluate
to a \emph{transformer} (Section~\ref{sec:transformers}).

The example below defines \scheme{let*} as a syntactic abstraction,
specifying the transformer with \scheme{syntax-rules} (see
Section~\ref{sec:derived}).

\schemedisplay
(define-syntax let*
  (syntax-rules ()
    [(_ () e1 e2 ...) (let () e1 e2 ...)]
    [(_ ([i1 v1] [i2 v2] ...) e1 e2 ...)
     (let ([i1 v1])
       (let* ([i2 v2] ...) e1 e2 ...))]))
\endschemedisplay

Keyword bindings established by \scheme{define-syntax} are visible
throughout the body in which they appear, except where shadowed by
other bindings, and nowhere else, just like variable bindings established
by \scheme{define}.
All bindings established by a set of internal definitions, whether
keyword or variable definitions, are visible within the definitions
themselves.
For example, the expression

\schemedisplay
(let ()
  (define even?
    (lambda (x)
      (or (= x 0) (odd? (- x 1)))))
  (define-syntax odd?
    (syntax-rules ()
      [(_ x) (not (even? x))]))
  (even? 10))
\endschemedisplay

is valid and should return \scheme{#t}.

An implication of the left-to-right processing order
(Section~\ref{sec:expansion}) is that one internal definition can affect
whether a subsequent form is also a definition.
For example, the expression

\schemedisplay
(let ()
  (define-syntax bind-to-zero
    (syntax-rules ()
      [(_ id) (define id 0)]))
  (bind-to-zero x)
  x)
\endschemedisplay

evaluates to \scheme{0}, regardless of any binding for
\scheme{bind-to-zero} that might appear outside of the \scheme{let}
expression.

\scheme{let-syntax} and \scheme{letrec-syntax} are analogous to \scheme{let}
and \scheme{letrec} but bind keywords rather than variables.
Like \scheme{begin}, a \scheme{let-syntax} or \scheme{letrec-syntax} form
may appear in a definition context, in which case it is treated as a
definition, and the forms in the body of the form must also be
definitions.
A \scheme{let-syntax} or \scheme{letrec-syntax} form may also appear in an
expression context, in which case the forms within their bodies must be
expressions.

The syntax

\schemedisplay
(let-syntax ((\var{keyword} \var{transformer-expr}) \dots) \var{form_1} \var{form_2} \dots)
\endschemedisplay

binds the keywords \scheme{\var{keyword} \dots} to the results of
evaluating, at expansion time, the expressions
\scheme{\var{transformer-expr} \dots}, which must evaluate to
transformers (Section~\ref{sec:transformers}).

Keyword bindings established by \scheme{let-syntax} are visible
throughout the forms in the body of the \scheme{let-syntax} form,
except where shadowed, and nowhere else.

The syntax

\schemedisplay
(letrec-syntax ((\var{keyword} \var{transformer-expr}) \dots) \var{form_1} \var{form_2} \dots)
\endschemedisplay

is similar, but the bindings established by \scheme{let-syntax} are
also visible within \scheme{\var{transformer-expr} ...}.

The forms in the of a \scheme{let-syntax} or \scheme{letrec-syntax} are
treated, whether in definition or expression context, as if wrapped in
an implicit \scheme{begin}.

The following example highlights how \scheme{let-syntax}
and \scheme{letrec-syntax} differ.

\schemedisplay
(let ([f (lambda (x) (+ x 1))])
  (let-syntax ([f (syntax-rules ()
                    [(_ x) x])]
               [g (syntax-rules ()
                    [(_ x) (f x)])])
    (list (f 1) (g 1)))) ;=> (1 2)

(let ([f (lambda (x) (+ x 1))])
  (letrec-syntax ([f (syntax-rules ()
                       [(_ x) x])]
                  [g (syntax-rules ()
                       [(_ x) (f x)])])
    (list (f 1) (g 1)))) ;=> (1 1)
\endschemedisplay

The two expressions are identical except that the \scheme{let-syntax} form
in the first expression is a \scheme{letrec-syntax} form in the second.
In the first expression, the \scheme{f} occurring in \scheme{g} refers to
the \scheme{let}-bound variable \scheme{f}, whereas in the second it refers
to the keyword \scheme{f} whose binding is established by the
\scheme{letrec-syntax} form.

Keywords occupy the same name space as variables, i.e., within the same
scope, an identifier can be bound as a variable or keyword, or neither, but
not both.

\subsection{Transformers\label{sec:transformers}}

A transformer is a \emph{transformation procedure} or a
\var{variable transformer}.
A transformation procedure is a procedure that must accept one
argument, a wrapped syntax object (Section~\ref{sec:syntaxobjects})
representing the input, and return a \emph{syntax object}
(Section~\ref{sec:syntaxobjects}) representing the output.
The procedure is called by the expander whenever a reference to
a keyword with which it has been associated is found.
If the keyword appears in the first position of a list-structured
input form, the transformer receives the entire list-structured
form, and its output replaces the entire form.
If the keyword is found in any other definition or expression context, the
transformer receives a wrapped syntax object representing just the keyword
reference, and its output replaces just the reference.
An error is signaled if the keyword appears on the
left-hand side of a \scheme{set!} expression.

Variable transformers are similar.
If a keyword associated with a variable transformer appears on
the left-hand side of a \scheme{set!} expression, however, an error
is not signaled.
Instead, the transformer receives a wrapped syntax object representing the
entire \scheme{set!} expression as its argument, and its output
replaces the entire \scheme{set!} expression.
A variable transformer is created by passing a transformation
procedure to \scheme{make-variable-transformer}:

\schemedisplay
(make-variable-transformer \var{procedure})
\endschemedisplay

\scheme{make-variable-transformer} returns
an implementation-dependent encapsulation of the transformation
procedure that allows the expander to recognize that it is a
variable transformer.

\subsection{Syntax objects\label{sec:syntaxobjects}}

A syntax object is a representation of a Scheme form that contains
contextual information about the form in addition to its structure.
This contextual information is used by the expander to maintain
lexical scoping and may also be used by an implementation to maintain
source-object correlation.

Syntax objects may be wrapped or unwrapped.
A wrapped syntax object (Section~\ref{sec:hygiene}), consists of a
\emph{wrap} (Section~\ref{sec:hygiene}) and some internal representation
of a Scheme form.
(The internal representation is unspecified, but is typically a Scheme
s-expression or s-expression annotated with source information.)
A wrapped syntax object representing an identifier is itself referred to as
an identifier; thus, the term \emph{identifier} may refer either to
the syntactic entity (symbol, variable, or keyword) or to the
concrete representation of the syntactic entity as a syntax object.
Wrapped syntax objects are distinct from other types of values.

% NOTE: if we want to allow wrapped syntax objects to be represented
% as unwrapped syntax objects, we can say instead that "identifiers" are
% distinct from other types of values

An unwrapped syntax object is one that is unwrapped, fully or partially,
i.e., whose outer layers consist of lists and vectors and whose leaves are
either wrapped syntax objects or nonsymbol values.

The term \emph{syntax object} is used in this document to refer to
a syntax object that is either wrapped or unwrapped.
More formally, a syntax object is:

\begin{itemize}
\item a pair or list of syntax objects,
\item a vector of syntax objects,
\item a nonlist, nonvector, nonsymbol value, or
\item a wrapped syntax object.
\end{itemize}

The distinction between the terms ``syntax object'' and ``wrapped syntax
object'' is important.
For example, when invoked by the expander, a transformer
(Section~\ref{sec:transformers}) must accept a wrapped syntax object but
may return any syntax object, including an unwrapped syntax object.

\subsection{Parsing input and producing output\label{sec:syntaxcase}}

Transformers destructure their input with \scheme{syntax-case} and rebuild
their output with \scheme{syntax}.

A \scheme{syntax-case} expression has the following syntax.

\schemedisplay
(syntax-case \var{expr} (\var{literal} \dots) \var{clause} \dots)
\endschemedisplay

Each \var{literal} must be an identifier.
Each \var{clause} must take one of the following two forms.

\schemedisplay
(\var{pattern} \var{output-expr})
(\var{pattern} \var{fender} \var{output-expr})
\endschemedisplay

A \var{pattern} is an identifier, constant, or one of the following.

\scheme{(}\var{pattern}{\kstar}\scheme{)}\\
\scheme{(}\var{pattern}\kplus\ . \var{pattern}\scheme{)}\\
\scheme{(}\var{pattern}{\kstar} \var{pattern} \var{ellipsis} \var{pattern}{\kstar}\scheme{)}\\
\scheme{(}\var{pattern}{\kstar} \var{pattern} \var{ellipsis} \var{pattern}{\kstar} . \var{pattern}\scheme{)}\\
\scheme{#(}\var{pattern}{\kstar}\scheme{)}\\
\scheme{#(}\var{pattern}{\kstar} \var{pattern} \var{ellipsis} \var{pattern}{\kstar}\scheme{)}

An identifier appearing within a \var{pattern} may be an underscore
(~\scheme{_}~), a literal identifier listed in the list of literals
\scheme{(\var{literal} \dots)}, or an ellipsis (~\scheme{...}~).
All other identifiers appearing within a \var{pattern} are
\emph{pattern variables}.
An ellipsis or underscore may not appear in \scheme{(\var{literal} \dots)}.

Pattern variables match arbitrary input subforms and
are used to refer to elements of the input.
The same pattern variable may not appear more than once in a
\var{pattern}.

Underscores also match arbitrary input subforms but are not pattern variables
and so cannot be used to refer to those elements.
Multiple underscores may appear in a \var{pattern}.

A literal identifier matches an input subform if and only if the input
subform is an identifier and either both its occurrence in the input
expression and its occurrence in the list of literals have the same
lexical binding, or the two identifiers have the same name and both have
no lexical binding.

A subpattern followed by an ellipsis can match zero or more elements of
the input.

More formally, an input form \var{F} matches a pattern \var{P} if and only if

\begin{itemize}
\item \var{P} is an underscore (~\scheme{_}~),

\item \var{P} is a pattern variable,

\item \var{P} is a literal identifier
and \var{F} is an equivalent identifier in the
sense of \scheme{free-identifier=?}
(Section~\ref{sec:identifierpredicates}),

\item \var{P} is of the form
\scheme{(\var{P_1} \dots \var{P_n})}
and \var{F} is a list of $n$ elements that match \var{P_1} through
\var{P_n},

\item \var{P} is of the form
\scheme{(\var{P_1} \dots \var{P_n} . \var{P_x})}
and \var{F} is a list or improper list of $n$ or more elements
whose first $n$ elements match \var{P_1} through \var{P_n}
and
whose $n$th cdr matches \var{P_x},

\item \var{P} is of the form
\scheme{(\var{P_1} \dots \var{P_k} \var{P_e} \var{ellipsis} \var{P_{m+1}} \dots \var{P_n})},
where \var{ellipsis} is the identifier \scheme{...}
and \var{F} is a proper list of $n$
elements whose first $k$ elements match \var{P_1} through \var{P_k},
whose next $m-k$ elements each match \var{P_e},
and
whose remaining $n-m$ elements match \var{P_{m+1}} through \var{P_n},

\item \var{P} is of the form
\scheme{(\var{P_1} \dots \var{P_k} \var{P_e} \var{ellipsis} \var{P_{m+1}} \dots \var{P_n} . \var{P_x})},
where \var{ellipsis} is the identifier \scheme{...}
and \var{F} is a list or improper list of $n$
elements whose first $k$ elements match \var{P_1} through \var{P_k},
whose next $m-k$ elements each match \var{P_e},
whose next $n-m$ elements match \var{P_{m+1}} through \var{P_n},
and 
whose $n$th and final cdr matches \var{P_x},

\item \var{P} is of the form
\scheme{#(\var{P_1} \dots \var{P_n})}
and \var{F} is a vector of $n$ elements that match \var{P_1} through
\var{P_n},

\item \var{P} is of the form
\scheme{#(\var{P_1} \dots \var{P_k} \var{P_e} \var{ellipsis} \var{P_{m+1}} \dots \var{P_n})},
where \var{ellipsis} is the identifier \scheme{...}
and \var{F} is a vector of $n$ or more elements
whose first $k$ elements match \var{P_1} through \var{P_k},
whose next $m-k$ elements each match \var{P_e},
and
whose remaining $n-m$ elements match \var{P_{m+1}} through \var{P_n},
or

\item \var{P} is a pattern datum (any nonlist, nonvector, nonsymbol
object) and \var{F} is equal to \var{P} in the sense of the
\scheme{equal?} procedure.
\end{itemize}

\scheme{syntax-case} first evaluates \var{expr}.
It then attempts to match
the \var{pattern} from the first \var{clause} against the resulting value,
which is unwrapped as necessary to perform the match.
If the pattern matches the value and no
\var{fender} is present,
\var{output-expr} is evaluated and its value returned as the
value of the \scheme{syntax-case} expression.
If the pattern does not match the value, \scheme{syntax-case} tries
the second \var{clause}, then the third, and so on.
An error is signaled if the value does not match any of the patterns.

If the optional \var{fender} is present, it serves as an additional
constraint on acceptance of a clause.
If the \var{pattern} of a given \var{clause} matches the input value,
the corresponding \var{fender} is evaluated.
If \var{fender} evaluates to a true value, the clause is accepted;
otherwise, the clause is rejected as if the pattern had failed to match
the value.
Fenders are logically a part of the matching process, i.e., they
specify additional matching constraints beyond the basic structure of
the input.

Pattern variables contained within a clause's
\var{pattern} are bound to the corresponding pieces of the input
value within the clause's \var{fender} (if present) and
\var{output-expr}.
Pattern variables can be referenced only within \scheme{syntax}
expressions (see below).
Pattern variables occupy the same name space as program variables and
keywords.

See the examples following the description of \scheme{syntax}.

A \scheme{syntax} form has the following syntax.

\schemedisplay
(syntax \var{template})
\endschemedisplay

\scheme{#'\var{template}} is equivalent to \scheme{(syntax \var{template})}.
The abbreviated form is converted into the longer form when the expression
is read, i.e., prior to expansion.

A \scheme{syntax} expression is similar to a \scheme{quote} expression
except that (1) the values of pattern variables appearing within
\var{template} are inserted into \var{template}, (2) contextual
information associated both with the input and with the template is
retained in the output to support lexical scoping, and (3) the value
of a \scheme{syntax} expression is a syntax object.

A \var{template} is a pattern variable, an identifier that
is not a pattern
variable, a pattern datum, or one of the following.

\scheme{(}\var{subtemplate}{\kstar}\scheme{)}\\
\scheme{(}\var{subtemplate}{\kplus} . \var{template}\scheme{)}\\
\scheme{#(}\var{subtemplate}{\kstar}\scheme{)}

A \var{subtemplate} is a \var{template} followed by zero or more ellipses.

The value of a \var{syntax} form is a copy of \var{template} in which
the pattern variables appearing within the template are replaced with
the input subforms to which they are bound.
Pattern data and identifiers that are not pattern variables
are copied directly into the output.
A subtemplate followed by an ellipsis expands
into zero or more occurrences of the subtemplate.
The subtemplate
must contain at least one pattern variable from a subpattern
followed by an ellipsis.
(Otherwise, the expander would not be able to determine how many times the
subform should be repeated in the output.)
Pattern variables that occur in subpatterns followed by one or more
ellipses may occur only in subtemplates that are
followed by (at least) as many ellipses.
These pattern variables are replaced in the output by the input
subforms to which they are bound, distributed as specified.
If a pattern variable is followed by more ellipses in the template
than in the associated pattern, the input form is replicated as
necessary.

A template of the form
\scheme{(\var{ellipsis} \var{template})} is identical to \var{template}, except that
ellipses within the template have no special meaning.
That is, any ellipses contained within \var{template} are
treated as ordinary identifiers.
In particular, the template \scheme{(... ...)} produces a single
ellipsis, \scheme{...}.
This allows syntactic abstractions to expand into forms containing
ellipses.

\label{wrappingrules}
The output produced by \scheme{syntax} is wrapped or unwrapped according to
the following rules.

\begin{itemize}
\item the copy of \scheme{(\var{t_1} .  \var{t_2})} is a pair if \var{t_1}
      or \var{t_2} contain any pattern variables,
\item the copy of \scheme{(\var{t} \var{ellipsis})} is a list if \var{t}
      contains any pattern variables,
\item the copy of \scheme{#(\var{t_1} ... \var{t_n})} is a vector if any of
      \var{t_1},~\dots,~\var{t_n} contain any pattern variables, and
\item the copy of any portion of \var{t} not containing any pattern variables
      is a wrapped syntax object.
\end{itemize}

The input subforms inserted in place of the pattern variables are wrapped
if and only if the corresponding input subforms are wrapped.

The following definitions of \scheme{or} illustrate \scheme{syntax-case}
and \scheme{syntax}.
The second is equivalent to the first but uses the the \scheme{#'}
prefix instead of the full \scheme{syntax} form.

\schemedisplay
(define-syntax or
  (lambda (x)
    (syntax-case x ()
      [(_) (syntax #f)]
      [(_ e) (syntax e)]
      [(_ e1 e2 e3 ...)
       (syntax (let ([t e1])
                 (if t t (or e2 e3 ...))))])))

(define-syntax or
  (lambda (x)
    (syntax-case x ()
      [(_) #'#f]
      [(_ e) #'e]
      [(_ e1 e2 e3 ...)
       #'(let ([t e1])
           (if t t (or e2 e3 ...)))])))

(define-syntax case
  (lambda (x)
    (syntax-case x (else)
      [(_ e0 [(k ...) e1 e2 ...] ... [else else-e1 else-e2 ...])
       #'(let ([t e0])
           (cond
             [(memv t '(k ...)) e1 e2 ...]
             ...
             [else else-e1 else-e2 ...]))]
      [(_ e0 [(ka ...) e1a e2a ...] [(kb ...) e1b e2b ...] ...)
       #'(let ([t e0])
           (cond
             [(memv t '(ka ...)) e1a e2a ...]
             [(memv t '(kb ...)) e1b e2b ...]
             ...))])))
\endschemedisplay

The examples below define \emph{identifier macros}, syntactic abstractions
supporting keyword references that do not necessarily appear in the first
position of a list-structured form.
The second example uses \scheme{make-variable-transformer} to handle the case
where the keyword appears on the left-hand side of a
\scheme{set!} expression.

\schemedisplay
(define p (cons 4 5))
(define-syntax p.car
  (lambda (x)
    (syntax-case x ()
      [(_ . rest) #'((car p) . rest)]
      [_  #'(car p)])))
p.car ;=> 4
(set! p.car 15) ;=> \var{syntax error}

(define p (cons 4 5))
(define-syntax p.car
  (make-variable-transformer
    (lambda (x)
      (syntax-case x (set!)
        [(set! _ e) #'(set-car! p e)]
        [(_ . rest) #'((car p) . rest)]
        [_  #'(car p)]))))
(set! p.car 15)
p.car           ;=> 15
p               ;=> (15 5)
\endschemedisplay

A derived \scheme{identifier-syntax} form that simplifies the definition
of identifier macros is described in Section~\ref{sec:derived}.

\subsection{Identifier predicates\label{sec:identifierpredicates}}

The procedure \scheme{identifier?} is used to determine if a value is
an identifier.

\schemedisplay
(identifier? \var{x})
\endschemedisplay

It returns \scheme{#t} if its argument \var{x} is an identifier, i.e., a
syntax object representing an identifier, and \scheme{#f} otherwise.

\scheme{identifier?} is often used within a fender to verify
that certain subforms of an input form are identifiers, as in the
definition of \scheme{rec}, which creates self-contained
recursive objects, below.

\schemedisplay
(define-syntax rec
  (lambda (x)
    (syntax-case x ()
      [(_ x e)
       (identifier? #'x)
       #'(letrec ([x e]) x)])))

(map (rec fact
       (lambda (n)
         (if (= n 0)                 ;=> (1 2 6 24 120)
             1
             (* n (fact (- n 1))))))
     '(1 2 3 4 5))

(rec 5 (lambda (x) x)) ;=> \var{syntax error}
\endschemedisplay

The procedures \scheme{bound-identifier=?} and \scheme{free-identifier=?}
each take two identifier arguments and return \scheme{#t} if their
arguments are equivalent and \scheme{#f} otherwise.
These predicates are used to compare identifiers according to their
\emph{intended use} as free references or bound identifiers in a given
context.

\schemedisplay
(bound-identifier=? \var{id_1} \var{id_2})
\endschemedisplay

The procedure \scheme{bound-identifier=?} returns true if and only if a
binding for one would capture a reference to the other in the output of
the transformer, assuming that the reference appears within the scope of
the binding.
In general, two identifiers are \scheme{bound-identifier=?} only if
both are present in the original program or both are introduced by the
same transformer application
(perhaps implicitly---see \scheme{datum->syntax}).
Operationally, two identifiers are
considered equivalent by \scheme{bound-identifier=?} if and only if they
have the same name and same marks (Section~\ref{sec:hygiene}).

\scheme{bound-identifier=?} can be used for detecting
duplicate identifiers in a binding construct or for other
preprocessing of a binding construct that requires detecting instances
of the bound identifiers.

\schemedisplay
(free-identifier=? \var{id_1} \var{id_2})
\endschemedisplay

The procedure \scheme{free-identifier=?} returns \scheme{#t} if and
only if the two identifiers would resolve to the same binding if both were
to appear in the output of a transformer outside of any bindings inserted
by the transformer.
(If neither of two like-named identifiers resolves to a binding, i.e., both
are unbound, they are considered to resolve to the same binding.)
Operationally, two identifiers are considered equivalent by
\scheme{free-identifier=?} if and only the topmost matching
substitution for each maps to the same binding (Section~\ref{sec:hygiene})
or the identifiers have the same name and no matching substitution.

\scheme{syntax-case} and \scheme{syntax-rules} use
\scheme{free-identifier=?} to compare identifiers listed in the literals
list against input identifiers.

The following definition of unnamed \scheme{let}
uses \scheme{bound-identifier=?} to detect duplicate identifiers.

\schemedisplay
(define-syntax let
  (lambda (x)
    (define unique-ids?
      (lambda (ls)
        (or (null? ls)
            (and (let notmem? ([x (car ls)] [ls (cdr ls)])
                   (or (null? ls)
                       (and (not (bound-identifier=? x (car ls)))
                            (notmem? x (cdr ls)))))
                 (unique-ids? (cdr ls))))))
    (syntax-case x ()
      [(_ ((i v) ...) e1 e2 ...)
       (unique-ids? #'(i ...))
       #'((lambda (i ...) e1 e2 ...) v ...)])))
\endschemedisplay

The argument \scheme{#'(i ...)} to \scheme{unique-ids?} is guaranteed
to be a list by the rules given in the description of \scheme{syntax}
above.

With this definition of \scheme{let}, the expression

\schemedisplay
(let ([a 3] [a 4]) (+ a a))
\endschemedisplay

causes a syntax error exception to be raised, whereas

\schemedisplay
(let-syntax ([dolet (lambda (x)
                      (syntax-case x ()
                        [(_ b)
                         #'(let ([a 3] [b 4]) (+ a b))]))])
  (dolet a))
\endschemedisplay

evaluates to \scheme{7}, since the identifier \scheme{a} introduced by \scheme{dolet}
and the identifier \scheme{a} extracted from the input form are not
\scheme{bound-identifier=?}.

The following definition of \scheme{case} is equivalent to the one in
Section~\ref{sec:syntaxcase}.
Rather than including \scheme{else} in the literals list as before,
this version explicitly tests for \scheme{else} using
\scheme{free-identifier=?}.

\schemedisplay
(define-syntax case
  (lambda (x)
    (syntax-case x ()
      [(_ e0 [(k ...) e1 e2 ...] ... [else-key else-e1 else-e2 ...])
       (and (identifier? #'else-key)
            (free-identifier=? #'else-key #'else))
       #'(let ([t e0])
           (cond
             [(memv t '(k ...)) e1 e2 ...]
             ...
             [else else-e1 else-e2 ...]))]
      [(_ e0 [(ka ...) e1a e2a ...] [(kb ...) e1b e2b ...] ...)
       #'(let ([t e0])
           (cond
             [(memv t '(ka ...)) e1a e2a ...]
             [(memv t '(kb ...)) e1b e2b ...]
             ...))])))
\endschemedisplay

With either definition of \scheme{case}, \scheme{else} is not
recognized as an auxiliary
keyword if an enclosing lexical binding for \scheme{else} exists.
For example,

\schemedisplay
(let ([else #f])
  (case 0 [else (write "oops")]))
\endschemedisplay

results in a syntax error, since \scheme{else} is bound
lexically and is
therefore not the same \scheme{else} that appears in the definition of
\scheme{case}.

\subsection{Syntax-object and datum conversions\label{sec:conversions}}

The procedure \scheme{syntax->datum}
strips all syntactic information from a syntax
object and returns the corresponding Scheme ``datum.''

\schemedisplay
(syntax->datum \var{syntax-object})
\endschemedisplay

Identifiers stripped in this manner are converted to their symbolic
names, which can then be compared with \scheme{eq?}.
Thus, a predicate \scheme{symbolic-identifier=?} might be defined as follows.

\schemedisplay
(define symbolic-identifier=?
  (lambda (x y)
    (eq? (syntax->datum x)
         (syntax->datum y))))
\endschemedisplay

% not be true with import alias and rename
%Two identifiers that are \scheme{bound-identifier=?} or
%\scheme{free-identifier=?} are \scheme{symbolic-identifier=?}; in order to
%refer to the same binding, two identifiers must have the same name.
%The converse is not always true, since two identifiers may have
%the same name but different bindings.

The procedure \scheme{datum->syntax} accepts two arguments, a
template identifier and a datum.

\schemedisplay
(datum->syntax \var{template-id} \var{datum})
\endschemedisplay

It returns a syntax object representation of \var{datum} that
contains the same contextual information as
\var{template-id}, with the effect that the
syntax object behaves
as if it were introduced into the code when
\var{template-id} was introduced.

\scheme{datum->syntax} allows a transformer to ``bend'' lexical
scoping rules by creating \emph{implicit identifiers}
that behave as if they were present in the input form,
thus permitting the definition of syntactic
abstractions that introduce visible bindings for or references to
identifiers that do not appear explicitly in the input form.
For example, the following defines a \scheme{loop} expression that
uses this controlled form of identifier capture to
bind the variable \scheme{break} to an escape procedure
within the loop body.
(The derived \scheme{with-syntax} form is like \scheme{let} but binds
pattern variables---see Section~\ref{sec:derived}.)

\schemedisplay
(define-syntax loop
  (lambda (x)
    (syntax-case x ()
      [(k e ...)
       (with-syntax ([break (datum->syntax #'k 'break)])
         #'(call-with-current-continuation
             (lambda (break)
               (let f () e ... (f)))))])))

(let ((n 3) (ls '()))
  (loop
    (if (= n 0) (break ls))
    (set! ls (cons 'a ls))
    (set! n (- n 1)))) ;=> (a a a)
\endschemedisplay

Were \scheme{loop} to be defined as

\schemedisplay
(define-syntax loop
  (lambda (x)
    (syntax-case x ()
      [(_ e ...)
       #'(call-with-current-continuation
           (lambda (break) (let f () e ... (f))))])))
\endschemedisplay

the variable \scheme{break} would not be visible in \scheme{e \dots}.

The datum argument \var{datum} may also represent an arbitrary
Scheme form, as demonstrated by the following definition of
\scheme{include}, an expand-time version of \scheme{load}.

\schemedisplay
(define-syntax include
  (lambda (x)
    (define read-file
      (lambda (fn k)
        (let ([p (open-input-file fn)])
          (let f ([x (read p)])
            (if (eof-object? x)
                (begin (close-input-port p) '())
                (cons (datum->syntax k x)
                      (f (read p))))))))
    (syntax-case x ()
      [(k filename)
       (let ([fn (syntax->datum #'filename)])
         (with-syntax ([(exp ...) (read-file fn #'k)])
           #'(begin exp ...)))])))
\endschemedisplay


\scheme{(include "filename")} expands into a \scheme{begin} expression
containing the forms found in the file named by
\scheme{"filename"}.
For example, if the file \scheme{flib.ss} contains
\scheme{(define f (lambda (x) (g (* x x))))}, and the file
\scheme{glib.ss} contains
\scheme{(define g (lambda (x) (+ x x)))},
the expression

\schemedisplay
(let ()
  (include "flib.ss")
  (include "glib.ss")
  (f 5))
\endschemedisplay

evaluates to \scheme{50}.

The definition of \scheme{include} uses \scheme{datum->syntax} to convert
the objects read from the file into syntax objects in the proper
lexical context, so that identifier references and definitions within
those expressions are scoped where the \scheme{include} form appears.

Using \scheme{datum->syntax}, it is even possible to break hygiene
entirely and write macros in the style of old Lisp macros.
The \scheme{lisp-transformer} procedure defined below creates a transformer
that converts its input into a datum, calls the programmer's procedure on
this datum, and converts the result back into a syntax object that is
scoped at top level (or, more accurately, wherever
\scheme{lisp-transformer} is defined).

\schemedisplay
(define lisp-transformer
  (lambda (p)
    (lambda (x)
      (datum->syntax #'lisp-transformer
        (p (syntax->datum x))))))
\endschemedisplay

Using \scheme{lisp-transformer}, defining a basic version of Common Lisp's
\scheme{defmacro} is a straightforward exercise.

\subsection{Generating lists of temporaries\label{sec:generatingtemporaries}}

Transformers can introduce a fixed number of identifiers into their
output simply by naming each identifier.
In some cases, however, the number of identifiers to be introduced depends
upon some characteristic of the input expression.
A straightforward definition of \scheme{letrec}, for example,
requires as many
temporary identifiers as there are binding pairs in the
input expression.
The procedure \scheme{generate-temporaries} is used to construct
lists of temporary identifiers.

\schemedisplay
(generate-temporaries \var{list})
\endschemedisplay

\var{list} may be any list or syntax object representing a list-structured
form; its contents are not important.
The number of temporaries generated is the number of elements in \var{list}.
Each temporary is guaranteed to be unique, i.e., different from all other
identifiers.

A definition of \scheme{letrec} that
uses \scheme{generate-temporaries} is shown below.

\schemedisplay\label{defn:letrec}
(define-syntax letrec
  (lambda (x)
    (syntax-case x ()
      ((_ ((i v) ...) e1 e2 ...)
       (with-syntax (((t ...) (generate-temporaries (syntax (i ...)))))
          (syntax (let ((i #f) ...)
                    (let ((t v) ...)
                      (set! i t) ...
                      (let () e1 e2 ...)))))))))
\endschemedisplay

Any transformer that uses \scheme{generate-temporaries} in this fashion can
be rewritten to avoid using it, albeit with a loss of clarity.
The trick is to use a recursively defined intermediate form that
generates one temporary per expansion step and completes the
expansion after enough temporaries have been generated.

\subsection{Derived forms and procedures\label{sec:derived}}

The forms and procedures described in this section are \emph{derived},
i.e., they can defined in terms of the forms and procedures described
in earlier sections of this document.

The {\r5rs} \scheme{syntax-rules} form is supported as a derived form,
with the following extensions:

\begin{itemize}
 \item Patterns are generalized slightly to allow a fixed number of   
   subpatterns to appear after an ellipsis, e.g.,
   \scheme{(\var{p_1} ... \var{p_2} \var{p_3})}.

 \item Underscores (~\scheme{_}~) may appear within the pattern
   and match any input, but are not pattern variables and so
   are not bound in the output \var{template}.

 \item The first position of a syntax-rules pattern may be any
   identifier, including an underscore, i.e., it need not be the name
   of the macro being defined.
   This position is always ignored.

 \item An optional fender may appear between the pattern and
   template of any clause and has the same meaning as a
   \scheme{syntax-case} fender.
\end{itemize}

A \scheme{syntax-rules} form has the syntax

\schemedisplay
(syntax-rules (\var{literal} \dots) \var{clause} \dots)
\endschemedisplay

Each \var{literal} must be an identifier.
Each \var{clause} must take one of the following two forms.

\schemedisplay
(\var{pattern} \var{template})
(\var{pattern} \var{fender} \var{template})
\endschemedisplay

Each \var{pattern} and \var{fender} are as in \scheme{syntax-case},
and each \var{template}
is as in \scheme{syntax}.
(See Section~\ref{sec:syntaxcase}.)

The definition of \scheme{or} below is like the ones given in
Section~\ref{sec:syntaxcase}, except
that \scheme{syntax-rules} is used in place of \scheme{syntax-case}
and \scheme{syntax}.

\schemedisplay
(define-syntax or
  (syntax-rules ()
    [(_) #f]
    [(_ e) e]
    [(_ e1 e2 e3 ...)
     (let ([t e1])
       (if t t (or e2 e3 ...)))]))
\endschemedisplay

The \scheme{lambda} expression used to produce the transformer is
implicit, as are the \scheme{syntax} forms used to construct the
output.

Any \scheme{syntax-rules} form can be expressed with
\scheme{syntax-case} by making the \scheme{lambda} expression and
\scheme{syntax} expressions explicit, and
\scheme{syntax-rules} may be defined in terms of \scheme{syntax-case}
as follows.

\schemedisplay
(define-syntax syntax-rules
  (lambda (x)
    (syntax-case x ()
      [(_ (k ...) [(_ . p) f ... t] ...)
       #'(lambda (x)
           (syntax-case x (k ...)
             [(_ . p) f ... #'t] ...))])))
\endschemedisplay

A more robust implementation would verify that the literals
\scheme{\var{k} \dots} are all identifiers, that the first position
of each pattern is an identifier, and that at most one fender
is present in each clause.

Since the \scheme{lambda} and \scheme{syntax} expressions are implicit in
a \scheme{syntax-rules} form, definitions expressed with
\scheme{syntax-rules} are shorter than the equivalent definitions
expressed with \scheme{syntax-case}.
The choice of which to use when either suffices is a matter of taste, but
some transformers that can be written easily with \scheme{syntax-case}
cannot be written easily or at all with \scheme{syntax-rules}.


The definitions of \scheme{p.car} in Section~\ref{sec:syntaxcase}
demonstrated how identifier macros might be written using
\scheme{syntax-case}.
Many identifier macros can be defined more succinctly using
the derived \scheme{identifier-syntax} form.
An \scheme{identifier-syntax} form has one of the following syntaxes:

\schemedisplay
(identifier-syntax \var{template})
(identifier-syntax (\var{id_1} \var{template_1}) ((set! \var{id_2} \var{pattern}) \var{template_2}))
\endschemedisplay

When a keyword is bound to a transformer produced by the first form of
\scheme{identifier-syntax}, references to the keyword within the scope
of the binding are replaced by \var{template}.

\schemedisplay
(define p (cons 4 5))
(define-syntax p.car (identifier-syntax (car p)))
p.car ;=> 4
(set! p.car 15) ;=> \var{syntax error}
\endschemedisplay

The second, more general, form of \scheme{identifier-syntax} permits
the transformer to determine what happens when \scheme{set!} is used.

\schemedisplay
(define p (cons 4 5))
(define-syntax p.car
  (identifier-syntax
    [_ (car p)]
    [(set! _ e) (set-car! p e)]))
(set! p.car 15)
p.car           ;=> 15
p               ;=> (15 5)
\endschemedisplay

\scheme{identifier-syntax} may be defined in terms of \scheme{syntax-case},
\scheme{syntax}, and \scheme{make-variable-transformer} as follows.

\schemedisplay
(define-syntax identifier-syntax
  (syntax-rules (set!)
    [(_ e)
     (lambda (x)
       (syntax-case x ()
         [id (identifier? #'id) #'e]
         [(_ x (... ...)) #'(e x (... ...))]))]
    [(_ (id exp1) ((set! var val) exp2))
     (and (identifier? #'id) (identifier? #'var))
     (make-variable-transformer
       (lambda (x)
         (syntax-case x (set!)
           [(set! var val) #'exp2]
           [(id x (... ...)) #'(exp1 x (... ...))]
           [id (identifier? #'id) #'exp1])))]))
\endschemedisplay

The derived \scheme{with-syntax} form is used to bind pattern variables,
just as \scheme{let} is used to bind variables.
This allows a transformer to construct its output in separate
pieces, then put the pieces together.

A \scheme{with-syntax} form has the following syntax.

\schemedisplay
(with-syntax ((\var{pattern} \var{expr_0}) \dots) \var{expr_1} \var{expr_2} \dots)
\endschemedisplay

Each \var{pattern} is identical in form to a \scheme{syntax-case} pattern.
The value of each \var{expr_0} is computed and destructured according
to the corresponding \var{pattern}, and pattern variables within
the \var{pattern} are bound as with \scheme{syntax-case} to the
corresponding portions of the value within
\scheme{\var{expr_1} \var{expr_2} \dots}.

\scheme{with-syntax} may be defined in terms of \scheme{syntax-case} as
follows.

\schemedisplay
(define-syntax with-syntax
  (lambda (x)
    (syntax-case x ()
      ((_ ((p e0) ...) e1 e2 ...)
       (syntax (syntax-case (list e0 ...) ()
                 ((p ...) (begin e1 e2 ...))))))))
\endschemedisplay

The following definition of \scheme{cond} demonstrates the use of
\scheme{with-syntax} to support transformers that employ recursion
internally to construct their output.
It handles all \scheme{cond} clause variations and takes care to produce
one-armed \scheme{if} expressions where appropriate.

\schemedisplay
(define-syntax cond
  (lambda (x)
    (syntax-case x ()
      [(_ c1 c2 ...)
       (let f ([c1 #'c1] [c2* #'(c2 ...)])
         (syntax-case c2* ()
           [()
            (syntax-case c1 (else =>)
              [(else e1 e2 ...) #'(begin e1 e2 ...)]
              [(e0) #'(let ([t e0]) (if t t))]
              [(e0 => e1) #'(let ([t e0]) (if t (e1 t)))]
              [(e0 e1 e2 ...) #'(if e0 (begin e1 e2 ...))])]
           [(c2 c3 ...)
            (with-syntax ([rest (f #'c2 #'(c3 ...))])
              (syntax-case c1 (=>)
                [(e0) #'(let ([t e0]) (if t t rest))]
                [(e0 => e1) #'(let ([t e0]) (if t (e1 t) rest))]
                [(e0 e1 e2 ...) #'(if e0 (begin e1 e2 ...) rest)]))]))])))
\endschemedisplay

\section{Reference Implementation\label{sec:implementation}}

A reference implementation is available in source
(\hyperlink{implementation/syntax.ss}{syntax.ss}) and expanded
(\hyperlink{implementation/syntax.pp}{syntax.pp}) forms.
The source form is the more readable of the two but uses
\scheme{syntax-case} and thus must be bootstrapped using the expanded
form.

\section{Issues\label{sec:issues}}

% \subsection{Possible additional features}

% include
% datum
% with-implicit
% syntax-error

% \subsection{Interspersed definitions and expressions}

\subsection{Library interaction}

This SRFI does not fully address the interaction between the proposed R6RS
library system and the macro system, nor does it specify the environment
in which a transformer is run.
These issues are still open to some extent, but we anticipate that the
environment in which a transformer runs will be dictated by the set of
libraries imported ``for syntax'' and possibly the ``meta level'' at
which the transformer is evaluated.

% The following is not an option: modules of transformer helpers need
% "for run-time" access to syntax-case, etc., since their run time is the
% meta time of transformer evaluation.

% It may be that \scheme{syntax-case}, \scheme{syntax-rules}, and most of
% the other features (aside from the keyword binding constructs) will be
% relegated to a module that is imported ``for syntax only'' by default so
% that they do not clutter the run-time name space.

\subsection{Name changes}

We have chosen the SRFI~72~\cite{vantonder:srfi72} names
\scheme{syntax->datum} and \scheme{datum->syntax} for the procedures that
Chez Scheme~\cite{Dybvig:csug7}, MzScheme~\cite{flatt:mzscheme}, and most
other systems call \scheme{syntax-object->datum} and
\scheme{datum->syntax-object}, because the SRFI~72 names are shorter.
While this change is incompatible with a large amount of existing code, it
is easy to identify and fix the incompatible code.

\subsection{Top-level keyword bindings}

This SRFI has nothing to say about top-level keyword bindings. 
Implementations that allow variable definitions at top level, however, are
encouraged to allow syntax definitions at top level as well.

\subsection{Expansion of internal definitions}

The operational semantics of \scheme{library} and \scheme{lambda} body
expansion has been specified more precisely than in {\r5rs}.
This operational description obviates the restriction below from
Section~5.3 of {\r5rs}:

\begin{quotation}
\noindent
Although macros may expand into definitions and syntax definitions in
any context that permits them, it is an error for a definition or syntax
definition to shadow a syntactic keyword whose meaning is needed to
determine whether some form in the group of forms that contains the
shadowing definition is in fact a definition, or, for internal definitions,
is needed to determine the boundary between the group and the expressions
that follow the group.
\end{quotation}

With the expansion algorithm described in this document, the expression

\schemedisplay
(let ([y 55] [z 73])
  (define foo (lambda (x) (set! y z)))
  (foo z)
  y)
\endschemedisplay

evaluates to \scheme{73}.
With the {\r5rs} restriction, however, it is an error, hence the
expression's behavior is unspecified, if it appears within the scope of a
keyword binding for \scheme{foo} that expands \scheme{(foo z)} into a
definition, e.g.:

\schemedisplay
(define-syntax foo
  (syntax-rules ()
    [(foo x) (define x 88)]))
\endschemedisplay

The algorithm signals an error when the forms in the expression above are
swapped, i.e.:

\schemedisplay
(let ([y 55] [z 73])
  (foo z)
  (define foo (lambda (x) (set! y z)))
  y)
\endschemedisplay

In this case, the keyword \scheme{foo} identifying the definition
\scheme{(foo z)} is redefined by the subsequent definition of
\scheme{foo}.

With internal syntax definitions, the algorithm produces similar results. 
For example, in

\schemedisplay
(let ()
  (define-syntax foo
    (syntax-rules ()
      ((foo x) (define x 37))))
  (foo a)
  a)
\endschemedisplay

the occurrence of \scheme{foo} in \scheme{(foo a)} is a reference to the
local binding of \scheme{foo}, as expected, and the expression evaluates
to \scheme{37},
whereas in

\schemedisplay
(let ()
  (foo a)
  (define-syntax foo
    (syntax-rules ()
      ((foo x) (define x 37))))
  a)
\endschemedisplay

the keyword \scheme{foo} identifying \scheme{(foo a)} as a definition
is subsequently redefined, and an error is signaled.

Specifying the expansion algorithm operationally, as we have done, may be
disquieting to some, yet we believe it is preferable to the more
declarative {\r5rs} semantics both in giving the expected meaning to
expressions like the first of each pair above and in signaling an error
for the second of each pair.

\subsection{Fluid identifiers or bindings}

Chez Scheme~\cite{Dybvig:csug7}, MzScheme~\cite{flatt:mzscheme}, and
various other systems support a \scheme{fluid-let-syntax} construct that
dynamically (at expansion time) rebinds an existing syntactic binding.
SRFI~72~\cite{vantonder:srfi72} supports a more general concept of fluid
identifiers.
Should we include either feature in {\r6rs}?

\subsection{Expand-time environment}

Chez Scheme~\cite{Dybvig:csug7} and various other systems allow arbitrary bindings to be added
to the expand-time environment and provide a mechanism for retrieving
those bindings.
Chez Scheme uses this feature, for example, to record information about
record definitions for use in subordinate record definitions.
Should we include such a feature in {\r6rs}?

\subsection{Quasisyntax}

MzScheme~\cite{flatt:mzscheme} provides \scheme{quasisyntax},
\scheme{unsyntax}, and \scheme{unsyntax-splicing} forms, analogous to
\scheme{quasiquote}, \scheme{unquote}, and \scheme{unquote-splicing}, with
the reader syntax \scheme{#`}, \scheme{#,}, and \scheme{#,@}.
SRFI~72~\cite{vantonder:srfi72} also includes \scheme{quasisyntax} but
overloads \scheme{unquote} and \scheme{unquote-splicing}, which reduces
the number of additional constructs but complicates the use of
\scheme{quasisyntax} for generating \scheme{quasiquote} expressions.
Should we include either variant in {\r6rs}?

\subsection{Fresh syntax}

SRFI~72~\cite{vantonder:srfi72} proposes that \scheme{syntax} apply a
fresh mark, so that identifiers contained within two different
\scheme{syntax} forms are not \scheme{bound-identifier=?}.
(It makes an exception, however, for identifiers that appear nested within
the same \scheme{quasisyntax} form.)
We have opted to keep the traditional semantics in which a fresh mark is
applied to all introduced portions of a transformer's output, as
described in Section~\ref{sec:hygiene}.
Ignoring the SRFI~72 \scheme{quasisyntax} exception, which muddies the
SRFI~72 semantics somewhat, both models are straightforward, logical
points in the design space.
The SRFI~72 semantics allows transformation helpers defined in
separate libraries to introduce their own unique identifier bindings.
On the other hand, the traditional semantics requires less work in the
common case where a macro and its transformation helpers are
self-contained and there is no reason to introduce two different
identifiers with the same name.
Of less concern but still relevant, the SRFI~72 semantics is also
potentially incompatible with a large amount of existing
\scheme{syntax-case} code, and identifying the affected code is not
straightforward.

This SRFI's \scheme{generate-temporaries}, while intended to generate
lists of temporaries as illustrated in the \scheme{letrec} example
of Section~\ref{sec:generatingtemporaries}, can of course be used to generate
single identifiers as well, and library helpers can use that feature
to introduce their own unique bindings if necessary.
Should we consider instead a variant of \scheme{syntax}, say
\scheme{fresh-syntax}, that applies a unique mark to its output?
Should we consider something more general, like MzScheme's
\scheme{make-syntax-introducer}~\cite{flatt:mzscheme}, which creates a
procedure that applies the same mark to a syntax object each time it is
applied?
Either can be used to define \scheme{generate-temporaries}, which can
then be considered a derived procedure.

\subsection{Degree of wrapping}

This SRFI strikes a middle ground between two extremes in the degree
to which syntax objects are wrapped.
At one extreme, syntax objects are always fully wrapped.
That is, the expander passes transformers fully wrapped input,
transformers must return fully wrapped output to the expander,
\scheme{syntax-case} must receive a fully wrapped input
value, and \scheme{syntax} forms always produce fully wrapped
output values.
Transformers and their helpers must use \scheme{syntax-case}
to destructure their input and \scheme{syntax} to produce their
output.
At the other extreme, syntax objects are mostly unwrapped: only the
identifiers occurring within a syntax object are wrapped.
Transformers and their helpers can use arbitrary list operations
to destructure their input or produce their output.

Each extreme has advantages. 
The fully wrapped representation frees an implementation to choose an
appropriate internal representation that allows the implementation to
provide more functionality and/or efficiency. 
In particular, it allows an implementation to avoid traversals of list-
and vector-structured constants to record binding information in embedded
identifiers that will end up being stripped of this information in the
end.
Because constants need not be traversed or copied, shared structure and
cycles among and within constants (more precisely, parts of the input that
will end up being constant in the final output) can be preserved for
free.
Also, again because constants need not be traversed or copied, the
expander can be written in such a way that it is linear in the size of the
input and new nodes added by transformers.
Since fully wrapped syntax objects can be destructured only via
syntax-case, which automates matching and syntax checking, it encourages a
safe, high-level style of transformer code, like syntax-rules. 

On the other hand, the unwrapped representation provides more flexibility
to the programmer, who can use familiar list-processing operations, like
map, to process portions of a transformer's input.

The flexibility of the unwrapped representation can obviously lead to an
undisciplined style in which appropriate matching is not done before a
transformer goes grabbing for a piece of the input or ignores extra pieces
of the input that should not be there. 
This is counter to one of the main goals of \scheme{syntax-case}, which is
to encourage a high-level style of transformer code, i.e., a style that is
both more readable and robust.
An important facet of this high-level style is that syntax checking is
performed automatically.
Such checking in low-level hand-written code is tedious and all too likely
to be incomplete.

Ironically, the fully wrapped representation can lead to similar problems.
With the fully wrapped representation, programmers are compelled to use
a procedure like MzScheme's \scheme{syntax->list}~\cite{flatt:mzscheme} to
convert wrapped syntax objects into lists in order to map helpers over
portions of the input.
When \scheme{syntax->list} is applied to properly matched input, there
is no problem, but \scheme{syntax->list} can also be applied to arbitrary
subforms for which no matching has been done.

The middle point described in this SRFI strikes a balance between the two
extremes, yielding the advantages of both without the shortcomings.
The \hpageref{wrappingrules}{rules for the degree to which the output of a
syntax form is unwrapped} are the key to this middle point between the two
extremes. 
They allow the programmer to use list- and vector-processing operations,
but only on portions of the transformer input that have been properly
matched. 
For example, if a transformer's input is matched using the following
\scheme{syntax-case} input pattern:

\schemedisplay
(_ ([x e] ...) b1 b2 ...)
\endschemedisplay

the corresponding output expression can treat \scheme{#'(x ...)},
\scheme{#'(e ...)}, \scheme{#'([x e] ...)}, and \scheme{#'(b1 b2 ...)} as
lists, but cannot delve into the \scheme{x}'s, \scheme{e}'s, \scheme{b1},
or \scheme{b2}'s without using syntax-case to further destructure them.

Compared with the fully wrapped extreme, the middle point eliminates the
need for potentially problematic helpers like \scheme{syntax->list}. 
Because a transformer can return partially unwrapped output, it also
eliminates the need for corresponding output constructors like
\scheme{list->syntax}. 
Compared with the mostly unwrapped extreme, the middle point ties the
implementation down less without unduly restricting the programmer from
using list- and vector-processing operations. 
It also encourages a safer, higher-level style of transformer code.

While the middle point is not as simple as either extreme and does not
provide quite as much implementation freedom as the fully wrapped extreme,
it is a natural and appropriate compromise between the two extremes.

Note:  Programmers wishing to employ the fully unwrapped representation
can define and use the following procedures that convert from the fully
wrapped to the unwrapped representation and back.

\schemedisplay
(define syntax->sexpr
  (lambda (x)
    (syntax-case x ()
      [(a . d) (cons (syntax->sexpr #'a) (syntax->sexpr #'d))]
      [#(a ...)
       (list->vector
         (map syntax->sexpr (syntax->sexpr #'(a ...))))]
      [_ (if (identifier? x) x (syntax->datum x))])))

(define sexpr->syntax
  (lambda (x)
    (cond
      [(pair? x)
       (with-syntax ([a (sexpr->syntax (car x))]
                     [d (sexpr->syntax (cdr x))]) 
         #'(a . d))]
      [(vector? x)
       (with-syntax ([(x ...) (map sexpr->syntax (vector->list x))])
         #'#(x ...))]
      [else (if (identifier? x) x (datum->syntax #'* x))])))
\endschemedisplay

A library that exports versions of \scheme{define-syntax},
\scheme{make-variable-expander}, and \scheme{syntax} to make these
operations transparent is left as an exercise for the reader.

It is cleaner and possibly much more efficient, however, for a transformer
to traverse only those parts of the input that it needs to traverse and to
treat as s-expressions only those parts that it has matched.

\section{Acknowledgments}

This SRFI was written in consultation with the full set of {\r6rs} editors: 
Will Clinger, Kent Dybvig, Matthew Flatt, Michael Sperber, and Anton
van~Straaten.

Much of this document has been copied from or adapted from Chapter~10 of the
\emph{Chez Scheme Version~7 User's Guide}~\cite{Dybvig:csug7}, some
of which also appears in Chapter~8 of \emph{The Scheme Programming
Language, 3rd edition}~\cite{Dybvig:tspl3}.

\iflatex
\def\url#1{{\let~=\schtilde\\\emph{#1}}}
\renewcommand{\bibname}{References}
\bibliographystyle{srfi}
\bibliography{syntax}
\fi
\ifhtml
\section{References}
\def\url#1{\hyperlink{#1}{Online version}}
\chapter{Primitive syntax}

After the {\cf import} form within a {\cf library} form or a top-level
program, the forms
that constitute the body of the library or the top-level program
depend on the libraries that are
imported. In particular, imported syntactic keywords determine 
the available syntactic abstractions and whether each form is a 
definition or expression. A few form types are
always available independent of imported libraries, however,
including constant literals, variable references, procedure calls,
 and macro uses.

\section{Primitive expression types}
\label{primitiveexpressionsection}

The entries in this section all describe expressions, which may occur
in the place of \hyper{expression} syntactic variables.  See
also section~\ref{expressionsection}.

\subsection*{Constant literals}\unsection

\begin{entry}{%
\pproto{\hyper{number}}{\exprtype}
\pproto{\hyper{boolean}}{\exprtype}
\pproto{\hyper{character}}{\exprtype}
\pproto{\hyper{string}}{\exprtype}
\pproto{\hyper{bytevector}}{\exprtype}}\mainindex{literal}

An expression consisting of a representation of a number object, a
boolean, a character, a string, or a bytevector, evaluates ``to
itself''.

\begin{scheme}
145932     \ev  145932
\schtrue   \ev  \schtrue
"abc"      \ev  "abc"
\#vu8(2 24 123) \ev \#vu8(2 24 123)%
\end{scheme}

As noted in section~\ref{storagemodel}, the value of a literal
expression is immutable.
\end{entry}

\subsection*{Variable references}\unsection
\begin{entry}{%
\pproto{\hyper{variable}}{\exprtype}}

An expression consisting of a variable\index{variable}
(section~\ref{variablesection}) is a variable reference if it is not a
macro use (see below).  The value of
the variable reference is the value stored in the location to which the
variable is bound.  It is a syntax violation to reference
an unbound\index{unbound} variable.

The following example examples assumes the base library
has been imported:
%
\begin{scheme}
(define x 28)
x   \ev  28%
\end{scheme}
\end{entry}

\subsection*{Procedure calls}\unsection

\begin{entry}{%
\pproto{(\hyper{operator} \hyperi{operand} \dotsfoo)}{\exprtype}}

A procedure call consists of expressions for the procedure to be
called and the arguments to be passed to it, with enclosing
parentheses.  A form in an expression context is a procedure call if
\hyper{operator} is not an identifier bound as a syntactic keyword
(see section~\ref{macrosection} below).

When a procedure call is evaluated, the operator and operand
expressions are evaluated (in an unspecified order) and the resulting
procedure is passed the resulting
arguments.\mainindex{call}\mainindex{procedure call}

The following examples assume the \rsixlibrary{base} library
has been imported:
%
\begin{scheme}%
(+ 3 4)                          \ev  7
((if \schfalse + *) 3 4)         \ev  12%
\end{scheme}
%
If the value of \hyper{operator} is not a procedure, an exception with
condition type {\cf\&assertion} is raised.  Also, if \hyper{operator}
does not accept as many arguments as there are \hyper{operand}s, an
exception with condition type {\cf\&assertion} is raised.

\begin{note} In contrast to other dialects of Lisp, the order of
evaluation is unspecified, and the operator expression and the operand
expressions are always evaluated with the same evaluation rules.

Although the order of evaluation is otherwise unspecified, the effect of
any concurrent evaluation of the operator and operand expressions is
constrained to be consistent with some sequential order of evaluation.
The order of evaluation may be chosen differently for each procedure call.
\end{note}

\begin{note} In many dialects of Lisp, the form {\tt
()} is a legitimate expression.  In Scheme, expressions written as
list/pair forms must have at
least one subexpression, so {\tt ()} is not a syntactically valid
expression.
\end{note}

\end{entry}

\section{Macros}
\label{macrosection}

Libraries and top-level programs can define and use new kinds of derived expressions and
definitions called {\em syntactic abstractions} or
{\em macros}.\mainindex{syntactic abstraction}\mainindex{macro}
A syntactic abstraction is created by binding a keyword to a
{\em macro transformer} or, simply, {\em transformer}.
\index{macro transformer}\index{transformer}
The transformer determines
how a use of the macro (called a \defining{macro use})
is transcribed into a more primitive form.

Most macro uses have the form:
\begin{scheme}
(\hyper{keyword} \hyper{datum} \dotsfoo)%
\end{scheme}%
where \hyper{keyword} is an identifier that uniquely determines the
kind of form.  This identifier is called the {\em syntactic
keyword}\index{syntactic keyword}, or simply {\em
keyword}\index{keyword}, of the macro\index{macro keyword}.
The number of \hyper{datum}s and the syntax
of each depends on the syntactic abstraction.

Macro uses can also take the form of improper lists, singleton
identifiers, or {\cf set!} forms, where the second subform of the
{\cf set!} is the keyword (see section~\ref{identifier-syntax})
library section~\extref{lib:make-variable-transformer}{{\cf make-variable-transformer}}):
\begin{scheme}
(\hyper{keyword} \hyper{datum} \dotsfoo . \hyper{datum})
\hyper{keyword}
(set! \hyper{keyword} \hyper{datum})%
\end{scheme}

The {\cf define-syntax}, {\cf let-syntax} and {\cf letrec-syntax}
forms, described in sections~\ref{define-syntax} and \ref{let-syntax},
create bindings for keywords, associate them with macro transformers,
and control the scope within which they are visible.

The {\cf syntax-rules} and {\cf identifier-syntax} forms, described in
section~\ref{syntaxrulessection}, create transformers via a pattern
language.  Moreover, the {\cf syntax-case} form, described in library
chapter~\extref{lib:syntaxcasechapter}{{\cf syntax-case}}, 
allows creating transformers via arbitrary Scheme code.

Keywords occupy the same name space as variables.
That is, within the same
scope, an identifier can be bound as a variable or keyword, or neither, but
not both, and local bindings of either kind may shadow other bindings of
either kind.

Macros defined using {\cf syntax-rules} and {\cf identifier-\hp{}syntax}
are ``hygienic'' and ``referentially transparent'' and thus preserve
Scheme's lexical scoping~\cite{Kohlbecker86,
  hygienic,Bawden88,macrosthatwork,syntacticabstraction}:
\mainindex{hygienic} \mainindex{referentially transparent}

\begin{itemize}
\item If a macro transformer inserts a binding for an identifier
(variable or keyword) not appearing in the macro use, the identifier is in effect renamed
throughout its scope to avoid conflicts with other identifiers.

\item If a macro transformer inserts a free reference to an
identifier, the reference refers to the binding that was visible
where the transformer was specified, regardless of any local
bindings that may surround the use of the macro.
\end{itemize}

Macros defined using the {\cf syntax-case} facility are also
hygienic unless {\cf datum\coerce{}syntax}
(see library section~\extref{lib:conversionssection}{Syntax-object and datum conversions}) is used.

%%% Local Variables: 
%%% mode: latex
%%% TeX-master: "r6rs"
%%% End: 

\fi

\section{Change Log}

Changes since first SRFI release.

\begin{itemize}
% Jorgen Schaefer <forcer@forcix.cx>
\item Section~\ref{sec:specification} ``singleton identifier'' sentence
      now refers to Section~\ref{sec:transformers} rather than
      Section~\ref{sec:syntaxcase}.

% Jorgen Schaefer <forcer@forcix.cx>
\item A statement about constructing residual \scheme{letrec*} forms has
      been added to the expression entry under body evaluation in
      Section~\ref{sec:expansion}.

% Arthur A. Gleckler <arthur@zurich.csail.mit.edu>
\item An errant ``s'' has been removed from ``variable references''
      in the Section~\ref{sec:expansion} that begins ``Expansion of
      each.''

% Arthur A. Gleckler <arthur@zurich.csail.mit.edu>
\item The output of the expander for the Secton~\ref{sec:expander}
      \scheme{even?}/\scheme{odd?} example has been corrected.

% Andre van Tonder <andre@now.het.brown.edu>
\item The last couple of sentences of the Section~\ref{sec:hygiene}
      paragraph that begins with ``When a substitution'' have been
      reworded.

% Arthur A. Gleckler <arthur@zurich.csail.mit.edu>
\item The word ``novector'' has been changed to ``nonvector'' in
      Section~\ref{sec:syntaxobjects}.

% Andre van Tonder <andre@now.het.brown.edu>
\item The phrase ``identifiers are equal'' has been corrected to read
      ``identifiers have the same name'' in Section~\ref{syntax-case}.

% Andre van Tonder <andre@now.het.brown.edu>
\item The description of \scheme{datum->syntax} in
      Section~\ref{sec:conversions} no longer claims that the datum
      argument is an ``arbitrary value.''
      (It should be a datum.)

% Andre van Tonder <andre@now.het.brown.edu>
\item The second occurrence of ``\var{datum}'' in the Section~\ref{sec:conversions}
      paragraph starting with ``It returns a syntax object'' has been
      corrected to read ``\var{template-id}.''

% Andre van Tonder <andre@now.het.brown.edu>
\item The Section~\ref{sec:hygiene} description description of marking
      now describes antimarks as an integral rather than optional part
      of the mechanism.

\item The text now says that an implementation ``can'' use marks and
      substitutions to avoid the appearance that their use is required.
      Any implementation would already have been free to use any
      mechanism as long as it achieves the same results.
\end{itemize}

\section{Copyright}

Copyright {\copyright} R. Kent Dybvig (2006). All Rights Reserved.

Permission is hereby granted, free of charge, to any person obtaining a
copy of this software and associated documentation files (the "Software"),
to deal in the Software without restriction, including without limitation
the rights to use, copy, modify, merge, publish, distribute, sublicense,
and/or sell copies of the Software, and to permit persons to whom the
Software is furnished to do so, subject to the following conditions:

The above copyright notice and this permission notice shall be included in
all copies or substantial portions of the Software.

THE SOFTWARE IS PROVIDED "AS IS", WITHOUT WARRANTY OF ANY KIND, EXPRESS OR
IMPLIED, INCLUDING BUT NOT LIMITED TO THE WARRANTIES OF MERCHANTABILITY,
FITNESS FOR A PARTICULAR PURPOSE AND NONINFRINGEMENT.  IN NO EVENT SHALL
THE AUTHORS OR COPYRIGHT HOLDERS BE LIABLE FOR ANY CLAIM, DAMAGES OR OTHER
LIABILITY, WHETHER IN AN ACTION OF CONTRACT, TORT OR OTHERWISE, ARISING
FROM, OUT OF OR IN CONNECTION WITH THE SOFTWARE OR THE USE OR OTHER
DEALINGS IN THE SOFTWARE. 

\end{document}

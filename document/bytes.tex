\section{Bytes objects}
\label{bytesobjects}

Many applications must deal with blocks of binary data by accessing
them in various ways---extracting signed or unsigned numbers of
various sizes.  Therefore, this library provides a single type for
blocks of binary data with multiple ways to access that data. It deals
only with integers in various sizes with specified endianness, because
these are the most frequent applications.

Bytes objects\mainschindex{bytes objects} are objects of a disjoint
type. Conceptually, a bytes object represents a sequence of 8-bit
bytes.  The description of bytes objects uses the term \defining{byte}
for an exact integer in the interval $\{-128, \ldots, 127\}$ and the
term \defining{octets} for a nexact integer in the interval $\{0,
\ldots, 255\}$.  A byte corresponds to its two's complement
representation as an octet.

The length of a bytes object is the number of bytes it contains. This
number is fixed. A valid index into a bytes object is an exact,
non-negative integer. The first byte of a bytes object has index 0,
the last byte has an index one less than the length of the bytes
object.

Generally, the access procedures come in different flavors according
to the size of the represented integer, and the endianness of the
representation.  The procedures also distinguish signed and unsigned representations.
The signed representations all use two's complement.

\begin{entry}{%
\proto{endianness}{ {\cf big}}{\exprtype}
\proto{endianness}{ {\cf little}}{\exprtype}}
   
{\cf (endianness big)} and {\cf (endianness little)} evaluate to the
symbols {\cf big} and {\cf little}, respectively. These symbols
represent an endianness, and whenever one of the procedures operating
on bytes objects accepts an endianness as an argument, that argument
must be one of these symbols. If the operarand to endianness is
anything other than big or little, an expansion-time error is
signalled. %% FIXME: error
\end{entry}

\begin{entry}{%
\proto{native-endianness}{}{procedure}}

This procedure returns the endianness of the underlying machine
architecture, either {\cf big} or {\cf little}.
\end{entry}   

\begin{entry}{%
\proto{bytes?}{ obj}{procedure}}
   
Returns \schtrue{} if \var{obj} is a bytes object, otherwise returns \schfalse{}.
\end{entry}

\begin{entry}{%
\proto{make-bytes}{ k}{procedure}
\proto{make-bytes k fill}{procedure}}
   
Returns a newly allocated bytes object of \var{k} bytes.
   
If the \var{fill} argument is missing, the initial contents of the
returned bytes object are unspecified.
   
If the \var{fill} argument is present, it must be an exact integer in
the interval $\{-128, \ldots 255\}$ that specifies the initial value
for the bytes of the bytes object: If \var{fill} is is positive, it is
interpreted as an octet, if it is negative, it is interpreted as a byte.
\end{entry}   

\begin{entry}{%
\proto{bytes-length}{ bytes}{procedure}}
   
Returns the number of bytes in bytes as an exact integer.
\end{entry}

\begin{entry}{%
\proto{bytes-u8-ref}{ bytes k}{procedure}
\proto{bytes-s8-ref}{ bytes k}{procedure}}
   
\domain{\var{K} must be a valid index of bytes.}
   
{\cf Bytes-u8-ref} returns the byte at index \var{k} of \var{bytes} as an octet.
   
{\cf Bytes-s8-ref} returns the byte at index \var{k} of \var{bytes}.

\begin{scheme}
(let ((b1 (make-bytes 16 -127))
      (b2 (make-bytes 16 255)))
  (list
    (bytes-s8-ref b1 0)
    (bytes-u8-ref b1 0)
    (bytes-s8-ref b2 0)
    (bytes-u8-ref b2 0))) \ev (-127 129 -1 255)
\end{scheme}
\end{entry}   

\begin{entry}{%
\proto{bytes-u8-set!}{ bytes k octet}{procedure}
\proto{bytes-s8-set!}{ bytes k byte}{procedure}}
   
\var{K} must be a valid index of \var{bytes}.
   
{\cf Bytes-u8-set!} stores \var{octet} in element \var{k} of
\var{bytes}.
   
{\cf Bytes-s8-set!} stores the two's complement representation of \var{byte} in element
\var{k} of \var{bytes}.
   
Both procedures return \unspecifiedreturn.

\begin{scheme}
(let ((b (make-bytes 16 -127)))

  (bytes-s8-set! b 0 -126)
  (bytes-u8-set! b 1 246)

  (list
    (bytes-s8-ref b 0)
    (bytes-u8-ref b 0)
    (bytes-s8-ref b 1)
    (bytes-u8-ref b 1))) \ev (-126 130 -10 246)
\end{scheme}
\end{entry}

\begin{entry}{%
\proto{bytes-uint-ref}{ bytes k endianness size}{procedure}
\proto{bytes-sint-ref}{ bytes k endianness size}{procedure}
\proto{bytes-uint-set!}{ bytes k n endianness size}{procedure}
\proto{bytes-sint-set!}{ bytes k n endianness size}{procedure}}
   
\domain{\var{Size} must be a positive exact integer. \var{K} must be a
  valid index of \var{bytes}; so must the indices $\{\var{k}, \ldots,
  \var{k} + \var{size} - 1\}$.}
   
{\cf Bytes-uint-ref} retrieves the exact integer corresponding to the
unsigned representation of size size and specified by endianness at
indices $\{\var{k}, \ldots, \var{k} + \var{size} - 1\}$.
   
{\cf Bytes-sint-ref} retrieves the exact integer corresponding to the two's
complement representation of size \var{size} and specified by \var{endianness} at
indices $\{\var{k}, \ldots, \var{k} + \var{size} - 1\}$.
   
\domain{For {\cf bytes-uint-set!}, \var{n} must be an exact integer in the
  set $\{0, \ldots, 256^{\var{size}}-1\}$.}

{\cf Bytes-uint-set!} stores the unsigned representation of size \var{size}
and specified by \var{endianness} into \var{bytes} at indices
$\{\var{k}, \ldots, \var{k} + \var{size} - 1\}$.
   
\domain{For {\cf bytes-uint-set!}, \var{n} must be an exact integer in
  the interval $\{-256^{\var{size}-1}, \ldots,
  256^{\var{size}-1}-1\}$.}
{\cf Bytes-sint-set!} stores the two's complement
representation of size \var{size} and specified by endianness into \var{bytes}
object at indices $\{\var{k}, \ldots, \var{k} + \var{size} - 1\}$.
   
The \ldots{\cf -set!} procedures return \unspecifiedreturn.

\begin{scheme}
(define b (make-bytes 16 -127))

(bytes-uint-set! b 0 (- (expt 2 128) 3)
                 (endianness little) 16)

(bytes-uint-ref b 0 (endianness little) 16)\lev (- (expt 2 128) 3))

(bytes-sint-ref b 0 (endianness little) 16)\lev -3

(bytes->u8-list b)\lev (253 255 255 255 255 255 255 255
               255 255 255 255 255 255 255 255)

(bytes-uint-set! b 0 (- (expt 2 128) 3)
                 (endianness big) 16)

(bytes-uint-ref b 0 (endianness big) 16) \lev (- (expt 2 128) 3)

(bytes-sint-ref b 0 (endianness big) 16) \lev -3

(bytes->u8-list b) \lev (255 255 255 255 255 255 255 255
               255 255 255 255 255 255 255 253))
\end{scheme}
\end{entry}

\begin{entry}{%
\proto{bytes-u16-ref}{ bytes k endianness}{procedure}
\proto{bytes-s16-ref}{ bytes k endianness}{procedure}
\proto{bytes-u16-native-ref}{ bytes k}{procedure}
\proto{bytes-s16-native-ref}{ bytes k}{procedure}
\proto{bytes-u16-set!}{ bytes k n endianness}{procedure}
\proto{bytes-s16-set!}{ bytes k n endianness}{procedure}
\proto{bytes-u16-native-set!}{ bytes k n}{procedure}
\proto{bytes-s16-native-set!}{ bytes k n}{procedure}}
   
\domain{\var{K} must be a valid index of \var{bytes}; so must $\var{k}
  + 1$.}
   
These retrieve and set two-byte representations of numbers at indices
\var{k} and $\var{k}+1$, according to the endianness specified by
\var{endianness}. The procedures with {\cf u16} in their names deal with the
unsigned representation, those with {cf s16} with the two's complement
representation.

The procedures with {\cf native} in their names employ the native endianness, and
only work at aligned indices: \var{k} must be a multiple of 2. It is an error to
use them at non-aligned indices.
   
The \ldots{\cf -set!} procedures return \unspecifiedreturn.

\begin{scheme}
(define b
  (u8-list->bytes
    '(255 255 255 255 255 255 255 255
      255 255 255 255 255 255 255 253)))

(bytes-u16-ref b 14 (endianness little)) \lev 65023
(bytes-s16-ref b 14 (endianness little)) \lev -513
(bytes-u16-ref b 14 (endianness big)) \lev 65533
(bytes-s16-ref b 14 (endianness big)) \lev -3

(bytes-u16-set! b 0 12345 (endianness little))
(bytes-u16-ref b 0 (endianness little)) \lev 12345

(bytes-u16-native-set! b 0 12345)
(bytes-u16-native-ref b 0) \ev 12345

(bytes-u16-ref b 0 (endianness little)) \lev \unspecified
\end{scheme}
\end{entry}

\begin{entry}{%
\proto{bytes-u32-ref}{ bytes k endianness}{procedure}
\proto{bytes-s32-ref}{ bytes k endianness}{procedure}
\proto{bytes-u32-native-ref}{ bytes k}{procedure}
\proto{bytes-s32-native-ref}{ bytes k}{procedure}
\proto{bytes-u32-set!}{ bytes k n endianness}{procedure}
\proto{bytes-s32-set!}{ bytes k n endianness}{procedure}
\proto{bytes-u32-native-set!}{ bytes k n}{procedure}
\proto{bytes-s32-native-set!}{ bytes k n}{procedure}}
   
\domain{\var{K} must be a valid index of \var{bytes}; so must the
  indices $\{\var{k}, \ldots, \var{k}+ 3\}$.}.
   
These retrieve and set four-byte representations of numbers at indices $\{\var{k},
\ldots, \var{k}+ 3\}$, according to the endianness specified by \var{endianness}. The
procedures with {\cf u32} in their names deal with the unsigned representation,
those with {\cf s32} with the two's complement representation.
   
The procedures with {\cf native} in their names employ the native endianness, and
only work at aligned indices: \var{k} must be a multiple of 4. It is an error to
use them at non-aligned indices.
   
The \ldots{\cf{}-set!} procedures return \unspecifiedreturn.

\begin{scheme}
(define b
  (u8-list->bytes
    '(255 255 255 255 255 255 255 255
      255 255 255 255 255 255 255 253)))

(bytes-u32-ref b 12 (endianness little)) \lev 4261412863
(bytes-s32-ref b 12 (endianness little)) \lev -33554433
(bytes-u32-ref b 12 (endianness big)) \lev 4294967293
(bytes-s32-ref b 12 (endianness big)) \lev -3
\end{scheme}
\end{entry}

\begin{entry}{%
\proto{bytes-u64-ref}{ bytes k endianness}{procedure}
\proto{bytes-s64-ref}{ bytes k endianness}{procedure}
\proto{bytes-u64-native-ref}{ bytes k}{procedure}
\proto{bytes-s64-native-ref}{ bytes k}{procedure}
\proto{bytes-u64-set!}{ bytes k n endianness}{procedure}
\proto{bytes-s64-set!}{ bytes k n endianness}{procedure}
\proto{bytes-u64-native-set!}{ bytes k n}{procedure}
\proto{bytes-s64-native-set!}{ bytes k n}{procedure}}
 
\domain{\var{K} must be a valid index of \var{bytes}; so must the indices
  $\{\var{k}, \ldots, \var{k}+ 7\}$.}
   
These retrieve and set eight-byte representations of numbers at
indices $\{\var{k}, \ldots, \var{k}+ 7\}$, according to the endianness
specified by \var{endianness}. The procedures with {\cf u64} in their names deal
with the unsigned representation, those with {\cf s64} with the two's
complement representation.
   
The procedures with {\cf native} in their names employ the native endianness, and
only work at aligned indices: \var{k} must be a multiple of 8. It is an error to
use them at non-aligned indices.
   
The \ldots{\cf{}-set!} procedures return \unspecifiedreturn.

\begin{scheme}
(define b
  (u8-list->bytes
    '(255 255 255 255 255 255 255 255
      255 255 255 255 255 255 255 253)))

(bytes-u64-ref b 8 (endianness little)) \lev 18302628885633695743
(bytes-s64-ref b 8 (endianness little)) \lev -144115188075855873
(bytes-u64-ref b 8 (endianness big)) \lev 18446744073709551613
(bytes-s64-ref b 8 (endianness big)) \lev -3
\end{scheme}
\end{entry}

\begin{entry}{%
\proto{bytes=?}{ bytes-1 bytes-2}{procedure}}
   
Returns \schtrue{} if \var{bytes-1} and \var{bytes-2} are equal---that
is, if they have the same length and equal bytes at all valid indices.
It returns \schfalse{} otherwise.
\end{entry}

\begin{entry}{%
\proto{bytes-copy!}{ source source-start target target-start n}{procedure}}

\domain{\var{Source-start}, \var{target-start},
  and \var{n} must be non-negative exact integers that satisfy
  
  \begin{displaymath}
    \begin{array}{rcccccl}
      0 & \leq & \var{source-start} & \leq & \var{source-start} + \var{n} & \leq & l_{\var{source}}
      \\
      0 & \leq & \var{target-start} & \leq & \var{target-start} + \var{n} & \leq & l_{\var{target}}
    \end{array}
  \end{displaymath}
  %
  where $l_{\var{source}}$ is the length of \var{source} and
  $l_{\var{target}}$ is the length of \var{target}.}
   
   
  This copies the bytes from \var{source} at indices 
  \begin{displaymath}
     \{ \var{source-start}, \ldots \var{source-start} + \var{n} - 1 \}
  \end{displaymath}
  to consecutive indices in \var{target} starting at \var{target-index}.
   
  This must work even if the memory regions for the source and the target
  overlap, i.e., the bytes at the target location after the copy must be
  equal to the bytes at the source location before the copy.
   
  This returns \unspecifiedreturn.
\begin{scheme}
(let ((b (u8-list->bytes '(1 2 3 4 5 6 7 8))))
  (bytes-copy! b 0 b 3 4)
  (bytes->u8-list b)) \ev (1 2 3 1 2 3 4 8)
\end{scheme}
\end{entry}

\begin{entry}{%
\proto{bytes-copy}{ bytes}{procedure}}
   
This returns a newly allocated copy of \var{bytes}.
\end{entry}

\begin{entry}{%
\proto{bytes->u8-list}{ bytes}{procedure}
\proto{u8-list->bytes}{ list}{procedure}}
   
\ide{Bytes->u8-list} returns a newly allocated list of the bytes of
\var{bytes} in the same order.

{\cf U8-list->bytes} returns a newly allocated bytes object whose
elements are the elements of list \var{list}, which must all be octets, in
the same order.  Analogous to {\cf list->vector}.
\end{entry}

\begin{entry}{%
\proto{bytes->uint-list}{ bytes endianness size}{procedure}
\proto{bytes->sint-list}{ bytes endianness size}{procedure}
\proto{uint-list->bytes}{ list endianness size}{procedure}
\proto{sint-list->bytes}{ list endianness size}{procedure}}
   
\domain{\var{Size} must be a positive exact integer.}
   
These convert between lists of integers and their consecutive
representations according to \var{size} and \var{endianness} in the
\var{bytes} objects in the same way as {\cf bytes->u8-list} and {\cf
  u8-list->bytes} do for one-byte representations.

\begin{scheme}
(let ((b (u8-list->bytes '(1 2 3 255 1 2 1 2))))
  (bytes->sint-list b (endianness little) 2)) \lev (513 -253 513 513)

(let ((b (u8-list->bytes '(1 2 3 255 1 2 1 2))))
  (bytes->uint-list b (endianness little) 2)) \lev (513 65283 513 513)
\end{scheme}
\end{entry}


%%% Local Variables: 
%%% mode: latex
%%% TeX-master: "r6rs"
%%% End: 

\chapter{Numbers}
\label{numbertypeschapter}
\index{number}

Numerical computation was long neglected by the Lisp
community.  Until Common Lisp there was no carefully thought out
strategy for organizing numerical computation, and with the exception of
the MacLisp system \cite{Pitman83} little effort was made to
execute numerical code efficiently.
The Scheme reports recognized the excellent work
of the Common Lisp committee and accepted many of their recommendations,
while simplifying and generalizing in some ways
consistent with the purposes of Scheme.

It is important to distinguish between the mathematical numbers, the
Scheme numbers that attempt to model them, the machine representations
used to implement the Scheme numbers, and notations used to write numbers.
This report uses the types \type{number}, \type{complex}, \type{real},
\type{rational}, and \type{integer} to refer to both mathematical numbers
and Scheme numbers.
The \type{fixnum} and \type{flonum} types refer to certain
subtypes of the Scheme numbers, as explained below.

\section{Numerical types}
\label{numericaltypes}
\index{numerical types}

\vest Mathematically, numbers may be arranged into a tower of subtypes
in which each level is a subset of the level above it:
\begin{tabbing}
\ \ \ \ \ \ \ \ \ \=\tupe{number} \\
\> \tupe{complex} \\
\> \tupe{real} \\
\> \tupe{rational} \\
\> \tupe{integer} 
\end{tabbing}

For example, 5 is an integer.  Therefore 5 is also a rational,
a real, and a complex.  The same is true of the Scheme numbers
that model 5.  For Scheme numbers, these types are defined by the
predicates \ide{number?}, \ide{complex?}, \ide{real?}, \ide{rational?},
and \ide{integer?}.

There is no simple relationship between a number's type and its
representation inside a computer.  Although most implementations of
Scheme will offer at least three different representations of 5, these
different representations denote the same integer.

Scheme's numerical operations treat numbers as abstract data, as
independent of their representation as possible.  Although an implementation
of Scheme may use many different representations for
numbers, this should not be apparent to a casual programmer writing
simple programs.

It is necessary, however, to distinguish between numbers that are
represented exactly and those that may not be.  For example, indexes
into data structures must be known exactly, as must some polynomial
coefficients in a symbolic algebra system.  On the other hand, the
results of measurements are inherently inexact, and irrational numbers
may be approximated by rational and therefore inexact approximations.
In order to catch uses of inexact numbers where exact numbers are
required, Scheme explicitly distinguishes exact from inexact numbers.
This distinction is orthogonal to the dimension of type.

A \defining{fixnum} is an exact integer whose value lies
within a certain implementation-dependent subrange of the
exact integers (section \ref{fixnumssection}).
Likewise, every implementation is required
to designate a subset of its inexact reals as \defining{flonum}s, and
to convert certain external representations into flonums.  Note that
this does not imply that an implementation is required to use
floating point representations.

\section{Exactness}
\label{exactly}

\mainindex{exactness} Scheme numbers are either \type{exact} or
\type{inexact}.  A number is exact if it was written as an exact
constant or was derived from exact numbers using only exact
operations.  A number is inexact if it was written as an inexact
constant or was derived from inexact numbers.  Thus inexactness is
contagious.  

Exact arithmetic is reliable in the following sense:
If exact numbers are passed to any of the arithmetic procedures
described in section~\ref{genericarithmeticsection}, and an
exact number is returned, then the result is mathematically
correct.
This is generally not true of
computations involving inexact numbers because approximate methods
such as floating point arithmetic may be used, but it is the duty of
each implementation to make the result as close as practical to the
mathematically ideal result.

\section{Implementation restrictions}

\index{implementation restriction}\label{restrictions}

\vest Implementations of Scheme are required to implement the whole
tower of subtypes given in section~\ref{numericaltypes}.

\vest Implementations are required to support
exact integers and exact rationals of
practically unlimited size and precision, and to implement
certain procedures (listed in \ref{propagationsection})
so they always return exact results when given exact
arguments.

\vest Implementations may also support only a limited range of
inexact numbers of
any type, subject to the requirements of this section.  For example,
an implementation may
limit the range of inexact reals (and therefore
the range of inexact integers and rationals)
to the dynamic range of the flonum format.
Furthermore
the gaps between the representable inexact integers and
rationals are
likely to be very large in such an implementation as the limits of this
range are approached.

\vest An implementation may use floating point and other approximate 
representation strategies for \tupe{inexact} numbers.
This report recommends, but does not require, that the IEEE 
floating point standards be followed by implementations that use
floating point representations, and that implementations using
other representations should match or exceed the precision achievable
using these floating point standards~\cite{IEEE}.

\vest In particular, implementations that use floating point
representations must follow these rules: A floating point result
must be represented with at least as much precision as is
used to express any of the inexact arguments to that operation.
It is desirable (but not required) for
potentially inexact operations such as {\cf sqrt}, when applied to exact
arguments, to produce exact answers whenever possible (for example the
square root of an exact 4 ought to be an exact 2).
If, however, an
exact number is operated upon so as to produce an inexact result
(as by {\cf sqrt}), and if the result is represented in floating
point, then the most precise floating point format available
must be used; but if the result
is represented in some other way then the representation must have
at least as much precision as the most precise
floating point format available.

It is the programmer's responsibility to avoid using inexact numbers
with magnitude or significand too large to be represented in the
implementation.

\section{Infinities and NaNs}

Positive infinity is regarded as a real (but not rational) number,
whose value is indeterminate but greater than all rational numbers.
Negative infinity is regarded as a real (but not rational) number,
whose value is indeterminate but less than all rational numbers.

A NaN is regarded as a real (but not rational) number whose value is
so indeterminate that it might represent any real number, including
positive or negative infinity, and might even be greater than positive
infinity or less than negative infinity.

%%% Local Variables: 
%%% mode: latex
%%% TeX-master: "r6rs"
%%% End: 

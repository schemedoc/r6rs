\chapter{Control structures}
\label{controllib}

This chapter describes the \defrsixlibrary{control} library.

\begin{entry}{%
\proto{when}{ \hyper{test} \hyperi{expression} \hyperii{expression} \dotsfoo}{\exprtype}
\proto{unless}{ \hyper{test} \hyperi{expression} \hyperii{expression} \dotsfoo}{\exprtype}}

\syntax \hyper{Test} must be an expression.

\semantics A {\cf when} expression is evaluated by evaluating the
\hyper{test} expression.  If \hyper{test} evaluates to a true value,
the remaining \hyper{expression}s are evaluated in order, and the
results of the last \hyper{expression} are returned as the
results of the entire {\cf when} expression.  Otherwise, the {\cf
	  when} expression returns \unspecifiedreturn.  An {\cf unless}
expression is evaluated by evaluating the \hyper{test} expression.
If \hyper{test} evaluates to \schfalse, the remaining
\hyper{expression}s are evaluated in order, and the results of the
last \hyper{expression} are returned as the results of the
entire {\cf unless} expression.  Otherwise, the {\cf unless} expression
returns \unspecifiedreturn.

\begin{scheme}
(when (> 3 2) 'greater) \ev greater
(when (< 3 2) 'greater) \ev \theunspecified
(unless (> 3 2) 'less) \ev \theunspecified
(unless (< 3 2) 'less) \ev less
\end{scheme}

The {\cf when} and {\cf unless} expressions are derived forms.  They
could be defined in terms of base library forms by the following macros:

\begin{scheme}
(define-syntax \ide{when}
  (syntax-rules ()
    ((when test result1 result2 ...)
     (if test
         (begin result1 result2 ...)))))

(define-syntax \ide{unless}
  (syntax-rules ()
    ((unless test result1 result2 ...)
     (if (not test)
         (begin result1 result2 ...)))))
\end{scheme}

\end{entry}

\noindent%
\pproto{(do ((\hyperi{variable} \hyperi{init} \hyperi{step})}{\exprtype}
\mainschindex{do}{\tt\obeyspaces%
     \dotsfoo)\\
    (\hyper{test} \hyper{expression} \dotsfoo)\\
  \hyper{command} \dotsfoo)}

\syntax
The \hyper{init}s, \hyper{step}s, \hyper{test}s, and \hyper{command}s must be
expressions.  The \hyper{variable}s must be pairwise distinct variables.

\semantics
The {\cf do} expression is an iteration construct.  It specifies a set of variables to
be bound, how they are to be initialized at the start, and how they are
to be updated on each iteration.

A {\cf do} expression is evaluated as follows:
The \hyper{init} expressions are evaluated (in some unspecified order),
the \hyper{variable}s are bound to fresh locations, the results of the
\hyper{init} expressions are stored in the bindings of the
\hyper{variable}s, and then the iteration phase begins.

\vest Each iteration begins by evaluating \hyper{test}; if the result is
\schfalse, then the \hyper{command}s
are evaluated in order for effect, the \hyper{step}
expressions are evaluated in some unspecified order, the
\hyper{variable}s are bound to fresh locations holding the results,
and the next iteration begins.

\vest If \hyper{test} evaluates to a true value, the
\hyper{expression}s are evaluated from left to right and the values of
the last \hyper{expression} are returned.  If no \hyper{expression}s
are present, then the values of the {\cf do} expression \areunspecified.

\vest The region\index{region} of the binding of a \hyper{variable}
consists of the entire {\cf do} expression except for the \hyper{init}s.
It is a syntax violation for a \hyper{variable} to appear more than once in the
list of {\cf do} variables.

\vest A \hyper{step} may be omitted, in which case the effect is the
same as if {\cf(\hyper{variable} \hyper{init} \hyper{variable})} had
been written instead of {\cf(\hyper{variable} \hyper{init})}.

If a {\cf do} expression appears in a tail context, the
\hyper{expression}s are a \hyper{tail sequence} in the sense of report
section~\extref{report:basetailcontextsection}{Tail calls and tail contexts},
i.e., the last \hyper{expression} is also in a tail context.

\begin{scheme}
(do ((vec (make-vector 5))
     (i 0 (+ i 1)))
    ((= i 5) vec)
  (vector-set! vec i i))          \ev  \#(0 1 2 3 4)

(let ((x '(1 3 5 7 9)))
  (do ((x x (cdr x))
       (sum 0 (+ sum (car x))))
      ((null? x) sum)))             \ev  25%
\end{scheme}

The following definition
of {\cf do} uses a trick to expand the variable clauses.

\begin{scheme}
(define-syntax \ide{do}
  (syntax-rules ()
    ((do ((var init step ...) ...)
         (test expr ...)
         command ...)
     (letrec
       ((loop
         (lambda (var ...)
           (if test
               (begin
                 \schfalse ; avoid empty begin
                 expr ...)
               (begin
                 command
                 ...
                 (loop (do "step" var step ...)
                       ...))))))
       (loop init ...)))
    ((do "step" x)
     x)
    ((do "step" x y)
     y)))%
\end{scheme}

%\end{entry}

\begin{entry}{%
\proto{case-lambda}{ \hyper{case-lambda clause} \dotsfoo}{\exprtype}}
    
\syntax
Each \hyper{case-lambda clause} must be of the form
%
\begin{scheme}
(\hyper{formals} \hyper{body})%
\end{scheme}

\hyper{Formals} must be as in a {\cf lambda} form
(report section~\extref{report:lambda}{Procedures}), and \hyper{body} is as described in
report section~\extref{report:bodiessection}{Bodies and sequences}.

\semantics A {\cf case-lambda} expression evaluates to a procedure.
This procedure, when applied, tries to match its arguments to the
\hyper{case-lambda clause}s in order.  The arguments match a clause if one of the
following conditions is fulfilled:
%
\begin{itemize}
\item \hyper{Formals} has the form {\tt (\hyper{variable} \dotsfoo)}
and the number of arguments is the same as the number of formal
parameters in \hyper{formals}.
\item \hyper{Formals} has the form\\ {\tt
(\hyperi{variable} \dotsfoo \hypern{variable} . \hyper{variable$_{n+1}$)}
}\\
and the number of arguments is at least $n$.
\item \hyper{Formals} has the form {\tt \hyper{variable}}.
\end{itemize}
%
For the first clause matched by the arguments, the variables of the
\hyper{formals} are bound to fresh locations containing the
argument values in the same arrangement as with {\cf lambda}.

If the arguments match none of the clauses, an exception with condition 
type {\cf\&assertion} is raised.

\begin{scheme}
(define foo
  (case-lambda 
   (() 'zero)
   ((x) (list 'one x))
   ((x y) (list 'two x y))
   ((a b c d . e) (list 'four a b c d e))
   (rest (list 'rest rest))))

(foo) \ev zero
(foo 1) \ev (one 1)
(foo 1 2) \ev (two 1 2)
(foo 1 2 3) \ev (rest (1 2 3))
(foo 1 2 3 4) \ev (four 1 2 3 4 ())
\end{scheme}

The {\cf case-lambda} keyword
can be defined in terms of {\cf lambda} by the following macros:
%
\begin{scheme}
(define-syntax \ide{case-lambda}
  (syntax-rules ()
    ((\_ (fmls b1 b2 ...))
     (lambda fmls b1 b2 ...))
    ((\_ (fmls b1 b2 ...) ...)
     (lambda args
       (let ((n (length args)))
         (case-lambda-help args n
           (fmls b1 b2 ...) ...))))))

(define-syntax case-lambda-help
  (syntax-rules ()
    ((\_ args n)
     (assertion-violation \schfalse{}
       "unexpected number of arguments"))
    ((\_ args n ((x ...) b1 b2 ...) more ...)
     (if (= n (length '(x ...)))
         (apply (lambda (x ...) b1 b2 ...) args)
         (case-lambda-help args n more ...)))
    ((\_ args n ((x1 x2 ... . r) b1 b2 ...) more ...)
     (if (>= n (length '(x1 x2 ...)))
         (apply (lambda (x1 x2 ... . r) b1 b2 ...)
                   args)
         (case-lambda-help args n more ...)))
    ((\_ args n (r b1 b2 ...) more ...)
     (apply (lambda r b1 b2 ...) args))))
\end{scheme}
\end{entry}


%%% Local Variables: 
%%% mode: latex
%%% TeX-master: "r6rs-lib"
%%% End: 

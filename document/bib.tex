
% My reference for proper reference format is:
%    Mary-Claire van Leunen.
%    {\em A Handbook for Scholars.}
%    Knopf, 1978.
% I think the references list would look better in ``open'' format,
% i.e. with the three blocks for each entry appearing on separate
% lines.  I used the compressed format for SIGPLAN in the interest of
% space.  In open format, when a block runs over one line,
% continuation lines should be indented; this could probably be done
% using some flavor of latex list environment.  Maybe the right thing
% to do in the long run would be to convert to Bibtex, which probably
% does the right thing, since it was implemented by one of van
% Leunen's colleagues at DEC SRC.
%  -- Jonathan

% I tried to follow Jonathan's format, insofar as I understood it.
% I tried to order entries lexicographically by authors (with singly
% authored papers first), then by date.
% In some cases I replaced a technical report or conference paper
% by a subsequent journal article, but I think there are several
% more such replacements that ought to be made.
%  -- Will, 1991.

% This is just a personal remark on your question on the RRRS:
% The language CUCH (Curry-Church) was implemented by 1964 and 
% is a practical version of the lambda-calculus (call-by-name).
% One reference you may find in Formal Language Description Languages
% for Computer Programming T.~B.~Steele, 1965 (or so).
%  -- Matthias Felleisen

% Rather than try to keep the bibliography up-to-date, which is hopeless
% given the time between updates, I replaced the bulk of the references
% with a pointer to the Scheme Repository.  Ozan Yigit's bibliography in
% the repository is a superset of the R4RS one.
% The bibliography now contains only items referenced within the report.
%  -- Richard, 1996.

\begin{thebibliography}{999}

\bibitem{SICP}
Harold Abelson and Gerald Jay Sussman with Julie Sussman.
{\em Structure and Interpretation of Computer Programs, second edition.}
MIT Press, Cambridge, 1996.

\bibitem{barendregt}H.~P.~Barendregt.
Introduction to the Lambda Calculus.
{\em Nieuw Archief voor Wisenkunde}
4 (2):337--372, 1984.

\bibitem{Bawden88} %new
Alan Bawden and Jonathan Rees.
Syntactic closures.
In {\em Proceedings of the 1988 ACM Symposium on Lisp and
  Functional Programming}, pages 86--95.

\bibitem{bawdenquasiquote}Alan Bawden.
Quasiquotation in Lisp.
In {\em Proceedings of the ACM SIGPLAN Workshop on Partial Evaluation and Semantics-Based Program Manipulation}.
Technical report BRICS-NS99 -1, University of Aarhus,
pages 4--12, 1999.

\bibitem{howtoprint}
Robert G. Burger~and R. Kent Dybvig.
Printing floating-point numbers quickly and accurately.
In {\em Proceedings of the ACM SIGPLAN '96 Conference
  on Programming Language Design and Implementation}, pages~108--116.

\bibitem{AITR633}
William Clinger.
Foundations of Actor Semantics.
MIT Artificial Intelligence Laboratory Technical Report 633, May 1981.

\bibitem{RRRS}
William Clinger, editor.
The revised revised report on Scheme, or an uncommon Lisp.
MIT Artificial Intelligence Memo 848, August 1985.
Also published as Computer Science Department Technical Report 174,
  Indiana University, June 1985.

\bibitem{howtoread} %new
William Clinger.
How to read floating point numbers accurately.
In {\em Proceedings of the ACM SIGPLAN '90 Conference
  on Programming Language Design and Implementation}, pages 92--101.
Proceedings published as {\em SIGPLAN Notices} 25(6), June 1990.

\bibitem{R4RS}
William Clinger and Jonathan Rees, editors.
The revised$^4$ report on the algorithmic language Scheme.
In {\em ACM Lisp Pointers} 4(3), pages~1--55, 1991.

\bibitem{macrosthatwork} %new
William Clinger and Jonathan Rees.
Macros that work.
In {\em Proceedings of the 1991 ACM Conference on Principles of
  Programming Languages}, pages~155--162.

\bibitem{propertailrecursion} %new
William Clinger.
Proper Tail Recursion and Space Efficiency.
To appear in {\em Proceedings of the 1998 ACM Conference on Programming
 Language Design and Implementation}, June 1998.

\bibitem{srfi76}Will Clinger, R. Kent Dybvig, Michael Sperber and Anton van Straaten.
SRFI 76: R6RS Records.
{\cf http://srfi.schemers.org/srfi-76/}, 2005.

\bibitem{srfi77}William D Clinger and Michael Sperber.
SRFI 77: Preliminary Proposal for R6RS Arithmetic.
{\cf http://srfi.schemers.org/srfi-77/}, 2005.

\bibitem{syntacticabstraction}
R.~Kent Dybvig, Robert Hieb, and Carl Bruggeman.
Syntactic abstraction in Scheme.
{\em Lisp and Symbolic Computation} 5(4):295--326, 1993.

\bibitem{tspl3}R.~Kent Dybvig
{\em The Scheme Programming Language.}
Third edition. MIT Press, Cambridge, 2003.
{\cf http://www.scheme.com/tspl3/}

\bibitem{csug7}R.~Kent Dybvig.
{\em Chez Scheme Version 7 User's Guide.}
Cadence Research Systems, 2005.
{\cf http://www.scheme.com/csug7/}

\bibitem{srfi93}R.\ Kent Dybvig
SRFI 93: R6RS {\cf syntax-case} macros.
{\cf http://srfi.schemers.org/srfi-93/}, 2006.

\bibitem{cleaninguptower}
Sebastian Egner, Richard Kelsey, and Michael Sperber.
Cleaning up the tower: Numbers in Scheme.
In {\em Proceedings of the Fifth ACM SIGPLAN
  Workshop on Scheme and Functional Programming},
pages 109--120,
September 22, 2004, Snowbird, Utah.
Technical report TR600,
{\cf http://www.cs.indiana.edu/\linebreak[1]cgi-bin/\linebreak[1]techreports/\linebreak[1]TRNNN.cgi?trnum=TR600}
Computer Science Department, Indiana University.

\bibitem{Scheme311}
Carol Fessenden, William Clinger, Daniel P.~Friedman, and Christopher Haynes.
Scheme 311 version 4 reference manual.
Indiana University Computer Science Technical Report 137, February 1983.
Superseded by~\cite{Scheme84}.

\bibitem{mzscheme301}Matthew Flatt.
{\em PLT MzScheme: Language Manual, No.~301.}
2006.
{\cf http://download.plt-scheme.org/\linebreak[1]doc/\linebreak[1]301/\linebreak[1]html/\linebreak[1]mzscheme/}

\bibitem{srfi83}Matthew Flatt and Kent Dybvig
SRFI 83: R6RS Library Syntax.
{\cf http://srfi.schemers.org/\linebreak[1]srfi-83/}, 2005.
 
\bibitem{srfi75}Matthew Flatt and Mark Feeley.
SRFI 75: R6RS Unicode data.
{\cf http://srfi.schemers.org/\linebreak[1]srfi-75/}, 2005.

\bibitem{Scheme84}
D.~Friedman, C.~Haynes, E.~Kohlbecker, and M.~Wand.
Scheme 84 interim reference manual.
Indiana University Computer Science Technical Report 153, January 1985.

\bibitem{srfi22}Martin Gasbichler and Michael Sperber
SRFI 22: Running Scheme Scripts on Unix.
{\cf http://srfi.schemers.org/srfi-22/}, 2002.

\bibitem{srfi11}Lars T Hansen.
SRFI 11: Syntax for receiving multiple values.
{\cf http://srfi.schemers.org/srfi-11/}, 2000.

\bibitem{IEEE}
{\em IEEE Standard 754-1985.  IEEE Standard for Binary Floating-Point
Arithmetic.}  IEEE, New York, 1985.

\bibitem{IEEEScheme}
{\em IEEE Standard 1178-1990.  IEEE Standard for the Scheme
  Programming Language.}  IEEE, New York, 1991.

\bibitem{R5RS}
Richard Kelsey, William Clinger and Jonathan Rees, editors.
The revised$^5$ report on the algorithmic language Scheme.
In {\em Higher-Order and Symbolic Computation} 11(1), pages~7--105, 1998.

\bibitem{srfi34}Richard Kelsey and Michael Sperber
SRFI 34: Exception Handling for Programs.
{\cf http://srfi.\linebreak[1]schemers.\linebreak[1]org/\linebreak[1]srfi-34/}, 2002.

\bibitem{srfi35}Richard Kelsey and Michael Sperber
SRFI 35: Conditions.
{\cf http://srfi.schemers.org/srfi-35/}, 2002.

\bibitem{Kohlbecker86}
Eugene E. Kohlbecker~Jr.
{\em Syntactic Extensions in the Programming Language Lisp.}
PhD thesis, Indiana University, August 1986.

\bibitem{hygienic}
Eugene E.~Kohlbecker~Jr., Daniel P.~Friedman, Matthias Felleisen, and Bruce Duba.
Hygienic macro expansion.
In {\em Proceedings of the 1986 ACM Conference on Lisp
  and Functional Programming}, pages 151--161.

\bibitem{Landin65}
Peter Landin.
A correspondence between Algol 60 and Church's lambda notation: Part I.
{\em Communications of the ACM} 8(2):89--101, February 1965.

\bibitem{PLTRedex}Jacob Matthews, Robert Bruce Findler, Matthew
  Flatt, and Matthias Felleisen.
A Visual Environment for Developing Context-Sensitive Term Rewriting Systems.
In {\em International Conference on Rewriting Techniques and Applications (RTA2004).}

\bibitem{R5RSSemantics}Jacob Matthews and Robert Bruce Findler.
An Operational Semantics for R5RS Scheme.
In {\em 2005 Workshop on Scheme and Functional Programming}.
September 2005.

\bibitem{MITScheme}
MIT Department of Electrical Engineering and Computer Science.
Scheme manual, seventh edition.
September 1984.

\bibitem{Naur63}
Peter Naur et al.
Revised report on the algorithmic language Algol 60.
{\em Communications of the ACM} 6(1):1--17, January 1963.

\bibitem{Penfield81}
Paul Penfield, Jr.
Principal values and branch cuts in complex APL.
In {\em APL '81 Conference Proceedings,} pages 248--256.
ACM SIGAPL, San Francisco, September 1981.
Proceedings published as {\em APL Quote Quad} 12(1), ACM, September 1981.

\bibitem{Pitman83}
Kent M.~Pitman.
The revised MacLisp manual (Saturday evening edition).
MIT Laboratory for Computer Science Technical Report 295, May 1983.

\bibitem{Rees82}
Jonathan A.~Rees and Norman I.~Adams IV.
T: A dialect of Lisp or, lambda: The ultimate software tool.
In {\em Conference Record of the 1982 ACM Symposium on Lisp and
  Functional Programming}, pages 114--122.

\bibitem{Rees84}
Jonathan A.~Rees, Norman I.~Adams IV, and James R.~Meehan.
The T manual, fourth edition.
Yale University Computer Science Department, January 1984.

\bibitem{R3RS}
Jonathan Rees and William Clinger, editors.
The revised$^3$ report on the algorithmic language Scheme.
In {\em ACM SIGPLAN Notices} 21(12), pages~37--79, December 1986.

\bibitem{Reynolds72}
John Reynolds.
Definitional interpreters for higher order programming languages.
In {\em ACM Conference Proceedings}, pages 717--740.
ACM, \todo{month?}~1972.

\bibitem{Scheme78}
Guy Lewis Steele Jr.~and Gerald Jay Sussman.
The revised report on Scheme, a dialect of Lisp.
MIT Artificial Intelligence Memo 452, January 1978.

\bibitem{Rabbit}
Guy Lewis Steele Jr.
Rabbit: a compiler for Scheme.
MIT Artificial Intelligence Laboratory Technical Report 474, May 1978.

\bibitem{CLtL}
Guy Lewis Steele Jr.
{\em Common Lisp: The Language, second edition.}
Digital Press, Burlington MA, 1990.

\bibitem{Scheme75}
Gerald Jay Sussman and Guy Lewis Steele Jr.
Scheme: an interpreter for extended lambda calculus.
MIT Artificial Intelligence Memo 349, December 1975.

\bibitem{SchemeCharter2006}
{\em Scheme Standardization charter.}
{\tt
  http://www.\linebreak[1]schemers.\linebreak[1]org/\linebreak[1]Documents/\linebreak[1]Standards/\linebreak[1]Charter/\linebreak[1]mar-2006.txt},
March 2006.

\bibitem{Stoy77}
Joseph E.~Stoy.
{\em Denotational Semantics: The Scott-Strachey Approach to
  Programming Language Theory.}
MIT Press, Cambridge, 1977.

\bibitem{TImanual85}
Texas Instruments, Inc.
TI Scheme Language Reference Manual.
Preliminary version 1.0, November 1985. 

\bibitem{Unicode}
The Unicode Consortium.
{\em  The Unicode Standard, Version 5.0.0},
defined by: {\em The Unicode Standard, Version 5.0} (Boston, MA,
 Addison-Wesley, 2007. ISBN 0-321-48091-0).

\bibitem{Waddellphd}Oscar Waddell.
{\em Extending the Scope of Syntactic Extension.}
PhD thesis, Indiana University, August 1999.

\bibitem{WaiteGoos}William M.~Waite and Gerhard Goos.
{\em Compiler Construction.}
Springer-Verlag, New York, 1984.

\end{thebibliography}

%%% Local Variables: 
%%% mode: latex
%%% TeX-master: "r6rs"
%%% End: 

\section{\tt{Eval}}

The \library{r6rs eval} library allows a program to create Scheme
expressions as data at run time and evaluate them.

\begin{entry}{%
\proto{eval}{ expression environment-specifier}{procedure}}

Evaluates \var{expression} in the specified environment and returns its value.
\var{Expression} must be a valid Scheme expression represented as a
datum value, and \var{environment-specifier} must be a 
\defining{library specifier}, which can be created using the {\cf
  environment} procedure described below.

If the first argument to {\cf eval} is not a syntactically correct
expression, then {\cf eval} must raise an exception with condition
type {\cf \&syntax}.  Specifically, if the first argument to {\cf
  eval} is a definition, it must raise an exception with condition
type {\cf \&eval-definition}.
\end{entry}

\begin{entry}{%
\proto{environment}{ import-spec \dots}{procedure}}

\domain{\var{Import-spec} shall be a datum representing an
  \hyper{import spec} (see section~\ref{librarysyntaxsection}).}
The {\cf environment} procedure returns an environment corresponding
to \var{import-spec}

The bindings of the environment represented by the specifier are
immutable: If {\cf eval} is applied to an expression that attempts to
assign to one of the variables of the environment, {\cf eval} must
raise an exception with a condition type {\cf\&contract}.

\begin{scheme}
(library (foo)
  (export)
  (import (r6rs))
  (write (eval '(let ((x 3)) x) (environment '(r6rs))) \\\> {\it writes} 3

(library foo
  (export)
  (import (r6rs)
  (write
    (eval
      '(eval:car (eval:cons 2 4))
      '(add-prefix (only (r6rs) car cdr cons null?)
                   eval:)))) \\\> {\it writes} 2
\end{scheme}
\end{entry}

%%% Local Variables: 
%%% mode: latex
%%% TeX-master: "r6rs"
%%% End: 

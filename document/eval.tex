\chapter{\tt{eval}}
\label{evalchapter}

The \rsixlibrary{eval} library allows a program to create Scheme
expressions as data at run time and evaluate them.

\begin{entry}{%
\proto{eval}{ expression environment}{procedure}}

Evaluates \var{expression} in the specified environment and returns its value.
\var{Expression} must be a syntactically valid Scheme expression represented as a
datum value, and \var{environment} must be an
\defining{environment}, which can be created using the {\cf
  environment} procedure described below.

If the first argument to {\cf eval} is determined not to be a syntactically correct
expression, then {\cf eval} must raise an exception with condition
type {\cf \&syntax}.  Specifically, if the first argument to {\cf
  eval} is a definition or a splicing {\cf begin} form containing a
definition, it must raise an exception with condition type {\cf
  \&syntax}.
\end{entry}

\begin{entry}{%
\proto{environment}{ import-spec \dots}{procedure}}

\domain{\var{Import-spec} must be a datum representing an
  \hyper{import spec} (see report
  section~\extref{report:librarysyntaxsection}{Library form}).}
The {\cf environment} procedure returns an environment corresponding
to \var{import-spec}.

The bindings of the environment represented by the specifier are
immutable: If {\cf eval} is applied to an expression that is
determined to contain an
assignment to one of the variables of the environment, then {\cf eval} must
raise an exception with a condition type {\cf\&syntax}.

\begin{scheme}
(library (foo)
  (export)
  (import (rnrs)
          (rnrs eval))
  (write
    (eval '(let ((x 3)) x)
          (environment '(rnrs))))) \\\> {\it writes} 3

(library (foo)
  (export)
  (import (rnrs)
          (rnrs eval))
  (write
    (eval
      '(eval:car (eval:cons 2 4))
      (environment
        '(prefix (only (rnrs) car cdr cons null?)
                 eval:))))) \\\> {\it writes} 2%
\end{scheme}
\end{entry}

%%% Local Variables: 
%%% mode: latex
%%% TeX-master: "r6rs-lib"
%%% End: 

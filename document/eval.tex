\section{\tt{Eval}}

FIXME: name of the library, condition types, library syntax

The {\cf eval} library allows a program to create Scheme expressions
as data at run time and evaluate them.

\begin{entry}{%
\proto{eval}{ expression environment-specifier}{procedure}}

Evaluates \var{expression} in the specified environment and returns its value.
\var{Expression} must be a valid Scheme expression represented as data,
and \var{environment-specifier} must be a 
\defining{library specifier}, which can be created using the {\cf
  the-library-environment} syntactic form described below.

If the first argument to {\cf eval} is not a syntactically correct
expression, then {\cf eval} must raise an exception with condition
type {\cf \&syntax}.  Specifically, if the first argument to {\cf
  eval} is a definition, it must raise an exception with condition
type {\cf \&eval-definition}.
\end{entry}

\begin{entry}{%
\proto{the-library-environment}{}{\exprtype}}

The {\cf (the-library-environment)} form evaluates to the specifier of
an environment containing the set of bindings imported {\cf for eval}
into the library it appears in.

If {\cf (the-library-environment)} appears within an expression passed
to {\cf eval}, it evaluates to the same library that was passed to
{\cf eval}.

The bindings of the environment represented by the specifier are
immutable: If {\cf eval} is applied to an expression that attempts to
assign to one of the variables of the environment, {\cf eval} must
raise an exception with a FIXME condition type.

For details on the semantics of invoking and visiting a library {\cf
  for eval}, refer to chapter~\ref{libraries}.

FIXME: write

\begin{scheme}
(library foo
  (import (for r6rs run eval))
  (write (eval '(let ((x 3)) x)))) \\\> {\it writes} 3

(library foo
  (import
    (for r6rs expand run)
    (for (add-prefix (only r6rs car cdr cons null?)
                     "eval:")
         eval))
  (write
    (eval
      '(eval:car (eval:cons 'foo 'bar))))) \\\> {\it writes} foo
\end{scheme}
\end{entry}

%%% Local Variables: 
%%% mode: latex
%%% TeX-master: "r6rs"
%%% End: 

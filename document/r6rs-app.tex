 \documentclass[twoside,twocolumn]{algol60}
%\documentclass[twoside]{algol60}

\pagestyle{headings} 
\showboxdepth=0
%\makeindex
% Macros for R^nRS.

\usepackage{makeidx}
\usepackage{url}

% tex2page.sty mucks with in some manner
\let\centerlinesaved=\centerline
\usepackage{tex2page}
\let\centerline=\centerlinesaved

% \let\htmlonly=\iffalse
% \let\endhtmlonly=\fi
% \let\texonly=\iftrue
% \let\endtexonly=\fi

\makeatletter

\texonly
\newcommand{\topnewpage}{\@topnewpage}
\endtexonly

\htmlonly
\newcommand{\topnewpage}[0][]{#1}
\endhtmlonly

\newcommand{\authorsc}[1]{{\scriptsize\scshape #1}}

% Chapters, sections, etc.

\newcommand{\extrapart}[1]{
 % \chapter{#1}
  \chapter*{#1}
  \markboth{#1}{#1}
  \vskip 1ex
  \addcontentsline{toc}{chapter}{#1}}

\newcommand{\clearchaptergroupstar}[1]{
  \texonly
  \clearpage
  \addcontentsline{toc}{chaptergroup}{#1}
  \topnewpage[
    \centerline{\large\bf\uppercase{#1}}
    \bigskip]
    \endtexonly
  }

\newcommand{\clearchapterstar}[1]{
  \clearpage
  \topnewpage[
    \centerline{\large\bf\uppercase{#1}}
    \bigskip]}

\newcommand{\clearextrapart}[1]{
  \clearchapterstar{#1}
  \markboth{#1}{#1}
  \addcontentsline{toc}{chapter}{#1}}

\newcommand{\vest}{}
\newcommand{\dotsfoo}{$\ldots\,$}

\newcommand{\sharpfoo}[1]{{\tt\##1}}
\newcommand{\schfalse}{\sharpfoo{f}}
\newcommand{\schtrue}{\sharpfoo{t}}

\newcommand{\ampfoo}[1]{{\tt\&#1}}

\newcommand{\libfoo}[1]{{\tt(#1)}}

\newcommand{\singlequote}{{\tt'}}  %\char19
\newcommand{\doublequote}{{\tt"}}
\newcommand{\backquote}{{\tt\char18}}
\newcommand{\backwhack}{{\tt\char`\\}}
\newcommand{\comma}{{\tt\char`\@}}
\newcommand{\atsign}{{\tt\char`\@}}
\newcommand{\bang}{{\tt\char`\!}}
\newcommand{\sharpsign}{{\tt\#}}
\newcommand{\verticalbar}{{\tt|}}
\newcommand{\openbracket}{{\tt\char`\[}}
\newcommand{\closedbracket}{{\tt\char`\]}}
\newcommand{\ampersand}{{\tt\char`\&}}

\newcommand{\coerce}{\discretionary{->}{}{->}}

% Knuth's \in sucks big boulders
\def\elem{\hbox{\raise.13ex\hbox{$\scriptstyle\in$}}}

\newcommand{\meta}[1]{{\noindent\hbox{\rm$\langle$#1$\rangle$}}}
\let\hyper=\meta
\newcommand{\hyperi}[1]{\hyper{#1$_1$}}
\newcommand{\hyperii}[1]{\hyper{#1$_2$}}
\newcommand{\hyperiii}[1]{\hyper{#1$_3$}}
\newcommand{\hyperj}[1]{\hyper{#1$_i$}}
\newcommand{\hypern}[1]{\hyper{#1$_n$}}
\texonly
\newcommand{\var}[1]{\noindent\hbox{\textnormal{\textit{#1}}}}
\endtexonly
\htmlonly
\newcommand{\var}[1]{\textnormal{\textit{#1}}}
\endhtmlonly
\newcommand{\vari}[1]{\var{#1$_1$}}
\newcommand{\varii}[1]{\var{#1$_2$}}
\newcommand{\variii}[1]{\var{#1$_3$}}
\newcommand{\variv}[1]{\var{#1$_4$}}
\newcommand{\varj}[1]{\var{#1$_j$}}
\newcommand{\varn}[1]{\var{#1$_n$}}

\newcommand{\vr}[1]{{\noindent\hbox{$#1$\/}}}  % Careful, is \/ always the right thing?
\newcommand{\vri}[1]{\vr{#1_1}}
\newcommand{\vrii}[1]{\vr{#1_2}}
\newcommand{\vriii}[1]{\vr{#1_3}}
\newcommand{\vriv}[1]{\vr{#1_4}}
\newcommand{\vrv}[1]{\vr{#1_5}}
\newcommand{\vrj}[1]{\vr{#1_j}}
\newcommand{\vrn}[1]{\vr{#1_n}}

%%R4%% The excessive use of the code font in the numbers section was
% confusing, somewhat obnoxious, and inconsistent with the rest
% of the report and with parts of the section itself.  I added
% a \tupe no-op, and changed most old uses of \type to \tupe,
% to make it easier to change the fonts back if people object
% to the change.

\newcommand{\type}[1]{{\it#1}}
\newcommand{\tupe}[1]{{#1}}

\newcommand{\defining}[1]{\mainindex{#1}{\em #1}}
\newcommand{\ide}[1]{{\schindex{#1}\frenchspacing\tt{#1}}}

\newcommand{\lambdaexp}{{\cf lambda} expression}

\newcommand{\callcc}{{\tt call-with-current-continuation}}

\newcommand{\mainschindex}[1]{\label{#1}\index{#1@\texttt{#1}}}
\newcommand{\mainindex}[1]{\index{#1}}
\newcommand{\schindex}[1]{\index{#1@\texttt{#1}}}
\newcommand{\sharpindex}[1]{\index{#1@\texttt{\#{}#1}}}
\newcommand{\ampindex}[1]{\index{#1@\texttt{\&{}#1}}}
\newcommand{\libindex}[1]{\index{#1@\texttt{(#1)}}}

\newcommand{\extref}[2]{\texonly\ref{#1}\endtexonly\htmlonly{on ``#2''}\endhtmlonly}

\renewenvironment{theindex}
{\clearpage
\topnewpage[
    \begin{center}
      \large\bf\MakeUppercase{\indexheading}
    \end{center}
    \vskip 1ex \bigskip]
    \markboth{Index}{Index}
    \addcontentsline{toc}{chapter}{\indexheading}
    \parindent\z@
    \texonly\parskip\z@ plus .1pt\endtexonly\relax\let\item\@idxitem
    \indexintro\par\bigskip}
               {\texonly\clearpage\endtexonly}


\newcommand{\domain}[1]{#1}
\newcommand{\nodomain}[1]{}
%\newcommand{\todo}[1]{{\rm$[\![$!!~#1$]\!]$}}
\newcommand{\todo}[1]{}

% \frobq will make quote and backquote look nicer.
\def\frobqcats{%\catcode`\'=13
\catcode`\`=13{}}
{\frobqcats
\gdef\frobqdefs{%\def'{\singlequote}
\def`{\backquote}}}
\def\frobq{\frobqcats\frobqdefs}

% \cf = code font
% Unfortunately, \cf \cf won't work at all, so don't even attempt to
% next constructions which use them...
\newcommand{\cf}{\frenchspacing\frobq\tt}

\texonly
% Same as \obeycr, but doesn't do a \@gobblecr.
{\catcode`\^^M=13 \gdef\myobeycr{\catcode`\^^M=13 \def^^M{\\}}%
\gdef\restorecr{\catcode`\^^M=5 }}
\endtexonly

{\obeyspaces\gdef {\hbox{\hskip0.5em}}}

\gdef\gobblecr{\@gobblecr}

\def\setupcode{\@makeother\^}

% Scheme example environment
% At 11 points, one column, these are about 56 characters wide.
% That's 32 characters to the left of the => and about 20 to the right.

\newcommand{\exception}[1]{{\cf#1} \textnormal{\textit{exception}}}
\newenvironment{schemenoindent}{
  % Commands for scheme examples
  \newcommand{\ev}{\>\>\evalsto}
  \newcommand{\lev}{\\\>\evalsto}
  \newcommand{\unspecified}{{\em{}unspecified}}
  \newcommand{\theunspecified}{{\em{}the unspecified value}}
  \setupcode
  \small \cf \obeyspaces \myobeycr
  \begin{tabbing}%
\qquad\=\hspace*{5em}\=\hspace*{9em}\=\evalsto~\=\kill%   was 16em
\gobblecr}{\unskip\end{tabbing}}

%\newenvironment{scheme}{\begin{schemenoindent}\+\kill}{\end{schemenoindent}}
\newenvironment{scheme}{
  % Commands for scheme examples
  \newcommand{\ev}{\>\>\evalsto}
  \newcommand{\lev}{\\\>\evalsto}
  \renewcommand{\em}{\rmfamily\itshape}
  \newcommand{\unspecified}{{\em{}unspecified}}
  \newcommand{\theunspecified}{{\em{}the unspecified value}}
  \setupcode
  \small \cf \obeyspaces \myobeycr
  \begin{tabbing}%
\qquad\=\hspace*{5em}\=\hspace*{9em}\=\evalsto~\=\+\kill%   was 16em
\gobblecr}{\unskip\end{tabbing}}

\newcommand{\evalsto}{$\Longrightarrow$}

% Rationale

\newenvironment{rationale}{%
\bgroup\small\noindent{\em Rationale:}\space}{%
\egroup}

% Notes

\newenvironment{note}{%
\bgroup\small\noindent{\em Note:}\space}{%
\egroup}

% Names of library modules

\newcommand{\library}[1]{{\tt (#1)}}
\newcommand{\deflibrary}[1]{\library{#1}\libindex{#1}}

% Manual entries

\newenvironment{entry}[1]{
  \vspace{3.1ex plus .5ex minus .3ex}\noindent#1%
\unpenalty\nopagebreak}{\vspace{0ex plus 1ex minus 1ex}}

\newcommand{\exprtype}{syntax}

\newcommand{\unspecifiedreturn}{the unspecified value}

% Primitive prototype
\newcommand{\pproto}[2]{\unskip%
\hbox{\cf\spaceskip=0.5em#1}\hfill\penalty 0%
\hbox{ }\nobreak\hfill\hbox{\rm #2}\break}

% Parenthesized prototype
\newcommand{\proto}[3]{\pproto{(\mainschindex{#1}\hbox{#1}{\it#2\/})}{#3}}

% Variable prototype
\newcommand{\vproto}[2]{\mainschindex{#1}\pproto{#1}{#2}}

% Condition-type prototype
 \newcommand{\ctproto}[1]{\ampindex{#1}\pproto{\ampfoo{#1}}{condition type}}

% Extending an existing definition (\proto without the index entry)
\newcommand{\rproto}[3]{\pproto{(\hbox{#1}{\it#2\/})}{#3}}

% Extending an existing definition, with index entry
\newcommand{\irproto}[3]{\schindex{#1}\rproto{#1}{#2}{#3}}

% Variable prototype
\newcommand{\rvproto}[2]{\pproto{#1}{#2}}

% Grammar environment

\newenvironment{grammar}{
  \def\:{\goesto{}}
  \def\|{$\vert$}
  \cf \myobeycr
  \begin{tabbing}
    %\qquad\quad \= 
    \qquad \= $\vert$ \= \kill
  }{\unskip\end{tabbing}}

%\newcommand{\unsection}{\unskip}
\newcommand{\unsection}{{\vskip -2ex}}

% Commands for grammars
\newcommand{\arbno}[1]{#1\hbox{\rm*}}  
\newcommand{\atleastone}[1]{#1\hbox{$^+$}}

\newcommand{\goesto}{$\longrightarrow$}

\newcommand{\syntax}{{\em Syntax: }}
\newcommand{\semantics}{{\em Semantics: }}
\newcommand{\implresp}{{\em Implementation responsibilities: }}

\newcommand{\rrs}[1]{\textit{Revised$^#1$ Report on the Algorithmic Language Scheme}}

\newcommand{\libindexentry}[1]{#1 (library)}

\makeatother


\texonly
\externaldocument[report:]{r6rs}
\externaldocument[lib:]{r6rs-lib}
\endtexonly

\def\headertitle{Revised$^{\rnrsrevision}$ Scheme Non-Normative Appendices (latest erratum: \rnrserratadate)}
\def\TZPtitle{Revised^\rnrsrevision{} Report on the Algorithmic Language Scheme - Non-Normative Appendices -}

\begin{document}

\thispagestyle{empty}

\topnewpage[{
\begin{center}   {\huge\bf
        Revised{\Huge$^{\mathbf{\htmlonly\tiny\endhtmlonly{}\rnrsrevision}}$} Report on the Algorithmic Language \\
                              \vskip 3pt
                              Scheme\\
                                \vskip 1.5ex
                              --- Non-Normative Appendices ---}

\vskip 1ex
$$
\begin{tabular}{l@{\extracolsep{.5in}}lll}
\multicolumn{4}{c}{M\authorsc{ICHAEL} S\authorsc{PERBER}}
\\
\multicolumn{4}{c}{R.\ K\authorsc{ENT} D\authorsc{YBVIG},
  M\authorsc{ATTHEW} F\authorsc{LATT},
  A\authorsc{NTON} \authorsc{VAN} S\authorsc{TRAATEN}}
\\
\multicolumn{4}{c}{(\textit{Editors})} \\[1ex]
\multicolumn{4}{c}{\bf\cancel \rnrsrevisiondate} \\
\multicolumn{4}{c}{\Huge{**Unofficial version incorporating errata**}} \\
\multicolumn{4}{c}{\url{https://weinholt.se/scheme/r6rs/}} \\
\multicolumn{4}{c}{\bf \rnrserratadate}
\end{tabular}
$$



\end{center}

\chapter*{Summary}
\medskip

{\parskip 1ex
This document contains non-normative appendices to the
\textit{Revised\itspace{}$^6$ Report on the Algorithmic Language Scheme}.  These
appendices contain advice for users and suggestions for implementors
on issues not fit for standardization, in particular on platform-specific
issues.

This document frequently refers back to the \textit{Revised\itspace{}$^6$ Report
  on the Algorithmic Language Scheme}~\cite{R6RS} and the
\textit{Revised\itspace{}$^6$ Report on the Algorithmic Language Scheme ---
  Libraries ---}~\cite{R6RS-libraries}; references to the report are
identified by designations such as ``report section'' or ``report
chapter'', and references to the library report are identified by
designations such as ``library section'' or ``library chapter''.
}

\medskip

We intend this report to belong to the entire Scheme community, and so
we grant permission to copy it in whole or in part without fee.  In
particular, we encourage implementors of Scheme to use this report as
a starting point for manuals and other documentation, modifying it as
necessary.
% We're done.
}]

\texonly\clearpage\endtexonly

\chapter*{Contents}
\addvspace{3.5pt}                  % don't shrink this gap
\renewcommand{\tocshrink}{-4.0pt}  % value determined experimentally
{%\footnotesize
\tableofcontents
}

\vfill

\texonly\clearpage\endtexonly

\appendix

\chapter{Standard-conformant mode}
\label{standardconformantmode}

Scheme implementations compliant with the report may operate in a
variety of modes.  In particular, in addition to one or more modes
conformant with the requirements of the report, an implementation
may offer non-conformant modes.  These modes are by nature
implementation-specific, and may differ in the language and available
libraries.  In particular, non-conformant language extensions may be
available, including unsafe libraries or otherwise unsafe features,
and the semantics of the language may differ from the semantics
described in the report.  Moreover, the default mode offered by a
Scheme implementation may be non-conformant, and such a Scheme
implementation may require special settings or declarations to enter
the report-conformant mode.  Implementors should clearly
document the nature of the default mode and how to enter a
report-conformant mode.

\chapter{Optional case insensitivity}
\label{caseinsensitivityappendix}

In contrast with earlier revisions of the report~\cite{R5RS}, the
syntax of data distinguishes upper and lower case in identifiers and in
characters specified via their names.  For example, the identifiers
{\cf X} and {\cf x} are different, and the character
{\cf\sharpsign\backwhack{}space} cannot be written
{\cf\sharpsign\backwhack{}SPACE}.

Implementors may wish to support case-insensitive syntax for backward
compatibility or other reasons.
If they do so, they should adopt the following directives
to control case folding.

\begin{entry}{%
{\cf{}\#!fold-case}\sharpbangindex{fold-case}\\
{\cf{}\#!no-fold-case}\sharpbangindex{no-fold-case}}

These directives may appear anywhere comments may appear and are
treated as comments, except that they affect the reading of subsequent
lexemes.
The {\cf{}\#!fold-case} causes the reader to case-fold
(see library section~\extref{lib:string-foldcase}{``Unicode''})
each \meta{identifier} and \meta{character name}.
The {\cf{}\#!no-fold-case} directive causes the reader to return
to the default, non-folding behavior.
\end{entry}

\chapter{Use of square brackets}

Even though matched square brackets are synonymous with parentheses in
the syntax, many programmers use square brackets only in a few select
places.  In particular, programmers should restrict use of
square brackets to places in syntactic forms where two consecutive
open parentheses would otherwise be common.  These are the applicable
forms specified in the report and the library report:
%
\begin{itemize}
\item For {\cf cond} forms (see report section~\extref{report:cond}{Derived
    conditionals}), a \hyper{cond clause} may take one of the follow
  forms:
\begin{scheme}
[\hyper{test} \hyperi{expression} \dotsfoo]
{}[\hyper{test} => \hyper{expression}]
{}[else \hyperi{expression} \hyperii{expression} \dotsfoo]%
\end{scheme}
\item For {\cf case} forms (see report section~\extref{report:case}{Derived
    conditionals}), a \hyper{case clause} may take one of the follow
  forms:
\begin{scheme}
[(\hyperi{datum} \dotsfoo) \hyperi{expression} \hyperii{expression} \dotsfoo]
{}[else \hyperi{expression} \hyperii{expression} \dotsfoo]%
\end{scheme}
\item For {\cf let}, {\cf let*}, {\cf letrec}, {\cf letrec*} forms
  (see report section~\extref{report:let}{Binding constructs}), 
  \hyper{bindings} may take the following form:
\begin{scheme}
([\hyperi{variable} \hyperi{init}] \dotsfoo)%
\end{scheme}
\item For {\cf let-values} and {\cf let-values*} forms 
  (see report section~\extref{report:let}{Binding constructs}),
  \hyper{mv-bindings} may take the following form:
\begin{scheme}
([\hyperi{formals} \hyperi{init}] \dotsfoo)%
\end{scheme}
\item For {\cf syntax-rules} forms (see report
  section~\extref{report:syntax-rules}{Macro transformers}), a
  \hyper{syntax rule} may take the following form:
\begin{scheme}
[\hyper{srpattern} \hyper{template}]%
\end{scheme}
\item For {\cf identifier-syntax} forms (see report
  section~\extref{report:identifier-syntax}{Macro transformers}), the two
  clauses may take the following form:
\begin{scheme}
[\hyperi{id} \hyperi{template}]
{}[(set! \hyperii{id} \hyper{pattern}) \hyperii{template}]%
\end{scheme}
\item For {\cf do} forms (see library
  section~\extref{lib:do}{Control structures}), the variable bindings
  may take the following form:
\begin{scheme}
([\hyperi{variable} \hyperi{init} \hyperi{step}] \dotsfoo)%
\end{scheme}
\item For {\cf case-lambda} forms (see library
  section~\extref{lib:case-lambda}{Control structures}), a
  \hyper{case-lambda clause} may take the following form:
%
\begin{scheme}
[\hyper{formals} \hyper{body}]%
\end{scheme}
\item For {\cf guard} forms (see library section~\extref{lib:guard}{Exceptions}),
a \hyper{cond clause} may take one of the follow forms:
\begin{scheme}
[\hyper{test} \hyperi{expression} \dotsfoo]
{}[\hyper{test} => \hyper{expression}] 
{}[else \hyperi{expression} \hyperii{expression} \dotsfoo]%
\end{scheme}
\item For {\cf syntax-case} forms (see library
  chapter~\extref{lib:syntax-case}{{\tt syntax-case}}), a
  \hyper{syntax-case rule} may take one of the following forms:
\begin{scheme}
[\hyper{pattern} \hyper{output expression}]
{}[\hyper{pattern} \hyper{fender} \hyper{output expression}]%
\end{scheme}
\end{itemize}

\chapter{Scripts}
\label{scriptappendix}

A \defining{Scheme script\mainindex{script}} is a top-level program 
(see report chapter~\extref{report:programchapter}{Top-level
  programs}) which is packaged such that 
it is directly executable by conforming implementations of Scheme, on 
one or more plaforms.

\section{Script interpreter}

Where applicable, implementations should provide a \defining{script interpreter} in 
the form of an executable program named {\cf scheme-script} that is 
capable of initiating the execution of Scheme scripts, as described 
below.

\begin{rationale}
Distributing a Scheme program that is portable with respect to both 
Scheme implementations and operating systems is challenging, even 
if that program has been written in standard Scheme.  Languages with 
a single or primary implementation can at least rely on a standard 
name for their script interpreters.  Standardizing the name of the 
executable used to start a Scheme script removes one barrier to the 
distribution of Scheme scripts.
\end{rationale}

\section{Syntax}
\label{scriptsyntaxsection}

A Scheme \defining{script} is a delimited piece of text, typically a
file, which consists of an optional script header, followed by a
top-level program.  A script header has the following syntax:
%
\begin{grammar}
\meta{script header} \: \meta{shebang} /usr/bin/env \meta{space} 
  \> scheme-script \meta{linefeed}
\meta{shebang} \: \#! \| \#! \meta{space}
\end{grammar}

\subsection{Script header}

The \defining{script header}, if present on the first line of a script,
is used by Unix-like operating systems to identify the interpreter to 
execute that script.

The script header syntax given above is the recommended portable 
form that programmers should use.  However, if the first line 
of a script begins with {\cf \#!/} or {\cf \#!\meta{space}}, 
implementations should ignore it on all platforms, even if it does not conform to 
the recommended syntax.  

\begin{rationale}
Requiring script interpreters to recognize and ignore the script 
header helps ensure that Scheme scripts written for Unix-like systems 
can also run on other kinds of systems. Furthermore, recognizing
all {\cf \#!/} or {\cf \#!\meta{space}} combinations 
allows local customizations to be 
performed by altering a script header from its default form.
\end{rationale}

\subsection{Example}
\begin{scheme}
\#!/usr/bin/env scheme-script
\#!r6rs
(import (rnrs base)
        (rnrs io ports)
        (rnrs programs))
(put-bytes (standard-output-port)
           (call-with-port
               (open-file-input-port
                 (cadr (command-line)))
             get-bytes-all))%
\end{scheme}

\section{Platform considerations}
Many platforms require that scripts be marked as executable in some way. 
The platform-specific details of this are beyond the scope of this 
report.  Scripts that are not suitably marked as executable will fail 
to execute on many platforms.
Other platform-specific notes for some popular operating systems follow.

\subsection{Apple Mac OS X}
The Mac OS X operating system supports the Unix-like script header for
shell scripts that run in the Terminal.  Depending on the intended
usage, it may be advisable to choose a file name ending in {\cf
  .command} for a script, as this makes the script double-clickable.
 %TODO: other kind of Mac script?

\subsection{Unix}
Scheme scripts on Unix-like operating systems are supported by the 
presence of the script header.  Scripts that omit the script header 
are unlikely to be directly executable on Unix-like systems.

To support installation of the Scheme script interpreter in 
non-standard paths, scripts should use the {\cf /usr/bin/env} 
program as specified in the recommended script header syntax.  
(Note that on many Unix-like systems, this also allows the 
script interpreter itself to be implemented as a shell script.)

\subsection{Microsoft Windows}
The Windows operating system allows a file extension to be associated 
with a script interpreter such as {\cf scheme-script}.  This 
association may be configured appropriately by Scheme implementations, 
installation programs, or by the user.

\subsection{Selecting an implementation}
If multiple implementations of Scheme are installed on a machine, the
user may wish to specify which implementation should be used to 
execute Scheme scripts by default.  Most platforms support some 
mechanism for choosing between alternative implementations of a 
program.  For example, Debian GNU/Linux uses the {\cf /etc/alternatives} 
mechanism to do this; Microsoft Windows uses file extension 
associations.  Implementations are expected to configure this 
appropriately, e.g., as part of their installation procedure.
Failing that, users must perform any necessary configuration to 
choose their preferred Scheme script interpreter.

\chapter{Source code representation}
\label{sourcecoderepappendix}

The report does not specify how source code is represented and stored.
The only requirement the report imposes is that the source code of a
top-level program (see report
section~\extref{report:programsyntaxsection}{Top-level program
  syntax}) or a script (see section~\ref{scriptsyntaxsection}) be
delimited.  The source code of a library is self-delimiting in the
sense that, if the beginning of a library form can be identified, so
can the end.

Implementations may take radically different approaches to storing
source code for libraries, among them: files in the file system where
each file contains an arbitrary number of library forms, files in the
file system where each file contains exactly one library form, records
in a database, and data structures in memory.

Similarly, programs and scripts may be stored in a variety of
formats.  Platform constraints may restrict the choices available to
an implementation, which is why the report neither mandates nor
recommends a specific method for storage.

Implementations may provide a means for importing libraries coming
from the outside via an interface that accepts a UTF-8 text file in
Unicode Normalization Form C where line endings are encoded as linefeed
characters.  Such text files may contain an arbitrary number of
library forms.  (Authors of such files should include an
{\cf\#!r6rs} comment if the file is written purely with the lexical
and datum syntax described in the report.  See report
section~\extref{report:whitespaceandcomments}{Whitespace and
  comments}.)  After importing such a file, the libraries defined in
it should be available to other libraries and files.  An
implementation may store the file as is, or convert it to some storage
format to achieve this.

Similarly, implementations may provide a means for executing a program
represented as a UTF-8 text file containing its source code.  Again,
authors of such files should include an {\cf\#!r6rs}
comment if the file is written purely with the lexical and datum syntax
described in the report.  This report does not describe a file format
that allows both libraries and programs to appear in the same file.

\chapter{Use of library versions}

Names for libraries may include a version.  An \hyper{import spec} may
designate a set of acceptable versions that may be imported.
Conversely, only one version of each library should be part of a
program.  This allows using the ``name part'' of a \hyper{library
  name} for different purposes than the \hyper{version}. 

In particular, if several different variants of a library exists where
it is feasible that they coexist in the same program, it is
recommended that different names be used for the variants.  In
contrast, for compatible versions of a library where coexistence of
several versions is unnecessary and undesirable, it is recommended
that the same name and different versions be used.  In particular, it
is recommended that new versions of libraries that are conservative
extensions of old ones differ only in the version, not in the name.

Correspondingly, it is recommended that \hyper{import spec}s do not
constrain an import to a single version, but instead specify a wide
range of acceptable versions of a library.

Implementations that allow two libraries of the same name with
different versions to coexist in the same program should
report when processing a program that actually makes use of this
extension.

\chapter{Unique library names}
\label{librarynamesappendix}

Programmers should choose names for distributed libraries
that do not collide with other libraries' names.
This appendix suggests a convention for generating unique library
names, similar to the convention for Java~\cite{JLS3}.

A unique library name can be formed by associating the library with an
Internet domain name, such as {\cf mit.edu}.  The lower-case
components of the domain are reversed to form a prefix for the library
name.  Adding further name components to establish a hierarchy may be
advisable, depending on the size of the organization associated with
the domain name, the number of libraries to be distributed from it,
and other organizational properties or conventions associated with the
library.

Programmers should use library names that are suitable for
use as part of file names.  Special characters in
domain names that do not fit conventions of commonly used file systems should be replaced by
hyphens or suitable ``escape sequences'' that, as much as possible,
are suitable for avoiding collisions.  Here are some examples for
possible library names according to this convention:
%
\begin{scheme}
(edu mit swiss cheese)
(de deinprogramm educational graphics turtle)
(com pan-am booking passenger)%
\end{scheme}
%
The name of a library does not necessarily indicate an Internet
address where the package is distributed.

\renewcommand{\bibname}{References}

\bibliographystyle{plain}
\bibliography{abbrevs,rrs}

% \newcommand{\indexheading}{Alphabetic index of definitions of
%   concepts, keywords, and procedures}
% \newcommand{\indexintro}{}
%
% \printindex

\end{document}

%%% Local Variables: 
%%% mode: latex
%%% TeX-master: "r6rs-app"
%%% End: 

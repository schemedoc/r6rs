\chapter{Numbers}
\label{numberchapter}
\index{number}

Numerical computation has traditionally been neglected by the Lisp
community.  Until Common Lisp there was no carefully thought out
strategy for organizing numerical computation, and with the exception of
the MacLisp system \cite{Pitman83} little effort was made to
execute numerical code efficiently.  This report recognizes the excellent work
of the Common Lisp committee and accepts many of their recommendations.
In some ways this report simplifies and generalizes their proposals in a manner
consistent with the purposes of Scheme.

It is important to distinguish between the mathematical numbers, the
Scheme numbers that attempt to model them, the machine representations
used to implement the Scheme numbers, and notations used to write numbers.
This report uses the types \type{number}, \type{complex}, \type{real},
\type{rational}, and \type{integer} to refer to both mathematical numbers
and Scheme numbers.  Machine representations such as fixed point and
floating point are referred to by names such as \type{fixnum} and
\type{flonum}.

\section{Numerical types}
\label{numericaltypes}
\index{numerical types}

\vest Mathematically, numbers may be arranged into a tower of subtypes
in which each level is a subset of the level above it:
\begin{tabbing}
\ \ \ \ \ \ \ \ \ \=\tupe{number} \\
\> \tupe{complex} \\
\> \tupe{real} \\
\> \tupe{rational} \\
\> \tupe{integer} 
\end{tabbing}

For example, 3 is an integer.  Therefore 3 is also a rational,
a real, and a complex.  The same is true of the Scheme numbers
that model 3.  For Scheme numbers, these types are defined by the
predicates \ide{number?}, \ide{complex?}, \ide{real?}, \ide{rational?},
and \ide{integer?}.

There is no simple relationship between a number's type and its
representation inside a computer.  Although most implementations of
Scheme will offer at least two different representations of 3, these
different representations denote the same integer.

Scheme's numerical operations treat numbers as abstract data, as
independent of their representation as possible.  Although an implementation
of Scheme may use fixnum, flonum, and perhaps other representations for
numbers, this should not be apparent to a casual programmer writing
simple programs.

It is necessary, however, to distinguish between numbers that are
represented exactly and those that may not be.  For example, indexes
into data structures must be known exactly, as must some polynomial
coefficients in a symbolic algebra system.  On the other hand, the
results of measurements are inherently inexact, and irrational numbers
may be approximated by rational and therefore inexact approximations.
In order to catch uses of inexact numbers where exact numbers are
required, Scheme explicitly distinguishes exact from inexact numbers.
This distinction is orthogonal to the dimension of type.

A subrange of the exact integers is designated as the set of fixnums.
Conversely, a \defining{fixnum} is an exact integer whose value lies
within this fixnum range.  Moreover, every implementation is required
to designate a subset of its inexact reals as \defining{flonum}s, and
to convert certain external representations into flonums.  Note that
this does not imply that an implementation is required to use
floating-point representations

\section{Exactness}
\label{exactly}

\mainindex{exactness} Scheme numbers are either \type{exact} or
\type{inexact}.  A number is exact if it was written as an exact
constant or was derived from exact numbers using only exact
operations.  A number is inexact if it was written as an inexact
constant or was derived from inexact numbers.  Thus inexactness is
contagious.  

It is the programmer's responsibility to avoid using numbers with
magnitude or significand too large to be represented in the
implementation.

If two implementations produce exact results for a computation that
did not involve inexact intermediate results, the two ultimate results
will be mathematically equivalent.  This is generally not true of
computations involving inexact numbers because approximate methods
such as floating point arithmetic may be used, but it is the duty of
each implementation to make the result as close as practical to the
mathematically ideal result.

\section{Implementation restrictions}

\index{implementation restriction}\label{restrictions}

\vest Implementations of Scheme are required to implement the whole
tower of subtypes given in section~\ref{numericaltypes}.

\vest Implementations may also support only a limited range of inexact numbers of
any type, subject to the requirements of this section.  For example,
an implementation that uses flonums to represent all its
inexact real numbers may
limit the range of inexact reals (and therefore
the range of inexact integers and rationals)
to the dynamic range of the flonum format.
Furthermore
the gaps between the representable inexact integers and
rationals are
likely to be very large in such an implementation as the limits of this
range are approached.

\vest Implementations are required to support
exact integers and exact rationals of
practically unlimited size and precision, and to implement the
above procedures and the {\cf /} procedure in
such a way that they always return exact results when given exact
arguments.

\vest An implementation may use floating point and other approximate 
representation strategies for \tupe{inexact} numbers.
This report recommends, but does not require, that the IEEE 
floating point standards be followed by implementations that use
flonum representations, and that implementations using
other representations should match or exceed the precision achievable
using these floating point standards~\cite{IEEE}.

\vest In particular, implementations that use flonum representations
must follow these rules: A flonum result
must be represented with at least as much precision as is used to express any of
the inexact arguments to that operation.  It is desirable (but not required) for
potentially inexact operations such as {\cf sqrt}, when applied to exact
arguments, to produce exact answers whenever possible (for example the
square root of an exact 4 ought to be an exact 2).
If, however, an
exact number is operated upon so as to produce an inexact result
(as by {\cf sqrt}), and if the result is represented as a flonum, then
the most precise flonum format available must be used; but if the result
is represented in some other way then the representation must have at least as
much precision as the most precise \tupe{flonum} format available.


\section{Infinities and NaNs}
\label{infinitiesnanssection}

Positive infinity is regarded as a real (but not rational) number,
whose value is indeterminate but greater than all rational numbers.
Negative infinity is regarded as a real (but not rational) number,
whose value is indeterminate but less than all rational numbers.

A NaN is regarded as a real (but not rational) number whose value is
so indeterminate that it might represent any real number, including
positive or negative infinity, and might even be greater than positive
infinity or less than negative infinity.

The specification of the numerical operations is written as though
infinities and NaNs are representable, and specifies many operations
with respect to these numbers in ways that are consistent with the
IEEE 754 standard for binary floating point arithmetic.  
An implementation of Scheme is not required to represent infinities and
NaNs. Such an implementation must raise a continuable exception with
condition type {\cf\&no-infinities} or {\cf\&no-nans} respectively
(see section~\ref{flonumssection})
whenever it is unable to represent an infinity or NaN as required by
the specification.  In this case, the continuation of the exception
handler will be the continuation that otherwise would have received
the infinity or NaN value.  This requirement also applies to
conversions between numbers and external representations, including
the reading of program source code.

\section{Semantics of common operations}

Some operations are the semantic basis for several arithmetic
procedures.  The behavior of these operations is described in this
section for later reference.

\subsection{Integer division}
\label{integerdivision}

For various kinds of arithmetic (fixnum, flonum, exact, inexact, and
generic), Scheme provides operations for performing integer
division.  They rely on mathematical operations $\mathrm{div}$,
$\mathrm{mod}$, $\mathrm{div}_0$, and
$\mathrm{mod}_0$, that are defined as follows:

$\mathrm{div}$, $\mathrm{mod}$, $\mathrm{div}_0$, and $\mathrm{mod}_0$
each accept two real numbers $x_1$ and $x_2$ as operands, where
$x_2$ must be nonzero.

$\mathrm{div}$ returns an integer, $\mathrm{mod}$ returns a real.
Their results are specified by
%
\begin{eqnarray*}
x_1~\mathrm{div}~x_2 &=& n_d\\
x_1~\mathrm{mod}~x_2 &=& x_m
\end{eqnarray*}
%
where
%
\begin{displaymath}
\begin{array}{c}
x_1 = n_d * x_2 + x_m\\
0 \leq x_m < |x_2|
\end{array}
\end{displaymath}
%
Examples:
\begin{eqnarray*}
5~\mathrm{div}~3    &=&  1\\
5~\mathrm{div}~-3   &=&  -1\\
5~\mathrm{mod}~3    &=&  2\\
5~\mathrm{mod}~-3   &=&  2
\end{eqnarray*}
%
$\mathrm{div}_0$ and $\mathrm{mod}_0$ are like $\mathrm{div}$ and
$\mathrm{mod}$, except the result of $\mathrm{mod}_0$ lies within a
half-open interval centered on zero.  The results are specified by
%
\begin{eqnarray*}
x_1~\mathrm{div}_0~x_2 &=& n_d\\
x_1~\mathrm{mod}_0~x_2 &=& x_m
\end{eqnarray*}
%
where:
%
\begin{displaymath}
\begin{array}{c}
x_1 = n_d * x_2 + x_m\\
-|\frac{x_2}{2}| \leq x_m < |\frac{x_2}{2}|
\end{array}
\end{displaymath}
%
Examples:
%
\begin{eqnarray*}
5~\mathrm{div}_0~3    &=&  2\\
5~\mathrm{div}_0~-3   &=&  -2\\
5~\mathrm{mod}_0~3    &=&  -1\\
5~\mathrm{mod}_0~-3   &=&  -1
\end{eqnarray*}

\begin{rationale}
The half-open symmetry about zero is convenient for some purposes.
\end{rationale}

\subsection{Transcendental functions}
\label{transcendentalfunctions}

In general, the transcendental functions $\log$, $\sin^{-1}$
(arcsine), $\cos^{-1}$ (arccosine), and $\tan^{-1}$ are multiply
defined.  The value of $\log z$ is defined to be the one whose
imaginary part lies in the range from $-\pi$ (inclusive if $-0.0$ is
distinguished, exclusive otherwise) to $\pi$ (inclusive).  $\log 0$ is
undefined.

The value of $\log z$ for non-real $z$ is defined in terms of log on real numbers as 

\begin{displaymath}
\log z = \log |z| + \mathrm{angle}~z
\end{displaymath}
%
where $\mathrm{angle}~z$ is the angle of $z = a\cdot e^{ib}$ specified
as:
$$\mathrm{angle}~z = b+2\pi n$$
with $-\pi \leq \mathrm{angle}~z\leq \pi$ and $\mathrm{angle}~z =
b+2\pi n$ for some integer $n$.

With the one-argument version of $\log$ defined this way, the values
of the two-argument-version of $\log$, $\sin^{-1} z$, $\cos^{-1} z$,
$\tan^{-1} z$, and the two-argument version of $\tan^{-1}$ are
according to the following formul\ae:
\begin{eqnarray*}
\log z~b &=& \frac{\log z}{\log b}\\
\sin^{-1} z &=& -i \log (i z + \sqrt{1 - z^2})\\
\cos^{-1} z &=& \pi / 2 - \sin^{-1} z\\
\tan^{-1} z &=& (\log (1 + i z) - \log (1 - i z)) / (2 i)\\
\tan^{-1} x~y &=& \mathrm{angle}(x+ yi)
\end{eqnarray*}

The range of $\tan^{-1} x~y$ is as in the following table. The
asterisk (*) indicates that the entry applies to implementations that
distinguish minus zero.

\begin{center}
\begin{tabular}{clll}
& $y$ condition & $x$ condition & range of result $r$\\\hline
& $y = 0.0$ & $x > 0.0$ & $0.0$\\
$\ast$ & $y = +0.0$  & $x > 0.0$ & $+0.0$\\	
$\ast$ & $y = -0.0$ & $x > 0.0$ & $-0.0$\\
& $y > 0.0$ & $x > 0.0$ & $0.0 < r < \frac{\pi}{2}$\\
& $y > 0.0$ & $x = 0.0$ & $\frac{\pi}{2}$\\
& $y > 0.0$ & $x < 0.0$ & $\frac{\pi}{2} < r < \pi$\\
& $y = 0.0$ & $x < 0$ & $\pi$\\
$\ast$ & $y = +0.0$ & $x < 0.0$ & $\pi$\\
$\ast$ & $y = -0.0$ & $x < 0.0$ & $-\pi$\\	
&$y < 0.0$ & $x < 0.0$ & $-\pi< r< -\frac{\pi}{2}$\\
&$y < 0.0$ & $x = 0.0$ & $-\frac{\pi}{2}$\\
&$y < 0.0$ & $x > 0.0$ & $-\frac{\pi}{2} < r< 0.0$\\	
&$y = 0.0$ & $x = 0.0$ & undefined\\
$\ast$& $y = +0.0$ & $x = +0.0$ & $+0.0$\\
$\ast$& $y = -0.0$ & $x = +0.0$& $-0.0$\\
$\ast$& $y = +0.0$ & $x = -0.0$ & $\pi$\\
$\ast$& $y = -0.0$ & $x = -0.0$ & $-\pi$\\
$\ast$& $y = +0.0$ & $x = 0$ & $\frac{\pi}{2}$\\
$\ast$& $y = -0.0$ & $x = 0$	& $-\frac{\pi}{2}$
\end{tabular}
\end{center}

The above specification follows~\cite{CLtL}, which in turn
cites~\cite{Penfield81}; refer to these sources for more detailed
discussion of branch cuts, boundary conditions, and implementation of
these functions.

\section{Fixnums}

A subrange of the exact integers is designated as the set of fixnums.
Conversely, a fixnum is an exact integer whose value lies within this
fixnum range.

Every implementation must define its fixnum range as a closed
interval
%
\begin{displaymath}
[-2^{w-1}, 2^{w-1} - 1]
\end{displaymath}
%
such that $w$ is a a (mathematical) integer $w \geq 24$.  Every
mathematical integer within an implementation's fixnum range must
correspond to an exact integer that is representable within the
implementation.

This section specifies two kinds of operations on fixnums.  Operations
whose names begin with {\cf fixnum} perform arithmetic modulo
$2^{w}$.  Operations whose names begin with {\cf fx}
perform integer arithmetic on their fixnum arguments, but raise an
exception with condition type {\cf\&implementation-restriction} 
if the result is not a fixnum.

\begin{rationale}
The operations whose names begin with {\cf fixnum}
implement arithmetic on a quotient ring of the integers,
but their results will not be the same in every implementation
because the particular ring is parameterized by $w$.
The operations whose names begin with {\cf fx} do
not have as nice a closure property, and the arguments that
cause them to raise an exception will not be the same in every
implementation, but any results they return without
raising an exception will be the same in all implementations.
\end{rationale}

Some operations (e.g.\ {\cf fixnum<} and {\cf fx<}) behave the same in
both sets.

\begin{rationale}
Duplication of names reduces bias toward either set,
and saves programmers from having to remember which
names are supplied.
\end{rationale}

This section uses \var{fx}, \vari{fx} and \varii{fx} as parameter
names for arguments that must be fixnums.

\subsection{Signalling fixnum operations}

\begin{entry}{%
\proto{fixnum?}{ obj}{procedure}}

This returns \schtrue{} if \var{obj} is an exact
integer within the fixnum range, and
otherwise returns \schfalse{}.
\end{entry}

\begin{entry}{%
\proto{fixnum-width}{}{procedure}
\proto{least-fixnum}{}{procedure}
\proto{greatest-fixnum}{}{procedure}}

These procedures return $w$,
$-2^{w-1}$ and $2^{w-1} - 1$, the
width, minimum and the maximum value of the fixnum range, respectively.
\end{entry}

\begin{entry}{%
\proto{fixnum=?}{ \vari{fx} \varii{fx} \variii{fx} \dotsfoo}{procedure}
\proto{fixnum>?}{ \vari{fx} \varii{fx} \variii{fx} \dotsfoo}{procedure}
\proto{fixnum<?}{ \vari{fx} \varii{fx} \variii{fx} \dotsfoo}{procedure}
\proto{fixnum>=?}{ \vari{fx} \varii{fx} \variii{fx} \dotsfoo}{procedure}
\proto{fixnum<=?}{ \vari{fx} \varii{fx} \variii{fx} \dotsfoo}{procedure}}

These procedures return \schtrue{} if their arguments are (respectively):
equal, monotonically increasing, monotonically decreasing,
monotonically nondecreasing, or monotonically nonincreasing,
\schfalse{} otherwise.
\end{entry}

\begin{entry}{%
\proto{fixnum-zero?}{ fx}{procedure}
\proto{fixnum-positive?}{ fx}{procedure}
\proto{fixnum-negative?}{ fx}{procedure}
\proto{fixnum-odd?}{ fx}{procedure}
\proto{fixnum-even?}{ fx}{procedure}}

These numerical predicates test a fixnum for a particular property,
returning \schtrue{} or \schfalse{}.  The five properties tested by
these procedures are: whether the number is zero, greater than zero,
less than zero, odd, or even.
\end{entry}

\begin{entry}{%
\proto{fixnum-max}{ \vari{fx} \varii{fx} \dotsfoo}{procedure}
\proto{fixnum-min}{ \vari{fx} \varii{fx} \dotsfoo}{procedure}}

These procedures return the maximum or minimum of their arguments.
\end{entry}

\begin{entry}{%
\proto{fixnum+}{ \vari{fx} \dotsfoo}{procedure}
\proto{fixnum*}{ \vari{fx} \dotsfoo}{procedure}}

These procedures return the unique fixnum that is congruent mod $2^w$
to the sum or product of their arguments.
\end{entry}

\begin{entry}{%
\proto{fixnum-}{ \vari{fx} \varii{fx} \dotsfoo}{procedure}
\rproto{fixnum-}{ fx}{procedure}}

With two or more arguments, this procedure returns the unique fixnum
that is congruent mod $2^w$ to the difference of its arguments,
associating to the left.  With one argument, however, it returns the
the unique fixnum that is congruent mod $2^w$ to the additive inverse
of its argument.
\end{entry}

\begin{entry}{%
\proto{fixnum-div+mod}{ \vari{fx} \varii{fx}}{procedure}
\proto{fixnum-div}{ \vari{fx} \varii{fx}}{procedure}
\proto{fixnum-mod}{ \vari{fx} \varii{fx}}{procedure}
\proto{fixnum-div0+mod0}{ \vari{fx} \varii{fx}}{procedure}
\proto{fixnum-div0}{ \vari{fx} \varii{fx}}{procedure}
\proto{fixnum-mod0}{ \vari{fx} \varii{fx}}{procedure}}

{\varii{Fx} must be nonzero.}
These procedures implement number-theoretic integer division modulo
$2^w$.  Each procedure returns the unique fixnum(s) congruent modulo
$2^w$ to the result(s) specified in section~\ref{integerdivision}.

\begin{scheme}
(fixnum-div \vari{ex} \varii{ex})         \ev \(\vari{ex}~\mathrm{div}~\varii{ex}\)
(fixnum-mod \vari{ex} \varii{ex})         \ev \(\vari{ex}~\mathrm{mod}~\varii{ex}\)
(fixnum-div+mod \vari{ex} \varii{ex})     \ev \(\vari{ex}~\mathrm{div}~\varii{ex}, \vari{ex}~\mathrm{mod}~\varii{ex}\)\\\>\>\>; two return values
(fixnum-div0 \vari{ex} \varii{ex})        \ev \(\vari{ex}~\mathrm{div}\sb{0}~\varii{ex}\)
(fixnum-mod0 \vari{ex} \varii{ex})        \ev \(\vari{ex}~\mathrm{mod}\sb{0}~\varii{ex}\)
(fixnum-div0+mod0 \vari{ex} \varii{ex})   \lev \(\vari{ex} \vari{ex}~\mathrm{div}\sb{0}~\varii{ex}, \vari{ex}~\mathrm{mod}\sb{0}~\varii{ex}\)\\\>\>\>; two return values
\end{scheme}
\end{entry}

\begin{entry}{%
\proto{fixnum+/carry}{ \vari{fx} \varii{fx} \variii{fx}}{procedure}}

Returns the two fixnum results of the following computation:
%
\begin{scheme}
(let* ((s (+ \vari{fx} \varii{fx} \variii{fx}))
       (s0 (mod0 s (expt 2 (fixnum-width))))
       (s1 (div0 s (expt 2 (fixnum-width)))))
  (values s0 s1))
\end{scheme}

\begin{note}
The results returned by the
{\cf fixnum+/carry},
{\cf fixnum-/carry}, and
{\cf fixnum*/carry}
procedures depend upon the precision $w$,
so there are no {\cf fx} equivalents to these procedures.
\end{note}
\end{entry}

\begin{entry}{%
\proto{fixnum-/carry}{ \vari{fx} \varii{fx} \variii{fx}}{procedure}}

Returns the two fixnum results of the following computation:
%
\begin{scheme}
(let* ((d (- \vari{fx} \varii{fx} \variii{fx}))
       (d0 (mod0 d (expt 2 (fixnum-width))))
       (d1 (div0 d (expt 2 (fixnum-width)))))
  (values d0 d1))
\end{scheme}
\end{entry}

\begin{entry}{%
\proto{fixnum*/carry}{ \vari{fx} \varii{fx} \variii{fx}}{procedure}}

Returns the two fixnum results of the following computation:
\begin{scheme}
(let* ((s (+ (* \vari{fx} \varii{fx}) \variii{fx}))
       (s0 (mod0 s (expt 2 (fixnum-width))))
       (s1 (div0 s (expt 2 (fixnum-width)))))
  (values s0 s1))
\end{scheme}
\end{entry}

\begin{entry}{%
\proto{fixnum-not}{ \var{fx}}{procedure}}

This procedure returns the unique fixnum that is congruent
mod $2^w$ to the one's-complement of their argument.
\end{entry}

\begin{entry}{%
\proto{fixnum-and}{ \vari{fx} \dotsfoo}{procedure}
\proto{fixnum-ior}{ \vari{fx} \dotsfoo}{procedure}
\proto{fixnum-xor}{ \vari{fx} \dotsfoo}{procedure}}

These procedures return the fixnum that is the bit-wise ``and,''
``inclusive or,'' or ``exclusive or'' of the two's complement
representations of their arguments.  If they are passed only one
argument, they return that argument.  If they are passed no arguments,
they return the fixnum (either $-1$ or $0$) that acts as identity for the
operation.
\end{entry}

\begin{entry}{%
\proto{fixnum-if}{ \vari{fx} \varii{fx} \variii{fx}}{procedure}}

This procedure return the fixnum result of the following
computation:
\begin{scheme}
(fixnum-ior (fixnum-and \vari{fx} \varii{fx})
            (fixnum-and (fixnum-not \vari{fx}) \variii{fx}))
\end{scheme}
\end{entry}

\begin{entry}{%
\proto{fixnum-bit-count}{ \var{fx}}{procedure}}

If the argument \var{fx} is non-negative, this procedure return the
number of 1 bits in the two's complement representation of \var{fx}.
Otherwise they return the number of 0 bits in the two's complement
representation of \var{fx}.
\end{entry}

\begin{entry}{%
\proto{fixnum-length}{ \var{fx}}{procedure}}

This procedure returns the fixnum result of the following
computation:
\begin{scheme}
(do ((result 0 (+ result 1))
     (bits (if (fixnum-negative? \var{fx})
               (fixnum- \var{fx})
               \var{fx})
           (fixnum-logical-shift-right bits 1)))
    ((fixnum-zero? bits)
     result))
\end{scheme}
\end{entry}

\begin{entry}{%
\proto{fixnum-first-bit-set}{ \var{fx}}{procedure}}

This procedure returns the index of the least significant $1$ bit in
the two's complement representation of their argument.  If the
argument is $0$, then $-1$ is returned.
%
\begin{scheme}
(fixnum-first-bit-set 0)        \ev  -1
(fixnum-first-bit-set 1)        \ev  0
(fixnum-first-bit-set -4)       \ev  2
\end{scheme}
\end{entry}

\begin{entry}{%
\proto{fixnum-bit-set?}{ \vari{fx} \varii{fx}}{procedure}}

\domain{\varii{Fx} must be non-negative.}  This procedure returns the
fixnum result of the following computation:
\begin{scheme}
(not (fixnum-zero?
      (fixnum-and \vari{fx}
                  (fixnum-logical-shift-left 1 \varii{fx}))))
\end{scheme}
%
\end{entry}

\begin{entry}{%
\proto{fixnum-copy-bit}{ \vari{fx} \varii{fx} \variii{fx}}{procedure}}

\domain{\varii{Fx} must be non-negative.}  This procedure
returns the fixnum result of the following computation:
\begin{scheme}
(let* ((mask (fixnum-logical-shift-left 1 \varii{fx})))
  (fixnum-if mask
             (fixnum-logical-shift-left \variii{fx} \varii{fx})
             \vari{fx}))
\end{scheme}
\end{entry}

\begin{entry}{%
\proto{fixnum-bit-field}{ \vari{fx} \varii{fx} \variii{fx}}{procedure}}

\domain{\varii{Fx} and \variii{fx} must be non-negative.}  This procedure
returns the fixnum result of the following computation:
%
\begin{scheme}
(let* ((mask (fixnum-not
              (fixnum-logical-shift-left -1 \variii{fx}))))
  (fixnum-logical-shift-right (fixnum-and \vari{fx} mask)
                              \varii{fx}))
\end{scheme}
%
\end{entry}

\begin{entry}{%
\proto{fixnum-copy-bit-field}{ \vari{fx} \varii{fx} \variii{fx} \variv{fx}}{procedure}}

\domain{The second and third arguments must be non-negative.} This procedure
returns the fixnum result of the following computation:
\begin{scheme}
(let* ((to    \vari{fx})
       (start \varii{fx})
       (end   \variii{fx})
       (from  \variv{fx})
       (mask1 (fixnum-logical-shift-left -1 start))
       (mask2 (fixnum-not
               (fixnum-logical-shift-left -1 end)))
       (mask (fixnum-and mask1 mask2)))
  (fixnum-if mask
             (fixnum-logical-shift-left from start)
             to))
\end{scheme}
\end{entry}

\begin{entry}{%
\proto{fixnum-arithmetic-shift}{ \vari{fx} \varii{fx}}{procedure}}

Returns the unique fixnum that is congruent mod $2^w$ to the result of
the following computation:
%
\begin{scheme}
(exact-floor (* \vari{fx} (expt 2 \varii{fx})))
\end{scheme}
\end{entry}

\begin{entry}{%
\proto{fixnum-arithmetic-shift-left}{ \vari{fx} \varii{fx}}{procedure}
\proto{fixnum-arithmetic-shift-right}{ \vari{fx} \varii{fx}}{procedure}}

\domain{\varii{Fx} must be non-negative.}  {\cf
  Fixnum-arithmetic-shift-left} returns the same result as {\cf
  fixnum-arithmetic-shift}, and {\cf (fixnum-arithmetic-shift-right
\vari{fx} \varii{fx})} returns the same result as
{\cf (fixnum-arithmetic-shift \vari{fx} (fixnum- \varii{fx}))}.
\end{entry}

\begin{entry}{%
\proto{fixnum-logical-shift-left}{ \vari{fx} \varii{fx}}{procedure}}

This procedure behaves the same as {\cf fixnum-arithmetic-shift-left}.
\end{entry}

\begin{entry}{%
\proto{fixnum-logical-shift-right}{ \vari{fx} \varii{fx}}{procedure}}

\domain{\varii{Fx} must be non-negative.}  This procedure returns
the result of the following computation:
\begin{scheme}
(let* ((n       \vari{fx})
       (shift   \varii{fx})
       (shifted
         (fixnum-arithmetic-shift-right n shift)))
  (let* ((mask-width
           (fixnum-
             (fixnum-width)
             (fixnum-mod shift (fixnum-width))))
         (mask (fixnum-not
                (fixnum-logical-shift-left
                  -1 mask-width))))
    (fixnum-and shifted mask)))
\end{scheme}
%
\begin{note}
The results of {\cf fixnum-logical-shift-left}
and {\cf fixnum-logical-shift-left}
can depend upon the precision $w$,
so they have no {\cf fx} equivalents.
\end{note}
\end{entry}

\begin{entry}{%
\proto{fixnum-rotate-bit-field}{ \vari{fx} \varii{fx} \variii{fx} \variv{fx}}{procedure}}

\domain{\varii{Fx}, \variii{fx}, and \variv{fx} must be non-negative.}
This procedure returns the result of the following computation:
\begin{scheme}
(let* ((n     \vari{fx})
       (start \varii{fx})
       (end   \variii{fx})
       (count \variv{fx})
       (width (fixnum- end start)))
  (if (fixnum-positive? width)
      (let* ((count (fixnum-mod count width))
             (field0
               (fixnum-bit-field n start end))
             (field1
               (fixnum-logical-shift-left
                 field0 count))
             (field2
               (fixnum-logical-shift-right
                 field0 (fixnum- width count)))
             (field (fixnum-ior field1 field2)))
        (fixnum-copy-bit-field n start end field))
      n))
\end{scheme}
\end{entry}

\begin{entry}{%
\proto{fixnum-reverse-bit-field}{ \vari{fx} \varii{fx} \variii{fx}}{procedure}}

Returns the fixnum obtained from the first argument by reversing the
bit field specified by the second and third arguments.
\begin{scheme}
(fixnum-reverse-bit-field \sharpsign{}b1010010 1 4)    \lev  88 ; \sharpsign{}b1011000
(fixnum-reverse-bit-field \sharpsign{}b1010010 91 -4)  \lev  82 ; \sharpsign{}b1010010
\end{scheme}
\end{entry}

\subsection{Quotient-ring fixnum operations}

\begin{entry}{%
\rproto{fixnum?}{ obj}{procedure}}

This performs the same operation as {\cf fixnum?} in the FIXME
library.
\end{entry}

\begin{entry}{%
\rproto{fixnum-width}{}{procedure}
\rproto{least-fixnum}{}{procedure}
\rproto{greatest-fixnum}{}{procedure}}

These procedures perform the same operations as {\cf fixnum-width},
{\cf least-fixnum}, and {\cf greatest-fixnum} in the FIXME library,
respectively.
\end{entry}

\begin{entry}{%
\proto{fx=?}{ \vari{fx} \varii{fx} \variii{fx} \dotsfoo}{procedure}
\proto{fx>?}{ \vari{fx} \varii{fx} \variii{fx} \dotsfoo}{procedure}
\proto{fx<?}{ \vari{fx} \varii{fx} \variii{fx} \dotsfoo}{procedure}
\proto{fx>=?}{ \vari{fx} \varii{fx} \variii{fx} \dotsfoo}{procedure}
\proto{fx<=?}{ \vari{fx} \varii{fx} \variii{fx} \dotsfoo}{procedure}}

These procedures perform the same operations as {\cf fixnum=?}, {\cf
  fixnum>?}, {\cf fixnum<?}, {\cf fixnum>=?}, and {\cf fixnum<=?},
respectively.
\end{entry}

\begin{entry}{%
\proto{fxzero?}{ fx}{procedure}
\proto{fxpositive?}{ fx}{procedure}
\proto{fxnegative?}{ fx}{procedure}
\proto{fxodd?}{ fx}{procedure}
\proto{fxeven?}{ fx}{procedure}}

These procedures perform the same operations as {\cf fixnum-zero?},
{\cf fixnum-positive?}, {\cf fixnum-negative?}, {\cf fixnum-odd?}, and
{\cf fixnum-even?}, respectively.
\end{entry}

\begin{entry}{%
\proto{fxmax}{ \vari{fx} \varii{fx} \dotsfoo}{procedure}
\proto{fxmin}{ \vari{fx} \varii{fx} \dotsfoo}{procedure}}

These procedures perform the same operations as {\cf fixnum-max},
and {\cf fixnum-min}, respectively.
\end{entry}

\begin{entry}{%
\proto{fx+}{ \vari{fx} \varii{fx}}{procedure}
\proto{fx*}{ \vari{fx} \varii{fx}}{procedure}}

These procedures return the sum or product of their arguments,
provided that sum or product is a fixnum.  An exception with condition
type {\cf\&implementation-restriction} is raised if
that sum or product is not a fixnum.

\begin{rationale}
These procedures are restricted to two arguments because their
generalizations to three or more arguments would require
precision proportional to the number of arguments.
\end{rationale}
\end{entry}

\begin{entry}{%
\proto{fx-}{ \vari{fx} \varii{fx}}{procedure}
\rproto{fx-}{ fx}{procedure}}

With two arguments, this procedure returns the difference of its
arguments, provided that difference is a fixnum.

With one argument, this procedure returns the additive
inverse of its argument, provided that integer is a
fixnum.

An exception with condition type {\cf\&contract} is raised if the
mathematically correct result of this procedure is not a fixnum.

\begin{scheme}
(fx- (least-fixnum))  \ev  \exception{\&contract}
\end{scheme}
\end{entry}

\begin{entry}{%
\proto{fxdiv+mod}{ \vari{fx} \varii{fx}}{procedure}
\proto{fxdiv}{ \vari{fx} \varii{fx}}{procedure}
\proto{fxmod}{ \vari{fx} \varii{fx}}{procedure}
\proto{fxdiv0+mod0}{ \vari{fx} \varii{fx}}{procedure}
\proto{fxdiv0}{ \vari{fx} \varii{fx}}{procedure}
\proto{fxmod0}{ \vari{fx} \varii{fx}}{procedure}}

\domain{\varii{Fx} must be nonzero.}
These procedures implement number-theoretic integer division and
return the results of the corresponding mathematical operations
specified in the section~\ref{integerdivision}.  An exception with
condition type {\cf\&implementation-restriction} is raised if a result
specified by that section is not a fixnum.

\begin{scheme}
(fxdiv \vari{ex} \varii{ex})         \ev \(\vari{ex}~\mathrm{div}~\varii{ex}\)
(fxmod \vari{ex} \varii{ex})         \ev \(\vari{ex}~\mathrm{mod}~\varii{ex}\)
(fxdiv+mod \vari{ex} \varii{ex})     \lev \(\vari{ex}~\mathrm{div}~\varii{ex}, \vari{ex}~\mathrm{mod}~\varii{ex}\)\\\>\>; two return values
(fxdiv0 \vari{ex} \varii{ex})        \ev \(\vari{ex}~\mathrm{div}\sb{0}~\varii{ex}\)
(fxmod0 \vari{ex} \varii{ex})        \ev \(\vari{ex}~\mathrm{mod}\sb{0}~\varii{ex}\)
(fxdiv0+mod0 \vari{ex} \varii{ex})   \lev \(\vari{ex} \vari{ex}~\mathrm{div}\sb{0}~\varii{ex}, \vari{ex}~\mathrm{mod}\sb{0}~\varii{ex}\)\\\>\>; two return values
\end{scheme}
\end{entry}

\begin{entry}{%
\proto{fxnot}{ \var{fx}}{procedure}}

This performs the same operation as {\cf fixnum-not}.
\end{entry}

\begin{entry}{%
\proto{fxand}{ \vari{fx} \dotsfoo}{procedure}
\proto{fxior}{ \vari{fx} \dotsfoo}{procedure}
\proto{fxxor}{ \vari{fx} \dotsfoo}{procedure}}

These procedures perform the same operations as {\cf fixnum-and},
{\cf fixnum-ior}, and {\cf fixnum-xor}, respectively.
\end{entry}

\begin{entry}{%
\proto{fxif}{ \vari{fx} \varii{fx} \variii{fx}}{procedure}}

This performs the same operation as {\cf fixnum-if}.
\end{entry}

\begin{entry}{%
\proto{fxbit-count}{ \var{fx}}{procedure}}

This performs the same operation as {\cf fixnum-bit-count}.
\end{entry}

\begin{entry}{%
\proto{fxlength}{ \var{fx}}{procedure}}

This performs the same operation as {\cf fixnum-length}.
\end{entry}

\begin{entry}{%
\proto{fxfirst-bit-set}{ \var{fx}}{procedure}}

This performs the same operation as {\cf fixnum-first-bit-set}.
\end{entry}

\begin{entry}{%
\proto{fxbit-set?}{ \var{fx} \varii{fx}}{procedure}}

\domain{\varii{Fx} must be non-negative and less than {\cf
  (fixnum-width)}.}  This procedure
returns the same result returned by {\cf fixnum-bit-set?}.
\end{entry}

\begin{entry}{%
\proto{fxcopy-bit}{ \vari{fx} \varii{fx} \variii{fx}}{procedure}}

\domain{\varii{Fx} must be non-negative and less than {\cf
  (fixnum-width)}. \variii{Fx} must be 0 or
1.}  This procedure returns the same result returned by {\cf
  fixnum-copy-bit}.
\end{entry}

\begin{entry}{%
\proto{fxbit-field}{ \vari{fx} \varii{fx} \variii{fx}}{procedure}}

\domain{\varii{Fx} and \variii{fx} must be non-negative and less than
  {\cf (fixnum-width)}.  Moreover, \varii{fx} must be less than or
    equal to \variii{fx}.}  This procedure returns the same result
  returned by {\cf fixnum-bit-field}.
\end{entry}

\begin{entry}{%
\proto{fxcopy-bit-field}{ \vari{fx} \varii{fx} \variii{fx} \variv{fx}}{procedure}}

\domain{\varii{Fx} and \variii{fx} must be non-negative and less than
  {\cf (fixnum-width)}.  Moreover, or \varii{fx} must be less than or
    equal to \variii{fx}.}  This procedure returns the same result
  returned by {\cf fixnum-copy-bit-field}.
\end{entry}

\begin{entry}{%
\proto{fxarithmetic-shift}{ \vari{fx} \varii{fx}}{procedure}}

\domain{The absolute value of the \varii{fx} must be less than 
{\cf (fixnum-width)}.}  If
%
\begin{scheme}
(exact-floor (* \vari{fx} (expt 2 \varii{fx})))
\end{scheme}
%
is a fixnum, then that fixnum is returned.  Otherwise an exception
with condition type {\cf\&implementation-restriction} is
raised.
\end{entry}

\begin{entry}{%
\proto{fxarithmetic-shift-left}{ \vari{fx} \varii{fx}}{procedure}
\proto{fxarithmetic-shift-right}{ \vari{fx} \varii{fx}}{procedure}}

\domain{\varii{Fx} must be non-negative.} {\cf
  Fxarithmetic-shift-left} behaves the same as {\cf
  fxarithmetic-shift}, and {\cf (fxarithmetic-shift-right \vari{fx}
  \varii{fx})} behaves the same as {\cf (fxarithmetic-shift \vari{fx}
  (fixnum- \varii{fx}))}.
\end{entry}

\begin{entry}{%
\proto{fxrotate-bit-field}{ \vari{fx} \varii{fx} \variii{fx} \variv{fx}}{procedure}}

\domain{\varii{Fx}, \variii{fx}, and \variv{fx} must be non-negative
  and less than {\cf (fixnum-width)}.  \variv{Fx} must be less than
  the difference between \variii{fx} and \variii{fx}.}  This procedure
returns the same result as the {\cf fixnum-rotate-bit-field}
procedure.
\end{entry}

\begin{entry}{%
\proto{fxreverse-bit-field}{ \vari{fx} \varii{fx} \variii{fx}}{procedure}}

\domain{\varii{Fx} and \variii{fx} must be non-negative and less than
  {\cf (fixnum-width)}.  Moreover, \varii{fx} must be less than or
  equal to \variii{fx}.}  This procedure returns the same result as
the {\cf fixnum-reverse-bit-field} procedure.
\end{entry}

\section{Flonums}
\label{flonumssection}

This section uses \var{fl}, \vari{fl} and \varii{fl} as
parameter names for arguments that must be flonums, and \var{ifl},
\vari{ifl} and \varii{ifl} as parameter names for arguments that 
must be integer-valued flonums, i.e. flonums for which the
{\cf integer-valued?} predicate is true.

\begin{entry}{%
\proto{flonum?}{ obj}{procedure}}

This returns \schtrue{} if \var{obj} is a flonum, and
otherwise returns \schfalse{}.
\end{entry}

\begin{entry}{%
\proto{fl=?}{ \vari{fl} \varii{fl} \variii{fl} \dotsfoo}{procedure}
\proto{fl<?}{ \vari{fl} \varii{fl} \variii{fl} \dotsfoo}{procedure}
\proto{fl<=?}{ \vari{fl} \varii{fl} \variii{fl} \dotsfoo}{procedure}
\proto{fl>?}{ \vari{fl} \varii{fl} \variii{fl} \dotsfoo}{procedure}
\proto{fl>=?}{ \vari{fl} \varii{fl} \variii{fl} \dotsfoo}{procedure}}

These procedures return \schtrue{} if their arguments are (respectively):
equal, monotonically increasing, monotonically decreasing,
monotonically nondecreasing, or monotonically nonincreasing,
\schfalse{} otherwise.  These
predicates are required to be transitive.

\begin{scheme}
(fl= +inf.0 +inf.0)           \ev  \schtrue{}
(fl= -inf.0 +inf.0)           \ev  \schfalse{}
(fl= -inf.0 -inf.0)           \ev  \schtrue{}
(fl= 0.0 -0.0)                \ev  \schtrue{}
(fl< 0.0 -0.0)                \ev  \schfalse{}
(fl= +nan.0 \var{fl})               \ev  \schfalse{}
(fl< +nan.0 \var{fl})               \ev  \schfalse{}
\end{scheme}
\end{entry}

\begin{entry}{%
\proto{flinteger?}{ fl}{procedure}
\proto{flzero?}{ fl}{procedure}
\proto{flpositive?}{ fl}{procedure}
\proto{flnegative?}{ fl}{procedure}
\proto{flodd?}{ ifl}{procedure}
\proto{fleven?}{ ifl}{procedure}
\proto{flfinite?}{ fl}{procedure}
\proto{flinfinite?}{ fl}{procedure}
\proto{flnan?}{ fl}{procedure}}

These numerical predicates test a flonum for a particular property,
returning \schtrue{} or \schfalse{}.
{\cf Flinteger?} tests it if the number is an integer,
{\cf flzero?} tests if
it is {\cf fl=?} to zero, {\cf flpositive?} tests if it is greater
than zero, {\cf flnegative?} tests if it is less
than zero, {\cf flodd?} tests if it is odd, 
{\cf fleven?} tests if it is even,
{\cf flfinite?} tests if it is not an infinity and not a NaN,
{\cf flinfinite?} tests if it is an infinity,
{\cf flnan?} tests if it is a NaN.

\begin{scheme}
(flnegative? -0.0)   \ev \schfalse{}
(flfinite? +inf.0)   \ev \schfalse{}
(flfinite? 5.0)      \ev \schtrue{}
(flinfinite? 5.0)    \ev \schfalse{}
(flinfinite? +inf.0) \ev \schtrue{}
\end{scheme}

\begin{note}
{\cf (flnegative? -0.0)} must return \schfalse{},
else it would lose the correspondence with
{(fl< -0.0 0.0)}, which is \schfalse{}
according to the IEEE standards.
\end{note}
\end{entry}

\begin{entry}{%
\proto{flmax}{ \vari{fl} \varii{fl} \dotsfoo}{procedure}
\proto{flmin}{ \vari{fl} \varii{fl} \dotsfoo}{procedure}}

These procedures return the maximum or minimum of their arguments.
\end{entry}

\begin{entry}{%
\proto{fl+}{ \vari{fl} \dotsfoo}{procedure}
\proto{fl*}{ \vari{fl} \dotsfoo}{procedure}}

These procedures return the flonum sum or product of their flonum
arguments.  In general, they should return the flonum that best
approximates the mathematical sum or product.  (For implementations
that represent flonums as IEEE binary floating point numbers, the
meaning of ``best'' is reasonably well-defined by the IEEE standards.)

\begin{scheme}
(fl+ +inf.0 -inf.0)      \ev  +nan.0
(fl+ +nan.0 \var{fl})          \ev  +nan.0
(fl* +nan.0 \var{fl})          \ev  +nan.0
\end{scheme}
\end{entry}

\begin{entry}{%
\proto{fl-}{ \vari{fl} \varii{fl} \dotsfoo}{procedure}
\rproto{fl-}{ fl}{procedure}
\proto{fl/}{ \vari{fl} \varii{fl} \dotsfoo}{procedure}
\rproto{fl/}{ fl}{procedure}}

With two or more arguments, these procedures return the flonum
difference or quotient of their flonum arguments, associating to the
left.  With one argument, however, they return the additive or
multiplicative flonum inverse of their argument.  In general, they
should return the flonum that best approximates the mathematical
difference or quotient.  (For implementations that represent flonums
as IEEE binary floating point numbers, the meaning of ``best'' is
reasonably well-defined by the IEEE standards.)

\begin{scheme}
(fl- +inf.0 +inf.0)      \ev  +nan.0
\end{scheme}

For undefined quotients, {\cf fl/} behaves as specified by the
IEEE standards:

\begin{scheme}
(fl/ 1.0 0.0)  \ev +inf.0
(fl/ -1.0 0.0) \ev -inf.0
(fl/ 0.0 0.0)  \ev +nan.0
\end{scheme}
\end{entry}

\begin{entry}{%
\proto{flabs}{ fl}{procedure}}

This returns the absolute value of its argument.
\end{entry}

\begin{entry}{%
\proto{fldiv+mod}{ \vari{fl} \varii{fl}}{procedure}
\proto{fldiv}{ \vari{fl} \varii{fl}}{procedure}
\proto{flmod}{ \vari{fl} \varii{fl}}{procedure}
\proto{fldiv0+mod0}{ \vari{fl} \varii{fl}}{procedure}
\proto{fldiv0}{ \vari{fl} \varii{fl}}{procedure}
\proto{flmod0}{ \vari{fl} \varii{fl}}{procedure}}

These procedures implement number-theoretic integer division and
return the results of the corresponding mathematical operations
specified in section~\ref{integerdivision}.  For zero divisors, these
procedures may return a NaN or some meaningless flonum.

\begin{scheme}
(fldiv \vari{fl} \varii{fl})         \ev \(\vari{fl}~\mathrm{div}~\varii{fl}\)
(flmod \vari{fl} \varii{fl})         \ev \(\vari{fl}~\mathrm{mod}~\varii{fl}\)
(fldiv+mod \vari{fl} \varii{fl})     \ev \(\vari{fl}~\mathrm{div}~\varii{fl}, \vari{fl}~\mathrm{mod}~\varii{fl}\)\\\>\>\>; two return values
(fldiv0 \vari{fl} \varii{fl})        \ev \(\vari{fl}~\mathrm{div}_0~\varii{fl}\)
(flmod0 \vari{fl} \varii{fl})        \ev \(\vari{fl}~\mathrm{mod}_0~\varii{fl}\)
(fldiv0+mod0 \vari{fl} \varii{fl})   \lev \(\vari{fl}~\mathrm{div}_0~\varii{fl}, \vari{fl}~\mathrm{mod}_0~\varii{fl}\)\\\>\>; two return values
\end{scheme}

\end{entry}

\begin{entry}{%
\proto{flnumerator}{ fl}{procedure}
\proto{fldenominator}{ fl}{procedure}}

These procedures return the numerator or denominator of their
argument as a flonum; the result is computed as if the argument was represented as
a fraction in lowest terms.  The denominator is always positive.  The
denominator of 0 is defined to be 1.
%
\begin{scheme}
(flnumerator +inf.0)           \ev  +inf.0
(flnumerator -inf.0)           \ev  -inf.0
(fldenominator +inf.0)         \ev  1.0
(fldenominator -inf.0)         \ev  1.0
(flnumerator 0.75)             \ev  3.0 ; example
(fldenominator 0.75)           \ev  4.0 ; example
\end{scheme}

The following behavior is strongly recommended but not required:

\begin{scheme}
(flnumerator -0.0)             \ev -0.0
\end{scheme}
\end{entry}

\begin{entry}{%
\proto{flfloor}{ fl}{procedure}
\proto{flceiling}{ fl}{procedure}
\proto{fltruncate}{ fl}{procedure}
\proto{flround}{ fl}{procedure}}

These procedures return integral flonums for flonum arguments that are
not infinities or NaNs.  For such arguments, {\cf flfloor} returns the
largest integral flonum not larger than \var{fl}.  {\cf Flceiling}
returns the smallest integral flonum not smaller than \var{fl}.  {\cf
  Fltruncate} returns the integral flonum closest to \var{fl} whose
absolute value is not larger than the absolute value of \var{fl}.
{\cf Flround} returns the closest integral flonum to \var{fl},
rounding to even when \var{fl} is halfway between two integers.

\begin{rationale}
{\cf Flround} rounds to even for consistency with the default rounding
mode specified by the IEEE floating point standard.
\end{rationale}

Although infinities and NaNs are not integers, these procedures return
an infinity when given an infinity as an argument, and a NaN when
given a NaN:

\begin{scheme}
(flfloor +inf.0)                       \ev  +inf.0
(flceiling -inf.0)                     \ev  -inf.0
(fltruncate +nan.0)                    \ev  +nan.0
\end{scheme}
\end{entry}

\begin{entry}{%
\proto{flexp}{ fl}{procedure}
\proto{fllog}{ fl}{procedure}
\rproto{fllog}{ \vari{fl} \varii{fl}}{procedure}
\proto{flsin}{ fl}{procedure}
\proto{flcos}{ fl}{procedure}
\proto{fltan}{ fl}{procedure}
\proto{flasin}{ fl}{procedure}
\proto{flatan}{ fl}{procedure}
\rproto{flatan}{ \vari{fl} \varii{fl}}{procedure}}

These procedures compute the usual transcendental functions.  
{\cf Flexp} computes the base-$e$ exponential of \var{fl}.
{\cf
  Fllog} with a single argument computes the natural logarithm of
\var{fl} (not the base ten logarithm); {\cf (fllog \vari{fl}
  \varii{fl})} computes the base-\varii{fl} logarithm of \vari{fl}.
{\cf Flasin}, {\cf flacos}, and {\cf flatan} compute arcsine,
arccosine, and arctangent, respectively.  {\cf (flatan \vari{fl}
  \varii{fl})} computes the arc tangent of \vari{fl}/\varii{fl}.

See section~\ref{transcendentalfunctions} for the underlying
mathematical operations.  In the event that these operations do not
yield a real result for the given arguments, the result may be a NaN,
or may be some meaningless flonum.

Implementations that use IEEE binary floating point arithmetic are
encouraged to follow the relevant standards for these procedures.

\begin{scheme}
(flexp +inf.0)                \ev +inf.0
(flexp -inf.0)                \ev 0.0
(fllog +inf.0)                \ev +inf.0
(fllog 0.0)                   \ev -inf.0
(fllog -0.0)                  \ev \unspecified\\\>\>; if -0.0 is distinguished
(fllog -inf.0)                \ev +nan.0
(flatan -inf.0)               \ev -1.5707963267948965\\\>\>; approximately
(flatan +inf.0)               \ev 1.5707963267948965\\\>\>; approximately
\end{scheme}
\end{entry}

\begin{entry}{%
\proto{flsqrt}{ fl}{procedure}}

Returns the principal square root of \var{fl}. For a negative argument,
the result may be a NaN, or may be some meaningless flonum.

\begin{scheme}
(flsqrt +inf.0)               \ev  +inf.0
\end{scheme}
\end{entry}

\begin{entry}{%
\proto{flexpt}{ \vari{fl} \varii{fl}}{procedure}}

Returns \vari{fl} raised to the power \varii{fl}.  \vari{fl} should be
non-negative; if \vari{fl} is negative, then the result may be a NaN,
or may be some meaningless flonum.  If \vari{fl} is zero, then the
result is zero.  For positive $\mathit{fl}_1$,% LaTeX kludge

\begin{displaymath}
  \mathit{fl}_1^{\mathit{fl}_2} = e^{\mathit{fl}_2 \log \mathit{fl}_1}
\end{displaymath}
\end{entry}

\begin{entry}{%
\ctproto{no-infinities}{condition type}
\proto{no-infinities?}{ obj}{procedure}
\ctproto{no-nans}{condition type}
\proto{no-nans?}{ obj}{procedure}}

These condition types could be defined by the following code:

\begin{scheme}
(define-condition-type \&no-infinities
    \&implementation-restriction
  no-infinities?)

(define-condition-type \&no-nans
    \&implementation-restriction
  no-nans?)
\end{scheme}

These types describe that a program has executed an arithmetic
operations that is specified to return an infinity or a NaN,
respectively, on a Scheme implementation that is not able to represent
the infinity or NaN.  (See section~\ref{infinitiesnanssection}.)
\end{entry}

\section{Fixnum/Flonum conversions}

\begin{entry}{%
\proto{fixnum->flonum}{ fx}{procedure}}

Returns a flonum that is numerically closest to its argument.

\begin{note}
The result of this procedure may not be
numerically equal to its argument, because the fixnum precision
may be greater than the flonum precision.
\end{note}
\end{entry}

\section{Exact arithmetic}

The exact arithmetic provides generic operations on exact numbers;
these operations correspond to their mathematical counterparts.  The
exact numbers include rationals of arbitrary precision, and exact
rectangular complex numbers.  A rational number with a denominator of
$1$ is indistinguishable from its numerator.  An exact rectangular
complex number with a zero imaginary part is indistinguishable from
its real part.

\begin{entry}{%
\proto{exact-number?}{ ex}{procedure}
\proto{exact-complex?}{ ex}{procedure}
\proto{exact-rational?}{ ex}{procedure}
\proto{exact-integer?}{ ex}{procedure}}

These numerical type predicates can be applied to any kind of
argument, including non-numbers.  They return \schtrue{} if the object
is an exact number of the named type, and otherwise return
\schfalse{}.  In general, if a type predicate is true of a number then
all higher type predicates are also true of that number.
Consequently, if a type predicate is false of a number, then all lower
type predicates are also false of that number.
\end{entry}

This section uses \var{ex}, \vari{ex}, \varii{ex}, and \variii{ex} as
parameter names for arguments that must be exact complex numbers, \var{ef},
\vari{ef}, \varii{ef}, and \variii{ef} as parameter names for arguments 
that must be exact rational numbers, and \var{ei}, \vari{ei}, \varii{ei}, and
\variii{ei} as parameter names that must be exact integers.

\begin{entry}{%
\proto{exact=?}{ \vari{ex} \varii{ex} \variii{ex} \dotsfoo}{procedure}
\proto{exact>?}{ \vari{ef} \varii{ef} \variii{ef} \dotsfoo}{procedure}
\proto{exact<?}{ \vari{ef} \varii{ef} \variii{ef} \dotsfoo}{procedure}
\proto{exact>=?}{ \vari{ef} \varii{ef} \variii{ef} \dotsfoo}{procedure}
\proto{exact<=?}{ \vari{ef} \varii{ef} \variii{ef} \dotsfoo}{procedure}}

These procedures return \schtrue{} if their arguments are
(respectively): equal, monotonically increasing, monotonically
decreasing, monotonically nondecreasing, or monotonically
nonincreasing \schfalse{} otherwise.
\end{entry}

\begin{entry}{%
\proto{exact-zero?}{ ex}{procedure}
\proto{exact-positive?}{ ef}{procedure}
\proto{exact-negative?}{ ef}{procedure}
\proto{exact-odd?}{ ei}{procedure}
\proto{exact-even?}{ ei}{procedure}}

These numerical predicates test an exact number for a particular
property, returning \schtrue{} or \schfalse{}.  {\cf Exact-zero?}
tests if the number is {\cf exact=?} to zero, {\cf exact-positive?}
tests if it is greater than zero, {\cf exact-negative?} tests if it is
less than zero, {\cf exact-odd?} tests if it is odd, {\cf exact-even?}
tests if it is even.
\end{entry}

\begin{entry}{%
\proto{exact-max}{ \vari{ef} \varii{ef} \dotsfoo}{procedure}
\proto{exact-min}{ \vari{ef} \varii{ef} \dotsfoo}{procedure}}

These procedures return the maximum or minimum of their arguments.
\end{entry}

\begin{entry}{%
\proto{exact+}{ \vari{ex} \varii{ex} \dotsfoo}{procedure}
\proto{exact*}{ \vari{ex} \varii{ex} \dotsfoo}{procedure}}

These procedures return the sum or product of their arguments.
\end{entry}

\begin{entry}{%
\proto{exact-}{ \vari{ex} \varii{ex} \dotsfoo}{procedure}
\rproto{exact-}{ ex}{procedure}
\proto{exact/}{ \vari{ex} \varii{ex} \dotsfoo}{procedure}
\rproto{exact/}{ ex}{procedure}}

With two or more arguments, these procedures return the difference or
quotient of their arguments, associating to the left.  With one
argument, however, they return the additive or multiplicative inverse
of their argument. {\cf Exact/} raises an exception with condition
type {\cf\&contract} if a divisor is $0$.
\end{entry}

\begin{entry}{%
\proto{exact-abs}{ ef}{procedure}}

This procedure returns the absolute value of its argument.
\end{entry}

\begin{entry}{%
\proto{exact-div+mod}{ \vari{ef} \varii{ef}}{procedure}
\proto{exact-div}{ \vari{ef} \varii{ef}}{procedure}
\proto{exact-mod}{ \vari{ef} \varii{ef}}{procedure}
\proto{exact-div0+mod0}{ \vari{ef} \varii{ef}}{procedure}
\proto{exact-div0}{ \vari{ef} \varii{ef}}{procedure}
\proto{exact-mod0}{ \vari{ef} \varii{ef}}{procedure}}

\domain{\varii{Ef} must be nonzero.}
These procedures implement number-theoretic integer division and
return the results of the corresponding mathematical operations
specified in section~\ref{integerdivision}.
%
\begin{scheme}
(exact-div \vari{ef} \varii{ef})         \ev \(\vari{ef}~\mathrm{div}~\varii{ef}\)
(exact-mod \vari{ef} \varii{ef})         \ev \(\vari{ef}~\mathrm{mod}~\varii{ef}\)
(exact-div+mod \vari{ef} \varii{ef})     \ev \(\vari{ef}~\mathrm{div}~\varii{ef}, \vari{ef}~\mathrm{mod}~\varii{ef}\)\\\>\>\>; two return values
(exact-div0 \vari{ef} \varii{ef})        \ev \(\vari{ef}~\mathrm{div}_0~\varii{ef}\)
(exact-mod0 \vari{ef} \varii{ef})        \ev \(\vari{ef}~\mathrm{mod}_0~\varii{ef}\)
(exact-div0+mod0 \vari{ef} \varii{ef})   \lev \(\vari{ef}~\mathrm{div}_0~\varii{ef}, \vari{ef}~\mathrm{mod}_0~\varii{ef}\)\\\>\>; two return values
\end{scheme}
\end{entry}

\begin{entry}{%
\proto{exact-gcd}{ \vari{ei} \varii{ei} \dotsfoo}{procedure}
\proto{exact-lcm}{ \vari{ei} \varii{ei} \dotsfoo}{procedure}}

These procedures return the greatest common divisor or least common
multiple of their arguments.
\end{entry}

\begin{entry}{%
\proto{exact-numerator}{ ef}{procedure}
\proto{exact-denominator}{ ef}{procedure}}

These procedures return the numerator or denominator of their argument.
The result is computed as if the argument was represented as a
fraction in lowest terms.  The denominator is always positive.  The
denominator of $0$ is defined to be $1$.
\end{entry}

\begin{entry}{%
\proto{exact-floor}{ ef}{procedure}
\proto{exact-ceiling}{ ef}{procedure}
\proto{exact-truncate}{ ef}{procedure}
\proto{exact-round}{ ef}{procedure}}

These procedures return exact integers.  {\cf Exact-floor} returns the
largest integer not larger than \var{ef}.  {\cf Exact-ceiling} returns
the smallest integer not smaller than \var{ef}.  {\cf Exact-truncate}
returns the integer closest to \var{ef} whose absolute value is not
larger than the absolute value of \var{ef}.  {\cf Exact-round} returns
the closest integer to \var{ef}, rounding to even when \var{ef} is
halfway between two integers.
\end{entry}

\begin{entry}{%
\proto{exact-expt}{ \vari{ef} \varii{ei}}{procedure}}

Returns \vari{ef} raised to the power \varii{ei}.  $0^{\var{ei}}$ is
$1$ if $\var{ei} = 0$ and $0$ if \var{ei} is positive.  Otherwise,
this procedure raises an exception with condition type {\cf\&contract}.
\end{entry}

\begin{entry}{%
\proto{exact-make-rectangular}{ \vari{ef} \varii{ef}}{procedure}
\proto{exact-real-part}{ ex}{procedure}
\proto{exact-imag-part}{ ex}{procedure}}

The arguments of {\cf exact-make-rectangular} must be exact
rationals.  Suppose $z$ is a complex number such that
%
\begin{displaymath}
z = \vari{ef} + \varii{ef}i\mathrm{.}
\end{displaymath}
%
Then:
\begin{scheme}
(exact-make-rectangular \vari{ef} \varii{ef}) \lev \var{z}
(exact-real-part \var{z})              \ev \vari{ef}
(exact-imag-part \var{z})              \ev \varii{ef}
\end{scheme}
\end{entry}

\begin{entry}{%
\proto{exact-sqrt}{ ei}{procedure}}

\domain{\var{Ei} must be a non-negative exact integer.}
{\cf Exact-sqrt} returns two non-negative exact
integers $s$ and $r$ where $\var{ei} = s^2 +
r$ and $\var{ei} < (s+1)^2$.
\end{entry}

\begin{entry}{%
\proto{exact-not}{ ei}{procedure}}

Returns the exact integer whose two's complement representation is the
one's complement of the two's complement representation of its
argument.
\end{entry}

\begin{entry}{%
\proto{exact-and}{ \vari{ei} \dotsfoo}{procedure}
\proto{exact-ior}{ \vari{ei} \dotsfoo}{procedure}
\proto{exact-xor}{ \vari{ei} \dotsfoo}{procedure}}

These procedures return the exact integer that is the bit-wise
``and,'' ``inclusive or,'' or ``exclusive or'' of the two's complement
representations of their arguments.  If they are passed only one
argument, they return that argument.  If they are passed no arguments,
they return the integer (either $-1$ or $0$) that acts as identity for
the operation.
\end{entry}

\begin{entry}{%
\proto{exact-if}{ \vari{ei} \varii{ei} \variii{ei}}{procedure}}

Returns the exact integer that is the result of the following computation:
\begin{scheme}
(exact-ior (exact-and \vari{ei} \varii{ei})
           (exact-and (exact-not \vari{ei}) \variii{ei}))
\end{scheme}
\end{entry}

\begin{entry}{%
\proto{exact-bit-count}{ ei}{procedure}}
 
If the argument is non-negative, this procedure returns the number of
1 bits in the two's complement representation of its argument.
Otherwise it returns the number of 0 bits in the two's complement
representation of its argument.
\end{entry}

\begin{entry}{%
\proto{exact-length}{ ei}{procedure}}

These procedures return the exact integer that is the result of the following
computation:
\begin{scheme}
(do ((result 0 (+ result 1))
     (bits (if (exact-negative? \var{ei})
               (exact-not \var{ei})
               \var{ei})
           (exact-arithmetic-shift bits -1)))
    ((exact-zero? bits)
     result))
\end{scheme}
\end{entry}

\begin{entry}{%
\proto{exact-first-bit-set}{ ei}{procedure}}

This procedures returns the index of the least significant $1$
bit in the two's complement representation of its argument.
If the argument is $0$, then $-1$ is returned.
\begin{scheme}
(exact-first-bit-set 0)        \ev  -1
(exact-first-bit-set 1)        \ev  0
(exact-first-bit-set -4)       \ev  2
\end{scheme}
\end{entry}

\begin{entry}{%
\proto{exact-bit-set?}{ \vari{ei} \varii{ei}}{procedure}}

\domain{\varii{Ei} must be a non-negative exact integer.}

Otherwise returns the result of the following computation:
\begin{scheme}
(not (exact-zero?
      (exact-and (exact-arithmetic-shift-left 1 \varii{ei})
                 \vari{ei})))
\end{scheme}
\end{entry}

\begin{entry}{%
\proto{exact-copy-bit}{ \vari{ei} \varii{ei} \variii{ei}}{procedure}}

\domain{\varii{Ei} must be non-negative, and \variii{ei}
must be either $0$ or $1$.}
This procedure returns the result of the following computation:
\begin{scheme}
(let* ((mask (exact-arithmetic-shift-left 1 \varii{ei})))
  (exact-if mask
            (exact-arithmetic-shift-left \variii{ei} \varii{ei})
            \vari{ei}))
\end{scheme}
\end{entry}

\begin{entry}{%
\proto{exact-bit-field}{ \vari{ei} \varii{ei} \variii{ei}}{procedure}}

\domain{\varii{Ei} and \variii{ei} must be non-negative, and
  \varii{fx} must be less than or equal to \variii{fx}.}  This
procedure returns the result of the following computation:
%
\begin{scheme}
(let* ((mask (exact-not
              (exact-arithmetic-shift-left -1 \variii{ei}))))
  (exact-arithmetic-shift-right (exact-and \vari{ei} mask)
                                \varii{ei}))
\end{scheme}
\end{entry}

\begin{entry}{%
\proto{exact-copy-bit-field}{ \vari{ei} \varii{ei} \variii{ei} \variv{ei}}{procedure}}

\domain{\varii{Fx} and \variii{fx} must be non-negative,
and \varii{fx} must be less than or equal to \variii{fx}.}
This procedure returns the result of the following computation:
%
\begin{scheme}
(let* ((to    \vari{ei})
       (start \varii{ei})
       (end   \variii{ei})
       (from  \variv{ei})
       (mask1
         (exact-arithmetic-shift-left -1 start))
       (mask2
         (exact-not
           (exact-arithmetic-shift-left -1 end)))
       (mask (exact-and mask1 mask2)))
  (exact-if mask
            (exact-arithmetic-shift-left from start)
            to))
\end{scheme}
\end{entry}

\begin{entry}{%
\proto{exact-arithmetic-shift}{ \vari{ei} \varii{ei}}{procedure}}

Returns the exact integer result of the following computation:
%
\begin{scheme}
(exact-floor (* \vari{ei} (expt 2 \varii{ei})))
\end{scheme}
\end{entry}

\begin{entry}{%
\proto{exact-arithmetic-shift-left}{ \vari{ei} \varii{ei}}{procedure}
\proto{exact-arithmetic-shift-right}{ \vari{ei} \varii{ei}}{procedure}}

\domain{\varii{Ei} must bge non-negative.}  {\cf
  Exact-arithmetic-shift-left} returns the same result as {\cf
  exact-arithmetic-shift}, and {\cf (exact-arithmetic-shift-right
  \vari{fx} \varii{fx})} returns the same result as {\cf
  (exact-arithmetic-shift \vari{fx} (exact- \varii{fx}))}.
\end{entry}

\begin{entry}{%
\proto{exact-rotate-bit-field}{ \vari{ei} \varii{ei} \variii{ei} \variv{ei}}{procedure}}

\domain{\varii{Ei}, \variii{ei}, \variv{ei} must be non-negative, and \variv{ei}
must be less than the difference between \varii{ei} and \variii{ei}.}
The procedure returns the result of the following computation:
%
\begin{scheme}
(let* ((n     \vari{ei})
       (start \varii{ei})
       (end   \variii{ei})
       (count \variv{ei})
       (width (exact- end start)))
  (if (exact-positive? width)
      (let* ((count (exact-mod count width))
             (field0
               (exact-bit-field n start end))
             (field1 (exact-arithmetic-shift-left
                       field0 count))
             (field2 (exact-arithmetic-shift-right
                       field0
                       (exact- width count)))
             (field (exact-ior field1 field2)))
        (exact-copy-bit-field n start end field))
      n))
\end{scheme}
\end{entry}

\begin{entry}{%
\proto{exact-reverse-bit-field}{ \vari{ei} \varii{ei} \variii{ei}}{procedure}}

\domain{\varii{Ei} and \variii{ei} must be non-negative, and
  \varii{ei} must be less than or equal to \variii{ei}.}  This procedure returns
the result obtained from the first argument by reversing the bit field
specified by the second and third arguments.
\begin{scheme}
(exact-reverse-bit-field \sharpsign{}b1010010 1 4)   \lev  88 ; \sharpsign{}b1011000
(exact-reverse-bit-field \sharpsign{}1010010 91 -4)  \lev  \var{error}
\end{scheme}
\end{entry}

\section{Inexact arithmetic}

The inexact arithmetic provides generic operations on inexact numbers.
The inexact numbers include inexact reals and inexact complex numbers,
both of which are distinguishable from the exact numbers.  The inexact
complex numbers include the flonums, and the procedures described here
behave consistently with the corresponding flonum procedures if passed
flonum arguments.

\begin{entry}{%
\proto{inexact-number?}{ obj}{procedure}
\proto{inexact-complex?}{ obj}{procedure}
\proto{inexact-real?}{ obj}{procedure}
\proto{inexact-rational?}{ obj}{procedure}
\proto{inexact-integer?}{ obj}{procedure}}

These numerical type predicates can be applied to any kind of
argument, including non-numbers.  They return \schtrue{} if the object
is an inexact number of the named type, and otherwise they return
\schfalse{}.  In general, if a type predicate is true of a number then
all higher type predicates are also true of that number.
Consequently, if a type predicate is false of a number, then all lower
type predicates are also false of that number.
\end{entry}

We will use \var{in}, \vari{in}, \varii{in}, and
\variii{in} as parameter names for arguments that must be inexact numbers,
\var{ir}, \vari{ir}, \varii{ir}, and \variii{ir} as
parameter names for arguments that must be inexact real numbers,
\var{if}, \vari{if}, \varii{if}, and \variii{if} as
parameter names for arguments that must be inexact rationals, and
\var{ii}, \vari{ii}, \varii{ii}, and \variii{ii} as
parameter names for arguments that must be the inexact integers.

\begin{entry}{%
\proto{inexact=?}{ \vari{in} \varii{in} \variii{in} \dotsfoo}{procedure}
\proto{inexact>?}{ \vari{ir} \varii{ir} \variii{ir} \dotsfoo}{procedure}
\proto{inexact<?}{ \vari{ir} \varii{ir} \variii{ir} \dotsfoo}{procedure}
\proto{inexact>=?}{ \vari{ir} \varii{ir} \variii{ir} \dotsfoo}{procedure}
\proto{inexact<=?}{ \vari{ir} \varii{ir} \variii{ir} \dotsfoo}{procedure}}

These procedures return \schtrue{} if their arguments are
(respectively): equal, monotonically increasing, monotonically
decreasing, monotonically nondecreasing, or monotonically
nonincreasing \schfalse{} otherwise.  These predicates are required to
be transitive.
\end{entry}

\begin{entry}{%
\proto{inexact-zero?}{ in}{procedure}
\proto{inexact-positive?}{ ir}{procedure}
\proto{inexact-negative?}{ ir}{procedure}
\proto{inexact-odd?}{ ii}{procedure}
\proto{inexact-even?}{ ii}{procedure}
\proto{inexact-finite?}{ in}{procedure}
\proto{inexact-infinite?}{ in}{procedure}
\proto{inexact-nan?}{ in}{procedure}}

These numerical predicates test an inexact number for a particular
property, returning \schtrue{} or \schfalse{}.  {\cf Inexact-zero?}
tests if the number is {\cf inexact=?} to zero, {\cf
  inexact-positive?} tests if it is greater than zero, {\cf
  inexact-negative?} tests if it is less than zero, {\cf inexact-odd?}
tests if it is odd, {\cf inexact-even?} tests if it is even, {\cf
  inexact-finite?} tests if it is not an infinity and not a NaN, {\cf
  inexact-infinite?} tests if it is an infinity, {\cf inexact-nan?}
tests if it is a NaN.
\end{entry}

\begin{entry}{%
\proto{inexact-max}{ \vari{ir} \varii{ir} \dotsfoo}{procedure}
\proto{inexact-min}{ \vari{ir} \varii{ir} \dotsfoo}{procedure}}

These procedures return the maximum or minimum of their arguments.
\end{entry}

\begin{entry}{%
\proto{inexact+}{ \vari{in} \varii{in} \dotsfoo}{procedure}
\proto{inexact*}{ \vari{in} \varii{in} \dotsfoo}{procedure}}

These procedures return the sum or product of their arguments.
\end{entry}

\begin{entry}{%
\proto{inexact-}{ \vari{in} \varii{in} \dotsfoo}{procedure}
\rproto{inexact-}{ in}{procedure}
\proto{inexact/}{ \vari{in} \varii{in} \dotsfoo}{procedure}
\rproto{inexact/}{ in}{procedure}}

With two or more arguments, these procedures return the difference or
quotient of their arguments, associating to the left.  With one
argument, however, they return the additive or multiplicative inverse
of their argument.
\end{entry}

\begin{entry}{%
\proto{inexact-abs}{ in}{procedure}}

This procedure returns the absolute value of its argument.
\end{entry}

\begin{entry}{%
\proto{inexact-div+mod}{ \vari{ir} \varii{ir}}{procedure}
\proto{inexact-div}{ \vari{ir} \varii{ir}}{procedure}
\proto{inexact-mod}{ \vari{ir} \varii{ir}}{procedure}
\proto{inexact-div0+mod0}{ \vari{ir} \varii{ir}}{procedure}
\proto{inexact-div0}{ \vari{ir} \varii{ir}}{procedure}
\proto{inexact-mod0}{ \vari{ir} \varii{ir}}{procedure}}

\domain{\vari{Ir}
must be neither infinite nor a NaN, and \varii{ir} must be nonzero.}
These procedures implement number-theoretic integer division and
return the results of the corresponding mathematical operations
specified in section~\ref{integerdivision}.  

\begin{scheme}
(inexact-div \vari{ir} \varii{ir})         \ev \(\vari{ir}~\mathrm{div}~\varii{ir}\)
(inexact-mod \vari{ir} \varii{ir})         \ev \(\vari{ir}~\mathrm{mod}~\varii{ir}\)
(inexact-div+mod \vari{ir} \varii{ir})     \lev \(\vari{ir}~\mathrm{div}~\varii{ir}, \vari{ir}~\mathrm{mod}~\varii{ir}\)\\\>\>; two return values
(inexact-div0 \vari{ir} \varii{ir})        \ev \(\vari{ir}~\mathrm{div}_0~\varii{ir}\)
(inexact-mod0 \vari{ir} \varii{ir})        \ev \(\vari{ir}~\mathrm{mod}_0~\varii{ir}\)
(inexact-div0+mod0 \vari{ir} \varii{ir})   \lev \(\vari{ir}~\mathrm{div}_0~\varii{ir}, \vari{ir}~\mathrm{mod}_0~\varii{ir}\)\\\>\>; two return values
\end{scheme}
\end{entry}

\begin{entry}{%
\proto{inexact-gcd}{ \vari{ii} \varii{ii} \dotsfoo}{procedure}
\proto{inexact-lcm}{ \vari{ii} \varii{ii} \dotsfoo}{procedure}}

These procedures return the greatest common divisor or least common
multiple of their arguments.
\end{entry}

\begin{entry}{%
\proto{inexact-numerator}{ if}{procedure}
\proto{inexact-denominator}{ if}{procedure}}

These procedures return the numerator or denominator of their
argument.  The result is computed as if the argument was represented
as a fraction in lowest terms.  The denominator is always positive.
The denominator of $0$ is defined to be $1$.
\end{entry}

\begin{entry}{%
\proto{inexact-floor}{ ir}{procedure}
\proto{inexact-ceiling}{ ir}{procedure}
\proto{inexact-truncate}{ ir}{procedure}
\proto{inexact-round}{ ir}{procedure}}

These procedures return inexact integers for real arguments that are
not infinities or NaNs.  For such arguments, {\cf inexact-floor}
returns the largest integer not larger than \var{ir}.  {\cf
  Inexact-ceiling} returns the smallest integer not smaller than
\var{ir}.  {\cf Inexact-truncate} returns the integer closest to
\var{in} whose absolute value is not larger than the absolute value of
\var{in}.  {\cf Inexact-round} returns the closest integer to
\var{in}, rounding to even when \var{in} is halfway between two
integers.

\begin{rationale}
{\cf Round} rounds to even for consistency with the default rounding
mode specified by the IEEE floating point standard.
\end{rationale}

Although infinities and NaNs are not integers, these procedures return
an infinity when given an infinity as an argument, and a NaN when
given a NaN.
\end{entry}

\begin{entry}{%
\proto{inexact-exp}{ in}{procedure}
\proto{inexact-log}{ in}{procedure}
\rproto{inexact-log}{ \vari{in} \varii{in}}{procedure}
\proto{inexact-sin}{ in}{procedure}
\proto{inexact-cos}{ in}{procedure}
\proto{inexact-tan}{ in}{procedure}
\proto{inexact-asin}{ in}{procedure}
\proto{inexact-atan}{ in}{procedure}
\rproto{inexact-atan}{ \vari{ir} \varii{ir}}{procedure}}

These procedures compute the usual transcendental functions.  {\cf
  Inexact-exp} computes the base-$e$ exponential of \var{in}.  {\cf
  Inexact-log} with a single argument computes the natural logarithm
of \var{in} (not the base ten logarithm); {\cf (inexact-log \vari{in}
  \varii{in})} computes the base-\varii{in} logarithm of \vari{in}.
{\cf Inexact-asin}, {\cf Inexact-acos}, and {\cf Inexact-atan} compute
arcsine, arccosine, and arctangent, respectively.  The two-argument
variant of {\cf Inexact-atan} computes {\cf (inexact-angle
  (inexact-make-rectangular \vari{ir} \varii{ir}))} (see below).

See section~\ref{transcendentalfunctions} for the underlying
mathematical operations.  In the event that these operations do not
yield a real result for the given arguments, the result may be {\cf
  +nan.0}, or may be some meaningless inexact number.
\end{entry}

\begin{entry}{%
\proto{inexact-sqrt}{ in}{procedure}}

Returns the principal square root of \var{z}. For a negative argument,
the result may be {\cf +nan.0}, or may be some meaningless inexact
number.  With $\log$ defined as in
section~\ref{transcendentalfunctions}, the value of {\cf (inexact-sqrt
  \var{in})} could be expressed as

\begin{displaymath}
e^{\frac{\log \var{in}}{2}}.
\end{displaymath}
\end{entry}

\begin{entry}{%
\proto{inexact-expt}{ \vari{in} \varii{in}}{procedure}}

Returns \vari{in} raised to the power \varii{in}.  For nonzero
\vari{in},
%
\begin{displaymath}
  \vari{in}^{\varii{in}} = e^{\varii{in} \log \vari{in}}
\end{displaymath}
%
$0.0^{\var{in}}$ is $1$ if $\var{in} = 0.0$, and $0.0$ if {\cf
  (inexact-real-part \var{in})} is positive.  Otherwise, this
procedure reports a violation of an implementation restriction or
returns an unspecified number.
\end{entry}

\begin{entry}{%
\proto{inexact-make-rectangular}{ \vari{ir} \varii{ir}}{procedure}
\proto{inexact-make-polar}{ \vari{ir} \varii{ir}}{procedure}
\proto{inexact-real-part}{ in}{procedure}
\proto{inexact-imag-part}{ in}{procedure}
\proto{inexact-magnitude}{ in}{procedure}
\proto{inexact-angle}{ in}{procedure}}

Suppose \vari{ir}, \varii{ir}, \variii{ir}, and \variv{ir} are inexact
real numbers, and \var{z} is a complex number, such that
%
\begin{displaymath}
\var{z} = \vari{ir} + \varii{ir}i = \variii{ir} e^{i\variv{ir}}.
\end{displaymath}
%
Then (inexactly):
%
\begin{scheme}
(inexact-make-rectangular \vari{ir} \varii{ir}) \lev \var{z}
(inexact-make-rectangular \variii{ir} \variv{ir}) \lev \var{z}
(inexact-real-part \var{z})              \ev \vari{ir}
(inexact-imag-part \var{z})              \ev \varii{ir}
(inexact-magnitude \var{z})              \ev |\variii{ir}|
(inexact-angle \var{z})                  \ev \var{ir}\(_{\mathrm{angle}}\)
\end{scheme}
%
where $-\pi \leq \var{ir}_{\mathrm{angle}} \leq \pi$ with
$\var{ir}_{\mathrm{angle}} = \variv{ir} + 2\pi n$ for
some integer $n$.

\begin{scheme}
(inexact-angle -1.0)         \ev \(\pi\)
(inexact-angle -1.0+0.0)     \ev \(\pi\)
(inexact-angle -1.0-0.0)     \ev -\(\pi\)\\\>; if -0.0 is distinguished
\end{scheme}

Moreover, suppose \vari{ir}, \varii{ir} are such that either \vari{ir}
or \varii{ir} is an infinity, then
%
\begin{scheme}
(inexact-make-rectangular \vari{ir} \varii{ir}) \lev \var{z}
(inexact-magnitude \var{z})              \ev +inf.0
\end{scheme}
\end{entry}

%%% Local Variables: 
%%% mode: latex
%%% TeX-master: "r6rs"
%%% End: 

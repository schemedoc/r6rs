% 1. Structure of the language

\chapter{Overview of Scheme}
\label{semanticchapter}

This chapter gives an overview of Scheme's semantics.  A
detailed informal semantics is the subject of
the following chapters.  For reference
purposes, chapter~\ref{formalsemanticschapter} provides a formal
semantics for a core subset of Scheme.

\vest Following Algol, Scheme is a statically scoped programming
language.  Each use of a variable is associated with a lexically
apparent binding of that variable.

\vest Scheme has latent as opposed to manifest types
\cite{WaiteGoos}.  Types
are associated with values (also called objects\mainindex{object}) rather than
with variables.  (Some authors refer to languages with latent types as
weakly typed or dynamically typed languages.)  Other languages with
latent types are Python, Ruby, Smalltalk, and other dialects of Lisp.  Languages
with manifest types (sometimes referred to as strongly typed or
statically typed languages) include Algol 60, C, C\#, Java, Haskell and ML.

\vest All objects created in the course of a Scheme computation, including
procedures and continuations, have unlimited extent.
No Scheme object is ever destroyed.  The reason that
implementations of Scheme do not (usually!)\ run out of storage is that
they are permitted to reclaim the storage occupied by an object if
they can prove that the object cannot possibly matter to any future
computation.  Other languages in which most objects have unlimited
extent include C\#, Haskell, ML, Python, Ruby, Smalltalk and other Lisp dialects.

\vest Implementations of Scheme are required to be properly tail-recursive.
This allows the execution of an iterative computation in constant space,
even if the iterative computation is described by a syntactically
recursive procedure.  Thus with a properly tail-recursive implementation,
iteration can be expressed using the ordinary procedure-call
mechanics, so that special iteration constructs are useful only as
syntactic sugar.  See section~\ref{proper tail recursion}.

\vest Scheme was one of the first languages to support procedures as
objects in their own right.  Procedures can be created dynamically,
stored in data structures, returned as results of procedures, and so
on.  Other languages with these properties include Common Lisp,
Haskell, ML, Smalltalk, and Ruby.

\vest One distinguishing feature of Scheme is that continuations, which
in most other languages only operate behind the scenes, also have
``first-class'' status.  Continuations are useful for implementing a
wide variety of advanced control constructs, including non-local exits,
backtracking, and coroutines.  See section~\ref{continuations}.

\vest Arguments to Scheme procedures are always passed by value, which
means that the actual argument expressions are evaluated before the
procedure gains control, whether the procedure needs the result of the
evaluation or not.  C, C\#, Common Lisp, Python, Ruby, and Smalltalk
are other languages that always pass arguments by value.  This is
distinct from the lazy-evaluation semantics of Haskell, or the
call-by-name semantics of Algol 60, where an argument expression is
not evaluated unless its value is needed by the procedure.  Note that
call-by-value refers to a different distinction than the distinction
between by-value and by-reference passing in Pascal.  In Scheme, all
data structures are passed by-reference.

\vest Scheme's model of arithmetic is designed to remain as
independent as possible of the particular ways in which numbers are
represented within a computer. In Scheme, every integer is a rational
number, every rational is a real, and every real is a complex number.
Scheme distinguishes between exact arithmetic, which corresponds to
the mathematical ideal, and inexact arithmetic on approximations.
Exact arithmetic includes arithmetic on integers, rationals and
complex numbers.

The following sections give a brief overview of the most fundamental
elements of the language.  The purpose of this overview is to explain
enough of the basic concepts of the language to facilitate
understanding of the subsequent chapters of the report, which are
organized as a reference manual.  Consequently, this overview is
not a complete introduction of the language, nor is it precise
in all respects.

\section{Basic types}

Scheme programs manipulate \textit{values}, which are also referred
to as \textit{objects}.
Scheme values are organized into sets of values called \textit{types}.
This gives an overview of the fundamentally important types of the
Scheme language.  More types are described in later chapters.

\begin{note}
  As Scheme is latently typed, the use of the term \textit{type} in
  this report differs from the use of the term in the context of other
  languages, particularly those with manifest typing.
\end{note}

\paragraph{Boolean values}

\mainindex{boolean}A boolean value denotes a truth value, and can either be
true or false.  In Scheme, the value for ``false'' is written
\schfalse{}.  The value ``true'' is written \schtrue{}.  In
most places where a truth value is expected, however, any value different from
\schfalse{} counts as true.

\paragraph{Numbers}

\mainindex{number}Scheme supports a rich variety of numerical data types, including
integers of arbitrary precision, rational numbers, complex numbers and
inexact numbers of various kinds.  Chapter~\ref{numbertypeschapter} gives an
overview of the structure of Scheme's numerical tower.

\paragraph{Characters}

\mainindex{character}Scheme characters mostly correspond to textual characters.
More precisely, they are isomorphic to the \textit{scalar values} of
the Unicode standard.

\paragraph{Strings}

\mainindex{string}Strings are finite sequences of characters with fixed length and thus
represent arbitrary Unicode texts.

\paragraph{Symbols}

\mainindex{symbol}A symbol is an object representing a string that
cannot be modified.  This string is called the symbol's \textit{name}.
Unlike strings, two symbols whose names are spelled the same
way are indistinguishable.  Symbols are useful for many applications;
for instance, they may be used the way enumerated values are used in
other languages.

\paragraph{Pairs and lists}

\mainindex{pair}\mainindex{list}
A pair is a data structure with two components.  The most common use
of pairs is to represent (singly linked) lists, where the first
component (the ``car'') represents the first element of the list, and
the second component (the ``cdr'') the rest of the list.  Scheme also
has a distinguished empty list, which is the last cdr in a chain of
pairs representing a list.

\paragraph{Vectors}

\mainindex{vector}Vectors, like lists, are linear data structures
representing finite sequences of arbitrary objects.
Whereas the elements of a list are accessed
sequentially through the pair chain representing it,
the elements of a vector are addressed by an integer index.
Thus, vectors are more appropriate than
lists for random access to elements.

\paragraph{Procedures}

\mainindex{procedure}As mentioned in the introduction, procedures are
values in Scheme.

\section{Expressions}

The most important elements of a Scheme program are
\mainindex{expression}\textit{expressions}.  Expressions can be
\textit{evaluated}, producing a \textit{value}.  (Actually, any number
of values---see section~\ref{multiplereturnvaluessection}.)  The most
fundamental expressions are literal expressions:

\begin{scheme}
\schtrue{} \ev \schtrue
23 \ev 23%
\end{scheme}

This notation means that the expression \schtrue{} evaluates to
\schtrue{}, that is, the value for ``true,''  and that the expression
{\cf 23} evaluates to the number 23.

Compound expressions are formed by placing parentheses around their
subexpressions.  The first subexpression is an
\textit{operator}\mainindex{operator} and identifies an operation; the
remaining subexpressions are \textit{operands}\mainindex{operand}:
%
\begin{scheme}
(+ 23 42) \ev 65
(+ 14 (* 23 42)) \ev 980%
\end{scheme}
%
In the first of these examples, {\cf +}, the operator, is the name of
the built-in operation for addition, and {\cf 23} and {\cf 42} are the
operands.  The expression {\cf (+ 23 42)} reads as ``the sum of 23 and
42.''  Compound expressions can be nested---the second example reads
as ``the sum of 14 and the product of 23 and 42.''

As these examples indicate, compound expressions in Scheme are always
written using the same prefix notation\mainindex{prefix notation}.  As
a consequence, the parentheses are needed to indicate structure, and
``superfluous'' parentheses, which are permissible in mathematics and
many programming languages, are not allowed in Scheme.

As in many other languages, whitespace and newlines are not
significant when they separate subexpressions of an expression, and
can be used to indicate structure.

\section{Variables and binding}

\mainindex{variable}\mainindex{binding}\mainindex{identifier}Scheme
allows identifiers to denote values.  These
identifiers are called variables. (More precisely, variables denote
locations.  This distinction is not important, however, for a large
proportion of Scheme code.)

\begin{scheme}
(let ((x 23)
      (y 42))
  (+ x y)) \ev 65%
\end{scheme}

In this case, the operator of the expression, {\cf let}, is a binding
construct.  The parenthesized structure following the {\cf let} lists
variables alongside expressions: the variable {\cf x} alongside {\cf
  23}, and the variable {\cf y} alongside {\cf 42}.  The {\cf let}
expression binds {\cf x} to 23, and {\cf y} to 42.  These bindings are
available in the \textit{body} of the {\cf let} expression, {\cf (+ x
  y)}, and only there.

\section{Definitions}

\index{definition}The variables bound by a {\cf let} expression
are \textit{local}, because their bindings are visible only in the
{\cf let}'s body.  Scheme also allows creating top-level bindings for
identifiers as follows:

\begin{scheme}
(define x 23)
(define y 42)
(+ x y) \ev 65%
\end{scheme}

(These are actually ``top-level'' in the body of a library or script;
see section~\ref{librariesintrosection} below.)

The first two parenthesized structures are \textit{definitions}; they
create top-level bindings, binding {\cf x} to 23 and {\cf y} to 42.
Definitions are not expressions, and cannot appear in all places
where an expression can occur.  Moreover, a definition has no value.

Bindings follow the lexical structure of the program:  When several
bindings with the same name exist, a variable refers to the binding
that is closest to it, starting with its occurrence in the program
and going from inside to outside, going all the way to a top-level
binding only if no
local binding can be found along the way:

\begin{scheme}
(define x 23)
(define y 42)
(let ((y 43))
  (+ x y)) \ev 66

(let ((y 43))
  (let ((y 44))
    (+ x y))) \ev 67%
\end{scheme}

\section{Procedures}
\label{proceduressection}

\index{procedure}Definitions can also be used to define
procedures:

\begin{scheme}
(define (f x)
  (+ x 42))

(f 23) \ev 65%
\end{scheme}

A procedure is, slightly simplified, an abstraction over an
expression.  In the example, the first definition defines a procedure
called {\cf f}.  (Note the parentheses around {\cf f x}, which
indicate that this is a procedure definition.)  The expression {\cf (f
  23)} is a \index{procedure call}procedure call, meaning,
roughly, ``evaluate {\cf (+ x 42)} (the body of the procedure) with
{\cf x} bound to 23.''

As procedures are regular values, they can be passed to other
procedures:
%
\begin{scheme}
(define (f x)
  (+ x 42))

(define (g p x)
  (p x))

(g f 23) \ev 65%
\end{scheme}

In this example, the body of {\cf g} is evaluated with {\cf p}
bound to {\cf f} and {\cf x} bound to 23, which is equivalent
to {\cf (f 23)}, which evaluates to 42.

In fact, many predefined operations of Scheme are
bindings for procedures.  {\cf +}, for example, which receives
special syntactic treatment in many other languages, is just a regular
identifier in Scheme, bound to a procedure that adds numbers.  The
same holds for {\cf *} and many others:

\begin{scheme}
(define (h op x y)
  (op x y))

(h + 23 42) \ev 65
(h * 23 42) \ev 966%
\end{scheme}

Procedure definitions are not the only way to create procedures.  A
{\cf lambda} expression creates a new procedure as a value, with no
need to specify a name:

\begin{scheme}
((lambda (x) (+ x 42)) 23) \ev 65%
\end{scheme}

The entire expression in this example is a procedure call; its
operator is {\cf (lambda (x) (+ x 42))}, which evaluates to a
procedure that takes a single number and add it to 42.

\section{Procedure calls and syntactic keywords}

Whereas {\cf (+ 23 42)}, {\cf (f 23)}, and {\cf ((lambda (x) (+ x 42))
  23)} are all examples of procedure calls, {\cf lambda} and {\cf
  let} expressions are not.  This is because {\cf let}, even though
it is an identifier, is not a variable, but is instead a \textit{syntactic
  keyword}\index{syntactic keyword}.  An expression that has a
syntactic keyword as its operator obeys special rules determined by
the keyword.  The {\cf define} identifier in a definition is also a
syntactic keyword.  Hence, definitions are also not procedure calls.

In the case of {\cf lambda}, these rules specify that the first
subform is a list of parameters, and the second subform is the body of
the procedure.  In {\cf let} expressions, the first subform is a list
of binding specifications, and the second is a body of
expressions.

Procedure calls can be distinguished from these ``special forms'' by
looking for a syntactic keyword in the first position of an
expression: if it is not a syntactic keyword, the expression
is a procedure call.  The set of syntactic keywords of Scheme is
fairly small, which usually makes this task fairly simple.
It is possible, however, to create new bindings for syntactic keywords; see
below.

\section{Assignment}

Scheme variables bound by definitions or {\cf let} or {\cf lambda}
forms are not actually bound directly to the values specified in the
respective bindings, but to locations containing these values.  The
contents of these locations can subsequently be modified destructively
via \textit{assignment}\index{assignment}:

%
\begin{scheme}
(let ((x 23))
  (set! x 42)
  x) \ev 42%
\end{scheme}

In this case, the body of the {\cf let} expression consists of two
expressions which are evaluated sequentially, with the value of the
final expression becoming the value of the entire {\cf let}
expression.  The expression {\cf (set! x 42)} is an assignment, saying
``replace the value in the location denoted by {\cf x} with 42.''
Thus, the previous value of 23 is replaced by 42.

\section{Derived forms and macros}

Many of the special forms specified in this report
can be translated into more basic special forms.
For example, {\cf let} expressions can be translated
into procedure calls and {\cf lambda} expressions.  The following two
expressions are equivalent:
%
\begin{scheme}
(let ((x 23)
      (y 42))
  (+ x y)) \ev 65

((lambda (x y) (+ x y)) 23 42) \lev 65%
\end{scheme}

Special forms like {\cf let} expressions are called \textit{derived
  forms}\index{derived form} because their semantics can be
derived from that of other kinds of forms by a syntactic
transformation.  Procedure definitions are also derived forms.  The
following two definitions are equivalent:

\begin{scheme}
(define (f x)
  (+ x 42))

(define f
  (lambda (x)
    (+ x 42)))%
\end{scheme}

In Scheme, it is possible for a program to create its own derived
forms by binding syntactic keywords to macros\index{macro}:

\begin{scheme}
(define-syntax def
  (syntax-rules ()
    ((def f (p ...) body)
     (define (f p ...)
       body))))

(def f (x)
  (+ x 32))%
\end{scheme}

The {\cf define-syntax} construct specifies that a parenthesized
structure matching the pattern {\cf (def f (p ...) body)}, where {\cf
  f}, {\cf p}, and {\cf body} are pattern variables, is translated to
{\cf (define (f p ...) body)}.  Thus, the {\cf def} form appearing in
the example gets translated to:

\begin{scheme}
(define (f x)
  (+ x 42))%
\end{scheme}

The ability to create new syntactic keywords makes Scheme extremely
flexible and expressive, enabling the formulation of many features
built into other languages as derived forms.

\section{Syntactic datums and datum values}

A subset of the Scheme values called \textit{datum
  values}\index{datum value} have a special
status in the language.  These include booleans, numbers, characters,
and strings as well as lists and vectors whose elements are datums.  Each
datum value may be represented in textual form as a
\textit{syntactic datum}\index{syntactic datum}, which can be written out
and read back in without loss of information.  
Several syntactic datums can represent the same datum value,
but the datum value corresponding to a syntactic datum is uniquely determined.
Moreover, each datum value
can be trivially translated to a literal expression in a program by
prepending a {\cf\singlequote} to a corresponding syntactic datum:

\begin{scheme}
'23 \ev 23
'\schtrue{} \ev \schtrue{}
'foo \ev foo
'(1 2 3) \ev (1 2 3)
'\#(1 2 3) \ev \#(1 2 3)%
\end{scheme}

The {\cf\singlequote} is, of course, not needed for number or boolean
literals.
The identifier {\cf foo} is a syntactic datum that can represent a
symbol with name ``foo,'' and {\cf 'foo} is a literal expression with
that symbol as its value.  {\cf (1 2 3)} is a syntactic datum that can
represent a list with elements 1, 2, and 3, and {\cf '(1 2 3)} is a literal
expression with this list as its value.  Likewise, {\cf \#(1 2 3)}
is a syntactic datum that can represent a vector with elements 1, 2 and 3, and
{\cf '\#(1 2 3)} is the corresponding literal.

The syntactic datums form a superset of the Scheme forms.  Thus, datums
can be used to represent Scheme programs as data objects.  In
particular, symbols can be used to represent identifiers.

\begin{scheme}
'(+ 23 42) \ev (+ 23 42)
'(define (f x) (+ x 42)) \lev (define (f x) (+ x 42))%
\end{scheme}

This facilitates writing programs that operate on Scheme source code,
in particular intpreters and program transformers.

\section{Libraries}
\label{librariesintrosection}

Scheme code is organized in components called
\textit{libraries}\index{library}.  Each library contains 
declarations, definitions and expressions.  It can import definitions
from other libraries, and export definitions to other libraries:

\begin{scheme}
(library (hello)
  (export)
  (import (r6rs base)
          (r6rs i/o simple))
  (display "Hello World")
  (newline))%
\end{scheme}

\section{Scripts}

A Scheme program is invoked via a \textit{script}\index{script}.
Like a library, a script contains declarations, definitions and
expressions, but specifies an entry point for execution.  Thus, a
script defines, via the transitive closure of the libraries it
imports, a Scheme program.

\begin{scheme}
\#! /usr/bin/env scheme-script
\#!r6rs
(import (r6rs base)
        (r6rs i/o ports))
(put-bytes (standard-output-port)
           (call-with-port
               (open-file-input-port
                 (cadr (command-line-arguments)))
             get-bytes-all))
0%
\end{scheme}

%%% Local Variables: 
%%% mode: latex
%%% TeX-master: "r6rs"
%%% End: 

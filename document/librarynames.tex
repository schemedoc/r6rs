\chapter{Unique library names}
\label{librarynamesappendix}

Programmers are encouraged choose names for distributed libraries
whose names are chosen not to collide with other libraries' names.
This appendix suggests a convention for generating unique library
names, similar to the convention for Java~\cite{JLS3}.

A unique library name can be formed by associating the library with an
Internet domain name, such as {\cf mit.edu}.  The lower-case
components of the domain are reversed to form a prefix for the library
name.  Adding further name components to establish a hierarchy may be
advisable, depending on the size of the organization associated with
the domain name, the number of libraries to be distributed from it,
and other organizational properties or conventions associated with the
library.

Programmers are encouraged to use library names that are suitable for
use in the file-system mapping described in
appendix~\ref{filesystemmappingappendix}.  Special characters in
domain names that do not fit the convention should be replaced by
hyphens or suitable ``escape sequences'' that, as much as possible,
are suitable for avoiding collisions.  Here are some examples for
possible library names according to this convention:
%
\begin{scheme}
(edu mit swiss cheese)
(de deinprogramm educational graphics turtle)
(com pan-am booking passenger)%
\end{scheme}
%
The name of a library does not necessarily indicate an Internet
address where the package is distributed.

%%% Local Variables: 
%%% mode: latex
%%% TeX-master: "r6rs-app"
%%% End: 

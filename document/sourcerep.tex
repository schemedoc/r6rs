\chapter{Source code representation}
\label{sourcecoderepappendix}

The report does not specify how source code is represented and stored.
The only requirement the report imposes is that the source code of a
top-level program (see report
section~\extref{report:programsyntaxsection}{Top-level program
  syntax}) or a script (see section~\ref{scriptsyntaxsection}) be
delimited.  The source code of a library is self-delimiting in the
sense that, if the beginning of a library form can be identified, so
can the end.

Implementations might take radically different approaches to storing
source code for libraries, among them:
%
\begin{itemize}
\item files in the file system where each file contains an
  arbitrary number of library forms,
\item files in the file system where each file contains an
  exactly one library forms,
\item records in a database, and
\item data structures in memory.
\end{itemize}
%
Similarly, programs and scripts might be stored in a variety of
formats.  Platform constraints might restrict the choices available to
an implementation, which is why the report neither mandates nor
recommends no specific method for storage.

Implementations may provide a means for importing libraries coming
from the outside via an interface that accepts a UTF-8 text file in
Unicode Normalization Form C where line endings are encoded as newline
characters.  Such text files may contain an arbitrary number of
library forms.  (Authors of such files are encouraged to include an
{\cf\#!r6rs} comment if the file is written purely with the lexical
and read syntax described in the report.  See report
section~\extref{report:whitespaceandcomments}{Whitespace and
  comments}.)  After importing such a file, the libraries defined in
it should be available to other libraries and files.  An
implementation may store the file as is, or convert it to some storage
format to achieve this.

Similarly, implementations may provide a means for executing a program
represented as a UTF-8 text file containing its source code.  Again,
authors of such files are encouraged to include an {\cf\#!r6rs}
comment if the file is written purely with the lexical and read syntax
described in the report.  This report does not describe a file format
that allows both libraries and programs are not expected to appear in
the same file.

%%% Local Variables: 
%%% mode: latex
%%% TeX-master: "r6rs-app"
%%% End: 

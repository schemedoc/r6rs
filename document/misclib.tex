\chapter{Miscellaneous libraries}
\label{misclibchapter}

\chapter{List utilities}
\label{listutilities}

This chapter describes the \deflibrary{r6rs lists} library.

\begin{entry}{%
\proto{find}{ proc list}{procedure}}

\domain{\var{Proc} must be a procedure; it must take a single argument
  if \var{list} is non-empty.}  The {\cf find} procedure applies
\var{proc} to the elements of \var{list} in order.  If
\var{proc} returns a true value for an element, {\cf find}
immediately returns that element.  If \var{proc} returns
\schfalse{} for all elements of the list, it returns \schfalse{}.

\begin{scheme}
(find even? '(3 1 4 1 5 9)) \ev 4
(find even? '(3 1 5 1 5 9)) \ev \schfalse{}
\end{scheme}
  
\end{entry}

\begin{entry}{%
\proto{forall}{ proc \vari{l} \varii{l} \dotsfoo{} \varn{l}}{procedure}
\proto{exists}{ proc \vari{l} \varii{l} \dotsfoo{} \varn{l}}{procedure}}

\domain{Each \var{l} must be the
  empty list or a chain of pairs according to the conditions specified
  below.  \var{Proc} must be a procedure; it must take as many
  arguments as there are \var{l}s if \vari{l} is non-empty.}

For natural numbers $i = 0, 1, \ldots$, the {\cf forall} procedure
successively applies \var{proc} to arguments $x_i^1 \ldots x_i^n$,
where $x_i^j$ is the $i$th element of \varj{l}, until \schfalse{} is
returned.  If \var{proc} returns true values for all but the last
element of \vari{l}, \var{forall} performs a tail call of \var{proc}
on the $k$th elements, where $k$ is the length of \vari{l}---in this
case, the \var{l}s must all be lists of length $k$.  If \var{proc}
returns \schfalse{} on any set of elements, {\cf forall} returns
\schfalse{} after the first such application of \var{proc} without
further traversing the \var{l}s.  If the \var{l}s are all empty, {\cf
  forall} returns \schtrue.

For natural numbers $i = 0, 1, \ldots$, the {\cf exists} procedure
applies \var{proc} successively to arguments $x_i^1 \ldots x_i^n$,
where $x_i^j$ is the $i$th element of \varj{l}, until a true value is
returned.  If \var{proc} returns \schfalse{} for all but the last
elements of the \var{l}s, \var{exists} performs a tail call of
\var{proc} on the $k$th elements, where $k$ is the length of
\vari{l}---in this case, the \var{l}s must all be lists of length $k$.
If \var{proc} returns a true value on any set of elements, {\cf
  exists} returns that value after the first such application of
\var{proc} without further traversing the \var{l}s.  If the \var{l}s
are all empty, {\cf exists} returns \schfalse.

\begin{scheme}
(forall even? '(3 1 4 1 5 9)) \lev \schfalse{}
(forall even? '(3 1 4 1 5 9 . 2)) \lev \schfalse{}
(forall even? '(2 4 14)) \ev \schtrue{}
(forall even? '(2 4 14 . 9)) \lev \exception{\cf\&contract}
(forall (lambda (n) (and (even? n) n)) '(2 4 14)) \lev 14
(forall < '(1 2 3) '(2 3 4)) \ev \schtrue{}
(forall < '(1 2 4) '(2 3 4)) \ev \schfalse{}

(exists even? '(3 1 4 1 5 9)) \lev \schtrue{}
(exists even? '(3 1 1 5 9)) \ev \schfalse{}
(exists even? '(3 1 1 5 9 . 2)) \lev \exception{\cf\&contract}
(exists (lambda (n) (and (even? n) n)) '(2 1 4 14)) \lev 2
(exists < '(1 2 4) '(2 3 4)) \ev \schtrue{}
(exists > '(1 2 3) '(2 3 4)) \ev \schfalse{}
\end{scheme}
\end{entry}

\begin{entry}{%
\proto{filter}{ proc list}{procedure}
\proto{partition}{ proc list}{procedure}
}

\domain{\var{Proc} must be a procedure; it must take a single argument
  if \var{list} is non-empty.}  The {\cf filter} procedure successively applies
\var{proc} to the elements of \var{list} and returns a list of
the values of \var{list} for which \var{proc} returned a true
value.  The {\cf partition} procedure also successively applies \var{proc} to
the elements of \var{list}, but returns two values, the first one a
list of the values of \var{list} for which \var{proc} returned a
true value, and the second a list of the values of \var{list} for
which \var{proc} returned \schfalse.

\begin{scheme}
(filter even? '(3 1 4 1 5 9 2 6)) \lev (4 2 6)

(partition even? '(3 1 4 1 5 9 2 6)) \lev (4 2 6) (3 1 1 5 9) ; two values
\end{scheme}

\end{entry}

\begin{entry}{%
\proto{fold-left}{ combine nil \vari{list} \varii{list} \dotsfoo \varn{list}}{procedure}}

\domain{If
  more than one \var{list} is given, then they must all be the same
  length.  \var{Combine} must be a
  procedure; if the \var{list}s are non-empty, it must take one more
  argument than there are {\it list}s.}
The {\cf fold-left} procedure iterates the \var{combine} procedure over an
accumulator value and the values of the {\it list}s from left to
right, starting with an accumulator value of \var{nil}.  More
specifically, {\cf fold-left} returns \var{nil} if the {\it list}s are
empty.  If they are not empty, \var{combine} is first applied to
\var{nil} and the respective first elements of the {\it list}s in
order.  The result becomes the new accumulator value, and \var{combine}
is applied to new accumulator value and the respective next elements
of the {\it list}.  This step is repeated until the end of the list is
reached; then the accumulator value is returned.

\begin{scheme}
(fold-left + 0 '(1 2 3 4 5)) \ev 15

(fold-left (lambda (a e) (cons e a)) '()
           '(1 2 3 4 5)) \lev (5 4 3 2 1)

(fold-left (lambda (count x)
             (if (odd? x) (+ count 1) count))
           0
           '(3 1 4 1 5 9 2 6 5 3)) \lev 7

(fold-left (lambda (max-len s)
             (max max-len (string-length s)))
           0
           '("longest" "long" "longer")) \lev 7

(fold-left cons '(q) '(a b c)) \lev ((((q) . a) . b) . c)

(fold-left + 0 '(1 2 3) '(4 5 6)) \lev 21
\end{scheme}
\end{entry}


\begin{entry}{%
\proto{fold-right}{ combine nil \vari{list} \varii{list} \dotsfoo \varn{list}}{procedure}}

\domain{ If
  more than one \var{list} is given, then they must all be the same
  length.  \var{Combine} must be a
  procedure; if the \var{list}s are non-empty, it must take one more
  argument than there are {\it list}s.}
The {\cf fold-right} procedure iterates the \var{combine} procedure over
the values of the {\it list}s from right to left and an accumulator
value, starting with an accumulator value of \var{nil}.  More
specifically, {\cf fold-right} returns \var{nil} if the {\it list}s
are empty.  If they are not empty, \var{combine} is first applied to the
respective last elements of the {\it list}s in order and \var{nil}.
The result becomes the new accumulator value, and \var{combine} is
applied to the respective previous elements of the {\it list} and the
new accumulator value.  This step is repeated until the beginning of the
list is reached; then the accumulator value is returned.

\begin{scheme}
(fold-right + 0 '(1 2 3 4 5)) \ev 15

(fold-right cons '() '(1 2 3 4 5)) \lev (1 2 3 4 5)

(fold-right (lambda (x l)
              (if (odd? x) (cons x l) l))
            '()
            '(3 1 4 1 5 9 2 6 5))
\ev (3 1 1 5 9 5)

(fold-right cons '(q) '(a b c)) \lev (a b c q)

(fold-right + 0 '(1 2 3) '(4 5 6)) \lev 21
\end{scheme}
\end{entry}

\begin{entry}{%
\proto{remp}{ proc list}{procedure}
\proto{remove}{ obj list}{procedure}
\proto{remv}{ obj list}{procedure}
\proto{remq}{ obj list}{procedure}}

\domain{\var{Proc} must be a procedure; it must take a single argument
  if \var{list} is non-empty.}
Each of these procedures returns a list of the elements of \var{list}
that do not satisfy a given condition.  The {\cf remp} procedure successively
applies \var{proc} to the elements of \var{list} and returns a
list of the values of \var{list} for which \var{proc} returned
\schfalse.  The {\cf remove}, {\cf remv}, and {\cf remq} procedures return a list of
the elements that are not \var{obj}.  The {\cf remq} procedure uses {\cf eq?}\ to
compare \var{obj} with the elements of \var{list}, while {\cf remv}
uses {\cf eqv?}\ and {\cf remove} uses {\cf equal?}.

\begin{scheme}
(remp even? '(3 1 4 1 5 9 2 6 5)) \lev (3 1 1 5 9 5)

(remove 1 '(3 1 4 1 5 9 2 6 5)) \lev (3 4 5 9 2 6 5)

(remv 1 '(3 1 4 1 5 9 2 6 5)) \lev (3 4 5 9 2 6 5)

(remq 'foo '(bar foo baz)) \ev (bar baz)
\end{scheme}
\end{entry}

\begin{entry}{%
\proto{memp}{ proc l}{procedure}
\proto{member}{ obj l}{procedure}
\proto{memv}{ obj l}{procedure}
\proto{memq}{ obj l}{procedure}
}

\domain{\var{Proc} must be a procedure; it must take a single argument
  if \var{l} is non-empty.
  \var{l} must be the empty list or a chain of pairs of size according
to the conditions stated below.}

These procedures return the first sublist of \var{l} whose car
satisfies a given condition, where the subchains of \var{l} are the
chains of pairs returned by {\tt (list-tail \var{l} \var{k})} for
\var{k} less than the length of \var{l}.  The {\cf memp} procedure applies
\var{proc} to the cars of the sublists of \var{l} until it
finds one for which \var{proc} returns a true value without traversing
\var{l} further.  The {\cf
  member}, {\cf memv}, and {\cf memq} procedures look for the first occurrence of
\var{obj}.  If \var{l} does not contain an element satisfying the
condition, then \schfalse{} (not the empty list) is returned; in that
case, \var{l} must be a list.  The {\cf
  member} procedure uses {\cf equal?}\ to compare \var{obj} with the elements of
\var{l}, while {\cf memv} uses {\cf eqv?}\ and {\cf memq} uses
{\cf eq?}.

\begin{scheme}
(memp even? '(3 1 4 1 5 9 2 6 5)) \lev (4 1 5 9 2 6 5)

(memq 'a '(a b c))              \ev  (a b c)
(memq 'b '(a b c))              \ev  (b c)
(memq 'a '(b c d))              \ev  \schfalse
(memq (list 'a) '(b (a) c))     \ev  \schfalse
(member (list 'a)
        '(b (a) c))             \ev  ((a) c)
(memq 101 '(100 101 102))       \ev  \unspecified
(memv 101 '(100 101 102))       \ev  (101 102)%
\end{scheme} 
\begin{rationale}
  Although they are ordinarily used as predicates, {\cf memp}, {\cf
    member}, {\cf memv}, {\cf memq}, do not have question marks in
  their names because they return useful values rather than just
  \schtrue{} or \schfalse{}.
\end{rationale}
\end{entry}

\begin{entry}{%
\proto{assp}{ proc al}{procedure}
\proto{assoc}{ obj al}{procedure}
\proto{assv}{ obj al}{procedure}
\proto{assq}{ obj al}{procedure}}

\domain{\var{Al} (for ``association list'') must be the empty list or
  a chain of pairs of size according to
  the conditions specified below, where each car contains a pair.
  \var{Proc} must be a procedure; it
  must take a single argument if \var{al} is non-empty.}

These procedures find the first pair in \var{al}
whose car field satisfies a given condition, and returns that pair
without traversing \var{al} further.
If no pair in \var{al} satisfies the condition, then \schfalse{}
is returned; in that case, \var{al} must be a
list.  The {\cf assp} procedure successively applies
\var{proc} to the car fields of \var{al} and looks for a pair
for which it returns a true value.  The {\cf assoc}, {\cf assv}, and {\cf
  assq} procedures look for a pair that has \var{obj} as its car.  The
{\cf assoc} procedure uses 
{\cf equal?}\ to compare \var{obj} with the car fields of the pairs in
\var{al}, while {\cf assv} uses {\cf eqv?}\ and {\cf assq} uses
{\cf eq?}.


\begin{scheme}
(define d '((3 a) (1 b) (4 c)))

(assp even? d) \ev (4 c)
(assp odd? d) \ev (3 a)

(define e '((a 1) (b 2) (c 3)))
(assq 'a e)     \ev  (a 1)
(assq 'b e)     \ev  (b 2)
(assq 'd e)     \ev  \schfalse
(assq (list 'a) '(((a)) ((b)) ((c))))
                \ev  \schfalse
(assoc (list 'a) '(((a)) ((b)) ((c))))   
                           \ev  ((a))
(assq 5 '((2 3) (5 7) (11 13)))    
                           \ev  \unspecified
(assv 5 '((2 3) (5 7) (11 13)))    
                           \ev  (5 7)%
\end{scheme}

\end{entry}

%%% Local Variables: 
%%% mode: latex
%%% TeX-master: "r6rs"
%%% End: 



\section{{\tt when} and {\tt unless}}

\begin{entry}{%
\proto{when}{ \hyper{test} \hyperi{expression} \hyperii{expression} \dotsfoo}{\exprtype}
\proto{unless}{ \hyper{test} \hyperi{expression} \hyperii{expression} \dotsfoo}{\exprtype}}

\semantics A {\cf when} expression is evaluated by evaluating the
\hyper{test} expression.  When \hyper{test} evaluates to a true value
then the remaining \hyper{expression}s are evaluated in order, and the
result(s) of the last \hyper{expression} is(are) returned as the
result(s) of the entire {\cf when} expression.  Otherwise, the {\cf
  when} expression evaluates to the unspecified value.  A {\cf unless}
expression is evaluated by evaluating the \hyper{test} expression.
When \hyper{test} evaluates to false then the remaining
\hyper{expression}s are evaluated in order, and the result(s) of the
last \hyper{expression} is(are) returned as the result(s) of the
entire {\cf unless} expression.  Otherwise, the {\cf unless} expression
evaluates to the unspecified value.

\begin{scheme}
(when (> 3 2) 'greater) \ev greater
(when (< 3 2) 'greater) \ev \theunspecified
(unless (> 3 2) 'less) \ev \theunspecified
(unless (< 3 2) 'less) \ev less
\end{scheme}

The {\cf when} and {\cf unless} expressions are derived forms.  They
could be defined in terms of core language by the following macros:

\begin{scheme}
(define-syntax \ide{when}
  (syntax-rules ()
    ((when test result1 result2 ...)
     (if test
         (begin result1 result2 ...)))))

(define-syntax \ide{unless}
  (syntax-rules ()
    ((when test result1 result2 ...)
     (if (not test)
         (begin result1 result2 ...)))))
\end{scheme}

\end{entry}

\section{{\tt case-lambda}}

\begin{entry}{%
\proto{case-lambda}{ \hyperi{clause} \hyperii{clause} \dotsfoo}{library \exprtype}}
    
\syntax
Each \hyper{clause} should be of the form
%
\begin{scheme}
(\hyper{formals} \hyperi{expression} \dotsfoo)%
\end{scheme}
%
If a \hyper{clause} has a \hyper{formals} that consists of a single
\hyper{variable}, then that clause must be last.

\semantics A {\cf case-lambda} expression evaluates to a procedure.
This procedure, when applied, tries to match its arguments to the
\hyper{clause}s in order.  The arguments match a clause, if one the
following conditions is fulfilled:
%
\begin{itemize}
\item \hyper{Formals} has the form {\tt (\hyper{variable} \dotsfoo)}
and the number of arguments is the same as the number of formal
parameters in \hyper{formals}.
\item \hyper{Formals} has the form\\ {\tt
(\hyperi{variable} \dotsfoo \hypern{variable} . \hyper{variable$_{n+1}$)}
}\\
and the number of arguments is at least $n$.
\item \hyper{Formals} has the form {\tt \hyper{variable}}.
\end{itemize}
%
For the first clause matched by the arguments, the variables of the
\hyper{formals} will be bound to fresh locations containing the
argument values in the same arrangement as with {\cf lambda}.

If the arguments match none of the clauses, a violation is raised.
FIXME

\begin{scheme}
(define foo
  (case-lambda 
   (() 'zero)
   ((x) (list 'one x))
   ((x y) (list 'two x y))
   ((a b c d . e) (list 'four a b c d e))
   (rest (list 'rest rest))))

(foo) \ev zero
(foo 1) \ev (one 1)
(foo 1 2) \ev (two 1 2)
(foo 1 2 3) \ev (rest (1 2 3))
(foo 1 2 3 4) \ev (four 1 2 3 4 ())
\end{scheme}

The {\cf case-lambda} expression is a derived form.  It
could be defined in terms of core language by the following macros:
%
\begin{scheme}
(define-syntax \ide{case-lambda}
  (syntax-rules ()
    ((case-lambda
      (formals-0 body0-0 body1-0 ...)
      (formals-1 body0-1 body1-1 ...)
      ...)
     (lambda args
       (let ((l (length args)))
         (case-lambda-helper
          l args
          (formals-0 body0-0 body1-0 ...)
          (formals-1 body0-1 body1-1 ...) ...))))))

(define-syntax \ide{case-lambda-helper}
  (syntax-rules ()
    ((case-lambda-helper
      l args)
     (violation 'case-lambda
                "wrong number of arguments" l args))
    ((case-lambda-helper
      l args
      ((formal ...) body ...)
      clause ...)
     (if (= l (length '(formal ...)))
         (apply (lambda (formal ...) body ...)
                args)
         (case-lambda-helper l args clause ...)))
    ((case-lambda-helper
      l args
      ((formal . formals-rest) body ...)
      clause ...)
     (case-lambda-helper-dotted l args
                                (body ...)
                                (formal . formals-rest)
                                formals-rest 1
                                clause ...))
    ((case-lambda-helper
      l args
      (formal body ...))
     (let ((formal args))
       body ...))))

(define-syntax \ide{case-lambda-helper-dotted}
  (syntax-rules ()
    ((case-lambda-helper-dotted
      l args
      (body ...)
      formals
      (formal . formals-rest) k
      clause ...)
     (case-lambda-helper-dotted
      l args
      (body ...)
      formals
      formals-rest (+ 1 k)
      clause ...))
    ((case-lambda-helper-dotted
      l args
      (body ...)
      formals
      rest-formal k
      clause ...)
     (if (>= l k)
         (apply (lambda formals body ...) args)
         (case-lambda-helper
          l args clause ...)))))
\end{scheme}
\end{entry}

\section{Delayed evaluation}

\begin{entry}{%
\proto{delay}{ \hyper{expression}}{library \exprtype}}

\todo{Fix.}

The {\cf delay} construct is used together with the procedure \ide{force} to
implement \defining{lazy evaluation} or \defining{call by need}.
{\tt(delay~\hyper{expression})} returns an object called a
\defining{promise} which at some point in the future may be asked (by
the {\cf force} procedure) \todo{Bartley's white lie; OK?} to evaluate
\hyper{expression}, and deliver the resulting value.
The effect of \hyper{expression} returning multiple values
is unspecified.

See the description of {\cf force} (section~\ref{force}) for a
more complete description of {\cf delay}.

\end{entry}

\begin{entry}{%
\proto{force}{ promise}{library procedure}}

Forces the value of \var{promise} (see \ide{delay},
section~\ref{delay}).\index{promise}  If no value has been computed for
the promise, then a value is computed and returned.  The value of the
promise is cached (or ``memoized'') so that if it is forced a second
time, the previously computed value is returned.
% without any recomputation.
% [As pointed out by Marc Feeley, the "without any recomputation"
% isn't necessarily true. --Will]

\begin{scheme}
(force (delay (+ 1 2)))   \ev  3
(let ((p (delay (+ 1 2))))
  (list (force p) (force p)))  
                               \ev  (3 3)

(define a-stream
  (letrec ((next
            (lambda (n)
              (cons n (delay (next (+ n 1)))))))
    (next 0)))
(define head car)
(define tail
  (lambda (stream) (force (cdr stream))))

(head (tail (tail a-stream)))  
                               \ev  2%
\end{scheme}

{\cf Force} and {\cf delay} are mainly intended for programs written in
functional style.  The following examples should not be considered to
illustrate good programming style, but they illustrate the property that
only one value is computed for a promise, no matter how many times it is
forced.
% the value of a promise is computed at most once.
% [As pointed out by Marc Feeley, it may be computed more than once,
% but as I observed we can at least insist that only one value be
% used! -- Will]

\begin{scheme}
(define count 0)
(define p
  (delay (begin (set! count (+ count 1))
                (if (> count x)
                    count
                    (force p)))))
(define x 5)
p                     \ev  {\it{}a promise}
(force p)             \ev  6
p                     \ev  {\it{}a promise, still}
(begin (set! x 10)
       (force p))     \ev  6%
\end{scheme}

Here is a possible implementation of {\cf delay} and {\cf force}.
Promises are implemented here as procedures of no arguments,
and {\cf force} simply calls its argument:

\begin{scheme}
(define force
  (lambda (object)
    (object)))%
\end{scheme}

We define the expression

\begin{scheme}
(delay \hyper{expression})%
\end{scheme}

to have the same meaning as the procedure call

\begin{scheme}
(make-promise (lambda () \hyper{expression}))\rm
\end{scheme}

as follows

\begin{scheme}
(define-syntax delay
  (syntax-rules ()
    ((delay expression)
     (make-promise (lambda () expression))))),%
\end{scheme}

where {\cf make-promise} is defined as follows:

% \begin{scheme}
% (define make-promise
%   (lambda (proc)
%     (let ((already-run? \schfalse) (result \schfalse))
%       (lambda ()
%         (cond ((not already-run?)
%                (set! result (proc))
%                (set! already-run? \schtrue)))
%         result))))%
% \end{scheme}

\begin{scheme}
(define make-promise
  (lambda (proc)
    (let ((result-ready? \schfalse)
          (result \schfalse))
      (lambda ()
        (if result-ready?
            result
            (let ((x (proc)))
              (if result-ready?
                  result
                  (begin (set! result-ready? \schtrue)
                         (set! result x)
                         result))))))))%
\end{scheme}

\begin{rationale}
A promise may refer to its own value, as in the last example above.
Forcing such a promise may cause the promise to be forced a second time
before the value of the first force has been computed.
This complicates the definition of {\cf make-promise}.
\end{rationale}

Various extensions to this semantics of {\cf delay} and {\cf force}
are supported in some implementations:

\begin{itemize}
\item Calling {\cf force} on an object that is not a promise may simply
return the object.

\item It may be the case that there is no means by which a promise can be
operationally distinguished from its forced value.  That is, expressions
like the following may evaluate to either \schtrue{} or to \schfalse{},
depending on the implementation:

\begin{scheme}
(eqv? (delay 1) 1)          \ev  \unspecified
(pair? (delay (cons 1 2)))  \ev  \unspecified%
\end{scheme}

\item Some implementations may implement ``implicit forcing,'' where
the value of a promise is forced by primitive procedures like \cf{cdr}
and \cf{+}:

\begin{scheme}
(+ (delay (* 3 7)) 13)  \ev  34%
\end{scheme}
\end{itemize}
\end{entry}

\section{Enumerations}
\label{enumerationssection}

This section describes the \deflibrary{r6rs enum} library for dealing with enumerated values
\mainschindex{enumeration}and sets of enumerated values.  Enumerated
values are represented by ordinary symbols, while finite sets of
enumerated values form a separate type, known as the
\defining{enumeration sets}.
The enumeration sets are further partitioned into sets that
share the same \defining{universe} and \defining{enumeration type}.
These universes and enumeration types are created by the
{\cf make-enumeration} procedure.  Each call to that procedure
creates a new enumeration type.

This library interprets each enumeration set with respect to
its specific universe of symbols and enumeration type.
This facilitates efficient implementation of enumeration sets
and enables the complement operation.

In the definition of the following procedures, let \var{enum-set}
range over the enumeration sets, which are defined as the subsets
of the universes that can be defined using {\cf make-enumeration}.

\begin{entry}{%
\proto{make-enumeration}{ list}{procedure}}

{\cf make-enumeration} takes an arbitrary list of symbols,
creates a new enumeration type whose universe consists of
those symbols (in canonical order of their first appearance
in the list) and returns that universe as an enumeration
set whose universe is itself and whose enumeration type is
the newly created enumeration type.
\end{entry}

\begin{entry}{%
\proto{enum-set-universe}{ enum-set}{procedure}}

{\cf enum-set-universe} returns the set of all symbols that comprise
the universe of its argument.
\end{entry}

\begin{entry}{%
\proto{enum-set-indexer}{ enum-set}{procedure}}

{\cf enum-set-indexer} returns a unary procedure that, given a symbol
that is in the universe of \var{enum-set}, returns its 0-origin index
within the canonical ordering of the symbols in the universe; given a
value not in the universe, the unary procedure returns \schfalse.

\begin{scheme}
(let* ((e (make-enumeration '(red green blue)))
       (i (enum-set-indexer e)))
  (list (i 'red) (i 'green) (i 'blue) (i 'yellow))) \lev (0 1 2 \schfalse)
\end{scheme}

{\cf enum-set-indexer} could be defined as follows (using the
{\cf memq} procedure from the \library{r6rs lists} library):

\begin{scheme}
(define (enum-set-indexer set)
  (let* ((symbols (enum-set->list
                    (enum-set-universe set)))
         (cardinality (length symbols)))
    (lambda (x)
      (let ((probe (memq x symbols)))
        (if probe
            (- cardinality (length probe))
            \schfalse)))))
\end{scheme}
\end{entry}

\begin{entry}{%
\proto{enum-set-constructor}{ enum-set}{procedure}}

{\cf enum-set-constructor} returns a unary procedure that, given a
list of symbols that belong to the universe of \var{enum-set}, returns
a subset of that universe that contains exactly the symbols in the
list.  If any value in the list is not a symbol that belongs to the
universe, then the unary procedure raises an exception with
condition type {\cf\&contract}.
\end{entry}

\begin{entry}{%
\proto{enum-set->list}{ enum-set}{procedure}}

{\cf enum-set->list} returns a list of the symbols that belong to its
argument, in the canonical order of the universe of \var{enum-set}.

\begin{scheme}
(let* ((e (make-enumeration '(red green blue)))
       (c (enum-set-constructor e)))
  (enum-set->list (c '(blue red)))) \lev (red blue)
\end{scheme}
\end{entry}

\begin{entry}{%
\proto{enum-set-member?}{ symbol enum-set}{procedure}
\proto{enum-set-subset?}{ \vari{enum-set} \varii{enum-set}}{procedure}
\proto{enum-set=?}{ \vari{enum-set} \varii{enum-set}}{procedure}}

{\cf enum-set-member?} returns \schtrue{} if its first argument is an
element of its second argument, \schfalse{} otherwise.

{\cf enum-set-subset?} returns \schtrue{} if the universe of
\vari{enum-set} is a subset of the universe of \varii{enum-set}
(considered as sets of symbols) and every element of \vari{enum-set}
is a member of its second.  It returns \schfalse{} otherwise.

{\cf enum-set=?} returns \schtrue{} if \vari{enum-set}  is a
subset of \varii{enum-set} and vice versa, as determined by the
{\cf enum-set-subset?} procedure.  This implies that the universes of
the two sets are equal as sets of symbols, but does not imply
that they are equal as enumeration types.  Otherwise, \schfalse{} is
returned.

\begin{scheme}
(let* ((e (make-enumeration '(red green blue)))
       (c (enum-set-constructor e)))
  (list
   (enum-set-member? 'blue (c '(red blue)))
   (enum-set-member? 'green (c '(red blue)))
   (enum-set-subset? (c '(red blue)) e)
   (enum-set-subset? (c '(red blue)) (c '(blue red)))
   (enum-set-subset? (c '(red blue)) (c '(red)))
   (enum-set=? (c '(red blue)) (c '(blue red)))))
\ev (\schtrue{} \schfalse{} \schtrue{} \schtrue{} \schfalse{} \schtrue{})
\end{scheme}
\end{entry}

\begin{entry}{%
\proto{enum-set-union}{ \vari{enum-set} \varii{enum-set}}{procedure}
\proto{enum-set-intersection}{ \vari{enum-set} \varii{enum-set}}{procedure}
\proto{enum-set-difference}{ \vari{enum-set} \varii{enum-set}}{procedure}}


\domain{\vari{enum-set} and \varii{enum-set} shall be enumeration sets 
  that have the same enumeration type.  If their enumeration types
  differ, a {\cf\&contract} violation is raised.}

{\cf enum-set-union} returns the union of \vari{enum-set} and \varii{enum-set}.
{\cf enum-set-intersection} returns the intersection of \vari{enum-set} and \varii{enum-set}.
{\cf enum-set-difference} returns the difference of \vari{enum-set}
and \varii{enum-set}.

\begin{scheme}
(let* ((e (make-enumeration '(red green blue)))
       (c (enum-set-constructor e)))
  (list (enum-set->list
         (enum-set-union (c '(blue)) (c '(red))))
        (enum-set->list
         (enum-set-intersection (c '(red green))
                                (c '(red blue))))
        (enum-set->list
         (enum-set-difference (c '(red green))
                              (c '(red blue))))))
\lev ((red blue) (red) (green))
\end{scheme}
\end{entry}

\begin{entry}{%
\proto{enum-set-complement}{ enum-set}{procedure}}

{\cf enum-set-complement} takes an enumeration set and returns its
complement with respect to its universe.


\begin{scheme}
(let* ((e (make-enumeration '(red green blue)))
       (c (enum-set-constructor e)))
  (enum-set->list
   (enum-set-complement (c '(red)))))
\ev (green blue)
\end{scheme}
\end{entry}

\begin{entry}{%
\proto{enum-set-projection}{ \vari{enum-set} \varii{enum-set}}{procedure}}

{\cf enum-set-projection} projects \vari{enum-set} into the universe
of \varii{enum-set}, dropping any elements of \vari{enum-set} that do
not belong to the universe of \varii{enum-set}.  (If \vari{enum-set}
is a subset of the universe of its second, then no elements are
dropped, and the injection is returned.)

\begin{scheme}
(let ((e1 (make-enumeration
           '(red green blue black)))
      (e2 (make-enumeration
           '(red black white))))
  (enum-set->list
   (enum-set-projection e1 e2))))
\ev (red black)
\end{scheme}
\end{entry}

\begin{entry}{}
\pproto{(define-enumeration \hyper{type-name}}{\exprtype}
\mainschindex{define-enumeration}{\tt\obeyspaces%
  (\hyper{symbol} \dotsfoo)\\
  \hyper{constructor-syntax})}

The {\cf define-enumeration} form defines an enumeration type and
provides two macros for constructing its members and sets of its
members.

A {\cf define-enumeration} form is a definition and can appear
anywhere any other \hyper{definition} can appear.

\hyper{type-name} is an identifier that will be bound to a macro;
\hyper{symbol}~\dotsfoo{} are the symbols that will comprise the
universe of the enumeration (in order).

{\cf (\hyper{type-name} \hyper{symbol})} checks at macro-expansion
time whether \hyper{symbol} is in the universe associated with
\hyper{type-name}.  If it is, then {\cf (\hyper{type-name}
  \hyper{symbol})} is equivalent to {\cf \hyper{symbol}}.  
It is a syntax violation if it is not.

\hyper{constructor-syntax} is an identifier that will be bound to a
macro that, given any finite sequence of the symbols in the universe,
possibly with duplicates, expands into an expression that evaluates
to the enumeration set of those symbols.

{\cf (\hyper{constructor-syntax} \hyper{symbol}~\dotsfoo{})} checks at
macro-expansion time whether every \hyper{symbol}~\dotsfoo{} is in the
universe associated with \hyper{type-name}.  It is a syntax violation
if one or more is not.
Otherwise
\begin{scheme}
(\hyper{constructor-syntax} \hyper{symbol}~\dotsfoo{})
\end{scheme}
%
is equivalent to
%
\begin{scheme}
((enum-set-constructor (\hyper{constructor-syntax}))
 (list '\hyper{symbol}~\dotsfoo{}))\rm.
\end{scheme}

\begin{scheme}
(define-enumeration color
  (black white purple maroon)
  color-set)

(color black)                      \ev black
(color purpel)                     \ev \exception{\&syntax}
(enum-set->list (color-set))       \ev ()
(enum-set->list
 (color-set maroon white))         \ev (white maroon)
\end{scheme}
\end{entry}

%%% Local Variables: 
%%% mode: latex
%%% TeX-master: "r6rs"
%%% End: 



\section{Bytes objects}
\label{bytessection}

Many applications must deal with blocks of binary data by accessing
them in various ways---extracting signed or unsigned numbers of
various sizes.  Therefore, the \deflibrary{r6rs bytes} library
provides a single type for
blocks of binary data with multiple ways to access that data. It deals
only with integers in various sizes with specified endianness, because
these are the most frequent applications.

Bytes objects\mainschindex{bytes objects} are objects of a disjoint
type. Conceptually, a bytes object represents a sequence of 8-bit
bytes.  The description of bytes objects uses the term \defining{byte}
for an exact integer in the interval $\{-128, \ldots, 127\}$ and the
term \defining{octet} for an exact integer in the interval $\{0,
\ldots, 255\}$.  A byte corresponds to its two's complement
representation as an octet.

The length of a bytes object is the number of bytes it contains. This
number is fixed. A valid index into a bytes object is an exact,
non-negative integer. The first byte of a bytes object has index 0;
the last byte has an index one less than the length of the bytes
object.

Generally, the access procedures come in different flavors according
to the size of the represented integer, and the endianness of the
representation.  The procedures also distinguish signed and unsigned
representations.
The signed representations all use two's complement.

Like list and vector literals, literals representing bytes objects
must be quoted:
%
\begin{scheme}
'\#vu8(12 23 123) \ev \#vu8(12 23 123)%
\end{scheme}


\begin{entry}{%
\proto{endianness}{ {\cf big}}{\exprtype}
\rproto{endianness}{ {\cf little}}{\exprtype}}
   
{\cf (endianness big)} and {\cf (endianness little)} evaluate to the
symbols {\cf big} and {\cf little}, respectively. These symbols
represent an endianness, and whenever one of the procedures operating
on bytes objects accepts an endianness as an argument, that argument
must be one of these symbols. It is a syntax violation for the operand to
{\cf endianness} to be
anything other than {\cf big} or {\cf little}.
\end{entry}

\begin{entry}{%
\proto{native-endianness}{}{procedure}}

This procedure returns the implementation's preferred endianness
(usually that of the underlying machine architecture),
either {\cf big} or {\cf little}.
\end{entry}   

\begin{entry}{%
\proto{bytes?}{ obj}{procedure}}
   
Returns \schtrue{} if \var{obj} is a bytes object,
otherwise returns \schfalse{}.
\end{entry}

\begin{entry}{%
\proto{make-bytes}{ k}{procedure}
\rproto{make-bytes}{ k fill}{procedure}}
   
Returns a newly allocated bytes object of \var{k} bytes.
   
If the \var{fill} argument is missing, the initial contents of the
returned bytes object are unspecified.
   
If the \var{fill} argument is present, it must be an exact integer in
the interval $\{-128, \ldots 255\}$ that specifies the initial value
for the bytes of the bytes object: If \var{fill} is positive, it is
interpreted as an octet; if it is negative, it is interpreted as a byte.
\end{entry}   

\begin{entry}{%
\proto{bytes-length}{ bytes}{procedure}}
   
Returns, as an exact integer, the number of bytes in \var{bytes}.
\end{entry}

\begin{entry}{%
\proto{bytes-u8-ref}{ bytes k}{procedure}
\proto{bytes-s8-ref}{ bytes k}{procedure}}
   
\domain{\var{k} must be a valid index of bytes.}
   
{\cf bytes-u8-ref} returns the byte at index \var{k} of \var{bytes},
as an octet.
   
{\cf bytes-s8-ref} returns the byte at index \var{k} of \var{bytes},
as a (signed) byte.

\begin{scheme}
(let ((b1 (make-bytes 16 -127))
      (b2 (make-bytes 16 255)))
  (list
    (bytes-s8-ref b1 0)
    (bytes-u8-ref b1 0)
    (bytes-s8-ref b2 0)
    (bytes-u8-ref b2 0))) \ev (-127 129 -1 255)
\end{scheme}
\end{entry}   

\begin{entry}{%
\proto{bytes-u8-set!}{ bytes k octet}{procedure}
\proto{bytes-s8-set!}{ bytes k byte}{procedure}}
   
\var{k} must be a valid index of \var{bytes}.
   
{\cf bytes-u8-set!} stores \var{octet} in element \var{k} of
\var{bytes}.
   
{\cf bytes-s8-set!} stores the two's complement representation of
\var{byte} in element \var{k} of \var{bytes}.
   
Both procedures return \unspecifiedreturn.

\begin{scheme}
(let ((b (make-bytes 16 -127)))

  (bytes-s8-set! b 0 -126)
  (bytes-u8-set! b 1 246)

  (list
    (bytes-s8-ref b 0)
    (bytes-u8-ref b 0)
    (bytes-s8-ref b 1)
    (bytes-u8-ref b 1))) \ev (-126 130 -10 246)
\end{scheme}
\end{entry}

\begin{entry}{%
\proto{bytes-uint-ref}{ bytes k endianness size}{procedure}
\proto{bytes-sint-ref}{ bytes k endianness size}{procedure}
\proto{bytes-uint-set!}{ bytes k n endianness size}{procedure}
\proto{bytes-sint-set!}{ bytes k n endianness size}{procedure}}
   
\domain{\var{size} must be a positive exact integer. $\{\var{k}, \ldots,
  \var{k} + \var{size} - 1\}$ must be valid indices of \var{bytes}.}
   
{\cf bytes-uint-ref} retrieves the exact integer corresponding to the
unsigned representation of size \var{size} and specified by \var{endianness}
at indices $\{\var{k}, \ldots, \var{k} + \var{size} - 1\}$.
   
{\cf bytes-sint-ref} retrieves the exact integer corresponding to the two's
complement representation of size \var{size} and specified by \var{endianness} at
indices $\{\var{k}, \ldots, \var{k} + \var{size} - 1\}$.
   
\domain{For {\cf bytes-uint-set!}, \var{n} must be an exact integer in the
  set $\{0, \ldots, 256^{\var{size}}-1\}$.}

{\cf bytes-uint-set!} stores the unsigned representation of size \var{size}
and specified by \var{endianness} into \var{bytes} at indices
$\{\var{k}, \ldots, \var{k} + \var{size} - 1\}$.
   
\domain{For {\cf bytes-sint-set!}, \var{n} must be an exact integer in
  the interval $\{-256^{\var{size}}/2, \ldots,
  256^{\var{size}}/2-1\}$.}
{\cf bytes-sint-set!} stores the two's complement
representation of size \var{size} and specified by \var{endianness}
into \var{bytes} at indices $\{\var{k}, \ldots, \var{k} + \var{size} - 1\}$.
   
The \ldots{\cf -set!} procedures return \unspecifiedreturn.

\begin{scheme}
(define b (make-bytes 16 -127))

(bytes-uint-set! b 0 (- (expt 2 128) 3)
                 (endianness little) 16)

(bytes-uint-ref b 0 (endianness little) 16)\lev
    \#xfffffffffffffffffffffffffffffffd

(bytes-sint-ref b 0 (endianness little) 16)\lev -3

(bytes->u8-list b)\lev (253 255 255 255 255 255 255 255
               255 255 255 255 255 255 255 255)

(bytes-uint-set! b 0 (- (expt 2 128) 3)
                 (endianness big) 16)

(bytes-uint-ref b 0 (endianness big) 16) \lev
    \#xfffffffffffffffffffffffffffffffd

(bytes-sint-ref b 0 (endianness big) 16) \lev -3

(bytes->u8-list b) \lev (255 255 255 255 255 255 255 255
               255 255 255 255 255 255 255 253))
\end{scheme}
\end{entry}

\begin{entry}{%
\proto{bytes-u16-ref}{ bytes k endianness}{procedure}
\proto{bytes-s16-ref}{ bytes k endianness}{procedure}
\proto{bytes-u16-native-ref}{ bytes k}{procedure}
\proto{bytes-s16-native-ref}{ bytes k}{procedure}
\proto{bytes-u16-set!}{ bytes k n endianness}{procedure}
\proto{bytes-s16-set!}{ bytes k n endianness}{procedure}
\proto{bytes-u16-native-set!}{ bytes k n}{procedure}
\proto{bytes-s16-native-set!}{ bytes k n}{procedure}}
   
\domain{\var{k} must be a valid index of \var{bytes}; so must $\var{k}
  + 1$.}
   
These retrieve and set two-byte representations of numbers at indices
\var{k} and $\var{k}+1$, according to the endianness specified by
\var{endianness}. The procedures with {\cf u16} in their names deal with the
unsigned representation; those with {\cf s16} in their names deal
with the two's complement representation.

The procedures with {\cf native} in their names employ the native
endianness, and only work at aligned indices:
\var{k} must be a multiple of 2.
   
The \ldots{\cf -set!} procedures return \unspecifiedreturn.

\begin{scheme}
(define b
  (u8-list->bytes
    '(255 255 255 255 255 255 255 255
      255 255 255 255 255 255 255 253)))

(bytes-u16-ref b 14 (endianness little)) \lev 65023
(bytes-s16-ref b 14 (endianness little)) \lev -513
(bytes-u16-ref b 14 (endianness big)) \lev 65533
(bytes-s16-ref b 14 (endianness big)) \lev -3

(bytes-u16-set! b 0 12345 (endianness little))
(bytes-u16-ref b 0 (endianness little)) \lev 12345

(bytes-u16-native-set! b 0 12345)
(bytes-u16-native-ref b 0) \ev 12345

(bytes-u16-ref b 0 (endianness little)) \lev \unspecified
\end{scheme}
\end{entry}

\begin{entry}{%
\proto{bytes-u32-ref}{ bytes k endianness}{procedure}
\proto{bytes-s32-ref}{ bytes k endianness}{procedure}
\proto{bytes-u32-native-ref}{ bytes k}{procedure}
\proto{bytes-s32-native-ref}{ bytes k}{procedure}
\proto{bytes-u32-set!}{ bytes k n endianness}{procedure}
\proto{bytes-s32-set!}{ bytes k n endianness}{procedure}
\proto{bytes-u32-native-set!}{ bytes k n}{procedure}
\proto{bytes-s32-native-set!}{ bytes k n}{procedure}}
   
\domain{$\{\var{k}, \ldots, \var{k}+ 3\}$ must be valid indices of \var{bytes}.}.
   
These retrieve and set four-byte representations of numbers at indices $\{\var{k},
\ldots, \var{k}+ 3\}$, according to the endianness specified by \var{endianness}. The
procedures with {\cf u32} in their names deal with the unsigned representation,
those with {\cf s32} with the two's complement representation.
   
The procedures with {\cf native} in their names employ the native endianness, and
only work at aligned indices: \var{k} must be a multiple of 4..
   
The \ldots{\cf{}-set!} procedures return \unspecifiedreturn.

\begin{scheme}
(define b
  (u8-list->bytes
    '(255 255 255 255 255 255 255 255
      255 255 255 255 255 255 255 253)))

(bytes-u32-ref b 12 (endianness little)) \lev 4261412863
(bytes-s32-ref b 12 (endianness little)) \lev -33554433
(bytes-u32-ref b 12 (endianness big)) \lev 4294967293
(bytes-s32-ref b 12 (endianness big)) \lev -3
\end{scheme}
\end{entry}

\begin{entry}{%
\proto{bytes-u64-ref}{ bytes k endianness}{procedure}
\proto{bytes-s64-ref}{ bytes k endianness}{procedure}
\proto{bytes-u64-native-ref}{ bytes k}{procedure}
\proto{bytes-s64-native-ref}{ bytes k}{procedure}
\proto{bytes-u64-set!}{ bytes k n endianness}{procedure}
\proto{bytes-s64-set!}{ bytes k n endianness}{procedure}
\proto{bytes-u64-native-set!}{ bytes k n}{procedure}
\proto{bytes-s64-native-set!}{ bytes k n}{procedure}}
 
\domain{$\{\var{k}, \ldots, \var{k}+ 7\}$ must be valid indices of \var{bytes}.}
   
These retrieve and set eight-byte representations of numbers at
indices $\{\var{k}, \ldots, \var{k}+ 7\}$, according to the endianness
specified by \var{endianness}. The procedures with {\cf u64} in their names deal
with the unsigned representation, those with {\cf s64} with the two's
complement representation.
   
The procedures with {\cf native} in their names employ the native endianness, and
only work at aligned indices: \var{k} must be a multiple of 8.
   
The \ldots{\cf{}-set!} procedures return \unspecifiedreturn.

\begin{scheme}
(define b
  (u8-list->bytes
    '(255 255 255 255 255 255 255 255
      255 255 255 255 255 255 255 253)))

(bytes-u64-ref b 8 (endianness little)) \lev 18302628885633695743
(bytes-s64-ref b 8 (endianness little)) \lev -144115188075855873
(bytes-u64-ref b 8 (endianness big)) \lev 18446744073709551613
(bytes-s64-ref b 8 (endianness big)) \lev -3
\end{scheme}
\end{entry}

\begin{entry}{%
\proto{bytes=?}{ bytes-1 bytes-2}{procedure}}
   
Returns \schtrue{} if \var{bytes-1} and \var{bytes-2} are equal---that
is, if they have the same length and equal bytes at all valid indices.
It returns \schfalse{} otherwise.
\end{entry}

\begin{entry}{%
\proto{bytes-ieee-single-native-ref}{ bytes k}{procedure}
\proto{bytes-ieee-single-ref}{ bytes k endianness}{procedure}}

\domain{$\{\var{k}, \ldots, \var{k}+3\}$ must be valid indices of
  \var{bytes}.  For {\cf bytes-ieee-single-native-ref}, \var{k} must
  be a multiple of $4$.}

Returns the inexact real that best represents the IEEE-754 single
precision number represented by the four bytes beginning at index
\var{k}.
\end{entry}

\begin{entry}{%
\proto{bytes-ieee-double-native-ref}{ bytes k}{procedure}
\proto{bytes-ieee-double-ref}{ bytes k endianness}{procedure}}

\domain{$\{\var{k}, \ldots, \var{k}+7\}$ must be valid indices of
  \var{bytes}.  For {\cf bytes-ieee-double-native-ref}, \var{k} must
  be a multiple of $8$.}

Returns the inexact real that best represents the IEEE-754 single
precision number represented by the eight bytes beginning at index
\var{k}.
\end{entry}

\begin{entry}{%
\proto{bytes-ieee-single-native-set!}{ bytes k x}{procedure}
\proto{bytes-ieee-single-set!}{ bytes k x endianness}{procedure}}

\domain{$\{\var{k}, \ldots, \var{k}+3\}$ must be valid indices of
  \var{bytes}.  For {\cf bytes-ieee-single-native-set!}, \var{k} must
  be a multiple of $4$.  \var{X} must be a real number.}

Stores an IEEE-754 single precision representation of \var{x} into
elements \var{k} through $\var{k}+3$ of \var{bytes}, and returns
\unspecifiedreturn.
\end{entry}

\begin{entry}{%
\proto{bytes-ieee-double-native-set!}{ bytes k x}{procedure}
\proto{bytes-ieee-double-set!}{ bytes k x endianness}{procedure}}

\domain{$\{\var{k}, \ldots, \var{k}+7\}$ must be valid indices of
  \var{bytes}.  For {\cf bytes-ieee-double-native-set!}, \var{k} must
  be a multiple of $8$.}

Stores an IEEE-754 double precision representation of \var{x} into
elements \var{k} through $\var{k}+7$ of \var{bytes}, and returns
\unspecifiedreturn.
\end{entry}

\begin{entry}{%
\proto{bytes-copy!}{ source source-start target target-start n}{procedure}}

\domain{\var{Source-start}, \var{target-start},
  and \var{n} must be non-negative exact integers that satisfy
  
  \begin{displaymath}
    \begin{array}{rcccccl}
      0 & \leq & \var{source-start} & \leq & \var{source-start} + \var{n} & \leq & l_{\var{source}}
      \\
      0 & \leq & \var{target-start} & \leq & \var{target-start} + \var{n} & \leq & l_{\var{target}}
    \end{array}
  \end{displaymath}
  %
  where $l_{\var{source}}$ is the length of \var{source} and
  $l_{\var{target}}$ is the length of \var{target}.}
   
   
  This copies the bytes from \var{source} at indices 
  \begin{displaymath}
     \{ \var{source-start}, \ldots \var{source-start} + \var{n} - 1 \}
  \end{displaymath}
  to consecutive indices in \var{target} starting at \var{target-index}.
   
  This must work even if the memory regions for the source and the target
  overlap, i.e., the bytes at the target location after the copy must be
  equal to the bytes at the source location before the copy.
   
  This returns \unspecifiedreturn.
\begin{scheme}
(let ((b (u8-list->bytes '(1 2 3 4 5 6 7 8))))
  (bytes-copy! b 0 b 3 4)
  (bytes->u8-list b)) \ev (1 2 3 1 2 3 4 8)
\end{scheme}
\end{entry}

\begin{entry}{%
\proto{bytes-copy}{ bytes}{procedure}}
   
This returns a newly allocated copy of \var{bytes}.
\end{entry}

\begin{entry}{%
\proto{bytes->u8-list}{ bytes}{procedure}
\proto{u8-list->bytes}{ list}{procedure}}
   
\ide{bytes->u8-list} returns a newly allocated list of the bytes of
\var{bytes} in the same order.

{\cf u8-list->bytes} returns a newly allocated bytes object whose
elements are the elements of list \var{list}, which must all be octets, in
the same order.  Analogous to {\cf list->vector}.
\end{entry}

\begin{entry}{%
\proto{bytes->uint-list}{ bytes endianness size}{procedure}
\proto{bytes->sint-list}{ bytes endianness size}{procedure}
\proto{uint-list->bytes}{ list endianness size}{procedure}
\proto{sint-list->bytes}{ list endianness size}{procedure}}
   
\domain{\var{Size} must be a positive exact integer.}
   
These convert between lists of integers and their consecutive
representations according to \var{size} and \var{endianness} in the
\var{bytes} objects in the same way as {\cf bytes->u8-list} and {\cf
  u8-list->bytes} do for one-byte representations.

\begin{scheme}
(let ((b (u8-list->bytes '(1 2 3 255 1 2 1 2))))
  (bytes->sint-list b (endianness little) 2)) \lev (513 -253 513 513)

(let ((b (u8-list->bytes '(1 2 3 255 1 2 1 2))))
  (bytes->uint-list b (endianness little) 2)) \lev (513 65283 513 513)
\end{scheme}
\end{entry}


%%% Local Variables: 
%%% mode: latex
%%% TeX-master: "r6rs"
%%% End: 


\chapter{\tt{eval}}
\label{evalchapter}

The \library{r6rs eval} library allows a program to create Scheme
expressions as data at run time and evaluate them.

\begin{entry}{%
\proto{eval}{ expression environment-specifier}{procedure}}

Evaluates \var{expression} in the specified environment and returns its value.
\var{Expression} must be a valid Scheme expression represented as a
datum value, and \var{environment-specifier} must be a 
\defining{library specifier}, which can be created using the {\cf
  environment} procedure described below.

If the first argument to {\cf eval} is determined not to be a syntactically correct
expression, then {\cf eval} must raise an exception with condition
type {\cf \&syntax}.  Specifically, if the first argument to {\cf
  eval} is a definition or a splicing {\cf begin} form containing a
definition, it must raise an exception with condition type {\cf
  \&syntax}.
\end{entry}

\begin{entry}{%
\proto{environment}{ import-spec \dots}{procedure}}

\domain{\var{Import-spec} must be a datum representing an
  \hyper{import spec} (see report
  section~\extref{report:librarysyntaxsection}{Library form}).}
The {\cf environment} procedure returns an environment corresponding
to \var{import-spec}

The bindings of the environment represented by the specifier are
immutable: If {\cf eval} is applied to an expression that is
determined to contain an
assignment to one of the variables of the environment, then {\cf eval} must
raise an exception with a condition type {\cf\&assertion}.

\begin{scheme}
(library (foo)
  (export)
  (import (r6rs))
  (write
    (eval '(let ((x 3)) x)
          (environment '(r6rs))))) \\\> {\it writes} 3

(library (foo)
  (export)
  (import (r6rs))
  (write
    (eval
      '(eval:car (eval:cons 2 4))
      (environment
        '(prefix (only (r6rs) car cdr cons null?)
                 eval:))))) \\\> {\it writes} 2
\end{scheme}
\end{entry}

%%% Local Variables: 
%%% mode: latex
%%% TeX-master: "r6rs-lib"
%%% End: 


%%% Local Variables: 
%%% mode: latex
%%% TeX-master: "r6rs"
%%% End: 




\chapter{Miscellaneous libraries}
\label{misclibchapter}

\section{{\tt when} and {\tt unless}}

This section describes the \deflibrary{r6rs when-unless} library.

\begin{entry}{%
\proto{when}{ \hyper{test} \hyperi{expression} \hyperii{expression} \dotsfoo}{\exprtype}
\proto{unless}{ \hyper{test} \hyperi{expression} \hyperii{expression} \dotsfoo}{\exprtype}}

\syntax \hyper{Test} must be an expression.
\semantics A {\cf when} expression is evaluated by evaluating the
\hyper{test} expression.  If \hyper{test} evaluates to a true value,
the remaining \hyper{expression}s are evaluated in order, and the
result(s) of the last \hyper{expression} is(are) returned as the
result(s) of the entire {\cf when} expression.  Otherwise, the {\cf
	  when} expression evaluates to the unspecified value.  An {\cf unless}
expression is evaluated by evaluating the \hyper{test} expression.
If \hyper{test} evaluates to false, the remaining
\hyper{expression}s are evaluated in order, and the result(s) of the
last \hyper{expression} is(are) returned as the result(s) of the
entire {\cf unless} expression.  Otherwise, the {\cf unless} expression
evaluates to the unspecified value.

\begin{scheme}
(when (> 3 2) 'greater) \ev greater
(when (< 3 2) 'greater) \ev \theunspecified
(unless (> 3 2) 'less) \ev \theunspecified
(unless (< 3 2) 'less) \ev less
\end{scheme}

The {\cf when} and {\cf unless} expressions are derived forms.  They
could be defined in terms of base library forms by the following macros:

\begin{scheme}
(define-syntax \ide{when}
  (syntax-rules ()
    ((when test result1 result2 ...)
     (if test
         (begin result1 result2 ...)))))

(define-syntax \ide{unless}
  (syntax-rules ()
    ((unless test result1 result2 ...)
     (if (not test)
         (begin result1 result2 ...)))))
\end{scheme}

\end{entry}

\section{{\tt case-lambda}}

This section describes the \deflibrary{r6rs case-lambda} library.

\begin{entry}{%
\proto{case-lambda}{ \hyperi{clause} \hyperii{clause} \dotsfoo}{\exprtype}}
    
\syntax
Each \hyper{clause} should be of the form
%
\begin{scheme}
(\hyper{formals} \hyper{body})%
\end{scheme}

\hyper{Formals} must be as in a {\cf lambda} form
(section~\ref{lambda}), \hyper{body} must be a body according to
section~\ref{bodiessection}.

\semantics A {\cf case-lambda} expression evaluates to a procedure.
This procedure, when applied, tries to match its arguments to the
\hyper{clause}s in order.  The arguments match a clause if one the
following conditions is fulfilled:
%
\begin{itemize}
\item \hyper{Formals} has the form {\tt (\hyper{variable} \dotsfoo)}
and the number of arguments is the same as the number of formal
parameters in \hyper{formals}.
\item \hyper{Formals} has the form\\ {\tt
(\hyperi{variable} \dotsfoo \hypern{variable} . \hyper{variable$_{n+1}$)}
}\\
and the number of arguments is at least $n$.
\item \hyper{Formals} has the form {\tt \hyper{variable}}.
\end{itemize}
%
For the first clause matched by the arguments, the variables of the
\hyper{formals} are bound to fresh locations containing the
argument values in the same arrangement as with {\cf lambda}.

If the arguments match none of the clauses, an exception with condition 
type {\cf\&contract} is raised.

\begin{scheme}
(define foo
  (case-lambda 
   (() 'zero)
   ((x) (list 'one x))
   ((x y) (list 'two x y))
   ((a b c d . e) (list 'four a b c d e))
   (rest (list 'rest rest))))

(foo) \ev zero
(foo 1) \ev (one 1)
(foo 1 2) \ev (two 1 2)
(foo 1 2 3) \ev (rest (1 2 3))
(foo 1 2 3 4) \ev (four 1 2 3 4 ())
\end{scheme}

A sample definition of {\cf case-lambda} in terms of simpler forms is in
appendix~\ref{derivedformsappendix}.
\end{entry}

\section{Delayed evaluation}

This section describes the \deflibrary{r6rs promises} library.

\begin{entry}{%
\proto{delay}{ \hyper{expression}}{\exprtype}}

\todo{Fix.}

The {\cf delay} construct is used together with the procedure \ide{force} to
implement \defining{lazy evaluation} or \defining{call by need}.
{\tt(delay~\hyper{expression})} returns an object called a
\defining{promise} which at some point in the future may be asked (by
the {\cf force} procedure) \todo{Bartley's white lie; OK?} to evaluate
\hyper{expression}, and deliver the resulting value.
The effect of \hyper{expression} returning multiple values
is unspecified.

See the description of {\cf force} (section~\ref{force}) for a
more complete description of {\cf delay}.

\end{entry}

\begin{entry}{%
\proto{force}{ promise}{procedure}}

{\var{Promise} must be a promise.}

Forces the value of \var{promise} (see \ide{delay},
section~\ref{delay}).\index{promise}  If no value has been computed for
the promise, then a value is computed and returned.  The value of the
promise is cached (or ``memoized'') so that if it is forced a second
time, the previously computed value is returned.
% without any recomputation.
% [As pointed out by Marc Feeley, the "without any recomputation"
% isn't necessarily true. --Will]

\begin{scheme}
(force (delay (+ 1 2)))   \ev  3
(let ((p (delay (+ 1 2))))
  (list (force p) (force p)))  
                               \ev  (3 3)

(define a-stream
  (letrec ((next
            (lambda (n)
              (cons n (delay (next (+ n 1)))))))
    (next 0)))
(define head car)
(define tail
  (lambda (stream) (force (cdr stream))))

(head (tail (tail a-stream)))  
                               \ev  2%
\end{scheme}

Promises are mainly intended for programs written in
functional style.  The following examples should not be considered to
illustrate good programming style, but they illustrate the property that
only one value is computed for a promise, no matter how many times it is
forced.
% the value of a promise is computed at most once.
% [As pointed out by Marc Feeley, it may be computed more than once,
% but as I observed we can at least insist that only one value be
% used! -- Will]

\begin{scheme}
(define count 0)
(define p
  (delay (begin (set! count (+ count 1))
                (if (> count x)
                    count
                    (force p)))))
(define x 5)
p                     \ev  {\it{}a promise}
(force p)             \ev  6
p                     \ev  {\it{}a promise, still}
(begin (set! x 10)
       (force p))     \ev  6%
\end{scheme}

Here is a possible implementation of {\cf delay} and {\cf force}.
Promises are implemented here as procedures of no arguments,
and {\cf force} simply calls its argument:

\begin{scheme}
(define force
  (lambda (object)
    (object)))%
\end{scheme}

The expression

\begin{scheme}
(delay \hyper{expression})%
\end{scheme}

has the same meaning as the procedure call

\begin{scheme}
(make-promise (lambda () \hyper{expression}))\rm
\end{scheme}

as follows

\begin{scheme}
(define-syntax delay
  (syntax-rules ()
    ((delay expression)
     (make-promise (lambda () expression))))),%
\end{scheme}

where {\cf make-promise} is defined as follows:

% \begin{scheme}
% (define make-promise
%   (lambda (proc)
%     (let ((already-run? \schfalse) (result \schfalse))
%       (lambda ()
%         (cond ((not already-run?)
%                (set! result (proc))
%                (set! already-run? \schtrue)))
%         result))))%
% \end{scheme}

\begin{scheme}
(define make-promise
  (lambda (proc)
    (let ((result-ready? \schfalse)
          (result \schfalse))
      (lambda ()
        (if result-ready?
            result
            (let ((x (proc)))
              (if result-ready?
                  result
                  (begin (set! result-ready? \schtrue)
                         (set! result x)
                         result))))))))%
\end{scheme}

\begin{rationale}
A promise may refer to its own value, as in the last example above.
Forcing such a promise may cause the promise to be forced a second time
before the value of the first force has been computed.
This complicates the definition of {\cf make-promise}.
\end{rationale}

Various extensions to this semantics of {\cf delay} and {\cf force}
are supported in some implementations:

\begin{itemize}
\item Calling {\cf force} on an object that is not a promise may simply
return the object.

\item It may be the case that there is no means by which a promise can be
operationally distinguished from its forced value.  That is, expressions
like the following may evaluate to either \schtrue{} or to \schfalse{},
depending on the implementation:

\begin{scheme}
(eqv? (delay 1) 1)          \ev  \unspecified
(pair? (delay (cons 1 2)))  \ev  \unspecified%
\end{scheme}

\item Some implementations may implement ``implicit forcing,'' where
the value of a promise is forced by primitive procedures like \cf{cdr}
and \cf{+}:

\begin{scheme}
(+ (delay (* 3 7)) 13)  \ev  34%
\end{scheme}
\end{itemize}
\end{entry}

\chapter{\tt{eval}}
\label{evalchapter}

The \library{r6rs eval} library allows a program to create Scheme
expressions as data at run time and evaluate them.

\begin{entry}{%
\proto{eval}{ expression environment-specifier}{procedure}}

Evaluates \var{expression} in the specified environment and returns its value.
\var{Expression} must be a valid Scheme expression represented as a
datum value, and \var{environment-specifier} must be a 
\defining{library specifier}, which can be created using the {\cf
  environment} procedure described below.

If the first argument to {\cf eval} is determined not to be a syntactically correct
expression, then {\cf eval} must raise an exception with condition
type {\cf \&syntax}.  Specifically, if the first argument to {\cf
  eval} is a definition or a splicing {\cf begin} form containing a
definition, it must raise an exception with condition type {\cf
  \&syntax}.
\end{entry}

\begin{entry}{%
\proto{environment}{ import-spec \dots}{procedure}}

\domain{\var{Import-spec} must be a datum representing an
  \hyper{import spec} (see report
  section~\extref{report:librarysyntaxsection}{Library form}).}
The {\cf environment} procedure returns an environment corresponding
to \var{import-spec}

The bindings of the environment represented by the specifier are
immutable: If {\cf eval} is applied to an expression that is
determined to contain an
assignment to one of the variables of the environment, then {\cf eval} must
raise an exception with a condition type {\cf\&assertion}.

\begin{scheme}
(library (foo)
  (export)
  (import (r6rs))
  (write
    (eval '(let ((x 3)) x)
          (environment '(r6rs))))) \\\> {\it writes} 3

(library (foo)
  (export)
  (import (r6rs))
  (write
    (eval
      '(eval:car (eval:cons 2 4))
      (environment
        '(prefix (only (r6rs) car cdr cons null?)
                 eval:))))) \\\> {\it writes} 2
\end{scheme}
\end{entry}

%%% Local Variables: 
%%% mode: latex
%%% TeX-master: "r6rs-lib"
%%% End: 


\section{Command-line access}
\label{scriptlibsection}

The procedure described in this section is exported by the
\deflibrary{r6rs programs} library.

\begin{entry}{%
\proto{command-line}{}{procedure}}

When a script is being executed, this returns a list of strings with
at least one element.  The first element is an implementation-specific
name for the running script.  The following elements are command-line
arguments according to the operating system's conventions.
\end{entry}

%%% Local Variables: 
%%% mode: latex
%%% TeX-master: "r6rs-lib"
%%% End: 


\chapter{R$^5$RS compatibility}
\label{r5rscompatchapter}

The features described in this chapter are exported from the
\defrsixlibrary{r5rs} library and provide some functionality of the
preceding revision of this report~\cite{R5RS} that was omitted from
the main part of the current report.

\begin{entry}{%
\proto{exact->inexact}{ z}{procedure}
\proto{inexact->exact}{ z}{procedure}}

These are the same as the {\cf inexact} and {\cf exact}
procedures; see report section~\extref{report:inexact}{Generic conversions}.
\end{entry}

\begin{entry}{%
\proto{quotient}{ \vari{n} \varii{n}}{procedure}
\proto{remainder}{ \vari{n} \varii{n}}{procedure}
\proto{modulo}{ \vari{n} \varii{n}}{procedure}}

These procedures implement number-theoretic (integer)
division.  \varii{N} must be non-zero.  All three procedures
return integer objects.  If \vari{n}/\varii{n} is an integer object:
\begin{scheme}
    (quotient \vari{n} \varii{n})   \ev \vari{n}/\varii{n}
    (remainder \vari{n} \varii{n})  \ev 0
    (modulo \vari{n} \varii{n})     \ev 0
\end{scheme}
If \vari{n}/\varii{n} is not an integer object:
\begin{scheme}
    (quotient \vari{n} \varii{n})   \ev \var{n$_q$}
    (remainder \vari{n} \varii{n})  \ev \var{n$_r$}
    (modulo \vari{n} \varii{n})     \ev \var{n$_m$}
\end{scheme}
where \var{n$_q$} is $\vari{n}/\varii{n}$ rounded towards zero,
$0 < |\var{n$_r$}| < |\varii{n}|$, $0 < |\var{n$_m$}| < |\varii{n}|$,
\var{n$_r$} and \var{n$_m$} differ from \vari{n} by a multiple of \varii{n},
\var{n$_r$} has the same sign as \vari{n}, and
\var{n$_m$} has the same sign as \varii{n}.

Consequently, for integer objects \vari{n} and \varii{n} with
\varii{n} not equal to 0,
\begin{scheme}
     (= \vari{n} (+ (* \varii{n} (quotient \vari{n} \varii{n}))
           (remainder \vari{n} \varii{n})))
                                 \ev  \schtrue%
\end{scheme}
provided all number object involved in that computation are exact.

\begin{scheme}
(modulo 13 4)           \ev  1
(remainder 13 4)        \ev  1

(modulo -13 4)          \ev  3
(remainder -13 4)       \ev  -1

(modulo 13 -4)          \ev  -3
(remainder 13 -4)       \ev  1

(modulo -13 -4)         \ev  -1
(remainder -13 -4)      \ev  -1

(remainder -13 -4.0)    \ev  -1.0%
\end{scheme}

\begin{note}
  These procedures could be defined in terms of {\cf div} and {\cf
    mod} (see report section~\extref{report:div}{Arithmetic operations}) as follows (without checking of the
  argument types):
\begin{scheme}
(define (sign n)
  (cond
    ((negative? n) -1)
    ((positive? n) 1)
    (else 0)))

(define (quotient n1 n2)
  (* (sign n1) (sign n2) (div (abs n1) (abs n2))))

(define (remainder n1 n2)
  (* (sign n1) (mod (abs n1) (abs n2))))

(define (modulo n1 n2)
  (* (sign n2) (mod (* (sign n2) n1) (abs n2))))
\end{scheme}
\end{note}
\end{entry}

\begin{entry}{%
\proto{delay}{ \hyper{expression}}{\exprtype}}

The {\cf delay} construct is used together with the procedure \ide{force} to
implement \defining{lazy evaluation} or \defining{call by need}.
{\tt(delay~\hyper{expression})} returns an object called a
\defining{promise} which at some point in the future may be asked (by
the {\cf force} procedure) to evaluate
\hyper{expression}, and deliver the resulting value.
The effect of \hyper{expression} returning multiple values
is unspecified.

\end{entry}

\begin{entry}{%
\proto{force}{ promise}{procedure}}

{\var{Promise} must be a promise.}

Forces the value of \var{promise}.  If no value has been computed for
the promise, then a value is computed and returned.  The value of the
promise is cached (or ``memoized'') so that if it is forced a second
time, the previously computed value is returned.

\begin{scheme}
(force (delay (+ 1 2)))   \ev  3
(let ((p (delay (+ 1 2))))
  (list (force p) (force p)))  
                               \ev  (3 3)

(define a-stream
  (letrec ((next
            (lambda (n)
              (cons n (delay (next (+ n 1)))))))
    (next 0)))
(define head car)
(define tail
  (lambda (stream) (force (cdr stream))))

(head (tail (tail a-stream)))  
                               \ev  2%
\end{scheme}

Promises are mainly intended for programs written in
functional style.  The following examples should not be considered to
illustrate good programming style, but they illustrate the property that
only one value is computed for a promise, no matter how many times it is
forced.

\begin{scheme}
(define count 0)
(define p
  (delay (begin (set! count (+ count 1))
                (if (> count x)
                    count
                    (force p)))))
(define x 5)
p                     \ev  {\it{}a promise}
(force p)             \ev  6
p                     \ev  {\it{}a promise, still}
(begin (set! x 10)
       (force p))     \ev  6%
\end{scheme}

Here is a possible implementation of {\cf delay} and {\cf force}.
Promises are implemented here as procedures of no arguments,
and {\cf force} simply calls its argument:

\begin{scheme}
(define force
  (lambda (object)
    (object)))%
\end{scheme}

The expression

\begin{scheme}
(delay \hyper{expression})%
\end{scheme}

has the same meaning as the procedure call

\begin{scheme}
(make-promise (lambda () \hyper{expression}))%
\end{scheme}

as follows

\begin{scheme}
(define-syntax delay
  (syntax-rules ()
    ((delay expression)
     (make-promise (lambda () expression))))),%
\end{scheme}

where {\cf make-promise} is defined as follows:

\begin{scheme}
(define make-promise
  (lambda (proc)
    (let ((result-ready? \schfalse)
          (result \schfalse))
      (lambda ()
        (if result-ready?
            result
            (let ((x (proc)))
              (if result-ready?
                  result
                  (begin (set! result-ready? \schtrue)
                         (set! result x)
                         result))))))))%
\end{scheme}
\end{entry}

\begin{entry}{%
\proto{null-environment}{ n}{procedure}}

\domain{\var{N} must be the exact integer object 5.}  The {\cf
  null-environment} procedure returns an
environment specifier suitable for use with {\cf eval} (see
chapter~\ref{evalchapter}) representing an environment that is empty except
for the (syntactic) bindings for all keywords described in
the previous revision of this report~\cite{R5RS}.
\end{entry}

\begin{entry}{%
\proto{scheme-report-environment}{ n}{procedure}}

\domain{\var{N} must be the exact integer object 5.}  The {\cf scheme-report-environment} procedure returns
an environment specifier for an environment that is empty except for
the bindings for the identifiers described in the previous
revision of this report~\cite{R5RS}, omitting {\cf load}, {\cf
  interaction-environment}, {\cf
  transcript-on}, {\cf transcript-off}, and {\cf char-ready?}.  The
bindings have as values the procedures of the same names described in
this report.
\end{entry}


%%% Local Variables: 
%%% mode: latex
%%% TeX-master: "r6rs-lib"
%%% End: 


%%% Local Variables: 
%%% mode: latex
%%% TeX-master: "r6rs"
%%% End: 




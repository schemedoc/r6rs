\chapter{Miscellaneous libraries}
\label{misclibchapter}

\subsection{Delayed evaluation}\unsection

\begin{entry}{%
\proto{delay}{ \hyper{expression}}{library \exprtype}}

\todo{Fix.}

The {\cf delay} construct is used together with the procedure \ide{force} to
implement \defining{lazy evaluation} or \defining{call by need}.
{\tt(delay~\hyper{expression})} returns an object called a
\defining{promise} which at some point in the future may be asked (by
the {\cf force} procedure) \todo{Bartley's white lie; OK?} to evaluate
\hyper{expression}, and deliver the resulting value.
The effect of \hyper{expression} returning multiple values
is unspecified.

See the description of {\cf force} (section~\ref{force}) for a
more complete description of {\cf delay}.

\end{entry}

\begin{entry}{%
\proto{force}{ promise}{library procedure}}

Forces the value of \var{promise} (see \ide{delay},
section~\ref{delay}).\index{promise}  If no value has been computed for
the promise, then a value is computed and returned.  The value of the
promise is cached (or ``memoized'') so that if it is forced a second
time, the previously computed value is returned.
% without any recomputation.
% [As pointed out by Marc Feeley, the "without any recomputation"
% isn't necessarily true. --Will]

\begin{scheme}
(force (delay (+ 1 2)))   \ev  3
(let ((p (delay (+ 1 2))))
  (list (force p) (force p)))  
                               \ev  (3 3)

(define a-stream
  (letrec ((next
            (lambda (n)
              (cons n (delay (next (+ n 1)))))))
    (next 0)))
(define head car)
(define tail
  (lambda (stream) (force (cdr stream))))

(head (tail (tail a-stream)))  
                               \ev  2%
\end{scheme}

{\cf Force} and {\cf delay} are mainly intended for programs written in
functional style.  The following examples should not be considered to
illustrate good programming style, but they illustrate the property that
only one value is computed for a promise, no matter how many times it is
forced.
% the value of a promise is computed at most once.
% [As pointed out by Marc Feeley, it may be computed more than once,
% but as I observed we can at least insist that only one value be
% used! -- Will]

\begin{scheme}
(define count 0)
(define p
  (delay (begin (set! count (+ count 1))
                (if (> count x)
                    count
                    (force p)))))
(define x 5)
p                     \ev  {\it{}a promise}
(force p)             \ev  6
p                     \ev  {\it{}a promise, still}
(begin (set! x 10)
       (force p))     \ev  6%
\end{scheme}

Here is a possible implementation of {\cf delay} and {\cf force}.
Promises are implemented here as procedures of no arguments,
and {\cf force} simply calls its argument:

\begin{scheme}
(define force
  (lambda (object)
    (object)))%
\end{scheme}

We define the expression

\begin{scheme}
(delay \hyper{expression})%
\end{scheme}

to have the same meaning as the procedure call

\begin{scheme}
(make-promise (lambda () \hyper{expression}))\rm
\end{scheme}

as follows

\begin{scheme}
(define-syntax delay
  (syntax-rules ()
    ((delay expression)
     (make-promise (lambda () expression))))),%
\end{scheme}

where {\cf make-promise} is defined as follows:

% \begin{scheme}
% (define make-promise
%   (lambda (proc)
%     (let ((already-run? \schfalse) (result \schfalse))
%       (lambda ()
%         (cond ((not already-run?)
%                (set! result (proc))
%                (set! already-run? \schtrue)))
%         result))))%
% \end{scheme}

\begin{scheme}
(define make-promise
  (lambda (proc)
    (let ((result-ready? \schfalse)
          (result \schfalse))
      (lambda ()
        (if result-ready?
            result
            (let ((x (proc)))
              (if result-ready?
                  result
                  (begin (set! result-ready? \schtrue)
                         (set! result x)
                         result))))))))%
\end{scheme}

\begin{rationale}
A promise may refer to its own value, as in the last example above.
Forcing such a promise may cause the promise to be forced a second time
before the value of the first force has been computed.
This complicates the definition of {\cf make-promise}.
\end{rationale}

Various extensions to this semantics of {\cf delay} and {\cf force}
are supported in some implementations:

\begin{itemize}
\item Calling {\cf force} on an object that is not a promise may simply
return the object.

\item It may be the case that there is no means by which a promise can be
operationally distinguished from its forced value.  That is, expressions
like the following may evaluate to either \schtrue{} or to \schfalse{},
depending on the implementation:

\begin{scheme}
(eqv? (delay 1) 1)          \ev  \unspecified
(pair? (delay (cons 1 2)))  \ev  \unspecified%
\end{scheme}

\item Some implementations may implement ``implicit forcing,'' where
the value of a promise is forced by primitive procedures like \cf{cdr}
and \cf{+}:

\begin{scheme}
(+ (delay (* 3 7)) 13)  \ev  34%
\end{scheme}
\end{itemize}
\end{entry}

\section{\tt{Eval}}

\begin{entry}{%
\proto{eval}{ expression environment-specifier}{procedure}}

Evaluates \var{expression} in the specified environment and returns its value.
\var{Expression} must be a valid Scheme expression represented as data,
and \var{environment-specifier} must be a value returned by one of the
three procedures described below.
Implementations may extend {\cf eval} to allow non-expression programs
(definitions) as the first argument and to allow other
values as environments, with the restriction that {\cf eval} is not
allowed to create new bindings in the environments associated with
{\cf null-environment} or {\cf scheme-report-environment}.

\begin{scheme}
(eval '(* 7 3) (scheme-report-environment 5))
                                                   \ev  21

(let ((f (eval '(lambda (f x) (f x x))
               (null-environment 5))))
  (f + 10))
                                                   \ev  20
\end{scheme}

\end{entry}

\begin{entry}{%
\proto{scheme-report-environment}{ version}{procedure}
\proto{null-environment}{ version}{procedure}}

\var{Version} must be the exact integer {\cf \integerversion},
corresponding to this revision of the Scheme report (the
Revised$^\integerversion$ Report on Scheme).
{\cf Scheme-report-environment} returns a specifier for an
environment that is empty except for all bindings defined in
this report that are either required or both optional and
supported by the implementation. {\cf Null-environment} returns
a specifier for an environment that is empty except for the
(syntactic) bindings for all syntactic keywords defined in
this report that are either required or both optional and
supported by the implementation.

Other values of \var{version} can be used to specify environments
matching past revisions of this report, but their support is not
required.  An implementation will signal an error if \var{version}
is neither {\cf \integerversion} nor another value supported by
the implementation.

The effect of assigning (through the use of {\cf eval}) a variable
bound in a {\cf scheme-report-environment}
(for example {\cf car}) is unspecified.  Thus the environments specified
by {\cf scheme-report-environment} may be immutable.

\end{entry}

\begin{entry}{%
\proto{interaction-environment}{}{optional procedure}}

This procedure returns a specifier for the environment that
contains imple\-men\-ta\-tion-defined bindings, typically a superset of
those listed in the report.  The intent is that this procedure
will return the environment in which the implementation would evaluate
expressions dynamically typed by the user.

\end{entry}

%%% Local Variables: 
%%% mode: latex
%%% TeX-master: "r6rs"
%%% End: 




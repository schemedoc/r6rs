\section{Port I/O}
\label{portsiosection}

The \library{r6rs i/o ports} library defines an I/O layer for
conventional, imperative buffered input and output with mixed text and
binary data.  A \defining{port} represents a buffered access object
for a data sink or source or both simultaneously.  The library allows
creating ports from from arbitrary input sources and sink represented
as readers and writers (see section~\ref{primiosection}), but does not
require that all ports be built from readers and writers.

The \library{r6rs i/o ports} distinguishes between \textit{input
  ports\mainindex{input port}} and \textit{output
  ports\mainindex{output ports}}.  An input port is a source for data,
whereas an output port is a sink for data.  A port may be both an
input and an output source; such a port typically provides
simultaneous read and write access to a file or other piece of data.

This section uses the \var{input-port}, \var{output-port}, \var{port}
parameter names for arguments that must be input ports (or combined
input/output ports), output ports (or combined input/output ports),
or any kind of port.

\subsection{Condition type}

This library introduces the following condition type:

\begin{entry}{%
\ctproto{i/o-port}
\proto{i/o-port-error}{ obj}{procedure}
\proto{i/o-error-port}{ condition}{procedure}}

This condition type could be defined by
%
\begin{scheme}
(define-condition-type \&i/o-port \&i/o
  i/o-port-error?
  (port i/o-error-port))
\end{scheme}

This condition type allows specifying with what particular port an I/O
error is associated. Except for condition objects provided for
encoding and decoding errors, conditions raised by procedures may
include an {\cf\&i/o-port-error} condition, but are not required to do
so.
\end{entry}

\subsection{Buffer modes}

Each output port has an associated buffer mode that defines when an
output operation will flush the buffer associated with the output
port. The possible buffer modes are the symbols {\cf none} for no buffering,
{\cf line} for flushing upon line feeds and line separators (U+2028), and
{\cf block} for block-based buffering.  This section uses the parameter name
\var{buffer-name} for arguments that must be buffer-mode symbols.

\begin{entry}{%
\proto{buffer-mode}{ \hyper{name}}{\exprtype}}
   
\hyper{Name} must be one of the symbols {\cf none}, {\cf line}, and
{\cf block}. This returns the corresponding symbol denoting the
associated buffer mode.

If name is not one of the valid identifiers, an exception must be
raised with condition type {\cf\&syntax} at macro-expansion time.
\end{entry}

\begin{entry}{%
\proto{buffer-mode?}{ obj}{procedure}}
   
This returns \schtrue{} if the argument is a valid buffer-mode symbol,
\schfalse{} otherwise.
\end{entry}

\subsection{Text transcoders}

Several different Unicode encoding schemes describe standard ways to
encode characters and strings as byte sequences and to decode those
sequences.  With this document, a \defining{codec} is a Scheme object
that somehow represents a Unicode or similar encoding scheme.

The text transcoders of this document generalize codecs
to deal with common end-of-line conventions, and different
error-handling modes.

A \defining{transcoder} is an opaque object that somehow represents
some specific bidirectional (but not necessarily lossless) translation
between byte sequences and Unicode characters and strings.  The
transcoder specifies how procedures that perform textual input are to
interpret input bytes as characters or strings.  The transcoder also
specifies how procedures that perform textual output are to translate
characters into bytes.  Moreover, the transcoder specifies a
mode for handling encoding or decoding errors.

A \var{transcoder} parameter name means that the corresponding
argument must be a transcoder.

\begin{entry}{%
\proto{utf-8-codec}{}{procedure}
\proto{latin-1-codec}{}{procedure}
\proto{utf-16le-codec}{}{procedure}
\proto{utf-16be-codec}{}{procedure}
\proto{utf-32le-codec}{}{procedure}
\proto{utf-32be-codec}{}{procedure}}

These are predefined codecs for the UTF-8, ISO8859-1,
UTF-16LE, UTF-16BE, UTF32-LE, and UTF-32BE encoding
schemes.

A call to any of these procedures returns a value that is equal in the
sense of {\cf eqv?} to the result of any other call to the same
procedure.
\end{entry}

\begin{entry}{%
\proto{utf-16bom-codec}{}{procedure}}
   
This predefined codec can only be used with input ports. The first
attempt to read from a port transcoded using a transcoder with this
codec will lead to the first two bytes of the port being read, which
must be a byte-order mark. (The byte-order mark is not returned as
part of the read operation.) If they are \#xFE, \#xFF, the decoding of
the following data happens according to UTF-16be, if they are \#xFF,
\#xFE, the decoding happens according to UTF-16le.  If they are
anything else, an exception with condition type {\cf\&i/o-decoding} is
raised.
\end{entry}

\begin{entry}{%
\proto{utf-32bom-codec}{}{procedure}}
   
This predefined codec can only be used with input ports.  The first
attempt to read from a port transcoded using a transcoder with this
codec will lead to the first four bytes of the port being read, which
must be a byte-order mark. (The byte-order mark is not returned as
part of the read operation.) If they are \#x00, \#x00, \#xFE, \#xFF,
the decoding of the following data happens according to UTF-32be, if
they are \#xFF, \#xFE, \#x00, \#x00 the decoding happens according to
UTF-16le.  If they are anything else, an exception with condition type
{\cf\&i/o-decoding} is raised.
\end{entry}

\begin{entry}{%
\proto{eol-style}{ name}{\exprtype}}

If \var{name} is one of the symbols {\cf lf}, {\cf cr}, {\cf crlf}, or
{\cf ls}, then this form evaluates to \var{name}.  If \var{name} is
not one of these symbols, the effect of evaluating this expression is
implementation-dependent.

\begin{rationale}
  End-of-line styles other than those listed might become commonplace
  in the future.
\end{rationale}
\end{entry}

\begin{entry}{%
\proto{native-eol-style}{}{procedure}}

This returns the default end-of-line style of the underlying platform, i.e.
{\cf lf} on Unix and {\cf crlf} on Windows.
\end{entry}

\begin{entry}{%
\ctproto{i/o-decoding}
\proto{i/o-decoding-error?}{ obj}{procedure}}

\begin{scheme}
(define-condition-type \&i/o-decoding \&i/o-port
  i/o-decoding-error?
  (transcoder i/o-decoding-error-transcoder))
\end{scheme}

An exception with this type is raised when one of the operations for
textual output to a port encounters a character not supported by the
codec of the specified transcoder.  The {\cf transcoder} field
contains the transcoder specified with the operation.

Exceptions of this type raised by the operations described in this
section are continuable.  When it is raised, the port's position is at
the beginning of the invalid encoding.  The exception handler must
return a character or string representing the decoded text starting at
the port's current position, and update the port's position to point
past the error.
\end{entry}

\begin{entry}{% 
\ctproto{i/o-encoding}
\proto{i/o-encoding-error?}{ obj}{procedure}
\proto{i/o-encoding-error-char}{ condition}{procedure}
\proto{i/o-encoding-error-transcoder}{ condition}{procedure}}

This condition type could be defined by
%
\begin{scheme}
(define-condition-type \&i/o-encoding \&i/o-port
  i/o-encoding-error?
  (char i/o-encoding-error-char)
  (transcoder i/o-encoding-error-transcoder))
\end{scheme}

An exception with this type is raised when one of the operations for
textual output to a port encounters a character not supported by the
codec of the specified transcoder.  The {\cf char} field of the
condition object contains the character that the operation was unable
to encode, and the {\cf transcoder} field contains the transcoder
specified with the operation.

Exceptions of this type raised by the operations described in this
section are continuable.  The handler, if it returns, is expected to
output to the port an appropriate encoding for the character that
caused the error.  The operation that raised the exception will then
continue after that character.
\end{entry}

\begin{entry}{%
\proto{error-handling-mode}{ name}{\exprtype}}

If \var{name} is one of the symbols {\cf ignore}, {\cf raise}, or {\cf
  replace}, then this form evaluates to \var{name}.  If \var{name} is
not one of these symbols, the effect of evaluating this expression is
implementation-dependent.

\begin{rationale}
  Implementations may support error-handling modes other than those
  listed.
\end{rationale}

When part of a transcoder, an error-handling mode specifies the behvior
of a text I/O operation in the presence of an encoding or decoding
error:

If a codec for an input port encounters an invalid or incomplete
character encoding, it will behave according to the specified
error-handling mode. If it is {\cf ignore}, the first byte of the
invalid encoding will be ignored and decoding continues with the next
byte.  If it is {\cf replace}, the replacement character U+FFFD will
be injected into the data stream.  Decoding subsequently continues
with the next byte. If it is {\cf raise}, a continuable exception with
condition type {\cf\&i/o-decoding} is raised.  See the description of
{\cf\&i/o-decoding} for details
on how to handle such an exception.

If a codec for an output port encounters a character it cannot encode,
it will behave according to the specified error-handling mode. If it
is {\cf ignore}, the character is ignored and encoding continues after
the encoding. If it is {\cf replace}, an encoding-specific replacement
character will be emitted by the transcoder, and decoding continues
after the encoding. This replacement character is U+FFFD for the
Unicode encodings capable of representing it, and the {\cf ?}
character for the Latin-1 encoding. If the mode is {\cf raise}, an
exception with condition type {\cf\&i/o-encoding} is raised.
See the description of
{\cf\&i/o-decoding} for details
on how to handle such an exception.
\end{entry}

\begin{entry}{%
\proto{make-transcoder}{ codec eol-style handling-mode}{procedure}
\rproto{make-transcoder}{ codec eol-style}{procedure}
\rproto{make-transcoder}{ codec}{procedure}}

\domain{\var{Codec} must be a codec, \var{eol-style}, if present, an
  eol-style symbol, and \var{handling-mode}, if present, an
  error-handling-mode symbol.}  \var{Eol-style} may be omitted, in
which case it defaults to the native end-of-line style of the
underlying platform.  \var{Handling-mode} may be omitted, in which
case it defaults to {\cf raise}.  This returns a transcoder with the
behavior specified by its arguments.

A transcoder returned by {\cf make-transcoder} is equal in the sense
of {\cf eqv?} to any other transcoder returnd by {\cf
  make-transcoder}, if and only if the \var{code}, \var{eol-style},
and \var{handling-mode} arguments are equal in the sense of {\cf
  eqv?}.
\end{entry}

\begin{entry}{%
\proto{transcoder-codec}{ transcoder}{procedure}
\proto{transcoder-eol-style}{ transcoder}{procedure}
\proto{transcoder-error-handling-mode}{ transcoder}{procedure}}

These are accessors for transcoder object; when applied to a
transcoder returned by {\cf make-transcoder}, they return the
\var{codec}, \var{eol-style}, \var{handling-mode} arguments.
\end{entry}

\section{Input and output ports}

The operations described in this section are common to input and
output ports.  A port may have an associated \defining{position} that
specifies a particular place within its data sink or source as a byte
count from the beginning of the sink or source, and operations for
inspecting and setting it.  A port may also have an associated
transcoder that represents a default text encoding associated with the
port.  Note that the transcoder associated with a port does not have
any direct effect to the procedures performing textual I/O.  However,
it can be passed to those procedures to specify that the text encoding
or decoding should happen according to that transcoder.

\begin{entry}{%
\proto{port?}{ obj}{procedure}}
   
Returns \schtrue{} if the argument is a port, and returns \schfalse{}
otherwise.
\end{entry}

\begin{entry}{%
\proto{port-transcoder}{ port}{procedure}}

This returns the transcoder associated with \var{port}, if it has one,
or \schfalse{} if no transcoder is associated with \var{port}.
\end{entry}

\begin{entry}{%
\proto{port-has-port-position?}{ port}{procedure}
\proto{port-position}{ port}{procedure}}

The {\cf port-has-port-position?} procedure returns \schtrue{} if the
port supports the {\cf port-position} operation, and \schfalse{}
otherwise.

Returns the exact non-negative integer index of the position at which the
next byte would be read from or written to the port.
   
This procedure raises an exception with condition type {\cf\&contract}
if the port does not support the operation.
\end{entry}   

\begin{entry}{%
\proto{port-has-set-port-position!?}{ port}{procedure}
\proto{set-port-position!}{ port pos}{procedure}

\domain{\var{Pos} must be a non-negative exact integer.}
   
The {\cf port-has-set-port-position?} procedure returns \schtrue{} if the port
supports the {\cf set-port-position!} operation, and \schfalse}
otherwise.
   
The {\cf set-port-position!} procedure sets the current byte position
of the port to \var{pos}.  If \var{port} is an output or combined
input and output port, this will first flush \var{pos}.  (See {\cf
  flush-output-port}, section~\ref{flush-output-port}.)
This procedure raises an exception with condition type {\cf\&contract}
if the port does not support the operation.
\end{entry}

\begin{entry}{%
\proto{close-port}{ port}{procedure}}
   
Closes the port, rendering the port incapable of delivering or
accepting data. If \var{port} is an output port, it is flushed before
being closed.  This has no effect if the port has already been closed.
A closed port is still a port. The unspecified value is returned.
\end{entry}

\begin{entry}{%
\proto{call-with-port}{ port proc}{procedure}}
   
\domain{\var{Proc} must be a procedure that accepts a single
  argument.}  Calls \var{proc} with \var{port} as an argument. If
\var{proc} returns, then the \var{port} is closed automatically and
the values returned by \var{proc} are returned. If \var{proc} does not
return, then the port will not be closed automatically, unless it is
possible to prove that the port will never again be used for a
{\cf lookahead}, {\cf get}, or {\cf put} operation.
\end{entry}

\section{Input ports}

An input port allows reading an infinite sequence of bytes puncuated
by end of file objects. An input port connected to a finite data
source ends in an infinite sequence of end of file objects.  All of
the procedures containing textual input accept a transcoder as an
optional argument. If no transcoder is supplied or the
\var{transcoder} argument is \schfalse{}, the input bytes are
interpreted as UTF-8 with a platform-specific end-of-line convention.

It is unspecified whether a character encoding consisting of several
bytes may have an end of file between the bytes.  If, for example,
{\cf get-char} raises an {\cf\&i/o-decoding} exception because the
character encoding at the port's position is incomplete up to the next
end of file, a subsequent call to {\cf get-char} may successfully
decode a character if bytes completing the encoding are available
after the end of file.

\begin{entry}{%
\proto{input-port?}{ obj}{procedure}}

Returns \schtrue{} if the argument is an input port (or combined input
and output port), and returns \schfalse{} otherwise.
\end{entry}

\begin{entry}{%
\proto{port-eof?}{ input-port}{procedure}}
   
Returns \schtrue{} if the {\cf lookahead-u8} procedure would return
the end-of-file object, and returns \schfalse{} otherwise.
\end{entry}

\begin{entry}{%
\proto{open-file-input-port}{ filename}{procedure}
\rproto{open-file-input-port}{ filename file-options}{procedure}
\rproto{open-file-input-port}{ filename file-options transcoder}{procedure}}
   
This returns an input port for the named file. The file-options object
defaults to {\cf (file-options)} if not present. It may determine
various aspects of the returned port; see section~\ref{fileoptionssection}.

The returned input port supports the {\cf port-position} and {\cf
  set-port-position!}  operations.

If \var{transcoder} is specified, it becomes the transcoder associated
with the returned port.
\end{entry}

\begin{entry}{%
\proto{open-bytes-input-port}{ bytes}{procedure}
\rproto{open-bytes-input-port}{ bytes transcoder}{procedure}}
   
This returns an input port whose bytes are drawn from a copy of the
bytes object \var{bytes}.

If \var{transcoder} is specified, it becomes the transcoder associated
with the returned port.
\end{entry}

\begin{entry}{%
\proto{open-string-input-port}{ string}{procedure}
\rproto{open-string-input-port}{ string transcoder}{procedure}}

This returns an input port whose bytes are drawn from an encoding of
\var{string}.  If \var{transcoder} is not specified, that encoding is
UTF-8.  If \var{transcoder} is specified, the encoding is according to
\var{transcoder}.  Also, \var{transcoder} becomes the transcoder
associated with the returned port.
\end{entry}

\begin{entry}{%
\proto{standard-input-port}{}{procedure}}
   
Returns an input port connected to standard input, possibly a fresh
one on each call.  If the returned port supports textual input, it has
an associated transcoder with its encoding, end-of-line style, and
error-handling mode {\cf raise}.
\end{entry}

\begin{entry}{%
\proto{get-u8}{ input-port}{procedure}}
   
This reads from input-port, blocking as necessary, until data is
available from \var{input-port} or an end of file is reached. If a
byte becomes available, this returns that byte as an octet, and
updates \var{input-port} to point just past that byte. If no input
byte is seen before an end of file is reached, then the end-of-file
object is returned.
\end{entry}

\begin{entry}{%
\proto{lookahead-u8}{ input-port}{procedure}}
   
This is the same as {\cf get-u8}, except it does not consume the
\var{byte} from the port.
\end{entry}

\begin{entry}{%
\proto{get-bytes-n}{ input-port k}{procedure}}
   
This reads from \var{input-port}, blocking as necessary, until \var{k}
bytes are available from \var{input-port} or an end of file is
reached. If \var{k} or more bytes are available before an apparent end
of the input, it returns a bytes object of size \var{k}. If fewer
bytes are available before an end of file, it returns a bytes object
containing those bytes. In either case, the input port is updated to
point just past the bytes read.  If an end of file is reached before
any bytes are available, this returns the end-of-file object.
\end{entry}

\begin{entry}{%
\proto{get-bytes-n!}{ input-port bytes start count}{procedure}}
   
\domain{\var{Count} must be an exact, non-negative integer, specifying
  the number of bytes to be read. \var{Bytes} must be a bytes object with at
  least $\var{start} + \var{count}$ elements.}
   
This reads from \var{input-port}, blocking as necessary, until
\var{count} are available from \var{input-port} or an end of file is
reached. If count or more bytes are available before an end of file,
they are written into \var{bytes} starting at index \var{start}, and
the procedure returns \var{count}. If fewer bytes are available before
the next end of file, it writes the available bytes into \var{bytes}
starting at index \var{start}, and returns the number of bytes actually
read. In either case, the input port is updated to point just past the
data read. If an end of file is reached before any bytes
are available, this returns the end-of-file object.
\end{entry}

\begin{entry}{%
\proto{get-bytes-some}{ input-port}{procedure}}
   
This reads from \var{input-port}, blocking as necessary, until data is
available from \var{input-port} or an end of file is reached.  If data
becomes available, this returns a freshly allocated bytes object of
non-zero size containing that data, and updates \var{input-port} to
point just past that data.  If no input bytes are seen before an end
of file is reached, then the end-of-file object is returned.
\end{entry}

\begin{entry}{%
\proto{get-bytes-all}{ input-port}{procedure}}
   
This attempts to read all data until the next end of file, blocking as
necessary. If one or more bytes are read, this returns a bytes object
containing all bytes up to the next end of file.  Otherwise {\cf
  get-bytes-all} returns the end-of-file object. Note that this
procedure may block indefinitely, waiting to see an apparent end of
file, even though some bytes are available.
\end{entry}

\begin{entry}{%
\proto{get-char}{ input-port}{procedure}
\rproto{get-char}{ input-port transcoder}}
   
This reads from \var{input-port}, blocking as necessary, until the
complete encoding for a character is available from input-port, or the
bytes that are available cannot be the prefix of any valid encoding,
or an end of file is reached.

If a complete character is available before the next end of file, {\cf
  get-char} returns that character, and updates the input port to
point past the bytes that encoded that character. If an end of file is
reached before any bytes are read, then {\cf get-char} returns the
end-of-file object.
\end{entry}

\begin{entry}{%
\proto{lookahead-char}{ input-port}{procedure}
\rproto{lookahead-char}{ input-port transcoder}{procedure}}
   
This is the same as {\cf get-char}, but does not consume any bytes
from the port.

\begin{note}
  With some of the standard transcoders
  described in this document, up to eight bytes of lookahead are
  required. Nonstandard transcoders may require even more lookahead.
\end{note}
\end{entry}

\begin{entry}{%
\proto{get-string-n}{ input-port k}{procedure}
\rproto{get-string-n}{ input-port k transcoder}{procedure}}
   
This reads from \var{input-port}, blocking as necessary, until the
encodings of \var{k} characters (including invalid encodings, if they
don't raise an exception) are available, or until an end of
file is reached.
   
If \var{k} or more characters are read before end of file, {\cf
  get-string-n} returns a string consisting of those \var{k}
characters. If fewer characters are available before an end of file,
but one or more characters can be read, it returns a string containing
those characters. In either case, the input port is updated to point
just past the data read. If no bytes can be read before an 
end of file, then the end-of-file object is returned.
\end{entry}

\begin{entry}{%
\proto{get-string-n!}{ input-port string start count}{procedure}
\rproto{get-string-n!}{ input-port string start count transcoder}{procedure}}

\domain{\var{Start} and \var{count} must be an exact, non-negative
  integer, specifying the number of characters to be read.
  \var{String} must be a string with at least $\var{start} +
  \var{count}$ characters.}

This reads from \var{input-port} in the same manner as {\cf
  get-string-n}.  If \var{count} or more characters are available
before an apparent end of the input, they are written into string
starting at index \var{start}, and \var{count} is returned. If fewer
characters are available before an apparent end of the input, but one
or more can be read, then those characters are written into string
starting at index \var{start}, and the number of characters actually read is
returned. If no characters can be read before an end of file,
then the end-of-file object is returned.
\end{entry}   

\begin{entry}{%
\proto{get-string-all}{ input-port}{procedure}
\rproto{get-string-all}{ input-port transcoder}{procedure}}
   
This reads from \var{input-port} until an apparent end of the input, decoding
characters in the same manner as {\cf get-string-n} and {\cf get-string-n!}.
   
If data is available before the end of file, a string
containing all the text decoded from that data is returned. If no data
precedes the end of file, the end-of-file object file object is
returned.
\end{entry}

\begin{entry}{%
\proto{get-line}{ input-port}{procedure}
\rproto{get-line}{ input-port transcoder}{procedure}}
   
This reads from \var{input-port} up to and including the next
end-of-line encoding or line separator character (U+2028) or apparent
end of the input, decoding characters in the same manner as {\cf
  get-string-n} and {\cf get-string-n!}.
   
If an end-of-line encoding or line separator is read, then a string
containing all of the text up to (but not including) the end-of-line
encoding is returned, and the port is updated to point just past the
end-of-line encoding or line separator. If an end of file is
encountered before any end-of-line encoding is read, but some bytes
have been read and decoded as characters, then a string containing
those characters is returned. If an end of file is encountered before
any bytes are read and decoded, then the end-of-file object is
returned.
\end{entry}

\section{Output ports}

An port is a sink to which bytes are written. These bytes may control
external devices, or may produce files and other objects that may
subsequently be opened for input.  The procedures in this section for
textual output accept a transcoder as an optional argument.  If no
transcoder is supplied, the character(s) output will be translated to
UTF-8 with a platform-specific end-of-line convention.

\begin{entry}{%
\proto{output-port?}{ obj}{procedure}}
   
This returns \schtrue{} if the argument is an output port (or a
combined input and output port), and returns \schfalse{} otherwise.
\end{entry}   

\begin{entry}{%
\proto{flush-output-port}{ output-port}{procedure}}
   
This flushes any output from the buffer of \var{output-port} to the
underlying data or device. The unspecified value is returned.
\end{entry}

\begin{entry}{%
\proto{output-port-buffer-mode}{ output-port}{procedure}}
   
This returns the symbol that represents the buffer-mode of
\var{output-port}.
\end{entry}

\begin{entry}{%
\proto{open-file-output-port}{ filename}{procedure}
\rproto{open-file-output-port}{ filename file-options}{procedure}
\rproto{open-file-output-port}{ filename file-options transcoder}{procedure}}

This returns an output port for the named file and the specified
options (which default to {\cf (file-options)}.) The \var{buffer-mode}
argument is optional; it defaults to {\cf block}.  If \var{transcoder}
is specified, it becomes the transcoder associated with the returned
port.
\end{entry}   

\begin{entry}{%
\proto{open-bytes-output-port}{ proc}{procedure}
\rproto{open-bytes-output-port}{ proc transcoder}{procedure}}

This creates an output port that accumulates a bytes object from the
output written to it.  If \var{transcoder} is specified, it becomes
the transcoder associated with the returned port.
\end{entry}

\begin{entry}{%
\proto{get-output-bytes}{ output-port}{procedure}}

\domain{\var{Output-port} must be a port that accumulates a bytes
  object from the output written to it, such as the ports created by
  {\cf open-bytes-output-port} and {\cf call-with-bytes-output-port}.}
This returns a copy of the bytes object that has been accumulated with
\var{output-port} so far.
\end{entry}

\begin{entry}{%
\proto{call-with-bytes-output-port}{ proc}{procedure}
\rproto{call-with-bytes-output-port}{ proc transcoder}{procedure}}

\domain{\var{Proc} is a procedure accepting one argument.} This
creates an output port that accumulates a bytes object from the output
written to it, and calls \var{proc} with that output port as an
argument. When \var{proc} returns for the first time, the port is
closed and the bytes object associated with the port is returned.  If
\var{transcoder} is specified, it becomes the transcoder associated
with the port.
\end{entry}

\begin{entry}{%
\proto{open-string-output-port}{ proc}{procedure}
\rproto{open-string-output-port}{ proc transcoder}{procedure}}

This is the same as {\cf open-bytes-output-port}.
\end{entry}

\begin{entry}{%
\proto{get-output-string}{ output-port}{procedure}
\rproto{get-output-string}{ output-port transcoder}{procedure}}

\domain{\var{Output-port} must be a port that accumulates a bytes
  object from the output written to it, such as the ports created by
  {\cf open-bytes-output-port} and {\cf call-with-bytes-output-port}.}
The decoding of the bytes object associated with that port is returned
as a freshly allocated string.  That decoding is according to
\var{transcoder} if it is specified, and UTF-8 if not.  In either
case, decoding errors are always handled analogously to the {\cf
  replace} error-handling mode: the first byte of each invalid
encoding is skipped and decoded as the U+FFFD replacement character.
\end{entry}

\begin{entry}{%
\proto{call-with-string-output-port}{ proc}{procedure}
\rproto{call-with-string-output-port}{ proc transcoder}{procedure}}
   
\domain{\var{Proc} is a procedure accepting one argument.} This
creates an output port that accumulates a bytes object from the output
written to it, and calls \var{proc} with that port as an argument.
When \var{proc} returns for the first time, the port is closed and the
decoding of the bytes object associated with that port is returned as
a freshly allocated string.  That decoding is according to
\var{transcoder} if it is specified, and UTF-8 if not.  In either
case, decoding errors are always handled analogously to the {\cf
  replace} error-handling mode: the first byte of each invalid
encoding is skipped and decoded as the U+FFFD replacement character.
If \var{transcoder} is specified, it also becomes the transcoder
associated with the port.
\end{entry}

\begin{entry}{%
\proto{standard-output-port}{}{procedure}
\proto{standard-error-port}{}{procedure}}
   
Returns a port connected to the standard output or standard error,
respectively. If the returned port supports textual input, it has an
associated transcoder with its encoding, end-of-line style, and
error-handling mode {\cf raise}.
\end{entry}

\begin{entry}{%
\proto{put-u8}{ output-port octet}{procedure}}

Writes \var{octet} to the output port. The unspecified value is returned.
\end{entry}

\begin{entry}{%
\proto{put-bytes}{ output-port bytes}{procedure}
\rproto{put-bytes}{ output-port bytes start}{procedure}
\rproto{put-bytes}{ output-port bytes start count}{procedure}}
   
\domain{\var{Start} and \var{count} must be non-negative exact
  integers that default to 0 and $\texttt{(bytes-length \var{bytes})}
  - \var{start}$, respectively. \var{Bytes} must have a size of at
  least $\var{start} + \var{count}$.}  This procedure writes the
\var{count} bytes of the bytes object \var{bytes}, starting at index
\var{start}, to the output port. The unspecified value is returned.
\end{entry}

\begin{entry}{%
\proto{put-char}{ output-port char}{procedure}
\rproto{put-char}{ output-port char transcoder}}
   
Writes an encoding of the character \var{char} to the port. The
unspecified value is returned.
\end{entry}

\begin{entry}{%
\proto{put-string}{ output-port string}{procedure}
\rproto{put-string}{ output-port string transcoder}{procedure}}
   
Writes an encoding of the string to the port. The unspecified value is
returned.
\end{entry}   

\begin{entry}{%
\proto{put-string-n}{ output-port string}{procedure}
\rproto{put-string-n}{ output-port string start}{procedure}
\rproto{put-string-n}{ output-port string start count}{procedure}
\rproto{put-string-n}{ output-port string start count transcoder}{procedure}}
   
\domain{\var{Start} and \var{count} must be non-negative exact
  integers.  var{String} must have a length of at least $\var{start} +
  \var{count}$.}  \var{Start} defaults to 0.  \var{Count} defaults to
$\texttt{(string-length \var{bytes})} - \var{start}$. Writes the
encodings of the \var{count} characters of \var{string}, starting at
index \var{start}, to the port. The unspecified value is returned.
\end{entry}

\section{Opening files for reading and writing}

\begin{entry}{%
\proto{open-file-input/output-port}{ filename}{procedure}
\rproto{open-file-input/output-port}{ filename file-options}{procedure}
\rproto{open-file-input/output-port}{ filename file-options buffer-mode}{procedure}
\rproto{open-file-input/output-port}{ filename file-options buffer-mode transcoder}{procedure}}
   
This returns a single port that is both an input port and an output
port, for the named file and options (which default to {\cf
  (file-options)}).  \var{Buffer-mode} optionally specifies the buffer
mode of the port; it defaults to {\cf block}.
   
The returned port supports the {\cf port-position} and {\cf
  set-port-position!}  operations. The same port position is used for
both input and output.

If \var{transcoder} is specified, it becomes the transcoder associated
with the returned port.
\end{entry}

\section{Ports from readers and writers}

\begin{entry}{%
\proto{open-reader-input-port}{ reader}{procedure}
\rproto{open-reader-input-port}{ reader transcoder}{procedure}}
   
This returns an input port connected to the reader \var{reader}.
If \var{transcoder} is specified, it becomes the transcoder associated
with the returned port.
\end{entry}   

\begin{entry}{%
\proto{open-writer-output-port}{ writer}{procedure}
\rproto{open-writer-output-port}{ writer buffer-mode}{procedure}
\rproto{open-writer-output-port}{ writer buffer-mode transcoder}{procedure}}
   
This returns an output port connected to the writer \var{writer}.
\var{Buffer-mode} optionally specifies the buffer mode of the port; it
defaults to {\cf block}.
If \var{transcoder} is specified, it becomes the transcoder associated
with the returned port.
\end{entry}

%%% Local Variables: 
%%% mode: latex
%%% TeX-master: "r6rs"
%%% End: 

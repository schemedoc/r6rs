\section{Port I/O}
\label{portsiosection}

The \defrsixlibrary{io ports} library defines an I/O layer for
conventional, imperative buffered input and output.
A \defining{port} represents a buffered access object
for a data sink or source or both simultaneously.
The library allows ports to be created from arbitrary data sources
and sinks.

The \rsixlibrary{io ports} library distinguishes between \textit{input
  ports\mainindex{input port}} and \textit{output
  ports\mainindex{output ports}}.  An input port is a source for data,
whereas an output port is a sink for data.  A port may be both an
input port and an output port; such a port typically provides
simultaneous read and write access to a file or other data.

The \rsixlibrary{io ports} library also distinguishes between
\textit{binary ports\mainindex{binary port}}, which are sources
or sinks for uninterpreted bytes, and
\textit{textual ports\mainindex{textual ports}}, which are sources
or sinks for characters and strings.

This section uses \var{input-port}, \var{output-port},
\var{binary-port}, \var{textual-port},
\var{binary-input-port}, \var{textual-input-port},
\var{binary-output-port}, \var{textual-output-port},
and \var{port} as
parameter names for arguments that must be input ports (or combined
input/output ports), output ports (or combined input/output ports),
binary ports, textual ports, binary input ports, textual input ports,
binary output ports, textual output ports, or any kind of port,
respectively.

\subsection{File names}
\label{filenamesection}

Some of the procedures described in this chapter accept a file name as an
argument. Valid values for such a file name include strings that name a file
using the native notation of filesystem paths on an implementation's
underlying operating system, and may include implementation-dependent
values as well.

A \var{filename} parameter name means that the
corresponding argument must be a file name.

\subsection{File options}
\label{fileoptionssection}

\mainindex{file options}
When opening a file, the various procedures in this library accept a
{\cf file-options} object that encapsulates flags to specify how
the file is to be opened. A {\cf file-options} object is an enum-set
(see chapter~\ref{enumerationschapter}) over the symbols constituting
valid file options.
A \var{file-options} parameter name means that the
corresponding argument must be a file-options object.

\begin{entry}{%
\proto{file-options}{ \hyper{file-options symbol} \dotsfoo}{\exprtype}}

\domain{Each \hyper{file-options symbol} must be a symbol.}
The {\cf file-options} syntax returns a file-options object that 
encapsulates the
specified options.

When supplied to an operation that opens a file for output, the
file-options object returned by {\cf (file-options)} specifies that the
file is created if it does not exist and an exception with condition type
{\cf\&i/o-file-already-exists} is raised if it does exist.
The following standard options can be included to modify the default behavior.

\begin{itemize}   
\item {\cf no-create}
      If the file does not already exist, it is not created;
      instead, an exception with condition type {\cf\&i/o-file-does-not-exist}
      is raised.
      If the file already exists, the exception with condition type
      {\cf\&i/o-file-already-exists} is not raised
      and the file is truncated to zero length.
\item {\cf no-fail}
      If the file already exists, the exception with condition type
      {\cf\&i/o-file-already-exists} is not raised,
      even if {\cf no-create} is not included,
      and the file is truncated to zero length.
\item {\cf no-truncate}
      If the file already exists and the exception with condition type
      {\cf\&i/o-file-already-exists} has been inhibited by inclusion of
      {\cf no-create} or {\cf no-fail}, the file is not truncated, but
      the port's current position is still set to the beginning of the
      file.
\end{itemize}

These options have no effect when a file is opened only for input.
Symbols
other than those listed above may be used as \hyper{file-options symbol}s;
they have implementation-specific meaning, if any.

\begin{note}
  Only the name of \hyper{file-options symbol} is significant.
\end{note}
\end{entry}   

\subsection{Buffer modes}

Each port has an associated buffer mode.  For an output port, the
buffer mode defines when an output operation flushes the buffer
associated with the output port.  For an input port, the buffer mode
defines how much data will be read to satisfy read operations.  The
possible buffer modes are the symbols {\cf none} for no buffering,
{\cf line} for flushing upon line endings and reading up to line
endings, or other implementation-dependent behavior,
and {\cf block} for arbitrary buffering.  This section uses
the parameter name \var{buffer-mode} for arguments that must be
buffer-mode symbols.

If two ports are connected to the same mutable source, both ports
are unbuffered, and reading a byte or character from that shared
source via one of the two ports would change the bytes or characters
seen via the other port, a lookahead operation on one port will
render the peeked byte or character inaccessible via the other port,
while a subsequent read operation on the peeked port will see the
peeked byte or character even though the port is otherwise unbuffered.

In other words, the semantics of buffering is defined in terms of side
effects on shared mutable sources, and a lookahead operation has the
same side effect on the shared source as a read operation.

\begin{entry}{%
\proto{buffer-mode}{ \hyper{buffer-mode symbol}}{\exprtype}}
   
\domain{\hyper{Buffer-mode symbol} must be a symbol whose name is one
  of {\cf none}, {\cf line}, and {\cf block}.} The result is the
corresponding symbol, and specifies the associated buffer mode.

\begin{note}
  Only the name of \hyper{buffer-mode symbol} is significant.
\end{note}
\end{entry}

\begin{entry}{%
\proto{buffer-mode?}{ obj}{procedure}}
   
Returns \schtrue{} if the argument is a valid buffer-mode symbol,
and returns \schfalse{} otherwise.
\end{entry}

\subsection{Transcoders}
\label{transcoderssection}

Several different Unicode encoding schemes describe standard ways to
encode characters and strings as byte sequences and to decode those
sequences~\cite{Unicode}.
Within this document, a \defining{codec} is an immutable Scheme
object that represents a Unicode or similar encoding scheme.

An \defining{end-of-line style} is a symbol that, if it is not {\cf
  none}, describes how a textual port transcodes representations of
line endings.

A \defining{transcoder} is an immutable Scheme object that combines
a codec with an end-of-line style and a method for handling
decoding errors.
Each transcoder represents some specific bidirectional (but not
necessarily lossless), possibly stateful translation between byte
sequences and Unicode characters and strings.
Every transcoder can operate in the input direction (bytes to characters)
or in the output direction (characters to bytes). 
A \var{transcoder} parameter name means that the corresponding
argument must be a transcoder.

A \defining{binary port} is a port that supports binary I/O, does not
have an associated transcoder and does not support textual I/O.  A
\defining{textual port} is a port that supports textual I/O, and does
not support binary I/O.  A textual port may or may not have an
associated transcoder.

\begin{entry}{%
\proto{latin-1-codec}{}{procedure}
\proto{utf-8-codec}{}{procedure}
\proto{utf-16-codec}{}{procedure}}

These are predefined codecs for the ISO 8859-1, UTF-8,
and UTF-16 encoding schemes \cite{Unicode}.

A call to any of these procedures returns a value that is equal in the
sense of {\cf eqv?} to the result of any other call to the same
procedure.
\end{entry}

\begin{entry}{%
\proto{eol-style}{ \hyper{eol-style symbol}}{\exprtype}}

\domain{\hyper{Eol-style symbol} should be a symbol whose name is one
  of {\cf lf}, {\cf cr}, {\cf crlf}, {\cf nel},
  {\cf crnel}, {\cf ls}, and {\cf none}.}  The form evaluates to the
corresponding symbol.  If the name of \var{eol-style symbol} is not
one of these symbols, the effect and result are
implementation-dependent; in particular, the result may be an
eol-style symbol acceptable as an \var{eol-style} argument to {\cf
  make-transcoder}.  Otherwise, an exception is raised.

All eol-style symbols except {\cf none} describe a specific
line-ending encoding:

\noindent\begin{tabular}{ll}
{\cf lf} & \meta{linefeed}\\
{\cf cr} & \meta{carriage return}\\
{\cf crlf} & \meta{carriage return} \meta{linefeed}\\
{\cf nel} & \meta{next line}\\
{\cf crnel} & \meta{carriage return} \meta{next line}\\
{\cf ls} & \meta{line separator}
\end{tabular}

For a textual port with a transcoder, and whose transcoder has an eol-style symbol {\cf
  none}, no conversion occurs.  For a textual input port, any
eol-style symbol other than {\cf none} means that all of the above
line-ending encodings are recognized and are translated into a single
linefeed.  For a textual output port, {\cf none} and {\cf lf} are
equivalent.  Linefeed characters are encoded according to the
specified eol-style symbol, and all other characters that participate
in possible line endings are encoded as is.

\begin{note}
  Only the name of \hyper{eol-style symbol} is significant.
\end{note}
\end{entry}

\begin{entry}{%
\proto{native-eol-style}{}{procedure}}

Returns the default end-of-line style of the underlying platform, e.g.,
{\cf lf} on Unix and {\cf crlf} on Windows.
\end{entry}

\begin{entry}{%
\ctproto{i/o-decoding}
\proto{make-i/o-decoding-error}{ pobj}{procedure}
\proto{i/o-decoding-error?}{ obj}{procedure}}

This condition type could be defined by
%
\begin{scheme}
(define-condition-type \&i/o-decoding \&i/o-port
  make-i/o-decoding-error i/o-decoding-error?)%
\end{scheme}

An exception with this type is raised when one of the operations for
textual input from a port encounters a sequence of bytes that cannot
be translated into a character or string by the input direction of the
port's transcoder.

When such an exception is raised, the port's position is past
the invalid encoding.
\end{entry}

\begin{entry}{% 
\ctproto{i/o-encoding}
\proto{make-i/o-encoding-error}{ pobj cobj}{procedure}
\proto{i/o-encoding-error?}{ obj}{procedure}
\proto{i/o-encoding-error-char}{ condition}{procedure}}

This condition type could be defined by
%
\begin{scheme}
(define-condition-type \&i/o-encoding \&i/o-port
  make-i/o-encoding-error i/o-encoding-error?
  (cobj i/o-encoding-error-char))%
\end{scheme}

An exception with this type is raised when one of the operations for
textual output to a port encounters a character that cannot be
translated into bytes by the output direction of the port's transcoder.
\var{Cobj} should be the character that could not be encoded.
\end{entry}

\begin{entry}{%
\proto{error-handling-mode}{ \hyper{error-handling-mode symbol}}{\exprtype}}

\domain{\hyper{Error-handling-mode symbol} should be a symbol whose
  name is one of {\cf ignore}, {\cf raise}, and {\cf replace}.} The
form evaluates to the corresponding symbol.  If
\hyper{error-handling-mode symbol} is not one of these identifiers,
effect and result are implementation-dependent: The result may be an
error-handling-mode symbol acceptable as a \var{handling-mode}
argument to {\cf make-transcoder}.  If it is not acceptable as a
\var{handling-mode} argument to {\cf make-transcoder}, an exception is
raised.

\begin{note}
  Only the name of \hyper{error-handling-style symbol} is significant.
\end{note}

The error-handling mode of a transcoder specifies the behavior
of textual I/O operations in the presence of encoding or decoding
errors.

If a textual input operation encounters an invalid or incomplete
character encoding, and the error-handling mode is {\cf ignore},
an appropriate number of bytes of the
invalid encoding are ignored and decoding continues with the
following bytes.
If the error-handling mode is {\cf replace}, the replacement
character U+FFFD is injected into the data stream, an appropriate
number of bytes are ignored, and decoding
continues with the following bytes.
If the error-handling mode is {\cf raise}, an 
exception with condition type {\cf\&i/o-decoding} is raised.

If a textual output operation encounters a character it cannot encode,
and the error-handling mode is {\cf ignore}, the character is
ignored and encoding continues with the next character.
If the error-handling mode is {\cf replace}, a codec-specific
replacement character is emitted by the transcoder, and encoding
continues with the next character.
The replacement character is U+FFFD for transcoders whose codec
is one of the Unicode encodings, but is the {\cf ?}
character for the Latin-1 encoding.
If the error-handling mode is {\cf raise}, an
exception with condition type {\cf\&i/o-encoding} is raised.
\end{entry}

\begin{entry}{%
\proto{make-transcoder}{ codec}{procedure}
\rproto{make-transcoder}{ codec eol-style}{procedure}
\rproto{make-transcoder}{ codec eol-style handling-mode}{procedure}}

\domain{\var{Codec} must be a codec; \var{eol-style}, if present, an
  eol-style symbol; and \var{handling-mode}, if present, an
  error-handling-mode symbol.}  \var{Eol-style} may be omitted, in
which case it defaults to the native end-of-line style of the
underlying platform.  \var{Handling-mode} may be omitted, in which
case it defaults to {\cf replace}.  The result is a transcoder with the
behavior specified by its arguments.
\end{entry}

\begin{entry}{
\proto{native-transcoder}{}{procedure}}

Returns an implementation-dependent transcoder that represents a
possibly locale-dependent ``native'' transcoding.
\end{entry}

\begin{entry}{%
\proto{transcoder-codec}{ transcoder}{procedure}
\proto{transcoder-eol-style}{ transcoder}{procedure}
\proto{transcoder-error-handling-mode}{ transcoder}{procedure}}

These are accessors for transcoder objects; when applied to a
transcoder returned by {\cf make-transcoder}, they return the
\var{codec}, \var{eol-style}, and \var{handling-mode} arguments,
respectively.
\end{entry}

\begin{entry}{%
\proto{bytevector->string}{ bytevector transcoder}{procedure}}

Returns the string that results from transcoding the
\var{bytevector} according to the input direction of the transcoder.
\end{entry}

\begin{entry}{%
\proto{string->bytevector}{ string transcoder}{procedure}}

Returns the bytevector that results from transcoding the
\var{string} according to the output direction of the transcoder.
\end{entry}

\subsection{End-of-file object}
\label{eofsection}

The end-of-file object is returned by various I/O procedures when they
reach end of file.\index{end-of-file object}

\begin{entry}{%
\proto{eof-object}{}{procedure}}

Returns the end-of-file object.
\begin{scheme}
(eqv? (eof-object) (eof-object)) \lev  \schtrue
(eq? (eof-object) (eof-object)) \lev  \schtrue%
\end{scheme}
\end{entry}

\begin{note}
  The end-of-file object  is not a datum value, and thus has no external
  representation.
\end{note}

\begin{entry}{%
\proto{eof-object?}{ obj}{procedure}}
  
Returns \schtrue{} if \var{obj} is the end-of-file object, \schfalse{} otherwise.
\end{entry}

\subsection{Input and output ports}

The operations described in this section are common to input and
output ports, both binary and textual.  A port may also have an
associated \defining{position} that specifies a particular place
within its data sink or source, and may also provide operations for
inspecting and setting that place.

\begin{entry}{%
\proto{port?}{ obj}{procedure}}
   
Returns \schtrue{} if the argument is a port, and returns \schfalse{}
otherwise.
\end{entry}

\begin{entry}{%
\proto{port-transcoder}{ port}{procedure}}

Returns the transcoder associated with \var{port} if \var{port} is
textual and has an associated transcoder, and returns \schfalse{} if
\var{port} is binary or does not have an associated transcoder.
\end{entry}

\begin{entry}{%
\proto{textual-port?}{ port}{procedure}
\proto{binary-port?}{ port}{procedure}}

The {\cf textual-port?} procedure returns \schtrue{} if \var{port} is
textual, and returns \schfalse{} otherwise.
The {\cf binary-port?} procedure returns \schtrue{} if \var{port} is
binary, and returns \schfalse{} otherwise.
\end{entry}

\begin{entry}{%
\proto{transcoded-port}{ binary-port transcoder}{procedure}}

The {\cf transcoded-port} procedure
returns a new textual port with the specified \var{transcoder}.
Otherwise the new textual port's state is largely the same as
that of \var{binary-port}.
If \var{binary-port} is an input port, the new textual
port will be an input port and
will transcode the bytes that have not yet been read from
\var{binary-port}.
If \var{binary-port} is an output port, the new textual
port will be an output port and
will transcode output characters into bytes that are
written to the byte sink represented by \var{binary-port}.

As a side effect, however, {\cf transcoded-port}
closes \var{binary-port} in
a special way that allows the new textual port to continue to
use the byte source or sink represented by \var{binary-port},
even though \var{binary-port} itself is closed and cannot
be used by the input and output operations described in this
chapter.
\end{entry}

\begin{entry}{%
\proto{port-has-port-position?}{ port}{procedure}
\proto{port-position}{ port}{procedure}}

The {\cf port-has-port-position?} procedure returns \schtrue{} if the
port supports the {\cf port-position} operation, and \schfalse{}
otherwise.

For a binary port, the {\cf port-position} procedure returns the index
of the position at which the next byte would be read from or written
to the port as an exact non-negative integer object.  For a textual
port, {\cf port-position} returns a value of some implementation-dependent
type representing the port's position; this value may be useful only as
the \var{pos} argument to {\cf set-port-position!}, if the latter is
supported on the port (see below).

If the port does not support the operation, {\cf port-position} raises
an exception with condition type {\cf\&assertion}.

\begin{note}
  For a textual port, the port position may or may not be an integer
  object.  If it is an integer object, the integer object does not
  necessarily correspond to a byte or character position.
\end{note}
\end{entry}   

\begin{entry}{%
\proto{port-has-set-port-position!?}{ port}{procedure}
\proto{set-port-position!}{ port pos}{procedure}}

\domain{If \var{port} is a binary port, \var{pos} should be a
  non-negative exact integer object.  If \var{port} is a textual port,
  \var{pos} should be the return value of a call to {\cf
    port-position} on \var{port}.}
   
The {\cf port-has-set-port-position!?} procedure returns \schtrue{} if the port
supports the {\cf set-port-position!} operation, and \schfalse{}
otherwise.
   
The {\cf set-port-position!} procedure raises an
exception with condition type {\cf\&assertion}
if the port does not support the operation, 
and an exception with condition type {\cf\&i/o-invalid-position} if
\var{pos} is not in the range of valid positions of \var{port}.
Otherwise, it sets the current position
of the port to \var{pos}.  If \var{port} is an output
port, {\cf set-port-position!} first flushes \var{port}.  (See {\cf
  flush-output-port}, section~\ref{flush-output-port}.)

If \var{port} is a binary output port and the current position is set
beyond the current end of the data in the underlying data sink, the object is
not extended until new data is written at that position.
The contents of any intervening positions are unspecified.
Binary ports created by {\cf open-file-output-port} and
{\cf open-file-input/output-port} can always be extended in this manner
within the limits of the underlying operating system.
In other cases, attempts to set the port beyond the current end of data
in the underlying object may result in an exception with condition
type {\cf\&i/o-invalid-position}.
\end{entry}

\begin{entry}{%
\proto{close-port}{ port}{procedure}}
   
Closes the port, rendering the port incapable of delivering or
accepting data. If \var{port} is an output port, it is flushed before
being closed.  This has no effect if the port has already been closed.
A closed port is still a port.  The {\cf close-port} procedure returns
\unspecifiedreturn.
\end{entry}

\begin{entry}{%
\proto{call-with-port}{ port proc}{procedure}}
   
\domain{\var{Proc} must accept one argument.}
The {\cf call-with-port} procedure
calls \var{proc} with \var{port} as an argument. If
\var{proc} returns, \var{port} is closed automatically and
the values returned by \var{proc} are returned. If \var{proc} does not
return, \var{port} is not closed automatically, except perhaps when it is
possible to prove that \var{port} will never again be used for an
input or output operation.
\end{entry}

\subsection{Input ports}

An input port allows the reading of an infinite sequence of bytes
or characters punctuated
by end-of-file objects. An input port connected to a finite data
source ends in an infinite sequence of end-of-file objects.

It is unspecified whether a character encoding consisting of several
bytes may have an end of file between the bytes.  If, for example,
{\cf get-char} raises an {\cf\&i/o-decoding} exception because the
character encoding at the port's position is incomplete up to the next
end of file, a subsequent call to {\cf get-char} may successfully
decode a character if bytes completing the encoding are available
after the end of file.

\begin{entry}{%
\proto{input-port?}{ obj}{procedure}}

Returns \schtrue{} if the argument is an input port (or a combined input
and output port), and returns \schfalse{} otherwise.
\end{entry}

\begin{entry}{%
\proto{port-eof?}{ input-port}{procedure}}
   
Returns \schtrue{}
if the {\cf lookahead-u8} procedure (if \var{input-port} is a binary port)
or the {\cf lookahead-char} procedure (if \var{input-port} is a textual port)
would return
the end-of-file object, and \schfalse{} otherwise.
The operation may block indefinitely if no data is available
but the port cannot be determined to be at end of file.
\end{entry}

\begin{entry}{%
\proto{open-file-input-port}{ filename}{procedure}
\rproto{open-file-input-port}{ filename file-options}{procedure}
\pproto{(open-file-input-port \var{filename}}{procedure}{\tt\obeyspaces%
\hspace*{2em}\var{file-options} \var{buffer-mode})}\\
\pproto{(open-file-input-port \var{filename}}{procedure}{\tt\obeyspaces%
\hspace*{2em}\var{file-options} \var{buffer-mode} \var{maybe-transcoder})}}
   
\domain{\var{Maybe-transcoder} must be either a transcoder or \schfalse.}

The {\cf open-file-input-port} procedure returns an
input port for the named file. The \var{file-options} and
\var{maybe-transcoder} arguments are optional.

The \var{file-options} argument, which may determine
various aspects of the returned port (see section~\ref{fileoptionssection}),
defaults to the value of {\cf (file-options)}.

The \var{buffer-mode} argument, if supplied,
must be one of the symbols that name a buffer mode.
The \var{buffer-mode} argument defaults to {\cf block}.

If \var{maybe-transcoder} is a transcoder, it becomes the transcoder associated
with the returned port.

If \var{maybe-transcoder} is \schfalse{} or absent,
the port will be a binary port and will support the
{\cf port-position} and {\cf set-port-position!}  operations.
Otherwise the port will be a textual port, and whether it supports
the {\cf port-position} and {\cf set-port-position!} operations
is implementation-dependent (and possibly transcoder-dependent).
\end{entry}

\begin{entry}{%
\proto{open-bytevector-input-port}{ bytevector}{procedure}
\pproto{(open-bytevector-input-port \var{bytevector}}{procedure}{\tt\obeyspaces%
\hspace*{2em}\var{maybe-transcoder})}}

\domain{\var{Maybe-transcoder} must be either a transcoder or \schfalse.}
   
The {\cf open-bytevector-input-port} procedure returns an input port whose bytes are drawn from 
\var{bytevector}.
If \var{transcoder} is specified, it becomes the transcoder associated
with the returned port.

If \var{maybe-transcoder} is \schfalse{} or absent,
the port will be a binary port and will support the
{\cf port-position} and {\cf set-port-position!}  operations.
Otherwise the port will be a textual port, and whether it supports
the {\cf port-position} and {\cf set-port-position!} operations
will be implementation-dependent (and possibly transcoder-dependent).

If \var{bytevector} is modified after {\cf open-\hp{}bytevector-\hp{}input-\hp{}port}
has been called, the effect on the returned
port is unspecified.
\end{entry}

\begin{entry}{%
\proto{open-string-input-port}{ string}{procedure}}

Returns a textual input port whose characters are drawn from
\var{string}.  The port may or may not have an associated transcoder;
if it does, the transcoder is implementation-dependent.
The port should support the
{\cf port-position} and {\cf set-port-position!} operations.

If \var{string} is modified after {\cf open-string-input-port}
has been called, the effect on the returned port is unspecified.
\end{entry}

\begin{entry}{%
\proto{standard-input-port}{}{procedure}}
   
Returns a fresh binary input port connected to standard input.
Whether the port supports the {\cf port-position} and {\cf
  set-port-position!} operations is implementation-dependent.
\end{entry}

\begin{entry}{%
\proto{current-input-port}{}{procedure}}
 
This returns a default textual port for input.  Normally, this default port
is associated with standard input, but can be dynamically re-assigned
using the {\cf with-input-from-file} procedure from the
\rsixlibrary{io simple} library (see section~\ref{with-input-from-file}).
The port may or may not have an associated transcoder; if it does, the
transcoder is implementation-dependent.
\end{entry}

\begin{entry}{%
\pproto{(make-custom-binary-input-port \var{id} \var{read!}}{procedure}}
\mainschindex{make-custom-binary-input-port}{\tt\obeyspaces\\
  \var{get-position} \var{set-position!} \var{close})}

Returns a newly created binary input port whose byte source is
an arbitrary algorithm represented by the \var{read!} procedure.
\var{Id} must be a string naming the new port,
provided for informational purposes only.
\var{Read!} must be a procedure and should behave as specified
below; it will be called by operations that perform binary input.

Each of the remaining arguments may be \schfalse{}; if any of
those arguments is not \schfalse{}, it must be a procedure and
should behave as specified below.
   
\begin{itemize}
\item {\cf (\var{read!} \var{bytevector} \var{start} \var{count})}
       
  \domain{\var{Start} will be a non-negative exact integer object,
  \var{count} will be a positive exact integer object,}
  and \var{bytevector} will be a bytevector whose length is at least
  $\var{start} + \var{count}$.
  The \var{read!} procedure should obtain up to \var{count} bytes
  from the byte source, and should write those bytes
  into \var{bytevector} starting at index \var{start}.
  The \var{read!} procedure should return an exact integer object.  This
  integer object should represent the number of bytes that it has read.
  To indicate an end of file, the \var{read!}
  procedure should write no bytes and return 0.

\item {\cf (\var{get-position})}
       
  The \var{get-position} procedure (if supplied) should return an exact
  integer object representing the current position of
  the input port.  If not supplied, the custom port will not support
  the {\cf port-position} operation.
  
\item {\cf (\var{set-position!} \var{pos})}
       
  \domain{\var{Pos} will be a non-negative exact integer object.}
  The \var{set-position!} procedure (if supplied) should set the
  position of the input port to \var{pos}.  If not supplied, the custom
  port will not support the {\cf set-port-position!} operation.
       
\item {\cf (\var{close})}
       
  The \var{close} procedure (if supplied) should perform any actions
  that are necessary when the input port is closed.
\end{itemize}

\implresp The implementation must check the return
values of \var{read!} and \var{get-position} only when it actually calls
them as part of an I/O operation requested by the program.  The
implementation is not required to check that these procedures
otherwise behave as described.  If they do not, however, the behavior
of the resulting port is unspecified.
\end{entry}

\begin{entry}{%
\pproto{(make-custom-textual-input-port \var{id} \var{read!}}{procedure}}
\mainschindex{make-custom-textual-input-port}{\tt\obeyspaces\\
  \var{get-position} \var{set-position!} \var{close})}

Returns a newly created textual input port whose character source is
an arbitrary algorithm represented by the \var{read!} procedure.
\var{Id} must be a string naming the new port,
provided for informational purposes only.
\var{Read!} must be a procedure and should behave as specified
below; it will be called by operations that perform textual input.

Each of the remaining arguments may be \schfalse{}; if any of
those arguments is not \schfalse{}, it must be a procedure and
should behave as specified below.
   
\begin{itemize}
\item {\cf (\var{read!} \var{string} \var{start} \var{count})}
       
  \domain{\var{Start} will be a non-negative exact integer object,
  \var{count} will be a positive exact integer object,}
  and \var{string} will be a string whose length is at least
  $\var{start} + \var{count}$.
  The \var{read!} procedure should obtain up to \var{count} characters
  from the character source, and should write those characters
  into \var{string} starting at index \var{start}.
  The \var{read!} procedure should return an exact integer object
  representing the number of characters that it has written.
  To indicate an end of file, the \var{read!}
  procedure should write no bytes and return 0.

\item {\cf (\var{get-position})}
       
  The \var{get-position} procedure (if supplied) should return a single
  value.  The return value should represent the current position of
  the input port.  If not supplied, the custom port will not support
  the {\cf port-position} operation.
  
\item {\cf (\var{set-position!} \var{pos})}
       
  The \var{set-position!} procedure (if supplied) should set the
  position of the input port to \var{pos} if \var{pos} is the return
  value of a call to \var{get-position}.  If not supplied, the custom
  port will not support the {\cf set-port-position!} operation.
       
\item {\cf (\var{close})}
       
  The \var{close} procedure (if supplied) should perform any actions
  that are necessary when the input port is closed.
\end{itemize}

The port may or may not have an an associated transcoder; if it does,
the transcoder is implementation-dependent.

\implresp The implementation must check the return
values of \var{read!} and \var{get-position} only when it actually calls
them as part of an I/O operation requested by the program.  The
implementation is not required to check that these procedures
otherwise behave as described.  If they do not, however, the behavior
of the resulting port is unspecified.

\begin{note}
  Even when the \var{get-position} procedure is supplied, the
  {\cf port-position} procedure cannot generally return a precise
  value for a custom textual input port if data has been
  read from the port.
\end{note}

\end{entry}

\subsection{Binary input}\unsection

\begin{entry}{%
\proto{get-u8}{ binary-input-port}{procedure}}
   
Reads from \var{binary-input-port}, blocking as necessary, until a
byte is available from \var{binary-input-port} or until an end of file is reached.
If a byte becomes available, {\cf get-u8} returns the byte as an octet and
updates \var{binary-input-port} to point just past that byte. If no input
byte is seen before an end of file is reached, the end-of-file
object is returned.
\end{entry}

\begin{entry}{%
\proto{lookahead-u8}{ binary-input-port}{procedure}}
   
The {\cf lookahead-u8} procedure is like {\cf get-u8}, but it does not 
update \var{binary-input-port} to point past the byte.
\end{entry}

\begin{entry}{%
\proto{get-bytevector-n}{ binary-input-port count}{procedure}}
   
\domain{\var{Count} must be an exact, non-negative integer object representing
  the number of bytes to be read.}
The {\cf get-bytevector-n} procedure reads
from \var{binary-input-port}, blocking as necessary, until \var{count}
bytes are available from \var{binary-input-port} or until an end of file is
reached. If \var{count} bytes are available before an end
of file, {\cf get-bytevector-n} returns a bytevector of size \var{count}.
If fewer bytes are available before an end of file, {\cf get-bytevector-n}
returns a bytevector
containing those bytes. In either case, the input port is updated to
point just past the bytes read.  If an end of file is reached before
any bytes are available, {\cf get-bytevector-n} returns the end-of-file object.
\end{entry}

\begin{entry}{%
\pproto{(get-bytevector-n! \var{binary-input-port}}{procedure}}
\mainschindex{get-bytevector-n!}{\tt\obeyspaces\\
    \var{bytevector} \var{start} \var{count})}

\domain{\var{Start} and \var{count} must be exact, non-negative
  integer objects, with \var{count} representing
  the number of bytes to be read. \var{bytevector} must be a bytevector
  with at
  least $\var{start} + \var{count}$ elements.}
   
The {\cf get-bytevector-n!} procedure reads from \var{binary-input-port},
blocking as necessary, until
\var{count} bytes are available from \var{binary-input-port} or until
an end of file is
reached. If \var{count} bytes are available before an end of file,
they are written into \var{bytevector} starting at index \var{start}, and
the result is \var{count}. If fewer bytes are available before
the next end of file, the available bytes are written into \var{bytevector}
starting at index \var{start}, and the result is a number object
representing the number of bytes actually
read. In either case, the input port is updated to point just past the
bytes read. If an end of file is reached before any bytes
are available, {\cf get-bytevector-n!} returns the end-of-file object.
\end{entry}

\begin{entry}{%
\proto{get-bytevector-some}{ binary-input-port}{procedure}}
   
Reads from \var{binary-input-port}, blocking as necessary, until bytes are
available from \var{binary-input-port} or until an end of file is reached.
If bytes become available,
{\cf get-bytevector-some} returns a freshly allocated
bytevector containing the initial available bytes (at least one),
and it updates \var{binary-input-port} to point just past these bytes.
If no input bytes are seen before an end
of file is reached, the end-of-file object is returned.
\end{entry}

\begin{entry}{%
\proto{get-bytevector-all}{ binary-input-port}{procedure}}
   
Attempts to read all bytes until the next end of file, blocking as
necessary. If one or more bytes are read, {\cf get-bytevector-all} returns
a bytevector
containing all bytes up to the next end of file and updates
\var{binary-input-port} to point just past these bytes.  Otherwise, {\cf
  get-bytevector-all} returns the end-of-file object.
The operation may block indefinitely waiting to see if more bytes
will become available, even if some bytes are already available.
\end{entry}

\subsection{Textual input}\unsection

\begin{entry}{%
\proto{get-char}{ textual-input-port}{procedure}}
   
Reads from \var{textual-input-port}, blocking as necessary, until a
complete character is available from \var{textual-input-port},
or until an end of file is reached.

If a complete character is available before the next end of file, {\cf
  get-char} returns that character and updates the input port to
point past the character. If an end of file is
reached before any character is read, {\cf get-char} returns the
end-of-file object.
\end{entry}

\begin{entry}{%
\proto{lookahead-char}{ textual-input-port}{procedure}}
  
The {\cf lookahead-char} procedure is like {\cf get-char}, but it does not 
update \var{textual-input-port} to point past the 
character.

\begin{note}
  With some of the standard transcoders
  described in this document, up to four bytes of lookahead are
  needed. Nonstandard transcoders may need even more lookahead.
\end{note}
\end{entry}

\begin{entry}{%
\proto{get-string-n}{ textual-input-port count}{procedure}}
   
\domain{\var{Count} must be an exact, non-negative integer object, representing
  the number of characters to be read.}

The {\cf get-string-n} procedure reads
from \var{textual-input-port}, blocking as necessary, until 
\var{count} characters are available, or until an end of
file is reached.
   
If \var{count} characters are available before end of file, {\cf
  get-string-n} returns a string consisting of those \var{count}
characters. If fewer characters are available before an end of file,
but one or more characters can be read,
{\cf get-string-n} returns a string containing
those characters. In either case, the input port is updated to point
just past the characters read. If no characters can be read before an 
end of file, the end-of-file object is returned.
\end{entry}

\begin{entry}{%
\proto{get-string-n!}{ textual-input-port string start count}{procedure}}

\domain{\var{Start} and \var{count} must be exact, non-negative
  integer objects, with \var{count} representing the number of characters to be read.
  \var{String} must be a string with at least $\var{start} +
  \var{count}$ characters.}

The {\cf get-string-n!} procedure reads from \var{textual-input-port} in the same manner as {\cf
  get-string-n}.  If \var{count} characters are available
before an end of file, they are written into \var{string}
starting at index \var{start}, and \var{count} is returned. If fewer
characters are available before an end of file, but one
or more can be read, those characters are written into \var{string}
starting at index \var{start} and the number of characters actually read is
returned as an exact integer object. If no characters can be read before an end of file,
the end-of-file object is returned.
\end{entry}   

\begin{entry}{%
\proto{get-string-all}{ textual-input-port}{procedure}}
   
Reads from \var{textual-input-port} until an end of file, decoding
characters in the same manner as {\cf get-string-n} and {\cf get-string-n!}.
   
If characters are available before the end of file, a string
containing all the characters decoded from that data are returned. If no character
precedes the end of file, the end-of-file object is
returned.
\end{entry}

\begin{entry}{%
\proto{get-line}{ textual-input-port}{procedure}}
   
Reads from \var{textual-input-port} up to and including the linefeed
character or end of file, decoding characters in the same manner as
{\cf get-string-n} and {\cf get-string-n!}.
   
If a linefeed character is read, a string
containing all of the text up to (but not including) the linefeed
character is returned, and the port is updated to point just past the
linefeed character. If an end of file is
encountered before any linefeed character is read, but some characters
have been read and decoded as characters, a string containing
those characters is returned. If an end of file is encountered before
any characters are read, the end-of-file object is
returned.

\begin{note}
  The end-of-line style, if not {\cf none}, will cause all line
  endings to be read as linefeed characters.  See
  section~\ref{transcoderssection}.
\end{note}
\end{entry}

\begin{entry}{%
\proto{get-datum}{ textual-input-port}{procedure}}
 
Reads an external representation from \var{textual-input-port} and returns the
datum it represents.  The {\cf get-datum} procedure returns the next
datum that can be parsed from the given \var{textual-input-port}, updating
\var{textual-input-port} to point exactly past the end of the external
representation of the object.

Any \meta{interlexeme space}
(see report section~\extref{report:lexicalsyntaxsection}{Lexical syntax}) in
the input is first skipped.  If an end of file occurs after the
\meta{interlexeme space}, the end-of-file object (see
section~\ref{eofsection}) is returned.

If a character inconsistent with an external representation is
encountered in the input, an exception with condition types
{\cf\&lexical} and {\cf\&i/o-read} is raised.
Also, if the end of file is encountered
after the beginning of an external representation, but the external
representation is incomplete and therefore cannot be parsed, an exception
with condition types {\cf\&lexical} and {\cf\&i/o-read} is raised.
\end{entry}

\subsection{Output ports}

An output port is a sink to which bytes or characters are written.
The written data may control
external devices or may produce files and other objects that may
subsequently be opened for input.

\begin{entry}{%
\proto{output-port?}{ obj}{procedure}}
   
Returns \schtrue{} if the argument is an output port (or a
combined input and output port), \schfalse{} otherwise.
\end{entry}   

\begin{entry}{%
\proto{flush-output-port}{ output-port}{procedure}}
   
Flushes any buffered output from the buffer of \var{output-port} to the
underlying file, device, or object. The {\cf flush-output-port}
procedure returns \unspecifiedreturn.
\end{entry}

\begin{entry}{%
\proto{output-port-buffer-mode}{ output-port}{procedure}}
   
Returns the symbol that represents the buffer mode of
\var{output-port}.
\end{entry}

\begin{entry}{%
\proto{open-file-output-port}{ filename}{procedure}
\rproto{open-file-output-port}{ filename file-options}{procedure}
\pproto{(open-file-output-port \var{filename}}{procedure}{\tt\obeyspaces%
\hspace*{2em}\var{file-options} \var{buffer-mode})}\\
\pproto{(open-file-output-port \var{filename}}{procedure}{\tt\obeyspaces%
\hspace*{2em}\var{file-options} \var{buffer-mode} \var{maybe-transcoder})}}

\domain{\var{Maybe-transcoder} must be either a transcoder or \schfalse.}

The {\cf open-file-output-port} procedure returns an output port for the named file.

The \var{file-options} argument, which may determine
various aspects of the returned port (see section~\ref{fileoptionssection}),
defaults to the value of {\cf (file-options)}.

The \var{buffer-mode} argument, if supplied,
must be one of the symbols that name a buffer mode.
The \var{buffer-mode} argument defaults to {\cf block}.

If \var{maybe-transcoder} is a transcoder, it becomes the transcoder
associated with the port.

If \var{maybe-transcoder} is \schfalse{} or absent,
the port will be a binary port and will support the
{\cf port-position} and {\cf set-port-position!}  operations.
Otherwise the port will be a textual port, and whether it supports
the {\cf port-position} and {\cf set-port-position!} operations
is implementation-dependent (and possibly transcoder-dependent).
\end{entry}   

\begin{entry}{%
\proto{open-bytevector-output-port}{}{procedure}
\rproto{open-bytevector-output-port}{ maybe-transcoder}{procedure}}

\domain{\var{Maybe-transcoder} must be either a transcoder or \schfalse.}

The {\cf open-bytevector-output-port} procedure returns 
two values: an output port and an extraction procedure.
The output port accumulates the bytes written to it for
later extraction by the procedure.

If \var{maybe-transcoder} is a transcoder, it becomes
the transcoder associated with the port.
If \var{maybe-transcoder} is \schfalse{} or absent,
the port will be a binary port and will support the
{\cf port-position} and {\cf set-port-position!}  operations.
Otherwise the port will be a textual port, and whether it supports
the {\cf port-position} and {\cf set-port-position!} operations
is implementation-dependent (and possibly transcoder-dependent).

The extraction procedure takes no arguments.
When called, it returns a
bytevector consisting of all the port's accumulated bytes (regardless
of the port's current position), removes
the accumulated bytes from the port, and resets the port's position.
\end{entry}

\begin{entry}{%
\proto{call-with-bytevector-output-port}{ proc}{procedure}
\pproto{(call-with-bytevector-output-port \var{proc}}{procedure}{\tt\obeyspaces%
\hspace*{2em}\var{maybe-transcoder})}}

\domain{\var{Proc} must accept one argument.
  \var{Maybe-transcoder} must be either a transcoder or \schfalse.}

The {\cf call-with-bytevector-output-port} procedure creates an output
port that accumulates the bytes written to it and calls \var{proc} with
that output port as an argument. Whenever \var{proc} returns, a
bytevector consisting of all of the port's accumulated bytes (regardless
of the port's current position) is returned and
the port is closed.

The transcoder associated with the output port is determined
as for a call to {\cf open-bytevector-output-port}.
\end{entry}

\begin{entry}{%
\proto{open-string-output-port}{}{procedure}}

Returns two values: a textual output port and an extraction procedure.
The output port accumulates the characters written to it for
later extraction by the procedure.

The port may or may not have an associated transcoder; if it does, the
transcoder is implementation-dependent.  The port should support the {\cf
  port-position} and {\cf set-port-position!} operations.

The extraction procedure takes no arguments.
When called, it returns a string consisting of all of the port's
accumulated characters (regardless of the current position),
removes the accumulated characters from the port, and resets
the port's position.
\end{entry}

\begin{entry}{%
\proto{call-with-string-output-port}{ proc}{procedure}}

\domain{\var{Proc} must accept one argument.}
The {\cf call-\hp{}with-\hp{}string-\hp{}output-\hp{}port} procedure creates a textual output port that accumulates the
characters written to it and calls \var{proc} with that output port
as an argument.  Whenever \var{proc} returns, a string consisting of all
of the port's accumulated characters (regardless of the port's current
position) is returned and the port is closed.

The port may or may not have an associated transcoder; if it does, the
transcoder is implementation-dependent.  The port should support the
{\cf port-position} and {\cf set-port-position!} operations.
\end{entry}

\begin{entry}{%
\proto{standard-output-port}{}{procedure}
\proto{standard-error-port}{}{procedure}}
   
Returns a fresh binary output port connected to the standard output or
standard error respectively.  Whether the port supports the {\cf
  port-position} and {\cf set-port-position!} operations is
implementation-dependent.
\end{entry}

\begin{entry}{%
\proto{current-output-port}{}{procedure}
\proto{current-error-port}{}{procedure}}
 
These return default textual ports for regular output and error
output.  Normally, these default ports are associated with standard
output, and standard error, respectively.  The return value of {\cf
  current-output-port} can be dynamically re-assigned using the {\cf
  with-output-to-file} procedure from the \rsixlibrary{io simple}
library (see section~\ref{with-output-to-file}).  A port returned by
one of these procedures may or may not have an associated transcoder;
if it does, the transcoder is implementation-dependent.
\end{entry}


\begin{entry}{%
\pproto{(make-custom-binary-output-port \var{id}}{procedure}}
\mainschindex{make-custom-binary-output-port}{\tt\obeyspaces\\
  \var{write!} \var{get-position} \var{set-position!} \var{close})}

Returns a newly created binary output port whose byte sink is
an arbitrary algorithm represented by the \var{write!} procedure.
\var{Id} must be a string naming the new port,
provided for informational purposes only.
\var{Write!} must be a procedure and should behave as specified
below; it will be called by operations that perform binary output.

Each of the remaining arguments may be \schfalse{}; if any of
those arguments is not \schfalse{}, it must be a procedure and
should behave as specified in the description of
{\cf make-custom-binary-input-port}.
   
\begin{itemize}
\item {\cf (\var{write!} \var{bytevector} \var{start} \var{count})}
       
  \domain{\var{Start} and \var{count} will be non-negative exact
    integer objects,
  and \var{bytevector} will be a bytevector whose length is at least
  $\var{start} + \var{count}$.}
  The \var{write!} procedure should write up to \var{count} bytes
  from \var{bytevector} starting at index \var{start} 
  to the byte sink.
  The \var{write!} procedure should return the number of
  bytes that it wrote, as an exact integer object.
\end{itemize}

\implresp The implementation must check the return
values of \var{write!} only when it actually calls \var{write!} as part of
an I/O operation requested by the program.  The implementation is not
required to check that \var{write!} otherwise behaves as described.
If it does not, however, the behavior of the resulting port is
unspecified.
\end{entry}

\begin{entry}{%
\pproto{(make-custom-textual-output-port \var{id}}{procedure}}
\mainschindex{make-custom-textual-output-port}{\tt\obeyspaces\\
  \var{write!} \var{get-position} \var{set-position!} \var{close})}

Returns a newly created textual output port whose byte sink is
an arbitrary algorithm represented by the \var{write!} procedure.
\var{Id} must be a string naming the new port,
provided for informational purposes only.
\var{Write!} must be a procedure and should behave as specified
below; it will be called by operations that perform textual output.

Each of the remaining arguments may be \schfalse{}; if any of
those arguments is not \schfalse{}, it must be a procedure and
should behave as specified in the description of
{\cf make-custom-textual-input-port}.
   
\begin{itemize}
\item {\cf (\var{write!} \var{string} \var{start} \var{count})}
       
  \domain{\var{Start} and \var{count} will be non-negative exact
    integer objects,
  and \var{string} will be a string whose length is at least
  $\var{start} + \var{count}$.}
  The \var{write!} procedure should write up to \var{count} characters
  from \var{string} starting at index \var{start} 
  to the character sink.
  The \var{write!} procedure should return the number of
  characters that it wrote, as an exact integer object.
\end{itemize}

The port may or may not have an associated transcoder; if it does, the
transcoder is implementation-dependent.

\implresp The implementation must check the return
values of \var{write!} only when it actually calls \var{write!} as part of
an I/O operation requested by the program.  The implementation is not
required to check that \var{write!} otherwise behaves as described.
If it does not, however, the behavior of the resulting port is
unspecified.
\end{entry}

\subsection{Binary output}\unsection

\begin{entry}{%
\proto{put-u8}{ binary-output-port octet}{procedure}}

Writes \var{octet} to the output port and returns \unspecifiedreturn.
\end{entry}

\begin{entry}{%
\proto{put-bytevector}{ binary-output-port bytevector}{procedure}
\rproto{put-bytevector}{ binary-output-port bytevector start}{procedure}
\pproto{(put-bytevector \var{binary-output-port}}{procedure}}
{\tt\obeyspaces\\
     \var{bytevector} \var{start} \var{count})}
   
\domain{\var{Start} and \var{count} must be non-negative exact
  integer objects that default to 0 and $\texttt{(bytevector-length \var{bytevector})}
  - \var{start}$, respectively. \var{Bytevector} must have a length of at
  least $\var{start} + \var{count}$.}  The {\cf put-bytevector} procedure writes
the \var{count} bytes of the bytevector \var{bytevector} starting at index
\var{start} to the output port.  The {\cf put-bytevector} procedure
returns \unspecifiedreturn.
\end{entry}

\subsection{Textual output}\unsection

\begin{entry}{%
\proto{put-char}{ textual-output-port char}{procedure}}
   
Writes \var{char} to the port. The {\cf put-char} procedure returns
\unspecifiedreturn.
\end{entry}

\begin{entry}{%
\proto{put-string}{ textual-output-port string}{procedure}
\rproto{put-string}{ textual-output-port string start}{procedure}
\rproto{put-string}{ textual-output-port string start count}{procedure}}
   
\domain{\var{Start} and \var{count} must be non-negative exact
  integer objects.  \var{String} must have a length of at least $\var{start} +
  \var{count}$.}  \var{Start} defaults to 0.  \var{Count} defaults to
$\texttt{(string-length \var{string})} - \var{start}$.  The {\cf
  put-string} procedure writes the
\var{count} characters of \var{string} starting at
index \var{start} to the port.  The {\cf put-string} procedure
returns \unspecifiedreturn.
\end{entry}


\begin{entry}{%
\proto{put-datum}{ textual-output-port datum}{procedure}}

\domain{\var{Datum} should be a datum value.}  The {\cf put-datum}
procedure writes an external representation of \var{datum} to
\var{textual-output-port}.  The specific external representation is
implementation-dependent.  However, whenever possible, an
implementation should produce a representation for which {\cf
  get-datum}, when reading the representation, will return an object
equal (in the sense of {\cf equal?}) to \var{datum}.

\begin{note}
  Not all datums may allow producing an external representation for which
  {\cf get-datum} will produce an object that is equal to the
  original.  Specifically, NaNs contained in \var{datum} may make
  this impossible.
\end{note}

\begin{note}
  The {\cf put-datum} procedure merely writes the external
  representation, but no trailing delimiter.  If {\cf put-datum} is
  used to write several subsequent external representations to an
  output port, care should be taken to delimit them properly so they can
  be read back in by subsequent calls to {\cf get-datum}.
\end{note}
\end{entry}


\subsection{Input/output ports}\unsection

\begin{entry}{%
\proto{open-file-input/output-port}{ filename}{procedure}
\rproto{open-file-input/output-port}{ filename file-options}{procedure}
\pproto{(open-file-input/output-port \var{filename}}{procedure}{\tt\obeyspaces%
\hspace*{2em}\var{file-options} \var{buffer-mode})}\\
\pproto{(open-file-input/output-port \var{filename}}{procedure}{\tt\obeyspaces%
\hspace*{2em}\var{file-options} \var{buffer-mode} \var{transcoder})}}
   
Returns a single port that is both an input port and an
output port for the named file.
The optional arguments default as described in the specification
of {\cf open-file-output-port}.
If the input/output port supports {\cf port-position} and/or
{\cf set-port-position!}, the same port position is used
for both input and output.
\end{entry}

\begin{entry}{%
\pproto{(make-custom-binary-input/output-port}{procedure}}
\mainschindex{make-custom-binary-input/output-port}{\tt\obeyspaces\\
  \var{id} \var{read!} \var{write!} \var{get-position} \var{set-position!} \var{close})}

Returns a newly created binary input/output port whose
byte source and sink are
arbitrary algorithms represented by the \var{read!} and \var{write!}
procedures.
\var{Id} must be a string naming the new port,
provided for informational purposes only.
\var{Read!} and \var{write!} must be procedures,
and should behave as specified for the
{\cf make-custom-binary-input-port} and
{\cf make-custom-binary-output-port} procedures.

Each of the remaining arguments may be \schfalse{}; if any of
those arguments is not \schfalse{}, it must be a procedure and
should behave as specified in the description of
{\cf make-custom-binary-input-port}.

\begin{note}
  Unless both \var{get-position} and \var{set-position!} procedures are
  supplied, a put operation cannot precisely position the port for output
  to a custom binary input/output port after data has been read from
  the port.
\end{note}
\end{entry}

\begin{entry}{%
\pproto{(make-custom-textual-input/output-port}{procedure}}
\mainschindex{make-custom-textual-input/output-port}{\tt\obeyspaces\\
  \var{id} \var{read!} \var{write!} \var{get-position} \var{set-position!} \var{close})}

Returns a newly created textual input/output port whose
textual source and sink are
arbitrary algorithms represented by the \var{read!} and \var{write!}
procedures.
\var{Id} must be a string naming the new port,
provided for informational purposes only.
\var{Read!} and \var{write!} must be procedures,
and should behave as specified for the
{\cf make-custom-textual-input-port} and
{\cf make-custom-textual-output-port} procedures.

Each of the remaining arguments may be \schfalse{}; if any of
those arguments is not \schfalse{}, it must be a procedure and
should behave as specified in the description of
{\cf make-custom-textual-input-port}.

\begin{note}
  Even when both \var{get-position} and \var{set-position!} procedures are
  supplied, the {\cf port-position} procedure cannot generally return a
  precise value for a custom textual input/output port, and a put operation
  cannot precisely position the port for output, after data has
  been read from the port.
\end{note}
\end{entry}


%%% Local Variables: 
%%% mode: latex
%%% TeX-master: "r6rs-lib"
%%% End: 

\section{Port I/O}
\label{portsiosection}

The \defrsixlibrary{i/o ports} library defines an I/O layer for
conventional, imperative buffered input and output.
A \defining{port} represents a buffered access object
for a data sink or source or both simultaneously.
The library allows ports to be created from arbitrary data sources
and sinks.

The \rsixlibrary{i/o ports} library distinguishes between \textit{input
  ports\mainindex{input port}} and \textit{output
  ports\mainindex{output ports}}.  An input port is a source for data,
whereas an output port is a sink for data.  A port may be both an
input port and an output port; such a port typically provides
simultaneous read and write access to a file or other data.

The \rsixlibrary{i/o ports} library also distinguishes between
\textit{binary ports\mainindex{binary port}}, which are sources
or sinks for uninterpreted bytes, and
\textit{textual ports\mainindex{textual ports}}, which are sources
or sinks for characters and strings.

This section uses \var{input-port}, \var{output-port},
\var{binary-port}, \var{textual-port},
\var{binary-input-port}, \var{textual-input-port},
\var{binary-output-port}, \var{textual-output-port},
and \var{port} as
parameter names for arguments that must be input ports (or combined
input/output ports), output ports (or combined input/output ports),
binary ports, textual ports, binary input ports, textual input ports,
binary output ports, textual output ports, or any kind of port,
respectively.

\subsection{File names}
\label{filenamesection}

Some of the procedures described in this chapter accept a file name as an
argument. Valid values for such a file name include strings that name a file
using the native notation of filesystem paths on an implementation's
underlying operating system, and may include implementation-dependent
values as well.

\begin{rationale}
Implementation-dependent file names may provide a more
abstract and/or more general representation. Indeed, most operating
systems do not use strings for representing file names, but rather byte
or word sequences.
Furthermore the string notation is not fully portable across operating
systems, and is difficult to manipulate.
\end{rationale}

A \var{filename} parameter name means that the
corresponding argument must be a file name.

\subsection{File options}
\label{fileoptionssection}

\mainindex{file options}
When opening a file, the various procedures in this library accept a
{\cf file-options} object that encapsulates flags to specify how
the file is to be opened. A {\cf file-options} object is an enum-set
(see chapter~\ref{enumerationschapter}) over the symbols constituting
valid file options.
A \var{file-options} parameter name means that the
corresponding argument must be a file-options object.

\begin{entry}{%
\proto{file-options}{ \hyper{file-options name} \dotsfoo}{\exprtype}}

Each \hyper{file-options name} must be an \meta{identifier}.
The {\cf file-options} syntax returns a file-options object that 
encapsulates the
specified options. The following options, which affect output
only, have standard meanings:

\begin{itemize}   
\item {\cf no-create} do not create file if it does not already exist
  (if this option is absent, the file is created)
\item {\cf no-fail} do not fail if this option is set and {\cf
    no-create} is absent (if the {\cf no-create} and {\cf no-fail} are
  both absent the file already exists, then an exception with condition type
  {\cf\&i/o-file-already-exists} is raised.)
\item {\cf no-truncate}
  file is not truncated to zero length (if absent, the file is truncated)
\end{itemize}

\meta{Identifiers}s
other than those listed above may be used as \hyper{file-options name}s;
they have implementation-specific meaning, if any.

When supplied to an operation that opens a file for output, the
file-options object returned by {\cf (file-options)} specifies that
the file is created if it does not exist, and is truncated and
overwritten if it exists.

\begin{rationale}
  The flags specified above represent only a common subset of
  meaningful options on popular platforms.  The {\cf file-options}
  form does not restrict the \hyper{file-options name}s, so 
  implementations can extend the file options by platform-specific
  flags.
\end{rationale}
\end{entry}   

\subsection{Buffer modes}

Each port has an associated buffer mode.  For an output port, the
buffer mode defines when an output operation flushes the buffer
associated with the output port.  For an input port, the buffer mode
defines how much data will be read to satisfy read operations.  The
possible buffer modes are the symbols {\cf none} for no buffering,
{\cf line} for flushing upon line endings or reading until line
endings, and {\cf block} for arbitrary buffering.  This section uses
the parameter name \var{buffer-mode} for arguments that must be
buffer-mode symbols.

If two ports are connected to the same mutable source, and both ports
are unbuffered, and reading a byte or character from that shared
source via one of the two ports would change the bytes or characters
seen via the other port, then a lookahead operation on one port will
render the peeked byte or character inaccessible via the other port,
while a subsequent read operation on the peeked port will see the
peeked byte or character even though the port is otherwise unbuffered.

In other words, the semantics of buffering is defined in terms of side
effects on shared mutable sources, and a lookahead operation has the
same side effect on the shared source as a read operation.

\begin{entry}{%
\proto{buffer-mode}{ \hyper{name}}{\exprtype}}
   
\hyper{Name} must be one of the \meta{identifier}s {\cf none}, {\cf line}, or
{\cf block}. The result is the corresponding symbol, denoting the
associated buffer mode.

It is a syntax violation if \hyper{name} is not one of the valid
identifiers.
\end{entry}

\begin{entry}{%
\proto{buffer-mode?}{ obj}{procedure}}
   
Returns \schtrue{} if the argument is a valid buffer-mode symbol,
and returns \schfalse{} otherwise.
\end{entry}

\subsection{Transcoders}
\label{transcoderssection}

Several different Unicode encoding schemes describe standard ways to
encode characters and strings as byte sequences and to decode those
sequences~\cite{Unicode}.
Within this document, a \defining{codec} is an immutable Scheme
object that represents a Unicode or similar encoding scheme.

An \defining{end-of-line style} is a symbol that, if it is not {\cf
  none}, describes how a textual port transcodes representations of
line endings.

A \defining{transcoder} is an immutable Scheme object that combines
a codec with an end-of-line style and a method for handling
decoding errors.
Each transcoder represents some specific bidirectional (but not
necessarily lossless), possibly stateful translation between byte
sequences and Unicode characters and strings.
Every transcoder can operate in the input direction (bytes to characters)
or in the output direction (characters to bytes),
but the composition of those directions need not be identity (and
often isn't).  The composition of two transcoders is not defined.
A \var{transcoder} parameter name means that the corresponding
argument must be a transcoder.

A \defining{binary port} is a port that does not have an associated
transcoder: It does not support textual I/O.  A \defining{textual
  port} is a port with an associated transcoder.

\begin{entry}{%
\proto{latin-1-codec}{}{procedure}
\proto{utf-8-codec}{}{procedure}
\proto{utf-16-codec}{}{procedure}}

These are predefined codecs for the ISO 8859-1, UTF-8,
and UTF-16 encoding schemes \cite{Unicode}.

A call to any of these procedures returns a value that is equal in the
sense of {\cf eqv?} to the result of any other call to the same
procedure.
\end{entry}

\begin{entry}{%
\proto{eol-style}{ name}{\exprtype}}

If \var{name} is one of the \meta{identifier}s {\cf lf}, {\cf cr},
{\cf crlf}, {\cf nel}, {\cf crnel}, {\cf ls}, or {\cf none}, then the form
evaluates to the corresponding symbol.  If \var{name} is not one of
these identifiers, effect and result are implementation-dependent: The
result may be an eol-style symbol acceptable as an \var{eol-mode}
argument to {\cf make-transcoder}.  Otherwise, an exception is raised.

All eol-style symbols except {\cf none} describe a specific
line-ending encoding:

\noindent\begin{tabular}{ll}
{\cf lf} & \meta{linefeed}\\
{\cf cr} & \meta{carriage return}\\
{\cf crlf} & \meta{carriage return} \meta{linefeed}\\
{\cf nel} & \meta{next line}\\
{\cf crnel} & \meta{carriage return} \meta{next line}\\
{\cf ls} & \meta{line separator}
\end{tabular}

For a textual port whose transcoder has an eol-style symbol {\cf
  none}, no conversion occurs.  For a textual input port, any
eol-style symbol other than {\cf none} means that all of the above
line-ending encodings are recognized and are translated into a single
linefeed.  For a textual output port, {\cf none} and {\cf lf} are
equivalent.  Linefeed characters are encoded according to the
specified eol-style symbol, and all other characters that participate
in possible line endings are encoded as is.

\begin{rationale}
  The set is not closed because end-of-line styles other than those
  listed might become commonplace in the future.
\end{rationale}
\end{entry}

\begin{entry}{%
\proto{native-eol-style}{}{procedure}}

Returns the default end-of-line style of the underlying platform, e.g.,
{\cf lf} on Unix and {\cf crlf} on Windows.
\end{entry}

\begin{entry}{%
\ctproto{i/o-decoding}
\proto{make-i/o-decoding-error}{ port}{procedure}
\proto{i/o-decoding-error?}{ obj}{procedure}
\proto{i/o-decoding-error-transcoder}{ condition}{procedure}}

This condition type could be defined by
%
\begin{scheme}
(define-condition-type \&i/o-decoding \&i/o-port
  make-i/o-decoding-error i/o-decoding-error?
  (transcoder i/o-decoding-error-transcoder))
\end{scheme}

An exception with this type is raised when one of the operations for
textual input from a port encounters a sequence of bytes that cannot
be translated into a character or string by the input direction of the
port's transcoder.  The {\cf transcoder} field contains the port's
transcoder.

Exceptions of this type raised by the operations described in this
section are continuable.
When such an exception is raised, the port's position is at
the beginning of the invalid encoding.
If the exception handler returns, it should
return a character or string representing the decoded text starting at
the port's current position, and the exception handler must update the 
port's position to point past the error.

\implresp The implementation must check that the exception handler
returns a character or a string only if it actually returns.
\end{entry}

\begin{entry}{% 
\ctproto{i/o-encoding}
\proto{make-i/o-encoding-error}{ port char transcoder}{procedure}
\proto{i/o-encoding-error?}{ obj}{procedure}
\proto{i/o-encoding-error-char}{ condition}{procedure}
\proto{i/o-encoding-error-transcoder}{ condition}{procedure}}

This condition type could be defined by
%
\begin{scheme}
(define-condition-type \&i/o-encoding \&i/o-port
  make-i/o-encoding-error i/o-encoding-error?
  (char i/o-encoding-error-char)
  (transcoder i/o-encoding-error-transcoder))
\end{scheme}

An exception with this type is raised when one of the operations for
textual output to a port encounters a character that cannot be
translated into bytes by the output direction of the port's transcoder.
The {\cf char} field of the
condition object contains the character that could not be encoded,
and the {\cf transcoder} field contains the transcoder associated
with the port.

Exceptions of this type raised by the operations described in this
section are continuable.  The handler, if it returns, should 
output to the port an appropriate encoding for the character that
caused the error.  The operation that raised the exception 
continues after that character.

\implresp The implementation is not required to check whether the
handler has output an encoding.
\end{entry}

\begin{entry}{%
\proto{error-handling-mode}{ name}{\exprtype}}

If \var{name} is one of the \meta{identifier}s {\cf ignore}, {\cf
  raise}, or {\cf replace}, then the result is the corresponding
symbol.  If \var{name} is not one of these identifiers, effect and
result are implementation-dependent: The result may be an
error-handling-mode symbol acceptable as a \var{handling-mode}
argument to {\cf make-transcoder}.  If it is not acceptable as a
\var{handling-mode} argument to {\cf make-transcoder}, an exception is raised.

\begin{rationale}
  Implementations may support error-handling modes other than those
  listed.
\end{rationale}

The error-handling mode of a transcoder specifies the behavior
of textual I/O operations in the presence of encoding or decoding
errors.

If a textual input operation encounters an invalid or incomplete
character encoding, and the error-handling mode is {\cf ignore},
then an appropriate number of bytes of the
invalid encoding are ignored and decoding continues with the
following bytes.
If the error-handling mode is {\cf replace}, then the replacement
character U+FFFD is injected into the data stream, an appropriate
number of bytes are ignored, and decoding
continues with the following bytes.
If the error-handling mode is {\cf raise}, then a continuable
exception with condition type {\cf\&i/o-decoding} is raised;
see the description of
{\cf\&i/o-decoding} for details
on how to handle such an exception.

If a textual output operation encounters a character it cannot encode,
and the error-handling mode is {\cf ignore}, then the character is
ignored and encoding continues with the next character.
If the error-handling mode is {\cf replace}, a codec-specific
replacement character is emitted by the transcoder, and encoding
continues with the next character.
The replacement character is U+FFFD for transcoders whose codec
is one of the Unicode encodings, but is the {\cf ?}
character for the Latin-1 encoding.
If the error-handling mode is {\cf raise}, an
exception with condition type {\cf\&i/o-encoding} is raised;
see the description of
{\cf\&i/o-decoding} for details
on how to handle such an exception.
\end{entry}

\begin{entry}{%
\proto{make-transcoder}{ codec}{procedure}
\rproto{make-transcoder}{ codec eol-style}{procedure}
\rproto{make-transcoder}{ codec eol-style handling-mode}{procedure}}

\domain{\var{Codec} must be a codec; \var{eol-style}, if present, an
  eol-style symbol; and \var{handling-mode}, if present, an
  error-handling-mode symbol.}  \var{eol-style} may be omitted, in
which case it defaults to the native end-of-line style of the
underlying platform.  \var{handling-mode} may be omitted, in which
case it defaults to {\cf raise}.  The result is a transcoder with the
behavior specified by its arguments.
\end{entry}

\begin{entry}{
\proto{native-transcoder}{}{procedure}}

Returns an implementation-dependent transcoder that represents a
possibly locale-dependent ``native'' transcoding.
\end{entry}

\begin{entry}{%
\proto{transcoder-codec}{ transcoder}{procedure}
\proto{transcoder-eol-style}{ transcoder}{procedure}
\proto{transcoder-error-handling-mode}{ transcoder}{procedure}}

These are accessors for transcoder objects; when applied to a
transcoder returned by {\cf make-transcoder}, they return the
\var{codec}, \var{eol-style}, and \var{handling-mode} arguments,
respectively.
\end{entry}

\begin{entry}{%
\proto{bytevector->string}{ bytevector transcoder}{procedure}}

Returns the string that results from transcoding the
\var{bytevector} according to the transcoders's input direction.
\end{entry}

\begin{entry}{%
\proto{string->bytevector}{ string transcoder}{procedure}}

Returns the bytevector that results from transcoding the
\var{string} according to the transcoder's output direction.
\end{entry}

\subsection{End of file object}
\label{eofsection}

The end of file object is returned by various I/O procedures when they
reach end of file.\index{end of file object}

\begin{entry}{%
\proto{eof-object}{}{procedure}}

Returns the end of file object.
\begin{scheme}
(eqv? (eof-object) (eof-object)) \lev  \schtrue
(eq? (eof-object) (eof-object)) \lev  \schtrue%
\end{scheme}
\end{entry}

\begin{note}
  The end of file object  is not a datum value, and thus has no external
  representation.
\end{note}

\begin{entry}{%
\proto{eof-object?}{ obj}{procedure}}
  
Returns \schtrue{} if \var{obj} is the end of file object, otherwise
returns \schfalse.
\end{entry}

\subsection{Input and output ports}

The operations described in this section are common to input and
output ports, both binary and textual.  A port may also have an
associated \defining{position} that specifies a particular place
within its data sink or source, and may also provide operations for
inspecting and setting that place.

\begin{entry}{%
\proto{port?}{ obj}{procedure}}
   
Returns \schtrue{} if the argument is a port, and returns \schfalse{}
otherwise.
\end{entry}

\begin{entry}{%
\proto{port-transcoder}{ port}{procedure}}

Returns the transcoder associated with \var{port} if \var{port} is
textual, and returns \schfalse{} if \var{port} is binary.
\end{entry}

\begin{entry}{%
\proto{textual-port?}{ port}{procedure}
\proto{binary-port?}{ port}{procedure}}

The {\cf textual-port} procedure returns \schtrue{} if \var{port} is
textual, \schfalse{} otherwise.
The {\cf binary-port} procedure returns \schtrue{} if \var{port} is
textual, \schfalse{} otherwise.
\end{entry}

\begin{entry}{%
\proto{transcoded-port}{ binary-port transcoder}{procedure}}

The {\cf transcoded-port} procedure
returns a new textual port with the specified \var{transcoder}.
Otherwise the new textual port's state is largely the same as
that of the \var{binary-port}.
If the \var{binary-port} is an input port, then the new textual
port will be an input port and
will transcode the bytes that have not yet been read from
the \var{binary-port}.
If the \var{binary-port} is an output port, then the new textual
port will be an output port and
will transcode output characters into bytes that are
written to the byte sink represented by the \var{binary-port}.

As a side effect, however, the {\cf transcoded-port} procedure
closes \var{binary-port} in
a special way that allows the new textual port to continue to
use the byte source or sink represented by the \var{binary-port},
even though the \var{binary-port} itself is closed and cannot
be used by the input and output operations described in this
chapter.

\begin{rationale}
Closing the \var{binary-port} precludes interference between
the \var{binary-port} and the textual port constructed from it.
\end{rationale}
\end{entry}

\begin{entry}{%
\proto{port-has-port-position?}{ port}{procedure}
\proto{port-position}{ port}{procedure}}

The {\cf port-has-port-position?} procedure returns \schtrue{} if the
port supports the {\cf port-position} operation, and \schfalse{}
otherwise.

For a binary port, the {\cf port-position} procedure returns the exact
non-negative integer index of the position at which the next byte
would be read from or written to the port.  For a textual port, {\cf
  port-position} returns an arbitrary value that is acceptable as
input for {\cf set-port-position} (see below).

If the port does not support the operation, {\cf port-position} raises
an exception with condition type {\cf\&assertion}.

\begin{note}
  For a textual port, the port position may or may not be an integer.
  If it is an integer, the integer does not necessarily correspond to
  a byte or character position.
\end{note}
\end{entry}   

\begin{entry}{%
\proto{port-has-set-port-position!?}{ port}{procedure}
\proto{set-port-position!}{ port pos}{procedure}}

\domain{If \var{port} is a binary port, \var{pos} must be a
  non-negative exact integer.  If \var{port} is a textual port,
  \var{pos} should be the return value of a call to {\cf port-position}.}
   
The {\cf port-has-set-port-position?} procedure returns \schtrue{} if the port
supports the {\cf set-port-position!} operation, and \schfalse{}
otherwise.
   
The {\cf set-port-position!} procedure sets the current position
of the port to \var{pos}.  If \var{port} is an output or combined
input and output port, this first flushes \var{port}.  (See {\cf
  flush-output-port}, section~\ref{flush-output-port}.)
An exception with condition type {\cf\&assertion} is raised
if the port does not support the operation.
\end{entry}

\begin{entry}{%
\proto{close-port}{ port}{procedure}}
   
Closes the port, rendering the port incapable of delivering or
accepting data. If \var{port} is an output port, it is flushed before
being closed.  This has no effect if the port has already been closed.
A closed port is still a port.  The {\cf close-port} procedure returns
\unspecifiedreturn.
\end{entry}

\begin{entry}{%
\proto{call-with-port}{ port proc}{procedure}}
   
\domain{\var{Proc} must be a procedure that accepts a single
  argument.}  The {\cf call-with-port} procedure
calls \var{proc} with \var{port} as an argument. If
\var{proc} returns, then the \var{port} is closed automatically and
the values returned by \var{proc} are returned. If \var{proc} does not
return, then the port is not closed automatically, except perhaps when it is
possible to prove that the port will never again be used for a
{\cf lookahead}, {\cf get}, or {\cf put} operation.
\end{entry}

\subsection{Input ports}

An input port allows reading an infinite sequence of bytes
or characters punctuated
by end of file objects. An input port connected to a finite data
source ends in an infinite sequence of end of file objects.

It is unspecified whether a character encoding consisting of several
bytes may have an end of file between the bytes.  If, for example,
{\cf get-char} raises an {\cf\&i/o-decoding} exception because the
character encoding at the port's position is incomplete up to the next
end of file, a subsequent call to {\cf get-char} may successfully
decode a character if bytes completing the encoding are available
after the end of file.

\begin{entry}{%
\proto{input-port?}{ obj}{procedure}}

Returns \schtrue{} if the argument is an input port (or a combined input
and output port), and returns \schfalse{} otherwise.
\end{entry}

\begin{entry}{%
\proto{port-eof?}{ input-port}{procedure}}
   
Returns \schtrue{}
if the {\cf lookahead-u8} procedure (if \var{input-port} is a binary port)
or the {\cf lookahead-char} procedure (if \var{input-port} is a textual port)
would return
the end-of-file object, and returns \schfalse{} otherwise.
\end{entry}

\begin{entry}{%
\proto{open-file-input-port}{ filename}{procedure}
\rproto{open-file-input-port}{ filename file-options}{procedure}
\pproto{(open-file-input-port \var{filename}}{procedure}{\tt\obeyspaces%
\hspace*{2em}\var{file-options} \var{buffer-mode})}\\
\pproto{(open-file-input-port \var{filename}}{procedure}{\tt\obeyspaces%
\hspace*{2em}\var{file-options} \var{buffer-mode} \var{maybe-transcoder})}}
   
\domain{\var{Maybe-transcoder} must either be a transcoder or \schfalse.}

Returns an input port for the named file. The \var{file-options} and
\var{maybe-transcoder} arguments are optional.

The \var{file-options} argument, which may determine
various aspects of the returned port (see section~\ref{fileoptionssection}),
defaults to {\cf (file-options)}.

The \var{buffer-mode} argument, if supplied,
must be one of the symbols that name a buffer mode.
The \var{buffer-mode} argument defaults to {\cf block}.

If \var{maybe-transcoder} is a transcoder, it becomes the transcoder associated
with the returned port.

If \var{maybe-transcoder} is \schfalse{} or absent,
then the port will be a binary port and will support the
{\cf port-position} and {\cf set-port-position!}  operations.
Otherwise the port will be a textual port, and whether it supports
the {\cf port-position} and {\cf set-port-position!} operations
will be implementation-dependent (and possibly transcoder-dependent).

\begin{rationale}
  The position of a transcoded port may not be
  well-defined, and may be hard to calculate even when defined,
  especially when transcoding is buffered.
\end{rationale}
\end{entry}

\begin{entry}{%
\proto{open-bytevector-input-port}{ bytevector}{procedure}
\pproto{(open-bytevector-input-port \var{bytevector}}{procedure}{\tt\obeyspaces%
\hspace*{2em}\var{maybe-transcoder})}}

\domain{\var{Maybe-transcoder} must either be a transcoder or \schfalse.}
   
The {\cf open-bytevector-input-port} procedure returns an input port whose bytes are drawn from the
\var{bytevector}.
If \var{transcoder} is specified, it becomes the transcoder associated
with the returned port.

If no \var{maybe-transcoder} argument is given, or
if \schfalse{} is passed as an explicit argument,
then the port will be a binary port and will support the
{\cf port-position} and {\cf set-port-position!}  operations.
Otherwise the port will be a textual port, and whether it supports
the {\cf port-position} and {\cf set-port-position!} operations
will be implementation-dependent (and possibly transcoder-dependent).

If \var{bytevector} is modified after {\cf open-\linebreak[0]bytevector-\linebreak[0]input-\linebreak[0]port}
has been called, the effect on the returned
port is unspecified.
\end{entry}

\begin{entry}{%
\proto{open-string-input-port}{ string}{procedure}}

Returns a textual input port whose characters are drawn from
\var{string}.  The port has an associated transcoder, which is implementation-dependent.
Whether the port supports
the {\cf port-position} and {\cf set-port-position!} operations
is implementation-dependent.

If \var{string} is modified after {\cf open-string-input-port}
has been called, the effect on the returned port is unspecified.
\end{entry}

\begin{entry}{%
\proto{standard-input-port}{}{procedure}}
   
Returns a fresh binary input port connected to standard input.
Whether the port supports the {\cf port-position} and {\cf
  set-port-position!} operations is implementation-dependent.

\begin{rationale}
  The port is fresh so it can be safely closed or converted to a
  textual port without risking the usability of an existing port.
\end{rationale}
\end{entry}

\begin{entry}{%
\proto{current-input-port}{}{procedure}}
 
This returns a default textual port for input.  Normally, this default port
is associated with standard input, but can be dynamically re-assigned
using the {\cf with-input-from-file} procedure from the
\rsixlibrary{i/o simple} library (see section~\ref{with-input-from-file}).
\end{entry}

\begin{entry}{%
\pproto{(make-custom-binary-input-port \var{id} \var{read!}}{procedure}}
\mainschindex{make-custom-binary-input-port}{\tt\obeyspaces\\
  \var{get-position} \var{set-position!} \var{close})}

Returns a newly created binary input port whose byte source is
an arbitrary algorithm represented by the \var{read!} procedure.
\var{Id} must be a string naming the new port,
provided for informational purposes only.
\var{Read!} must be a procedure, and should behave as specified
below; it will be called by operations that perform binary input.

Each of the remaining arguments may be \schfalse{}; if any of
those arguments is not \schfalse{}, it must be a procedure and
should behave as specified below.
   
\begin{itemize}
\item {\cf (\var{read!} \var{bytevector} \var{start} \var{count})}
       
  \domain{\var{Start} will be a non-negative exact integer,
  \var{count} will be a positive exact integer,}
  and \var{bytevector} will be a bytevector whose length is at least
  $\var{start} + \var{count}$.
  The \var{read!} procedure should obtain up to \var{count} bytes
  from the byte source, and should write those bytes
  into \var{bytevector} starting at index \var{start}.
  The \var{read!} procedure should return an exact integer.  This
  integer should be the number of bytes that it has written.
  To indicate an end of file condition, the \var{read!}
  procedure should write no bytes and return 0.

\item {\cf (\var{get-position})}
       
  The \var{get-position} procedure (if supplied) should return an exact
  integer.  The return value should represent the current position of
  the input port.  If not supplied, the custom port will not support
  the {\cf port-position} operation.
  
\item {\cf (\var{set-position!} \var{k})}
       
  The \var{set-position!} procedure (if supplied) should set the
  position of the input port to \var{k}.  If not supplied, the custom
  port will not support the {\cf set-port-position!} operation.
       
\item {\cf (\var{close})}
       
  The \var{close} procedure (if supplied) should perform any actions
  that are necessary when the input port is closed.
\end{itemize}

\implresp The implementation is only required to check the return
values of \var{read!} and \var{get-position} when it actually calls
them as part of an I/O operation requested by the program.  The
implementation is not required to check that these procedures
otherwise behave as described.  If they do not, however, the behavior
of the resulting port is unspecified.
\end{entry}

\begin{entry}{%
\pproto{(make-custom-textual-input-port \var{id} \var{read!}}{procedure}}
\mainschindex{make-custom-textual-input-port}{\tt\obeyspaces\\
  \var{get-position} \var{set-position!} \var{close})}

Returns a newly created textual input port whose character source is
an arbitrary algorithm represented by the \var{read!} procedure.
\var{Id} must be a string naming the new port,
provided for informational purposes only.
\var{Read!} must be a procedure, and should behave as specified
below; it will be called by operations that perform textual input.

Each of the remaining arguments may be \schfalse{}; if any of
those arguments is not \schfalse{}, it must be a procedure and
should behave as specified below.
   
\begin{itemize}
\item {\cf (\var{read!} \var{string} \var{start} \var{count})}
       
  \domain{\var{Start} will be a non-negative exact integer,
  \var{count} will be a positive exact integer,}
  and \var{string} will be a string whose length is at least
  $\var{start} + \var{count}$.
  The \var{read!} procedure should obtain up to \var{count} characters
  from the character source, and should write those characters
  into \var{string} starting at index \var{start}.
  The \var{read!} procedure must return an exact integer.  This
  integer should be the number of characters that it has written.
  To indicate an end of file condition, the \var{read!}
  procedure should write no bytes and return 0.

\item {\cf (\var{get-position})}
       
  The \var{get-position} procedure (if supplied) should return a single
  value.  The return value should represent the current position of
  the input port.  If not supplied, the custom port will not support
  the {\cf port-position} operation.
  
\item {\cf (\var{set-position!} \var{pos})}
       
  The \var{set-position!} procedure (if supplied) should set the
  position of the input port to \var{pos} if \var{pos} is the return
  value of a call to \var{get-position}.  If not supplied, the custom
  port will not support the {\cf set-port-position!} operation.
       
\item {\cf (\var{close})}
       
  The \var{close} procedure (if supplied) should perform any actions
  that are necessary when the input port is closed.
\end{itemize}

\implresp The implementation is only required to check the return
values of \var{read!} and \var{get-position} when it actually calls
them as part of an I/O operation requested by the program.  The
implementation is not required to check that these procedures
otherwise behave as described.  If they do not, however, the behavior
of the resulting port is unspecified.
\end{entry}

\subsection{Binary input}

\begin{entry}{%
\proto{get-u8}{ binary-input-port}{procedure}}
   
Reads from \var{binary-input-port}, blocking as necessary, until data are
available from \var{binary-input-port} or until an end of file is reached.
If a byte becomes available, {\cf get-u8} returns the byte as an octet, and
updates \var{binary-input-port} to point just past that byte. If no input
byte is seen before an end of file is reached, then the end-of-file
object is returned.
\end{entry}

\begin{entry}{%
\proto{lookahead-u8}{ binary-input-port}{procedure}}
   
The {\cf lookahead-u8} procedure is like {\cf get-u8}, but it does not 
update \var{binary-input-port} to point past the byte.
\end{entry}

\begin{entry}{%
\proto{get-bytevector-n}{ binary-input-port k}{procedure}}
   
Reads from \var{binary-input-port}, blocking as necessary, until \var{k}
bytes are available from \var{binary-input-port} or until an end of file is
reached. If \var{k} or more bytes are available before an end
of file, {\cf get-bytevector-n} returns a bytevector of size \var{k}.
If fewer bytes are available before an end of file, {\cf get-bytevector-n}
returns a bytevector
containing those bytes. In either case, the input port is updated to
point just past the bytes read.  If an end of file is reached before
any bytes are available, {\cf get-bytevector-n} returns the end-of-file object.
\end{entry}

\begin{entry}{%
\pproto{(get-bytevector-n! \var{binary-input-port}}{procedure}}
\mainschindex{get-bytevector-n!}{\tt\obeyspaces\\
    \var{bytevector} \var{start} \var{count})}

\domain{\var{Count} must be an exact, non-negative integer, specifying
  the number of bytes to be read. \var{bytevector} must be a bytevector
  with at
  least $\var{start} + \var{count}$ elements.}
   
The {\cf get-bytevector-n!} procedure reads from \var{binary-input-port},
blocking as necessary, until
\var{count} bytes are available from \var{binary-input-port} or until
an end of file is
reached. If \var{count} or more bytes are available before an end of file,
they are written into \var{bytevector} starting at index \var{start}, and
the result is \var{count}. If fewer bytes are available before
the next end of file, the available bytes are written into \var{bytevector}
starting at index \var{start}, and the result is the number of bytes actually
read. In either case, the input port is updated to point just past the
data read. If an end of file is reached before any bytes
are available, {\cf get-bytevector-n!} returns the end-of-file object.
\end{entry}

\begin{entry}{%
\proto{get-bytevector-some}{ binary-input-port}{procedure}}
   
Reads from \var{binary-input-port}, blocking as necessary, until data are
available from \var{binary-input-port} or until an end of file is reached.
If data become available,
{\cf get-bytevector-some} returns a freshly allocated
bytevector of non-zero size containing the available data, and it updates
\var{binary-input-port} to point just past that data.
If no input bytes are seen before an end
of file is reached, then the end-of-file object is returned.
\end{entry}

\begin{entry}{%
\proto{get-bytevector-all}{ binary-input-port}{procedure}}
   
Attempts to read all data until the next end of file, blocking as
necessary. If one or more bytes are read, {\cf get-bytevector-all} returns
a bytevector
containing all bytes up to the next end of file.  Otherwise, {\cf
  get-bytevector-all} returns the end-of-file object.
Note that  {\cf get-bytevector-all}
may block indefinitely, waiting to see an end of
file, even though some bytes are available.
\end{entry}

\subsection{Textual input}

\begin{entry}{%
\proto{get-char}{ textual-input-port}{procedure}}
   
Reads from \var{textual-input-port}, blocking as necessary, until the
complete encoding for a character is available from \var{textual-input-port},
or until the available input data cannot be the prefix of any valid encoding,
or until an end of file is reached.

If a complete character is available before the next end of file, {\cf
  get-char} returns that character, and updates the input port to
point past the data that encoded that character. If an end of file is
reached before any data are read, then {\cf get-char} returns the
end-of-file object.
\end{entry}

\begin{entry}{%
\proto{lookahead-char}{ textual-input-port}{procedure}}
  
The {\cf lookahead-char} procedure is like {\cf get-char}, but it does not 
update \var{textual-input-port} to point past the data
that encode the character.

\begin{note}
  With some of the standard transcoders
  described in this document, up to four bytes of lookahead are
  required. Nonstandard transcoders may require even more lookahead.
\end{note}
\end{entry}

\begin{entry}{%
\proto{get-string-n}{ textual-input-port k}{procedure}}
   
Reads from \var{textual-input-port}, blocking as necessary, until the
encodings of \var{k} characters (including invalid encodings, if they
don't raise an exception) are available, or until an end of
file is reached.
   
If \var{k} or more characters are read before end of file, {\cf
  get-string-n} returns a string consisting of those \var{k}
characters. If fewer characters are available before an end of file,
but one or more characters can be read,
{\cf get-string-n} returns a string containing
those characters. In either case, the input port is updated to point
just past the data read. If no data can be read before an 
end of file, then the end-of-file object is returned.
\end{entry}

\begin{entry}{%
\proto{get-string-n!}{ textual-input-port string start count}{procedure}}

\domain{\var{Start} and \var{count} must be an exact, non-negative
  integer, specifying the number of characters to be read.
  \var{string} must be a string with at least $\var{start} +
  \var{count}$ characters.}

Reads from \var{textual-input-port} in the same manner as {\cf
  get-string-n}.  If \var{count} or more characters are available
before an end of file, they are written into string
starting at index \var{start}, and \var{count} is returned. If fewer
characters are available before an end of file, but one
or more can be read, then those characters are written into string
starting at index \var{start}, and the number of characters actually read is
returned. If no characters can be read before an end of file,
then the end-of-file object is returned.
\end{entry}   

\begin{entry}{%
\proto{get-string-all}{ textual-input-port}{procedure}}
   
Reads from \var{textual-input-port} until an end of file, decoding
characters in the same manner as {\cf get-string-n} and {\cf get-string-n!}.
   
If data are available before the end of file, a string
containing all the text decoded from that data are returned. If no data
precede the end of file, the end-of-file object file object is
returned.
\end{entry}

\begin{entry}{%
\proto{get-line}{ textual-input-port}{procedure}}
   
Reads from \var{textual-input-port} up to and including the linefeed
character or end of file, decoding characters in the same manner as
{\cf get-string-n} and {\cf get-string-n!}.
   
If a linefeed character is read, then a string
containing all of the text up to (but not including) the linefeed
character is returned, and the port is updated to point just past the
linefeed character. If an end of file is
encountered before any linefeed character is read, but some data
have been read and decoded as characters, then a string containing
those characters is returned. If an end of file is encountered before
any data are read, then the end-of-file object is
returned.

\begin{note}
  The end-of-line style, if not {\cf none}, will cause all line
  endings to be read as linefeed characters.  See
  section~\ref{transcoderssection}.
\end{note}
\end{entry}

\begin{entry}{%
\proto{get-datum}{ textual-input-port}{procedure}}
 
Reads an external representation from \var{textual-input-port} and returns the
datum it represents.  The {\cf get-datum} procedure returns the next
datum that can be parsed from the given \var{textual-input-port}, updating
\var{textual-input-port} to point exactly past the end of the external
representation of the object.

Any \meta{interlexeme space}
(see report section~\extref{report:lexicalsyntaxsection}{Lexical syntax}) in
the input is first skipped.  If an end of file occurs after the
\meta{interlexeme space}, the end of file object (see
section~\ref{eofsection}) is returned.

If a character inconsistent with an external representation is
encountered in the input, an exception with condition types
{\cf\&lexical} and {\cf\&i/o-read} is raised.
Also, if the end of file is encountered
after the beginning of an external representation, but the external
representation is incomplete and therefore cannot be parsed, an exception
with condition types {\cf\&lexical} and {\cf\&i/o-read} is raised.
\end{entry}

\subsection{Output ports}

An output port is a sink to which bytes or characters are written.
These data may control
external devices, or may produce files and other objects that may
subsequently be opened for input.

\begin{entry}{%
\proto{output-port?}{ obj}{procedure}}
   
Returns \schtrue{} if the argument is an output port (or a
combined input and output port), and returns \schfalse{} otherwise.
\end{entry}   

\begin{entry}{%
\proto{flush-output-port}{ output-port}{procedure}}
   
Flushes any output from the buffer of \var{output-port} to the
underlying file, device, or object. The {\cf flush-output-port}
procedure returns \unspecifiedreturn.
\end{entry}

\begin{entry}{%
\proto{output-port-buffer-mode}{ output-port}{procedure}}
   
Returns the symbol that represents the buffer-mode of
\var{output-port}.
\end{entry}

\begin{entry}{%
\proto{open-file-output-port}{ filename}{procedure}
\rproto{open-file-output-port}{ filename file-options}{procedure}
\pproto{(open-file-output-port \var{filename}}{procedure}{\tt\obeyspaces%
\hspace*{2em}\var{file-options} \var{buffer-mode})}\\
\pproto{(open-file-output-port \var{filename}}{procedure}{\tt\obeyspaces%
\hspace*{2em}\var{file-options} \var{buffer-mode} \var{maybe-transcoder})}}

\domain{\var{Maybe-transcoder} must be either a transcoder or \schfalse.}

The {\cf open-file-output-port} procedure returns an output port for the named file.

The \var{file-options} argument, which may determine
various aspects of the returned port (see section~\ref{fileoptionssection}),
defaults to {\cf (file-options)}.

The \var{buffer-mode} argument, if supplied,
must be one of the symbols that name a buffer mode.
The \var{buffer-mode} argument defaults to {\cf block}.

If \var{maybe-transcoder} is a transcoder, it becomes the transcoder
associated with the port.

If \var{maybe-transcoder} is \schfalse{} or absent,
then the port will be a binary port and will support the
{\cf port-position} and {\cf set-port-position!}  operations.
Otherwise the port will be a textual port, and whether it supports
the {\cf port-position} and {\cf set-port-position!} operations
will be implementation-dependent (and possibly transcoder-dependent).

\begin{rationale}
  The byte position of a transcoded port may not be
  well-defined, and may be hard to calculate even when defined,
  especially when transcoding is buffered.
\end{rationale}
\end{entry}   

\begin{entry}{%
\proto{open-bytevector-output-port}{}{procedure}
\rproto{open-bytevector-output-port}{ maybe-transcoder}{procedure}}

\domain{\var{Maybe-transcoder} must be either a transcoder or \schfalse.}

The {\cf open-bytevector-output-port} procedure returns 
two values: an output port and a procedure.
The output port accumulates the data written to it for
later extraction by the procedure.

If \var{maybe-transcoder} is a transcoder, it becomes
the transcoder associated with the port.
If no \var{maybe-transcoder} argument is given, or
if it is \schfalse,
then the port will be a binary port and will support the
{\cf port-position} and {\cf set-port-position!}  operations.
Otherwise the port will be a textual port, and whether it supports
the {\cf port-position} and {\cf set-port-position!} operations
will be implementation-dependent (and possibly transcoder-dependent).

When the procedure is called without arguments, it returns a
bytevector consisting of all the port's accumulated data and removes
the accumulated data from the port.

\begin{note}
  When the returned procedure is called without arguments, the
  returned data is independent of the port's current position.
\end{note}
\end{entry}

\begin{entry}{%
\proto{call-with-bytevector-output-port}{ proc}{procedure}
\pproto{(call-with-bytevector-output-port \var{proc}}{procedure}{\tt\obeyspaces%
\hspace*{2em}\var{maybe-transcoder})}}

\domain{\var{Proc} must be a procedure accepting one argument.
  \var{Maybe-transcoder} must be either a transcoder or \schfalse.}

The {\cf call-with-bytevector-output-port} procedure creates an output
port that accumulates the data written to it and calls \var{proc} with
that output port as an argument. Whenever \var{proc} returns, a
bytevector consisting of the port's accumulated data is returned and
the port is closed.

The transcoder associated with the output port is determined
as for a call to {\cf open-bytevector-output-port}.
\end{entry}

\begin{entry}{%
\proto{open-string-output-port}{}{procedure}}

Returns two values: a textual output port and a procedure.
The output port accumulates the characters written to it for
later extraction by the procedure.

The port has an associated transcoder, which is
implementation-dependent.  The port should support the {\cf
  port-position} and {\cf set-port-position!} operations.

When the procedure is called without arguments, it returns a string consisting of the port's
accumulated characters and removes the accumulated characters from the port.
\end{entry}

\begin{entry}{%
\proto{call-with-string-output-port}{ proc}{procedure}}

\domain{\var{Proc} must be a procedure accepting one argument.}
Creates a textual output port that accumulates the
characters written to it and calls \var{proc} with that output port
as an argument. Whenever \var{proc} returns, a string consisting of the
port's accumulated characters is returned and the port is closed.

The port has an associated transcoder, which is implementation-dependent.
The port should support
the {\cf port-position} and {\cf set-port-position!} operations.
\end{entry}

\begin{entry}{%
\proto{standard-output-port}{}{procedure}
\proto{standard-error-port}{}{procedure}}
   
Returns a fresh binary output port connected to the standard output or
standard error respectively.  Whether the port supports the {\cf
  port-position} and {\cf set-port-position!} operations is
implementation-dependent.
\end{entry}

\begin{entry}{%
\proto{current-output-port}{}{procedure}
\proto{current-error-port}{}{procedure}}
 
These return default textual ports for regular output and error output.
Normally, these default ports are associated with standard output, and
standard error, respectively.  The return value of {\cf
  current-output-port} can be dynamically re-assigned using the {\cf
  with-output-to-file} procedure from the \rsixlibrary{i/o simple}
library (see section~\ref{with-output-to-file}).
\end{entry}


\begin{entry}{%
\pproto{(make-custom-binary-output-port \var{id}}{procedure}}
\mainschindex{make-custom-binary-output-port}{\tt\obeyspaces\\
  \var{write!} \var{get-position} \var{set-position!} \var{close})}

Returns a newly created binary output port whose byte sink is
an arbitrary algorithm represented by the \var{write!} procedure.
\var{Id} must be a string naming the new port,
provided for informational purposes only.
\var{Write!} must be a procedure, and should behave as specified
below; it will be called by operations that perform binary output.

Each of the remaining arguments may be \schfalse{}; if any of
those arguments is not \schfalse{}, it must be a procedure and
should behave as specified in the description of
{\cf make-custom-binary-input-port}.
   
\begin{itemize}
\item {\cf (\var{write!} \var{bytevector} \var{start} \var{count})}
       
  \domain{\var{Start} and \var{count} will be non-negative exact integers,
  and \var{bytevector} will be a bytevector whose length is at least
  $\var{start} + \var{count}$.}
  The \var{write!} procedure should read up to \var{count} bytes
  from \var{bytevector} starting at index \var{start}, and forward
  them to the byte sink.
  If \var{count} is 0, then the \var{write!} procedure should
  have the effect of passing an end-of-file object to the byte sink.
  In any case, the \var{write!} procedure should return the number of
  bytes that it wrote, as an exact integer.
\end{itemize}

\implresp The implementation is only required to check the return
values of \var{write!} when it actually calls \var{write!} as part of
an I/O operation requested by the program.  The implementation is not
required to check that \var{write!} otherwise behaves as described.
If it does not, however, the behavior of the resulting port is
unspecified.
\end{entry}

\begin{entry}{%
\pproto{(make-custom-textual-output-port \var{id}}{procedure}}
\mainschindex{make-custom-textual-output-port}{\tt\obeyspaces\\
  \var{write!} \var{get-position} \var{set-position!} \var{close})}

Returns a newly created textual output port whose byte sink is
an arbitrary algorithm represented by the \var{write!} procedure.
\var{Id} must be a string naming the new port,
provided for informational purposes only.
\var{Write!} must be a procedure, and should behave as specified
below; it will be called by operations that perform textual output.

Each of the remaining arguments may be \schfalse{}; if any of
those arguments is not \schfalse{}, it must be a procedure and
should behave as specified in the description of
{\cf make-custom-textual-input-port}.
   
\begin{itemize}
\item {\cf (\var{write!} \var{string} \var{start} \var{count})}
       
  \domain{\var{Start} and \var{count} will be non-negative exact integers,
  and \var{string} will be a string whose length is at least
  $\var{start} + \var{count}$.}
  The \var{write!} procedure should read up to \var{count} characters
  from \var{string} starting at index \var{start}, and forward
  them to the character sink.
  If \var{count} is 0, then the \var{write!} procedure should
  have the effect of passing an end-of-file object to the character sink.
  In any case, the \var{write!} procedure should return the number of
  characters that it wrote, as an exact integer.
\end{itemize}

\implresp The implementation is only required to check the return
values of \var{write!} when it actually calls \var{write!} as part of
an I/O operation requested by the program.  The implementation is not
required to check that \var{write!} otherwise behaves as described.
If it does not, however, the behavior of the resulting port is
unspecified.
\end{entry}

\subsection{Binary output}

\begin{entry}{%
\proto{put-u8}{ binary-output-port octet}{procedure}}

Writes \var{octet} to the output port and returns \unspecifiedreturn.
\end{entry}

\begin{entry}{%
\proto{put-bytevector}{ binary-output-port bytevector}{procedure}
\rproto{put-bytevector}{ binary-output-port bytevector start}{procedure}
\pproto{(put-bytevector \var{binary-output-port}}{procedure}}
{\tt\obeyspaces\\
     \var{bytevector} \var{start} \var{count})}
   
\domain{\var{Start} and \var{count} must be non-negative exact
  integers that default to 0 and $\texttt{(bytevector-length \var{bytevector})}
  - \var{start}$, respectively. \var{bytevector} must have a length of at
  least $\var{start} + \var{count}$.}  The {\cf put-bytevector} procedure writes
\var{count} bytes of the bytevector \var{bytevector}, starting at index
\var{start}, to the output port.  The {\cf put-bytevector} procedure
returns \unspecifiedreturn.
\end{entry}

\subsection{Textual output}

\begin{entry}{%
\proto{put-char}{ textual-output-port char}{procedure}}
   
Writes \var{char} to the port. The {\cf put-char} procedure returns
\unspecifiedreturn.
\end{entry}

\begin{entry}{%
\proto{put-string}{ textual-output-port string}{procedure}
\rproto{put-string}{ textual-output-port string start}{procedure}
\rproto{put-string}{ textual-output-port string start count}{procedure}}
   
\domain{\var{Start} and \var{count} must be non-negative exact
  integers.  \var{String} must have a length of at least $\var{start} +
  \var{count}$.}  \var{Start} defaults to 0.  \var{Count} defaults to
$\texttt{(string-length \var{string})} - \var{start}$. Writes the
\var{count} characters of \var{string}, starting at
index \var{start}, to the port.  The {\cf put-string-n} procedure
returns \unspecifiedreturn.
\end{entry}


\begin{entry}{%
\proto{put-datum}{ textual-output-port datum}{procedure}}

\domain{\var{Datum} should be a datum value.}
The {\cf put-datum} procedure writes an external representation of
\var{datum} to \var{textual-output-port}.
The specific external representation is implementation-dependent.

\begin{note}
  The {\cf put-datum} procedure merely writes the external
  representation, but no trailing delimiter.  If {\cf put-datum} is
  used to write several subsequent external representations to an
  output port, care should be taken to delimit them properly so they can
  be read back in by subsequent calls to {\cf get-datum}.
\end{note}
\end{entry}


\subsection{Input/output ports}

\begin{entry}{%
\proto{open-file-input/output-port}{ filename}{procedure}
\rproto{open-file-input/output-port}{ filename file-options}{procedure}
\pproto{(open-file-input/output-port \var{filename}}{procedure}{\tt\obeyspaces%
\hspace*{2em}\var{file-options} \var{buffer-mode})}\\
\pproto{(open-file-input/output-port \var{filename}}{procedure}{\tt\obeyspaces%
\hspace*{2em}\var{file-options} \var{buffer-mode} \var{transcoder})}}
   
Returns a single port that is both an input port and an
output port for the named file.
The optional arguments default as described in the specification
of {\cf open-file-output-port}.
If the input/output port supports {\cf port-position} and/or
{\cf set-port-position!}, then the same port position is used
for both input and output.
\end{entry}

\begin{entry}{%
\pproto{(make-custom-binary-input/output-port}{procedure}}
\mainschindex{make-custom-binary-input/output-port}{\tt\obeyspaces\\
  \var{id} \var{read!} \var{write!} \var{get-position} \var{set-position!} \var{close})}

Returns a newly created binary input/output port whose
byte source and sink are
arbitrary algorithms represented by the \var{read!} and \var{write!}
procedures.
\var{Id} must be a string naming the new port,
provided for informational purposes only.
\var{Read!} and \var{write!} must be procedures,
and should behave as specified for the
{\cf make-custom-binary-input-port} and
{\cf make-custom-binary-output-port} procedures.

Each of the remaining arguments may be \schfalse{}; if any of
those arguments is not \schfalse{}, it must be a procedure and
should behave as specified in the description of
{\cf make-custom-binary-input-port}.
\end{entry}


%%% Local Variables: 
%%% mode: latex
%%% TeX-master: "r6rs-lib"
%%% End: 

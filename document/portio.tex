\section{Port I/O}
\label{portsiosection}

The \deflibrary{r6rs i/o ports} library defines an I/O layer for
conventional, imperative buffered input and output with mixed text and
binary data.  A \defining{port} represents a buffered access object
for a data sink or source or both simultaneously.  The library allows
creating ports from arbitrary input sources and sinks represented
as readers and writers (see section~\ref{primiosection}), but does not
require that all ports be built from readers and writers.

The \library{r6rs i/o ports} library distinguishes between \textit{input
  ports\mainindex{input port}} and \textit{output
  ports\mainindex{output ports}}.  An input port is a source for data,
whereas an output port is a sink for data.  A port may be both an
input port and an output port; such a port typically provides
simultaneous read and write access to a file or other data.

This section uses the \var{input-port}, \var{output-port}, and \var{port}
parameter names for arguments that must be input ports (or combined
input/output ports), output ports (or combined input/output ports),
or any kind of port.

\subsection{Condition type}

This library introduces the following condition type:

\begin{entry}{%
\ctproto{i/o-port}
\proto{i/o-port-error?}{ obj}{procedure}
\proto{i/o-error-port}{ condition}{procedure}}

This condition type could be defined by
%
\begin{scheme}
(define-condition-type \&i/o-port \&i/o
  i/o-port-error?
  (port i/o-error-port))
\end{scheme}

This condition type allows specifying with what particular port an I/O
error is associated. Except for condition objects provided for
encoding and decoding errors, conditions raised by procedures may
include an {\cf\&i/o-port-error} condition, but are not required to do
so.
\end{entry}

\subsection{Buffer modes}

Each output port has an associated buffer mode that defines when an
output operation flushes the buffer associated with the output
port. The possible buffer modes are the symbols {\cf none} for no buffering,
{\cf line} for flushing upon line feeds and line separators (U+2028), and
{\cf block} for arbitrary buffering.  This section uses the parameter name
\var{buffer-mode} for arguments that must be buffer-mode symbols.

\begin{entry}{%
\proto{buffer-mode}{ \hyper{name}}{\exprtype}}
   
\hyper{Name} must be one of the \meta{identifier}s {\cf none}, {\cf line}, or
{\cf block}. The result is the corresponding symbol, denoting the
associated buffer mode.

It is a syntax violation if \hyper{name} is not one of the valid
identifiers.
\end{entry}

\begin{entry}{%
\proto{buffer-mode?}{ obj}{procedure}}
   
Returns \schtrue{} if the argument is a valid buffer-mode symbol,
\schfalse{} otherwise.
\end{entry}

\subsection{Text transcoders}

Several different Unicode encoding schemes describe standard ways to
encode characters and strings as byte sequences and to decode those
sequences~\cite{Unicode}.  Within this document, a \defining{codec} is a Scheme object
that represents a Unicode or similar encoding scheme.

The text transcoders of this document generalize codecs
to deal with common end-of-line conventions and different
error-handling modes.

A \defining{transcoder} is an opaque object that represents
some specific bidirectional (but not necessarily lossless) translation
between byte sequences and Unicode characters and strings.  The
transcoder specifies how procedures that perform textual input are to
interpret input bytes as characters or strings.  The transcoder also
specifies how procedures that perform textual output are to translate
characters into bytes.  Moreover, the transcoder specifies a
mode for handling encoding or decoding errors.

A \var{transcoder} parameter name means that the corresponding
argument must be a transcoder.

\begin{entry}{%
\proto{utf-8-codec}{}{procedure}
\proto{latin-1-codec}{}{procedure}
\proto{utf-16le-codec}{}{procedure}
\proto{utf-16be-codec}{}{procedure}
\proto{utf-32le-codec}{}{procedure}
\proto{utf-32be-codec}{}{procedure}}

These are predefined codecs for the UTF-8, ISO8859-1,
UTF-16LE, UTF-16BE, UTF32-LE, and UTF-32BE encoding
schemes.

A call to any of these procedures returns a value that is equal in the
sense of {\cf eqv?} to the result of any other call to the same
procedure.
\end{entry}

\begin{entry}{%
\proto{eol-style}{ name}{\exprtype}}

If \var{name} is one of the \meta{identifier}s {\cf lf}, {\cf cr},
{\cf crlf}, or {\cf ls}, then the form evaluates to the corresponding
symbol.  If \var{name} is not one of these identifiers, effect and
result are implementation-dependent: The result may be an
eol-style symbol acceptable as an \var{eol-mode}
argument to {\cf make-transcoder}.  Otherwise, an exception is raised.

\begin{rationale}
  End-of-line styles other than those listed might become commonplace
  in the future.
\end{rationale}
\end{entry}

\begin{entry}{%
\proto{native-eol-style}{}{procedure}}

Returns the default end-of-line style of the underlying platform, e.g.
{\cf lf} on Unix and {\cf crlf} on Windows.
\end{entry}

\begin{entry}{%
\ctproto{i/o-decoding}
\proto{i/o-decoding-error?}{ obj}{procedure}}

\begin{scheme}
(define-condition-type \&i/o-decoding \&i/o-port
  i/o-decoding-error?
  (transcoder i/o-decoding-error-transcoder))
\end{scheme}

An exception with this type is raised when one of the operations for
textual output to a port encounters a character not supported by the
codec of the specified transcoder.  The {\cf transcoder} field
contains the transcoder specified with the operation.

Exceptions of this type raised by the operations described in this
section are continuable.  When such an exception is raised, the port's position is at
the beginning of the invalid encoding.  The exception handler must
return a character or string representing the decoded text starting at
the port's current position, and the exception handler must update the 
port's position to point past the error.
\end{entry}

\begin{entry}{% 
\ctproto{i/o-encoding}
\proto{i/o-encoding-error?}{ obj}{procedure}
\proto{i/o-encoding-error-char}{ condition}{procedure}
\proto{i/o-encoding-error-transcoder}{ condition}{procedure}}

This condition type could be defined by
%
\begin{scheme}
(define-condition-type \&i/o-encoding \&i/o-port
  i/o-encoding-error?
  (char i/o-encoding-error-char)
  (transcoder i/o-encoding-error-transcoder))
\end{scheme}

An exception with this type is raised when one of the operations for
textual output to a port encounters a character not supported by the
codec of the specified transcoder.  The {\cf char} field of the
condition object contains the character that the operation was unable
to encode, and the {\cf transcoder} field contains the transcoder
specified with the operation.

Exceptions of this type raised by the operations described in this
section are continuable.  The handler, if it returns, is expected to
output to the port an appropriate encoding for the character that
caused the error.  The operation that raised the exception 
continues after that character.
\end{entry}

\begin{entry}{%
\proto{error-handling-mode}{ name}{\exprtype}}

If \var{name} is one of the \meta{identifier}s {\cf ignore}, {\cf
  raise}, or {\cf replace}, then the result is the corresponding
symbol.  If \var{name} is not one of these identifiers, effect and
result are implementation-dependent: The result may be an
error-handling-mode symbol acceptable as a \var{handling-mode}
argument to {\cf make-transcoder}.  Otherwise, an exception is raised.

\begin{rationale}
  Implementations may support error-handling modes other than those
  listed.
\end{rationale}

When part of a transcoder, an error-handling mode specifies the behavior
of a text I/O operation in the presence of an encoding or decoding
error:

If a codec for an input port encounters an invalid or incomplete
character encoding, it behaves according to the specified
error-handling mode. If it is {\cf ignore}, the first byte of the
invalid encoding is ignored and decoding continues with the next
byte.  If it is {\cf replace}, the replacement character U+FFFD is
injected into the data stream.  Decoding subsequently continues
with the next byte. If it is {\cf raise}, a continuable exception with
condition type {\cf\&i/o-decoding} is raised.  See the description of
{\cf\&i/o-decoding} for details
on how to handle such an exception.

If a codec for an output port encounters a character it cannot encode,
it behaves according to the specified error-handling mode. If it
is {\cf ignore}, the character is ignored and encoding continues after
the encoding. If it is {\cf replace}, an encoding-specific replacement
character is emitted by the transcoder, and decoding continues
after the encoding. This replacement character is U+FFFD for the
Unicode encodings capable of representing it, and the {\cf ?}
character for the Latin-1 encoding. If the mode is {\cf raise}, an
exception with condition type {\cf\&i/o-encoding} is raised.
See the description of
{\cf\&i/o-decoding} for details
on how to handle such an exception.
\end{entry}

\begin{entry}{%
\proto{make-transcoder}{ codec eol-style handling-mode}{procedure}
\rproto{make-transcoder}{ codec eol-style}{procedure}
\rproto{make-transcoder}{ codec}{procedure}}

\domain{\var{Codec} must be a codec; \var{eol-style}, if present, an
  eol-style symbol; and \var{handling-mode}, if present, an
  error-handling-mode symbol.}  \var{eol-style} may be omitted, in
which case it defaults to the native end-of-line style of the
underlying platform.  \var{handling-mode} may be omitted, in which
case it defaults to {\cf raise}.  The result is a transcoder with the
behavior specified by its arguments.

A transcoder returned by {\cf make-transcoder} is equal in the sense
of {\cf eqv?} to any other transcoder returned by {\cf
  make-transcoder}, if and only if the \var{code}, \var{eol-style},
and \var{handling-mode} arguments are equal in the sense of {\cf
  eqv?}.
\end{entry}

\begin{entry}{%
\proto{transcoder-codec}{ transcoder}{procedure}
\proto{transcoder-eol-style}{ transcoder}{procedure}
\proto{transcoder-error-handling-mode}{ transcoder}{procedure}}

These are accessors for transcoder objects; when applied to a
transcoder returned by {\cf make-transcoder}, they return the
\var{codec}, \var{eol-style}, \var{handling-mode} arguments.
\end{entry}

\subsection{Input and output ports}

The operations described in this section are common to input and
output ports.  A port may have an associated \defining{position} that
specifies a particular place within its data sink or source as a byte
count from the beginning of the sink or source, and operations for
inspecting and setting it.  (Ends of file do not count as bytes.)
A port may also have an associated
transcoder that represents a default text encoding associated with the
port.  Note that the transcoder associated with a port does not have
any direct effect on the behavior of procedures that perform
textual I/O except for {\cf get-output-string} and {\cf
  call-with-string-output-port}.
However,
it can be passed to those procedures to specify that the text encoding
or decoding should happen according to that transcoder.

\begin{entry}{%
\proto{port?}{ obj}{procedure}}
   
Returns \schtrue{} if the argument is a port, and returns \schfalse{}
otherwise.
\end{entry}

\begin{entry}{%
\proto{port-transcoder}{ port}{procedure}}

Returns the transcoder associated with \var{port}, if it has one,
or \schfalse{} if no transcoder is associated with \var{port}.
\end{entry}

\begin{entry}{%
\proto{port-has-port-position?}{ port}{procedure}
\proto{port-position}{ port}{procedure}}

The {\cf port-has-port-position?} procedure returns \schtrue{} if the
port supports the {\cf port-position} operation, and \schfalse{}
otherwise.

The {\cf port-position} procedure
returns the exact non-negative integer index of the position at which the
next byte would be read from or written to the port.
This procedure raises an exception with condition type {\cf\&contract}
if the port does not support the operation.
\end{entry}   

\begin{entry}{%
\proto{port-has-set-port-position!?}{ port}{procedure}
\proto{set-port-position!}{ port pos}{procedure}}

\domain{\var{Pos} must be a non-negative exact integer.}
   
The {\cf port-has-set-port-position?} procedure returns \schtrue{} if the port
supports the {\cf set-port-position!} operation, and \schfalse{}
otherwise.
   
The {\cf set-port-position!} procedure sets the current byte position
of the port to \var{pos}.  If \var{port} is an output or combined
input and output port, this first flushes \var{port}.  (See {\cf
  flush-output-port}, section~\ref{flush-output-port}.)
This procedure raises an exception with condition type {\cf\&contract}
if the port does not support the operation.
\end{entry}

\begin{entry}{%
\proto{close-port}{ port}{procedure}}
   
Closes the port, rendering the port incapable of delivering or
accepting data. If \var{port} is an output port, it is flushed before
being closed.  This has no effect if the port has already been closed.
A closed port is still a port. The unspecified value is returned.
\end{entry}

\begin{entry}{%
\proto{call-with-port}{ port proc}{procedure}}
   
\domain{\var{Proc} must be a procedure that accepts a single
  argument.}  The {\cf call-with-port} procedure
calls \var{proc} with \var{port} as an argument. If
\var{proc} returns, then the \var{port} is closed automatically and
the values returned by \var{proc} are returned. If \var{proc} does not
return, then the port is not closed automatically, except perhaps when it is
possible to prove that the port will never again be used for a
{\cf lookahead}, {\cf get}, or {\cf put} operation.
\end{entry}

\subsection{Input ports}

An input port allows reading an infinite sequence of bytes punctuated
by end of file objects. An input port connected to a finite data
source ends in an infinite sequence of end of file objects.  All of
the procedures that perform textual input accept a transcoder as an
optional argument. If no transcoder is supplied or the
\var{transcoder} argument is \schfalse{}, the input bytes are
interpreted as UTF-8 with a platform-specific end-of-line convention.

It is unspecified whether a character encoding consisting of several
bytes may have an end of file between the bytes.  If, for example,
{\cf get-char} raises an {\cf\&i/o-decoding} exception because the
character encoding at the port's position is incomplete up to the next
end of file, a subsequent call to {\cf get-char} may successfully
decode a character if bytes completing the encoding are available
after the end of file.

\begin{entry}{%
\proto{input-port?}{ obj}{procedure}}

Returns \schtrue{} if the argument is an input port (or combined input
and output port), and returns \schfalse{} otherwise.
\end{entry}

\begin{entry}{%
\proto{port-eof?}{ input-port}{procedure}}
   
Returns \schtrue{} if the {\cf lookahead-u8} procedure would return
the end-of-file object, and returns \schfalse{} otherwise.
\end{entry}

\begin{entry}{%
\proto{open-file-input-port}{ filename}{procedure}
\rproto{open-file-input-port}{ filename file-options}{procedure}
\rproto{open-file-input-port}{ filename file-options transcoder}{procedure}}
   
Returns an input port for the named file. The file-options object,
which may determine
various aspects of the returned port (see section~\ref{fileoptionssection}),
defaults to {\cf (file-options)}.

The returned input port supports the {\cf port-position} and {\cf
  set-port-position!}  operations.

If \var{transcoder} is specified, it becomes the transcoder associated
with the returned port.
\end{entry}

\begin{entry}{%
\proto{open-bytes-input-port}{ bytes}{procedure}
\rproto{open-bytes-input-port}{ bytes transcoder}{procedure}}
   
Returns an input port whose bytes are drawn from the
bytes object \var{bytes}.
If \var{transcoder} is specified, it becomes the transcoder associated
with the returned port.

If \var{bytes} is modified after {\cf open-bytes-input-port}
has been called, the effect on the returned
port is unspecified.
\end{entry}

\begin{entry}{%
\proto{open-string-input-port}{ string}{procedure}
\rproto{open-string-input-port}{ string transcoder}{procedure}}

Returns an input port whose bytes are drawn from an encoding of
\var{string}.  If \var{transcoder} is not specified, that encoding is
UTF-8.  If \var{transcoder} is specified, the encoding is according to
\var{transcoder}.  Also, \var{transcoder} becomes the transcoder
associated with the returned port.
If \var{string} is modified after {\cf open-string-input-port}
has been called, the effect on the returned port is unspecified.
\end{entry}

\begin{entry}{%
\proto{standard-input-port}{}{procedure}}
   
Returns an input port connected to standard input, possibly a fresh
one on each call.  If the returned port supports textual input, it has
an associated transcoder with some encoding, some end-of-line style,
and error-handling mode {\cf raise}.

\begin{note}
  Implementations are encouraged to provide a transcoder appropriate
  for reading text from the port.
\end{note}
\end{entry}

\begin{entry}{%
\proto{get-u8}{ input-port}{procedure}}
   
Reads from \var{input-port}, blocking as necessary, until data is
available from \var{input-port} or until an end of file is reached. If a
byte becomes available, {\cf get-u8} returns the byte as an octet, and it
updates \var{input-port} to point just past that byte. If no input
byte is seen before an end of file is reached, then the end-of-file
object is returned.
\end{entry}

\begin{entry}{%
\proto{lookahead-u8}{ input-port}{procedure}}
   
The {\cf lookahead-u8} procedure is like {\cf get-u8}, but it does not 
update \var{input-port} to point past the byte.
\end{entry}

\begin{entry}{%
\proto{get-bytes-n}{ input-port k}{procedure}}
   
Reads from \var{input-port}, blocking as necessary, until \var{k}
bytes are available from \var{input-port} or until an end of file is
reached. If \var{k} or more bytes are available before an end
of file, {\cf get-bytes-n} returns a bytes object of size \var{k}. If fewer
bytes are available before an end of file, {\cf get-bytes-n} returns a bytes object
containing those bytes. In either case, the input port is updated to
point just past the bytes read.  If an end of file is reached before
any bytes are available, {\cf get-bytes-n} returns the end-of-file object.
\end{entry}

\begin{entry}{%
\proto{get-bytes-n!}{ input-port bytes start count}{procedure}}
   
\domain{\var{Count} must be an exact, non-negative integer, specifying
  the number of bytes to be read. \var{bytes} must be a bytes object with at
  least $\var{start} + \var{count}$ elements.}
   
The {\cf get-bytes-n!} procedure reads from \var{input-port}, blocking as necessary, until
\var{count} bytes are available from \var{input-port} or until an end of file is
reached. If \var{count} or more bytes are available before an end of file,
they are written into \var{bytes} starting at index \var{start}, and
the result is \var{count}. If fewer bytes are available before
the next end of file, the available bytes are written into \var{bytes}
starting at index \var{start}, and the result is the number of bytes actually
read. In either case, the input port is updated to point just past the
data read. If an end of file is reached before any bytes
are available, {\cf get-bytes-n!} returns the end-of-file object.
\end{entry}

\begin{entry}{%
\proto{get-bytes-some}{ input-port}{procedure}}
   
Reads from \var{input-port}, blocking as necessary, until data is
available from \var{input-port} or until an end of file is reached.  If data
becomes available, {\cf get-bytes-some} returns a freshly allocated bytes object of
non-zero size containing the available data, and it updates \var{input-port} to
point just past that data.  If no input bytes are seen before an end
of file is reached, then the end-of-file object is returned.
\end{entry}

\begin{entry}{%
\proto{get-bytes-all}{ input-port}{procedure}}
   
Attempts to read all data until the next end of file, blocking as
necessary. If one or more bytes are read, {\cf get-bytes-all} returns a bytes object
containing all bytes up to the next end of file.  Otherwise, {\cf
  get-bytes-all} returns the end-of-file object. Note that  {\cf get-bytes-all}
may block indefinitely, waiting to see an end of
file, even though some bytes are available.
\end{entry}

\begin{entry}{%
\proto{get-char}{ input-port}{procedure}
\rproto{get-char}{ input-port transcoder}}
   
Reads from \var{input-port}, blocking as necessary, until the
complete encoding for a character is available from input-port, or until
the bytes that are available cannot be the prefix of any valid encoding,
or until an end of file is reached.

If a complete character is available before the next end of file, {\cf
  get-char} returns that character, and it updates the input port to
point past the bytes that encoded that character. If an end of file is
reached before any bytes are read, then {\cf get-char} returns the
end-of-file object.
\end{entry}

\begin{entry}{%
\proto{lookahead-char}{ input-port}{procedure}
\rproto{lookahead-char}{ input-port transcoder}{procedure}}
  
The {\cf lookahead-char} procedure is like {\cf get-char}, but it does not 
update \var{input-port} to point past the bytes that encode the character.

\begin{note}
  With some of the standard transcoders
  described in this document, up to eight bytes of lookahead are
  required. Nonstandard transcoders may require even more lookahead.
\end{note}
\end{entry}

\begin{entry}{%
\proto{get-string-n}{ input-port k}{procedure}
\rproto{get-string-n}{ input-port k transcoder}{procedure}}
   
Reads from \var{input-port}, blocking as necessary, until the
encodings of \var{k} characters (including invalid encodings, if they
don't raise an exception) are available, or until an end of
file is reached.
   
If \var{k} or more characters are read before end of file, {\cf
  get-string-n} returns a string consisting of those \var{k}
characters. If fewer characters are available before an end of file,
but one or more characters can be read, {\cf get-string-n} returns a string containing
those characters. In either case, the input port is updated to point
just past the data read. If no bytes can be read before an 
end of file, then the end-of-file object is returned.
\end{entry}

\begin{entry}{%
\proto{get-string-n!}{ input-port string start count}{procedure}
\rproto{get-string-n!}{ input-port string start count transcoder}{procedure}}

\domain{\var{Start} and \var{count} must be an exact, non-negative
  integer, specifying the number of characters to be read.
  \var{string} must be a string with at least $\var{start} +
  \var{count}$ characters.}

Reads from \var{input-port} in the same manner as {\cf
  get-string-n}.  If \var{count} or more characters are available
before an end of file, they are written into string
starting at index \var{start}, and \var{count} is returned. If fewer
characters are available before an end of file, but one
or more can be read, then those characters are written into string
starting at index \var{start}, and the number of characters actually read is
returned. If no characters can be read before an end of file,
then the end-of-file object is returned.
\end{entry}   

\begin{entry}{%
\proto{get-string-all}{ input-port}{procedure}
\rproto{get-string-all}{ input-port transcoder}{procedure}}
   
Reads from \var{input-port} until an end of file, decoding
characters in the same manner as {\cf get-string-n} and {\cf get-string-n!}.
   
If data are available before the end of file, a string
containing all the text decoded from that data is returned. If no data
precede the end of file, the end-of-file object file object is
returned.
\end{entry}

\begin{entry}{%
\proto{get-line}{ input-port}{procedure}
\rproto{get-line}{ input-port transcoder}{procedure}}
   
Reads from \var{input-port} up to and including the next
end-of-line encoding or line separator character (U+2028) or 
end of file, decoding characters in the same manner as {\cf
  get-string-n} and {\cf get-string-n!}.
   
If an end-of-line encoding or line separator is read, then a string
containing all of the text up to (but not including) the end-of-line
encoding is returned, and the port is updated to point just past the
end-of-line encoding or line separator. If an end of file is
encountered before any end-of-line encoding is read, but some bytes
have been read and decoded as characters, then a string containing
those characters is returned. If an end of file is encountered before
any bytes are read, then the end-of-file object is
returned.
\end{entry}

\begin{entry}{%
\proto{get-datum}{ input-port}{procedure}
\rproto{get-datum}{ input-port transcoder}{procedure}}
 
Reads an external representation from \var{input-port} and returns the
datum it represents.  The {\cf get-datum} procedure returns the next
datum parsable from the given \var{input-port}, updating
\var{input-port} to point exactly past the end of the external
representation of the object.

Any \meta{interlexeme space} (see report section~\ref{report:lexicalsyntaxsection}) in
the input is first skipped.  If an end of file occurs after the
\meta{interlexeme space}, the end of file object (see
report section~\ref{report:eofsection}) is returned.

If a character inconsistent with an external representation is
encountered in the input, an exception with condition types
{\cf\&lexical} and {\cf\&i/o-read} is raised.  Also, if the end of file is encountered
after the beginning of an external representation, but the external
representation is incomplete and therefore not parsable, an exception
with condition types {\cf\&lexical} and {\cf\&i/o-read} is raised.
\end{entry}

\subsection{Output ports}

An output port is a sink to which bytes are written. These bytes may control
external devices, or may produce files and other objects that may
subsequently be opened for input.  The procedures in this section
that perform
textual output accept a transcoder as an optional argument.  If no
transcoder is supplied, the character(s) output is translated to
UTF-8 with a platform-specific end-of-line convention.

\begin{entry}{%
\proto{output-port?}{ obj}{procedure}}
   
Returns \schtrue{} if the argument is an output port (or a
combined input and output port), and returns \schfalse{} otherwise.
\end{entry}   

\begin{entry}{%
\proto{flush-output-port}{ output-port}{procedure}}
   
Flushes any output from the buffer of \var{output-port} to the
underlying file, device, or object. The unspecified value is returned.
\end{entry}

\begin{entry}{%
\proto{output-port-buffer-mode}{ output-port}{procedure}}
   
Returns the symbol that represents the buffer-mode of
\var{output-port}.
\end{entry}

\begin{entry}{%
\proto{open-file-output-port}{ filename}{procedure}
\rproto{open-file-output-port}{ filename file-options}{procedure}
\rproto{open-file-output-port}{ filename file-options transcoder}{procedure}}

Returns an output port for the named file and the specified
options (which default to {\cf (file-options)}). The \var{buffer-mode}
argument is optional; it defaults to {\cf block}.  If \var{transcoder}
is specified, it becomes the transcoder associated with the returned
port.
\end{entry}   

\begin{entry}{%
\proto{open-bytes-output-port}{ proc}{procedure}
\rproto{open-bytes-output-port}{ proc transcoder}{procedure}
\proto{open-string-output-port}{ proc}{procedure}
\rproto{open-string-output-port}{ proc transcoder}{procedure}}

Creates an output port that accumulates a bytes object from the
output written to it.  If \var{transcoder} is specified, it becomes
the transcoder associated with the returned port.

The {\cf open-string-output-port} procedure is the same as {\cf
  open-bytes-output-port}.
\end{entry}

\begin{entry}{%
\proto{get-output-bytes}{ output-port}{procedure}}

\domain{\var{Output-port} must be a port that accumulates a bytes
  object from the output written to it, such as the ports created by
  {\cf open-bytes-output-port} and {\cf call-with-bytes-output-port}.}
Returns a bytes object containing the output that has been accumulated with
\var{output-port} so far.
If the returned bytes object is modified, the
the effect on \var{output-port} is unspecified.
\end{entry}

\begin{entry}{%
\proto{call-with-bytes-output-port}{ proc}{procedure}
\rproto{call-with-bytes-output-port}{ proc transcoder}{procedure}}

\domain{\var{Proc} must be a procedure accepting one argument.}
Creates an output port that accumulates a bytes object from the output
written to it, and calls \var{proc} with that output port as an
argument. When \var{proc} returns for the first time, the port is
closed and the bytes object associated with the port is returned.  If
\var{transcoder} is specified, it becomes the transcoder associated
with the port.
\end{entry}

\begin{entry}{%
\proto{get-output-string}{ output-port}{procedure}
\rproto{get-output-string}{ output-port transcoder}{procedure}}

\domain{\var{Output-port} must be a port that accumulates a bytes
  object from the output written to it, such as the ports created by
  {\cf open-bytes-output-port} and {\cf call-with-bytes-output-port}.}
The decoding of the bytes object associated with \var{output-port} is returned
as a string.  That decoding is according to
\var{transcoder} if it is specified.  If it is not specified,
the decoding is according to the transcoder associated with the port.
If no transcoder is associated with the port, the decoding is
according to UTF-8.
In any
case, decoding errors are always handled analogously to the {\cf
  replace} error-handling mode: the first byte of each invalid
encoding is skipped and decoded as the U+FFFD replacement character.

If the returned string is modified, 
the effect on \var{output-port} is unspecified.
\end{entry}

\begin{entry}{%
\proto{call-with-string-output-port}{ proc}{procedure}
\rproto{call-with-string-output-port}{ proc transcoder}{procedure}}
   
\domain{\var{Proc} must be a procedure accepting one argument.}
Creates an output port that accumulates a bytes object from the output
written to it, and calls \var{proc} with that port as an argument.
When \var{proc} returns for the first time, the port is closed and the
decoding of the bytes object associated with the port is returned as
a string.  That decoding is according to
\var{transcoder} if it is specified.  If it is not specified,
the decoding is according to the transcoder associated with the port.
If no transcoder is associated with the port, the decoding is
according to UTF-8.
In any
case, decoding errors are always handled analogously to the {\cf
  replace} error-handling mode.
If \var{transcoder} is specified, it also becomes the transcoder
associated with the port.
\end{entry}

\begin{entry}{%
\proto{clear-bytes-output-port!}{ output-port}{procedure}
\proto{clear-string-output-port!}{ output-port}{procedure}}

\domain{\var{Output-port} must be a port that accumulates a bytes
  object from the output written to it, such as the ports created by
the {\cf open-bytes-output-port} and {\cf call-with-bytes-output-port}
procedures.}
The {\cf clear-bytes-output-port!} and {\cf clear-string-output-port!}
procedures clear the bytes object associated with \var{output-port},
associating it with an empty bytes object.
\end{entry}

\begin{entry}{%
\proto{standard-output-port}{}{procedure}
\proto{standard-error-port}{}{procedure}}
   
Returns a port connected to the standard output or standard error,
respectively. If the returned port supports textual output, it has an
associated transcoder with some encoding, some end-of-line style, and
error-handling mode {\cf raise}.

\begin{note}
  Implementations are encouraged to provide transcoders appropriate
  for writing text to these ports.
\end{note}
\end{entry}

\begin{entry}{%
\proto{put-u8}{ output-port octet}{procedure}}

Writes \var{octet} to the output port and returns the unspecified value.
\end{entry}

\begin{entry}{%
\proto{put-bytes}{ output-port bytes}{procedure}
\rproto{put-bytes}{ output-port bytes start}{procedure}
\rproto{put-bytes}{ output-port bytes start count}{procedure}}
   
\domain{\var{Start} and \var{count} must be non-negative exact
  integers that default to 0 and $\texttt{(bytes-length \var{bytes})}
  - \var{start}$, respectively. \var{bytes} must have a size of at
  least $\var{start} + \var{count}$.}  The {\cf put-bytes} procedure writes
\var{count} bytes of the bytes object \var{bytes}, starting at index
\var{start}, to the output port. The unspecified value is returned.
\end{entry}

\begin{entry}{%
\proto{put-char}{ output-port char}{procedure}
\rproto{put-char}{ output-port char transcoder}}
   
Writes an encoding of \var{char} to the port. The
unspecified value is returned.
\end{entry}

\begin{entry}{%
\proto{put-string}{ output-port string}{procedure}
\rproto{put-string}{ output-port string transcoder}{procedure}}
   
Writes an encoding of \var{string} to the port. The unspecified value is
returned.
\end{entry}   

\begin{entry}{%
\proto{put-string-n}{ output-port string}{procedure}
\rproto{put-string-n}{ output-port string start}{procedure}
\rproto{put-string-n}{ output-port string start count}{procedure}
\rproto{put-string-n}{ output-port string start count transcoder}{procedure}}
   
\domain{\var{Start} and \var{count} must be non-negative exact
  integers.  \var{string} must have a length of at least $\var{start} +
  \var{count}$.}  \var{start} defaults to 0.  \var{count} defaults to
$\texttt{(string-length \var{bytes})} - \var{start}$. Writes the
encoding of the \var{count} characters of \var{string}, starting at
index \var{start}, to the port. The unspecified value is returned.
\end{entry}


\begin{entry}{%
\proto{put-datum}{ output-port datum}{procedure}
\rproto{put-datum}{ output-port datum transcoder}{procedure}}

\domain{\var{Datum} should be a datum value.}
The {\cf put-datum} procedure writes an external representation of
\var{datum} to \var{output-port}.
The specific external representation is implementation-dependent.

\begin{note}
  The {\cf put-datum} procedure merely writes the external
  representation.  If {\cf put-datum} is used to write several
  subsequent external representations to an output port, care must be
  taken to delimit them properly so they can be read back in by
  subsequent calls to {\cf get-datum}.
\end{note}
\end{entry}

\subsection{Opening files for reading and writing}

\begin{entry}{%
\proto{open-file-input/output-port}{ filename}{procedure}
\rproto{open-file-input/output-port}{ filename file-options}{procedure}
\pproto{(open-file-input/output-port \var{filename}}{procedure}
{\tt\hspace*{4em}\var{file-options} \var{buffer-mode})}\\
\pproto{(open-file-input/output-port \var{filename}}{procedure}
{\tt\hspace*{4em}\var{file-options} \var{buffer-mode} \var{transcoder})}}
   
Returns a single port that is both an input port and an output
port for the named file and options (which default to {\cf
  (file-options)}).  \var{buffer-mode} optionally specifies the buffer
mode of the port; it defaults to {\cf block}.
   
The returned port supports the {\cf port-position} and {\cf
  set-port-position!}  operations. The same port position is used for
both input and output.

If \var{transcoder} is specified, it becomes the transcoder associated
with the returned port.
\end{entry}

\subsection{Ports from readers and writers}

\begin{entry}{%
\proto{open-reader-input-port}{ reader}{procedure}
\rproto{open-reader-input-port}{ reader transcoder}{procedure}}

Returns an input port connected to the reader \var{reader}.
If \var{transcoder} is specified, it becomes the transcoder associated
with the returned port.

The returned port supports the {\cf port-position} operation if and
only if \var{reader} supports the {\cf reader-get-position} operation.
It supports the {\cf set-port-position!} operation if and
only if \var{reader} supports the {\cf reader-set-position!} operation.
\end{entry}   

\begin{entry}{%
\proto{open-writer-output-port}{ writer}{procedure}
\rproto{open-writer-output-port}{ writer buffer-mode}{procedure}
\pproto{(open-writer-output-port \var{writer}}{procedure}
{\tt\hspace*{4em}\var{buffer-mode} \var{transcoder})}}
   
Returns an output port connected to the writer \var{writer}.
\var{buffer-mode} optionally specifies the buffer mode of the port; it
defaults to {\cf block}.
If \var{transcoder} is specified, it becomes the transcoder associated
with the returned port.

The returned port supports the {\cf port-position} operation if and
only if \var{writer} supports the {\cf writer-get-position} operation.
It supports the {\cf set-port-position!} operation if and
only if \var{writer} supports the {\cf writer-set-position!} operation.
\end{entry}

\begin{entry}{%
\pproto{(open-reader/writer-input/output-port}{procedure}
{\tt\hspace*{4em}\var{reader} \var{writer} \var{buffer-mode} \var{transcoder})}}

Returns a combined input/output port connected to \var{reader} for
reading operations and to \var{writer} for writing operations. 
\var{Buffer-mode} optionally specifies the buffer mode of the port; it
defaults to {\cf block}.
If \var{transcoder} is specified, it becomes the transcoder associated
with the returned port.

The returned port supports the {\cf port-position} operation if and
only if \var{reader} supports the {\cf reader-get-position} operation and
\var{writer} supports the {\cf writer-get-position} operation.
It supports the {\cf set-port-position!} operation if and
only if \var{reader} supports the {\cf reader-set-position!} operation and
\var{writer} supports the {\cf writer-set-position!} operation.
In that case, each call to {\cf set-port-position!} will set the
position of both \var{reader} and \var{writer}.

\begin{note}
  It is expected that \var{reader} and \var{writer} access the same
  underlying data, as is the case with the reader and writer and
  writer returned by {\cf open-file-reader+writer}.
\end{note}
\end{entry}

%%% Local Variables: 
%%% mode: latex
%%% TeX-master: "r6rs-lib"
%%% End: 

\chapter{Bytevectors}
\label{bytevectorschapter}

Many applications deal with blocks of binary data by accessing
them in various ways---extracting signed or unsigned numbers of
various sizes.  Therefore, the \defrsixlibrary{bytevector} library
provides a single type for
blocks of binary data with multiple ways to access that data. It deals
with integers and floating-point representations 
in various sizes with specified endianness, because
these are the most frequent applications.

Bytevectors\mainindex{bytevector} are objects of a disjoint
type. Conceptually, a bytevector represents a sequence of 8-bit
bytes.  The description of bytevectors uses the term \defining{byte}
for an exact integer in the interval $\{-128, \ldots, 127\}$ and the
term \defining{octet} for an exact integer in the interval $\{0,
\ldots, 255\}$.  A byte corresponds to its two's complement
representation as an octet.

The length of a bytevector is the number of bytes it contains. This
number is fixed. A valid index into a bytevector is an exact,
non-negative integer. The first byte of a bytevector has index 0;
the last byte has an index one less than the length of the bytevector.

Generally, the access procedures come in different flavors according
to the size of the represented integer and the endianness of the
representation.  The procedures also distinguish signed and unsigned
representations.
The signed representations all use two's complement.

Like list and vector literals, literals representing bytevectors
must be quoted:
%
\begin{scheme}
'\#vu8(12 23 123) \ev \#vu8(12 23 123)%
\end{scheme}

\section{Endianness}

Many operations described in this chapter accept an
\defining{endianness} argument.  Endianness describes the encoding of
exact integers as several contiguous bytes in a bytevector~\cite{IEN137}. 
For this purpose, the binary representation of the integer is split into
consecutive bytes.  \mainindex{little-endian}The little-endian
encoding places the least significant byte of an integer first, with
the other bytes following in increasing order of significance.
\mainindex{big-endian}The big-endian encoding places the most
significant byte of an integer first, with the other bytes following
in decreasing order of significance. 

This terminology also applies to IEEE-754 numbers: IEEE-754 describes
how to represent a floating-point number as an exact integer, and
endianness describes how the bytes of such an integer are laid out in
a bytevector.

\begin{note}
  Little- and big-endianness are only the most common kinds of
  endianness.  Some architectures distinguish between the endianness
  at different levels of a binary representation.
\end{note}

\section{General operations}

\begin{entry}{%
\proto{endianness}{ \hyper{endianness symbol}}{\exprtype}}

\domain{\hyper{Endianness symbol} must be a symbol describing an
  endianness.  An implementation must support at least the symbols
  {\cf big} and {\cf little}, but may support other endianness
  symbols.}  {\cf (endianness \hyper{endianness symbol})} evaluates to
the symbol named \hyper{endianness symbol}.  Whenever one of the
procedures operating on bytevectors accepts an endianness as an
argument, that argument must be one of these symbols.  It is a syntax
violation for \hyper{endianness symbol} to be anything other than an
endianness symbol supported by the implementation.

\begin{note}
  Implementors are encouraged to use widely accepted designations
  for endianness symbols other than {\cf big} and {\cf little}.
\end{note}
\end{entry}

\begin{entry}{%
\proto{native-endianness}{}{procedure}}

Returns the endianness symbol associated implementation's preferred
endianness (usually that of the underlying machine architecture).
This may be any \hyper{endianness symbol}, including a symbol other
than {\cf big} and {\cf little}.
\end{entry}   

\begin{entry}{%
\proto{bytevector?}{ obj}{procedure}}
   
Returns \schtrue{} if \var{obj} is a bytevector,
otherwise returns \schfalse{}.
\end{entry}

\begin{entry}{%
\proto{make-bytevector}{ k}{procedure}
\rproto{make-bytevector}{ k fill}{procedure}}
   
Returns a newly allocated bytevector of \var{k} bytes.
   
If the \var{fill} argument is missing, the initial contents of the
returned bytevector are unspecified.
   
If the \var{fill} argument is present, it must be an exact integer in
the interval $\{-128, \ldots 255\}$ that specifies the initial value
for the bytes of the bytevector: If \var{fill} is positive, it is
interpreted as an octet; if it is negative, it is interpreted as a byte.
\end{entry}   

\begin{entry}{%
\proto{bytevector-length}{ bytevector}{procedure}}
   
Returns, as an exact integer, the number of bytes in \var{bytevector}.
\end{entry}

\begin{entry}{%
\proto{bytevector=?}{ \vari{bytevector} \varii{bytevector}}{procedure}}
   
Returns \schtrue{} if \vari{bytevector} and \varii{bytevector} are equal---that
is, if they have the same length and equal bytes at all valid indices.
It returns \schfalse{} otherwise.
\end{entry}

\begin{entry}{%
\proto{bytevector-fill!}{ bytevector fill}}

\domain{The \var{fill} argument is as in the description of the {\cf
    make-bytevector} procedure.}
Stores \var{fill} in every element of \var{bytevector}
and returns \unspecifiedreturn.  Analogous to {\cf vector-fill!}.
\end{entry}

\begin{entry}{%
\pproto{(bytevector-copy! \var{source} \var{source-start}}{procedure}}
\mainschindex{bytevector-copy!}{\tt\obeyspaces\\
     \var{target} \var{target-start} \var{k})}

\domain{\var{Source} and \var{target} must be bytevectors.
  \var{Source-start}, \var{target-start},
  and \var{k} must be non-negative exact integers that satisfy
  
  \begin{displaymath}
    \begin{array}{rcccccl}
      0 & \leq & \var{source-start} & \leq & \var{source-start} + \var{k} & \leq & l_{\var{source}}
      \\
      0 & \leq & \var{target-start} & \leq & \var{target-start} + \var{k} & \leq & l_{\var{target}}
    \end{array}
  \end{displaymath}
  %
  where $l_{\var{source}}$ is the length of \var{source} and
  $l_{\var{target}}$ is the length of \var{target}.}
   
   
  The {\cf bytevector-copy!} procedure copies the bytes from \var{source} at indices 
  \begin{displaymath}
     \{ \var{source-start}, \ldots \var{source-start} + \var{k} - 1 \}
  \end{displaymath}
  to consecutive indices in \var{target} starting at \var{target-index}.
   
  This must work even if the memory regions for the source and the target
  overlap, i.e., the bytes at the target location after the copy must be
  equal to the bytes at the source location before the copy.
   
  This returns \unspecifiedreturn.
\begin{scheme}
(let ((b (u8-list->bytevector '(1 2 3 4 5 6 7 8))))
  (bytevector-copy! b 0 b 3 4)
  (bytevector->u8-list b)) \ev (1 2 3 1 2 3 4 8)
\end{scheme}
\end{entry}

\begin{entry}{%
\proto{bytevector-copy}{ bytevector}{procedure}}
   
Returns a newly allocated copy of \var{bytevector}.
\end{entry}

\section{Operations on bytes and octets}

\begin{entry}{%
\proto{bytevector-u8-ref}{ bytevector k}{procedure}
\proto{bytevector-s8-ref}{ bytevector k}{procedure}}
   
\domain{\var{K} must be a valid index of \var{bytevector}.}
   
The {\cf bytevector-u8-ref} procedure returns the byte at index \var{k} of \var{bytevector},
as an octet.
   
The {\cf bytevector-s8-ref} procedure returns the byte at index \var{k} of \var{bytevector},
as a (signed) byte.

\begin{scheme}
(let ((b1 (make-bytevector 16 -127))
      (b2 (make-bytevector 16 255)))
  (list
    (bytevector-s8-ref b1 0)
    (bytevector-u8-ref b1 0)
    (bytevector-s8-ref b2 0)
    (bytevector-u8-ref b2 0))) \ev (-127 129 -1 255)
\end{scheme}
\end{entry}   

\begin{entry}{%
\proto{bytevector-u8-set!}{ bytevector k octet}{procedure}
\proto{bytevector-s8-set!}{ bytevector k byte}{procedure}}
   
\domain{\var{K} must be a valid index of \var{bytevector}.}
   
The {\cf bytevector-u8-set!} procedure stores \var{octet} in element \var{k} of
\var{bytevector}.
   
The {\cf bytevector-s8-set!} procedure stores the two's complement representation of
\var{byte} in element \var{k} of \var{bytevector}.
   
Both procedures return \unspecifiedreturn.

\begin{scheme}
(let ((b (make-bytevector 16 -127)))

  (bytevector-s8-set! b 0 -126)
  (bytevector-u8-set! b 1 246)

  (list
    (bytevector-s8-ref b 0)
    (bytevector-u8-ref b 0)
    (bytevector-s8-ref b 1)
    (bytevector-u8-ref b 1))) \ev (-126 130 -10 246)
\end{scheme}
\end{entry}

\begin{entry}{%
\proto{bytevector->u8-list}{ bytevector}{procedure}
\proto{u8-list->bytevector}{ list}{procedure}}
   
\domain{\var{List} must be a list of octets.}

The {\cf bytevector->u8-list} procedure returns a newly allocated list of the octets of
\var{bytevector} in the same order.

The {\cf u8-list->bytevector} procedure returns a newly allocated bytevector whose
elements are the elements of list \var{list}, in
the same order.  It is analogous to {\cf list->vector}.
\end{entry}

\section{Operations on integers of arbitrary size}

\begin{entry}{%
\proto{bytevector-uint-ref}{ bytevector k endianness size}{procedure}
\proto{bytevector-sint-ref}{ bytevector k endianness size}{procedure}
\proto{bytevector-uint-set!}{ bytevector k n endianness size}{procedure}
\proto{bytevector-sint-set!}{ bytevector k n endianness size}{procedure}}
   
\domain{\var{Size} must be a positive exact integer. $\{\var{k}, \ldots,
  \var{k} + \var{size} - 1\}$ must be valid indices of \var{bytevector}.}
   
{\cf bytevector-uint-ref} retrieves the exact integer corresponding to the
unsigned representation of size \var{size} and specified by \var{endianness}
at indices $\{\var{k}, \ldots, \var{k} + \var{size} - 1\}$.
   
{\cf bytevector-sint-ref} retrieves the exact integer corresponding to the two's
complement representation of size \var{size} and specified by \var{endianness} at
indices $\{\var{k}, \ldots, \var{k} + \var{size} - 1\}$.
   
\domain{For {\cf bytevector-uint-set!}, \var{n} must be an exact integer in the
  interval $\{0, \ldots, 256^{\var{size}}-1\}$.}

{\cf bytevector-uint-set!} stores the unsigned representation of size \var{size}
and specified by \var{endianness} into \var{bytevector} at indices
$\{\var{k}, \ldots, \var{k} + \var{size} - 1\}$.
   
\domain{For {\cf bytevector-sint-set!}, \var{n} must be an exact integer in
  the interval $\{-256^{\var{size}}/2, \ldots,
  256^{\var{size}}/2-1\}$.}
{\cf bytevector-sint-set!} stores the two's complement
representation of size \var{size} and specified by \var{endianness}
into \var{bytevector} at indices $\{\var{k}, \ldots, \var{k} + \var{size} - 1\}$.
   
The \ldots{\cf -set!} procedures return \unspecifiedreturn.

\begin{scheme}
(define b (make-bytevector 16 -127))

(bytevector-uint-set! b 0 (- (expt 2 128) 3)
                     (endianness little) 16)

(bytevector-uint-ref b 0 (endianness little) 16)\lev
    \#xfffffffffffffffffffffffffffffffd

(bytevector-sint-ref b 0 (endianness little) 16)\lev -3

(bytevector->u8-list b)\lev (253 255 255 255 255 255 255 255
               255 255 255 255 255 255 255 255)

(bytevector-uint-set! b 0 (- (expt 2 128) 3)
                 (endianness big) 16)

(bytevector-uint-ref b 0 (endianness big) 16) \lev
    \#xfffffffffffffffffffffffffffffffd

(bytevector-sint-ref b 0 (endianness big) 16) \lev -3

(bytevector->u8-list b) \lev (255 255 255 255 255 255 255 255
               255 255 255 255 255 255 255 253))
\end{scheme}
\end{entry}

\begin{entry}{%
\proto{bytevector->uint-list}{ bytevector endianness size}{procedure}
\proto{bytevector->sint-list}{ bytevector endianness size}{procedure}
\proto{uint-list->bytevector}{ list endianness size}{procedure}
\proto{sint-list->bytevector}{ list endianness size}{procedure}}
   
\domain{\var{Size} must be a positive exact integer.  For {\cf
    uint-list->bytevector}, \var{list} must be a list of exact
  integers in the interval $\{0, \ldots, 256^{\var{size}}-1\}$.  For
  {\cf sint-list->bytevector}, \var{list} must be a list of exact
  integers in the interval $\{-256^{\var{size}}/2, \ldots,
  256^{\var{size}}/2-1\}$.  The length of \var{bytevector} or,
  respectively, of \var{list} must be divisible by \var{size}.}
   
These procedures convert between lists of integers and their consecutive
representations according to \var{size} and \var{endianness} in the
\var{bytevector} objects in the same way as {\cf bytevector->u8-list} and {\cf
  u8-list->bytevector} do for one-byte representations.

\begin{scheme}
(let ((b (u8-list->bytevector '(1 2 3 255 1 2 1 2))))
  (bytevector->sint-list b (endianness little) 2)) \lev (513 -253 513 513)

(let ((b (u8-list->bytevector '(1 2 3 255 1 2 1 2))))
  (bytevector->uint-list b (endianness little) 2)) \lev (513 65283 513 513)
\end{scheme}
\end{entry}

\section{Operations on 16-bit integers}

\begin{entry}{%
\proto{bytevector-u16-ref}{ bytevector k endianness}{procedure}
\proto{bytevector-s16-ref}{ bytevector k endianness}{procedure}
\proto{bytevector-u16-native-ref}{ bytevector k}{procedure}
\proto{bytevector-s16-native-ref}{ bytevector k}{procedure}
\proto{bytevector-u16-set!}{ bytevector k n endianness}{procedure}
\proto{bytevector-s16-set!}{ bytevector k n endianness}{procedure}
\proto{bytevector-u16-native-set!}{ bytevector k n}{procedure}
\proto{bytevector-s16-native-set!}{ bytevector k n}{procedure}}
   
\domain{\var{K} must be a valid index of \var{bytevector}; so must
  $\var{k} + 1$. For {\cf bytevector-u16-set!} and {\cf
    bytevector-u16-native-set!}, \var{n} must be an exact integer in
  the interval $\{0, \ldots, 2^{16}-1\}$.  For {\cf bytevector-s16-set!}
  and {\cf bytevector-s16-native-set!}, \var{n} must be an exact
  integer in the interval $\{-2^{15}, \ldots, 2^{15}-1\}$.}
   
These retrieve and set two-byte representations of numbers at indices
\var{k} and $\var{k}+1$, according to the endianness specified by
\var{endianness}. The procedures with {\cf u16} in their names deal with the
unsigned representation; those with {\cf s16} in their names deal
with the two's complement representation.

The procedures with {\cf native} in their names employ the native
endianness, and work only at aligned indices:
\var{k} must be a multiple of 2.
   
The \ldots{\cf -set!} procedures return \unspecifiedreturn.

\begin{scheme}
(define b
  (u8-list->bytevector
    '(255 255 255 255 255 255 255 255
      255 255 255 255 255 255 255 253)))

(bytevector-u16-ref b 14 (endianness little)) \lev 65023
(bytevector-s16-ref b 14 (endianness little)) \lev -513
(bytevector-u16-ref b 14 (endianness big)) \lev 65533
(bytevector-s16-ref b 14 (endianness big)) \lev -3

(bytevector-u16-set! b 0 12345 (endianness little))
(bytevector-u16-ref b 0 (endianness little)) \lev 12345

(bytevector-u16-native-set! b 0 12345)
(bytevector-u16-native-ref b 0) \ev 12345

(bytevector-u16-ref b 0 (endianness little)) \lev \unspecified
\end{scheme}
\end{entry}

\section{Operations on 32-bit integers}

\begin{entry}{%
\proto{bytevector-u32-ref}{ bytevector k endianness}{procedure}
\proto{bytevector-s32-ref}{ bytevector k endianness}{procedure}
\proto{bytevector-u32-native-ref}{ bytevector k}{procedure}
\proto{bytevector-s32-native-ref}{ bytevector k}{procedure}
\proto{bytevector-u32-set!}{ bytevector k n endianness}{procedure}
\proto{bytevector-s32-set!}{ bytevector k n endianness}{procedure}
\proto{bytevector-u32-native-set!}{ bytevector k n}{procedure}
\proto{bytevector-s32-native-set!}{ bytevector k n}{procedure}}
   
\domain{$\{\var{k}, \ldots, \var{k}+ 3\}$ must be valid indices of
  \var{bytevector}.
  For {\cf bytevector-u32-set!} and {\cf
    bytevector-u32-native-set!}, \var{n} must be an exact integer in
  the interval $\{0, \ldots, 2^{32}-1\}$.  For {\cf bytevector-s32-set!}
  and {\cf bytevector-s32-native-set!}, \var{n} must be an exact
  integer in the interval $\{-2^{31}, \ldots, 2^{32}-1\}$.}
   
These retrieve and set four-byte representations of numbers at indices $\{\var{k},
\ldots, \var{k}+ 3\}$, according to the endianness specified by \var{endianness}. The
procedures with {\cf u32} in their names deal with the unsigned representation;
those with {\cf s32} with the two's complement representation.
   
The procedures with {\cf native} in their names employ the native endianness, and
work only at aligned indices: \var{k} must be a multiple of 4.
   
The \ldots{\cf{}-set!} procedures return \unspecifiedreturn.

\begin{scheme}
(define b
  (u8-list->bytevector
    '(255 255 255 255 255 255 255 255
      255 255 255 255 255 255 255 253)))

(bytevector-u32-ref b 12 (endianness little)) \lev 4261412863
(bytevector-s32-ref b 12 (endianness little)) \lev -33554433
(bytevector-u32-ref b 12 (endianness big)) \lev 4294967293
(bytevector-s32-ref b 12 (endianness big)) \lev -3
\end{scheme}
\end{entry}

\section{Operations on 64-bit integers}

\begin{entry}{%
\proto{bytevector-u64-ref}{ bytevector k endianness}{procedure}
\proto{bytevector-s64-ref}{ bytevector k endianness}{procedure}
\proto{bytevector-u64-native-ref}{ bytevector k}{procedure}
\proto{bytevector-s64-native-ref}{ bytevector k}{procedure}
\proto{bytevector-u64-set!}{ bytevector k n endianness}{procedure}
\proto{bytevector-s64-set!}{ bytevector k n endianness}{procedure}
\proto{bytevector-u64-native-set!}{ bytevector k n}{procedure}
\proto{bytevector-s64-native-set!}{ bytevector k n}{procedure}}
 
\domain{$\{\var{k}, \ldots, \var{k}+ 7\}$ must be valid indices of
  \var{bytevector}.
  For {\cf bytevector-u64-set!} and {\cf
    bytevector-u64-native-set!}, \var{n} must be an exact integer in
  the interval $\{0, \ldots, 2^{64}-1\}$.  For {\cf bytevector-s64-set!}
  and {\cf bytevector-s64-native-set!}, \var{n} must be an exact
  integer in the interval $\{-2^{63}, \ldots, 2^{64}-1\}$.}
   
These retrieve and set eight-byte representations of numbers at
indices $\{\var{k}, \ldots, \var{k}+ 7\}$, according to the endianness
specified by \var{endianness}. The procedures with {\cf u64} in their names deal
with the unsigned representation; those with {\cf s64} with the two's
complement representation.
   
The procedures with {\cf native} in their names employ the native endianness, and
work only at aligned indices: \var{k} must be a multiple of 8.
   
The \ldots{\cf{}-set!} procedures return \unspecifiedreturn.

\begin{scheme}
(define b
  (u8-list->bytevector
    '(255 255 255 255 255 255 255 255
      255 255 255 255 255 255 255 253)))

(bytevector-u64-ref b 8 (endianness little)) \lev 18302628885633695743
(bytevector-s64-ref b 8 (endianness little)) \lev -144115188075855873
(bytevector-u64-ref b 8 (endianness big)) \lev 18446744073709551613
(bytevector-s64-ref b 8 (endianness big)) \lev -3
\end{scheme}
\end{entry}

\section{Operations on IEEE-754 numbers}

\begin{entry}{%
\proto{bytevector-ieee-single-native-ref}{ bytevector k}{procedure}
\proto{bytevector-ieee-single-ref}{ bytevector k endianness}{procedure}}

\domain{$\{\var{k}, \ldots, \var{k}+3\}$ must be valid indices of
  \var{bytevector}.  For {\cf bytevector-ieee-single-native-ref}, \var{k} must
  be a multiple of $4$.}

These procedures return the inexact real that best represents the IEEE-754 single
precision number represented by the four bytes beginning at index
\var{k}.
\end{entry}

\begin{entry}{%
\proto{bytevector-ieee-double-native-ref}{ bytevector k}{procedure}
\proto{bytevector-ieee-double-ref}{ bytevector k endianness}{procedure}}

\domain{$\{\var{k}, \ldots, \var{k}+7\}$ must be valid indices of
  \var{bytevector}.  For {\cf bytevector-ieee-double-native-ref}, \var{k} must
  be a multiple of $8$.}

These procedures return the inexact real that best represents the IEEE-754 single
precision number represented by the eight bytes beginning at index
\var{k}.
\end{entry}

\begin{entry}{%
\proto{bytevector-ieee-single-native-set!}{ bytevector k x}{procedure}
\pproto{(bytevector-ieee-single-set! \var{bytevector}}{procedure}}
{\tt\obeyspaces\\
     \var{k} \var{x} \var{endianness})}

\domain{$\{\var{k}, \ldots, \var{k}+3\}$ must be valid indices of
  \var{bytevector}.  For {\cf bytevector-ieee-single-native-set!}, \var{k} must
  be a multiple of $4$.}

These procedures store an IEEE-754 single precision representation of \var{x} into
elements \var{k} through $\var{k}+3$ of \var{bytevector}, and return
\unspecifiedreturn.
\end{entry}

\begin{entry}{%
\proto{bytevector-ieee-double-native-set!}{ bytevector k x}{procedure}
\pproto{(bytevector-ieee-double-set! \var{bytevector}}{procedure}}
{\tt\obeyspaces\\
     \var{k} \var{x} \var{endianness})}

\domain{$\{\var{k}, \ldots, \var{k}+7\}$ must be valid indices of
  \var{bytevector}.  For {\cf bytevector-ieee-double-native-set!}, \var{k} must
  be a multiple of $8$.}

These procedures store an IEEE-754 double precision representation of \var{x} into
elements \var{k} through $\var{k}+7$ of \var{bytevector}, and return
\unspecifiedreturn.
\end{entry}

\section{Operations on strings}

This section describes procedures that convert between strings and
bytevectors containing Unicode encodings of those strings.  When
decoding bytevectors, encoding errors are handled as with the {\cf
  replace} semantics of textual I/O (see
section~\ref{transcoderssection}): If an invalid or incomplete
character encoding is encountered, then the replacement character
U+FFFD is appended to the string being generated, an appropriate
number of bytes are ignored, and decoding continues with the following
bytes.

\begin{entry}{%
\proto{string->utf8}{ string}{procedure}}

Returns a newly allocated (unless empty) bytevector that
contains the UTF-8 encoding of the given string.
\end{entry}

\begin{entry}{%
\proto{string->utf16}{ string}{procedure}
\rproto{string->utf16}{ string endianness}{procedure}}

\domain{If \var{endianness} is specified, it must be the symbol {\cf
    big} or the symbol {\cf little}.}  The {\cf string->utf16
  procedure} returns a newly allocated (unless empty) bytevector that
contains the UTF-16BE or UTF-16LE encoding of the given string (with
no byte-order mark).  If endianness is not specified or is {\cf big},
then UTF-16BE is used.  If endianness is {\cf little}, then UTF-16LE
is used.
\end{entry}

\begin{entry}{%
\proto{string->utf32}{ string}{procedure}
\rproto{string->utf32}{ string endianness}{procedure}}

\domain{If \var{endianness} is specified, it must be the symbol {\cf
    big} or the symbol {\cf little}.}  The {\cf string->utf32} returns
a newly allocated (unless empty) bytevector that contains the UTF-32BE
or UTF-32LE encoding of the given string (with no byte mark).  If
endianness is not specified or is {\cf big}, then UTF-32BE is used.
If endianness is {\cf little}, then UTF-32LE is used.
\end{entry}

\begin{entry}{%
\proto{utf8->string}{ bytevector}{procedure}}

Returns a newly allocated (unless empty) string whose character
sequence is encoded by the given bytevector.
\end{entry}

\begin{entry}{%
\proto{utf16->string}{ bytevector}{procedure}
\rproto{utf16->string}{ bytevector endianness}{procedure}}

\domain{If \var{endianness} is specified, it must be the symbol {\cf
    big} or the symbol {\cf little}.}  The {\cf utf16->string}
procedure returns a newly allocated (unless empty) string whose
character sequence is encoded by the given bytevector.  If \var{endianness}
is {\cf big}, then UTF-16BE is used.  If \var{endianness} is {\cf little},
then UTF-16LE is used.  If \var{endianness} is not specified,
\var{bytevector} is decoded according to UTF-16: If it starts with the
bytes \sharpsign{}xFE, \sharpsign{}xFF (a big-endian byte-order mark),
then UTF-16BE is used for the remaining contents of \var{bytevector};
if it starts with the bytes \sharpsign{}xFF, \sharpsign{}xFE (a
little-endian byte-order mark), then UTF-16LE is used for the
remaining contents of \var{bytevector}; otherwise the entire contents
of \var{bytevector} are decoded according to UTF-16BE.
\end{entry}

\begin{entry}{%
\proto{utf32->string}{ bytevector}{procedure}
\rproto{utf32->string}{ bytevector endianness}{procedure}}

\domain{If \var{endianness} is specified, it must be the symbol {\cf
    big} or the symbol {\cf little}.}  The {\cf utf32->string}
procedure returns a newly allocated (unless empty) string whose
character sequence is encoded by the given bytevector.  If
\var{endianness} is {\cf big}, then UTF-32BE is used.  If
\var{endianness} is {\cf little}, then UTF-32LE is used.  If
\var{endianness} is not specified, \var{bytevector} is decoded
according to UTF-32: If it starts with the bytes \sharpsign{}x00,
\sharpsign{}x00, \sharpsign{}xFE, \sharpsign{}xFF (a big-endian
byte-order mark), then UTF-32BE is used for the remaining contents of
\var{bytevector}; if it starts with the bytes \sharpsign{}xFF,
\sharpsign{}xFE, \sharpsign{}x00, \sharpsign{}x00 (a little-endian
byte-order mark), then UTF-32LE is used for the remaining contents of
\var{bytevector}; otherwise the entire contents of \var{bytevector}
are decoded according to UTF-32BE.
\end{entry}

%%% Local Variables: 
%%% mode: latex
%%% TeX-master: "r6rs-lib"
%%% End: 

\chapter{Library file system mapping}
\label{filesystemmappingappendix}

Storing library source code as files in a file system is a common way to
support the use of standard tools for editing and other kinds of
source code processing.

The following recommendation specifies a standard way to map library
names to file names in popular filesystems, using an approach in which 
each file contains exactly one library form.  Following this recommendation
will allow users to work with a familiar source code structure across 
implementations, and can also allow multiple implementations to share 
a common repository of library source code.

The form of a library name is specified in 
section~\extref{report:librarysyntaxsection}{Library form}.  It can be 
expressed as follows:

\begin{scheme}
(\hyperi{identifier} \hyperii{identifier} \ldots \hypern{identifier} \hyper{version})%
\end{scheme}

where \hyper{version} is empty or has the following form:
%
\begin{scheme}
(\hyperi{sub-version} \hyperii{sub-version} \ldots)%
\end{scheme}

Such a library name should be mapped to a file in the file system with
the following relative path:
%TODO: not scheme
\begin{scheme}
\hyperi{identifier} \hyper{sep} \hyperii{identifier} \hyper{sep} \ldots \hypern{identifier} \hyper{xsep} \hyper{extension}
\end{scheme}
where \hyper{sep} is the platform-specific character (such as ``/'')
used to separate path elements (typically directory names); \hyper{xsep} 
is the platform-specific character (typically a period) used to separate parts 
of a filename; and \hyper{extension} has the following form:

%TODO: not scheme
\begin{scheme}
\hyperi{sub-version} \hyper{xsep} \hyperii{sub-version} \hyper{xsep} \ldots sls%
\end{scheme}

Note that the resulting path is relative to some implementation-dependent root 
directory.

%TODO:filename length limits, e.g. DOS? VMS? :)

%%% Local Variables: 
%%% mode: latex
%%% TeX-master: "r6rs-app"
%%% End: 

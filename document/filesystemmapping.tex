\chapter{Library / file system mapping}
\label{filesystemmappingappendix}

Storing library source code as files in a hierarchical file system is
a common way to support the use of standard tools for editing and other
kinds of source code processing.

The following recommendation specifies a standard way to map library
names to file names in widely-used file systems, using an approach in which 
each file contains exactly one library form.  Following this recommendation
will allow users to work with a familiar source code structure across 
implementations, and can also allow multiple implementations to share 
a common repository of library source code.

The form of a library name is specified in 
section~\extref{report:librarysyntaxsection}{Library form}.  It can be 
expressed as follows:

\begin{scheme}
(\hyperi{identifier} \ldots \hypern{identifier} \hyper{version})%
\end{scheme}

where \hyper{version} is empty or has the following form:
%
\begin{scheme}
(\hyperi{sub-version} \hyperii{sub-version} \ldots)%
\end{scheme}

Such a library name should be mapped to a file in the file system with
a relative path formed by the concatenation of the following components:
%TODO: not scheme
\begin{scheme}
\hyperi{identifier} \hyper{sep} \ldots \hypern{identifier} \hyper{xsep} \hyper{extension}
\end{scheme}
where \hyper{sep} is the platform-specific character (such as ``/'')
used to separate path elements (which are typically directory names); 
\hyper{xsep} is the platform-specific character (typically a period) 
used to separate parts of a file name; and \hyper{extension} has the 
following form:

%TODO: not scheme
\begin{scheme}
\hyperi{sub-version} \hyper{xsep} \hyperii{sub-version} \hyper{xsep} \ldots sls%
\end{scheme}
where {\cf sls} is the recommended extension used to identify Scheme library 
source files.

Note that the resulting path is relative to some implementation-dependent root 
directory.
%TODO: note that multiple roots are possible

According to this mapping, the source code for a library named {\cf (mylib examples
  hello)} would be stored in a file {\cf mylib/examples/hello.sls};
the source code for a library named {\cf (mylib examples hello (0 4
  2))} would be stored in a file {\cf mylib/examples/hello.0.4.2.sls}.

%TODO: define "known to the implementation"
A library source file may define a library with a library name consisting
of the same sequence of identifiers as another library known to the 
implementation, iff each library name includes a distinct and non-empty 
\hyper{version}.
%TODO: note about default version

If a library source file defines a library with a library name for which 
\hyper{version} is {\cf()} or empty, then the source file must similarly 
have no version embedded within its name.  In that case, to avoid 
confusion, no other library with a library name consisting of the same
sequence of identifiers, but with a non-empty version, should be known 
to the implementation.

This report does not describe a file system mapping for compiled code.

%TODO:filename length limits, e.g. DOS? VMS? :)

%%% Local Variables: 
%%% mode: latex
%%% TeX-master: "r6rs-app"
%%% End: 

\documentclass[twoside,twocolumn]{algol60}
%\documentclass[twoside]{algol60}

\usepackage{xr}
\externaldocument[lib:]{r6rs-lib}

\pagestyle{headings}
\showboxdepth=0
\makeindex
\input{commands}
%!TEX root = paper.tex

\usepackage{latexsym}
\usepackage{mathrsfs}
\usepackage{stmaryrd}

\newcommand{\pltreducks}{PLT Redex}
\newcommand{\rnrs}{Report}
\newcommand{\rnrslongspace}{\mbox{Revised\ensuremath{\,^{\mbox{\textrm{\scriptsize 5}}}} Report on Scheme}}
\newcommand{\rnrslong}{\mbox{Revised\ensuremath{^{\mbox{\textrm{\scriptsize 5}}}} Report on Scheme}}
\newcommand{\largernrslong}{\mbox{Revised\ensuremath{\,^{\mbox{\textrm{\large 5}}}} Report on Scheme}}

%\newenvironment*{proof}
%{\noindent\textbf{Proof} }
%{$\Box$ \\}

\newcommand*{\startappendixcode}{\protect\begin{verbatim}}
\newcommand*{\stopappendixcode}{\protect\end{verbatim}}

\newcommand*{\lvs}{\lambda_{\mathit{vs}}}
\newcommand*{\lunspec}{\lambda_{!?}}

%\newcommand{\either}{*\!{}\!{}\!\!\circ}
\newcommand{\either}{*\!\circ}

\newcommand{\hole}{[~]}
\newcommand{\holes}{\ensuremath{\hole_{\star}}}
\newcommand{\holeone}{\ensuremath{\hole_\circ}}
\newcommand{\holeany}{\ensuremath{\hole_{\either}}}

\newcommand{\sy}[1]{\textnormal{\textbf{#1}}}
\newcommand{\va}[1]{\textnormal{\textsf{#1}}}
% multi-letter nonterminals (one-letter can be done with $_$)
\newcommand{\nt}[1]{\textnormal{\textit{#1}}}

\newcommand{\beginF}{\ensuremath{\textbf{begin}^{\mbox{\textrm{\textbf{\scriptsize F}}}}}}
\newcommand{\Eo}{\ensuremath{E^{\circ}}}
\newcommand{\Estar}{\ensuremath{E^{\star}}}
\newcommand{\Fo}{\ensuremath{F^{\circ}}}
\newcommand{\Fstar}{\ensuremath{F^{\star}}}
\newcommand{\Io}{\ensuremath{I^{\circ}}}
\newcommand{\Istar}{\ensuremath{I^{\star}}}

\newcommand{\semfalse}{\textsf{\#f}}
\newcommand{\semtrue}{\textsf{\#t}}

\newcommand{\aline}{\noindent\hrulefill\par}

%\def\beginfig{\begin{figure*}[t]{\noindent\hrulefill\par}\small}
%\def\endfig{{\noindent\hrulefill\par}\end{figure*}}

\def\beginfig{\begin{figure*}[tb!]{\noindent\par}\small}
\def\endfig{{\noindent\hrulefill\par}\end{figure*}}

\newcommand{\dom}{\textit{dom}}

\newcommand{\gopen}{{^{\scriptscriptstyle\lceil}\!\!}}
\newcommand{\gclose}{\!\!{}^{\scriptscriptstyle\rceil}}

\newcommand*{\goesn}[1]{\stackrel{#1}{\rightarrow}}
\newcommand*{\goesone}{\goesn{C_1}}
\newcommand*{\goesall}{\goesn{C_*}}

\newcommand*{\goestrans}{\rightarrow}

\newcommand{\mrk}{\diamond}
\newcommand{\umrk}{^\mrk}

\newcommand{\rulename}[1]{\textsf{[#1]}}

%% reduction rule figure commands

\def\beginrules{\begin{displaymath}\begin{array}{l@{}l@{}lr}}
\def\endrules{\end{array}\end{displaymath}}

\newcommand{\extraspterm}{\\[6pt]}

\newcommand{\twolinerule}[3]{\twolineruleA{#1}{#2}{\rulename{#3}}{\rightarrow}}
\newcommand{\twolinescrule}[4]{\twolinescruleA{#1}{#2}{\rulename{#3}}{#4}{\rightarrow}}
\newcommand{\onelinerule}[3]{\onelineruleA{#1}{#2}{\rulename{#3}}{\rightarrow}}
\newcommand{\onelinescrule}[4]{\onelinescruleA{#1}{#2}{\rulename{#3}}{#4}{\rightarrow}}

\newcommand{\twolineruleA}[4]{
\multicolumn{3}{l}{{#1} {#4}} & {#3}\\ 
\multicolumn{3}{l}{{#2}} & \extraspterm}

\newcommand{\twolinescruleA}[5]{
\multicolumn{3}{l}{{#1} {#5}} & {#3}\\ 
\multicolumn{4}{l}{{#2 ~ ~ ~ {#4}}} \extraspterm}

\newcommand{\twolinescruleB}[5]{
\multicolumn{3}{l}{{#1} {#5}} & {#3}\\ 
\multicolumn{4}{l}{#2} \\
\multicolumn{4}{l}{~ ~ ~ #4} \extraspterm}

\newcommand{\onelineruleA}[4]{
\multicolumn{1}{l}{#1} & {#4} ~ & {#2} & {#3} \extraspterm}

\newcommand{\onelinescruleA}[5]{
\multicolumn{1}{l}{#1} & {#5} ~ & {#2} & {#3} \\
& & {#4} \extraspterm}



\def\headertitle{Revised$^{5.92}$ Scheme}
\def\integerversion{6}

\begin{document}

\thispagestyle{empty}

\topnewpage[{
\begin{center}   {\huge\bf
        Revised{\Huge$^{\mathbf{5.92}}$} Report on the Algorithmic Language \\
                              \vskip 3pt
                                Scheme}

\vskip 1ex
$$
\begin{tabular}{l@{\extracolsep{.5in}}lll}
\multicolumn{4}{c}{M\authorsc{ICHAEL} S\authorsc{PERBER}}
\\
\multicolumn{4}{c}{W\authorsc{ILLIAM} C\authorsc{LINGER},
  R.\ K\authorsc{ENT} D\authorsc{YBVIG},
  M\authorsc{ATTHEW} F\authorsc{LATT},
  A\authorsc{NTON} \authorsc{VAN} S\authorsc{TRAATEN}}
\\
\multicolumn{4}{c}{(\textit{Editors})} \\
\multicolumn{4}{c}{
  R\authorsc{ICHARD} K\authorsc{ELSEY}, J\authorsc{ONATHAN} R\authorsc{EES}} \\
\multicolumn{4}{c}{(\textit{Editors, Revised$^5$ Report on the Algorithmic Language Scheme})} \\[1ex]
\multicolumn{4}{c}{\bf 18 January 2007}
\end{tabular}
$$



\end{center}

\chapter*{Summary}
\medskip

The report gives a defining description of the programming language
Scheme.  Scheme is a statically scoped and properly tail-recursive
dialect of the Lisp programming language invented by Guy Lewis
Steele~Jr.\ and Gerald Jay~Sussman.  It was designed to have an
exceptionally clear and simple semantics and few different ways to
form expressions.  A wide variety of programming paradigms, including
imperative, functional, and message passing styles, find convenient
expression in Scheme.

The introduction offers a brief history of the language and of
the report.  It also gives a short introduction to the basic concepts
of the language.

Chapter~\ref{numbertypeschapter} explains Scheme's number types.
Chapter~\ref{readsyntaxchapter} defines the read syntax of Scheme
programs.  Chapter~\ref{basicchapter} presents the fundamental
semantic ideas of the language.  Chapter~\ref{terminologychapter}
defines notational conventions used in the rest of the report.
Chapters~\ref{librarychapter} and \ref{programchapter} describe
libraries and top-level programs, the basic organizational units of Scheme
programs.  Chapter~\ref{expansionchapter} explains the expansion
process for Scheme code.

Chapter~\ref{baselibrarychapter} explains the Scheme base library which
contains the fundamental forms useful to programmers.

Appendix~\ref{formalsemanticschapter} provides a formal semantics for a
core of Scheme.  Appendix~\ref{derivedformsappendix} contains
definitions for some of the derived forms described in the report.

The report concludes with a list of references and an
alphabetic index.

This report is accompanied by a report describing standard
libraries~\cite{R6RS-libraries}; references to this document are
identified by designations such as ``library section'' or ``library
chapter''.

\bigskip

\begin{center}
{\large \bf
*** DRAFT*** \\
}\end{center}

This is a preliminary draft.  It is intended to reflect the decisions
taken by the editors' committee, but contains many mistakes,
ambiguities and inconsistencies.

}]

\texonly\clearpage\endtexonly

\chapter*{Contents}
\addvspace{3.5pt}                  % don't shrink this gap
\renewcommand{\tocshrink}{-4.0pt}  % value determined experimentally
{
\tableofcontents
}

\vfill
\eject


\clearextrapart{Introduction}

\label{historysection}

Programming languages should be designed not by piling feature on top of
feature, but by removing the weaknesses and restrictions that make additional
features appear necessary.  Scheme demonstrates that a very small number
of rules for forming expressions, with no restrictions on how they are
composed, suffice to form a practical and efficient programming language
that is flexible enough to support most of the major programming
paradigms in use today.

Scheme
was one of the first programming languages to incorporate first class
procedures as in the lambda calculus, thereby proving the usefulness of
static scope rules and block structure in a dynamically typed language.
Scheme was the first major dialect of Lisp to distinguish procedures
from lambda expressions and symbols, to use a single lexical
environment for all variables, and to evaluate the operator position
of a procedure call in the same way as an operand position.  By relying
entirely on procedure calls to express iteration, Scheme emphasized the
fact that tail-recursive procedure calls are essentially gotos that
pass arguments.  Scheme was the first widely used programming language to
embrace first class escape procedures, from which all previously known
sequential control structures can be synthesized.  A subsequent
version of Scheme introduced the concept of exact and inexact numbers,
an extension of Common Lisp's generic arithmetic.
More recently, Scheme became the first programming language to support
hygienic macros, which permit the syntax of a block-structured language
to be extended in a consistent and reliable manner.

Numerical computation was long neglected by the Lisp
community.  Until Common Lisp there was no carefully thought out
strategy for organizing numerical computation, and with the exception of
the MacLisp system \cite{Pitman83} little effort was made to
execute numerical code efficiently.
The Scheme reports recognized the excellent work
of the Common Lisp committee and accepted many of their recommendations,
while simplifying and generalizing in some ways
consistent with the purposes of Scheme.

\todo{Ramsdell:
I would like to make a few comments on presentation.  The most
important comment is about section organization.  Newspaper writers
spend most of their time writing the first three paragraphs of any
article.  This part of the article is often the only part read by
readers, and is important in enticing readers to continue.  In the
same way, The first page is most likely to be the only page read by
many SIGPLAN readers.  If I had my choice of what I would ask them to
read, it would be the material in section 1.1, the Semantics section
that notes that scheme is lexically scoped, tail recursive, weakly
typed, ... etc.  I would expand on the discussion on continuations,
as they represent one important difference between Scheme and other
languages.  The introduction, with its history of scheme, its history
of scheme reports and meetings, and acknowledgements giving names of
people that the reader will not likely know, is not that one page I
would like all to read.  I suggest moving the history to the back of
the report, and use the first couple of pages to convince the reader
that the language documented in this report is worth studying.
}

\subsection*{Background}

\vest The first description of Scheme was written by Gerald Jay
Sussman and Guy Lewis Steele Jr.\ in
1975~\cite{Scheme75}.  A revised report by Steele and
Sussman~\cite{Scheme78}
appeared in 1978 and described the evolution
of the language as its MIT implementation was upgraded to support an
innovative compiler~\cite{Rabbit}.  Three distinct projects began in
1981 and 1982 to use variants of Scheme for courses at MIT, Yale, and
Indiana University~\cite{Rees82,MITScheme,Scheme311}.  An introductory
computer science textbook using Scheme was published in
1984~\cite{SICP}.  A number of textbooks describing and using Scheme
have been published since~\cite{tspl3}.

\vest As Scheme became more widespread,
local dialects began to diverge until students and researchers
occasionally found it difficult to understand code written at other
sites.
Fifteen representatives of the major implementations of Scheme therefore
met in October 1984 to work toward a better and more widely accepted
standard for Scheme.
%Participating in this workshop were Hal Abelson, Norman Adams, David
%Bartley, Gary Brooks, William Clinger, Daniel Friedman, Robert Halstead,
%Chris Hanson, Christopher Haynes, Eugene Kohlbecker, Don Oxley, Jonathan Rees,
%Guillermo Rozas, Gerald Jay Sussman, and Mitchell Wand.  Kent Pitman
%made valuable contributions to the agenda for the workshop but was
%unable to attend the sessions.
%
%Subsequent electronic mail discussions and committee work completed the
%definition of the language.
%Gerry Sussman drafted the section on numbers, Chris Hanson drafted the
%sections on characters and strings, and Gary Brooks and William Clinger
%drafted the sections on input and output.
%William Clinger recorded the decisions of the workshop and
%compiled the pieces into a coherent document.
%The ``Revised revised report on Scheme''~\cite{RRRS}
Their report~\cite{RRRS}, edited by Will Clinger,
was published at MIT and Indiana University in the summer of 1985.
Further revision took place in the spring of 1986~\cite{R3RS} (edited
by Jonathan Rees and Will Clinger),
and in the spring of 1988~\cite{R4RS} (also edited by Will Clinger and
Jonathan Rees).  Another revision published in 1998, edited
by Richard Kelsey, Will Clinger and Jonathan Rees,
reflected further revisions agreed upon in a meeting at Xerox PARC in
June 1992~\cite{R5RS}.

Attendees of the Scheme Workshop in Pittsburgh in October 2002 formed
a Strategy Committee to discuss a process for producing new revisions
of the report.  The strategy committee drafted a charter for Scheme
standardization.  This charter, together with a process for selecting
editorial committees for producing new revisions for the report, was
confirmed by the attendees of the Scheme Workshop in Boston in
November 2003.  Subsequently, a Steering Committee according to the
charter was selected, consisting of Alan Bawden, Guy L.\ Steele Jr.,
and Mitch Wand.  An editors' committee charged with producing this report was
also formed at the end of 2003, consisting of Will Clinger,
R.\ Kent Dybvig, Marc Feeley, Matthew Flatt, Richard Kelsey, Manuel
Serrano, and Mike Sperber, with Marc Feeley acting as Editor-in-Chief.
Richard Kelsey resigned from the committee in April 2005, and was
replaced by Anton van Straaten.  
Marc Feeley and Manuel Serrano
resigned from the committee in January 2006.  Subsequently, the charter
was revised to reduce the size of the editors' committee to five and
to replace the office of Editor-in-Chief by a Chair and a Project
Editor~\cite{SchemeCharter2006}.  R.\ Kent Dybvig served as Chair, and
Mike Sperber served as Project Editor.
Parts of the report were posted as Scheme Requests for Implementation
(SRFIs) and discussed by the community before being revised and finalized for
the report~\cite{srfi75,srfi76,srfi77,srfi83,srfi93}.
Jacob Matthews and Robby
Findler wrote the operational semantics for the language core.

\medskip

We intend this report to belong to the entire Scheme community, and so
we grant permission to copy it in whole or in part without fee.  In
particular, we encourage implementors of Scheme to use this report as
a starting point for manuals and other documentation, modifying it as
necessary.

\subsection*{Guiding principles}

To help guide the standardization effort, the editors have adopted a
set of principles, presented below.
Like the Scheme language defined in \rrs{5}~\cite{R5RS}, the language described
in this report is intended to:

\begin{itemize}
\item allow programmers to read each other's code, and allow
  development of portable programs that can be executed in any
  conforming implementation of Scheme;

\item derive its power from simplicity, a small number of generally
  useful core syntactic forms and procedures, and no unnecessary
  restrictions on how they are composed;
  
\item allow programs to define new procedures and new hygienic
  syntactic forms;
  
\item support the representation of program source code as data;
  
\item make procedure calls powerful enough to express any form of
  sequential control, and allow programs to perform non-local control
  operations without the use of global program transformations;
  
\item allow interesting, purely functional programs to run indefinitely
  without terminating or running out of memory on finite-memory
  machines;
  
\item allow educators to use the language to teach programming
  effectively, at various levels and with a variety of pedagogical
  approaches; and

\item allow researchers to use the language to explore the design,
  implementation, and semantics of programming languages.
\end{itemize}

In addition, this report is intended to:

\begin{itemize}
\item allow programmers to create and distribute substantial programs
  and libraries, e.g., SRFI implementations, that run without
  modification in a variety of Scheme implementations;
  
\item support procedural, syntactic, and data abstraction more fully
  by allowing programs to define hygiene-bending and hygiene-breaking
  syntactic abstractions and new unique datatypes along with
  procedures and hygienic macros in any scope;
  
\item allow programmers to rely on a level of automatic run-time type
  and bounds checking sufficient to ensure type safety; and

\item allow implementations to generate efficient code, without
  requiring programmers to use implementation-specific operators or
  declarations.
\end{itemize}

While it was possible to write portable programs in Scheme as
described in \rrs{5}, and indeed portable Scheme programs were written
prior to this report, many Scheme programs were not, primarily because
of the lack of substantial standardized libraries and the
proliferation of implementation-specific language additions.

In general, Scheme should include building blocks that allow a wide
variety of libraries to be written, include commonly used user-level
features to enhance portability and readability of library and
application code, and exclude features that are less commonly used and
easily implemented in separate libraries.

The language described in this report is inteded to also be backward
compatible with programs written in Scheme as described in \rrs{5} to
the extent possible without compromising the above principles and
future viability of the language.  With respect to future viability,
the editors have operated under the assumption that many more Scheme
programs will be written in the future than exist in the present, so
the future programs are those with which we must be most concerned.

\subsection*{Acknowledgements}

We would like to thank the following people for their help: Eli
Barzilay, Alan Bawden, Michael
Blair, Per Bothner, Trent Buck, Thomas Bushnell, Taylor Campbell, Pascal Costanza,
John Cowan, George Carrette, Andy Cromarty, David Cuthbert, Pavel Curtis, Jeff Dalton, Olivier Danvy,
Ken Dickey, Ray Dillinger, Blake Coverett, Jed Davis, Bruce Duba, Carl Eastlund,
Sebastian Egner, Tom Emerson, Marc Feeley,
Andy Freeman, Richard Gabriel, Martin Gasbichler, Peter Gavin, Arthur A.\ Gleckler,
Aziz Ghuloum, Yekta G\"ursel, Ken Haase, Lars T Hansen,
Dave Herman, Robert Hieb, Nils M.\ Holm, Paul Hudak, Stanislav Ievlev, Aubrey Jaffer, Shiro Kawai,
Michael Lenaghan, Morry Katz, Felix Klock, Donovan Kolbly,
Marcin Kowalczyk, Chris Lindblad, Thomas Lord, Bradley
Lucier, Mark Meyer, Jim Miller, Dan Muresan, Jason Orendorff, Jim Philbin,
John Ramsdell, Jeff Read, Jorgen Schaefer, Paul Schlie, Manuel Serrano,
Mike Shaff, Olin Shivers, Jonathan Shapiro, Jens Axel S\o{}gaard,
Pinku Surana, Julie Sussman, Sam Tobin-Hochstadt,
David Van Horn, Andre van Tonder, Reinder Verlinde, Oscar Waddell,
Perry Wagle, Alan Watson, Daniel Weise, Andrew Wilcox, Henry Wu,
and Ozan Yigit.
We thank Carol Fessenden, Daniel
Friedman, and Christopher Haynes for permission to use text from the Scheme 311
version 4 reference manual.  We thank Texas Instruments, Inc.~for permission to
use text from the {\em TI Scheme Language Reference Manual}~\cite{TImanual85}.
We gladly acknowledge the influence of manuals for MIT Scheme~\cite{MITScheme},
T~\cite{Rees84}, Scheme 84~\cite{Scheme84}, Common Lisp~\cite{CLtL},
Chez Scheme~\cite{csug7}, PLT~Scheme~\cite{mzscheme352},
and Algol 60~\cite{Naur63}.

\vest We also thank Betty Dexter for the extreme effort she put into
setting this report in \TeX, and Donald Knuth for designing the program
that caused her troubles.

\vest The Artificial Intelligence Laboratory of the
Massachusetts Institute of Technology, the Computer Science
Department of Indiana University, the Computer and Information
Sciences Department of the University of Oregon, and the NEC Research
Institute supported the preparation of this report.  Support for the MIT
work was provided in part by
the Advanced Research Projects Agency of the Department of Defense under Office
of Naval Research contract N00014-80-C-0505.  Support for the Indiana
University work was provided by NSF grants NCS 83-04567 and NCS
83-03325.


%%% Local Variables: 
%%% mode: latex
%%% TeX-master: "r6rs"
%%% End: 
   \par
\vskip 2ex
\clearchaptergroupstar{Description of the language} %\unskip\vskip -2ex
% 1. Structure of the language

\chapter{Overview of Scheme}
\label{semanticchapter}

This chapter gives an overview of Scheme's semantics.  A
detailed informal semantics is the subject of
the following chapters.  For reference
purposes, section~\ref{formalsemanticssection} provides a formal
semantics of Scheme.

\vest Following Algol, Scheme is a statically scoped programming
language.  Each use of a variable is associated with a lexically
apparent binding of that variable.

\vest Scheme has latent as opposed to manifest types.  Types
are associated with values (also called objects\mainindex{object}) rather than
with variables.  (Some authors refer to languages with latent types as
weakly typed or dynamically typed languages.)  Other languages with
latent types are Python, Ruby, Smalltalk, and other dialects of Lisp.  Languages
with manifest types (sometimes referred to as strongly typed or
statically typed languages) include Algol 60, C, C\#, Java, Haskell and ML.

\vest All objects created in the course of a Scheme computation, including
procedures and continuations, have unlimited extent.
No Scheme object is ever destroyed.  The reason that
implementations of Scheme do not (usually!)\ run out of storage is that
they are permitted to reclaim the storage occupied by an object if
they can prove that the object cannot possibly matter to any future
computation.  Other languages in which most objects have unlimited
extent include C\#, Haskell, ML, Python, Ruby, Smalltalk and other Lisp dialects.

\vest Implementations of Scheme are required to be properly tail-recursive.
This allows the execution of an iterative computation in constant space,
even if the iterative computation is described by a syntactically
recursive procedure.  Thus with a properly tail-recursive implementation,
iteration can be expressed using the ordinary procedure-call
mechanics, so that special iteration constructs are useful only as
syntactic sugar.  See section~\ref{proper tail recursion}.

\vest Scheme was one of the first languages to support procedures as
objects in their own right.  Procedures can be created dynamically,
stored in data structures, returned as results of procedures, and so
on.  Other languages with these properties include Common Lisp,
Haskell, ML, Python, Ruby, and Smalltalk.

\vest One distinguishing feature of Scheme is that continuations, which
in most other languages only operate behind the scenes, also have
``first-class'' status.  Continuations are useful for implementing a
wide variety of advanced control constructs, including non-local exits,
backtracking, and coroutines.  See section~\ref{continuations}.

\vest Arguments to Scheme procedures are always passed by value, which
means that the actual argument expressions are evaluated before the
procedure gains control, whether the procedure needs the result of the
evaluation or not.  C, C\#, Common Lisp, Python, Ruby, and Smalltalk
are other languages that always pass arguments by value.  This is
distinct from the lazy-evaluation semantics of Haskell, or the
call-by-name semantics of Algol 60, where an argument expression is
not evaluated unless its value is needed by the procedure.  Note that
call-by-value refers to a different distinction than the distinction
between by-value and by-reference passing in Pascal.  In Scheme, all
data structures are passed by-reference.

\vest Scheme's model of arithmetic is designed to remain as
independent as possible of the particular ways in which numbers are
represented within a computer. In Scheme, every integer is a rational
number, every rational is a real, and every real is a complex number.
Scheme distinguishes between exact arithmetic, which corresponds to
the mathematical ideal, and inexact arithmetic on approximations.
Exact arithmetic includes arithmetic on integers, rationals and
complex numbers.

FIXME: overview of the language to be added

\section{Programs and libraries}

A Scheme consists of a set of \textit{libraries\index{library}}, each
of which defines a part of the program connected to the others through
explicitly specified exports and imports.  A library consists of a set
of import and export FIXME specifications and a body, which contains
the code defining the library.  Chapter~\ref{librarychapter} describes
the syntax and semantics for libraries.  The subsequent chapters
describe various standard libraries provided by a Scheme system.  In
particular, chapter~\ref{baselibrarychapter} describes a base
library defining most of the constructs traditionally associated with
Scheme programs.

The division between the base library and other standard libraries is
based on use, not on construction.  In particular, some facilities
that are typically implemented as ``primitives'' by a compiler or
run-time libraries rather than in terms of other standard procedures
 or syntactic forms are not part of the base library, but defined in
separate libraries.  Examples include the fixnum and flonum libraries,
the exceptions and conditions libraries, and the libraries for
records.

%%% Local Variables: 
%%% mode: latex
%%% TeX-master: "r6rs"
%%% End: 
  \par
\chapter{Numbers}
\label{numbertypeschapter}
\index{number}

Numerical computation has traditionally been neglected by the Lisp
community.  Until Common Lisp there was no carefully thought out
strategy for organizing numerical computation, and with the exception of
the MacLisp system \cite{Pitman83} little effort was made to
execute numerical code efficiently.  This report recognizes the excellent work
of the Common Lisp committee and accepts many of their recommendations.
In some ways this report simplifies and generalizes their proposals in a manner
consistent with the purposes of Scheme.

It is important to distinguish between the mathematical numbers, the
Scheme numbers that attempt to model them, the machine representations
used to implement the Scheme numbers, and notations used to write numbers.
This report uses the types \type{number}, \type{complex}, \type{real},
\type{rational}, and \type{integer} to refer to both mathematical numbers
and Scheme numbers.  Machine representations such as fixed point and
floating point are referred to by names such as \type{fixnum} and
\type{flonum}.

\section{Numerical types}
\label{numericaltypes}
\index{numerical types}

\vest Mathematically, numbers may be arranged into a tower of subtypes
in which each level is a subset of the level above it:
\begin{tabbing}
\ \ \ \ \ \ \ \ \ \=\tupe{number} \\
\> \tupe{complex} \\
\> \tupe{real} \\
\> \tupe{rational} \\
\> \tupe{integer} 
\end{tabbing}

For example, 3 is an integer.  Therefore 3 is also a rational,
a real, and a complex.  The same is true of the Scheme numbers
that model 3.  For Scheme numbers, these types are defined by the
predicates \ide{number?}, \ide{complex?}, \ide{real?}, \ide{rational?},
and \ide{integer?}.

There is no simple relationship between a number's type and its
representation inside a computer.  Although most implementations of
Scheme will offer at least two different representations of 3, these
different representations denote the same integer.

Scheme's numerical operations treat numbers as abstract data, as
independent of their representation as possible.  Although an implementation
of Scheme may use fixnum, flonum, and perhaps other representations for
numbers, this should not be apparent to a casual programmer writing
simple programs.

It is necessary, however, to distinguish between numbers that are
represented exactly and those that may not be.  For example, indexes
into data structures must be known exactly, as must some polynomial
coefficients in a symbolic algebra system.  On the other hand, the
results of measurements are inherently inexact, and irrational numbers
may be approximated by rational and therefore inexact approximations.
In order to catch uses of inexact numbers where exact numbers are
required, Scheme explicitly distinguishes exact from inexact numbers.
This distinction is orthogonal to the dimension of type.

A subrange of the exact integers is designated as the set of fixnums.
Conversely, a \defining{fixnum} is an exact integer whose value lies
within this fixnum range.  Moreover, every implementation is required
to designate a subset of its inexact reals as \defining{flonum}s, and
to convert certain external representations into flonums.  Note that
this does not imply that an implementation is required to use
floating-point representations

\section{Exactness}
\label{exactly}

\mainindex{exactness} Scheme numbers are either \type{exact} or
\type{inexact}.  A number is exact if it was written as an exact
constant or was derived from exact numbers using only exact
operations.  A number is inexact if it was written as an inexact
constant or was derived from inexact numbers.  Thus inexactness is
contagious.  

It is the programmer's responsibility to avoid using numbers with
magnitude or significand too large to be represented in the
implementation.

If two implementations produce exact results for a computation that
did not involve inexact intermediate results, the two ultimate results
will be mathematically equivalent.  This is generally not true of
computations involving inexact numbers because approximate methods
such as floating point arithmetic may be used, but it is the duty of
each implementation to make the result as close as practical to the
mathematically ideal result.

\section{Implementation restrictions}

\index{implementation restriction}\label{restrictions}

\vest Implementations of Scheme are required to implement the whole
tower of subtypes given in section~\ref{numericaltypes}.

\vest Implementations may also support only a limited range of inexact numbers of
any type, subject to the requirements of this section.  For example,
an implementation that uses flonums to represent all its
inexact real numbers may
limit the range of inexact reals (and therefore
the range of inexact integers and rationals)
to the dynamic range of the flonum format.
Furthermore
the gaps between the representable inexact integers and
rationals are
likely to be very large in such an implementation as the limits of this
range are approached.

\vest Implementations are required to support
exact integers and exact rationals of
practically unlimited size and precision, and to implement the
above procedures and the {\cf /} procedure in
such a way that they always return exact results when given exact
arguments.

\vest An implementation may use floating point and other approximate 
representation strategies for \tupe{inexact} numbers.
This report recommends, but does not require, that the IEEE 
floating point standards be followed by implementations that use
flonum representations, and that implementations using
other representations should match or exceed the precision achievable
using these floating point standards~\cite{IEEE}.

\vest In particular, implementations that use flonum representations
must follow these rules: A flonum result
must be represented with at least as much precision as is used to express any of
the inexact arguments to that operation.  It is desirable (but not required) for
potentially inexact operations such as {\cf sqrt}, when applied to exact
arguments, to produce exact answers whenever possible (for example the
square root of an exact 4 ought to be an exact 2).
If, however, an
exact number is operated upon so as to produce an inexact result
(as by {\cf sqrt}), and if the result is represented as a flonum, then
the most precise flonum format available must be used; but if the result
is represented in some other way then the representation must have at least as
much precision as the most precise \tupe{flonum} format available.

\section{Infinities and NaNs}

Positive infinity is regarded as a real (but not rational) number,
whose value is indeterminate but greater than all rational numbers.
Negative infinity is regarded as a real (but not rational) number,
whose value is indeterminate but less than all rational numbers.

A NaN is regarded as a real (but not rational) number whose value is
so indeterminate that it might represent any real number, including
positive or negative infinity, and might even be greater than positive
infinity or less than negative infinity.

%%% Local Variables: 
%%% mode: latex
%%% TeX-master: "r6rs"
%%% End: 
 \par
% Lexical structure
\hyphenation{white-space}
%%\vfill\eject
\chapter{Lexical syntax and read syntax}
\label{readsyntaxchapter}

The syntax of Scheme code is organized in three levels:
%
\begin{enumerate}
\item the \textit{lexical syntax} that describes how a program text is split
  into a sequence of lexemes,
\item the \textit{read syntax}, formulated in terms of the lexical
  syntax, that structures the lexeme sequence as a sequence of
  \textit{syntactic datums\mainindex{datum}\mainindex{syntactic
      datum}}, where a syntactic datum is
    a recursively structured entity,
\item the \textit{program syntax} formulated in terms of the read
  syntax, imposing further structure and assigning meaning to
  syntactic datums.
\end{enumerate}
%
Syntactic datums (also called \textit{external
  representations\index{external representation}}) double
as a notation for data, and Scheme's \rsixlibrary{i/o ports} library
(library section~\extref{lib:portsiosection}{Port I/O})
provides the {\cf get-datum} and {\cf put-datum} procedures
for reading and writing syntactic datums, converting between their
textual representation and the corresponding values. 
Each syntactic datum uniquely determines a corresponding \defining{datum value}.
A syntactic datum can be used in a program to obtain the corresponding
datum value using {\cf quote} (see section~\ref{quote}).

The Scheme program consists of syntactic entities called
\textit{forms}\mainindex{form}, which, when they occur in source code, 
are a subset of the syntactic datums.
Consequently, Scheme's syntax has the property that any sequence of
characters that is an expression is also a syntactic datum representing
some object.  This can lead to confusion, since it may not be obvious
out of context whether a given sequence of characters is intended to
denote data or program. It is also a source of power, since it
facilitates writing programs such as interpreters and compilers that
treat programs as data (or vice versa).

A datum value may have several different external representations.
For example, both ``{\tt \#e28.000}'' and
``{\tt\#x1c}'' are syntactic datums representing the exact integer 28,
and the syntactic datums ``{\tt(8 13)}'', ``{\tt( 08 13 )}'', ``{\tt(8 .\
  (13 .\ ()))}''
all represent a list containing the integers 8 and 13. 
Syntactic datums that denote equal objects (in the sense of {\cf
  equal?}; see section~\ref{equal?}) are always equivalent 
as forms of a program.

Because of the close correspondence between syntactic datums and datum
values, this report sometimes uses the term \defining{datum} to denote
either a syntactic datum or a datum value when the exact meaning
is apparent from the context.

An implementation is not permitted to extend the lexical or read syntax in
any way, with one exception: it need not treat the syntax
{\cf \sharpsign{}!\meta{identifier}}, for any \meta{identifier} (see
section~\ref{identifiersection}) that is not {\cf r6rs}, as a syntax
violation, and it may use specific {\cf \sharpsign{}!}-prefixed
identifiers as flags indicating that subsequent input contains extensions
to the standard lexical or read syntax. 
The syntax {\cf \sharpsign{}!r6rs} may be used to signify that
the input afterward is written with the lexical syntax and
read syntax described by
this report when no other {\cf \sharpsign{}!\meta{identifier}} appears;
{\cf \sharpsign{}!r6rs} is otherwise treated as a comment; see section~\ref{whitespaceandcomments}.

This chapter overviews and provides formal accounts of the lexical
syntax and the read syntax.

\section{Notation}
\label{BNF}

The formal syntax for Scheme is written in an extended BNF.
Non-terminals are written using angle brackets.  Case is insignificant
for non-terminal names.

All spaces in the grammar are for legibility.
\meta{Empty} stands for the empty string.

The following extensions to BNF are used to make the description more
concise:  \arbno{\meta{thing}} means zero or more occurrences of
\meta{thing}, and \atleastone{\meta{thing}} means at least one
\meta{thing}.

Some non-terminal names refer to the Unicode scalar values of the same
name: \meta{character tabulation} (U+0009), \meta{linefeed} (U+000A),
\meta{carriage return} (U+000D), \meta{line tabulation} (U+000B),
\meta{form feed} (U+000C), \meta{carriage return} (U+000D),
\meta{space} (U+0020), \meta{next line} (U+0085), \meta{line
  separator} (U+2028), and \meta{paragraph separator} (U+2029).

\section{Lexical syntax}
\label{lexicalsyntaxsection}

The lexical syntax determines how a character sequence is split into a
sequence of lexemes\index{lexeme}, omitting non-significant portions
such as comments and whitespace.  The character sequence is assumed to
be text according to the Unicode standard~\cite{Unicode}.  Some of
the lexemes, such as numbers, identifiers, strings etc., of the lexical
syntax are syntactic datums in the read syntax, and thus represent data.
Besides the formal account of the syntax, this section also describes
what datum values are denoted by these syntactic datums.

The lexical syntax, in the description of comments, contains
a forward reference to \meta{datum}, which is described as part of the
read syntax.  Being comments, however, these \meta{datum}s do not play
a significant role in the syntax.

Case is significant except in boolean datums, number datums, and
hexadecimal numbers denoting Unicode scalar values.  For example, {\cf \#x1A}
and {\cf \#X1a} are equivalent.  The identifier {\cf Foo} is, however,
distinct from the identifier {\cf FOO}.

\subsection{Formal account}
\label{lexicalgrammarsection}

\meta{Interlexeme space} may occur on either side of any lexeme, but not
within a lexeme.

\vest Identifiers, numbers, characters, booleans, and dot must be terminated
by a \meta{delimiter} (e.g., parenthesis, space, or comment) or by the
end of the input.

The following two characters are reserved for future extensions to the
language: {\tt \verb"{" \verb"}"}

\begin{grammar}%
\meta{lexeme} \: \meta{identifier} \| \meta{boolean} \| \meta{number}\index{identifier}
\>  \| \meta{character} \| \meta{string}
\>  \| ( \| ) \| \openbracket{} \| \closedbracket{} \| \sharpsign( \| \sharpsign{}vu8( | \singlequote{} \| \backquote{} \| , \| ,@ \| {\bf.}
\>  \| \sharpsign\singlequote{} \| \sharpsign\backquote{} \| \sharpsign, \| \sharpsign,@
\meta{delimiter} \: \meta{interlexeme space} \| ( \| ) \| \openbracket{} \| \closedbracket{} \| " \| ;
\meta{whitespace} \: \meta{character tabulation}
\> \| \meta{linefeed} \| \meta{line tabulation} \| \meta{form feed}
\> \| \meta{carriage return} \| \meta{next line}
\> \| \meta{any character whose category is Zs, Zl, or Zp}
\meta{line ending} \: \meta{linefeed} \| \meta{carriage return}
\> \| \meta{carriage return} \meta{linefeed} \| \meta{next line}
\> \| \meta{carriage return} \meta{next line} \| \meta{line separator}
\meta{comment} \: ; \= $\langle$\rm all subsequent characters up to a
                    \>\ \rm \meta{line ending} or \meta{paragraph separator}$\rangle$\index{comment}
\qquad \= \| \meta{nested comment}
\> \| \#; \meta{datum}
\> \| \#!r6rs
\meta{nested comment} \: \#| \= \meta{comment text}
\> \arbno{\meta{comment cont}} |\#
\meta{comment text} \: \= $\langle$\rm character sequence not containing
\>\ \rm {\tt \#|} or {\tt |\#}$\rangle$
\meta{comment cont} \: \meta{nested comment} \meta{comment text}
\meta{atmosphere} \: \meta{whitespace} \| \meta{comment}
\meta{interlexeme space} \: \arbno{\meta{atmosphere}}%
\end{grammar}

\label{extendedalphas}
\label{identifiersyntax}

% This is a kludge, but \multicolumn doesn't work in tabbing environments.
\setbox0\hbox{\cf\meta{variable} \goesto{} $\langle$}

\begin{grammar}%
\meta{identifier} \: \meta{initial} \arbno{\meta{subsequent}}
 \>  \| \meta{peculiar identifier}
\meta{initial} \: \meta{constituent} \| \meta{special initial}
 \> \| \meta{inline hex escape}
\meta{letter} \:  a \| b \| c \| ... \| z
\> \| A \| B \| C \| ... \| Z
\meta{constituent} \: \meta{letter}
 \> \| $\langle${\rm any character whose Unicode scalar value is greater than}
 \> \quad {\rm 127, and whose category is Lu, Ll, Lt, Lm, Lo, Mn,}
 \> \quad {\rm Nl, No, Pd, Pc, Po, Sc, Sm, Sk, So, or Co}$\rangle$
\meta{special initial} \: ! \| \$ \| \% \| \verb"&" \| * \| / \| : \| < \| =
 \>  \| > \| ? \| \verb"^" \| \verb"_" \| \verb"~"
\meta{subsequent} \: \meta{initial} \| \meta{digit}
 \>  \| \meta{any character whose category is Nd, Mc, or Me}
 \>  \| \meta{special subsequent}
\meta{digit} \: 0 \| 1 \| 2 \| 3 \| 4 \| 5 \| 6 \| 7 \| 8 \| 9
\meta{hex digit} \: \meta{digit}
 \> \| a \| A \| b \| B \| c \| C \| d \| D \| e \| E \| f \| F
\meta{special subsequent} \: + \| - \| .\ \| @
\meta{inline hex escape} \: \backwhack{}x\meta{hex scalar value};
\meta{hex scalar value} \: \atleastone{\meta{hex digit}}
\meta{peculiar identifier} \: + \| - \| ... \| -> \arbno{\meta{subsequent}}
\meta{boolean} \: \schtrue{} \| \#T \| \schfalse{} \| \#F
\meta{character} \: \#\backwhack{}\meta{any character}
 \>  \| \#\backwhack{}\meta{character name}
 \>  \| \#\backwhack{}x\meta{hex scalar value}
\meta{character name} \: nul \| alarm \| backspace \| tab
\> \| linefeed \| vtab \| page \| return \| esc
\> \| space \| delete
\meta{string} \: " \arbno{\meta{string element}} "
\meta{string element} \: \meta{any character other than \doublequote{} or \backwhack}
 \> \| \meta{line ending}
 \> \| \backwhack{}a \| \backwhack{}b \| \backwhack{}t \| \backwhack{}n \| \backwhack{}v \| \backwhack{}f \| \backwhack{}r
 \>  \| \backwhack\doublequote{} \| \backwhack\backwhack 
 \>  \| \backwhack\meta{line ending} \| \backwhack\meta{space}
 \>  \| \meta{inline hex escape}%
\end{grammar}

A \meta{hex scalar value} represents a Unicode scalar value
between 0 and \sharpsign{}x10FFFF, excluding the range
$\left[\sharpsign{}x\textrm{D800}, \sharpsign{}x\textrm{DFFF}\right]$.

\label{numbersyntax}%
The rules for \meta{num $R$}, \meta{complex $R$}, \meta{real
$R$}, \meta{ureal $R$}, \meta{uinteger $R$}, and \meta{prefix $R$} below
should be replicated for \hbox{$R = 2, 8, 10,$}
and $16$.  There are no rules for \meta{decimal $2$}, \meta{decimal
$8$}, and \meta{decimal $16$}, which means that numbers containing
decimal points or exponents must be in decimal radix.

\begin{grammar}%
\meta{number} \: \meta{num $2$} \| \meta{num $8$}
   \>  \| \meta{num $10$} \| \meta{num $16$}
\meta{num $R$} \: \meta{prefix $R$} \meta{complex $R$}
\meta{complex $R$} \: %
         \meta{real $R$} %
      \| \meta{real $R$} @ \meta{real $R$}
   \> \| \meta{real $R$} + \meta{ureal $R$} i %
      \| \meta{real $R$} - \meta{ureal $R$} i
   \> \| \meta{real $R$} + \meta{naninf} i %
      \| \meta{real $R$} - \meta{naninf} i
   \> \| \meta{real $R$} + i %
      \| \meta{real $R$} - i
   \> \| + \meta{ureal $R$} i %
      \| - \meta{ureal $R$} i 
   \> \| + \meta{naninf} i %
      \| - \meta{naninf} i
   \> \| + i %
      \| - i
\meta{real $R$} \: \meta{sign} \meta{ureal $R$}
  \> \| + \meta{naninf} \| - \meta{naninf}
\meta{naninf} \: nan.0 \| inf.0
\meta{ureal $R$} \: %
         \meta{uinteger $R$}
   \> \| \meta{uinteger $R$} / \meta{uinteger $R$}
   \> \| \meta{decimal $R$} \meta{mantissa width}
\meta{decimal $10$} \: %
         \meta{uinteger $10$} \meta{suffix}
   \> \| . \atleastone{\meta{digit $10$}} \arbno{\#} \meta{suffix}
   \> \| \atleastone{\meta{digit $10$}} . \arbno{\meta{digit $10$}} \arbno{\#} \meta{suffix}
   \> \| \atleastone{\meta{digit $10$}} \atleastone{\#} . \arbno{\#} \meta{suffix}
\meta{uinteger $R$} \: \atleastone{\meta{digit $R$}} \arbno{\#}
\meta{prefix $R$} \: %
         \meta{radix $R$} \meta{exactness}
   \> \| \meta{exactness} \meta{radix $R$}
\end{grammar}

\begin{grammar}%
\meta{suffix} \: \meta{empty} 
   \> \| \meta{exponent marker} \meta{sign} \atleastone{\meta{digit $10$}}
\meta{exponent marker} \: e \| E \| s \| S \| f \| F
   \> \| d \| D \| l \| L
\meta{mantissa width} \: \meta{empty}
   \> \| | \atleastone{\meta{digit 10}}
\meta{sign} \: \meta{empty}  \| + \|  -
\meta{exactness} \: \meta{empty}
   \> \| \#i\sharpindex{i} \| \#I \| \#e\sharpindex{e} \| \#E
\meta{radix 2} \: \#b\sharpindex{b} \| \#B
\meta{radix 8} \: \#o\sharpindex{o} \| \#O
\meta{radix 10} \: \meta{empty} \| \#d \| \#D
\meta{radix 16} \: \#x\sharpindex{x} \| \#X
\meta{digit 2} \: 0 \| 1
\meta{digit 8} \: 0 \| 1 \| 2 \| 3 \| 4 \| 5 \| 6 \| 7
\meta{digit 10} \: \meta{digit}
\meta{digit 16} \: \meta{hex digit}
\end{grammar}

\subsection{Line endings}
\label{lineendings}

Line endings are significant in Scheme in single-line comments (see
section~\ref{whitespaceandcomments}) and within string literals.  In
Scheme source code, any of the line endings in \meta{line ending}
marks the end of a line.  Moreover, the two-character line endings
\meta{carriage return} \meta{linefeed} and \meta{carriage return}
\meta{next line} each count as a single line ending.

In a string literal, a line ending not preceded by a {\cf\backwhack}
denotes a linefeed character, which is the standard line-ending
character of Scheme.

\subsection{Whitespace and comments}
\label{whitespaceandcomments}

\defining{Whitespace} characters are spaces, linefeeds,
carriage returns, character tabulations, form feeds, line tabulations,
and any other character whose category is Zs, Zl, or Zp.
Whitespace is used for improved readability and
as necessary to separate lexemes from each other.  Whitespace may
occur between any two lexemes,
but not within a lexeme.  Whitespace may also occur inside a string,
where it is significant.

The lexical syntax includes several comment forms. In all cases,
comments are invisible to Scheme, except that they act as delimiters,
so, for example, a comment cannot appear in the middle of an identifier or number.

A semicolon ({\tt;}) indicates the start of a line
comment.\mainindex{comment}\mainschindex{;} The comment continues to
the end of the line on which the semicolon appears.

Another way to indicate a comment is to prefix a \hyper{datum}
(cf.\ Section~\ref{datumsyntax}) with {\tt \#;}\sharpindex{;}, possibly with
\meta{interlexeme space} before the \hyper{datum}.  The comment consists of
the comment prefix {\tt \#;} and the \hyper{datum} together.  (This
notation is useful for ``commenting out'' sections of code.)

Block comments may be indicated with properly nested {\tt
  \#|}\index{#"|@\texttt{\sharpsign\verticalbar}}\index{"|#@\texttt{\verticalbar\sharpsign}}
and {\tt |\#} pairs.

\begin{scheme}
\#|
   The FACT procedure computes the factorial
   of a non-negative integer.
|\#
(define fact
  (lambda (n)
    ;; base case
    (if (= n 0)
        \#;(= n 1)
        1       ; identity of *
        (* n (fact (- n 1))))))%
\end{scheme}

\begin{rationale}
  {\cf \sharpsign\verticalbar} \ldots {\cf \verticalbar\sharpsign}
  cannot be used to comment out an arbitrary datum or set of datums;
  it works only when none of the datums include a string with an
  unmatched {\cf \sharpsign\verticalbar} or {\cf
    \verticalbar\sharpsign} character sequence.  While {\cf
    \sharpsign\verticalbar} \ldots {\cf \verticalbar\sharpsign} and
  {\cf ;} can often be used, with care, to comment out a datum, only
  {\sharpsign;} allows the programmer to clearly communicate that a
  single datum has been commented out, as opposed to a block or line
  of arbitrary text.
\end{rationale}

The lexeme {\cf \sharpsign{}!r6rs}, which signifies that the program text
that follows is written with the lexical and read syntax described in this
report, is also otherwise treated as a comment.

\subsection{Identifiers}
\label{identifiersection}

Most identifiers\mainindex{identifier} allowed by other programming
languages are also acceptable to Scheme.  In general,
a sequence of letters, digits, and ``extended alphabetic
characters'' is
an identifier when it begins with a character that cannot begin a number.
In addition, \ide{+}, \ide{-}, and \ide{...} are identifiers, as is
a sequence of letters, digits, and extended alphabetic
characters that begins with the two-character sequence \ide{->}.
Here are some examples of identifiers:

\begin{scheme}
lambda         q                soup
list->vector   {+}                V17a
<=             a34kTMNs         ->-
the-word-recursion-has-many-meanings%
\end{scheme}

Extended alphabetic characters may be used within identifiers as if
they were letters.  The following are extended alphabetic characters:

\begin{scheme}
!\ \$ \% \verb"&" * + - . / :\ < = > ? @ \verb"^" \verb"_" \verb"~" %
\end{scheme}

Moreover, all characters whose Unicode scalar values are greater than 127 and
whose Unicode category is Lu, Ll, Lt, Lm, Lo, Mn, Mc, Me, Nd, Nl, No, Pd,
Pc, Po, Sc, Sm, Sk, So, or Co can be used within identifiers.
In addtion, any character can be used within an identifier
when denoted via an \meta{inline hex escape}.  For example, the
identifier \verb|H\x65;llo| is the same as the identifier
\verb|Hello|, and the identifier \verb|\x3BB;| is the same as the
identifier $\lambda$.

Any identifier may be used as a variable\index{variable} or as a
syntactic keyword\index{syntactic keyword} (see
sections~\ref{variablesection} and~\ref{macrosection}) in a Scheme
program.
Any identifier may also be used as a syntactic datum, in which case it
denotes a \textit{symbol}\index{symbol} (see section~\ref{symbolsection}).

\subsection{Booleans}

The standard boolean objects for true and false are written as
\schtrue{} and \schfalse.\sharpindex{t}\sharpindex{f}

\subsection{Characters}

Characters are written using the notation
\sharpsign\backwhack\hyper{character}\index{#\@\texttt{\sharpsign\backwhack}} or
\sharpsign\backwhack\hyper{character name} or
\sharpsign\backwhack{}x\meta{hex scalar value}.

For example:

\begin{schemenoindent}
\#\backwhack{}a          \ev \textrm{lower case letter a}
\#\backwhack{}A          \ev \textrm{upper case letter A}
\#\backwhack{}(          \ev \textrm{left parenthesis}
\#\backwhack{}           \ev \textrm{space character}
\#\backwhack{}nul        \ev \textrm{U+0000}
\#\backwhack{}alarm      \ev \textrm{U+0007}
\#\backwhack{}backspace  \ev \textrm{U+0008}
\#\backwhack{}tab        \ev \textrm{U+0009}
\#\backwhack{}linefeed   \ev \textrm{U+000A}
\#\backwhack{}vtab       \ev \textrm{U+000B}
\#\backwhack{}page       \ev \textrm{U+000C}
\#\backwhack{}return     \ev \textrm{U+000D}
\#\backwhack{}esc        \ev \textrm{U+001B}
\#\backwhack{}space      \ev \textrm{U+0020}
\>\>; \textrm{preferred way to write a space}
\#\backwhack{}delete     \ev \textrm{U+007F}

\#\backwhack{}xFF        \ev \textrm{U+00FF}
\#\backwhack{}x03BB      \ev \textrm{U+03BB}
\#\backwhack{}x00006587  \ev \textrm{U+6587}
\#\backwhack{}\(\lambda\) \ev \textrm{U+03BB}

\#\backwhack{}x0001z     \ev \exception{\&lexical}
\#\backwhack{}\(\lambda\)x         \ev \exception{\&lexical}
\#\backwhack{}alarmx     \ev \exception{\&lexical}
\#\backwhack{}alarm x    \ev \textrm{U+0007}
\>\>; \textrm{followed by {\cf{}x}}
\#\backwhack{}Alarm      \ev \exception{\&lexical}
\#\backwhack{}alert      \ev \exception{\&lexical}
\#\backwhack{}xA         \ev \textrm{U+000A}
\#\backwhack{}xFF        \ev \textrm{U+00FF}
\#\backwhack{}xff        \ev \textrm{U+00FF}
\#\backwhack{}x ff       \ev \textrm{U+0078}
\>\>; \textrm{followed by another datum, {\cf{}ff}}
\#\backwhack{}x(ff)      \ev \textrm{U+0078}
\>\>; \textrm{followed by another datum,}
\>\>; \textrm{a parenthesized {\cf{}ff}}
\#\backwhack{}(x)        \ev \exception{\&lexical}
\#\backwhack{}(x         \ev \exception{\&lexical}
\#\backwhack{}((x)       \ev \textrm{U+0028}
\>\>; \textrm{followed by another datum,}
\>\>; \textrm{parenthesized {\cf{}x}}
\#\backwhack{}x00110000  \ev \exception{\&lexical}
\>\>; \textrm{out of range}
\#\backwhack{}x000000001 \ev \textrm{U+0001}  
\#\backwhack{}xD800      \ev \exception{\&lexical}
\>\>; \textrm{in excluded range}
\end{schemenoindent}

(The notation \exception{\&lexical} means that the line in question is
a lexical syntax violation.)

Case is significant in \sharpsign\backwhack\hyper{character}, and in
\sharpsign\backwhack{\rm$\langle$character name$\rangle$}, % \hyper doesn't allow a linebreak
but not in \sharpsign\backwhack{}x\meta{hex scalar value}.  
A \meta{character} must be followed by a \meta{delimiter} or by the end of the input.
This rule resolves various ambiguous cases involving named characters,
requiring, for
example, the sequence of characters ``{\tt\sharpsign\backwhack space}''
to be interpreted as the space character rather than as
the character ``{\tt\sharpsign\backwhack s}'' followed
by the identifier ``{\tt pace}''.

\subsection{Strings}

\vest String are written as sequences of characters enclosed within doublequotes
({\cf "}).  Within a string literal, various escape
sequences\mainindex{escape sequence} denote characters other than
themselves.  Escape sequences always start with a backslash (\backwhack{}):

\begin{itemize}
\item{\tt \backwhack{}a} : alarm, U+0007
\item{\tt \backwhack{}b} : backspace, U+0008 
\item{\tt \backwhack{}t} : character tabulation, U+0009 
\item{\tt \backwhack{}n} : linefeed, U+000A 
\item{\tt \backwhack{}v} : line tabulation, U+000B 
\item{\tt \backwhack{}f} : formfeed, U+000C 
\item{\tt \backwhack{}r} : return, U+000D 
\item{\tt \backwhack{}}\verb|"| : doublequote, U+0022 
\item{\tt \backwhack{}\backwhack{}} : backslash, U+005C 
\item{\tt \backwhack{}\hyper{linefeed}} : nothing
\item{\tt \backwhack{}\hyper{space}} : space, U+0020 (useful for terminating the
  previous escape sequence before continuing with whitespace)
\item{\tt \backwhack{}x\meta{hex scalar value};} : specified character (note the
  terminating semi-colon).
\end{itemize}

These escape sequences are case-sensitive, except that the alphabetic
digits of a \meta{hex scalar value} can be uppercase or lowercase.

Any other character in a string after a backslash is an error. Except
for a line ending, any
character outside of an escape sequence and not a doublequote stands
for itself in the string literal. For example the single-character
string {\tt "$\lambda$"} (doublequote, a lower case lambda, doublequote)
denotes the same string literal as {\tt "\backwhack{}x03bb;"}.
A line ending stands for a linefeed character.

Examples:

\begin{schemenoindent}
"abc" \ev  \textrm{U+0061, U+0062, U+0063}
"\backwhack{}x41;bc" \ev  "Abc" ; \textrm{U+0041, U+0062, U+0063}
"\backwhack{}x41; bc" \ev "A bc"
\>\>; \textrm{U+0041, U+0020, U+0062, U+0063}
"\backwhack{}x41bc;" \ev  \textrm{U+41BC}
"\backwhack{}x41" \ev \exception{\&lexical}
"\backwhack{}x;" \ev \exception{\&lexical}
"\backwhack{}x41bx;" \ev \exception{\&lexical}
"\backwhack{}x00000041;" \ev  "A" ; \textrm{U+0041}
"\backwhack{}x0010FFFF;" \ev \textrm{U+10FFFF}
"\backwhack{}x00110000;" \ev  \exception{\&lexical}
\>\>; \textrm{out of range}
"\backwhack{}x000000001;" \ev \textrm{U+0001}
"\backwhack{}xD800;" \ev \exception{\&lexical}
\>\>; \textrm{in excluded range}
"A
bc" \ev \textrm{U+0041, U+000A, U+0062, U+0063}
\>\>; \textrm{if no space occurs after the {\cf{}A}}
\end{schemenoindent}
  
\subsection{Numbers}
\label{numbernotations}

The syntax of written representations for numbers is described
formally by the \meta{number} rule in the formal grammar.
Case is not significant in numerical constants.

A number may be written in binary, octal, decimal, or
hexadecimal by the use of a radix prefix.  The radix prefixes are {\cf
\#b}\sharpindex{b} (binary), {\cf \#o}\sharpindex{o} (octal), {\cf
\#d}\sharpindex{d} (decimal), and {\cf \#x}\sharpindex{x} (hexadecimal).  With
no radix prefix, a number is assumed to be expressed in decimal.

A
numerical constant may be specified to be either exact or
inexact by a prefix.  The prefixes are {\cf \#e}\sharpindex{e}
for exact, and {\cf \#i}\sharpindex{i} for inexact.  An exactness
prefix may appear before or after any radix prefix that is used.  If
the written representation of a number has no exactness prefix, the
constant is
inexact if it contains a decimal point, an
exponent, a ``\sharpsign'' character in the place of a digit, or
a nonempty mantissa width;
otherwise it is exact.

In systems with inexact numbers
of varying precisions, it may be useful to specify
the precision of a constant.  For this purpose, numerical constants
may be written with an exponent marker that indicates the
desired precision of the inexact
representation.  The letters {\cf s}, {\cf f},
{\cf d}, and {\cf l} specify the use of \var{short}, \var{single},
\var{double}, and \var{long} precision, respectively.  (When fewer
than four internal
inexact
representations exist, the four size
specifications are mapped onto those available.  For example, an
implementation with two internal representations may map short and
single together and long and double together.)  In addition, the
exponent marker {\cf e} specifies the default precision for the
implementation.  The default precision has at least as much precision
as \var{double}, but
implementations may wish to allow this default to be set by the user.

\begin{scheme}
3.1415926535898F0 {\rm{}Round to single, perhaps} 3.141593
0.6L0 {\rm{}Extend to long, perhaps} .600000000000000%
\end{scheme}

An inexact real number with nonempty mantissa width,
{\cf \var{x}|\var{p}}, denotes the best binary
floating point approximation of \var{x} using a \var{p}-bit significand. 
For example, {\cf 1.1|53} is an external
representation of the best approximation of 1.1 in IEEE double
precision.
If \var{x} is an external representation of an inexact real number
that contains no vertical bar, it should be treated as if specified
with a mantissa width of {\cf 53}.

Implementations that use binary floating point representations
of real numbers should represent {\cf \var{x}|\var{p}}
using a \var{p}-bit significand if practical, or by a greater
precision if a \var{p}-bit significand is not practical, or
by the largest available precision if \var{p} or more bits
of significand are not practical within the implementation.

\begin{note}
The precision of a significand should not be confused with the
number of bits used to represent the significand.  In the IEEE
floating point standards, for example, the significand's most
significant bit is implicit in single and double precision but
is explicit in extended precision.  Whether that bit is implicit
or explicit does not affect the mathematical precision.
In implementations that use binary floating point, the default
precision can be calculated by calling the following procedure:

\begin{scheme}
(define (precision)
  (do ((n 0 (+ n 1))
       (x 1.0 (/ x 2.0)))
    ((= 1.0 (+ 1.0 x)) n)))
\end{scheme}
\end{note}      

\begin{note}
When the underlying floating-point representation is IEEE double
precision, the {\cf |\var{p}} suffix should not always be omitted:
Denormalized numbers have diminished precision, and therefore should
carry a {\cf |\var{p}} suffix with the actual width of the
significand.
\end{note}

The literals {\cf +inf.0} and {\cf -inf.0} represent positive and
negative infinity, respectively.  The {\cf +nan.0}
literal represents the NaN that is the result of {\cf (/ 0.0 0.0)},
and may represent other NaNs as well.

If \var{x} is an external representation of an inexact real number and
contains no vertical bar and no exponent marker
other than {\cf e}, the inexact real number it denotes is a flonum
(see library section~\extref{lib:flonumssection}{Flonums}).
Some or all of the other external representations of
inexact reals may also denote flonums, but that is not required by
this report.

\section{Read syntax}
\label{readsyntaxsection}

The read syntax describes the syntax of
syntactic datums\mainindex{syntactic datum} in terms of a sequence of
\meta{lexeme}s, as defined in the lexical syntax.

Syntactic datums include the lexeme datums described in the
previous section as well as the following constructs for forming
compound datums:
%
\begin{itemize}
\item pairs and lists, enclosed by \verb|( )| or \verb|[ ]| (see
  section~\ref{pairlistsyntax})
\item vectors (see section~\ref{vectorsyntax})
\item bytevectors (see section~\ref{bytevectorsyntax})
\end{itemize}

\subsection{Formal account}
\label{datumsyntax}

The following grammar describes the syntax of syntactic datums in terms
of various kinds of lexemes defined in the grammar in
section~\ref{lexicalsyntaxsection}:

\begin{grammar}%
\meta{datum} \: \meta{lexeme datum}
\>  \| \meta{compound datum}
\meta{lexeme datum} \: \meta{boolean} \| \meta{number}
\>  \| \meta{character} \| \meta{string} \|  \meta{symbol}
\meta{symbol} \: \meta{identifier}
\meta{compound datum} \: \meta{list} \| \meta{vector} \| \meta{bytevector}
\meta{list} \: (\arbno{\meta{datum}})
\>    \| [\arbno{\meta{datum}}]
\>    \| (\atleastone{\meta{datum}} .\ \meta{datum})
\>    \| [\atleastone{\meta{datum}} .\ \meta{datum}]
\>    \| \meta{abbreviation}
\meta{abbreviation} \: \meta{abbrev prefix} \meta{datum}
\meta{abbrev prefix} \: ' \| ` \| , \| ,@ \| \#' | \#` | \#, | \#,@
\meta{vector} \: \#(\arbno{\meta{datum}})
\meta{bytevector} \: \#vu8(\arbno{\meta{u8}})
\meta{u8} \: $\langle${\rm any \meta{number} denoting an exact}
 \>\>\quad\quad {\rm integer in $\{0, \ldots, 255\}$}$\rangle$%
\end{grammar}

\subsection{Pairs and lists}
\label{pairlistsyntax}

List and pair datums, denoting pairs and lists of values
(see section~\ref{listsection}) are written using parentheses or brackets.
Matching pairs of parentheses that occur in the rules of \meta{list} are
equivalent to matching pairs of brackets.

The most general notation for Scheme pairs as syntactic datums is
the ``dotted'' notation \hbox{\cf (\hyperi{datum} .\ \hyperii{datum})} where
\hyperi{datum} is the representation of the value of the car field and
\hyperii{datum} is the representation of the value of the
cdr field.  For example {\cf (4 .\ 5)} is a pair whose car is 4 and whose
cdr is 5.

A more streamlined notation can be used for lists: the elements of the
list are simply enclosed in parentheses and separated by spaces.  The
empty list\index{empty list} is written {\tt()} .  For example,

\begin{scheme}
(a b c d e)%
\end{scheme}

and

\begin{scheme}
(a . (b . (c . (d . (e . ())))))%
\end{scheme}

are equivalent notations for a list of symbols.

The general rule is that, if a dot is followed by an open parenthesis,
the dot, open parenthesis, and matching closing parenthesis
can be omitted in the external representation.

The sequence of characters ``{\cf (4 .\ 5)}'' is the external representation of a
pair, not an expression that evaluates to a pair.
Similarly, the sequence of characters ``{\tt(+ 2 6)}'' is {\em not} an
external representation of the integer 8, even though it {\em is} a
base-library expression evaluating to the integer 8; rather, it is a
syntactic datum representing a three-element list, the elements of which
are the symbol {\tt +} and the integers 2 and 6.

\subsection{Vectors}
\label{vectorsyntax}

Vector datums, denoting vectors of values (see
section~\ref{vectorsection}), are written using the notation
{\tt\#(\hyper{datum} \dotsfoo)}.  For example, a vector of length 3
containing the number zero in element 0, the list {\cf(2 2 2 2)} in
element 1, and the string {\cf "Anna"} in element 2 can be written as
following:

\begin{scheme}
\#(0 (2 2 2 2) "Anna")%
\end{scheme}

This is the external representation of a vector, not a
base-library expression that evaluates to a vector.

\subsection{Bytevectors}
\label{bytevectorsyntax}

Bytevector datums, denoting bytevectors (see
library chapter~\extref{lib:bytevectorschapter}{Bytevectors}), are written using the notation
{\tt\#vu8(\hyper{u8} \dotsfoo)}, where the \hyper{u8}s represent the octets of
the bytevector.  For example, a bytevector of length 3 containing the
octets 2, 24, and 123 can be written as follows:

\begin{scheme}
\#vu8(2 24 123)%
\end{scheme}

This is the external representation of a bytevector, not an
expression that evaluates to a bytevector.

\subsection{Abbreviations}\unsection
\label{quotesection}

\begin{entry}{%
\pproto{\singlequote\hyper{datum}}{}
\pproto{\backquote\hyper{datum}}{}
\pproto{,\hyper{datum}}{}
\pproto{,\atsign\hyper{datum}}{}
\pproto{\#'\hyper{datum}}{}
\pproto{\#\backquote\hyper{datum}}{}
\pproto{\#,\hyper{datum}}{}
\pproto{\#,@\hyper{datum}}{}
}

Each of these is an abbreviation:
\\\quad\schindex{'}\singlequote\hyper{datum}
for {\cf (quote \hyper{datum})},
\\\quad\schindex{`}\backquote\hyper{datum}
for {\cf (quasiquote \hyper{datum})},
\\\quad\schindex{,}{\cf,}\hyper{datum}
for {\cf (unquote \hyper{datum})},
\\\quad\index{,@\texttt{,\atsign}}{\cf,}\atsign\hyper{datum}
for {\cf (unquote-splicing \hyper{datum})},
\\\quad\sharpindex{'}{\cf\#'}\hyper{datum}
for {\cf (syntax \hyper{datum})},
\\\quad\sharpindex{`}{\cf\#`}\hyper{datum}
for {\cf (quasisyntax \hyper{datum})},
\\\quad\sharpindex{,}{\cf\#,}\hyper{datum}
for {\cf (unsyntax \hyper{datum})}, and
\\\quad\index{#,@\texttt{\#,\atsign}}{\cf\#,@}\hyper{datum}
for {\cf (unsyntax-splicing \hyper{datum})}.
\end{entry}

%%% Local Variables: 
%%% mode: latex
%%% TeX-master: "r6rs"
%%% End: 
	\par
%\vfill\eject
\chapter{Semantic concepts}
\label{basicchapter}

\section{Programs and libraries}

A Scheme program consists of a \textit{top-level program\index{top-level program}}
together with a set of \textit{libraries\index{library}}, each
of which defines a part of the program connected to the others through
explicitly specified exports and imports.  A library consists of a set
of export and import specifications and a body, which consists of
definitions, and expressions;
a top-level program is similar to a library, but
has no export specifications.
Chapters~\ref{librarychapter} and \ref{programchapter}
describe the syntax and semantics of libraries and top-level programs,
respectively.  Subsequent chapters
describe various standard libraries provided by a Scheme system.  In
particular, chapter~\ref{baselibrarychapter} describes a base
library that defines many of the constructs traditionally associated with
Scheme.

The division between the base library and other standard libraries is
based on use, not on construction.  In particular, some facilities
that are typically implemented as ``primitives'' by a compiler or
run-time libraries rather than in terms of other standard procedures
 or syntactic forms are not part of the base library, but are defined in
separate libraries.  Examples include the fixnums and flonums libraries,
the exceptions and conditions libraries, and the libraries for
records.

\section{Variables, syntactic keywords, and regions}
\label{specialformsection}
\label{variablesection}

In a library body,
an identifier\index{identifier} may name a type of syntax, or it may name
a location where a value can be stored.  An identifier that names a type
of syntax is called a {\em syntactic keyword}\mainindex{syntactic keyword}
and is said to be {\em bound} to that syntax.  An identifier that names a
location is called a {\em variable}\mainindex{variable} and is said to be
{\em bound} to that location.  The set of all visible
bindings\mainindex{binding} in effect at some point in a top-level program or
library body is
known as the {\em environment} in effect at that point.  The value
stored in the location to which a variable is bound is called the
variable's value.  By abuse of terminology, the variable is sometimes
said to name the value or to be bound to the value.  This is not quite
accurate, but confusion rarely results from this practice.

\todo{Define ``assigned'' and ``unassigned'' perhaps?}

\todo{In programs without side effects, one can safely pretend that the
variables are bound directly to the arguments.  Or:
In programs without \ide{set!}, one can safely pretend that the
variable is bound directly to the value. }

\vest Certain expression types are used to create new kinds of syntax
and to bind syntactic keywords to those new syntaxes, while other
expression types create new locations and bind variables to those
locations.  These expression types are called {\em binding
  constructs}.\mainindex{binding construct} Scheme has two kinds of
binding constructs: A \textit{definition}\index{definition} binds a
variable in a top-level program or library body.  All other binding
constructs create bindings that are only locally visible in the form
that creates them.  Variable definitions are created by {\cf define}
forms (see section~\ref{defines}), and definitions for syntactic
keywords are created by {\cf define-syntax} forms (see
section~\ref{syntaxdefinitionsection}).

The most fundamental of the local variable binding constructs is the
{\cf lambda} expression, because all other local variable binding
constructs can be explained in terms of {\cf lambda} expressions.  The
other variable binding constructs are {\cf let}, {\cf let*}, {\cf
  letrec*}, {\cf letrec}, {\cf let-values}, {\cf let*-values}, {\cf
  do}, and {\cf case-lambda} expressions (see sections~\ref{lambda},
\ref{letrec}, \ref{do}, and library section
\extref{lib:case-lambda}{{\cf case-lambda}}).
The constructs in the base library that bind syntactic keywords are
listed in section~\ref{bindsyntax}.  Local bindings can
also be created in the form of internal {\cf define} and {\cf
  define-syntax} forms that appear inside another form rather than at
the top level of a program or a library body.

\vest Scheme is a statically scoped language with
block structure.  To each place in a top-level program or library body where an identifier is bound 
there corresponds a \defining{region} of code within which
the binding is visible.  The region is determined by the particular
binding construct that establishes the binding; if the binding is
established by a {\cf lambda} expression, for example, then its region
is the entire {\cf lambda} expression.  Every mention of an identifier
refers to the binding of the identifier that established the
innermost of the regions containing the use.  If there is no binding of
the identifier whose region contains the use, then the use refers to the
binding for the variable in the top level environment of the library
body or a binding imported from another library.  (See
chapter~\ref{librarychapter}.)
If there is no binding for the identifier,
it is said to be \defining{unbound}.\mainindex{bound}

\section{Exceptional situations}
\label{exceptionalsituationsection}

\mainindex{exceptional situation}A variety of exceptional situations
are distinguished in this report, among them violations of 
syntax, violations of a procedure's specification, violations of
implementation restrictions, and exceptional situations in the
environment.  When an exception is raised, an object is provided that
describes the nature of the exceptional situation.  The report uses
the condition system described in library section~\extref{lib:conditionssection}{Conditions} to
describe exceptional situations, classifying them by condition types.

For most of the exceptional situations described in this report,
portable programs cannot rely upon the exception being continuable
at the place where the situation was detected.
For those exceptions, the exception handler that is invoked by the
exception should not return.
In some cases, however, continuing is permissible; the
handler may return.  See library section~\extref{lib:exceptionssection}{Exceptions}.

An \defining{implementation restriction} is a limitation imposed by an
implementation.  Implementations must raise an exception or abort
program execution when they are unable to continue correct execution
of a correct program due to some implementation restriction.  For
example, an implementation must raise an exception with condition type
{\cf\&implementation-restriction} or abort the execution of the
program if it does not have enough storage to run a program.

Some possible implementation restrictions such as the lack of
representations for NaNs and infinities (see
section~\ref{infinitiesnanssection}) are anticipated by this report,
and implementations typically must raise an exception of the
appropriate condition type if they encounter such a situation, i.e.\
not abort the execution of the program.

\section{Argument checking}
\label{argumentcheckingsection}

\mainindex{argument checking}
Many procedures and forms specified in this report or as part of a
standard library only accept arguments of specific types or adhering
to other restrictions.  These restrictions imply responsibilities\mainindex{responsibility} for
both the programmer and the implementation of the specified forms and
procedures.  Specifically, the programmer is responsible for ensuring
that the arguments passed indeed adhere to the restrictions described
in the specification.  The implementation must check
that the restrictions in the specification are indeed met, to the
extent that it is reasonable, possible and necessary to allow the
specified operation to complete successfully.

It is not always possible for an implementation to completely check
the restrictions set forth in the specifications.  Specifically, if an
operation is specified to accept a procedure with specific properties,
checking of these properties is undecidable in general.  Moreover,
some operations accept both list arguments and procedures that are
called by these operations.  As lists are mutable in programs that
make use of the \library{r6rs mutable-pairs} library (see library
chapter~\extref{lib:pairmutationchapter}{Mutable pairs}), an argument that is a list
when the operation starts may be mutated by the passed procedure so
that it becomes a non-list during the execution of the operation.
Also, the procedure might escape to a different continuation,
preventing the operation from performing more checks.
Even if not, requiring the operation to check that the argument is a list after
each call to such a procedure would be impractical.  Furthermore, some
operations that accept list arguments only need to traverse the lists
partially to perform their function---requiring the implementation to
check that the arguments are lists would be impractical or potentially
violate reasonable performance assumptions.  For these reasons, the
programmer's obligations may exceed the checking obligations of the
implementation.  Implementations should, however, perform
as much checking as possible and give detailed feedback about
violations.

When an implementation detects a violation of an argument
specification at run time, it must either raise an exception with
condition type {\cf\&violation}, or abort the program in a way
consistent with the safety of execution as described in the next
section.

\section{Safety}
\label{safeunsafemodesection}

The standard libraries whose exports are described by this document
are said to be \defining{safe libraries}.  Libraries and top-level
program that import only from safe libraries are also said to be safe.

As defined by this document, the Scheme programming language
is safe in the following sense:
The execution of a safe top-level program consisting
cannot go so badly wrong as to crash or to continue to
execute while behaving in ways that are
inconsistent with the semantics described in this document,
unless said execution first encounters some implementation
restriction or other defect in the implementation of Scheme
that is executing the program.

Violations of an implementation restriction must raise an
exception with condition type {\cf\&implementation-restriction},
as must all
violations and errors that would otherwise threaten system
integrity in ways that might result in execution that is
inconsistent with the semantics described in this document.

The above safety properties are guaranteed only for top-level programs
and libraries that are said to be safe.  In particular, Implementations
may provide access to unsafe libraries, and may interpret
implementation-specific declarations in ways that
cannot guarantee safety.

\section{Boolean values}
\label{booleanvaluessection}

Although there is a separate boolean type, any Scheme value can be
used as a boolean value for the purpose of a conditional test.  In a
conditional test, all values count as true in such a test except for
\schfalse{}.  This report uses the word ``true'' to refer to any
Scheme value except \schfalse{}, and the word ``false'' to refer to
\schfalse{}. \mainindex{true} \mainindex{false}

\section{Multiple return values}
\label{multiplereturnvaluessection}

A Scheme expression can evaluate to an arbitrary finite number of
values.  These values are passed to the expression's continuation.

Not all continuations accept any number of values: A continuation that
accepts the argument to a procedure call is guaranteed to accept
exactly one value.  The effect of passing some other number of values
to such a continuation is unspecified.  The {\cf call-with-values}
procedure
described in section~\ref{controlsection} makes it possible to create
continuations that accept specified numbers of return values.
If the number of
return values passed to a continuation created by a call to
{\cf call-with-values} is not accepted by its consumer
that was passed in that call, then an exception is raised.
A more complete description of the number of values accepted by
different continuations and the consequences of passing an unexpected
number of values is given in the description of the {\cf values}
procedure in section~\ref{values}.

A number of forms in the base library have sequences of expressions
as subforms that are evaluated sequentially, with the return values of
all but the last expression being discarded.  The continuations
discarding these values accept any number of values.

\section{Storage model}
\label{storagemodel}

Variables and objects such as pairs, vectors, and strings implicitly
denote locations\mainindex{location} or sequences of locations.  A string, for
example, denotes as many locations as there are characters in the string. 
(These locations need not correspond to a full machine word.) A new value may be
stored into one of these locations using the {\tt string-set!} procedure, but
the string continues to denote the same locations as before.

An object fetched from a location, by a variable reference or by
a procedure such as {\cf car}, {\cf vector-ref}, or {\cf string-ref}, is
equivalent in the sense of \ide{eqv?} % and \ide{eq?} ??
(section~\ref{equivalencesection})
to the object last stored in the location before the fetch.

Every location is marked to show whether it is in use.
No variable or object ever refers to a location that is not in use.
Whenever this report speaks of storage being allocated for a variable
or object, what is meant is that an appropriate number of locations are
chosen from the set of locations that are not in use, and the chosen
locations are marked to indicate that they are now in use before the variable
or object is made to denote them.

It is desirable for constants\index{constant} (i.e. the values of
literal expressions) to reside in read-only-memory.  To express this,
it is convenient to imagine that every object that denotes locations
is associated with a flag telling whether that object is
mutable\index{mutable} or immutable\index{immutable}.  Literal
constants, the strings returned by \ide{symbol->string}, records with
no mutable fields, and other values explicitly designated as immutable
are immutable objects, while all objects created by the other
procedures listed in this report are mutable.  An attempt to store a
new value into a location that is denoted by an immutable object
should raise an exception with condition type {\cf\&assertion}.

\section{Proper tail recursion}
\label{proper tail recursion}

Implementations of Scheme are required to be
{\em properly tail-recursive}\mainindex{proper tail recursion}.
Procedure calls that occur in certain syntactic
contexts defined below are `tail calls'.  A Scheme implementation is
properly tail-recursive if it supports an unbounded number of active
tail calls.  A call is {\em active} if the called procedure may still
return.  Note that this includes calls that may be returned from either
by the current continuation or by continuations captured earlier by
{\cf call-with-current-continuation} that are later invoked.
In the absence of captured continuations, calls could
return at most once and the active calls would be those that had not
yet returned.
A formal definition of proper tail recursion can be found
in~\cite{propertailrecursion}.  The rules for identifying tail calls
in base-library constructs are described in
section~\ref{basetailcontextsection}.

\begin{rationale}

Intuitively, no space is needed for an active tail call because the
continuation that is used in the tail call has the same semantics as the
continuation passed to the procedure containing the call.  Although an improper
implementation might use a new continuation in the call, a return
to this new continuation would be followed immediately by a return
to the continuation passed to the procedure.  A properly tail-recursive
implementation returns to that continuation directly.

Proper tail recursion was one of the central ideas in Steele and
Sussman's original version of Scheme.  Their first Scheme interpreter
implemented both functions and actors.  Control flow was expressed using
actors, which differed from functions in that they passed their results
on to another actor instead of returning to a caller.  In the terminology
of this section, each actor finished with a tail call to another actor.

Steele and Sussman later observed that in their interpreter the code
for dealing with actors was identical to that for functions and thus
there was no need to include both in the language.

\end{rationale}

%%% Local Variables: 
%%% mode: latex
%%% TeX-master: "r6rs"
%%% End: 
	\par
%\vfill\eject
\chapter{Notation and terminology}
\label{terminologychapter}

\section{Requirement levels} 
\label{requirementsection}

The key words ``must'', ``must not'', ``required'', ``shall'', ``shall
not'', ``should'', ``should not'', ``recommended'', ``may'', and
``optional'' in this document are to be interpreted as described in
RFC~2119~\cite{mustard}.  Specifically:

\begin{description}
\item[must]\mainindex{must} This word means that a statement is an absolute
  requirement of the specification.
\item[must not]\mainindex{must not} This phrase means that a statement is an absolute
  prohibition of the specification.
\item[should]\mainindex{should} This word, or the adjective ``recommended'', mean that
  valid reasons may exist in particular circumstances to ignore a
  statement, but that the implications must be understood and weighed
  before choosing a different course.
\item[should not]\mainindex{should not} This phrase, or the phrase ``not recommended'', mean
  that valid reasons max exist in particular circumstances when the
  behavior of a statement is acceptable, but that the implications
  should be understood and weighed before choosing the course described
  by the statement.
\item[may]\mainindex{may} This word, or the adjective ``optional'', mean that an item
  is truly optional.
\end{description}

\section{Entry format}

The chapters describing bindings in the base library and the standard
libraries are organized
into entries.  Each entry describes one language feature or a group of
related features, where a feature is either a syntactic construct or a
built-in procedure.  An entry begins with one or more header lines of the form

\noindent\pproto{\var{template}}{\var{category}}\unpenalty

If \var{category} is ``\exprtype'', the entry describes a 
special syntactic form, and the template gives the syntax of the form.  Even
though the template is written in a notation similar to a right-hand
side of the BNF rules in chapter~\ref{readsyntaxchapter}, it describes
the set of forms equivalent to the forms matching the
template as syntactic datums.  Some ``\exprtype'' entries carry a
suffix ({\cf expand}), specifying that the form is exported with level
$1$.  Otherwise, the form is exported with level $0$; see
section~\ref{phasessection}.

Components of the form described by a template are designated
by syntactic variables, which are written using angle brackets, for
example, \hyper{expression}, \hyper{variable}.  Case is insignificant
in syntactic variables.  Syntactic variables
denote other forms, or, in some cases,
sequences of them.  A syntactic variable may refer to a non-terminal
in the grammar for syntactic datums, in which case only forms matching
that non-terminal are permissible in that position.  For example,
\hyper{expression} stands for any form which is a
syntactically valid expression.  Other non-terminals that are used in
templates will be defined as part of the specification.

The notation
\begin{tabbing}
\qquad \hyperi{thing} $\ldots$
\end{tabbing}
indicates zero or more occurrences of a \hyper{thing}, and
\begin{tabbing}
\qquad \hyperi{thing} \hyperii{thing} $\ldots$
\end{tabbing}
indicates one or more occurrences of a \hyper{thing}.

It is the programmer's responsibility to ensure that the component of
a form has the shape specified by a template.  Descriptions of syntax
may express other restrictions on the components of a form.
Typically, such a restriction is formulated as a phrase of the form
``\hyper{x} must be\mainindex{must be} a \ldots''.  Again, these
specify the programmer's responsibility.  It is the implementation's
responsibility to check that these restrictions are satisfied, as long
as the macro transformers involved in expanding the form terminate.
If the implementation detects that a component does not meet the
restriction, an exception with condition type {\cf\&syntax} is raised.

If \var{category} is ``procedure'', then the entry describes a procedure, and
the header line gives a template for a call to the procedure.  Parameter
names in the template are \var{italicized}.  Thus the header line

\noindent\pproto{(vector-ref \var{vector} \var{k})}{procedure}\unpenalty

indicates that the built-in procedure {\tt vector-ref} takes
two arguments, a vector \var{vector} and an exact non-negative integer
\var{k} (see below).  The header lines

\noindent%
\pproto{(make-vector \var{k})}{procedure}
\pproto{(make-vector \var{k} \var{fill})}{procedure}\unpenalty

indicate that the {\tt make-vector} procedure takes
either one or two arguments.  The parameter names are
case-insensitive: \var{Vector} is the same as \var{vector}.

As with syntax templates, an ellipsis \dotsfoo{} at the end of a header
line, as in

\noindent\pproto{(= \vari{z} \varii{z} \variii{z} \dotsfoo)}{procedure}\unpenalty

indicates that the procedure takes arbitrarily many arguments of the
same type as specified for the last parameter name.  In this case,
{\cf =} accepts two or more arguments that must all be complex
numbers.

\label{typeconventions}
A procedure that detects an argument that it is not specified to
handle must either raise an exception with condition type
{\cf\&assertion} or abort the execution of the program.  Also, if the
number of arguments provided in a procedure call does not match any
argument count specified for the called procedure, an exception with
condition type {\cf\&assertion} must be raised or the execution of the
program must be aborted.  See section~\ref{argumentcheckingsection}.

For succinctness, we follow the convention
that if a parameter name is also the name of a type, then the corresponding argument must be of the named type.
For example, the header line for {\tt vector-ref} given above dictates that the
first argument to {\tt vector-ref} must be a vector.  The following naming
conventions imply type restrictions:
%
\begin{center}
\begin{tabular}{ll}
\var{obj}&any object\\
\var{z}&complex number\\
\var{x}&real number\\
\var{y}&real number\\
\var{q}&rational number\\
\var{n}&integer\\
\var{k}&exact non-negative integer\\
\var{octet}&exact integer in $\{0, \ldots, 255\}$\\
\var{byte}&exact integer in $\{-128, \ldots, 127\}$\\
\var{char}&character (see section~\ref{charactersection})\\
\var{pair}&pair (see section~\ref{listsection})\\
\var{vector}&vector (see section~\ref{vectorsection})\\
\var{string}&string (see section~\ref{stringsection})\\
\var{condition}&condition (see library section~\extref{lib:conditionssection}{Conditions})\\
\var{bytevector}&bytevector (see library chapter~\extref{lib:bytevectorschapter}{Bytevectors})\\
\var{proc}&procedure (see section~\ref{proceduressection})
\end{tabular}
\end{center}

Other type restrictions are expressed through parameter naming
conventions that are described in specific chapters.  For example,
library chapter~\extref{lib:numberchapter}{Arithmetic} uses a number of special
parameter variables for the various subsets of the numbers.

With the listed type restrictions, the programmer's responsibility of
ensuring that the corresponding argument is of the specified type
corresponds to the implementation's responsibility of checking for
that type, see section~\ref{argumentcheckingsection}.

The \var{list} parameter naming conventions means that it is the
programmer's responsibility to pass a list argument (see
section~\ref{listsection}).  It is the implementation's responsibility
to check that the argument is appropriately structured for the
operation to perform its function, to the extent that this is possible
and reasonable.  The implementation must at least check that the
argument is either an empty list or a pair.

Descriptions of procedures may express other restrictions on the
arguments of a procedure.  Typically, such a restriction is formulated
as a phrase of the form ``\var{x} must be a \ldots''. (or otherwise
using the word ``must''.) 

If the description does not explicitly distinguish between the
programmer's and the implementation's responsibilities, the
restrictions describe both the programmer's responsibility, who must
ensure that an appropriate argument is passed, and the
implementation's responsibilities, which must check that the argument
is appropriate.  A description may explicitly list the
implementation's responsibilities for some arguments in a paragraph
labelled ``\textit{Implementation responsibilities}''.  In that case,
the responsibilities specified for these arguments in the rest of the
description are only for the programmer.

If \var{category} is something other than ``syntax'' and
``procedure'', then the entry describes a non-procedural value, and
the \var{category} describes the type of that value.  The header line

\noindent\rvproto{\&who}{condition type}

indicates that {\cf\&who} is a condition type.

The description of an entry occasionally states that it is \textit{the
  same} as another entry.  This means that both entries are
equivalent.  Specifically, it means that if both entries have the same
name and are thus exported from different libraries, the entries from
both libraries can be imported under the same name without conflict.

\section{Evaluation examples}

The symbol ``\evalsto'' used in program examples can be read
``evaluates to''.  For example,

\begin{scheme}
(* 5 8)      \ev  40%
\end{scheme}

means that the expression {\tt(* 5 8)} evaluates to the object {\tt 40}.
Or, more precisely:  the expression given by the sequence of characters
``{\tt(* 5 8)}'' evaluates, in the initial environment, to an object
that may be represented externally by the sequence of characters ``{\tt
40}''.  See section~\ref{readsyntaxsection} for a discussion of external
representations of objects.

The ``\evalsto'' symbol is also used when the evaluation of an
expression causes a violation.  For example,

\begin{scheme}
(integer->char \sharpsign{}xD800) \ev \exception{\&assertion}
\end{scheme}

means that the evaluation of the expression {\cf (integer->char
  \sharpsign{}xD800)} must raise an exception with condition type
{\cf\&assertion} or abort the execution of the program.

\section{Unspecified behavior}

\vest If the value of an expression is said to be ``unspecified'',
then the expression must evaluate without raising an exception, but
the values returned depend on the implementation; this report
explicitly does not say what values should be returned.
\mainindex{unspecified behavior}

Some expressions are specified to return \emph{the} unspecified value,
which is a special value returned by the \texttt{unspecified}
procedure.  (See section~\ref{unspecifiedvalue}.)  In this case, the
return value is meaningless, and programmers are discouraged from
relying on its specific nature.

\section{Exceptional situations}

When speaking of an exceptional situation (see section~\ref{exceptionalsituationsection}), this
report uses the phrase ``an exception is raised'' to indicate
that implementations must detect the situation and report it to the
program through the exception system described in
library chapter~\extref{lib:exceptionsconditionschapter}{Exceptions
  and conditions}.

Several variations on ``an exception is raised'' using the keywords
described in section~\ref{requirementsection} are possible, in
particular ``an exception must be raised'' (equivalent to ``an
exception is raised''), ``an exception should be raised'', and ``an
exception may be raised''.

This report uses the phrase ``an exception with condition type \var{t}''
to indicate that the object provided with the
exception is a condition object of the specified type.

The phrase ``a continuable exception is raised'' indicates an
exceptional situation that permits the exception handler to return,
thereby allowing program execution to continue at the place where the
original exception occurred.  See library
section~\extref{lib:exceptionssection}{Exceptions}.

\section{Naming conventions}

By convention, the names of procedures that store values into previously
allocated locations (see section~\ref{storagemodel}) usually end in
``\ide{!}''.
Such procedures are called mutation procedures.
By convention, the value returned by a mutation procedure is
\unspecifiedreturn{} (see section~\ref{unspecifiedvalue}),
but this convention is not always followed.

By convention, ``\ide{->}'' appears within the names of procedures that
take an object of one type and return an analogous object of another type.
For example, {\cf list->vector} takes a list and returns a vector whose
elements are the same as those of the list.

By convention, the names of condition types usually start with
``{\cf\&}''\index{&@\texttt{\&}}.

By convention, the names of predicates---procedures that always return
a boolean value---end in ``\ide{?}'' when the name contains any
letters; otherwise, the predicate's name does not end with a question
mark.

The components of compound names are usually separated by ``\ide{-}''
In particular, prefixes that are actual words or can be pronounced as
though they were actual words are followed by a hyphen, except when
the first character following the hyphen would be something other than
a letter, in which case the hyphen is omitted.  Short,
unpronounceable prefixes (``\ide{fx}'' and ``\ide{fl}'') are not
followed by a hyphen.

\section{Syntax violations}

Scheme implementations conformant with this report must detect
violations of the syntax.  A \defining{syntax violation} is an error
with respect to the syntax of library bodies, top-level bodies,
or the ``\exprtype'' entries in the
specification of the base library or the standard libraries.
Moreover, attempting to assign to an immutable variable (i.e., the
variables exported by a library; see
section~\ref{importsareimmutablesection}) is also
considered a syntax violation.

If a top-level or library form is not syntactically correct, then the
execution of that top-level program or library must not be allowed to begin.

%%% Local Variables: 
%%% mode: latex
%%% TeX-master: "r6rs"
%%% End: 
 \par
\documentclass{monograph}

% look for FIXME
\ifhtml\def\long{}\fi
\long\def\FIXME#1{{\bf FIXME}: #1}

\usepackage{scheme}

\def\r#1rs{R#1RS}
\iflatex
\input{fullpage.sty}
\fi

\ifhtml
\headerstuff{\raw{
<style type="text/css">
<!--
 a:link, a:active, a:visited {color:blue}
 a:hover {color:white; background:blue}
 a.plain:link, a.plain:active, a.plain:visited {color:blue; text-decoration:none}
 a.plain:hover {color:white; text-decoration:none; background:blue}
 table.indent {margin-left: 20px}
 h1 { font-size: 1.75em }
 h2 { font-size: 1.25em }
 h3 { font-size: 1.12em }
 h4 { font-size: 1em }
-->
</style>
}}
\documenttitle{R6RS Library Syntax}
\fi

\iflatex
\pagestyle{plain}
\fi

\ifhtml
\renewcommand{\sectionstar}[1]{\raw{\raw{<h1>}}#1\raw{\raw{</h1>}}}
\renewcommand{\subsectionstar}[1]{\raw{\raw{<h2>}}#1\raw{\raw{</h2>}}}
\renewcommand{\subsubsectionstar}[1]{\raw{\raw{<h3>}}#1\raw{\raw{</h3>}}}
\fi

\begin{document}

\iflatex
% block paragraphs
\schemeindent=0pt
\parskip=4pt
\parindent=0pt
\fi

\sectionstar{Title}

R6RS Library Syntax

\sectionstar{Authors}

Matthew Flatt and Kent Dybvig

\sectionstar{Status}

This SRFI is being submitted by a member of the Scheme Language Editor's
Committee as part of the {\r6rs} Scheme standardization process.  The purpose
of such ``{\r6rs} SRFIs'' is to inform the Scheme community of features and
design ideas under consideration by the editors and to allow the community
to give the editors some direct feedback that will be considered during
the design process.

At the end of the discussion period, this SRFI will be withdrawn.  When
the {\r6rs} specification is finalized, the SRFI may be revised to conform to
the {\r6rs} specification and then resubmitted with the intent to finalize
it.  This procedure aims to avoid the situation where this SRFI is
inconsistent with {\r6rs}.  An inconsistency between {\r6rs} and this SRFI could
confuse some users.  Moreover it could pose implementation problems for
{\r6rs} compliant Scheme systems that aim to support this SRFI.  Note that
departures from the SRFI specification by the Scheme Language Editor's
Committee may occur due to other design constraints, such as design
consistency with other features that are not under discussion as SRFIs.

\ifhtml
\sectionstar{Table of Contents}
\tableofcontents
\fi

\section{Abstract}

The library system presented here is designed to let programmers share
libraries, i.e., code that is intended to be incorporated into larger
programs, and especially into programs that use library code from multiple
sources.  The library system supports macro definitions within libraries,
allows macro exports, and distinguishes the phases in which definitions
and imports are needed.  This SRFI defines a standard notation for
libraries, a semantics for library expansion and execution, and a simple
format for sharing libraries.

\section{Rationale\label{sec:rationale}}

This standard addresses the following specific goals:

\begin{itemize}
\item Separate compilation and analysis; no two libraries have to be compiled at the same time (i.e., the meanings of two libraries cannot depend on each other cyclically, and compilation of two different libraries cannot rely on state shared across compilations), and significant program analysis does not require a whole program.
\item Independent compilation/analysis of unrelated libraries, where ``unrelated'' means that neither depends on the other through a transitive closure of imports.
\item Explicit declaration of dependencies, so that the meaning of each identifier is clear at compile time, and so that there is no ambiguity about whether a library needs to be executed for another library's compile time and/or run time.
\item Namespace management, so that different library producers are unlikely to define the same top-level name. 
\end{itemize}

It does not address the following:

\begin{itemize}
\item Mutually dependent libraries.
\item Separation of library interface from library implementation.
\item Code outside of a library (e.g., 5 by itself as a program).
\item Local modules and local imports. 
\end{itemize}


\section{Specification\label{sec:specification}}

\subsection{Library form}

A library declaration contains the following elements:

\begin{itemize}
\item a name for the library (possibly compound, with versioning),
\item a list of import dependencies, where each dependency specifies the
      following:
\begin{itemize}
\item the imported library's name,
\item the relevant phases, e.g., expand or run time, and
\item the subset of the library's exports to make available within the
      importing library, and the local names to use within the importing
      library for each of the library's exports,
\end{itemize}
\item a list of exports, which name a subset of the library's imports and
      definitions, and
\item a library body, consisting of a sequence of definitions followed
      by a sequence of expressions.
\end{itemize}

\subsection{Syntax and Semantics}

A library definition is written with the library form:

\schemedisplay
(library \raw{\ang{library~name}}
  (export \raw{\ang{export~spec}}\raw{\kstar})
  (import \raw{\ang{import~spec}}\raw{\kstar})
  \raw{\ang{library~body}})
\endschemedisplay

The \ang{library~name} specifies the name of the library, the
\scheme{import} form specifies the imported bindings, and the
\scheme{export} form specifies the exported bindings.
The \ang{library~body} specifies the set of definitions, both for local
(unexported) and exported bindings, and the set of initialization
expressions (commands) to be evaluated for their effects.
The exported bindings may be defined within the library or imported into
the library.

An identifier can be imported from two or more libraries only if the
binding exported by each library is the same (i.e., the binding is
defined in one library, and it arrives throgh the imports only by
exporting and re-exporting).  Otherwise, no identifier can be imported
multiple times, defined multiple times, or both defined and imported.

Library names have the following syntax.

\begin{grammar}
\ang{library~name}\longis \ang{identifier} \bar\ \scheme{(\raw{\ang{identifier}}\raw{\kplus} \raw{\ang{version}})}
\\
\ang{version}\longis \ang{empty} \bar\ \scheme{(\raw{\ang{subversion}}\raw{\kplus})}\\
\\
\ang{subversion}\longis \ang{exact nonnegative integer}
\end{grammar}

where \ang{identifier} is shorthand for \scheme{(\ang{identifier})}.

Each \ang{import~spec} specifies a set of bindings to be imported into
the library, the phases in which they are to be available, and the local
names by which they are to be known.

\begin{grammar}
\ang{import~spec}\longis \ang{import~set}\\
  \orbar \scheme{(for \raw{\ang{import~set}} \raw{\ang{import~phase}}\raw{\kstar})}
\end{grammar}

Valid import phases are \scheme{run}, \scheme{expand}, and
\scheme{(meta \var{n})}, where \scheme{run} is an abbreviation for
\scheme{(meta 0)} and \scheme{expand} is an abbreviation for
\scheme{(meta 1)}.

\begin{grammar}
\ang{import~phase}\longis \scheme{run} \bar\ \scheme{expand} \bar\ \scheme{(meta \raw{\ang{level}})}\\
\\
\ang{level}\longis \ang{exact nonnegative integer}
\end{grammar}

Phases are discussed in Section~\ref{sec:phases}.

An \ang{import~set} names a set of bindings from another library, and
possibly specifies local names for the imported bindings.

\begin{grammar}
\ang{import~set}\longis \ang{library~reference}\\
  \orbar \scheme{(only \raw{\ang{import~set}} \raw{\ang{identifier}}\raw{\kstar})}\\
  \orbar \scheme{(except \raw{\ang{import~set}} \raw{\ang{identifier}}\raw{\kstar})}\\
  \orbar \scheme{(add-prefix \raw{\ang{import~set}} \raw{\ang{identifier}})}\\
  \orbar \scheme{(rename \raw{\ang{import~set}} (\raw{\ang{identifier}} \raw{\ang{identifier}})\raw{\kstar})}
\end{grammar}

A \ang{library~reference} identifies a library by its (possibly compound)
name and optionally by its version.

\begin{grammar}
\ang{library~reference}\longis \scheme{\raw{\ang{identifier}}} \bar\ \scheme{(\raw{\ang{identifier}}\raw{\kplus} \raw{\ang{version~reference}})}\\
\\
\ang{version~reference}\longis \ang{empty} \bar\ \scheme{(\raw{\ang{subversion~reference}}\raw{\kplus})}\\
\\
\ang{subversion~reference}\longis \ang{subversion} \bar\ \ang{subversion~condition}\\
\\
\ang{subversion~condition}\longis \scheme{(>= \raw{\ang{subversion}})}\\
                           \orbar \scheme{(<= \raw{\ang{subversion}})}\\
                           \orbar \scheme{(and \raw{\ang{subversion~condition}}\raw{\kplus})}\\
                           \orbar \scheme{(or \raw{\ang{subversion~condition}}\raw{\kplus})}\\
                           \orbar \scheme{(not \raw{\ang{subversion~condition}})}
\end{grammar}

where \ang{identifier} is shorthand for \scheme{(\ang{identifier})}.

The sequence of identifiers in the importing library's
\ang{library~reference} must match the sequence of identifiers in the
imported library's \ang{library~name}.
The importing library's \ang{version~reference} specifies a predicate on a
prefix of the imported library's \ang{version}.
Each integer must match exactly and each condition has the expected meaning.
Everything beyond the prefix specified in the version reference matches
unconditionally.
When more than one library is identified by a library reference, the
choice of libraries is determined in some implementation-dependent manner.

To avoid problems such as incompatible types and replicated state, two
libraries whose library names contain the same sequence of identifiers but
whose versions do not match cannot co-exist in the same program.

By default, all of an imported library's exported bindings are made
visible within an importing library using the names given to the bindings
by the imported library.
The preceise set of bindings to be imported and the names of those
bindings can be adjusted with the \scheme{only}, \scheme{except},
\scheme{add-prefix}, and \scheme{rename} forms as described below.

\begin{itemize}
\item The \scheme{only} form produces a subset of the bindings from another
\ang{import~set}, including only the listed
\ang{identifier}s; if any of the included \ang{identifier}s is not in
\ang{import~set}, an exception is raised.
\item The \scheme{except} form produces a subset of the bindings from another
\ang{import~set}, including all but the listed
\ang{identifier}s; if any of the excluded \ang{identifier}s is not in
\ang{import~set}, an exception is raised.
\item The \scheme{add-prefix} adds a prefix to each
name from another \ang{import~set}.
\item The \scheme{rename} form, for each pair of identifiers (\ang{identifier}
\ang{identifier}), removes a binding from the set from \ang{import~set},
and adds it back with a different name. 
The first identifier is the original name, and the
second identifier is the new name. 
If the original name is not in \ang{import~set}, or
if the new name is already in \ang{import~set}, an exception is raised.
\end{itemize}

An \ang{export~set} names a set of imported and locally defined bindings to
be exported fpr optionally specified phases, possibly with different
external names.

\begin{grammar}
\ang{export~spec}\longis \ang{export~set}\\
  \orbar \scheme{(for (\raw{\ang{export~set}}\raw{\kstar}) \raw{\ang{import~phase}}\raw{\kstar})}\\
\\
\ang{export~set}\longis \ang{identifier}\\
  \orbar \scheme{(rename (\raw{\ang{identifier}} \raw{\ang{identifier}})\raw{\kstar})}
\end{grammar}

In an \ang{export~set}, an \ang{identifier} names a single binding defined
within the library or imported, where the external name for the export is
the same as the name of the binding within the library. 
A \scheme{rename} set exports the binding named by the first
\ang{identifier} in each pair, using the second \ang{identifier} as the
external name.

The \ang{library~body} of a \scheme{library} form contains definitions for
local and exported bindings and initalization expressions to be evaluated
when the library is invoked.

A \ang{library~body} is like a \scheme{lambda} body (see below) except that
\ang{library~bodies} need not include any expressions.

\begin{grammar}
\ang{library~body}\longis \ang{declaration}\kstar\ \ang{definition}\kstar\ \ang{command}\kstar
\end{grammar}

\FIXME{need to update syntax of {definition} to include {syntax definition}.}

\FIXME{replace this to be consistent with bodiessection.
Matthew and I have given up trying to produce a suitable grammar
and suggest that the splicing behavior of \scheme{begin} be described
in prose, something like: \textit{Any subsequence of zero or more body forms
may be wrapped, possibly in nested fashion, in a \scheme{begin}
form, provided that no empty \scheme{begin} form appears after the first
\ang{expression}.  This allows macros to expand into such subsequences
of body forms.  In this context, \scheme{begin} is treated as a
splicing form, i.e., as if the \scheme{begin} wrapper were not
actually present.}
This note will have to be modified for the script syntax to leave out
the prohibition of empty \scheme{begin} forms after the first
\ang{expression}.}

\FIXME{with the above, we no longer need \scheme{(begin <definition>*)} as
a definition.}

The definitions of a \ang{library~body} or \ang{body} are mutually
recursive.
The transformer expressions and transformer bindings are created
from left to right, as described in Chapter~\label{expansionchapter}.
The variable-definition right-hand-side expressions are evalated
from left to right, as if in an implicit \scheme{letrec*},
and the body expressions are also evaluated from left to right
after the variable-definition right-hand-side expressions.
The location of each exported variable is then initialized to the value
of its local counterpart.
The effect of returning twice to the continuation of the last body
expression is unspecified.

The names \scheme{library}, \scheme{export}, \scheme{import},
\scheme{for}, \scheme{run}, \scheme{expand}, \scheme{meta},
\scheme{import}, \scheme{export}, \scheme{only}, \scheme{except}, and
\scheme{rename} appearing in the library syntax are part of the
syntax and are not reserved, i.e, the same can be used for other
purposes within the library or even exported from or imported 
into a library with different meanings, without affecting their
use in the \scheme{library} form.

In the case of any ambiguities that arise from the use of one of
these names as a library name when using the shorthand (single-identifier)
\ang{library~reference} syntax should be resolved in favor of the interpretation
given to the name by the library syntax.
For example, \scheme{(import (for lib expand))} should be taken as
importing library \scheme{lib} for \scheme{expand}, not as importing
library \scheme{(for lib expand)}.
The user can always eliminate such ambiguities by avoiding the shorthand
\ang{library~reference} syntax when such an ambiguity might arise.

Bindings defined with a library are not visible in code that appears
outside of the library unless they are explicitly exported from the
library. 
An exported macro may, however, \emph{implicitly export} an identifier
defined within or imported into the library.
That is, it may insert a reference to that identifier into the output code
it produces.

All explicitly exported variables are immutable both in the exporting and
importing libraries.
All implicitly exported variables are mutable in the exporting library but
immutable in the importing libraries.
In consequence, any change after the initial assignment to the value of an
implicitly exported variable is not reflected by the references inserted
by an exported macro outside of the exporting library, and exported macros
may not insert assignments to implicitly exported variables outside of the
exporting library.

\subsection{Import phases\label{sec:phases}}

All bindings imported via a library's \scheme{import} form are
\emph{visible} throughout the library's \ang{library~body}.
An exception may be raised, however, if a binding is used out of its declared
phase(s):

\begin{itemize}
\item Bindings used in run-time code must be imported ``for \scheme{run},''
which is equivalent to ``for \scheme{(meta 0)}.''
\item Bindings used in the body of a transformer (appearing on the
right-hand-side of a transformer binding) in run-time code must be
imported ``for \scheme{expand},'' which is equivalent to
``for \scheme{(meta 1)},
\item Bindings used in the body of a transformer appearing within the body of a
transformer in run-time code must be imported ``for \scheme{(meta 2)},''
and so on.
\end{itemize}

The valid import phases of an imported binding are determined by the enclosing
\scheme{for} form, if any, in the \scheme{import} form of the importing
library, in addition to the phase of the identifier in the exporting library.
An \ang{import~set} without an enclosing \scheme{for} is equivalent to
\scheme{(for \raw{\ang{import~set}} run)}.

Import and export phases are combined by pairwise addition of all phase
combinations.  For example, references to an imported identifier exported
for phases $p_a$ and $p_b$ and imported for phases $q_a$, $q_b$, and $q_c$
are valid at phases $p_a+q_q$, $p_a+q_b$, $p_a+q_c$, $p_b+q_q$, $p_b+q_b$,
$nd p_b+q_c$.

The import phases implicitly determine when information about a
library must be available and also when the various forms contained within
a library must be evaluated.

Every library can be characterized by expand-time information (minimally,
its imported libraries, a list of the exported keywords, a list of the
exported variables, and code to evaluate the transformer expressions) and
run-time information (minimally, code to evaluate the variable definition
right-hand-side expressions, and code to evaluate the body expressions).
The expand-time information must be available to expand references to
any exported binding, and the run-time information must be available to
evaluate references to any exported variable binding.

If any of a library's bindings is imported by another library ``for
\scheme{expand}'' (or for any meta level greater than 0), both expand-time and
run-time information for the first library is made available when the second
library is expanded.
If any of a library's bindings is imported by another library ``for
\scheme{run},'' the expand-time information for the first library is made available when
the second library is expanded, and the run-time information for the first
library is made available when the run-time information for the second library
is made available.

We must also consider when the code to evaluate a library's transformer
expressions is executed and when the code to evaluate the library's
variable-definition right-hand-side expressions and body expressions is
executed.
We refer to executing the transformer expressions as \emph{visiting}
the library and to executing the variable-definition right-hand-side 
expressions and body expressions as \emph{invoking} the library.
A library must be visited before code that uses its bindings can be
expanded, and it must be invoked before code that uses its bindings can be
executed.
Visiting or invoking a library may also trigger the visiting or
invoking of other libraries.

More precisely, visiting a library at phase $N$ causes the system to:

\begin{itemize}
\item Visit at phase $N$ any library that is imported by this library
      ``for \scheme{run}'' and that is not yet visited at phase $N$.
\item Visit at phase $N+M$ any library that is imported by this
      library ``for \scheme{(meta \var{M})},'' $M>0$ and that is not yet
      visited at phase $N+M$.
\item Invoke at phase $N+M$ any library that is imported by this
      library ``for \scheme{(meta \var{M})},'' $M>0$ and that is not yet
      invoked at phase $N+M$.
\item Evaluate the library's transformer expressions.
\end{itemize}

The order in which imported libraries are visited and invoked is not
defined, but imported libraries must be visited and invoked before the
library's transformer expressions are evaluated.

Similarly, invoking a library at meta phase $N$ causes the system to:

\begin{itemize}
\item Invoke at phase $N$ any library that is imported by this library
      ``for \scheme{run}'' and that is not yet invoked at phase $N$.
\item Evaluate the library's variable-definition right-hand-side and body
      expressions.
\end{itemize}

The order in which imported libraries are invoked is not defined, but
imported libraries must be invoked before the library's variable-definition
right-hand-side and body expressions are evaluated.

An implementation is allowed to distinguish visits of a library across
different phases or to treat a visit at any phase as a visit at all
phases.
Similarly, an implementation is allowed to distinguish invocations of a
library across different phases or to treat an invocation at any phase as
an invocation at all phases.
An implementation is further allowed to start each expansion of a
\scheme{library} form by removing all library bindings above phase 0.
Thus, a portable library's meaning must not depend on whether the
invocations are distinguished or preserved across phases or \scheme{library}
expansions.

\subsection{Eval\label{sec:eval}}

The \scheme{eval} procedure accepts two arguments, an expression to
evaluate, represented as an s-expression, and an environment:

\schemedisplay
(eval \var{expression} \var{environment})
\endschemedisplay

Environments can be constructed with the \scheme{environment} procedure,
which accepts a set of import specifiers represented as datums.

\schemedisplay
(environment \var{import-spec} \dots) ;=> \var{environment}
\endschemedisplay

The s-expression syntax of an \var{import-spec} mirrors the external
syntax of an \ang{import spec}.
For example:

\schemedisplay
(eval '(+ 3 4) (environment '(r6rs base))) ;=> 7
\endschemedisplay

An exception is raised if the expand-time or run-time information for a
library named in one of the \var{import-specs} is not \emph{available}
when the call to \scheme{environment} occurs, in the sense of
Section~\ref{sec:phases}.

\section{Examples}

\FIXME{compare examples with von Tonder macros.test file.}
\FIXME{need some eval examples}

Hello world:

\schemedisplay
(library hello
  (import (r6rs))
  (export)
  (display "Hello World")
  (newline))
\endschemedisplay

Examples for various \ang{import~spec}s and \ang{export~spec}s:

\schemedisplay
(library (stack)
  (import (r6rs))
  (export make push! pop! empty!)

  (define (make) (list '()))
  (define (push! s v) (set-car! s (cons v (car s))))
  (define (pop! s) (let ([v (caar s)])
                     (set-car! s (cdar s))
                     v))
  (define (empty! s) (set-car! s '())))

(library (balloons)
  (import (r6rs))
  (export make push pop)

  (define (make w h) (cons w h))
  (define (push b amt) (cons (- (car b) amt) (+ (cdr b) amt)))
  (define (pop b) (display "Boom! ") 
                  (display (* (car b) (cdr b))) 
                  (newline)))

(library (party)
  (import (r6rs)
          (only (stack) make push! pop!) ; not empty!
          (add-prefix (balloons) balloon:))
  ;; Total exports: make, push, push!, make-party, pop!
  (export (rename (balloon:make make)
	          (balloon:push push))
	  push!
	  make-party
	  (rename (party-pop! pop!)))

  ;; Creates a party as a stack of balloons, starting with
  ;;  two balloons
  (define (make-party)
    (let ([s (make)]) ; from stack
      (push! s (balloon:make 10 10))
      (push! s (balloon:make 12 9))
      s))
  (define (party-pop! p)
    (balloon:pop (pop! p))))


(library (main)
  (import (r6rs) (party))

  (define p (make-party))
  (pop! p)        ; displays "Boom! 108"
  (push! p (push (make 5 5) 1))
  (pop! p))       ; displays "Boom! 24"
\endschemedisplay

Examples for macros and phases:

\schemedisplay
(library (id-stuff)
  (import (r6rs))
  (export find-dup)

  (define (find-dup l)
    (and (pair? l)
         (let loop ((rest (cdr l)))
           (cond
            [(null? rest) (find-dup (cdr l))]
            [(bound-identifier=? (car l) (car rest)) (car rest)]
            [else (loop (cdr rest))])))))

(library (values-stuff)
  (import (r6rs) (import (for (id-stuff) expand)))
  (export (for mvlet expand run))

  (define-syntax mvlet
    (lambda (stx)
      (syntax-case stx ()
        [(_ [(id ...) expr] body0 body ...)
         (not (find-dup (syntax-object->list (syntax (id ...)))))
         (syntax (call-with-values (lambda () expr) 
                                   (lambda (id ...) body0 body ...)))]))))

(library (let-div)
  (import (r6rs) (mvlet))
  (export let-div)

  (define (quotient+remainder n d)
    (let ([q (quotient n d)])
      (values q (- n (* q d)))))
  (define-syntax let-div
    (syntax-rules ()
     [(_ n d (q r) body0 body ...)
      (mvlet [(q r) (quotient+remainder n d)]
        body0 body ...)])))
\endschemedisplay


\end{document}
 \par
\chapter{Top-level programs}
\label{programchapter}

A \defining{top-level program} specifies an entry point for defining and running
a Scheme program.  A top-level program specifies a set of libraries to import and
code to run.  Through the imported libraries, whether directly or through the
transitive closure of importing, a top-level program defines a complete Scheme
program.

Top-level programs follow the convention of many common platforms of accepting 
a list of string \defining{command-line arguments} that may be used to
pass data to the script.

\section{Top-level program syntax}
\label{programsyntaxsection}

A top-level program is a delimited piece of text, typically a file, that follows
the following syntax:
%
\begin{grammar}
\meta{toplevel program} \: \meta{import form} \meta{toplevel body}
\meta{import form} \: (import \arbno{\meta{import spec}})
\meta{toplevel body} \: \arbno{\meta{toplevel body form}}
\meta{toplevel body form} \: \meta{definition} \| \meta{expression}
\end{grammar}
%
The rules for \meta{toplevel program} specify syntax at the form level.

The \meta{import form} is identical to the import clause in
libraries (see section~\ref{librarysyntaxsection}), 
and specifies a set of libraries to import.  A \meta{toplevel 
 body} is like a \meta{library body} (see
section~\ref{librarybodysection}), except that 
definitions and expressions may occur in any order.  Thus, the syntax
specified by \meta{toplevel body form} refers to the result of macro
expansion.

\begin{rationale}
By allowing the interleaving of definitions and expressions, top-level 
programs support exploratory and interactive development, without 
imposing unnecessary organizational overhead on code which may not be 
intended for reuse.
\end{rationale}

When base-library {\cf begin} forms occur anywhere within a top-level body,
they are spliced into the body; see section~\ref{begin}.
Some or all of the top-level body, including portions wrapped in {\cf begin}
forms, may be specified by a syntactic abstraction
(see section~\ref{macrosection}).

\section{Top-level program semantics}

A top-level program is executed by treating the program similarly to a library, and
instantiating it.  The semantics of a top-level body may be roughly explained by
a simple translation into a library body: 
Each \hyper{expression} that appears before a
definition in
the top-level body is converted into a dummy definition 
{\cf (define \hyper{variable} (begin \hyper{expression} (unspecified)))},
where \hyper{variable} is a fresh identifier.
(It is generally impossible to determine which forms are 
definitions and expressions without concurrently expanding the body, so
the actual translation is somewhat more complicated; see
chapter~\ref{expansionchapter}.)

On platforms that support it, a top-level program may access its command-line 
arguments by calling the {\cf command-line} procedure (see library 
section~\extref{lib:command-line}{Command-line access and exit values}).

%%% Local Variables: 
%%% mode: latex
%%% TeX-master: "r6rs"
%%% End: 
 \par
\chapter{Expansion process}
\label{expansionchapter}

Macro uses (see section~\ref{macrosection}) are expanded into \textit{core
forms}\mainindex{core form} at the start of evaluation (before compilation or interpretation)
by a syntax \emph{expander}.
(The set of core forms is implementation-dependent, as is the
representation of these forms in the expander's output.)
If the expander encounters a syntactic abstraction, it invokes
the associated transformer to expand the syntactic abstraction, then
repeats the expansion process for the form returned by the transformer.
If the expander encounters a core form, it recursively
processes the subforms, if any, and reconstructs the form from the
expanded subforms.
Information about identifier bindings is maintained during expansion
to enforce lexical scoping for variables and keywords.

To handle internal definitions, the expander processes the initial
forms in a \hyper{body} (see section~\ref{bodiessection}) or
\hyper{library body} (see section~\ref{librarybodysection})
from left to
right.  How the expander processes each form encountered as it does so
depends upon the kind of form.

\begin{description}
\item[macro use]
The expander invokes the associated transformer to transform the macro
use, then recursively performs whichever of these actions are appropriate
for the resulting form.

\item[{\cf declaration} form]
If none of the body forms processed so far is a definition, the
declaration is handled in some implementation-dependent fashion.
It is a syntax violation for a declaration to appear after a definition.

\item[{\cf define-syntax} form]
The expander expands and evaluates the right-hand-side expression and binds the
keyword to the resulting transformer.

\item[{\cf define} form]
The expander records the fact that the defined identifier is a variable but defers
expansion of the right-hand-side expression until after all of the
definitions have been processed.

\item[{\cf begin} form]
The expander splices the subforms into the list of body forms it is
processing.  (See section~\ref{begin}.)

\item[{\cf let-syntax} or {\cf letrec-syntax} form]
The expander splices the inner body forms into the list of (outer) body forms it is
processing, arranging for the keywords bound by the {\cf let-syntax}
and {\cf letrec-syntax} to be visible only in the inner body forms.

\item[expression, i.e., nondefinition]
The expander completes the expansion of the deferred right-hand-side forms
and the current and remaining expressions in the body, then
constructs a residual {\cf letrec*} form from the defined variables,
expanded right-hand-side expressions, and expanded body expressions.
\end{description}

It is a syntax violation
if the keyword that identifies one of the body forms
as a definition (derived or core) is redefined by the same definition or a
later definition in the same body.
To detect this error, the expander records the identifying keyword for each
macro use, {\cf define-syntax} form, {\cf define}
form, {\cf begin} form, {\cf let-syntax} form, and {\cf letrec-syntax}
form it encounters while processing the definitions and checks each
newly defined identifier ({\cf define} or {\cf define-syntax}
left-hand side) against the recorded keywords, as with
{\cf bound-identifier=?} (see library section~\ref{lib:identifierpredicatessection}).
For example, the following forms are syntax violations.

\begin{scheme}
(let ()
  (define define 17)
  define)

(let-syntax ([def0 (syntax-rules ()
                     [(\_ x) (define x 0)])])
  (let ()
    (def0 z)
    (define def0 '(def 0))
    (list z def0)))
\end{scheme}

Expansion of each variable definition right-hand side is deferred until
after all of the definitions have been seen so that each keyword and
variable reference within the right-hand side resolves to the local
binding, if any.

Note that this algorithm does not directly reprocess any form.
It requires a single left-to-right pass over the definitions followed by a
single pass (in any order) over the body expressions and deferred
right-hand sides.

For example, in

\begin{scheme}
(lambda (x)
  (define-syntax defun
    (syntax-rules ()
      [(\_ (x . a) e) (define x (lambda a e))]))
  (defun (even? n) (or (= n 0) (odd? (- n 1))))
  (define-syntax odd?
    (syntax-rules () [(\_ n) (not (even? n))]))
  (odd? (if (odd? x) (* x x) x)))
\end{scheme}

The definition of {\cf defun} is encountered first, and the keyword
{\cf defun} is associated with the transformer resulting from
the expansion and evaluation of the corresponding right-hand side.
A use of {\cf defun} is encountered next and expands into a
{\cf define} form.
Expansion of the right-hand side of this define form is deferred.
The definition of {\cf odd?} is next and results in the association
of the keyword {\cf odd?} with the transformer resulting from
expanding and evaluating the corresponding right-hand side.
A use of {\cf odd?} appears next and is expanded; the resulting
call to {\cf not} is recognized as an expression
because {\cf not} is bound as a variable.
At this point, the expander completes the expansion of the current
expression (the {\cf not} call) and the deferred right-hand side of the
{\cf even?} definition;
the uses of {\cf odd?} appearing in these expressions are expanded
using the transformer associated with the keyword {\cf odd?}.
The final output is the equivalent of

\begin{scheme}
(lambda (x)
  (letrec* ([even?
              (lambda (n)
                (or (= n 0)
                    (not (even? (- n 1)))))])
    (not (even? (if (not (even? x)) (* x x) x)))))
\end{scheme}

although the structure of the output is implementation dependent.

Because definitions and expressions can be interleaved in a
\hyper{script body} (see chapter~\ref{scriptchapter}),
the expander's processing of a \hyper{script body} is somewhat
more complicated.
It behaves as described above for a \hyper{body} or
\hyper{library body} with the following exceptions.
First, it treats declarations that appear after any definitions or
expressions as if they appeared before all of the definitions and
expressions.
Second, when the expander finds a nondefinition,
it defers its expansion and continues scanning for definitions.
Once it reaches the end of set of forms, it processes the
deferred right-hand-side and body expressions, then
constructs a residual {\cf letrec*} form from the defined variables,
expanded right-hand-side expressions, and expanded body expressions.
For each body expression that appears before a variable definition
in the body, a dummy binding is created at the corresponding place within
the set of {\cf letrec*} bindings, with a fresh temporary variable on the
left-hand side and the expression on the right-hand side, so that
left-to-right evaluation order is preserved.

%%% Local Variables: 
%%% mode: latex
%%% TeX-master: "r6rs"
%%% End: 
 \par
%\vfill\eject
\chapter{Base library}
\label{baselibrarychapter}

\newcommand{\syntax}{{\em Syntax: }}
\newcommand{\semantics}{{\em Semantics: }}

This chapter describes Scheme's base library.  FIXME  The initial (or
``top level'') Scheme environment starts out with a number of variables
bound to locations containing useful values, most of which are primitive
procedures that manipulate data.  For example, the variable {\cf abs} is
bound to (a location initially containing) a procedure of one argument
that computes the absolute value of a number, and the variable {\cf +}
is bound to a procedure that computes sums.

A program may use a top-level definition to bind any variable.  It may
subsequently alter any such binding by an assignment (see \ref{assignment}).
These operations do not modify the behavior of Scheme's built-in
procedures.  Altering any top-level binding that has not been introduced by a
definition has an unspecified effect on the behavior of the built-in procedures.

%[Deleted for R5RS because of multiple-value returns. -RK]
%A Scheme expression is a construct that returns a value, such as a
%variable reference, literal, procedure call, or conditional.

Section~\ref{basetailcontextsection} defines the rules that identify
tail calls and tail contexts in base-library constructs.

\section{Base types}
\label{disjointness}

FIXME: write something on base types

No object satisfies more than one of the following predicates:

\begin{scheme}
boolean?          pair?
symbol?           number?
char?             string?
vector?           procedure?
unspecified? 
\end{scheme}

These predicates define the types {\em boolean}, {\em pair}, {\em
symbol}, {\em number}, {\em char} (or {\em character}), {\em string}, {\em
vector}, and {\em procedure}.  The empty list is a special
object of its own type; it satisfies none of the above predicates.
The same holds for the unspecified value, which is also distinct
from the empty list.
\mainindex{type}\schindex{boolean?}\schindex{pair?}\schindex{symbol?}
\schindex{number?}\schindex{char?}\schindex{string?}\schindex{vector?}
\schindex{procedure?}\index{empty list}\index{unspecified value}

Although there is a separate boolean type,
any Scheme value can be used as a boolean value for the purpose of a
conditional test.  As explained in section~\ref{booleansection}, all
values count as true in such a test except for \schfalse{}.
% and possibly the empty list.
% The only value that is guaranteed to count as
% false is \schfalse{}.  It is explicitly unspecified whether the empty list
% counts as true or as false.
This report uses the word ``true'' to refer to any
Scheme value except \schfalse{}, and the word ``false'' to refer to
\schfalse{}. \mainindex{true} \mainindex{false}

\section{Declarations}
\label{declarations}

\schindex{declarations}
A declaration affects a range of code, and indicates that the code
within that range should be compiled or executed to have certain
qualities.  A declaration appears at the beginning of a \hyper{body}
(see section~\ref{bodiessection}), and its range is the library in which it
appears, the expression that immediately surrounds the \hyper{body} at
whose head it appears, or the group of definitions at whose head it
appears, possibly including code inserted in the range in the course
of macro expansion.

A declaration \hyper{declaration} has the following form:\mainschindex{declare}

{\cf (declare \hyper{declare spec})}

A \hyper{declare spec} has one of the following forms:

\begin{itemize}
\item {\cf (\hyper{quality} \hyper{priority})}

  \hyper{Quality} has to be one of {\cf safe}, {\cf fast}, {\cf
    small}, and {\cf debug}.  \hyper{Priority} has to be one of {\cf
    0}, {\cf 1}, {\cf 2}, and {\cf 3}.

  This specifies that the code in the range of the declaration should
  have the indicated quality at the indicated priority, where priority
  3 is the highest priority.  Priority 0 means the quality is not a
  priority at all.

\item {\cf \hyper{quality}}
  
  This is a synonym for {\cf (\hyper{quality} 3)}.
  
\item {\cf unsafe}

  This is a synonym for {\cf (safe 0)}.
\end{itemize}

For {\cf safe}, the default priority must be 1 or higher.  When the
priority for {\cf safe} is 1 or higher, implementations must raise all
required exceptions and let them be handled by the exception mechanism
(see chapter~\ref{exceptionsconditionschapter}).

Beyond that, the detailed interpretation of declarations will vary in
different implementations.  In particular, implementations are free to
ignore declarations, and may observe some declarations while ignoring
others.

The following descriptions of each quality may provide some guidance
for programmers and implementors.

\paragraph{{\tt safe}}

This quality's priority influences the degree of checking for
exceptional situations, and the raising and handling of exceptions in
response to those situations.  The higher the priority, the more
likely an exception will be raised.

At priority 0, an implementation is allowed to ignore any requirements
for raising an exception with condition type {\cf\&violation} (or one
of its subtypes).  In situations for which this report allows or
requires the imple mentation to raise an exception with condition type
{\cf\&violation}, the implementation may ignore the situation and
continue the computation with an incorrect result, may terminate the
computation in an unpleasant fashion, or may destroy the invariants of
run-time data structures in ways that cause unexpected and mysterious
misbehavior even in code that comes within the scope of a safe
declaration.  All bets are off.

At priority 1 and higher, an implementation must raise all exceptions
required by this report, handle those exceptions using the exception
mechanism described in chapter~\ref{exceptionsconditionschapter}, and
use the default exception handlers described in that chapter.  See
also section~\ref{safetysection}.

At higher priorities, implementations may be more likely to raise
exceptions that are allowed but not required by this report.

Most implementations are able to recognize some defects when
parsing, expanding macros, or compiling a definition or expression
whose evaluation has not yet commenced in the usual sense.
Implementations are allowed to use nonstandard exception handlers at
those times, and are encouraged to raise {\cf\&syntax} exceptions for
defects detected at those times, even if the definition or
expression that contains the violation will never be executed.
Implementations are also allowed to raise a {\cf\&warning} exception
at those times if they determine that some subexpression would
inevitably raise some kind of {\cf\&violation} exception were it ever
to be evaluated.

\paragraph{{\tt fast}}

This quality's priority influences the speed of the code it governs.
At high priorities, the code is likely to run faster, but that
improvement is constrained by other qualities and may come at the
expense of the small and debug qualities.

\paragraph{{\tt small}}

This quality's priority influences the amount of computer memory
needed to represent and to run the code.  At high priorities, the code
is likely to occupy less memory and to require less memory during
evaluation.

\paragraph{{\tt debug}}

This quality's priority influences the programmer's ability to debug
the code.  At high priorities, the programmer is more likely to
understand the correspondence between the original source code and
information displayed by debugging tools.  At low priorities, some
debugging tools may not be usable.

\section{Definitions}
\label{defines}

Definitions are valid in some, but not all, contexts where expressions
are allowed.  They are valid only at the top level of a \hyper{program}
and after the declarations in a \hyper{body} (see section~\ref{bodiessection}).
\mainindex{definition}

A definition \hyper{definition} should have one of the following forms:\mainschindex{define}

\begin{itemize}

\item{\tt(define \hyper{variable} \hyper{expression})}
  This will bind \hyper{variable} to a new
  location before assigning the value of \hyper{expression} to it.
\begin{scheme}
(define add3
  (lambda (x) (+ x 3)))
(add3 3)                            \ev  6
(define first car)
(first '(1 2))                      \ev  1%
\end{scheme}

\item{\tt(define \hyper{variable})}

This form is equivalent to
\begin{scheme}
(define \hyper{variable} (unspecified))
\end{scheme}

\item{\tt(define (\hyper{variable} \hyper{formals}) \hyper{body})}

\hyper{Formals} should be either a
sequence of zero or more variables, or a sequence of one or more
variables followed by a space-delimited period and another variable (as
in a lambda expression, see section~\ref{lambda}).  This form is equivalent to
\begin{scheme}
(define \hyper{variable}
  (lambda (\hyper{formals}) \hyper{body}))\rm.%
\end{scheme}

\item{\tt(define (\hyper{variable} .\ \hyper{formal}) \hyper{body})}

\hyper{Formal} should be a single
variable.  This form is equivalent to
\begin{scheme}
(define \hyper{variable}
  (lambda \hyper{formal} \hyper{body}))\rm.%
\end{scheme}

\end{itemize}


\section{Syntax definitions}

Syntax definitions are valid only at the top level of a
\hyper{program}.  \mainindex{syntax definition} Syntax definitions are
also \hyper{definition}s and have the following
form:\mainschindex{define-syntax}

{\tt(define-syntax \hyper{keyword} \hyper{transformer spec})}

\hyper{Keyword} is an identifier, and
the \hyper{transformer spec} should be an instance of \ide{syntax-rules}.
The top-level syntactic environment is extended by binding the
\hyper{keyword} to the specified transformer.

FIXME (library chapter?)

Although macros may expand into definitions and syntax definitions in
any context that permits them, it is an error for a definition or syntax
definition to shadow a syntactic keyword whose meaning is needed to
determine whether some form in the group of forms that contains the
shadowing definition is in fact a definition, or, for internal definitions,
is needed to determine the boundary between the group and the expressions
that follow the group.  For example, the following are errors:

\begin{scheme}
(define define 3)

(begin (define begin list))

(let-syntax
  ((foo (syntax-rules ()
          ((foo (proc args ...) body ...)
           (define proc
             (lambda (args ...)
               body ...))))))
  (let ((x 3))
    (foo (plus x y) (+ x y))
    (define foo x)
    (plus foo x)))
\end{scheme}

\section{Bodies and sequences}
\label{bodiessection}

\index{body}The body \hyper{body} of a \ide{lambda}, \ide{let}, \ide{let*},
\ide{let-values}, \ide{let*-values}, \ide{letrec*}, \ide{letrec},
\ide{let-syntax}, or \ide{letrec-syntax} expression or that of a
definition with a body has the following form:

{\cf \arbno{\hyper{declaration}} \arbno{\hyper{definition}} \hyper{sequence}}

\hyper{Declaration} is according to section~\ref{declarations}.

A \hyper{sequence} has the following form:

{\cf \arbno{\hyper{expression}}}

Definitions may occur after the declarations in a \hyper{body}.
Such definitions are known as {\em internal definitions}
\mainindex{internal definition} as opposed to the top level
definitions described above. 

With \ide{lambda}, \ide{let}, \ide{let*}, \ide{let-values},
\ide{let*-values}, \ide{letrec*}, and \ide{letrec},
the variable defined by an internal
definition is local to the \hyper{body}.  That is, \hyper{variable} is
bound rather than assigned, and the region of the binding is the
entire \hyper{body}.  For example,

\begin{scheme}
(let ((x 5))
  (define foo (lambda (y) (bar x y)))
  (define bar (lambda (a b) (+ (* a b) a)))
  (foo (+ x 3)))                \ev  45%
\end{scheme}

An expanded \hyper{body} containing internal definitions, where the
declarations and original syntax definitions have been elided, can
always be converted into a completely equivalent {\cf letrec*}
expression.  For example, the {\cf let} expression in the above
example is equivalent to

\begin{scheme}
(let ((x 5))
  (letrec* ((foo (lambda (y) (bar x y)))
            (bar (lambda (a b) (+ (* a b) a))))
    (foo (+ x 3))))%
\end{scheme}

In the body of a {\cf let-values} or {\cf letrec-values} form, the
region of the binding of an internal definition is the same as that of
an internal definition that would appear in place of the {\cf
  let-values} or {\cf letrec-values} form.  That is, a {\cf
  let-values} or {\cf letrec-values} form does not establish a new
region for internal definitions appearing in its body; the expanded
body is spliced into the surrounding program in the place of the {\cf
  let-values} or {\cf letrec-values} form.

\begin{scheme}
(let ()
  (let-syntax
    ((def (syntax-rules ()
            ((def stuff ...) (define stuff ...)))))
    (def foo 42))
  foo) \ev 42
\end{scheme}

Wherever an internal definition may occur
{\tt(begin \hyperi{definition} \dotsfoo)}
is equivalent to the sequence of definitions
that form the body of the \ide{begin}.


\section{Expressions}
\label{expressionsection}

The entries in this section describe the expressions of the base
language, which may occur in the position of the \hyper{expression}
syntactic variable.  The expressions also include constant literals,
variable references and procedure calls a described in
section~\ref{primitiveexpressionsection}.

\subsection{Literal expressions}\unsection
\label{literalsection}

\begin{entry}{%
\proto{quote}{ \hyper{S-expression}}{\exprtype}}

{\cf (quote \hyper{S-expression})} evaluates to the datum
denoted by \hyper{S-expression}.
(See
section~\ref{readsyntaxsection}.).  This notation is used to include literal
constants in Scheme code.

\begin{scheme}%
(quote a)                     \ev  a
(quote \sharpsign(a b c))     \ev  \#(a b c)
(quote (+ 1 2))               \ev  (+ 1 2)%
\end{scheme}

As noted in section~\ref{quotesection}, {\cf (quote \hyper{S-expression})}
may be abbreviated as \singlequote\hyper{S-expression}:

\begin{scheme}
'"abc"               \ev  "abc"
'145932              \ev  145932
'a                   \ev  a
'\#(a b c)           \ev  \#(a b c)
'()                  \ev  ()
'(+ 1 2)             \ev  (+ 1 2)
'(quote a)           \ev  (quote a)
''a                  \ev  (quote a)%
\end{scheme}

Numerical constants, string constants, character constants, and boolean
constants evaluate ``to themselves''; they need not be quoted.

\begin{scheme}
'"abc"     \ev  "abc"
"abc"      \ev  "abc"
'145932    \ev  145932
145932     \ev  145932
'\schtrue  \ev  \schtrue
\schtrue   \ev  \schtrue%
\end{scheme}

As noted in section~\ref{storagemodel}, the value of a literal
expression may be immutable.
\end{entry}


\subsection{Procedures}\unsection
\label{lamba}

\begin{entry}{%
\proto{lambda}{ \hyper{formals} \hyper{body}}{\exprtype}}

\syntax
\hyper{Formals} must be a formal arguments list as described below,
and \hyper{body} must be according to section~\ref{bodiessection}.

\semantics
\vest A \lambdaexp{} evaluates to a procedure.  The environment in
effect when the \lambdaexp{} was evaluated is remembered as part of the
procedure.  When the procedure is later called with some actual
arguments, the environment in which the \lambdaexp{} was evaluated will
be extended by binding the variables in the formal argument list to
fresh locations, the corresponding actual argument values will be stored
in those locations, and the expressions in the body of the \lambdaexp{}
will be evaluated sequentially in the extended environment.
The result(s) of the last expression in the body will be returned as
the result(s) of the procedure call.

\begin{scheme}
(lambda (x) (+ x x))      \ev  {\em{}a procedure}
((lambda (x) (+ x x)) 4)  \ev  8

(define reverse-subtract
  (lambda (x y) (- y x)))
(reverse-subtract 7 10)         \ev  3

(define add4
  (let ((x 4))
    (lambda (y) (+ x y))))
(add4 6)                        \ev  10%
\end{scheme}

\hyper{Formals} should have one of the following forms:

\begin{itemize}
\item {\tt(\hyperi{variable} \dotsfoo)}:
The procedure takes a fixed number of arguments; when the procedure is
called, the arguments will be stored in the bindings of the
corresponding variables.

\item \hyper{variable}:
The procedure takes any number of arguments; when the procedure is
called, the sequence of actual arguments is converted into a newly
allocated list, and the list is stored in the binding of the
\hyper{variable}.

\item {\tt(\hyperi{variable} \dotsfoo{} \hyper{variable$_{n}$}\ {\bf.}\
\hyper{variable$_{n+1}$})}:
If a space-delimited period precedes the last variable, then
the procedure takes $n$ or more arguments, where $n$ is the
number of formal arguments before the period (there must
be at least one).
The value stored in the binding of the last variable will be a
newly allocated
list of the actual arguments left over after all the other actual
arguments have been matched up against the other formal arguments.
\end{itemize}

It is a syntax defect for a \hyper{variable} to appear more than once in
\hyper{formals}.

\begin{scheme}
((lambda x x) 3 4 5 6)          \ev  (3 4 5 6)
((lambda (x y . z) z)
 3 4 5 6)                       \ev  (5 6)%
\end{scheme}

Each procedure created as the result of evaluating a \lambdaexp{} is
(conceptually) tagged
with a storage location, in order to make \ide{eqv?} and
\ide{eq?} work on procedures (see section~\ref{equivalencesection}).

\end{entry}


\subsection{Conditionals}\unsection

\begin{entry}{%
\proto{if}{ \hyper{test} \hyper{consequent} \hyper{alternate}}{\exprtype}
\rproto{if}{ \hyper{test} \hyper{consequent}}{\exprtype}}  %\/ if hyper = italic

\syntax
\hyper{Test}, \hyper{consequent}, and \hyper{alternate} may be arbitrary
expressions.

\semantics
An {\cf if} expression is evaluated as follows: first,
\hyper{test} is evaluated.  If it yields a true value\index{true} (see
section~\ref{booleansection}), then \hyper{consequent} is evaluated and
its value(s) is(are) returned.  Otherwise \hyper{alternate} is evaluated and its
value(s) is(are) returned.  If \hyper{test} yields a false value and no
\hyper{alternate} is specified, then the result of the expression is
the unspecified value.

\begin{scheme}
(if (> 3 2) 'yes 'no)           \ev  yes
(if (> 2 3) 'yes 'no)           \ev  no
(if (> 3 2)
    (- 3 2)
    (+ 3 2))                    \ev  1
(if \#f \#f)                    \ev \theunspecified%
\end{scheme}

\end{entry}


\subsection{Assignments}\unsection
\label{assignment}

\begin{entry}{%
\proto{set!}{ \hyper{variable} \hyper{expression}}{\exprtype}}

\hyper{Expression} is evaluated, and the resulting value is stored in
the location to which \hyper{variable} is bound.  \hyper{Variable} must
be bound either in some region\index{region} enclosing the {\cf set!}\ expression
or at top level.  The result of the {\cf set!} expression is
the unspecified value.

\begin{scheme}
(define x 2)
(+ x 1)                 \ev  3
(set! x 4)              \ev  \theunspecified
(+ x 1)                 \ev  5%
\end{scheme}

It is a syntax defect if the binding \hyper{variable} refers to is
immutable.
\end{entry}

\subsection{Conditionals}\unsection

\begin{entry}{%
\proto{cond}{ \hyperi{clause} \hyperii{clause} \dotsfoo}{\exprtype}}

\syntax
Each \hyper{clause} should be of the form
\begin{scheme}
(\hyper{test} \hyperi{expression} \dotsfoo)%
\end{scheme}
where \hyper{test} is any expression.  Alternatively, a \hyper{clause} may be
of the form
\begin{scheme}
(\hyper{test} => \hyper{expression})%
\end{scheme}
The last \hyper{clause} may be
an ``else clause,'' which has the form
\begin{scheme}
(else \hyperi{expression} \hyperii{expression} \dotsfoo)\rm.%
\end{scheme}
\mainschindex{else}
\mainschindex{=>}

\semantics
A {\cf cond} expression is evaluated by evaluating the \hyper{test}
expressions of successive \hyper{clause}s in order until one of them
evaluates to a true value\index{true} (see
section~\ref{booleansection}).  When a \hyper{test} evaluates to a true
value, then the remaining \hyper{expression}s in its \hyper{clause} are
evaluated in order, and the result(s) of the last \hyper{expression} in the
\hyper{clause} is(are) returned as the result(s) of the entire {\cf cond}
expression.  If the selected \hyper{clause} contains only the
\hyper{test} and no \hyper{expression}s, then the value of the
\hyper{test} is returned as the result.  If the selected \hyper{clause} uses the
\ide{=>} alternate form, then the \hyper{expression} is evaluated.
Its value must be a procedure that accepts one argument; this procedure is then
called on the value of the \hyper{test} and the value(s) returned by this
procedure is(are) returned by the {\cf cond} expression.
If all \hyper{test}s evaluate
to false values, and there is no else clause, then the result of
the conditional expression is the unspecified value; if there is an else
clause, then its \hyper{expression}s are evaluated, and the value(s) of
the last one is(are) returned.

\begin{scheme}
(cond ((> 3 2) 'greater)
      ((< 3 2) 'less))         \ev  greater%

(cond ((> 3 3) 'greater)
      ((< 3 3) 'less)
      (else 'equal))            \ev  equal%

(cond ((assv 'b '((a 1) (b 2))) => cadr)
      (else \schfalse{}))         \ev  2%
\end{scheme}

{\cf Cond} could be defined in terms of {\cf if}, {\cf let} and {\cf
  begin} using {\cf syntax-rules} (see
section~\ref{syntaxrulessection}) as follows:

\begin{scheme}
(define-syntax \ide{cond}
  (syntax-rules (else =>)
    ((cond (else result1 result2 ...))
     (begin result1 result2 ...))
    ((cond (test => result))
     (let ((temp test))
       (if temp (result temp))))
    ((cond (test => result) clause1 clause2 ...)
     (let ((temp test))
       (if temp
           (result temp)
           (cond clause1 clause2 ...))))
    ((cond (test)) test)
    ((cond (test) clause1 clause2 ...)
     (let ((temp test))
       (if temp
           temp
           (cond clause1 clause2 ...))))
    ((cond (test result1 result2 ...))
     (if test (begin result1 result2 ...)))
    ((cond (test result1 result2 ...)
           clause1 clause2 ...)
     (if test
         (begin result1 result2 ...)
         (cond clause1 clause2 ...)))))
\end{scheme}
\end{entry}


\begin{entry}{%
\proto{case}{ \hyper{key} \hyperi{clause} \hyperii{clause} \dotsfoo}{\exprtype}}

\syntax
\hyper{Key} may be any expression.  Each \hyper{clause} should have
the form
\begin{scheme}
((\hyperi{S-expression} \dotsfoo) \hyperi{expression} \hyperii{expression} \dotsfoo)\rm,%
\end{scheme}
where each \hyper{S-expression} is an external representation of some object.
The datums denoted by the the \hyper{S-expression}s must be distinct.
The last \hyper{clause} may be an ``else clause,'' which has the form
\begin{scheme}
(else \hyperi{expression} \hyperii{expression} \dotsfoo)\rm.%
\end{scheme}
\schindex{else}

\semantics
A {\cf case} expression is evaluated as follows.  \hyper{Key} is
evaluated and its result is compared against each the datum denoted by
each \hyper{S-expression}.  If the
result of evaluating \hyper{key} is equivalent (in the sense of
{\cf eqv?}; see section~\ref{eqv?}) to a datum, then the
expressions in the corresponding \hyper{clause} are evaluated from left
to right and the result(s) of the last expression in the \hyper{clause} is(are)
returned as the result(s) of the {\cf case} expression.  If the result of
evaluating \hyper{key} is different from every datum, then if
there is an else clause its expressions are evaluated and the
result(s) of the last is(are) the result(s) of the {\cf case} expression;
otherwise the result of the {\cf case} expression is the unspecified value.

\begin{scheme}
(case (* 2 3)
  ((2 3 5 7) 'prime)
  ((1 4 6 8 9) 'composite))     \ev  composite
(case (car '(c d))
  ((a) 'a)
  ((b) 'b))                     \ev  \theunspecified
(case (car '(c d))
  ((a e i o u) 'vowel)
  ((w y) 'semivowel)
  (else 'consonant))            \ev  consonant%
\end{scheme}

{\cf Case} can be defined in terms of {\cf let}, {\cf cond}, and
{\cf memv} (see section~\ref{listutilities} using {\cf syntax-rules}
(see section~\ref{syntaxrulessection}) as follows:

\begin{scheme}
(define-syntax \ide{case}
  (syntax-rules (else)
    ((case expr0
       ((key ...) res1 res2 ...)
       ...
       (else else-res1 else-res2 ...))
     (let ((tmp expr0))
       (cond
         ((memv tmp '(key ...)) res1 res2 ...)
         ...
         (else else-res1 else-res2 ...))))
    ((case expr0
       ((keya ...) res1a res2a ...)
       ((keyb ...) res1b res2b ...)
       ...)
     (let ((tmp expr0))
       (cond
         ((memv tmp '(keya ...)) res1a res2a ...)
         ((memv tmp '(keyb ...)) res1b res2b ...)
         ...)))))
\end{scheme}

\end{entry}


\begin{entry}{%
\proto{and}{ \hyperi{test} \dotsfoo}{\exprtype}}

The \hyper{test} expressions are evaluated from left to right, and the
value of the first expression that evaluates to a false value (see
section~\ref{booleansection}) is returned.  Any remaining expressions
are not evaluated.  If all the expressions evaluate to true values, the
value of the last expression is returned.  If there are no expressions
then \schtrue{} is returned.

\begin{scheme}
(and (= 2 2) (> 2 1))           \ev  \schtrue
(and (= 2 2) (< 2 1))           \ev  \schfalse
(and 1 2 'c '(f g))             \ev  (f g)
(and)                           \ev  \schtrue%
\end{scheme}

{\cf And} could be defined in terms of {\cf if} using {\cf
  syntax-rules} (see section~\ref{syntaxrulessection}) as follows:

\begin{scheme}
(define-syntax \ide{and}
  (syntax-rules ()
    ((and) \sharpfoo{t})
    ((and test) test)
    ((and test1 test2 ...)
     (if test1 (and test2 ...) \sharpfoo{f}))))
\end{scheme}
\end{entry}


\begin{entry}{%
\proto{or}{ \hyperi{test} \dotsfoo}{\exprtype}}

The \hyper{test} expressions are evaluated from left to right, and the value of the
first expression that evaluates to a true value (see
section~\ref{booleansection}) is returned.  Any remaining expressions
are not evaluated.  If all expressions evaluate to false values, the
value of the last expression is returned.  If there are no
expressions then \schfalse{} is returned.

\begin{scheme}
(or (= 2 2) (> 2 1))            \ev  \schtrue
(or (= 2 2) (< 2 1))            \ev  \schtrue
(or \schfalse \schfalse \schfalse) \ev  \schfalse
(or (memq 'b '(a b c)) 
    (/ 3 0))                    \ev  (b c)%
\end{scheme}

{\cf Or} could be defined in terms of {\cf if} using {\cf
  syntax-rules} (see section~\ref{syntaxrulessection}) as follows:

\begin{scheme}
(define-syntax \ide{or}
  (syntax-rules ()
    ((or) \sharpfoo{f})
    ((or test) test)
    ((or test1 test2 ...)
     (let ((x test1))
       (if x x (or test2 ...))))))
\end{scheme}
\end{entry}


\subsection{Binding constructs}

The four binding constructs {\cf let}, {\cf let*}, {\cf letrec*}, and {\cf letrec}
give Scheme a block structure, like Algol 60.  The syntax of the four
constructs is identical, but they differ in the regions\index{region} they establish
for their variable bindings.  In a {\cf let} expression, the initial
values are computed before any of the variables become bound; in a
{\cf let*} expression, the bindings and evaluations are performed
sequentially; while in a {\cf letrec*} and in a {\cf letrec}
expression, all the bindings are in
effect while their initial values are being computed, thus allowing
mutually recursive definitions.

In addition, the binding constructs {\cf let-values} and {\cf
  let*-values} allow the binding of results of expression returning
multiple values.  They are analogous to {\cf let} and {\cf let*} the
way they establish regions: in a {\cf let-values} expression, the
initial values are computed before any of the variables become bound;
in a {\cf let*-values} expression, the bindings are performed
sequentially.  (These forms are compatible with SRFI~11~\cite{srfi11}.)

\begin{entry}{%
\proto{let}{ \hyper{bindings} \hyper{body}}{\exprtype}}

\syntax
\hyper{Bindings} should have the form
\begin{scheme}
((\hyperi{variable} \hyperi{init}) \dotsfoo)\rm,%
\end{scheme}
where each \hyper{init} is an expression, and \hyper{body} should be a
sequence of one or more expressions.  It is
syntax defect for a \hyper{variable} to appear more than once in the list of variables
being bound.

\semantics
The \hyper{init}s are evaluated in the current environment (in some
unspecified order), the \hyper{variable}s are bound to fresh locations
holding the results, the \hyper{body} is evaluated in the extended
environment, and the value(s) of the last expression of \hyper{body}
is(are) returned.  Each binding of a \hyper{variable} has \hyper{body} as its
region.\index{region}

\begin{scheme}
(let ((x 2) (y 3))
  (* x y))                      \ev  6

(let ((x 2) (y 3))
  (let ((x 7)
        (z (+ x y)))
    (* z x)))                   \ev  35%
\end{scheme}

See also named {\cf let}, section \ref{namedlet}.

\end{entry}


\begin{entry}{%
\proto{let*}{ \hyper{bindings} \hyper{body}}{\exprtype}}\nobreak

\nobreak
\syntax
\hyper{Bindings} should have the form
\begin{scheme}
((\hyperi{variable} \hyperi{init}) \dotsfoo)\rm,%
\end{scheme}
and \hyper{body} should be a sequence of
one or more expressions.

\semantics
{\cf Let*} is similar to {\cf let}, but the bindings are performed
sequentially from left to right, and the region\index{region} of a binding indicated
by {\cf(\hyper{variable} \hyper{init})} is that part of the {\cf let*}
expression to the right of the binding.  Thus the second binding is done
in an environment in which the first binding is visible, and so on.

\begin{scheme}
(let ((x 2) (y 3))
  (let* ((x 7)
         (z (+ x y)))
    (* z x)))             \ev  70%
\end{scheme}

{\cf Let*} could be defined in terms of {\cf let} using {\cf
  syntax-rules} (see section~\ref{syntaxrulessection}) as follows:

\begin{scheme}
(define-syntax \ide{let*}
  (syntax-rules ()
    ((let* () body1 body2 ...)
     (let () body1 body2 ...))
    ((let* ((name1 expr1) (name2 expr2) ...)
       body1 body2 ...)
     (let ((name1 expr1))
       (let* ((name2 expr2) ...)
         body1 body2 ...)))))
\end{scheme}

\end{entry}

\begin{entry}{%
\proto{letrec}{ \hyper{bindings} \hyper{body}}{\exprtype}}

\syntax
\hyper{Bindings} should have the form
\begin{scheme}
((\hyperi{variable} \hyperi{init}) \dotsfoo)\rm,%
\end{scheme}
and \hyper{body} should be a sequence of
one or more expressions. It is a syntax defect for a \hyper{variable} to appear more
than once in the list of variables being bound.

\semantics
The \hyper{variable}s are bound to fresh locations holding undefined
values, the \hyper{init}s are evaluated in the resulting environment (in
some unspecified order), each \hyper{variable} is assigned to the result
of the corresponding \hyper{init}, the \hyper{body} is evaluated in the
resulting environment, and the value(s) of the last expression in
\hyper{body} is(are) returned.  Each binding of a \hyper{variable} has the
entire {\cf letrec} expression as its region\index{region}, making it possible to
define mutually recursive procedures.

\begin{scheme}
%(letrec ((x 2) (y 3))
%  (letrec ((foo (lambda (z) (+ x y z))) (x 7))
%    (foo 4)))                   \ev  14
%
(letrec ((even?
          (lambda (n)
            (if (zero? n)
                \schtrue
                (odd? (- n 1)))))
         (odd?
          (lambda (n)
            (if (zero? n)
                \schfalse
                (even? (- n 1))))))
  (even? 88))   
                \ev  \schtrue%
\end{scheme}

One restriction on {\cf letrec} is very important: it must be possible
to evaluate each \hyper{init} without assigning or referring to the value of any
\hyper{variable}.  If this restriction is violated, an exception  with
condition type {\cf\&contract} is
raised.  The restriction is necessary because Scheme passes arguments by value rather than by
name.
In the most common uses of {\cf letrec}, all the \hyper{init}s are
\lambdaexp{}s and the restriction is satisfied automatically.

{\cf Letrec} can be defined approximately in terms of {\cf let}
and {\cf set!} using {\cf syntax-rules} (see
section~\ref{syntaxrulessection}), using a helper
to generate the temporary variables
needed to hold the values before the assignments are made,
as follows:

\begin{scheme}
(define-syntax \ide{letrec}
  (syntax-rules ()
    ((letrec () body1 body2 ...)
     (let () body1 body2 ...))
    ((letrec ((var init) ...) body1 body2 ...)
     (letrec-helper
       (var ...)
       ()
       ((var init) ...)
       body1 body2 ...))))

(define-syntax letrec-helper
  (syntax-rules ()
    ((letrec-helper
       ()
       (temp ...)
       ((var init) ...)
       body1 body2 ...)
     (let ((var <undefined>) ...)
       (let ((temp init) ...)
         (set! var temp)
         ...)
       (let () body1 body2 ...)))
    ((letrec-helper
       (x y ...)
       (temp ...)
       ((var init) ...)
       body1 body2 ...)
     (letrec-helper
       (y ...)
       (newtemp temp ...)
       ((var init) ...)
       body1 body2 ...))))
\end{scheme}

The syntax {\cf <undefined>} represents an expression that
returns something that, when stored in a location, causes an exception to
be raised if an attempt to read to or write from the location occurs before the
assignments generated by the {\cf letrec} transformation take place.
(No such expression is defined in Scheme.)
\end{entry}

\begin{entry}{%
\proto{letrec*}{ \hyper{bindings} \hyper{body}}{\exprtype}}

\syntax
\hyper{Bindings} should have the form
\begin{scheme}
((\hyperi{variable} \hyperi{init}) \dotsfoo)\rm,%
\end{scheme}
and \hyper{body} should be a sequence of
one or more expressions. It is a syntax defect for a \hyper{variable} to appear more
than once in the list of variables being bound.

\semantics
The \hyper{variable}s are bound to fresh locations holding undefined
values, each \hyper{variable} is assigned in left-to-right order to the
result of evaluating the corresponding \hyper{init}, the \hyper{body} is
evaluated in the resulting environment, and the value(s) of the last
expression in \hyper{body} is(are) returned. 
Despite the left-to-right evaluation and assignment order, each binding of
a \hyper{variable} has the entire {\cf letrec*} expression as its
region\index{region}, making it possible to define mutually recursive
procedures.

\begin{scheme}
(letrec* ((p
           (lambda (x)
             (+ 1 (q (- x 1)))))
          (q
           (lambda (y)
             (if (zero? y)
                 0
                 (+ 1 (p (- y 1))))))
          (x (p 5))
          (y x))
  y)
                \ev  5%
\end{scheme}

One restriction on {\cf letrec*} is very important: it must be possible
to evaluate each \hyper{init} without assigning or referring to the value
the corresponding \hyper{variable} or the \hyper{variable} of any of
the bindings that follow it in \hyper{bindings}.
If this restriction is violated, an exception with condition type
{\cf\&contract} is raised.
The restriction is necessary because Scheme passes arguments by value
rather than by name. 

{\cf Letrec*} can be defined approximately in terms of {\cf let}
and {\cf set!}
using {\cf syntax-rules} (see section~\ref{syntaxrulessection})
as follows:

\begin{scheme}
(define-syntax \ide{letrec*}
  (syntax-rules ()
    ((letrec* ((var1 init1) ...) body1 body2 ...)
     (let ((var1 <undefined>) ...)
       (set! var1 init1)
       ...
       (let () body1 body2 ...)))))
\end{scheme}

The syntax {\cf <undefined>} represents an expression that
returns something that, when stored in a location, causes an exception to
be raised if an attempt to read from or write to the location occurs before the
assignments generated by the {\cf letrec*} transformation take place.
(No such expression is defined in Scheme.)
\end{entry}

% \todo{use or uses?  --- Jinx.}

\begin{entry}{%
\proto{let-values}{ \hyper{mv-bindings} \hyper{body}}{\exprtype}}

\syntax
\hyper{Mv-bindings} should have the form
\begin{scheme}
((\hyperi{formals} \hyperi{init}) \dotsfoo)\rm,%
\end{scheme}
and \hyper{body} should be a sequence of
one or more expressions. It is syntax defect for a variable to appear more
than once in the list of variables that appear as part of the formals.

\semantics The \hyper{init}s are evaluated in the current environment
(in some unspecified order), and the variables occurring in the
\hyper{formals} are bound to fresh locations containing the values
returned by the \hyper{init}s, where the \hyper{formals} are matched
to the return values in the same way that the \hyper{formals} in a
\lambdaexp{} are matched to the actual arguments in a procedure call.
Then, the \hyper{body} is evaluated in the extended environment, and the
value(s) of the last expression of \hyper{body} is(are) returned.
Each binding of a variable has \hyper{body} as its
region.\index{region}

\begin{scheme}
(let-values (((a b) (values 1 2))
             ((c d) (values 3 4)))
  (list a b c d)) \ev (1 2 3 4)

(let-values (((a b . c) (values 1 2 3 4)))
  (list a b c))            \ev (1 2 (3 4))

(let ((a 'a) (b 'b) (x 'x) (y 'y))
  (let-values (((a b) (values x y))
               ((x y) (values a b)))
    (list a b x y)))       \ev (x y a b)
\end{scheme}

The following definition of {\cf let-values}
using {\cf syntax-rules} (see section~\ref{syntaxrulessection})
employs a pair of helpers to
create temporary names for the formals.

\begin{scheme}
(define-syntax let-values
  (syntax-rules ()
    ((let-values (binding ...) body1 body2 ...)
     (let-values-helper1
       ()
       (binding ...)
       body1 body2 ...))))

(define-syntax let-values-helper1
  ;; map over the bindings
  (syntax-rules ()
    ((let-values
       ((id temp) ...)
       ()
       body1 body2 ...)
     (let ((id temp) ...) body1 body2 ...))
    ((let-values
       assocs
       ((formals1 expr1) (formals2 expr2) ...)
       body1 body2 ...)
     (let-values-helper2
       formals1
       ()
       expr1
       assocs
       ((formals2 expr2) ...)
       body1 body2 ...))))

(define-syntax let-values-helper2
  ;; create temporaries for the formals
  (syntax-rules ()
    ((let-values-helper2
       ()
       temp-formals
       expr1
       assocs
       bindings
       body1 body2 ...)
     (call-with-values
       (lambda () expr1)
       (lambda temp-formals
         (let-values-helper1
           assocs
           bindings
           body1 body2 ...))))
    ((let-values-helper2
       (first . rest)
       (temp ...)
       expr1
       (assoc ...)
       bindings
       body1 body2 ...)
     (let-values-helper2
       rest
       (temp ... newtemp)
       expr1
       (assoc ... (first newtemp))
       bindings
       body1 body2 ...))
    ((let-values-helper2
       rest-formal
       (temp ...)
       expr1
       (assoc ...)
       bindings
       body1 body2 ...)
     (call-with-values
       (lambda () expr1)
       (lambda (temp ... . newtemp)
         (let-values-helper1
           (assoc ... (rest-formal newtemp))
           bindings
           body1 body2 ...))))))
\end{scheme}

\end{entry}

\begin{entry}{%
\proto{let*-values}{ \hyper{mv-bindings} \hyper{body}}{\exprtype}}

{\cf Let*-values} is similar to {\cf let-values}, but the bindings are
processed sequentially from left to right, and the
region\index{region} of the bindings indicated by {\cf(\hyper{formals}
  \hyper{init})} is that part of the {\cf let*-values} expression to
the right of the bindings.  Thus, the second set of bindings is evaluated in
an environment in which the first set of bindings is visible, and so
on.

\begin{scheme}
(let ((a 'a) (b 'b) (x 'x) (y 'y))
  (let*-values (((a b) (values x y))
                ((x y) (values a b)))
    (list a b x y)))  \ev (x y x y)
\end{scheme}

The following macro defines {\cf let*-values} in terms of {\cf let}
and {\cf let-values}:

\begin{scheme}
(define-syntax let*-values
  (syntax-rules ()
    ((let*-values () body1 body2 ...)
     (let () body1 body2 ...))
    ((let*-values (binding1 binding2 ...)
       body1 body2 ...)
     (let-values (binding1)
       (let*-values (binding2 ...)
         body1 body2 ...)))))
\end{scheme}

\end{entry}

\subsection{Sequencing}\unsection

\begin{entry}{%
\proto{begin}{ \hyperi{expression} \hyperii{expression} \dotsfoo}{\exprtype}}

The \hyper{expression}s are evaluated sequentially from left to right,
and the value(s) of the last \hyper{expression} is(are) returned.  This
expression type is used to sequence side effects such as input and
output.

\begin{scheme}
(define x 0)

(begin (set! x 5)
       (+ x 1))                  \ev  6

(begin (display "4 plus 1 equals ")
       (display (+ 4 1)))      \ev  \unspecified
 \>{\em and prints}  4 plus 1 equals 5%
\end{scheme}

The following macro, which uses {\cf syntax-rules} (see
section~\ref{syntaxrulessection}), defines {\cf begin} in terms of {\cf
  lambda}:
%
\begin{scheme}
(define-syntax \ide{begin}
  (syntax-rules ()
    ((begin exp ...)
     ((lambda () exp ...)))))
\end{scheme}

The following alternative expansion for {\cf begin} does not make use of
the ability to write more than one expression in the body of a lambda
expression.  In any case, note that these rules apply only if the body
of the {\cf begin} contains no definitions.

\begin{scheme}
(define-syntax begin
  (syntax-rules ()
    ((begin exp)
     exp)
    ((begin exp1 exp2 ...)
     (call-with-values
         (lambda () exp1)
       (lambda ignored
         (begin exp2 ...))))))
\end{scheme}

\end{entry}


\subsection{Iteration}%\unsection

\noindent%
\pproto{(do ((\hyperi{variable} \hyperi{init} \hyperi{step})}{\exprtype}
\mainschindex{do}{\tt\obeyspaces%
     \dotsfoo)\\
    (\hyper{test} \hyper{expression} \dotsfoo)\\
  \hyper{command} \dotsfoo)}

{\cf Do} is an iteration construct.  It specifies a set of variables to
be bound, how they are to be initialized at the start, and how they are
to be updated on each iteration.  When a termination condition is met,
the loop exits after evaluating the \hyper{expression}s.

{\cf Do} expressions are evaluated as follows:
The \hyper{init} expressions are evaluated (in some unspecified order),
the \hyper{variable}s are bound to fresh locations, the results of the
\hyper{init} expressions are stored in the bindings of the
\hyper{variable}s, and then the iteration phase begins.

\vest Each iteration begins by evaluating \hyper{test}; if the result is
false (see section~\ref{booleansection}), then the \hyper{command}
expressions are evaluated in order for effect, the \hyper{step}
expressions are evaluated in some unspecified order, the
\hyper{variable}s are bound to fresh locations, the results of the
\hyper{step}s are stored in the bindings of the
\hyper{variable}s, and the next iteration begins.

\vest If \hyper{test} evaluates to a true value, then the
\hyper{expression}s are evaluated from left to right and the value(s) of
the last \hyper{expression} is(are) returned.  If no \hyper{expression}s
are present, then the value of the {\cf do} expression is the
unspecified value.

\vest The region\index{region} of the binding of a \hyper{variable}
consists of the entire {\cf do} expression except for the \hyper{init}s.
It is syntax defect for a \hyper{variable} to appear more than once in the
list of {\cf do} variables.

\vest A \hyper{step} may be omitted, in which case the effect is the
same as if {\cf(\hyper{variable} \hyper{init} \hyper{variable})} had
been written instead of {\cf(\hyper{variable} \hyper{init})}.

\begin{scheme}
(do ((vec (make-vector 5))
     (i 0 (+ i 1)))
    ((= i 5) vec)
  (vector-set! vec i i))          \ev  \#(0 1 2 3 4)

(let ((x '(1 3 5 7 9)))
  (do ((x x (cdr x))
       (sum 0 (+ sum (car x))))
      ((null? x) sum)))             \ev  25%
\end{scheme}

The following definition
of {\cf do} uses a trick to expand the variable clauses.
As with {\cf letrec} above, an auxiliary macro would also work.

\begin{scheme}
(define-syntax \ide{do}
  (syntax-rules ()
    ((do ((var init step ...) ...)
         (test expr ...)
         command ...)
     (letrec
       ((loop
         (lambda (var ...)
           (if test
               (begin
                 (unspecified)
                 expr ...)
               (begin
                 command
                 ...
                 (loop (do "step" var step ...)
                       ...))))))
       (loop init ...)))
    ((do "step" x)
     x)
    ((do "step" x y)
     y)))
\end{scheme}

%\end{entry}


\begin{entry}{%
\rproto{let}{ \hyper{variable} \hyper{bindings} \hyper{body}}{\exprtype}}

\label{namedlet}
``Named {\cf let}'' is a variant on the syntax of \ide{let} which provides
a more general looping construct than {\cf do} and may also be used to express
recursions.
It has the same syntax and semantics as ordinary {\cf let}
except that \hyper{variable} is bound within \hyper{body} to a procedure
whose formal arguments are the bound variables and whose body is
\hyper{body}.  Thus the execution of \hyper{body} may be repeated by
invoking the procedure named by \hyper{variable}.

%                                              |  <-- right margin
\begin{scheme}
(let loop ((numbers '(3 -2 1 6 -5))
           (nonneg '())
           (neg '()))
  (cond ((null? numbers) (list nonneg neg))
        ((>= (car numbers) 0)
         (loop (cdr numbers)
               (cons (car numbers) nonneg)
               neg))
        ((< (car numbers) 0)
         (loop (cdr numbers)
               nonneg
               (cons (car numbers) neg))))) %
  \lev  ((6 1 3) (-5 -2))%
\end{scheme}

{\cf Let} could be defined in terms of {\cf lambda} and {\cf letrec}
using {\cf syntax-rules} (see section~\ref{syntaxrulessection}) as
follows:

\begin{scheme}
(define-syntax \ide{let}
  (syntax-rules ()
    ((let ((name val) ...) body1 body2 ...)
     ((lambda (name ...) body1 body2 ...)
      val ...))
    ((let tag ((name val) ...) body1 body2 ...)
     ((letrec ((tag (lambda (name ...)
                      body1 body2 ...)))
        tag)
      val ...))))
\end{scheme}

\end{entry}

\subsection{Quasiquotation}\unsection
\label{quasiquotesection}

\begin{entry}{%
\proto{quasiquote}{ \hyper{qq template}}{\exprtype}}

``Backquote'' or ``quasiquote''\index{backquote} expressions are useful
for constructing a list or vector structure when some but not all of the
desired structure is known in advance.  If no
{\cf unquote} or {\cf unquote-splicing} forms
appear within the \hyper{qq template}, the result of
evaluating
{\cf (quasiquote \hyper{qq template})} is equivalent to the result of evaluating
{\cf (quote \hyper{qq template})}.

If an {\cf (unquote \hyper{expression} \dotsfoo)} form appears inside a
\hyper{qq template}, however, the \hyper{expression}s are evaluated
(``unquoted'') and their results are inserted into the structure instead
of the {\cf unquote} form.

If an {\cf (unquote-splicing \hyper{expression} \dotsfoo)} form
appears inside a \hyper{qq template}, then the \hyper{expression}s must
evaluate to lists; the opening and closing parentheses of the list are
then ``stripped away'' and the elements of the lists are inserted in
place of the {\cf unquote-splicing} form.  {\cf
Unquote-splicing} and multi-operand {\cf unquote} forms should
appear only within a list or vector \hyper{qq template}.

As noted in section~\ref{quotesection},
{\cf (quasiquote \hyper{qq template})} may be abbreviated
\backquote\hyper{qq template},
{\cf (unquote \hyper{expression})} may be abbreviated
{\cf,}\hyper{expression}, and
{\cf (unquote-splicing \hyper{expression})} may be abbreviated
{\cf,}\atsign\hyper{expression}.

% struck: "(in the sense of {\cf equal?})" after "equivalent"

\begin{scheme}
`(list ,(+ 1 2) 4)  \ev  (list 3 4)
(let ((name 'a)) `(list ,name ',name)) %
          \lev  (list a (quote a))
`(a ,(+ 1 2) ,@(map abs '(4 -5 6)) b) %
          \lev  (a 3 4 5 6 b)
`(({\cf foo} ,(- 10 3)) ,@(cdr '(c)) . ,(car '(cons))) %
          \lev  ((foo 7) . cons)
`\#(10 5 ,(sqrt 4) ,@(map sqrt '(16 9)) 8) %
          \lev  \#(10 5 2 4 3 8)
(let ((name 'foo))
  `((unquote name name name)))%
          \lev (foo foo foo)
(let ((name '(foo)))
  `((unquote-splicing name name name)))%
          \lev (foo foo foo)
\end{scheme}

Quasiquote forms may be nested.  Substitutions are made only for
unquoted components appearing at the same nesting level
as the outermost {\cf quasiquote}.  The nesting level increases by one inside
each successive quasiquotation, and decreases by one inside each
unquotation.

\begin{scheme}
`(a `(b ,(+ 1 2) ,(foo ,(+ 1 3) d) e) f) %
          \lev  (a `(b ,(+ 1 2) ,(foo 4 d) e) f)
(let ((name1 'x)
      (name2 'y))
  `(a `(b ,,name1 ,',name2 d) e)) %
          \lev  (a `(b ,x ,'y d) e)%
\end{scheme}

FIXME: The external syntax generated by \ide{write} for two-element lists whose
car is one of these symbols may vary between implementations.

\begin{scheme}
(quasiquote (list (unquote (+ 1 2)) 4)) %
          \lev  (list 3 4)
'(quasiquote (list (unquote (+ 1 2)) 4)) %
          \lev  `(list ,(+ 1 2) 4)
     {\em{}i.e.,} (quasiquote (list (unquote (+ 1 2)) 4))%
\end{scheme}

Unpredictable behavior can result if any of the identifiers
\ide{quasiquote}, \ide{unquote}, or \ide{unquote-splicing} appear in
positions within a \hyper{qq template} otherwise than as described above.

The following grammar for quasiquote expressions is not context-free.
It is presented as a recipe for generating an infinite number of
production rules.  Imagine a copy of the following rules for $D = 1, 2,
3, \ldots$.  $D$ keeps track of the nesting depth.

\begin{grammar}%
\meta{quasiquotation} \: \meta{quasiquotation 1}
\meta{qq template 0} \: \meta{expression}
\meta{quasiquotation $D$} \: `\meta{qq template $D$}
\>    \| (quasiquote \meta{qq template $D$})
\meta{qq template $D$} \: \meta{simple datum}
\>    \| \meta{list qq template $D$}
\>    \| \meta{vector qq template $D$}
\>    \| \meta{unquotation $D$}
\meta{list qq template $D$} \: (\arbno{\meta{qq template or splice $D$}})
\>    \| (\atleastone{\meta{qq template or splice $D$}} .\ \meta{qq template $D$})
\>    \| '\meta{qq template $D$}
\>    \| \meta{quasiquotation $D+1$}
\meta{vector qq template $D$} \: \#(\arbno{\meta{qq template or splice $D$}})
\meta{unquotation $D$} \: ,\meta{qq template $D-1$}
\>    \| (unquote \meta{qq template $D-1$})
\meta{qq template or splice $D$} \: \meta{qq template $D$}
\>    \| \meta{splicing unquotation $D$}
\meta{splicing unquotation $D$} \: ,@\meta{qq template $D-1$}
\>    \| (unquote-splicing \arbno{\meta{qq template $D-1$}})
\>    \| (unquote \arbno{\meta{qq template $D-1$}}) %
\end{grammar}

In \meta{quasiquotation}s, a \meta{list qq template $D$} can sometimes
be confused with either an \meta{un\-quota\-tion $D$} or a \meta{splicing
un\-quo\-ta\-tion $D$}.  The interpretation as an
\meta{un\-quo\-ta\-tion} or \meta{splicing
un\-quo\-ta\-tion $D$} takes precedence.

\end{entry}

\section{Binding constructs for syntactic keywords}
\label{bindsyntax}

{\cf Let-syntax} and {\cf letrec-syntax} are
analogous to {\cf let} and {\cf letrec}, but they bind
syntactic keywords to macro transformers instead of binding variables
to locations that contain values.  Syntactic keywords may also be
bound at top level; see section~\ref{define-syntax}.

\begin{entry}{%
\proto{let-syntax}{ \hyper{bindings} \hyper{body}}{\exprtype}}

\syntax
\hyper{Bindings} should have the form
\begin{scheme}
((\hyper{keyword} \hyper{transformer spec}) \dotsfoo)%
\end{scheme}
Each \hyper{keyword} is an identifier,
each \hyper{transformer spec} is an instance of {\cf syntax-rules}, and
\hyper{body} should be a sequence of one or more expressions.  It is a
syntax defect for \hyper{keyword} to appear more than once in the list of keywords
being bound.

\semantics
The \hyper{body} is expanded in the syntactic environment
obtained by extending the syntactic environment of the
{\cf let-syntax} expression with macros whose keywords are
the \hyper{keyword}s, bound to the specified transformers.
Each binding of a \hyper{keyword} has \hyper{body} as its region.

\begin{scheme}
(let-syntax ((when (syntax-rules ()
                     ((when test stmt1 stmt2 ...)
                      (if test
                          (begin stmt1
                                 stmt2 ...))))))
  (let ((if \schtrue))
    (when if (set! if 'now))
    if))                           \ev  now

(let ((x 'outer))
  (let-syntax ((m (syntax-rules () ((m) x))))
    (let ((x 'inner))
      (m))))                       \ev  outer%
\end{scheme}

\end{entry}

\begin{entry}{%
\proto{letrec-syntax}{ \hyper{bindings} \hyper{body}}{\exprtype}}

\syntax
Same as for {\cf let-syntax}.

\semantics
 The \hyper{body} is expanded in the syntactic environment obtained by
extending the syntactic environment of the {\cf letrec-syntax}
expression with macros whose keywords are the
\hyper{keyword}s, bound to the specified transformers.
Each binding of a \hyper{keyword} has the \hyper{bindings}
as well as the \hyper{body} within its region,
so the transformers can
transcribe expressions into uses of the macros
introduced by the {\cf letrec-syntax} expression.

\begin{scheme}
(letrec-syntax
  ((my-or (syntax-rules ()
            ((my-or) \schfalse)
            ((my-or e) e)
            ((my-or e1 e2 ...)
             (let ((temp e1))
               (if temp
                   temp
                   (my-or e2 ...)))))))
  (let ((x \schfalse)
        (y 7)
        (temp 8)
        (let odd?)
        (if even?))
    (my-or x
           (let temp)
           (if y)
           y)))        \ev  7%
\end{scheme}

\end{entry}

\section{Equivalence predicates}
\label{equivalencesection}

A \defining{predicate} is a procedure that always returns a boolean
value (\schtrue{} or \schfalse).  An \defining{equivalence predicate} is
the computational analogue of a mathematical equivalence relation (it is
symmetric, reflexive, and transitive).  Of the equivalence predicates
described in this section, {\cf eq?}\ is the finest or most
discriminating, and {\cf equal?}\ is the coarsest.  {\cf Eqv?}\ is
slightly less discriminating than {\cf eq?}.  \todo{Pitman doesn't like
this paragraph.  Lift the discussion from the Maclisp manual.  Explain
why there's more than one predicate.}


\begin{entry}{%
\proto{eqv?}{ \vari{obj} \varii{obj}}{procedure}}

The {\cf eqv?} procedure defines a useful equivalence relation on objects.
Briefly, it returns \schtrue{} if \vari{obj} and \varii{obj} should
normally be regarded as the same object.  This relation is left slightly
open to interpretation, but the following partial specification of
{\cf eqv?} holds for all implementations of Scheme.

The {\cf eqv?} procedure returns \schtrue{} if:

\begin{itemize}
\item \vari{obj} and \varii{obj} are both \schtrue{} or both \schfalse.

\item \vari{obj} and \varii{obj} are both symbols and

\begin{scheme}
(string=? (symbol->string obj1)
          (symbol->string obj2))
    \ev  \schtrue%
\end{scheme}

\begin{note} 
This assumes that neither \vari{obj} nor \varii{obj} is an ``uninterned
symbol'' as alluded to in section~\ref{symbolsection}.  This report does
not presume to specify the behavior of {\cf eqv?} on implementation-dependent
extensions.
\end{note}

\item \var{obj1} and \var{obj2} are both exact\index{exact} numbers,
  and are numerically equal (see {\cf =}, section see
  section~\ref{genericarithmetic}),

\item \var{obj1} and \var{obj2} are both inexact\index{inexact} numbers, are numerically
  equal (see {\cf =}, section see section~\ref{genericarithmetic}, and
  yield the same results (in the sense of {\cf eqv?}) when passed
  as arguments to any other procedure that can be defined
  as a finite composition of Scheme's standard arithmetic
  procedures.

\item \vari{obj} and \varii{obj} are both characters and are the same
character according to the {\cf char=?} procedure
(section~\ref{charactersection}).

\item both \vari{obj} and \varii{obj} are the empty list.

\item \vari{obj} and \varii{obj} are pairs, vectors, or strings that denote the
same locations in the store (section~\ref{storagemodel}).

\item \vari{obj} and \varii{obj} are procedures whose location tags are
equal (section~\ref{lambda}).
\end{itemize}

The {\cf eqv?} procedure returns \schfalse{} if:

\begin{itemize}
\item \vari{obj} and \varii{obj} are of different types
(section~\ref{disjointness}).

\item one of \vari{obj} and \varii{obj} is \schtrue{} but the other is
\schfalse{}.

\item \vari{obj} and \varii{obj} are symbols but

\begin{scheme}
(string=? (symbol->string \vari{obj})
          (symbol->string \varii{obj}))
    \ev  \schfalse%
\end{scheme}

\item one of \var{obj1} and \var{obj2} is an exact number but the other is
        an inexact number, or 

\item \var{obj1} and \var{obj2} are rational numbers for which the {\cf =} procedure
  returns \schfalse{}, or 

\item \var{obj1} and \var{obj2} yield different results (in the sense of
  {\cf eqv?}) when passed as arguments to any other procedure
  that can be defined as a finite composition of Scheme's
  standard arithmetic procedures.

\item \vari{obj} and \varii{obj} are characters for which the {\cf char=?}
  procedure returns \schfalse{}.

\item one of \vari{obj} and \varii{obj} is the empty list but the other
is not.

\item \vari{obj} and \varii{obj} are pairs, vectors, or strings that denote
distinct locations.

\item \vari{obj} and \varii{obj} are procedures that would behave differently
(return different value(s) or have different side effects) for some arguments.

\end{itemize}

\begin{scheme}
(eqv? 'a 'a)                     \ev  \schtrue
(eqv? 'a 'b)                     \ev  \schfalse
(eqv? 2 2)                       \ev  \schtrue
(eqv? '() '())                   \ev  \schtrue
(eqv? 100000000 100000000)       \ev  \schtrue
(eqv? (cons 1 2) (cons 1 2))     \ev  \schfalse
(eqv? (lambda () 1)
      (lambda () 2))             \ev  \schfalse
(eqv? \#f 'nil)                  \ev  \schfalse
(let ((p (lambda (x) x)))
  (eqv? p p))                    \ev  \schtrue%
\end{scheme}

The following examples illustrate cases in which the above rules do
not fully specify the behavior of {\cf eqv?}.  All that can be said
about such cases is that the value returned by {\cf eqv?} must be a
boolean.

\begin{scheme}
(eqv? "" "")             \ev  \unspecified
(eqv? '\#() '\#())         \ev  \unspecified
(eqv? (lambda (x) x)
      (lambda (x) x))    \ev  \unspecified
(eqv? (lambda (x) x)
      (lambda (y) y))    \ev  \unspecified%
\end{scheme}

The next set of examples shows the use of {\cf eqv?}\ with procedures
that have local state.  {\cf Gen-counter} must return a distinct
procedure every time, since each procedure has its own internal counter.
{\cf Gen-loser}, however, returns equivalent procedures each time, since
the local state does not affect the value or side effects of the
procedures.

\begin{scheme}
(define gen-counter
  (lambda ()
    (let ((n 0))
      (lambda () (set! n (+ n 1)) n))))
(let ((g (gen-counter)))
  (eqv? g g))           \ev  \schtrue
(eqv? (gen-counter) (gen-counter))
                        \ev  \schfalse
(define gen-loser
  (lambda ()
    (let ((n 0))
      (lambda () (set! n (+ n 1)) 27))))
(let ((g (gen-loser)))
  (eqv? g g))           \ev  \schtrue
(eqv? (gen-loser) (gen-loser))
                        \ev  \unspecified

(letrec ((f (lambda () (if (eqv? f g) 'both 'f)))
         (g (lambda () (if (eqv? f g) 'both 'g))))
  (eqv? f g))
                        \ev  \unspecified

(letrec ((f (lambda () (if (eqv? f g) 'f 'both)))
         (g (lambda () (if (eqv? f g) 'g 'both))))
  (eqv? f g))
                        \ev  \schfalse%
\end{scheme}

% Objects of distinct types must never be regarded as the same object,
% except that \schfalse{} and the empty list\index{empty list} are permitted to
% be identical.
%
% \begin{scheme}
% (eqv? '() \schfalse)    \ev  \unspecified%
% \end{scheme}

Since it is the effect of trying to modify constant objects (those returned by
literal expressions) is unspecified, implementations are permitted, though not
required, to share structure between constants where appropriate.  Thus
the value of {\cf eqv?} on constants is sometimes
implementation-dependent.

\begin{scheme}
(eqv? '(a) '(a))                 \ev  \unspecified
(eqv? "a" "a")                   \ev  \unspecified
(eqv? '(b) (cdr '(a b)))         \ev  \unspecified
(let ((x '(a)))
  (eqv? x x))                    \ev  \schtrue%
\end{scheme}

\begin{rationale} 
The above definition of {\cf eqv?} allows implementations latitude in
their treatment of procedures and literals:  implementations are free
either to detect or to fail to detect that two procedures or two literals
are equivalent to each other, and can decide whether or not to
merge representations of equivalent objects by using the same pointer or
bit pattern to represent both.
\end{rationale}

\end{entry}


\begin{entry}{%
\proto{eq?}{ \vari{obj} \varii{obj}}{procedure}}

{\cf Eq?}\ is similar to {\cf eqv?}\ except that in some cases it is
capable of discerning distinctions finer than those detectable by
{\cf eqv?}.

\vest {\cf Eq?}\ and {\cf eqv?}\ are guaranteed to have the same
behavior on symbols, booleans, the empty list, pairs, procedures,
and non-empty
strings and vectors.  {\cf Eq?}'s behavior on numbers and characters is
implementation-dependent, but it will always return either true or
false, and will return true only when {\cf eqv?}\ would also return
true.  {\cf Eq?} may also behave differently from {\cf eqv?} on empty
vectors and empty strings.

\begin{scheme}
(eq? 'a 'a)                     \ev  \schtrue
(eq? '(a) '(a))                 \ev  \unspecified
(eq? (list 'a) (list 'a))       \ev  \schfalse
(eq? "a" "a")                   \ev  \unspecified
(eq? "" "")                     \ev  \unspecified
(eq? '() '())                   \ev  \schtrue
(eq? 2 2)                       \ev  \unspecified
(eq? \#\backwhack{}A \#\backwhack{}A) \ev  \unspecified
(eq? car car)                   \ev  \schtrue
(let ((n (+ 2 3)))
  (eq? n n))      \ev  \unspecified
(let ((x '(a)))
  (eq? x x))      \ev  \schtrue
(let ((x '\#()))
  (eq? x x))      \ev  \schtrue
(let ((p (lambda (x) x)))
  (eq? p p))      \ev  \schtrue%
\end{scheme}

\todo{Needs to be explained better above.  How can this be made to be
not confusing?  A table maybe?}

\begin{rationale} It will usually be possible to implement {\cf eq?}\ much
more efficiently than {\cf eqv?}, for example, as a simple pointer
comparison instead of as some more complicated operation.  One reason is
that it may not be possible to compute {\cf eqv?}\ of two numbers in
constant time, whereas {\cf eq?}\ implemented as pointer comparison will
always finish in constant time.  {\cf Eq?}\ may be used like {\cf eqv?}\
in applications using procedures to implement objects with state since
it obeys the same constraints as {\cf eqv?}.
\end{rationale}

\end{entry}


\begin{entry}{%
\proto{equal?}{ \vari{obj} \varii{obj}}{procedure}}

The {\cf equal?}  predicate returns \schtrue{} if and only if the
(possibly infinite) unfoldings of its arguments into regular trees are
equal as ordered trees.

{\cf Equal?} treats pairs and vectors as nodes with outgoing edges,
uses {\cf string=?} to compare strings, and uses {\cf eqv?} to compare
other nodes.

\begin{scheme}
(equal? 'a 'a)                  \ev  \schtrue
(equal? '(a) '(a))              \ev  \schtrue
(equal? '(a (b) c)
        '(a (b) c))             \ev  \schtrue
(equal? "abc" "abc")            \ev  \schtrue
(equal? 2 2)                    \ev  \schtrue
(equal? (make-vector 5 'a)
        (make-vector 5 'a))     \ev  \schtrue
(equal? (lambda (x) x)
        (lambda (y) y))  \ev  \unspecified%


(let* ((x (list 'a))
       (y (list 'a))
       (z (list x y)))
  (list (equal? z (list y x))
        (equal? z (list x x))))             \lev  (\schtrue{} \schtrue{})

(let ((x (list 'a 'b 'c 'a))
      (y (list 'a 'b 'c 'a 'b 'c 'a)))
  (set-cdr! (list-tail x 2) x)
  (set-cdr! (list-tail y 5) y)
  (list
   (equal? x x)
   (equal? x y)
   (equal? (list x y 'a) (list y x 'b)))) \lev  (\schtrue{} \schtrue{} \schfalse{})
\end{scheme}

\end{entry}

\section{Unspecified value}
\label{unspecifiedvalue}

\begin{entry}{%
\proto{unspecified}{}{procedure}}

The {\cf unspecified} procedure returns the unspecified
value.\index{unspecified value}  (See section \ref{disjointness}.)
\end{entry}

\todo{{\tt Procedure?} doesn't belong in a section with the name
``control features.''  What to do?}

\section{Procedure predicate}

\begin{entry}{%
\proto{procedure?}{ obj}{procedure}}

Returns \schtrue{} if \var{obj} is a procedure, otherwise returns \schfalse.

\begin{scheme}
(procedure? car)            \ev  \schtrue
(procedure? 'car)           \ev  \schfalse
(procedure? (lambda (x) (* x x)))   
                            \ev  \schtrue
(procedure? '(lambda (x) (* x x)))  
                            \ev  \schfalse
\end{scheme}

\end{entry}

\section{Generic arithmetic}
\label{genericarithmetic}

The procedures described here implement generic arithmetic based on
the numerical tower described in chapter~\ref{numberchapter}.  They
are exported by the base library, but also through the \defining{r6rs
  arithmetic generic} library.  Unlike
the libraries in chapter~\ref{numberchapter}, the procedures generic
arithmetic accept both exact and inexact numbers as arguments, and
perform coercions and the appropriate operations as necessary.  Refer
to chapter~\ref{numberchapter} for a detailed description of the
numerical types of Scheme, a discussion of the concept of exactness,
the definition of some of the numerical operations used here, and
libraries defining more numerical operations.

\subsection{Propagation of exactness}

The procedures listed below will always return an exact integer result
provided all their arguments are exact integers and the mathematically
expected result is representable as an exact integer within the
implementation:

FIXME: formatting; complete?

\begin{scheme}
+            -             *
div          mod           div0
mod0 
max          min           abs
numerator    denominator   gcd
lcm          floor         ceiling
truncate     round         rationalize
expt%
\end{scheme}

The generic operations generally return the correct exact
result when all of their arguments are exact and the result is
mathematically well-defined, but return an inexact result when any
argument is inexact.  Exceptions are {\cf sqrt}, {\cf exp}, {\cf log},
{\cf sin}, {\cf cos}, {\cf tan}, {\cf asin}, {\cf acos}, {\cf atan},
{\cf expt}, {\cf make-polar}, {\cf magnitude}, and {\cf angle}, which
are allowed (but not required) to return inexact results even when
given exact arguments, as indicated in the specification of these
procedures.

One general exception to the rule above is that an implementation may
return an exact result despite inexact arguments if that exact result
would be the correct result for all possible substitutions of exact
arguments for the inexact ones.

Rational operations such as {\cf +} should always produce exact
results when given exact arguments. 

\subsection{Numerical operations}

The procedures described here behave consistently with the
corresponding {\cf inexact-} procedure if passed inexact arguments,
and with the corresponding {\cf exact-} procedure if passed exact
arguments.

We will use \var{z}, \vari{z}, \varii{z}, and \variii{z} as
metavariables that range over the complex numbers, \var{x}, \vari{x},
\varii{x}, and \variii{x} as metavariables that range over the real
numbers, \var{q}, \vari{q}, \varii{q}, and \variii{q} as metavariables
that range over the rationals. and \var{n}, \vari{n}, \varii{n}, and
\variii{n} as metavariables that range over the inexact integers.

\subsubsection{Numerical type predicates}

\begin{entry}{%
\proto{number?}{ obj}{procedure}
\proto{complex?}{ obj}{procedure}
\proto{real?}{ obj}{procedure}
\proto{rational?}{ obj}{procedure}
\proto{integer?}{ obj}{procedure}}

These numerical type predicates can be applied to any kind ofargument, including non-numbers.  They return \schtrue{} if the object is
of the named type, and otherwise they return \schfalse{}.
In general, if a type predicate is true of a number then all higher
type predicates are also true of that number.  Consequently, if a type
predicate is false of a number, then all lower type predicates are
also false of that number.

If \var{z} is a complex number, then {\cf (real? \var{z})} is true if
and only if {\cf (zero? (imag-part \var{z}))} and {\cf (exact?
  (imag-part \var{z}))} are both true.

If \var{x} is a real number, then {\cf (rational? \var{x})} is true if
and only if there exist exact integers \var{k1} and \var{k2} such that
{\cf (= \var{x} (/ \var{k1} \var{k2}))} and {\cf (= (numerator
  \var{x}) \var{k1})} and {\cf (= (denominator \var{x}) \var{k2})} are
all true.  Thus infinities and NaNs are not rational numbers.

If \var{q} is a rational number, then {\cf (integer?
\var{q})} is true if and only if {\cf`(= (denominator
\var{q}) 1)} is true.  If \var{q} is not a rational number,
then {\cf (integer? \var{q})} is false.

\begin{scheme}
(complex? 3+4i)                        \ev  \schtrue{}
(complex? 3)                           \ev  \schtrue{}
(real? 3)                              \ev  \schtrue{}
(real? -2.5+0.0i)                      \ev  \schfalse{}
(real? -2.5+0i)                        \ev  \schtrue{}
(real? -2.5)                           \ev  \schtrue{}
(real? \sharpsign{}e1e10)                         \ev  \schtrue{}
(rational? 6/10)                       \ev  \schtrue{}
(rational? 6/3)                        \ev  \schtrue{}
(rational? 2)                          \ev  \schtrue{}
(integer? 3+0i)                        \ev  \schtrue{}
(integer? 3.0)                         \ev  \schtrue{}
(integer? 8/4)                         \ev  \schtrue{}

(number? +nan.0)                       \ev  \schtrue{}
(complex? +nan.0)                      \ev  \schtrue{}
(real? +nan.0)                         \ev  \schtrue{}
(rational? +nan.0)                     \ev  \schfalse{}
(complex? +inf.0)                      \ev  \schtrue{}
(real? -inf.0)                         \ev  \schtrue{}
(rational? -inf.0)                     \ev  \schfalse{}
(integer? -inf.0)                      \ev  \schfalse{}
\end{scheme}

\begin{note}
The behavior of these type predicates on inexact numbers is
unreliable, because any inaccuracy may
affect the result.
\end{note}
\end{entry}

\begin{entry}{%
\proto{real-valued?}{ obj}{procedure}
\proto{rational-valued?}{ obj}{procedure}
\proto{integer-valued?}{ obj}{procedure}}

These numerical type predicates can be applied to any kind of
argument, including non-numbers.  They return \schtrue{} if the object
is a number and is equal in the sense of {\cf =} to some object of the
named type, and otherwise they return \schfalse{}.

\begin{scheme}
(real-valued? +nan.0)                  \ev  \schfalse{}
(real-valued? -inf.0)                  \ev  \schtrue{}
(real-valued? 3)                       \ev  \schtrue{}
(real-valued? -2.5+0.0i)               \ev  \schtrue{}
(real-valued? -2.5+0i)                 \ev  \schtrue{}
(real-valued? -2.5)                    \ev  \schtrue{}
(real-valued? \sharpsign{}e1e10)                  \ev  \schtrue{}

(rational-valued? +nan.0)              \ev  \schfalse{}
(rational-valued? -inf.0)              \ev  \schfalse{}
(rational-valued? 6/10)                \ev  \schtrue{}
(rational-valued? 6/10+0.0i)           \ev  \schtrue{}
(rational-valued? 6/10+0i)             \ev  \schtrue{}
(rational-valued? 6/3)                 \ev  \schtrue{}

(integer-valued? 3+0i)                 \ev  \schtrue{}
(integer-valued? 3+0.0i)               \ev  \schtrue{}
(integer-valued? 3.0)                  \ev  \schtrue{}
(integer-valued? 3.0+0.0i)             \ev  \schtrue{}
(integer-valued? 8/4)                  \ev  \schtrue{}
\end{scheme}

\begin{note}
The behavior of these type predicates on inexact numbers is
unreliable, because any inaccuracy may
affect the result.
\end{note}
\end{entry}

\begin{entry}{%
\proto{exact?}{ z}{procedure}
\proto{inexact?}{ z}{procedure}}

These numerical predicates provide tests for the exactness of a
quantity.  For any Scheme number, precisely one of these predicates is
true.

\begin{scheme}
(exact? 5)                   \ev  \schtrue{}
(inexact? +inf.0)            \ev  \schtrue{}
\end{scheme}
\end{entry}

\subsubsection{Generic conversions}

\begin{entry}{%
\proto{->inexact}{ z}{procedure}
\proto{->exact}{ z}{procedure}}

{\cf ->inexact} returns an inexact representation of \var{z}.  If
inexact numbers of the appropriate type have bounded precision, then
the value returned is an inexact number that is nearest to the
argument.  If an exact argument has no reasonably close inexact
equivalent, an exception with condition type
{\cf\&implementation-violation} may be
raised.

{\cf ->exact} returns an exact representation of \var{z}.  The value
returned is the exact number that is numerically closest to the
argument; in most cases, the result of this procedure should be
numerically equal to its argument.  If an inexact argument has no
reasonably close exact equivalent, an exception with condition type
{\cf\&implementation-violation} may be
raised.

These procedures implement the natural one-to-one correspondence
between exact and inexact integers throughout an
implementation-dependent range.

{\cf ->inexact} and {\cf ->exact} are idempotent.
\end{entry}

\begin{entry}{%
\proto{real->flonum}{ x}{procedure}}

{\cf Real->flonum} returns a flonum representation of
\var{x}, which must be a real number.

The value returned is a flonum that is numerically closest to the
argument.

\begin{rationale}
  The flonums are a subset of the inexact reals, but may be a proper
  subset.  The {\cf real->flonum} procedure converts an arbitrary real
  to the flonum type required by flonum-specific procedures.
\end{rationale}

\begin{note}
  If flonums are represented in binary floating point, then
  implementations are strongly encouraged to break ties by preferring
  the floating point representation whose least significant bit is
  zero.
\end{note}
\end{entry}

\begin{entry}{%
\proto{real->single}{ x}{procedure}
\proto{real->double}{ x}{procedure}}

Given a real number \var{x}, these procedures compute the best
IEEE-754 single or double precision approximation to \var{x} and
return that approximation as an inexact real.

\begin{note}:
  Both of the two conversions performed by these procedures (to
  IEEE-754 single or double, and then to an inexact real) may lose
  precision, introduce error, or may underflow or overflow.
\end{note}

\begin{rationale}
  The ability to round to IEEE-754 single or double precision is
  occasionally needed for control of precision or for
  interoperability.
\end{rationale}

\end{entry}
\subsubsection{Arithmetic operations}

\begin{entry}{%
\proto{=}{ \vari{z} \varii{z} \variii{z} \dotsfoo}{procedure}
\proto{<}{ \var{x1} \varii{x} \variii{x} \dotsfoo}{procedure}
\proto{>}{ \var{x1} \varii{x} \variii{x} \dotsfoo}{procedure}
\proto{<=}{ \var{x1} \varii{x} \variii{x} \dotsfoo}{procedure}
\proto{>=}{ \var{x1} \varii{x} \variii{x} \dotsfoo}{procedure}}

These procedures return \schtrue{} if their arguments are
(respectively): equal, monotonically increasing, monotonically
decreasing, monotonically nondecreasing, or monotonically
nonincreasing \schfalse{} otherwise.

\begin{scheme}
(= +inf.0 +inf.0)           \ev  \schtrue{}
(= -inf.0 +inf.0)           \ev  \schfalse{}
(= -inf.0 -inf.0)           \ev  \schtrue{}
\end{scheme}

For any real number \var{x} that is neither infinite nor NaN:

\begin{scheme}
(< -inf.0 \var{x} +inf.0))        \ev  \schtrue{}
(> +inf.0 \var{x} -inf.0))        \ev  \schtrue{}
\end{scheme}

These predicates are required to be transitive.

\begin{note}
The traditional implementations of these predicates in Lisp-like
languages are not transitive.
\end{note}

\begin{note}
While it is possible to compare inexact numbers using these
predicates, the results may be unreliable because a small inaccuracy
may affect the result; this is especially true of {\cf =} and {\cf zero?}.

When in doubt, consult a numerical analyst.
\end{note}
\end{entry}

\begin{entry}{%
\proto{zero?}{ z}{procedure}
\proto{positive?}{ x}{procedure}
\proto{negative?}{ x}{procedure}
\proto{odd?}{ n}{procedure}
\proto{even?}{ n}{procedure}
\proto{finite?}{ x}{procedure}
\proto{infinite?}{ x}{procedure}
\proto{nan?}{ x}{procedure}}

These numerical predicates test a number for a particular property,
returning \schtrue{} or \schfalse{}.  See note above.  {\cf Zero?}
tests if the number is {\cf =} to zero, {\cf positive?} tests if it is
greater than zero, {\cf negative?} tests if it is less than zero, {\cf
  odd?} tests if it is odd, {\cf even?} tests if it is even, {\cf
  finite?} tests if it is not an infinity and not a NaN, {\cf
  infinite?} tests if it is an infinity, {\cf nan?} tests if it is a
NaN.

\begin{scheme}
(positive? +inf.0)            \ev  \schtrue{}
(negative? -inf.0)            \ev  \schtrue{}
(finite? +inf.0)              \ev  \schfalse{}
(finite? 5)                   \ev  \schtrue{}
(finite? 5.0)                 \ev  \schtrue{}
(infinite? 5.0)               \ev  \schfalse{}
(infinite? +inf.0)            \ev  \schtrue{}
\end{scheme}
\end{entry}

\begin{entry}{%
\proto{max}{ \vari{x} \varii{x} \dotsfoo}{procedure}
\proto{min}{ \vari{x} \varii{x} \dotsfoo}{procedure}}

These procedures return the maximum or minimum of their arguments.

\begin{scheme}
(max 3 4)                              \ev  4    ; exact
(max 3.9 4)                            \ev  4.0  ; inexact
\end{scheme}

For any real number \var{x}:

\begin{scheme}
(max +inf.0 \var{x})                         \ev  +inf.0
(min -inf.0 \var{x})                         \ev  -inf.0
\end{scheme}

\begin{note}
If any argument is inexact, then the result will also be inexact (unless
the procedure can prove that the inaccuracy is not large enough to affect the
result, which is possible only in unusual implementations).  If {\cf min} or
{\cf max} is used to compare numbers of mixed exactness, and the numerical
value of the result cannot be represented as an inexact number without loss of
accuracy, then the procedure may raise an exception with condition
type {\cf\&implementation-restriction}.
\end{note}

\end{entry}

\begin{entry}{%
\proto{+}{ \vari{z} \dotsfoo}{procedure}
\proto{*}{ \vari{z} \dotsfoo}{procedure}}

These procedures return the sum or product of their arguments.

\begin{scheme}
(+ 3 4)                                \ev  7
(+ 3)                                  \ev  3
(+)                                    \ev  0
(+ +inf.0 +inf.0)                      \ev  +inf.0
(+ +inf.0 -inf.0)                      \ev  +nan.0

(* 4)                                  \ev  4
(*)                                    \ev  1
(* 5 +inf.0)                           \ev  +inf.0
(* -5 +inf.0)                          \ev  -inf.0
(* +inf.0 +inf.0)                      \ev  +inf.0
(* +inf.0 -inf.0)                      \ev  -inf.0
(* 0 +inf.0)                           \ev  0 \textit{or} +nan.0
(* 0 +nan.0)                           \ev  0 \textit{or} +nan.0
\end{scheme}

For any real number \var{x} that is neither infinite nor NaN:

\begin{scheme}
(+ +inf.0 \var{x})                           \ev  +inf.0
(+ -inf.0 \var{x})                           \ev  -inf.0
(+ +nan.0 \var{x})                           \ev  +nan.0
\end{scheme}

For any real number \var{x} that is neither
infinite nor NaN nor 0:

\begin{scheme}
(* +nan.0 \var{x})                           \ev  +nan.0
\end{scheme}

If any of these procedures are applied to mixed non-rational real and
non-real complex arguments, they either raise an exception with
condition type {\cf\&implementation-restriction} or return an unspecified number.
\end{entry}

\begin{entry}{%
\proto{-}{ z}{procedure}
\rproto{-}{ \vari{z} \varii{z} \dotsfoo}{procedure}
\proto{/}{ z}{procedure}
\rproto{/}{ \vari{z} \varii{z} \dotsfoo}{procedure}}

With two or more arguments, these procedures return the difference or
quotient of their arguments, associating to the left.  With one
argument, however, they return the additive or multiplicative inverse
of their argument.

\begin{scheme}
(- 3 4)                                \ev  -1
(- 3 4 5)                              \ev  -6
(- 3)                                  \ev  -3
(- +inf.0 +inf.0)                      \ev  +nan.0

(/ 3 4 5)                              \ev  3/20
(/ 3)                                  \ev  1/3
(/ 0.0)                                \ev  +inf.0
(/ 1.0 0)                              \ev  +inf.0
(/ -1 0.0)                             \ev  -inf.0
(/ +inf.0)                             \ev  0.0
(/ 0 0.0)                              \ev  0 or +nan.0
(/ 0.0 0)                              \ev  +nan.0
(/ 0.0 0.0)                            \ev  +nan.0
\end{scheme}

If any of these procedures are applied to mixed non-rational real and
non-real complex arguments, they either raise an exception with
condition type {\cf\&implementation-restriction} or return an
unspecified number.
\end{entry}

\begin{entry}{%
\proto{abs}{ x}{procedure}}

{\cf Abs} returns the absolute value of its argument.

\begin{scheme}
(abs -7)                               \ev  7
(abs -inf.0)                           \ev  +inf.0
\end{scheme}

\end{entry}

\begin{entry}{%
\proto{div+mod}{ \vari{x} \varii{x}}{procedure}
\proto{div}{ \vari{x} \varii{x}}{procedure}
\proto{mod}{ \vari{x} \varii{x}}{procedure}
\proto{div0+mod0}{ \vari{x} \varii{x}}{procedure}
\proto{div0}{ \vari{x} \varii{x}}{procedure}
\proto{mod0}{ \vari{x} \varii{x}}{procedure}}

These procedures implement number-theoretic integer division and
return the results of the corresponding mathematical operations
specified in section~\ref{integerdivision}.  In each case, \vari{x}
must be neither infinite nor a NaN, and \varii{x} must be nonzero;
otherwise, an exception is raised.

\begin{scheme}
(div \vari{x} \varii{x})         \ev \(\vari{x}~\mathrm{div}~\varii{x}\)
(mod \vari{x} \varii{x})         \ev \(\vari{x}~\mathrm{mod}~\varii{x}\)
(div+mod \vari{x} \varii{x})     \ev \(\vari{x}~\mathrm{div}~\varii{x}, \vari{x}~\mathrm{mod}~\varii{x}\)\\\>\>\>; two return values
(div0 \vari{x} \varii{x})        \ev \(\vari{x}~\mathrm{div}_0~\varii{x}\)
(mod0 \vari{x} \varii{x})        \ev \(\vari{x}~\mathrm{mod}_0~\varii{x}\)
(div0+mod0 \vari{x} \varii{x})   \lev \(\vari{x}~\mathrm{div}_0~\varii{x}, \vari{x}~\mathrm{mod}_0~\varii{x}\)\\\>\>; two return values
\end{scheme}

\begin{entry}{%
\proto{gcd}{ \vari{n} \dotsfoo}{procedure}
\proto{lcm}{ \vari{n} \dotsfoo}{procedure}}

These procedures return the greatest common divisor or least common
multiple of their arguments.  The result is always non-negative.

\begin{scheme}
(gcd 32 -36)                           \ev  4
(gcd)                                  \ev  0
(lcm 32 -36)                           \ev  288
(lcm 32.0 -36)                         \ev  288.0 ; inexact
(lcm)                                  \ev  1
\end{scheme}
\end{entry}

\begin{entry}{%
\proto{numerator}{ q}{procedure}
\proto{denominator}{ q}{procedure}}

These procedures return the numerator or denominator of their
argument; the result is computed as if the argument was represented as
a fraction in lowest terms.  The denominator is always positive.  The
denominator of $0$ is defined to be $1$.

\begin{scheme}
(numerator (/ 6 4))                    \ev  3
(denominator (/ 6 4))                  \ev  2
(denominator (->inexact (/ 6 4)))                 \ev  2.0
\end{scheme}
\end{entry}

\begin{entry}{%
\proto{floor}{ x}{procedure}
\proto{ceiling}{ x}{procedure}
\proto{truncate}{ x}{procedure}
\proto{round}{ x}{procedure}}

These procedures return inexact integers on inexact arguments that are
not infinities or NaNs, and exact integers on exact rational
arguments.  For such arguments, {\cf floor} returns the largest
integer not larger than \var{x}.  {\cf Ceiling} returns the smallest
integer not smaller than \var{x}.  {\cf Truncate} returns the integer
closest to \var{x} whose absolute value is not larger than the
absolute value of \var{x}.  {\cf Round} returns the closest integer to
\var{x}, rounding to even when \var{x} is halfway between two
integers.

\begin{rationale}
{\cf Round} rounds to even for consistency with the default rounding
mode specified by the IEEE floating point standard.
\end{rationale}

\begin{note}
If the argument to one of these procedures is inexact, then the result
will also be inexact.  If an exact value is needed, the
result should be passed to the {\cf ->exact} procedure.
\end{note}

Although infinities and NaNs are not integers, these procedures return
an infinity when given an infinity as an argument, and a NaN when
given a NaN.

\begin{scheme}
(floor -4.3)                           \ev  -5.0
(ceiling -4.3)                         \ev  -4.0
(truncate -4.3)                        \ev  -4.0
(round -4.3)                           \ev  -4.0

(floor 3.5)                            \ev  3.0
(ceiling 3.5)                          \ev  4.0
(truncate 3.5)                         \ev  3.0
(round 3.5)                            \ev  4.0  ; inexact

(round 7/2)                            \ev  4    ; exact
(round 7)                              \ev  7

(floor +inf.0)                         \ev  +inf.0
(ceiling -inf.0)                       \ev  -inf.0
(round +nan.0)                         \ev  +nan.0
\end{scheme}

\end{entry}

\begin{entry}{%
\proto{rationalize}{ \vari{x} \varii{x}}{procedure}}

{\cf Rationalize} returns the {\em simplest} rational number
differing from \vari{x} by no more than \varii{x}.    A rational number $r_1$ is
{\em simpler} \mainindex{simplest rational} than another rational number
$r_2$ if $r_1 = p_1/q_1$ and $r_2 = p_2/q_2$ (in lowest terms) and $|p_1|
\leq |p_2|$ and $|q_1| \leq |q_2|$.  Thus $3/5$ is simpler than $4/7$.
Although not all rationals are comparable in this ordering (consider $2/7$
and $3/5$) any interval contains a rational number that is simpler than
every other rational number in that interval (the simpler $2/5$ lies
between $2/7$ and $3/5$).  Note that $0 = 0/1$ is the simplest rational of
all.

\begin{scheme}
(rationalize
  (->exact .3) 1/10)                   \ev 1/3    ; exact
(rationalize .3 1/10)                  \ev \sharpsign{}i1/3  ; inexact

(rationalize +inf.0 3)                 \ev  +inf.0
(rationalize +inf.0 +inf.0)            \ev  +nan.0
(rationalize 3 +inf.0)                 \ev  0.0
\end{scheme}

\end{entry}

\begin{entry}{%
\proto{exp}{ z}{procedure}
\proto{log}{ z}{procedure}
\rproto{log}{ \vari{z} \varii{z}}{procedure}
\proto{sin}{ z}{procedure}
\proto{cos}{ z}{procedure}
\proto{tan}{ z}{procedure}
\proto{asin}{ z}{procedure}
\proto{acos}{ z}{procedure}
\proto{atan}{ z}{procedure}
\rproto{atan}{ \vari{x} \varii{x}}{procedure}}

These procedures compute the usual transcendental functions.  {\cf
  Exp} computes the base-$e$ exponential of \var{z}.  {\cf Log}
computes the natural logarithm of \var{z} (not the base ten
logarithm).  {\cf Asin}, {\cf acos}, and {\cf atan} compute arcsine,
arccosine, and arctangent, respectively.  The two-argument variant of
{\cf atan} computes {\cf (angle (make-rectangular \varii{x}
\vari{x}))}.

See section~\ref{transcendentalfunctions} for the underlying
mathematical operations. These procedures may return inexact results
even when given exact arguments.

\begin{scheme}
(exp +inf.0)                   \ev +inf.0
(exp -inf.0)                   \ev 0.0
(log +inf.0)                   \ev +inf.0
(log 0.0)                      \ev -inf.0
(log 0)                        \exception{\&contract}
(log -inf.0)                   \ev +inf.0+\(\pi\)i
(atan -inf.0)                  \ev -1.5707963267948965 \\\>\>\>; approximately
(atan +inf.0)                  \ev 1.5707963267948965� \\\>\>\>; approximately
(log -1.0+0.0i)                \ev 0.0+\(\pi\)i
(log -1.0-0.0i)                \ev 0.0-\(\pi\)i\\\>; if -0.0 is distinguished
\end{scheme}
\end{entry}

\begin{entry}{%
\proto{sqrt}{ z}{procedure}}

Returns the principal square root of \var{z}.  For rational \var{z},
the result will have either positive real part, or zero real part and
non-negative imaginary part.  With $\log$ defined as in
section~\ref{transcendentalfunctions}, the value of {\cf (sqrt
  \var{z})} could be expressed as
%
\begin{displaymath}
e^{\frac{\log \var{z}}{2}}.
\end{displaymath}

{\cf Sqrt} may return an inexact result even when given an exact
argument.

\begin{scheme}
(sqrt -5)                   \ev  0.0+2.23606797749979i
(sqrt +inf.0)               \ev  +inf.0
(sqrt -inf.0)               \ev  +inf.0i
\end{scheme}
\end{entry}

\begin{entry}{%
\proto{expt}{ \vari{z} \varii{z}}{procedure}}

Returns \vari{z} raised to the power \varii{z}.  For nonzero \vari{z},
%
\begin{displaymath}
  \vari{z}^{\varii{z}} = e^{\varii{z} \log \vari{z}}
\end{displaymath}

$0.0^{\var{z}}$ is $1$ if $\var{z} = 0.0$, and $0.0$ if {\cf
  (real-part \var{z})} is positive.  Otherwise, an exception is
raised with condition type {\cf\&implementation-restriction} or an unspecified
number is returned.

For an exact \vari{z} and an exact
integer \varii{z}, {\cf (expt \vari{z}
\varii{z})} must return an exact result.  For all other
values of \vari{z} and \varii{z}, {\cf (expt \vari{z}
\varii{z})} may return an inexact result, even when both
\vari{z} and \var{z2} are exact.

\begin{scheme}
(expt 5 3)                  \ev  125
(expt 5 -3)                 \ev  1/125
(expt 5 0)                  \ev  1
(expt 0 5)                  \ev  0
(expt 0 5+.0000312i)        \ev  0
(expt 0 -5)                 \ev  \unspecified
(expt 0 -5+.0000312i)       \ev  \unspecified
(expt 0 0)                  \ev  1
(expt 0.0 0.0)              \ev  1.0
\end{scheme}
\end{entry}

\begin{entry}{%
\proto{make-rectangular}{ \vari{x} \varii{x}}{procedure}
\proto{make-polar}{ \variii{x} \variv{x}}{procedure}
\proto{real-part}{ z}{procedure}
\proto{imag-part}{ z}{procedure}
\proto{magnitude}{ z}{procedure}
\proto{angle}{ z}{procedure}}

Suppose \vari{x}, \varii{x}, \variii{x}, and \variv{x} are real
numbers and \var{z} is a complex number such that
%
\begin{displaymath}
\var{z} = \vari{x} + \varii{x}i = \variii{x} e^{i\variv{x}}.
\end{displaymath}

Then:
%
\begin{scheme}
(make-rectangular \vari{x} \varii{x}) \ev \var{z}
(make-rectangular \variii{x} \variv{x}) \ev \var{z}
(real-part \var{z})              \ev \vari{x}
(imag-part \var{z})              \ev \varii{x}
(magnitude \var{z})              \ev |\variii{x}|
(angle \var{z})                  \ev \var{x}\(_{\mathrm{angle}}\)
\end{scheme}
%
where $-\pi \leq \var{x}_{\mathrm{angle}} \leq \pi$ with
$\var{x}_{\mathrm{angle}} = \variv{x} + 2\pi n$ for
some integer $n$.

\begin{scheme}
(angle -1.0)         \ev \(\pi\)
(angle -1.0+0.0)     \ev \(\pi\)
(angle -1.0-0.0)     \ev -\(\pi\)\\\>; if -0.0 is distinguished
\end{scheme}

Moreover, suppose \vari{x}, \varii{x} are such that either \vari{x}
or \varii{x} is an infinity, then
%
\begin{scheme}
(make-rectangular \vari{x} \varii{x}) \lev \var{z}
(magnitude \var{z})              \ev +inf.0
\end{scheme}
\end{entry}

{\cf Make-polar}, {\cf magnitude}, and
{\cf angle} may return inexact results even when given exact
arguments.

\begin{scheme}
(angle -1)                    \ev \(\pi\)
(angle +inf.0)                \ev 0.0
(angle -inf.0)                \ev \(\pi\)
(angle -1.0+0.0)              \ev \(\pi\)
(angle -1.0-0.0)              \ev \(-\pi\)\\\>; if -0.0 is distinguished
\end{scheme}
\end{entry}

\subsubsection{Numerical Input and Output}

\begin{entry}{%
\proto{number->string}{ z}{procedure}
\rproto{number->string}{ z radix}{procedure}
\rproto{number->string}{ z radix precision}{procedure}}

\var{Radix} must be an exact integer, either 2, 8, 10, or 16.  If
omitted, \var{radix} defaults to 10.  If a \var{precision} is
specified, then \var{z} must be an inexact complex number,
\var{precision} must be an exact positive integer, and the \var{radix}
must be 10.  The procedure {\cf number->string} takes a number and a
radix and returns as a string an external representation of the given
number in the given radix such that
%
\begin{scheme}
(let ((number \var{number})
      (radix \var{radix}))
  (eqv? number
        (string->number (number->string number
                                        radix)
                        radix)))
\end{scheme}
%
is true.  If no possible result makes this expression
true, an exception with condition type
{\cf\&implementation-restriction} is raised.

If a \var{precision} is specified, then the representations of the
inexact real components of the result, unless they are infinite or
NaN, specify an explicit \meta{mantissa width} \var{p}, and \var{p} is the
least $\var{p} \geq \var{precision}$ for which the above expression is
true.

If \var{z} is inexact, the radix is 10, and the above expression and
condition can be satisfied by a result that contains a decimal point,
then the result contains a decimal point and is expressed using the
minimum number of digits (exclusive of exponent, trailing zeroes, and
mantissa width) needed to make the above expression and condition
true~\cite{howtoprint,howtoread}; otherwise the format of the result
is unspecified.

The result returned by {\cf number->string} never contains an explicit
radix prefix.

\begin{note}
The error case can occur only when \var{z} is not a complex number
or is a complex number with a non-rational real or imaginary part.
\end{note}

\begin{rationale}
If \var{z} is an inexact number represented using binary floating
point, and the radix is 10, then the above expression is normally satisfied by
a result containing a decimal point.  The unspecified case
allows for infinities, NaNs, and representations other than binary
floating point.
\end{rationale}
\end{entry}

\begin{entry}{%
\proto{string->number}{ string}{procedure}
\rproto{string->number}{ string radix}{procedure}}

Returns a number of the maximally precise representation expressed by the
given \var{string}.  \var{Radix} must be an exact integer, either 2, 8, 10,
or 16.  If supplied, \var{radix} is a default radix that may be overridden
by an explicit radix prefix in \var{string} (e.g. {\tt "\#o177"}).  If \var{radix}
is not supplied, then the default radix is 10.  If \var{string} is not
a syntactically valid notation for a number, then {\cf string->number}
returns \schfalse{}.
%
\begin{scheme}
(string->number "100")                 \ev  100
(string->number "100" 16)              \ev  256
(string->number "1e2")                 \ev  100.0
(string->number "15\sharpsign\sharpsign")                \ev  1500.0
(string->number "+inf.0")              \ev  +inf.0
(string->number "-inf.0")              \ev  -inf.0
(string->number "+nan.0")              \ev  +nan.0
\end{scheme}
\end{entry}


\section{Booleans}
\label{booleansection}

The standard boolean objects for true and false are written as
\schtrue{} and \schfalse.\sharpindex{t}\sharpindex{f}  What really
matters, though, are the objects that the Scheme conditional expressions
({\cf if}, {\cf cond}, {\cf and}, {\cf or}, {\cf do}) treat as
true\index{true} or false\index{false}.  The phrase ``a true value''\index{true}
(or sometimes just ``true'') means any object treated as true by the
conditional expressions, and the phrase ``a false value''\index{false} (or
``false'') means any object treated as false by the conditional expressions.

\vest Of all the standard Scheme values, only \schfalse{}
% is guaranteed to count
counts as false in conditional expressions.
%  It is not
% specified whether the empty list\index{empty list} counts as false
% or as true in conditional expressions.
Except for \schfalse{},
% and possibly the empty list,
all standard Scheme values, including \schtrue,
pairs, the empty list, symbols, numbers, strings, vectors, and procedures,
count as true.

%\begin{note}
%In some implementations the empty list counts as false, contrary
%to the above.
%Nonetheless a few examples in this report assume that the
%empty list counts as true, as in \cite{IEEEScheme}.
%\end{note}

% \begin{rationale}
% For historical reasons some implementations regard \schfalse{} and the
% empty list as the same object.  These implementations therefore cannot
% make the empty list count as true in conditional expressions.
% \end{rationale}

\begin{note}
Programmers accustomed to other dialects of Lisp should be aware that
Scheme distinguishes both \schfalse{} and the empty list \index{empty list}
from the symbol \ide{nil}.
\end{note}

\vest Boolean constants evaluate to themselves, so they do not need to be quoted
in programs.

\begin{scheme}
\schtrue         \ev  \schtrue
\schfalse        \ev  \schfalse
'\schfalse       \ev  \schfalse%
\end{scheme}


\begin{entry}{%
\proto{not}{ obj}{procedure}}

{\cf Not} returns \schtrue{} if \var{obj} is false, and returns
\schfalse{} otherwise.

\begin{scheme}
(not \schtrue)   \ev  \schfalse
(not 3)          \ev  \schfalse
(not (list 3))   \ev  \schfalse
(not \schfalse)  \ev  \schtrue
(not '())        \ev  \schfalse
(not (list))     \ev  \schfalse
(not 'nil)       \ev  \schfalse%
\end{scheme}

\end{entry}


\begin{entry}{%
\proto{boolean?}{ obj}{procedure}}

{\cf Boolean?} returns \schtrue{} if \var{obj} is either \schtrue{} or
\schfalse{} and returns \schfalse{} otherwise.

\begin{scheme}
(boolean? \schfalse)  \ev  \schtrue
(boolean? 0)          \ev  \schfalse
(boolean? '())        \ev  \schfalse%
\end{scheme}

\end{entry}

 
\section{Pairs and lists}
\label{listsection}

A \defining{pair} (sometimes called a \defining{dotted pair}) is a
record structure with two fields called the car and cdr fields (for
historical reasons).  Pairs are created by the procedure {\cf cons}.
The car and cdr fields are accessed by the procedures {\cf car} and
{\cf cdr}.

Pairs are used primarily to represent lists.  A list can
be defined recursively as either the empty list\index{empty list} or a pair whose
cdr is a list.  More precisely, the set of lists is defined as the smallest
set \var{X} such that

\begin{itemize}
\item The empty list is in \var{X}.
\item If \var{list} is in \var{X}, then any pair whose cdr field contains
      \var{list} is also in \var{X}.
\end{itemize}

The objects in the car fields of successive pairs of a list are the
elements of the list.  For example, a two-element list is a pair whose car
is the first element and whose cdr is a pair whose car is the second element
and whose cdr is the empty list.  The length of a list is the number of
elements, which is the same as the number of pairs.

The empty list\mainindex{empty list} is a special object of its own type
(it is not a pair); it has no elements and its length is zero.

\begin{note}
The above definitions imply that all lists have finite length and are
terminated by the empty list.
\end{note}

A chain of pairs not ending in the empty list is called an
\defining{improper list}.  Note that an improper list is not a list.
The list and dotted notations can be combined to represent
improper lists:

\begin{scheme}
(a b c . d)%
\end{scheme}

is equivalent to

\begin{scheme}
(a . (b . (c . d)))%
\end{scheme}

Whether a given pair is a list depends upon what is stored in the cdr
field.  When the \ide{set-cdr!} procedure is used, an object can be a
list one moment and not the next:

\begin{scheme}
(define x (list 'a 'b 'c))
(define y x)
y                       \ev  (a b c)
(list? y)               \ev  \schtrue
(set-cdr! x 4)          \ev  \theunspecified
x                       \ev  (a . 4)
(eqv? x y)              \ev  \schtrue
y                       \ev  (a . 4)
(list? y)               \ev  \schfalse
(set-cdr! x x)          \ev  \theunspecified
(list? x)               \ev  \schfalse%
\end{scheme}

%It is often convenient to speak of a homogeneous list of objects
%of some particular data type, as for example \hbox{\cf (1 2 3)} is a list of
%integers.  To be more precise, suppose \var{D} is some data type.  (Any
%predicate defines a data type consisting of those objects of which the
%predicate is true.)  Then
%
%\begin{itemize}
%\item The empty list is a list of \var{D}.
%\item If \var{list} is a list of \var{D}, then any pair whose cdr is
%      \var{list} and whose car is an element of the data type \var{D} is also a
%      list of \var{D}.
%\item There are no other lists of \var{D}.
%\end{itemize}

\begin{entry}{%
\proto{pair?}{ obj}{procedure}}

{\cf Pair?} returns \schtrue{} if \var{obj} is a pair, and otherwise
returns \schfalse.

\begin{scheme}
(pair? '(a . b))        \ev  \schtrue
(pair? '(a b c))        \ev  \schtrue
(pair? '())             \ev  \schfalse
(pair? '\#(a b))         \ev  \schfalse%
\end{scheme}
\end{entry}


\begin{entry}{%
\proto{cons}{ \vari{obj} \varii{obj}}{procedure}}

Returns a newly allocated pair whose car is \vari{obj} and whose cdr is
\varii{obj}.  The pair is guaranteed to be different (in the sense of
{\cf eqv?}) from every existing object.

\begin{scheme}
(cons 'a '())           \ev  (a)
(cons '(a) '(b c d))    \ev  ((a) b c d)
(cons "a" '(b c))       \ev  ("a" b c)
(cons 'a 3)             \ev  (a . 3)
(cons '(a b) 'c)        \ev  ((a b) . c)%
\end{scheme}
\end{entry}


\begin{entry}{%
\proto{car}{ pair}{procedure}}

\nodomain{\var{Pair} must be a pair.}
Returns the contents of the car field of \var{pair}.  Note that 
taking the car of the empty list\index{empty list}
causes an exception with condition type {\cf\&contract} to be raised.

\begin{scheme}
(car '(a b c))          \ev  a
(car '((a) b c d))      \ev  (a)
(car '(1 . 2))          \ev  1
(car '())               \ev  \exception{\&contract}%
\end{scheme}
 
\end{entry}


\begin{entry}{%
\proto{cdr}{ pair}{procedure}}

\nodomain{\var{Pair} must be a pair.}
Returns the contents of the cdr field of \var{pair}.
  Note that 
taking the cdr of the empty list\index{empty list}
causes an exception with condition type {\cf\&contract} to be raised.

\begin{scheme}
(cdr '((a) b c d))      \ev  (b c d)
(cdr '(1 . 2))          \ev  2
(cdr '())               \ev  \exception{\&contract}%
\end{scheme}
 
\end{entry}



\setbox0\hbox{\tt(cadr \var{pair})}
\setbox1\hbox{procedure}


\begin{entry}{%
\proto{caar}{ pair}{procedure}
\proto{cadr}{ pair}{procedure}
\pproto{\hbox to 1\wd0 {\hfil$\vdots$\hfil}}{\hbox to 1\wd1 {\hfil$\vdots$\hfil}}
\proto{cdddar}{ pair}{procedure}
\proto{cddddr}{ pair}{procedure}}

These procedures are compositions of {\cf car} and {\cf cdr}, where
for example {\cf caddr} could be defined by

\begin{scheme}
(define caddr (lambda (x) (car (cdr (cdr x))))){\rm.}%
\end{scheme}

Arbitrary compositions, up to four deep, are provided.  There are
twenty-eight of these procedures in all.

\end{entry}


\begin{entry}{%
\proto{null?}{ obj}{procedure}}

Returns \schtrue{} if \var{obj} is the empty list\index{empty list},
otherwise returns \schfalse.

% \begin{note}
% In implementations in which the empty
% list is the same as \schfalse{}, {\cf null?} will return \schtrue{}
% if \var{obj} is \schfalse{}.
% \end{note}
 
\end{entry}

\begin{entry}{%
\proto{list?}{ obj}{procedure}}

Returns \schtrue{} if \var{obj} is a list, otherwise returns \schfalse{}.
By definition, all lists have finite length and are terminated by
the empty list.

\begin{scheme}
        (list? '(a b c))     \ev  \schtrue
        (list? '())          \ev  \schtrue
        (list? '(a . b))     \ev  \schfalse
        (let ((x (list 'a)))
          (set-cdr! x x)
          (list? x))         \ev  \schfalse%
\end{scheme}
\end{entry}


\begin{entry}{%
\proto{list}{ \var{obj} \dotsfoo}{procedure}}

Returns a newly allocated list of its arguments.

\begin{scheme}
(list 'a (+ 3 4) 'c)            \ev  (a 7 c)
(list)                          \ev  ()%
\end{scheme}
\end{entry}


\begin{entry}{%
\proto{length}{ list}{procedure}}

\nodomain{\var{List} must be a list.}
Returns the length of \var{list}.

\begin{scheme}
(length '(a b c))               \ev  3
(length '(a (b) (c d e)))       \ev  3
(length '())                    \ev  0%
\end{scheme}
\end{entry}


\begin{entry}{%
\proto{append}{ list \dotsfoo{} obj}{procedure}}

\nodomain{All \var{list}s should be lists.}
Returns a possibly improper list consisting of the elements of the first \var{list}
followed by the elements of the other \var{list}s, with \var{obj} as
the cdr of the final pair.
An improper list results if \var{obj} is not a
proper list.

\begin{scheme}
(append '(x) '(y))              \ev  (x y)
(append '(a) '(b c d))          \ev  (a b c d)
(append '(a (b)) '((c)))        \ev  (a (b) (c))
(append '(a b) '(c . d))        \ev  (a b c . d)
(append '() 'a)                 \ev  a%
\end{scheme}

The resulting improper list is always newly allocated, except that it shares
structure with the \var{obj} argument.
\end{entry}


\begin{entry}{%
\proto{reverse}{ list}{procedure}}

\nodomain{\var{List} must be a list.}
Returns a newly allocated list consisting of the elements of \var{list}
in reverse order.

\begin{scheme}
(reverse '(a b c))              \ev  (c b a)
(reverse '(a (b c) d (e (f))))  \lev  ((e (f)) d (b c) a)%
\end{scheme}
\end{entry}


\begin{entry}{%
\proto{list-tail}{ l \vr{k}}{procedure}}

\domain{\var{L} must a chain of pairs of size at least \var{k}, or, if
  \var{k} is 0, the empty list.}

Returns the subchain of pairs of \var{l} obtained by omitting the first \vr{k}
elements.
\begin{scheme}
(list-tail '(a b c d) 2)                 \ev  (c d)
(list-tail '(a b c . d) 2)                 \ev  (c . d)
\end{scheme}
\end{entry}


\begin{entry}{%
\proto{list-ref}{ l \vr{k}}{procedure}}

\domain{\var{L} must be a chain of pairs of size at least $\var{k}+1$.}

Returns the \vr{k}th element of \var{l}.

\begin{scheme}
(list-ref '(a b c d) 2)                 \ev  c
(list-ref '(a b c d)
          (inexact->exact (round 1.8))) \lev  c
(list-ref '(a b c . d) 2) \ev c
\end{scheme}
\end{entry}


%\begin{entry}{%
%\proto{last-pair}{ list}{procedure}}
%
%Returns the last pair in the nonempty, possibly improper, list \var{list}.
%{\cf Last-pair} could be defined by
%
%\begin{scheme}
%(define last-pair
%  (lambda (x)
%    (if (pair? (cdr x))
%        (last-pair (cdr x))
%        x)))%
%\end{scheme} 
% 
%\end{entry}

\begin{entry}{%
\proto{map}{ proc \vari{list} \varii{list} \dotsfoo}{procedure}}

\domain{The \var{list}s must all have the same length.  \var{Proc}
  must be a procedure.  If the \var{list}s are non-empty, they must
  take as many arguments as there are {\it list}s and returning a
  single value.}

{\cf Map} applies \var{proc} element-wise to the elements of the
\var{list}s and returns a list of the results, in order.
The dynamic order in which \var{proc} is applied to the elements of the
\var{list}s is unspecified.

\begin{scheme}
(map cadr '((a b) (d e) (g h)))   \lev  (b e h)

(map (lambda (n) (expt n n))
     '(1 2 3 4 5))                \lev  (1 4 27 256 3125)

(map + '(1 2 3) '(4 5 6))         \ev  (5 7 9)

(let ((count 0))
  (map (lambda (ignored)
         (set! count (+ count 1))
         count)
       '(a b)))                 \ev  (1 2) \var{or} (2 1)
\end{scheme}

\end{entry}


\begin{entry}{%
\proto{for-each}{ proc \vari{list} \varii{list} \dotsfoo}{procedure}}

The arguments to {\cf for-each} are like the arguments to {\cf map}, but
{\cf for-each} calls \var{proc} for its side effects rather than for its
values.  Unlike {\cf map}, {\cf for-each} is guaranteed to call \var{proc} on
the elements of the \var{list}s in order from the first element(s) to the
last, and the value returned by {\cf for-each} is the unspecified value.

\begin{scheme}
(let ((v (make-vector 5)))
  (for-each (lambda (i)
              (vector-set! v i (* i i)))
            '(0 1 2 3 4))
  v)                                \ev  \#(0 1 4 9 16)%
\end{scheme}

\end{entry}


\section{Symbols}
\label{symbolsection}

Symbols are objects whose usefulness rests on the fact that two
symbols are identical (in the sense of {\cf eqv?}) if and only if their
names are spelled the same way.  This is exactly the property needed to
represent identifiers\index{identifier} in programs, and so most
implementations of Scheme use them internally for that purpose.  Symbols
are useful for many other applications; for instance, they may be used
the way enumerated values are used in Pascal.

A symbol literal is formed using {\cf quote}.  Any character within a
symbol literal may be specified by its scalar value, using the {\tt
  \sharpsign\backwhack{}x} escape notation.

\begin{scheme}
Hello \ev Hello
'H\backwhack{}x65;llo \ev Hello
'$\lambda$ \ev $\lambda$
'\backwhack{}x3BB; \ev $\lambda$
(string->symbol "a b") \ev a\backwhack{}x20;b
(string->symbol "a\backwhack{}\backwhack{}b") \ev a\backwhack{}x5C;b
'a\backwhack{}x20;b \ev a\backwhack{}x20;b
'|a b| \>\>; \emph{syntax defect}
\>\>\textrm{; (illegal character}
\>\>\textrm{; vertical bar)}
'a\backwhack{}nb  \>\>; \emph{syntax defect}
\>\>\textrm{; (illegal use of backslash)}
'a\backwhack{}x20 \>\>; \emph{syntax defect}
\>\>\textrm{; (missing semi-colon to}
\>\>\textrm{; terminate \backwhack{}x escape)}
\end{scheme}

\begin{entry}{%
\proto{symbol?}{ obj}{procedure}}

Returns \schtrue{} if \var{obj} is a symbol, otherwise returns \schfalse.

\begin{scheme}
(symbol? 'foo)          \ev  \schtrue
(symbol? (car '(a b)))  \ev  \schtrue
(symbol? "bar")         \ev  \schfalse
(symbol? 'nil)          \ev  \schtrue
(symbol? '())           \ev  \schfalse
(symbol? \schfalse)     \ev  \schfalse%
\end{scheme}
\end{entry}


\begin{entry}{%
\proto{symbol->string}{ symbol}{procedure}}

Returns the name of \var{symbol} as a string.  
The returned string may be immutable.

\begin{scheme}
(symbol->string 'flying-fish)     
                                  \ev  "flying-fish"
(symbol->string 'Martin)          \ev  "Martin"
(symbol->string
   (string->symbol "Malvina"))     
                                  \ev  "Malvina"%
\end{scheme}
\end{entry}


\begin{entry}{%
\proto{string->symbol}{ string}{procedure}}

Returns the symbol whose name is \var{string}. 

\begin{scheme}
(eq? 'mISSISSIppi 'mississippi)  \lev  \schfalse
(string->symbol "mISSISSIppi")  \lev%
  {\rm{}the symbol with name} "mISSISSIppi"
(eq? 'bitBlt (string->symbol "bitBlt"))     \lev  \schtrue
(eq? 'JollyWog
     (string->symbol
       (symbol->string 'JollyWog)))  \lev  \schtrue
(string=? "K. Harper, M.D."
          (symbol->string
            (string->symbol "K. Harper, M.D.")))  \lev  \schtrue%
\end{scheme}

\end{entry}


\section{Characters}
\label{charactersection}

\mainindex{Unicode}
\mainindex{scalar value}

\defining{Characters} are objects that represent Unicode scalar
values~\cite{Unicode41}.

\begin{note}
  Unicode defines a standard mapping between sequences of {\em code
  points}\mainindex{code point} (integers in the range 0 to \#x10FFFF
  in the latest version of the standard) and human-readable
  ``characters.'' More precisely, Unicode distinguishes between
  glyphs, which are printed for humans to read, and characters, which
  are abstract entities that map to glyphs (sometimes in a way that's
  sensitive to surrounding characters).  Furthermore, different
  sequences of code points sometimes correspond to the same character.
  The relationships among code points, characters, and glyphs are
  subtle and complex.

  Despite this complexity, most things that a literate human would
  call a ``character'' can be represented by a single code point in
  Unicode (though there may exist code-point sequences that represent
  that same character). For example, Roman letters, Cyrillic letters,
  Hebrew consonants, and most Chinese characters fall into this
  category. Thus, the ``code point'' approximation of ``character''
  works well for many purposes. More specifically, Scheme characters
  correspond to Unicode {\em scalar values}\mainindex{scalar
    value}, which includes all code points except those designated as
  surrogates. A \defining{surrogate} is a code point in the range
  \#xD800 to \#xDFFF that is used in pairs in the UTF-16 encoding to
  encode a supplementary character (whose code is in the range
  \#x10000 to \#x10FFFF).
\end{note}


%There is no requirement that the data type of
%characters be disjoint from other data types; implementations are
%encouraged to have a separate character data type, but may choose to
%represent characters as integers, strings, or some other type.

\todo{Fix}
Characters written in the \sharpsign\backwhack{} notation are self-evaluating.
That is, they do not have to be quoted in programs.  
%The \sharpsign\backwhack{}
%notation is not an essential part of Scheme, however.  Even implementations
%that support the \sharpsign\backwhack{} notation for input do not have to
%support it for output.

\vest Some of the procedures that operate on characters ignore the
difference between upper case and lower case.  The procedures that
ignore case have \hbox{``{\tt -ci}''} (for ``case
insensitive'') embedded in their names.


\begin{entry}{%
\proto{char?}{ obj}{procedure}}

Returns \schtrue{} if \var{obj} is a character, otherwise returns \schfalse.

\end{entry}

\begin{entry}{%
\proto{char->integer}{ char}{procedure}
\proto{integer->char}{ \vr{s}}{procedure}}

\domain{\var{Sv} must be a scalar value, i.e.\ a non-negative exact
  integer in $\left[0, \#x\textrm{D7FF}\right] \cup
  \left[\#x\textrm{E000}, \#x\textrm{10FFFF}\right]$.}

Given a character, {\cf char\coerce{}integer} returns its scalar value
as an exact integer.  
For a scalar value \var{sv}, {\cf integer\coerce{}char}
returns its associated character.

\begin{scheme}
(integer->char 32) \ev \sharpsign\backwhack{}space
(char->integer (integer->char 5000))
\ev 5000
(integer->char \sharpsign{}xD800) \ev \exception{\&contract}
\end{scheme}
\end{entry}


\begin{entry}{%
\proto{char=?}{ \vari{char} \varii{char} \variii{char} \dotsfoo}{procedure}
\proto{char<?}{ \vari{char} \varii{char} \variii{char} \dotsfoo}{procedure}
\proto{char>?}{ \vari{char} \varii{char} \variii{char} \dotsfoo}{procedure}
\proto{char<=?}{ \vari{char} \varii{char} \variii{char} \dotsfoo}{procedure}
\proto{char>=?}{ \vari{char} \varii{char} \variii{char} \dotsfoo}{procedure}}

\label{characterequality}
These procedures impose a total ordering on the set of characters
according to their scalar values.

\begin{scheme}
(char<? \sharpsign\backwhack{}z \sharpsign\backwhack{}\ss) \ev \schtrue
(char<? \sharpsign\backwhack{}z \sharpsign\backwhack{}Z) \ev \schfalse
\end{scheme}

\end{entry}


\section{Strings}
\label{stringsection}

Strings are sequences of characters.  

\vest The {\em length} of a string is the number of characters that it
contains.  This number is an exact, non-negative integer that is fixed when the
string is created.  The \defining{valid indexes} of a string are the
exact non-negative integers less than the length of the string.  The first
character of a string has index 0, the second has index 1, and so on.

\vest In phrases such as ``the characters of \var{string} beginning with
index \var{start} and ending with index \var{end},'' it is understood
that the index \var{start} is inclusive and the index \var{end} is
exclusive.  Thus if \var{start} and \var{end} are the same index, a null
substring is referred to, and if \var{start} is zero and \var{end} is
the length of \var{string}, then the entire string is referred to.

\vest Some of the procedures that operate on strings ignore the
difference between upper and lower case.  The versions that ignore case
have \hbox{``{\cf -ci}''} (for ``case insensitive'') embedded in their
names.


\begin{entry}{%
\proto{string?}{ obj}{procedure}}

Returns \schtrue{} if \var{obj} is a string, otherwise returns \schfalse.
\end{entry}


\begin{entry}{%
\proto{make-string}{ \vr{k}}{procedure}
\rproto{make-string}{ \vr{k} char}{procedure}}

%\domain{\vr{k} must be a non-negative integer, and \var{char} must be
%a character.}  
{\cf Make-string} returns a newly allocated string of
length \vr{k}.  If \var{char} is given, then all elements of the string
are initialized to \var{char}, otherwise the contents of the
\var{string} are unspecified.

\end{entry}

\begin{entry}{%
\proto{string}{ char \dotsfoo}{procedure}}

Returns a newly allocated string composed of the arguments.

\end{entry}

\begin{entry}{%
\proto{string-length}{ string}{procedure}}

Returns the number of characters in the given \var{string}.
\end{entry}


\begin{entry}{%
\proto{string-ref}{ string \vr{k}}{procedure}}

\domain{\vr{k} must be a valid index of \var{string}.}
{\cf String-ref} returns character \vr{k} of \var{string} using zero-origin indexing.
\end{entry}


\begin{entry}{%
\proto{string-set!}{ string k char}{procedure}}

\domain{%\var{String} must be a string, 
\vr{k} must be a valid index of \var{string}%, and \var{char} must be a character
.}
{\cf String-set!} stores \var{char} in element \vr{k} of \var{string}
and returns the unspecified value.  % <!>

Passing an immutable string to {\cf string-set!} should cause an exception
with condition type {\cf\&contract} to be raised.
\begin{scheme}
(define (f) (make-string 3 \sharpsign\backwhack{}*))
(define (g) "***")
(string-set! (f) 0 \sharpsign\backwhack{}?)  \ev  \theunspecified
(string-set! (g) 0 \sharpsign\backwhack{}?)  \ev  \unspecified
             ; should raise \exception{\&contract}
(string-set! (symbol->string 'immutable)
             0
             \sharpsign\backwhack{}?)  \ev  \unspecified%
             ; should raise \exception{\&contract}
\end{scheme}

\end{entry}


\begin{entry}{%
\proto{string=?}{ \vari{string} \varii{string} \variii{string} \dotsfoo}{procedure}}

Returns \schtrue{} if the strings are the same length and contain the same
characters in the same positions, otherwise returns \schfalse.

\begin{scheme}
(string=? "Stra�e" "Strasse") \ev \schfalse
\end{scheme}
\end{entry}


\begin{entry}{%
\proto{string<?}{ \vari{string} \varii{string} \variii{string} \dotsfoo}{procedure}
\proto{string>?}{ \vari{string} \varii{string} \variii{string} \dotsfoo}{procedure}
\proto{string<=?}{ \vari{string} \varii{string} \variii{string} \dotsfoo}{procedure}
\proto{string>=?}{ \vari{string} \varii{string} \variii{string} \dotsfoo}{procedure}}

These procedures are the lexicographic extensions to strings of the
corresponding orderings on characters.  For example, {\cf string<?}\ is
the lexicographic ordering on strings induced by the ordering
{\cf char<?}\ on characters.  If two strings differ in length but
are the same up to the length of the shorter string, the shorter string
is considered to be lexicographically less than the longer string.

\begin{scheme}
(string<? "z" "\ss") \ev \schtrue
(string<? "z" "zz") \ev \schtrue
(string<? "z" "Z") \ev \schfalse
\end{scheme}
\end{entry}


\begin{entry}{%
\proto{substring}{ string start end}{procedure}}

\domain{\var{String} must be a string, and \var{start} and \var{end}
must be exact integers satisfying
$$0 \leq \var{start} \leq \var{end} \leq \hbox{\tt(string-length \var{string})\rm.}$$}
{\cf Substring} returns a newly allocated string formed from the characters of
\var{string} beginning with index \var{start} (inclusive) and ending with index
\var{end} (exclusive).
\end{entry}


\begin{entry}{%
\proto{string-append}{ \var{string} \dotsfoo}{procedure}}

Returns a newly allocated string whose characters form the concatenation of the
given strings.

\end{entry}


\begin{entry}{%
\proto{string->list}{ string}{procedure}
\proto{list->string}{ list}{procedure}}

\domain{\var{List} must be a list of characters.}
{\cf String\coerce{}list} returns a newly allocated list of the
characters that make up the given string.  {\cf List\coerce{}string}
returns a newly allocated string formed from the characters in the list
\var{list}. {\cf String\coerce{}list}
and {\cf list\coerce{}string} are
inverses so far as {\cf equal?}\ is concerned.  
%Implementations that provide
%destructive operations on strings should ensure that the result of
%{\cf list\coerce{}string} is newly allocated.

\end{entry}


\begin{entry}{%
\proto{string-copy}{ string}{procedure}}

Returns a newly allocated copy of the given \var{string}.

\end{entry}


\begin{entry}{%
\proto{string-fill!}{ string char}{procedure}}

Stores \var{char} in every element of the given \var{string} and returns the
unspecified value.  % <!>

\end{entry}

\section{Vectors}
\label{vectorsection}

Vectors are heterogenous structures whose elements are indexed
by integers.  A vector typically occupies less space than a list
of the same length, and the average time required to access a randomly
chosen element is typically less for the vector than for the list.

\vest The {\em length} of a vector is the number of elements that it
contains.  This number is a non-negative integer that is fixed when the
vector is created.  The {\em valid indexes}\index{valid indexes} of a
vector are the exact non-negative integers less than the length of the
vector.  The first element in a vector is indexed by zero, and the last
element is indexed by one less than the length of the vector.

Like list constants, vector constants must be quoted:

\begin{scheme}
'\#(0 (2 2 2 2) "Anna")  \lev  \#(0 (2 2 2 2) "Anna")%
\end{scheme}

\todo{Pitman sez: The visual similarity to lists is bound to be confusing
to some.  Elaborate on the distinction.}


\begin{entry}{%
\proto{vector?}{ obj}{procedure}}
 
Returns \schtrue{} if \var{obj} is a vector, otherwise returns \schfalse.
\end{entry}


\begin{entry}{%
\proto{make-vector}{ k}{procedure}
\rproto{make-vector}{ k fill}{procedure}}

Returns a newly allocated vector of \var{k} elements.  If a second
argument is given, then each element is initialized to \var{fill}.
Otherwise the initial contents of each element is unspecified.

\end{entry}


\begin{entry}{%
\proto{vector}{ obj \dotsfoo}{procedure}}

Returns a newly allocated vector whose elements contain the given
arguments.  Analogous to {\cf list}.

\begin{scheme}
(vector 'a 'b 'c)               \ev  \#(a b c)%
\end{scheme}
\end{entry}


\begin{entry}{%
\proto{vector-length}{ vector}{procedure}}

Returns the number of elements in \var{vector} as an exact integer.
\end{entry}


\begin{entry}{%
\proto{vector-ref}{ vector k}{procedure}}

\domain{\vr{k} must be a valid index of \var{vector}.}
{\cf Vector-ref} returns the contents of element \vr{k} of
\var{vector}.

\begin{scheme}
(vector-ref '\#(1 1 2 3 5 8 13 21)
            5)  \lev  8
(vector-ref '\#(1 1 2 3 5 8 13 21)
            (let ((i (round (* 2 (acos -1)))))
              (if (inexact? i)
                  (inexact->exact i)
                  i))) \lev 13%
\end{scheme}
\end{entry}


\begin{entry}{%
\proto{vector-set!}{ vector k obj}{procedure}}

\domain{\vr{k} must be a valid index of \var{vector}.}
{\cf Vector-set!} stores \var{obj} in element \vr{k} of \var{vector}.
The value returned by {\cf vector-set!}\ is the unspecified value.  % <!>

Passing an immutable vector to {\cf vector-set!} should cause an exception
with condition type {\cf\&contract} to be raised.

\begin{scheme}
(let ((vec (vector 0 '(2 2 2 2) "Anna")))
  (vector-set! vec 1 '("Sue" "Sue"))
  vec)      \lev  \#(0 ("Sue" "Sue") "Anna")

(vector-set! '\#(0 1 2) 1 "doe")  \lev  \unspecified
             ; constant vector
             ; may raise \exception{\&contract}
\end{scheme}

\end{entry}


\begin{entry}{%
\proto{vector->list}{ vector}{procedure}
\proto{list->vector}{ list}{procedure}}

{\cf Vector->list} returns a newly allocated list of the objects contained
in the elements of \var{vector}.  {\cf List->vector} returns a newly
created vector initialized to the elements of the list \var{list}.

\begin{scheme}
(vector->list '\#(dah dah didah))  \lev  (dah dah didah)
(list->vector '(dididit dah))   \lev  \#(dididit dah)%
\end{scheme}
\end{entry}


\begin{entry}{%
\proto{vector-fill!}{ vector fill}{procedure}}

Stores \var{fill} in every element of \var{vector}
and returns the unspecified value.  % <!>

\end{entry}

\section{Errors and violations}
\label{errorviolation}

\begin{entry}{%
\proto{error}{ who message \vari{irritant} \dotsfoo}{procedure}
\proto{violation}{ who message \vari{irritant} \dotsfoo}{procedure}}

\domain{\var{Who} must be a string or a symbol or \schfalse{}.
  \var{Message} must be a string.
  The \var{irritant}s are arbitrary objects.}

Both of these procedures raise an exception with a condition (see
chapter~\ref{exceptionsconditionschapter}) of the following condition types:
%
\begin{itemize}
\item If \var{who} is not \schfalse, the condition has condition type
  {\cf \&who}, with \var{who} as the value of the {\cf who} field.  In
  that case, \var{who} should identify the procedure or entity that
  detected the exception.  If it is \schfalse, the condition does not
  have condition type {\cf \&who}.
\item The condition has condition type {\cf \&message}, with
  \var{message} as the value of the {\cf message} field.
\item The condition has condition type {\cf \&irritants}, and the {\cf
    irritants} field has as its value a list of the \var{irritant}s.
\end{itemize}
%
Moreover, the condition created by {\cf error} has condition type 
{\cf \&error}, and the condition created by {\cf violation} has
condition type {\cf \&violation}.

\begin{scheme}
(define (fac n)
  (if (not (integer-valued? n))
      (violation 'fac "non-integral argument" n))
  (if (negative? n)
      (violation 'fac "negative argument" n))
  (letrec
    ((loop (lambda (n r)
             (if (zero? n)
                 r
                 (loop (- n 1) (* r n))))))
      (loop n 1)))

(fac 5) \ev 120
(fac 4.5) \ev \exception{\&violation}
(fac -3) \ev \exception{\&violation}
\end{scheme}

\begin{rationale}
  The procedures encode a common pattern of raising exceptions.
\end{rationale}
\end{entry}

\section{Control features}
\label{controlsection}
\label{valuessection}
 
% Intro flushed; not very a propos any more.
% Procedures should be discussed somewhere, however.

This chapter describes various primitive procedures which control the
flow of program execution in special ways.

\begin{entry}{%
\proto{apply}{ proc \vari{arg} $\ldots$ args}{procedure}}

\domain{\var{Proc} must be a procedure and \var{args} must be a
  list.}
Calls \var{proc} with the elements of the list
{\cf(append (list \vari{arg} \dotsfoo) \var{args})} as the actual
arguments.

\begin{scheme}
(apply + (list 3 4))              \ev  7

(define compose
  (lambda (f g)
    (lambda args
      (f (apply g args)))))

((compose sqrt *) 12 75)              \ev  30%
\end{scheme}
\end{entry}


\begin{entry}{%
\proto{call-with-current-continuation}{ proc}{procedure}
\proto{call/cc}{ proc}{procedure}}

\label{continuations} \domain{\var{Proc} must be a procedure of one
argument.} The procedure {\cf call-with-current-continuation} 
(which is the same as the procedure {\cf call/cc}) packages
up the current continuation (see the rationale below) as an ``escape
procedure''\mainindex{escape procedure} and passes it as an argument to
\var{proc}.  The escape procedure is a Scheme procedure that, if it is
later called, will abandon whatever continuation is in effect at that later
time and will instead use the continuation that was in effect
when the escape procedure was created.  Calling the escape procedure
may cause the invocation of \var{before} and \var{after} thunks installed using
\ide{dynamic-wind}.

The escape procedure accepts the same number of arguments as the
continuation of the original call to \callcc.

\vest The escape procedure that is passed to \var{proc} has
unlimited extent just like any other procedure in Scheme.  It may be stored
in variables or data structures and may be called as many times as desired.

\vest The following examples show only the most common ways in which
{\cf call-with-current-continuation} is used.  If all real uses were as
simple as these examples, there would be no need for a procedure with
the power of {\cf call-with-current-continuation}.

\begin{scheme}
(call-with-current-continuation
  (lambda (exit)
    (for-each (lambda (x)
                (if (negative? x)
                    (exit x)))
              '(54 0 37 -3 245 19))
    \schtrue))                        \ev  -3

(define list-length
  (lambda (obj)
    (call-with-current-continuation
      (lambda (return)
        (letrec ((r
                  (lambda (obj)
                    (cond ((null? obj) 0)
                          ((pair? obj)
                           (+ (r (cdr obj)) 1))
                          (else (return \schfalse))))))
          (r obj))))))

(list-length '(1 2 3 4))            \ev  4

(list-length '(a b . c))            \ev  \schfalse%

(call-with-current-continuation procedure?)
                            \ev  \schtrue%
\end{scheme}

\begin{rationale}

\vest A common use of {\cf call-with-current-continuation} is for
structured, non-local exits from loops or procedure bodies, but in fact
{\cf call-with-current-continuation} is extremely useful for implementing a
wide variety of advanced control structures.

\vest Whenever a Scheme expression is evaluated there is a
\defining{continuation} wanting the result of the expression.  The continuation
represents an entire (default) future for the computation.  If the expression is
evaluated at top level, for example, then the continuation might take the
result, print it on the screen, prompt for the next input, evaluate it, and
so on forever.  Most of the time the continuation includes actions
specified by user code, as in a continuation that will take the result,
multiply it by the value stored in a local variable, add seven, and give
the answer to the top level continuation to be printed.  Normally these
ubiquitous continuations are hidden behind the scenes and programmers do not
think much about them.  On rare occasions, however, a programmer may
need to deal with continuations explicitly.
{\cf Call-with-current-continuation} allows Scheme programmers to do
that by creating a procedure that acts just like the current
continuation.

\vest Most programming languages incorporate one or more special-purpose
escape constructs with names like {\tt exit}, \hbox{{\cf return}}, or
even {\tt goto}.  In 1965, however, Peter Landin~\cite{Landin65}
invented a general purpose escape operator called the J-operator.  John
Reynolds~\cite{Reynolds72} described a simpler but equally powerful
construct in 1972.  The {\cf catch} special form described by Sussman
and Steele in the 1975 report on Scheme is exactly the same as
Reynolds's construct, though its name came from a less general construct
in MacLisp.  Several Scheme implementors noticed that the full power of the
\ide{catch} construct could be provided by a procedure instead of by a
special syntactic construct, and the name
{\cf call-with-current-continuation} was coined in 1982.  This name is
descriptive, but opinions differ on the merits of such a long name, and
some people use the name \ide{call/cc} instead.
\end{rationale}

\end{entry}

\begin{entry}{%
\proto{values}{ obj $\ldots$}{procedure}}

Delivers all of its arguments to its continuation.
{\tt Values} might be defined as follows:
\begin{scheme}
(define (values . things)
  (call-with-current-continuation 
    (lambda (cont) (apply cont things))))
\end{scheme}

The continuations of all non-final expressions within a sequence of
expressions in {\cf lambda}, {\cf begin}, {\cf let}, {\cf let*}, {\cf
  letrec}, {\cf letrec*}, {\cf let-values}, {\cf let*-values}, {\cf
  case}, {\cf cond}, and {\cf do} forms as well as the continuations
of the \var{before} and \var{after} arguments to {\cf dynamic-wind}
take an arbitrary number of values.

Except for these and the continuations created by the {\cf
  call-with-values} procedure, all other continuations take exactly
one value.  The effect of passing an inappropriate number of values to
a continuation not created by {\cf call-with-values} is undefined.
\end{entry}

\begin{entry}{%
\proto{call-with-values}{ producer consumer}{procedure}}

Calls its \var{producer} argument with no values and
a continuation that, when passed some values, calls the
\var{consumer} procedure with those values as arguments.
The continuation for the call to \var{consumer} is the
continuation of the call to {\tt call-with-values}.

\begin{scheme}
(call-with-values (lambda () (values 4 5))
                  (lambda (a b) b))
                                                   \ev  5

(call-with-values * -)                             \ev  -1
\end{scheme}

If an inappropriate number of values is passed to a continuation
created by {\cf call-with-values}, an exception with condition type
{\cf\&contract} is raised.
\end{entry}

\begin{entry}{%
\proto{dynamic-wind}{ before thunk after}{procedure}}

\domain{\var{Before}, \var{thunk}, and \var{after} must be procedures
accepting zero arguments and returning any number of values.}

In the absense of any calls to escape procedures
(see \ide{call-with-current-continuation}),
{\cf dynamic-wind} behaves as if defined as follows.

\begin{scheme}
(define dynamic-wind
  (lambda (before thunk after)
    (before)
    (call-with-values
      (lambda () (thunk))
      (lambda vals
        (after)
        (apply values vals)))))
\end{scheme}

That is, \var{before} is called without arguments.
If \var{before} returns, \var{thunk} is called without arguments.
If \var{thunk} returns, \var{after} is called without arguments.
Finally, if \var{after} returns, the values resulting from the
call to \var{thunk} are returned.

Invoking an escape procedure to transfer control into or out of the
dynamic extent of the call to \var{thunk} can cause additional calls to
\var{before} and \var{after}.
When an escape procedure created outside the dynamic extent of the call to
\var{thunk} is invoked from within the dynamic extent, \var{after} is
called just after control leaves the dynamic extent.
Similarly, when an escape procedure created within the dynamic extent of
the call to \var{thunk} is invoked from outside the dynamic extent,
\var{before} is called just before control reenters the dynamic extent.
In the latter case, if \var{thunk} returns, \var{after} is called even
if \var{thunk} has returned previously.
While the calls to \var{before} and \var{after} are not considered to be
within the dynamic extent of the call to \var{thunk}, calls to the before
and after thunks of any other calls to {\cf dynamic-wind} that occur
within the dynamic extent of the call to \var{thunk} are considered to be
within the dynamic extent of the call to \var{thunk}.

More precisely, an escape procedure used to transfer control out of the
dynamic extent of a set of zero or more active {\cf dynamic-wind}
\var{thunk} calls $x\ \dots$ and transfer control into the dynamic extent
of a set of zero or more active {\cf dynamic-wind} \var{thunk} calls
$y\ \dots$ proceeds as follows.
It leaves the dynamic extent of the most recent $x$ and calls without
arguments the corresponding \var{after} thunk.
If the \var{after} thunk returns, the escape procedure proceeds to
the next most recent $x$, and so on.
Once each $x$ has been handled in this manner,
the escape procedure calls without arguments the \var{before} thunk
corresponding to the least recent $y$.
If the \var{before} thunk returns, the escape procedure reenters the
dynamic extent of the least recent $y$ and proceeds with the next least
recent $y$, and so on.
Once each $y$ has been handled in this manner, control is transfered to
the continuation packaged in the escape procedure.

\begin{scheme}
(let ((path '())
      (c \#f))
  (let ((add (lambda (s)
               (set! path (cons s path)))))
    (dynamic-wind
      (lambda () (add 'connect))
      (lambda ()
        (add (call-with-current-continuation
               (lambda (c0)
                 (set! c c0)
                 'talk1))))
      (lambda () (add 'disconnect)))
    (if (< (length path) 4)
        (c 'talk2)
        (reverse path))))
    \lev (connect talk1 disconnect
               connect talk2 disconnect)

(let ((n 0))
  (call-with-current-continuation
    (lambda (k)
      (dynamic-wind
        (lambda ()
          (set! n (+ n 1))
          (k))
        (lambda ()
          (set! n (+ n 2)))
        (lambda ()
          (set! n (+ n 4))))))
  n) \ev 1

(let ((n 0))
  (call-with-current-continuation
    (lambda (k)
      (dynamic-wind
        values
        (lambda ()
          (dynamic-wind
            values
            (lambda ()
              (set! n (+ n 1))
              (k))
            (lambda ()
              (set! n (+ n 2))
              (k))))
        (lambda ()
          (set! n (+ n 4))))))
  n) \ev 7
\end{scheme}
\end{entry}

\section{Tail calls and tail contexts}
\label{basetailcontextsection}

A {\em tail call}\mainindex{tail call} is a procedure call that occurs
in a {\em tail context}.  Tail contexts are defined inductively.  Note
that a tail context is always determined with respect to a particular lambda
expression.

\begin{itemize}
\item The last expression within the body of a lambda expression,
  shown as \hyper{tail expression} below, occurs in a tail context.
\begin{grammar}%
(l\=ambda \meta{formals}
  \>\arbno{\hyper{declaration}} \arbno{\hyper{definition}} \arbno{\meta{expression}} \meta{tail expression})
\end{grammar}%

\item If one of the following expressions is in a tail context,
then the subexpressions shown as \meta{tail expression} are in a tail context.
These were derived from rules in the grammar given in
chapter~\ref{formalchapter} by replacing some occurrences of \meta{expression}
with \meta{tail expression}.  Only those rules that contain tail contexts
are shown here.

\begin{grammar}%
(if \meta{expression} \meta{tail expression} \meta{tail expression})
(if \meta{expression} \meta{tail expression})

(cond \atleastone{\meta{cond clause}})
(cond \arbno{\meta{cond clause}} (else \meta{tail sequence}))

(c\=ase \meta{expression}
  \>\atleastone{\meta{case clause}})
(c\=ase \meta{expression}
  \>\arbno{\meta{case clause}}
  \>(else \meta{tail sequence}))

(and \arbno{\meta{expression}} \meta{tail expression})
(or \arbno{\meta{expression}} \meta{tail expression})

(let (\arbno{\meta{binding spec}}) \meta{tail body})
(let \meta{variable} (\arbno{\meta{binding spec}}) \meta{tail body})
(let* (\arbno{\meta{binding spec}}) \meta{tail body})
(letrec* (\arbno{\meta{binding spec}}) \meta{tail body})
(letrec (\arbno{\meta{binding spec}}) \meta{tail body})
(let-values (\arbno{\meta{mv binding spec}}) \meta{tail body})
(let*-values (\arbno{\meta{mv binding spec}}) \meta{tail body})

(let-syntax (\arbno{\meta{syntax spec}}) \meta{tail body})
(letrec-syntax (\arbno{\meta{syntax spec}}) \meta{tail body})

(begin \meta{tail sequence})

(d\=o \=(\arbno{\meta{iteration spec}})
  \>  \>(\meta{test} \meta{tail sequence})
  \>\arbno{\meta{expression}})

{\rm where}

\meta{cond clause} \: (\meta{test} \meta{tail sequence})
\meta{case clause} \: ((\arbno{\meta{S-expression}}) \meta{tail sequence})

\meta{tail body} \: \arbno{\meta{definition}} \meta{tail sequence}
\meta{tail sequence} \: \arbno{\meta{expression}} \meta{tail expression}
\end{grammar}%

\item
If a {\cf cond} expression is in a tail context, and has a clause of
the form {\cf (\hyperi{expression} => \hyperii{expression})}
then the (implied) call to
the procedure that results from the evaluation of \hyperii{expression} is in a
tail context.  \hyperii{expression} itself is not in a tail context.

\end{itemize}

Certain built-in procedures are also required to perform tail calls.
The first argument passed to \ide{apply} and to
\ide{call-with-current-continuation}, and the second argument passed to
\ide{call-with-values}, must be called via a tail call.
Similarly, \ide{eval} must evaluate its argument as if it
were in tail position within the \ide{eval} procedure.

In the following example the only tail call is the call to {\cf f}.
None of the calls to {\cf g} or {\cf h} are tail calls.  The reference to
{\cf x} is in a tail context, but it is not a call and thus is not a
tail call.
\begin{scheme}%
(lambda ()
  (if (g)
      (let ((x (h)))
        x)
      (and (g) (f))))
\end{scheme}%

\begin{note}
Implementations are allowed, but not required, to
recognize that some non-tail calls, such as the call to {\cf h}
above, can be evaluated as though they were tail calls.
In the example above, the {\cf let} expression could be compiled
as a tail call to {\cf h}. (The possibility of {\cf h} returning
an unexpected number of values can be ignored, because in that
case the effect of the {\cf let} is explicitly unspecified and
implementation-dependent.)
\end{note}

%%% Local Variables: 
%%% mode: latex
%%% TeX-master: "r6rs"
%%% End: 

	\par
\clearchaptergroupstar{Formal Semantics}
\chapter{Formal semantics}
\label{formalsemanticschapter}
%!TEX root = r6rs.tex

%\noindent\textbf{TODO}
%\begin{itemize}
%\item mention things are not meant to be in original programs (handlers ...?)
%\item exception handlers change
%\end{itemize}

This section gives a formal, operational semantics for Scheme. It does not cover the entire language. The notable missing features are the macro system, IO, and the numeric tower. The precise list of features included is given in section~\ref{sec:semantics:grammar}.

The core of the specification is a single-step term rewriting relation that indicates how an (abstract) machine behaves. In general, the Report is not a complete specification, giving implementations freedom to behave differently, typically to allow optimizations. This underspecification shows up in two ways in our semantics. 

The first is reduction rules that reduce to special ``\textbf{unknown:} \textit{string}'' states (where the string provides a description of the unknown state). The intention is that rules that reduce to such states can be replaced with arbitrary reduction rules. The precise specification of how to replace those rules is given in section~\ref{sec:semantics:underspecification}.

The other is that the single-step relation relates one program to multiple different programs, each corresponding to a legal transition that an abstract machine might take. Accordingly we use the transitive closure of the single step relation to define the semantics, $\mathcal{S}$, as a function from programs ($\mathcal{P}$) to sets of observable results ($\mathcal{R}$):
\begin{center}
\begin{tabular}{l}
$\mathcal{S} : \mathcal{P} \longrightarrow 2^{\mathcal{R}}$ \\
$\mathcal{S}(\mathcal{P}) = \{ \mathscr{O}(\mathcal{A}) \mid \mathcal{P} \rightarrow^* \mathcal{A} \}$
\end{tabular}
\end{center}
where the function $\mathscr{O}$ turns an answer from the semantics into an observable result. Intuitively, $\mathscr{O}$ is the identity function on simple base values, and returns a special tag for more complex values, like procedure and pairs.

So, an implementation conforms to the semantics if, for every program $\mathcal{P}$, the implementation produces one of the results in $\mathcal{S}(\mathcal{P})$ or, if the implementation loops forever, then there is an infinite reduction sequence starting at $\mathcal{P}$, where the reduction relation $\rightarrow$ has been adjusted to replace the \textbf{unknown:} states.

The precise definitions of $\mathcal{P}$, $\mathcal{A}$, $\mathcal{R}$, and $\mathscr{O}$ are also given in section~\ref{sec:semantics:grammar}.

\section{Background}

We assume the reader has a basic familiarity with context-sensitive
reduction semantics. Readers unfamiliar with this system may wish to
consult Felleisen and Flatt's monograph~\cite{ff:monograph} or Wright
and Felleisen~\cite{wf:type-soundness} for a thorough introduction,
including the relevant technical background, or an introduction to PLT
Redex~\cite{mfff:plt-redex} for a somewhat lighter one.

As a rough guide, we define the operational semantics of a language
via a relation on program terms, where the relation corresponds to a
single step of an abstract machine. The relation is defined using
evaluation contexts, namely terms with a distinguished place in them,
called \emph{holes}\index{hole}, where the next step of evaluation
occurs. We say that a term $e$ decomposes into an evaluation
context $E$ and another term $e'$ if $e$ is the
same as $E$ but with the hole replaced by $e'$. We write
$E[e']$ to indicate the term obtained by replacing the hole in
$E$ with $e'$.

For example, assuming that we have defined a grammar containing
non-terminals for evaluation contexts ($E$), expressions
($e$), variables ($x$), and values ($v$), we
would write:
%
\begin{displaymath}
  \begin{array}{l}
    E_1[((\sy{lambda}~(x_1 \cdots)~e_1)~v_1~\cdots)] \rightarrow
    \\
    E_1[\{ x_1 \cdots \mapsto v_1 \cdots \} e_1] ~~~~~ (\#x_1 = \#v_1)
  \end{array}
\end{displaymath}
%
to define the $\beta_v$ rewriting rule (as a part of the $\rightarrow$
single step relation). We use the names of the non-terminals (possibly
with subscripts) in a rewriting rule to restrict the application of
the rule, so it applies only when some term produced by that grammar
appears in the corresponding position in the term. If the same
non-terminal with an identical subscript appears multiple times, the
rule only applies when the corresponding terms are structurally
identical (nonterminals without subscripts are not constrained to
match each other). Thus, the occurrence of $E_1$ on both the
left-hand and right-hand side of the rule above means that the context
of the application expression does not change when using this rule.
The ellipses are a form of Kleene star, meaning that zero or more
occurrences of terms matching the pattern proceeding the ellipsis may
appear in place of the the ellipsis and the pattern preceding it. We
use the notation $\{ x_1 \cdots \mapsto v_1 \cdots \} e_1$ for
capture-avoiding substitution; in this case it means that each
$x_1$ is replaced with the corresponding $v_1$ in
$e_1$. Finally, we write side-conditions in parenthesis beside
a rule; the side-condition in the above rule indicates that the number
of $x_1$s must be the same as the number of $v_1$s.
Sometimes we use equality in the side-conditions; when we do it merely
means simple term equality, {\it i.e.} the two terms must have the
same syntactic shape.

Making the evaluation context $E$ explicit in the rule allows
us to define relations that manipulate their context. As a simple
example, we can add another rule that signals an error when a
procedure is applied to the wrong number of arguments by discarding
the evaluation context on the right-hand side of a rule:
%
\begin{displaymath}
  \begin{array}{l}
    E[((\sy{lambda}~(x_1 \cdots)~e)~v_1~\cdots)] \rightarrow
    \\
    \textrm{\textbf{error:} wrong argument count} ~~~~~ (\#x_1 \neq \#v_1)
  \end{array}
\end{displaymath}
%
Later we take advantage of the explicit evaluation context in more
sophisticated ways.

To help understand the semantics and how it behaves, we have
implemented it in PLT Redex. The implementation is available at the
Report's website: \url{http://www.r6rs.org/}. All of the reduction
rules and the metafunctions shown in the figures in this semantics
were generated automatically from the source code.



\section{Grammar}\label{sec:semantics:grammar}

\beginfig
\input{r6-fig-grammar-parti.tex}
\caption{Program Grammar}\label{fig:grammar}
\endfig

\beginfig
\input{r6-fig-grammar-partii.tex}
\caption{Evaluation Context Grammar}\label{fig:ec-grammar}
\endfig

\beginfig
\begin{center}
\parbox{3.6in}{\input{r6-fig-observable}

~

~

~

~

~
}
\parbox{2.4in}{\input{r6-fig-observable-value}}
\end{center}
\caption{Observable results}\label{fig:observable}
\endfig


Figure~\ref{fig:grammar} shows the grammar for the subset of the
Report this semantics models. Non-terminals are written in
\textit{italics} or in a calligraphic font ($\mathcal{P}$ and
$\mathcal{A}$), syntactic keywords are written in \textbf{boldface} and
other primitives are written in a \textsf{sans-serif} font.

The $\mathcal{P}$ non-terminal represents possible program states. The
first alternative is a program with a store and a series of
definitions. The second alternative is an error, and the final one is
used to indicate a place where the model does not completely specify
the behavior of the primitives it models. The $\mathcal{A}$ non-terminal
represents a final result of a program. It is just like $\mathcal{P}$
except that all of the definitions have been evaluated and the
value(s) of the last one is retained.

The $\mathcal{R}$ and $\mathcal{R}_v$ non-terminals specify the observable results of a program. Each $\mathcal{R}$ is either a sequence of values that correspond to the values produced by the program that terminates normally, or a tag indicating an uncaught exception was raised, or \sy{unknown} if the program encounters a situation the semantics does not cover (see section~\ref{sec:semantics:underspecification} for details of those situations). The $\mathcal{R}_v$ non-terminal specifies what the observable results are for a particular value: the unspecified value, a pair, the empty list, a symbol, a self-quoting value (true, false, and numbers), a condition, or a procedure.

The $\mathit{sf}$ non-terminal generates individual elements of the
store. The store holds all of the mutable state of a program. It is
explained in more detail along with the rules that manipulate it.

The $\mathit{ds}$ non-terminal generates definitions. Each definition
either binds variables with \sy{define}, is a sequence of
definitions wrapped in \beginF{}, or is an expression.  Rather
than synthesize the distinction between Scheme's two \sy{begin}
forms from the context, the \textbf{F} superscript on the
\sy{begin} indicates that this is the begin whose arguments are
expected to be forms, not expressions.

Expressions include quoted data, \sy{begin} expressions, \sy{begin0}
expressions%
\footnote{ \sy{begin0} is not part of the standard, but we include it
  to make the rules for \va{dynamic-wind} easier to read. Although
  we model it directly, it can be defined in terms of other forms we
  model here that do come from the standard:
\begin{displaymath}
  \begin{array}{rcl}
    (\sy{begin0}~e_1~e_2~\cdots) &=&
    \begin{array}{l}
      (\va{call-with-values}\\
      ~(\sy{lambda}~()~e_1)\\
      ~(\sy{lambda}~(\sy{dot}~x)\\
      ~~e_2 \cdots\\
      ~~(\va{apply}~\va{values}~x)))
    \end{array}
  \end{array}
\end{displaymath}
}, application expressions, \sy{if} expressions, \sy{set!}
expressions, \sy{handlers} expressions (used to model exceptions),
variables, non-procedure values (\nt{nonproc}), primitive
procedures (\nt{proc}), \sy{dw} expressions (used to model
\va{dynamic-wind}), continuations (written with \sy{throw}),
and lambda expressions. The \sy{dot} is used instead of a period
for procedures that accept an arbitrary number of arguments, in order
to avoid meta-circular confusion in our PLT Redex model. Quoted
expressions are either sequences, symbols, or self-quoting values
(numbers and the booleans \semtrue{} and \semfalse{}).

\beginfig
\begin{center}
\input{r6-fig-Quote.tex}

\input{r6-fig-Qtoc.tex}
\end{center}
\caption{Quote}\label{fig:quote}
\endfig

The $p$ non-terminal represents programs that have no quoted
data. Most of the reduction rules rewrite $p$ to $p$,
rather than $\mathcal{P}$ to $\mathcal{P}$, since quoted data is first
rewritten into calls to the list construction functions before
ordinary evaluation proceeds. Much like \nt{ds}, $d$
represents definitions and like \nt{es}, $e$ represents
expressions.

The values ($v$) are divided into five categories:
%
\begin{itemize}
\item the unspecified value, written $(\va{unspecified})$,
\item Non-procedures (\nt{nonproc}) include pair pointers
  (\va{pp}), \va{null}, symbols, self-quoting values
  (\nt{sqv}), and conditions. The self-quoting values are numbers,
  and the booleans \semtrue{} and \semfalse{}. Conditions represent
  the Report's condition values, but here just contain a message and
  are otherwise inert.
\item User procedure (\nt{uproc}) include multi-arity
  \sy{lambda} expressions and lambda expressions with dotted
  argument lists,
\item Primitive procedures (\nt{pproc}) include arithmetic procedures
  (\nt{aproc}): \va{+}, \va{-}, \va{/}, and \va{*}, procedures of one
  argument (\nt{proc1}): \va{null?}, \va{pair?}, \va{car}, \va{cdr},
  \va{call/cc}, \va{procedure?}, \va{condition?}, \va{unspecified?}, \va{raise}, and \va{raise-continuable}, procedures of
  two arguments: \va{cons}, \va{set-car!}, \va{set-cdr!}, \va{eqv?},
  and \va{call-with-values}, as well as \va{list}, \va{dynamic-wind},
  \va{apply}, \va{values}, \va{with-exception-handler}, and \va{unspecified}, the zero-arity
  procedure that produces the unspecified value.
\item Finally, continuations are represented as \sy{throw} expressions
  whose body consists of the context where the continuation was
  grabbed.
\end{itemize}
%
The final set of non-terminals in figure~\ref{fig:grammar}, \nt{sym},
$x$, \nt{pp}, and $n$ represent symbols, variables, pair pointers, and
numbers respectively. They are assumed to all be disjoint.
Additionally, the variables $x$ are assumed not to include any
keywords, so any program variables whose names coincide with keywords
must be renamed before the semantics can give the meaning of a
program.

The set of non-terminals for evaluation contexts is shown in
figure~\ref{fig:ec-grammar}. The $P$ non-terminal controls where
evaluation happens in a program that does not contain any quoted data.
In particular, it allows evaluation in the first definition or
expression in the sequence of expressions past the store. The $E$ and
$F$ evaluation contexts are for expressions.  They are factored in
that manner so that the \nt{PG}, \nt{G}, and \nt{H} evaluation contexts can
re-use \nt{F} and have fine-grained control over the context to support
exceptions and \va{dynamic-wind}. The starred and circled variants,
\Estar{}, \Eo, \Fstar, and \Fo{} dictate where a single value is
promoted to multiple values and where multiple values are demoted to a
single value. The \nt{U} context is used to manage the Report's underspecification of the results of \sy{set!}, \va{set-car!}, and \va{set-cdr!}. Finally, \nt{SD} and $S$ are the contexts where quoted
expressions can be simplified. The precise use of the evaluation
contexts is explained along with the relevant rules.

Figure~\ref{fig:observable} specifies a function that takes an answer ($\mathcal{A}$) and produces an observable result. It eliminates the store, and replaces complex values with simple tags that indicate only the kind of value that was produced or, if no values were produced, indicates that either an uncaught exception was raised, or that the program reached a state that is not specified by the semantics.

\section{Quote}\label{sec:semantics:quote}

The first reduction rule that applies to any program is the
\rulename{6qcons} that replaces a quoted expression with a reference
to a defined variable, and introduces a new definition. This rule
applies before any other because of the contexts in which it, and all
of the other rules, apply. In particular, this rule applies in an
\nt{SD} context. Figure~\ref{fig:ec-grammar} shows that the
\nt{SD} and $S$ contexts allow this reduction to apply in
any subexpression of a $d$ or $e$, as long as all of the
subexpressions to the left have no quoted expressions in them,
although expressions to the right may have quoted expressions.
Accordingly, this rule applies once for each quoted expression in the
program, moving them to definitions at the beginning of the program.
The rest of the rules apply in contexts that do not contain any quoted
expressions, ensuring that \rulename{6qcons} converts all quoted data
into lists before those rules apply.

Although the identifier \nt{qp} does not have a subscript, the semantics of PLT Redex's ``fresh'' declaration takes special care to ensures that the \nt{qp} on the right-hand side of the rule is indeed the same as the one in the side-condition.

\section{Multiple values}

\beginfig
\begin{center}
\input{r6-fig-Multiple--values--and--call-with-values.tex}
\end{center}
\caption{Multiple Values and Call-with-values}\label{fig:multiple-values-and-call-with-values}
\endfig

The basic strategy for multiple values is to add a rule that demotes
$(\va{values}~v)$ to $v$ and another rule that promotes
$v$ to $(\va{values}~v)$. If we allowed these rules to apply
in an arbitrary evaluation context, however, we would get infinite
reduction sequences of endless alternation between promotion and
demotion. So, the semantics allows demotion only in a context
expecting a single value and allows promotion only in a context
expecting multiple values. We obtain this behavior with a small
extension to the Felleisen-Hieb framework (also present in the
operational model for R$^5$RS~\cite{mf:op-r5rs} and work on
interoperability~\cite{mf:interop}). We extend the notation so that
holes have names (written with a subscript), and the context-matching
syntax may also demand a hole of a particular name (also written with
a subscript, for instance $E[e]_{\star}$).  The extension
allows us to give different names to the holes in which multiple
values are expected and those in which single values are expected, and
structure the grammar of contexts accordingly.

To exploit this extension, we use three kinds of holes in the
evaluation context grammar in figure~\ref{fig:ec-grammar}. The
ordinary hole \hole{} appears where the usual kinds of
evaluation can occur. The hole \holes{} appears in contexts that
allows multiple values and the hole \holeone{} appears in
contexts that expect a single value. Accordingly, the rules
\rulename{6promote} only applies in \holes{} contexts, and the
rule \rulename{6demote} only applies in \holeone{} contexts.

To see how the evaluation contexts are organized to ensure that
promotion and demotion occur in the right places, consider the $F$,
\Fstar{} and \Fo{} evaluation contexts. The \Fstar{} and \Fo{}
evaluation contexts are just the same as $F$, except that they allow
promotion to multiple values and demotion to a single value,
respectively. So, the $F$ evaluation context, rather than being
defined in terms of itself, exploits \Fstar{} and \Fo{} to dictate
where promotion and demotion can occur. For example, $F$ can be
$(\sy{if}~\Fo{}~e~e)$ meaning that demotion from $(\va{values}~v)$ to
$v$ can occur in the first argument to an \sy{if} expression.
Similarly, $F$ can be $(\sy{begin}~\Fstar{}~e~e~\cdots)$ meaning that
$v$ can be promoted to $(\va{values}~v)$ in the first argument to a
\sy{begin}.

In general, the promotion and demotion rules simplify the definitions
of the other rules. For instance, the rule for \sy{if} does not
need to consider multiple values in its first subexpression.
Similarly, the rule for \sy{begin} does not need to consider the
case of a single value as its first subexpression.

The other three rules in
figure~\ref{fig:multiple-values-and-call-with-values} handle
\va{call-with-values}. The evaluation contexts for
\va{call-with-values} (in the $F$ non-terminal) allow
evaluation in the body of a thunk that has been passed as the first
argument to \va{call-with-values}, as long as the second argument
has been reduced to a value. Once evaluation inside that thunk
completes, it will produce multiple values (since it is an \Fstar{}
position), and the entire \va{call-with-values} expression reduces
to an application of its second argument to those values, via the rule
\rulename{6cwvd}. If the thunk passed to \va{call-with-values} has
multiple body expressions, the rule \rulename{6cwvc} drops the first
one, allowing evaluation to continue with the second. Finally, in the
case that the first argument to \va{call-with-values} is a value,
but is not of the form $(\sy{lambda}~()~e)$, the rule
\rulename{6cwvw} wraps it in a thunk to trigger evaluation.

\section{Exceptions}

\beginfig
\begin{center}
\input{r6-fig-Exceptions}
\end{center}
\caption{Exceptions}\label{fig:exceptions}
\endfig

\beginfig
\begin{center}
\begin{minipage}{0.45\textwidth}
\input{r6-fig-A_0.tex}
\end{minipage}
~
\begin{minipage}{0.45\textwidth}
\input{r6-fig-A_1.tex}
\end{minipage}
\end{center}
\caption{Arity Testing Functions}\label{fig:arity}
\endfig

The workhorses for the exception system are $$(\sy{handlers}~v~\cdots{}~e)$$ expressions and the \nt{G} and \nt{PG} evaluation contexts (shown in
figure~\ref{fig:ec-grammar}). The \sy{handlers} expression records the
active exception handlers ($v \cdots$) in some expression ($e$). The
intention is that only the nearest enclosing \sy{handlers} expression
is relevant to raised exceptions, and the $G$ and \nt{PG} evaluation
contexts help achieve that goal. They are just like their counterparts
$E$ and $P$, except that \sy{handlers} expressions cannot occur on the
path to the hole, and the exception system rules take advantage of
that context to find the closest enclosing handler.

To see how the contexts work together with \sy{handler}
expressions, consider the left-hand side of the \rulename{6xunee}
rule. It matches expressions that have a call to \va{raise} or
\va{raise-continuable} (the non-terminal \nt{raise*} matches
both exception-raising procedures) expression in a \nt{PG}
evaluation context. Since the \nt{PG} context does not contain any
\sy{handlers} expressions, this exception cannot be caught, so
this expression reduces to a final state indicating the uncaught
exception. The rule \rulename{6xuneh} also signals an uncaught
exception, but it covers the case where a \sy{handlers} expression
has exhausted all of the handlers available to it. The rule applies to
expressions that have a \sy{handlers} expression (with no
exception handlers) in an arbitrary evaluation context where a call to
one of the exception-raising functions is nested in the
\sy{handlers} expression. The use of the $G$ evaluation
context ensures that there are no other \sy{handler} expressions
between this one and the raise.

The next two rules handle calls to \va{with-exception-handler}.
The \rulename{6xweh1} rule applies when there are no \sy{handler}
expressions. It constructs a new one and applies $v_2$ as a
thunk in the \sy{handler} body. If there already is a handler
expression, the \rulename{6xwehn} applies. It collects the current
handlers and adds the new one into a new \sy{handlers} expression
and, as with the previous rule, invokes the second argument to
\va{with-exception-handlers}.

The next two rules cover exceptions that are raised in the context of
a \sy{handlers} expression. If a continuable exception is raised,
\rulename{6xraisec} applies. It takes the most recently installed
handler from the nearest enclosing \sy{handlers} expression and
applies it to the argument to \va{raise-continuable}, but in a
context where the exception handlers do not include that latest
handler. The \rulename{6xraise} rule behaves similarly, except it
raises a new exception if the handler returns. The new exception is
created with the \sy{condition} special form.

The \sy{condition} special form is a stand-in for the Report's
conditions. It does not evaluate its argument (note its absence from
the $E$ grammar in figure~\ref{fig:ec-grammar}). That argument
is just a literal string describing the context in which the exception
was raised. The only operation on conditions is \va{condition?},
whose semantics are given by the two rules \rulename{6ct} and
\rulename{6cf}.

Finally, the rule \rulename{6xdone} drops a \sy{handlers} expression
when its body is fully evaluated, and the rule \rulename{6weherr}
raises an exception when \va{with-exception-handler} is supplied with
incorrect arguments.  The metafunctions in the side-condition
guarantee that this rule only applies when the arguments are not
suitable functions. Their definitions are given in
figure~\ref{fig:arity}.

\section{Arithmetic \& basic forms}

\beginfig
\begin{center}
\input{r6-fig-Arithmetic.tex}
\end{center}
\caption{Arithmetic}\label{fig:arithmetic}
\endfig

\beginfig
\begin{center}
\input{r6-fig-Basic--syntactic--forms.tex}
\end{center}
\caption{Basic Syntactic Forms}\label{fig:basic-syntactic-forms}
\endfig

This model does not include the Report's arithmetic, but does include
an idealized form in order to make experimentation with other features
simpler. Figure~\ref{fig:arithmetic} shows the reduction rules for the
primitive procedures that implement addition, subtraction,
multiplication, and division. They defer to their mathematical
analogues. In addition, when the subtraction or divison operator are
applied to no arguments, or when division receives a zero as a
divisor, or when any of the arithmetic operations receive a
non-number, an exception is raised.

Figure~\ref{fig:basic-syntactic-forms} shows the rules for
\sy{if}, \sy{begin}, and \sy{begin0}. The relevant
evaluation contexts are given by the $F$ non-terminal.

The evaluation contexts for \sy{if} only allow evaluation in its
first argument. Once that is a value, the rules for \sy{if} reduce
an \sy{if} expression to its first argument if the test is not
\semfalse{}, and to its third subexpression (or to the value
\va{unspecified} if there are only two subexpressions).

The \sy{begin} evaluation contexts allow evaluation in the first
subexpression of a begin, but only if there are two or more
subexpressions. In that case, once the first expression has been fully
simplified, the reduction rules drop its value. If there is only a
single subexpression, the \sy{begin} itself is dropped.

Like the \sy{begin} evaluation contexts, the \sy{begin0}
evaluation contexts allow evaluation of the first argument of a
\sy{begin0} expression when there are two or more subexpressions.
The \sy{begin0} evaluation contexts also allow evaluation in the
second argument of a \sy{begin0} expression, as long as the first
argument has been fully simplified. The \rulename{6begin0n} rule for
\sy{begin0} then drops a fully simplified second argument.
Eventually, there is only a single expression in the \sy{begin0},
at which point the \rulename{begin01} rule fires, and removes the
\sy{begin0} expression.

\section{Pairs \& eqv}

\beginfig
\begin{center}
\input{r6-fig-Cons.tex}
\end{center}
\caption{Lists}\label{fig:cons-cells}
\endfig

\beginfig
\begin{center}
\input{r6-fig-Cons-cell--mutation.tex}
\end{center}
\caption{Cons Cell Mutation}\label{fig:cons-cell-mutation}
\endfig

The rules in figure~\ref{fig:cons-cells} handle the pure subset of
lists (although they do use the store, to pave the way for mutation).
The first two rules handle \va{list} by reducing it to a
succession of calls to \va{cons}, followed by \va{null}.

The next rule, \rulename{6cons}, allocates a new \va{cons} cell.
It moves $(\va{cons}~v_1~v_2)$ into the store, bound to a fresh
\nt{pp}, for pair pointer (see also section~\ref{sec:semantics:quote} for a description of ``fresh'')

The rules \rulename{6car} and \rulename{6cdr} extract the components of a pair from the store when presented with a pair pointer. The next four rules handle the \va{null?} predicate and the \va{pair?} predicate, and the final two rules raise exceptions when \va{car} or \va{cdr} receive non pairs.

\section{Procedures \& application}

\beginfig
\begin{center}
\input{r6-fig-Procedure--application.tex}
\end{center}
\caption{Procedures \& Application}\label{fig:procedure-application}
\endfig

\beginfig
\begin{center}
\input{r6-fig-Apply.tex}
\end{center}
\caption{Apply}\label{fig:apply}
\endfig

\beginfig
\begin{center}
\input{r6-fig-Var-set!d_.tex}
\end{center}
\caption{Variable Assignment Metafunction}\label{fig:varsetd}
\endfig

In evaluating a procedure call, the report deliberately leaves
unspecified the order in which arguments are evaluated. To model that,
we use a reduction system with non-unique decomposition to model the
choice of which argument to evaluate. The intention is that a single
term decomposes into multiple different combinations of an evaluation
context and a reducible expression and that each choice corresponds to
a different order of evaluation.

To capture unspecified evaluation order but allow only evaluation that
is consistent with some sequential ordering of the evaluation of an
application's subexpressions, we use non-deterministic choice to pick
a subexpression to reduce only when we have not already committed to
reducing some other subexpression. To achieve that effect, we limit
the evaluation of application expressions to only those that have a
single expression that isn't fully reduced, as shown in the
non-terminal $F$, in figure~\ref{fig:ec-grammar}. To evaluate
application expressions that have more than two arguments to evaluate,
the rule \rulename{6mark} picks one of the subexpressions of an
application that is not fully simplified and lifts it out in its own
application, allowing it to be evaluated. Once one of the lifted
expressions is evaluated, the \rulename{6appN} substitutes its value
back into the original application.

The \rulename{6appN} rule also handles other applications whose
arguments are finished by substituting the first actual parameter for
the first formal parameter in the expression. Its side-condition uses
the function in figure~\ref{fig:varsetd} to ensure that there are no
\sy{set!} expressions with the parameter $x_1$ as a target.
If there is such an assignment, the \rulename{6appN!} rule applies (see also section~\ref{sec:semantics:quote} for a description of ``fresh'').
Instead of directly substituting the actual parameter for the formal
parameter, it creates a new location in the store, initially bound the
actual parameter, and substitutes a variable standing for that
location in place of the formal parameter. The store, then, handles
any eventual assignment to the parameter. Once all of the parameters
have been substituted away, the rule \rulename{6app0} applies and
evaluation of the body of the procedure begins.

The next two rules handle parameters with dotted argument lists. The
rule \rulename{6app} turns a well-formed application of a
parameter with a dotted argument list into an application of an
ordinary procedure by constructing a list of the extra arguments. The
\rulename{6arity} rule raises an exception when such a procedure
is applied to too few arguments.

The next three rules \rulename{6proct}, \rulename{6procf}, and
\rulename{6procu} handle applications of \va{procedure?}, and the
remaining rules cover applications of non-procedures and other arity
errors.

The rules in figure~\ref{fig:apply} cover 
cover \va{apply}. The first
rule, \rulename{6applyf}, covers the case where the last argument to
\va{apply} is the empty list, and simply reduces by erasing the
empty list and the \va{apply}. The second rule, \rulename{6applyc}
covers the case where \va{apply}'s final argument is a pair. It
reduces by extracting the components of the pair from the store and
putting them into the application of \va{apply}. Repeated
application of this rule thus extracts all of the list elements passed
to \va{apply} out of the store. The remaining four rules cover the
various errors that can occur when using \va{apply}: applying a
non-procedure, passing a non-list as the last argument, and supplying
too few arguments to \va{apply}.

\section{Call/cc and dynamic wind}

\beginfig
\begin{center}
\input{r6-fig-Call-cc--and--dynamic-wind.tex} \\
\input{r6-fig-TrimpoStpRe.tex}
\end{center}
\caption{Call/cc and Dynamic Wind}\label{fig:call-cc-and-dynamic-wind}
\endfig

The specification of \va{dynamic-wind} uses $(\sy{dw}~x~e~e~e)$
expressions to record which dynamic-wind middle thunks are active at
each point in the computation. Its first argument is an identifier
that is globally unique and serves to identify invocations of
\va{dynamic-wind}, in order to avoid exiting and re-entering the
same dynamic context during a continuation switch. The second, third,
and fourth arguments are calls to some pre-thunk, middle thunk, and
post thunks from a call to \va{dynamic-wind}. Evaluation only
occurs in the middle expression; the \sy{dw} expression only
serves to record which pre- and post- thunks need to be run during a
continuation switch. Accordingly, the reduction rule for an
application of \va{dynamic-wind} reduces to a call to the
pre-thunk, a \sy{dw} expression and a call to the post-thunk, as
shown in rule \rulename{6wind} in
figure~\ref{fig:call-cc-and-dynamic-wind}. The next two rules cover
abuses of the \va{dynamic-wind} procedure: calling it with
non-thunks, and calling it with the wrong number of arguments. The
\rulename{6dwdone} rule erases a \sy{dw} expression when its second
argument has finished evaluating.

The next two rules cover \va{call/cc}. The rule
\rulename{6call/cc} creates a new continuation. It takes the context
of the \va{call/cc} expression and packages it up into a
\sy{throw} expression, representing the continuation. The
\sy{throw} expression uses the fresh variable $x$ to record
where the application of \va{call/cc} occurred in the context for
use in the \rulename{6throw} rule when the continuation is applied.
That rule takes the arguments of the continuation, wraps them with a
call to \va{values}, and puts them back into the place where the
original call to \va{call/cc} occurred, replacing the current
context with the context returned by the $\mathscr{T}$ metafunction.

The $\mathscr{T}$ metafunction accepts two $D$ contexts and
builds a context that matches its second argument, the destination
context, except that additional calls to the pre- and post- thunks
from \sy{dw} expressions in the context have been added. The first
three cases in the function just simplify both the arguments so that
they are expression contexts. If the destination context is a
definition, it preserves the definition and otherwise it abandons it.

The fourth clause of the $\mathscr{T}$ metafunction exploits the
$H$ context, a context that contains everything except
\sy{dw} expressions. It ensures that shared parts of the
\va{dynamic-wind} context are ignored, recurring deeper into the
two expression contexts as long as the first \sy{dw} expression in
each have matching identifiers ($x_1$). The final rule is a
catchall; it only applies when all the others fail and thus applies
either when there are no \sy{dw}s in the context, or when the
\sy{dw} expressions do not match. It calls the two other
metafunctions defined in figure~\ref{fig:call-cc-and-dynamic-wind} and
puts their results together into a \sy{begin} expression.

The $\mathscr{S}$ metafunction extracts all of the post thunks from
its argument and the $\mathscr{R}$ metafunction extracts all of the pre
thunks from its argument. They each construct new contexts and exploit
$H$ to work through their arguments, one \sy{dw} at a time.
In each case, the metafunctions are careful to keep the right
\sy{dw} context around each of the thunks in case a continuation
jump occurs during one of their evaluations. In the case of
$\mathscr{S}$, all of the context except the \sy{dw}s are
discarded, since that was the context where the call to the
continuation occured. In contrast, the $\mathscr{R}$ metafunction
receives the destination context, and thus keeps the intermediate
parts of the context in its result.

\section{Library top level}

\beginfig
\begin{center}
\input{r6-fig-Top--level--and--Variables.tex}
\end{center}
\caption{Library Top Level}\label{fig:top-level-and-variables}
\endfig

The sequence of definitions in the body of a $p$ models the
body of a library that does not export anything and imports the
primitives described by the semantics. The grammar for $p$ does
not preclude alternating definitions and expressions (and indeed, the
semantics assigns a meaning to such programs), but the informal
semantics does, so we consider such programs to be malformed. They are
only modeled here to avoid the complexity of enforcing the requirement
that all definitions appear before any expression. Similarly, the
semantics covers multiple definitions of the same identifier, but this
also would be a syntax error, according to the informal semantics. In
this case, however, such expressions are modeled because they can also
arise via a continuation throw and via programs that use \sy{set!}
like this:
%
\begin{displaymath}
  \begin{array}{l}
    (\sy{define}~x~(\sy{set!}~y~1))\\
    (\sy{define}~y~2)
  \end{array}
\end{displaymath}
%
The only other departure from standard top-level library syntax is the
\beginF{} expressions. The super-script \textbf{F} serves to
distinguish a \sy{begin} expression whose subexpressions can be
forms from one whose subexpressions are ordinary expressions.

The first rule in figure~\ref{fig:top-level-and-variables} covers the
definition of a variable, and merely moves it into the store. The
second rule covers re-definition of a variable, and it updates the
store with the new value. The third rule drops a fully evaluated
expression, unless it is the last one and the fourth rule adds a
single expression if there are none, in order to guarantee that there
is always some result to a program.

The \rulename{6beginF} rule splices \beginF{} expressions into
their context. The \rulename{6var} rule extracts a value from the
store and \rulename{6setd} updates a value in the store, returning the
unspecified value. The next two rules, \rulename{6setu} and \rulename{6refu} handle reference and assignment of free variables. Finally, the last two rules dictate the behavior of the \va{unspecified?} predicate.

\section{Underspecification}\label{sec:semantics:underspecification}

\beginfig
\begin{center}
\input{r6-fig-Underspecification.tex}
\end{center}
\caption{Explicitly Unspecified Behavior}\label{fig:underspecification}
\endfig

The rules in figure~\ref{fig:underspecification} cover aspects of the
semantics that are explicitly unspecified. Implementations can replace
the rules \rulename{6ueqv}, \rulename{6ueqc}, \rulename{6uval} and with different rules that cover the left-hand sides and, as long as they follow the informal specification, any replacement is valid. Those three situations correspond to the case when \va{eqv?} applied to two procedures or
two conditions, and multiple values are used in a single-value context.

The remaining rules in figure~\ref{fig:underspecification} cover the results from the assignment operations, \sy{set!}, \va{set-car!}, and \va{set-cdr!}. An implementation does not adjust those rules, but instead renders them useless by adjusting the rules that insert \va{unspecified}: \rulename{6setcar}, \rulename{6setcdr}, \rulename{6set}, and \rulename{6setd}. Those rules can be adjusted by replacing \va{unspecified} with any number of values in those rules.

So, the remaining rules just specify the minimal behavior that we know that a value or values must have and otherwise reduce to an \textbf{unknown:} state. The rule \rulename{6udemand} drops \va{unspecified} in the \sy{U} context. See figure~\ref{fig:ec-grammar} for the precise definition of \sy{U}, but intuitively it is a context that is only a single expression layer deep that contains expressions whose value depends on the value of their subexpressions, like the first subexpression of a \sy{if}. Following that are rules that discard \va{unspecified} in expressions that discard the results of some of their subexpressions. The \rulename{6ubegin} shows how \sy{begin} discards its first expression when there are more expressions to evaluate. The next two rules, \rulename{6uhandlers} and \rulename{6udw} propagate \va{unspecified} to their context, since they also return any number of values to their context. Finally, the two \va{begin0} rules preserve \va{unspecified} until the rule \rulename{6begin01} can return it to its context.

\section*{Acknowledgments}

Thanks to Will Clinger and Mike Sperber for many helpful discussions of the semantics, and for careful readings of earlier drafts of the semantics.

%%% Local Variables: 
%%% mode: latex
%%% TeX-master: "paper"
%%% End: 
 \par
\clearchaptergroupstar{Appendices}
\appendix
\chapter{Sample definitions for derived forms}
\label{derivedformsappendix}

This appendix contains sample definitions for some of the keywords
described in this report in terms of simpler forms:

\subsubsection*{{\tt cond}}
The {\cf cond} keyword (section~\ref{cond}) 
could be defined in terms of {\cf if}, {\cf let} and {\cf
  begin} using {\cf syntax-rules} (see
section~\ref{syntaxrulessection}) as follows:

\begin{scheme}
(define-syntax \ide{cond}
  (syntax-rules (else =>)
    ((cond (else result1 result2 ...))
     (begin result1 result2 ...))
    ((cond (test => result))
     (let ((temp test))
       (if temp (result temp))))
    ((cond (test => result) clause1 clause2 ...)
     (let ((temp test))
       (if temp
           (result temp)
           (cond clause1 clause2 ...))))
    ((cond (test)) test)
    ((cond (test) clause1 clause2 ...)
     (let ((temp test))
       (if temp
           temp
           (cond clause1 clause2 ...))))
    ((cond (test result1 result2 ...))
     (if test (begin result1 result2 ...)))
    ((cond (test result1 result2 ...)
           clause1 clause2 ...)
     (if test
         (begin result1 result2 ...)
         (cond clause1 clause2 ...)))))
\end{scheme}
\subsubsection*{{\tt case}}
The {\cf case} keyword (section~\ref{case}) could be defined in terms of {\cf let}, {\cf cond}, and
{\cf memv} (see library chapter~\ref{lib:listutilities}) using {\cf syntax-rules}
(see section~\ref{syntaxrulessection}) as follows:

\begin{scheme}
(define-syntax \ide{case}
  (syntax-rules (else)
    ((case expr0
       ((key ...) res1 res2 ...)
       ...
       (else else-res1 else-res2 ...))
     (let ((tmp expr0))
       (cond
         ((memv tmp '(key ...)) res1 res2 ...)
         ...
         (else else-res1 else-res2 ...))))
    ((case expr0
       ((keya ...) res1a res2a ...)
       ((keyb ...) res1b res2b ...)
       ...)
     (let ((tmp expr0))
       (cond
         ((memv tmp '(keya ...)) res1a res2a ...)
         ((memv tmp '(keyb ...)) res1b res2b ...)
         ...)))))
\end{scheme}

\subsubsection*{{\tt letrec}}
The {\cf letrec} keyword (section~\ref{letrec})
could be defined approximately in terms of {\cf let}
and {\cf set!} using {\cf syntax-rules} (see
section~\ref{syntaxrulessection}), using a helper
to generate the temporary variables
needed to hold the values before the assignments are made,
as follows:

\begin{scheme}
(define-syntax \ide{letrec}
  (syntax-rules ()
    ((letrec () body1 body2 ...)
     (let () body1 body2 ...))
    ((letrec ((var init) ...) body1 body2 ...)
     (letrec-helper
       (var ...)
       ()
       ((var init) ...)
       body1 body2 ...))))

(define-syntax letrec-helper
  (syntax-rules ()
    ((letrec-helper
       ()
       (temp ...)
       ((var init) ...)
       body1 body2 ...)
     (let ((var <undefined>) ...)
       (let ((temp init) ...)
         (set! var temp)
         ...)
       (let () body1 body2 ...)))
    ((letrec-helper
       (x y ...)
       (temp ...)
       ((var init) ...)
       body1 body2 ...)
     (letrec-helper
       (y ...)
       (newtemp temp ...)
       ((var init) ...)
       body1 body2 ...))))
\end{scheme}

The syntax {\cf <undefined>} represents an expression that
returns something that, when stored in a location, causes an exception
with condition type {\cf\&contract} to
be raised if an attempt to read to or write from the location occurs before the
assignments generated by the {\cf letrec} transformation take place.
(No such expression is defined in Scheme.)

\subsubsection*{{\tt let-values}}
The following definition of {\cf let-values} (section~\ref{let-values})
using {\cf syntax-rules} (see section~\ref{syntaxrulessection})
employs a pair of helpers to
create temporary names for the formals.

\begin{scheme}
(define-syntax let-values
  (syntax-rules ()
    ((let-values (binding ...) body1 body2 ...)
     (let-values-helper1
       ()
       (binding ...)
       body1 body2 ...))))

(define-syntax let-values-helper1
  ;; map over the bindings
  (syntax-rules ()
    ((let-values
       ((id temp) ...)
       ()
       body1 body2 ...)
     (let ((id temp) ...) body1 body2 ...))
    ((let-values
       assocs
       ((formals1 expr1) (formals2 expr2) ...)
       body1 body2 ...)
     (let-values-helper2
       formals1
       ()
       expr1
       assocs
       ((formals2 expr2) ...)
       body1 body2 ...))))

(define-syntax let-values-helper2
  ;; create temporaries for the formals
  (syntax-rules ()
    ((let-values-helper2
       ()
       temp-formals
       expr1
       assocs
       bindings
       body1 body2 ...)
     (call-with-values
       (lambda () expr1)
       (lambda temp-formals
         (let-values-helper1
           assocs
           bindings
           body1 body2 ...))))
    ((let-values-helper2
       (first . rest)
       (temp ...)
       expr1
       (assoc ...)
       bindings
       body1 body2 ...)
     (let-values-helper2
       rest
       (temp ... newtemp)
       expr1
       (assoc ... (first newtemp))
       bindings
       body1 body2 ...))
    ((let-values-helper2
       rest-formal
       (temp ...)
       expr1
       (assoc ...)
       bindings
       body1 body2 ...)
     (call-with-values
       (lambda () expr1)
       (lambda (temp ... . newtemp)
         (let-values-helper1
           (assoc ... (rest-formal newtemp))
           bindings
           body1 body2 ...))))))
\end{scheme}


%%% Local Variables: 
%%% mode: latex
%%% TeX-master: "r6rs"
%%% End: 
 \par
\chapter{Additional material}

The Schemers web site at
\begin{center}
{\cf http://www.schemers.org/}
\end{center}
as well as the Readscheme site at
\begin{center}
{\cf http://library.readscheme.org/}
\end{center}
contain extensive Scheme bibliographies, as well as papers,
programs, implementations, and other material related to Scheme.

%%% Local Variables: 
%%% mode: latex
%%% TeX-master: "r6rs"
%%% End: 
 \par
\input{example}	\par
%\newpage                   %  Put bib on it's own page (it's just one)
%\twocolumn[\vspace{-.18in}]%  Last bib item was on a page by itself.
\renewcommand{\bibname}{References}

\bibliographystyle{plain}
\bibliography{abbrevs,rrs}

\vfill\eject


\newcommand{\indexheading}{Alphabetic index of definitions of
  concepts, keywords, and procedures}
\newcommand{\indexintro}{The index includes entries from the library
  document; the entries are marked with ``(library)''.}

\printindex

\end{document}

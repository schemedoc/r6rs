\documentclass[twoside,twocolumn]{algol60}
%\documentclass[twoside]{algol60}

\pagestyle{headings}
\showboxdepth=0
\makeindex
% Macros for R^nRS.

\usepackage{makeidx}
\usepackage{url}

% tex2page.sty mucks with in some manner
\let\centerlinesaved=\centerline
\usepackage{tex2page}
\let\centerline=\centerlinesaved

% \let\htmlonly=\iffalse
% \let\endhtmlonly=\fi
% \let\texonly=\iftrue
% \let\endtexonly=\fi

\makeatletter

\texonly
\newcommand{\topnewpage}{\@topnewpage}
\endtexonly

\htmlonly
\newcommand{\topnewpage}[0][]{#1}
\endhtmlonly

\newcommand{\authorsc}[1]{{\scriptsize\scshape #1}}

% Chapters, sections, etc.

\newcommand{\extrapart}[1]{
 % \chapter{#1}
  \chapter*{#1}
  \markboth{#1}{#1}
  \vskip 1ex
  \addcontentsline{toc}{chapter}{#1}}

\newcommand{\clearchaptergroupstar}[1]{
  \texonly
  \clearpage
  \addcontentsline{toc}{chaptergroup}{#1}
  \topnewpage[
    \centerline{\large\bf\uppercase{#1}}
    \bigskip]
    \endtexonly
  }

\newcommand{\clearchapterstar}[1]{
  \clearpage
  \topnewpage[
    \centerline{\large\bf\uppercase{#1}}
    \bigskip]}

\newcommand{\clearextrapart}[1]{
  \clearchapterstar{#1}
  \markboth{#1}{#1}
  \addcontentsline{toc}{chapter}{#1}}

\newcommand{\vest}{}
\newcommand{\dotsfoo}{$\ldots\,$}

\newcommand{\sharpfoo}[1]{{\tt\##1}}
\newcommand{\schfalse}{\sharpfoo{f}}
\newcommand{\schtrue}{\sharpfoo{t}}

\newcommand{\ampfoo}[1]{{\tt\&#1}}

\newcommand{\libfoo}[1]{{\tt(#1)}}

\newcommand{\singlequote}{{\tt'}}  %\char19
\newcommand{\doublequote}{{\tt"}}
\newcommand{\backquote}{{\tt\char18}}
\newcommand{\backwhack}{{\tt\char`\\}}
\newcommand{\comma}{{\tt\char`\@}}
\newcommand{\atsign}{{\tt\char`\@}}
\newcommand{\bang}{{\tt\char`\!}}
\newcommand{\sharpsign}{{\tt\#}}
\newcommand{\verticalbar}{{\tt|}}
\newcommand{\openbracket}{{\tt\char`\[}}
\newcommand{\closedbracket}{{\tt\char`\]}}
\newcommand{\ampersand}{{\tt\char`\&}}

\newcommand{\coerce}{\discretionary{->}{}{->}}

% Knuth's \in sucks big boulders
\def\elem{\hbox{\raise.13ex\hbox{$\scriptstyle\in$}}}

\newcommand{\meta}[1]{{\noindent\hbox{\rm$\langle$#1$\rangle$}}}
\let\hyper=\meta
\newcommand{\hyperi}[1]{\hyper{#1$_1$}}
\newcommand{\hyperii}[1]{\hyper{#1$_2$}}
\newcommand{\hyperiii}[1]{\hyper{#1$_3$}}
\newcommand{\hyperj}[1]{\hyper{#1$_i$}}
\newcommand{\hypern}[1]{\hyper{#1$_n$}}
\texonly
\newcommand{\var}[1]{\noindent\hbox{\textnormal{\textit{#1}}}}
\endtexonly
\htmlonly
\newcommand{\var}[1]{\textnormal{\textit{#1}}}
\endhtmlonly
\newcommand{\vari}[1]{\var{#1$_1$}}
\newcommand{\varii}[1]{\var{#1$_2$}}
\newcommand{\variii}[1]{\var{#1$_3$}}
\newcommand{\variv}[1]{\var{#1$_4$}}
\newcommand{\varj}[1]{\var{#1$_j$}}
\newcommand{\varn}[1]{\var{#1$_n$}}

\newcommand{\vr}[1]{{\noindent\hbox{$#1$\/}}}  % Careful, is \/ always the right thing?
\newcommand{\vri}[1]{\vr{#1_1}}
\newcommand{\vrii}[1]{\vr{#1_2}}
\newcommand{\vriii}[1]{\vr{#1_3}}
\newcommand{\vriv}[1]{\vr{#1_4}}
\newcommand{\vrv}[1]{\vr{#1_5}}
\newcommand{\vrj}[1]{\vr{#1_j}}
\newcommand{\vrn}[1]{\vr{#1_n}}

%%R4%% The excessive use of the code font in the numbers section was
% confusing, somewhat obnoxious, and inconsistent with the rest
% of the report and with parts of the section itself.  I added
% a \tupe no-op, and changed most old uses of \type to \tupe,
% to make it easier to change the fonts back if people object
% to the change.

\newcommand{\type}[1]{{\it#1}}
\newcommand{\tupe}[1]{{#1}}

\newcommand{\defining}[1]{\mainindex{#1}{\em #1}}
\newcommand{\ide}[1]{{\schindex{#1}\frenchspacing\tt{#1}}}

\newcommand{\lambdaexp}{{\cf lambda} expression}

\newcommand{\callcc}{{\tt call-with-current-continuation}}

\newcommand{\mainschindex}[1]{\label{#1}\index{#1@\texttt{#1}}}
\newcommand{\mainindex}[1]{\index{#1}}
\newcommand{\schindex}[1]{\index{#1@\texttt{#1}}}
\newcommand{\sharpindex}[1]{\index{#1@\texttt{\#{}#1}}}
\newcommand{\ampindex}[1]{\index{#1@\texttt{\&{}#1}}}
\newcommand{\libindex}[1]{\index{#1@\texttt{(#1)}}}

\newcommand{\extref}[2]{\texonly\ref{#1}\endtexonly\htmlonly{on ``#2''}\endhtmlonly}

\renewenvironment{theindex}
{\clearpage
\topnewpage[
    \begin{center}
      \large\bf\MakeUppercase{\indexheading}
    \end{center}
    \vskip 1ex \bigskip]
    \markboth{Index}{Index}
    \addcontentsline{toc}{chapter}{\indexheading}
    \parindent\z@
    \texonly\parskip\z@ plus .1pt\endtexonly\relax\let\item\@idxitem
    \indexintro\par\bigskip}
               {\texonly\clearpage\endtexonly}


\newcommand{\domain}[1]{#1}
\newcommand{\nodomain}[1]{}
%\newcommand{\todo}[1]{{\rm$[\![$!!~#1$]\!]$}}
\newcommand{\todo}[1]{}

% \frobq will make quote and backquote look nicer.
\def\frobqcats{%\catcode`\'=13
\catcode`\`=13{}}
{\frobqcats
\gdef\frobqdefs{%\def'{\singlequote}
\def`{\backquote}}}
\def\frobq{\frobqcats\frobqdefs}

% \cf = code font
% Unfortunately, \cf \cf won't work at all, so don't even attempt to
% next constructions which use them...
\newcommand{\cf}{\frenchspacing\frobq\tt}

\texonly
% Same as \obeycr, but doesn't do a \@gobblecr.
{\catcode`\^^M=13 \gdef\myobeycr{\catcode`\^^M=13 \def^^M{\\}}%
\gdef\restorecr{\catcode`\^^M=5 }}
\endtexonly

{\obeyspaces\gdef {\hbox{\hskip0.5em}}}

\gdef\gobblecr{\@gobblecr}

\def\setupcode{\@makeother\^}

% Scheme example environment
% At 11 points, one column, these are about 56 characters wide.
% That's 32 characters to the left of the => and about 20 to the right.

\newcommand{\exception}[1]{{\cf#1} \textnormal{\textit{exception}}}
\newenvironment{schemenoindent}{
  % Commands for scheme examples
  \newcommand{\ev}{\>\>\evalsto}
  \newcommand{\lev}{\\\>\evalsto}
  \newcommand{\unspecified}{{\em{}unspecified}}
  \newcommand{\theunspecified}{{\em{}the unspecified value}}
  \setupcode
  \small \cf \obeyspaces \myobeycr
  \begin{tabbing}%
\qquad\=\hspace*{5em}\=\hspace*{9em}\=\evalsto~\=\kill%   was 16em
\gobblecr}{\unskip\end{tabbing}}

%\newenvironment{scheme}{\begin{schemenoindent}\+\kill}{\end{schemenoindent}}
\newenvironment{scheme}{
  % Commands for scheme examples
  \newcommand{\ev}{\>\>\evalsto}
  \newcommand{\lev}{\\\>\evalsto}
  \renewcommand{\em}{\rmfamily\itshape}
  \newcommand{\unspecified}{{\em{}unspecified}}
  \newcommand{\theunspecified}{{\em{}the unspecified value}}
  \setupcode
  \small \cf \obeyspaces \myobeycr
  \begin{tabbing}%
\qquad\=\hspace*{5em}\=\hspace*{9em}\=\evalsto~\=\+\kill%   was 16em
\gobblecr}{\unskip\end{tabbing}}

\newcommand{\evalsto}{$\Longrightarrow$}

% Rationale

\newenvironment{rationale}{%
\bgroup\small\noindent{\em Rationale:}\space}{%
\egroup}

% Notes

\newenvironment{note}{%
\bgroup\small\noindent{\em Note:}\space}{%
\egroup}

% Names of library modules

\newcommand{\library}[1]{{\tt (#1)}}
\newcommand{\deflibrary}[1]{\library{#1}\libindex{#1}}

% Manual entries

\newenvironment{entry}[1]{
  \vspace{3.1ex plus .5ex minus .3ex}\noindent#1%
\unpenalty\nopagebreak}{\vspace{0ex plus 1ex minus 1ex}}

\newcommand{\exprtype}{syntax}

\newcommand{\unspecifiedreturn}{the unspecified value}

% Primitive prototype
\newcommand{\pproto}[2]{\unskip%
\hbox{\cf\spaceskip=0.5em#1}\hfill\penalty 0%
\hbox{ }\nobreak\hfill\hbox{\rm #2}\break}

% Parenthesized prototype
\newcommand{\proto}[3]{\pproto{(\mainschindex{#1}\hbox{#1}{\it#2\/})}{#3}}

% Variable prototype
\newcommand{\vproto}[2]{\mainschindex{#1}\pproto{#1}{#2}}

% Condition-type prototype
 \newcommand{\ctproto}[1]{\ampindex{#1}\pproto{\ampfoo{#1}}{condition type}}

% Extending an existing definition (\proto without the index entry)
\newcommand{\rproto}[3]{\pproto{(\hbox{#1}{\it#2\/})}{#3}}

% Extending an existing definition, with index entry
\newcommand{\irproto}[3]{\schindex{#1}\rproto{#1}{#2}{#3}}

% Variable prototype
\newcommand{\rvproto}[2]{\pproto{#1}{#2}}

% Grammar environment

\newenvironment{grammar}{
  \def\:{\goesto{}}
  \def\|{$\vert$}
  \cf \myobeycr
  \begin{tabbing}
    %\qquad\quad \= 
    \qquad \= $\vert$ \= \kill
  }{\unskip\end{tabbing}}

%\newcommand{\unsection}{\unskip}
\newcommand{\unsection}{{\vskip -2ex}}

% Commands for grammars
\newcommand{\arbno}[1]{#1\hbox{\rm*}}  
\newcommand{\atleastone}[1]{#1\hbox{$^+$}}

\newcommand{\goesto}{$\longrightarrow$}

\newcommand{\syntax}{{\em Syntax: }}
\newcommand{\semantics}{{\em Semantics: }}
\newcommand{\implresp}{{\em Implementation responsibilities: }}

\newcommand{\rrs}[1]{\textit{Revised$^#1$ Report on the Algorithmic Language Scheme}}

\newcommand{\libindexentry}[1]{#1 (library)}

\makeatother

%!TEX root = paper.tex

\usepackage{latexsym}
\usepackage{mathrsfs}
\usepackage{stmaryrd}

\newcommand{\pltreducks}{PLT Redex}
\newcommand{\rnrs}{Report}
\newcommand{\rnrslongspace}{\mbox{Revised\ensuremath{\,^{\mbox{\textrm{\scriptsize 5}}}} Report on Scheme}}
\newcommand{\rnrslong}{\mbox{Revised\ensuremath{^{\mbox{\textrm{\scriptsize 5}}}} Report on Scheme}}
\newcommand{\largernrslong}{\mbox{Revised\ensuremath{\,^{\mbox{\textrm{\large 5}}}} Report on Scheme}}

%\newenvironment*{proof}
%{\noindent\textbf{Proof} }
%{$\Box$ \\}

%\newcommand{\either}{*\!{}\!{}\!\!\circ}
\newcommand{\either}{*\!\circ}

\newcommand{\hole}{[~]}
\newcommand{\holes}{\ensuremath{\hole_{\star}}}
\newcommand{\holeone}{\ensuremath{\hole_\circ}}
\newcommand{\holeany}{\ensuremath{\hole_{\either}}}

\newcommand{\sy}[1]{\textnormal{\textbf{#1}}}
\newcommand{\va}[1]{\textnormal{\textsf{#1}}}
% multi-letter nonterminals (one-letter can be done with $_$)
\newcommand{\nt}[1]{\textnormal{\textit{#1}}}

\newcommand{\beginF}{\ensuremath{\textbf{begin}^{\mbox{\textrm{\textbf{\scriptsize F}}}}}}
\newcommand{\Eo}{\ensuremath{E^{\circ}}}
\newcommand{\Estar}{\ensuremath{E^{\star}}}
\newcommand{\Fo}{\ensuremath{F^{\circ}}}
\newcommand{\Fstar}{\ensuremath{F^{\star}}}
\newcommand{\Io}{\ensuremath{I^{\circ}}}
\newcommand{\Istar}{\ensuremath{I^{\star}}}

\newcommand{\semfalse}{\textsf{\#f}}
\newcommand{\semtrue}{\textsf{\#t}}

\newcommand{\aline}{\noindent\hrulefill\par}

%\def\beginfig{\begin{figure*}[t]{\noindent\hrulefill\par}\small}
%\def\endfig{{\noindent\hrulefill\par}\end{figure*}}

\def\beginfig{\begin{figure*}[tb!]{\noindent\par}\small}
\def\endfig{{\noindent\hrulefill\par}\end{figure*}}

\newcommand{\dom}{\textit{dom}}

\newcommand{\gopen}{{^{\scriptscriptstyle\lceil}\!\!}}
\newcommand{\gclose}{\!\!{}^{\scriptscriptstyle\rceil}}

\newcommand{\mrk}{\diamond}
\newcommand{\umrk}{^\mrk}

\newcommand{\rulename}[1]{\textsf{[#1]}}

\newcommand{\extraspterm}{\\[6pt]}

\newcommand{\twolinerule}[3]{\twolineruleA{#1}{#2}{\rulename{#3}}{\rightarrow}}
\newcommand{\twolinescrule}[4]{\twolinescruleA{#1}{#2}{\rulename{#3}}{#4}{\rightarrow}}
\newcommand{\onelinerule}[3]{\onelineruleA{#1}{#2}{\rulename{#3}}{\rightarrow}}
\newcommand{\onelinescrule}[4]{\onelinescruleA{#1}{#2}{\rulename{#3}}{#4}{\rightarrow}}

\newcommand{\twolineruleA}[4]{
\multicolumn{3}{l}{{#1} {#4}} & {#3}\\ 
\multicolumn{3}{l}{{#2}} & \extraspterm}

\newcommand{\twolinescruleA}[5]{
\multicolumn{3}{l}{{#1} {#5}} & {#3}\\ 
\multicolumn{4}{l}{{#2 ~ ~ ~ {#4}}} \extraspterm}

\newcommand{\twolinescruleB}[5]{
\multicolumn{3}{l}{{#1} {#5}} & {#3}\\ 
\multicolumn{4}{l}{#2} \\
\multicolumn{4}{l}{~ ~ ~ #4} \extraspterm}

\newcommand{\onelineruleA}[4]{
\multicolumn{1}{l}{#1} & {#4} ~ & {#2} & {#3} \extraspterm}

\newcommand{\onelinescruleA}[5]{
\multicolumn{1}{l}{#1} & {#5} ~ & {#2} & {#3} \\
& & {#4} \extraspterm}



\texonly
\usepackage{xr}
\externaldocument[lib:]{r6rs-lib}
\endtexonly

\def\headertitle{Revised$^{5.94}$ Scheme}
\def\integerversion{6}

\begin{document}

\thispagestyle{empty}

\topnewpage[{
\begin{center}   {\huge\bf
        Revised{\Huge$^{\mathbf{\htmlonly\tiny\endhtmlonly{}5.94}}$} Report on the Algorithmic Language \\
                              \vskip 3pt
                                Scheme}

\vskip 1ex
$$
\begin{tabular}{l@{\extracolsep{.5in}}lll}
\multicolumn{4}{c}{M\authorsc{ICHAEL} S\authorsc{PERBER}}
\\
\multicolumn{4}{c}{W\authorsc{ILLIAM} C\authorsc{LINGER},
  R.\ K\authorsc{ENT} D\authorsc{YBVIG},
  M\authorsc{ATTHEW} F\authorsc{LATT},
  A\authorsc{NTON} \authorsc{VAN} S\authorsc{TRAATEN}}
\\
\multicolumn{4}{c}{(\textit{Editors})} \\
\multicolumn{4}{c}{
  R\authorsc{ICHARD} K\authorsc{ELSEY}, W\authorsc{ILLIAM} C\authorsc{LINGER},
  J\authorsc{ONATHAN} R\authorsc{EES}} \\
\multicolumn{4}{c}{(\textit{Editors, Revised$^5$ Report on the
    Algorithmic Language Scheme})} \\
\multicolumn{4}{c}{
  R\authorsc{OBERT} B\authorsc{RUCE} F\authorsc{INDLER}, J\authorsc{ACOB} M\authorsc{ATTHEWS}} \\
\multicolumn{4}{c}{(\textit{Authors, formal semantics})} \\[1ex]
\multicolumn{4}{c}{\bf 11 June 2007}
\end{tabular}
$$



\end{center}

\chapter*{Summary}
\medskip

{\parskip 1ex
The report gives a defining description of the programming language
Scheme.  Scheme is a statically scoped and properly tail-recursive
dialect of the Lisp programming language invented by Guy Lewis
Steele~Jr.\ and Gerald Jay~Sussman.  It was designed to have an
exceptionally clear and simple semantics and few different ways to
form expressions.  A wide variety of programming paradigms, including
functional, imperative, and message passing styles, find convenient
expression in Scheme.

This report is accompanied by a report describing standard
libraries~\cite{R6RS-libraries}; references to this document are
identified by designations such as ``library section'' or ``library
chapter''.  It is also accompanied by a report containing
non-normative appendices~\cite{R6RS-appendices}.  A third report gives
some historical background and rationales for many aspects of the
language and its libraries~\cite{R6RS-rationale}.

\medskip

The individuals listed above are not the sole authors of the text of
the report.  Over the years, the following individuals were involved
in discussions contributing to the design of the Scheme language, and
were listed as authors of prior reports:

Hal Abelson, Norman Adams, David Bartley, Gary Brooks, William
Clinger, R.\ Kent Dybvig, Daniel Friedman, Robert Halstead, Chris
Hanson, Christopher Haynes, Eugene Kohlbecker, Don Oxley, Kent Pitman,
Jonathan Rees, Guillermo Rozas, Guy L.\ Steele Jr., Gerald Jay Sussman, and
Mitchell Wand.

In order to highlight recent contributions, they are not listed as
authors of this version of the report.  However, their contribution
and service is gratefully acknowledged.

\medskip

We intend this report to belong to the entire Scheme community, and so
we grant permission to copy it in whole or in part without fee.  In
particular, we encourage implementors of Scheme to use this report as
a starting point for manuals and other documentation, modifying it as
necessary.
}

\bigskip

\begin{center}
{\large \bf
*** DRAFT*** \\
}\end{center}

This is a preliminary draft.  It is intended to reflect the decisions
taken by the editors' committee, but likely contains many mistakes,
ambiguities, and inconsistencies.

}]

\texonly\clearpage\endtexonly

\chapter*{Contents}
\addvspace{3.5pt}                  % don't shrink this gap
\renewcommand{\tocshrink}{-4.0pt}  % value determined experimentally
{
\tableofcontents
}

\vfill
\eject


\clearextrapart{Introduction}

\label{historysection}

Programming languages should be designed not by piling feature on top of
feature, but by removing the weaknesses and restrictions that make additional
features appear necessary.  Scheme demonstrates that a very small number
of rules for forming expressions, with no restrictions on how they are
composed, suffice to form a practical and efficient programming language
that is flexible enough to support most of the major programming
paradigms in use today.

Scheme
was one of the first programming languages to incorporate first class
procedures as in the lambda calculus, thereby proving the usefulness of
static scope rules and block structure in a dynamically typed language.
Scheme was the first major dialect of Lisp to distinguish procedures
from lambda expressions and symbols, to use a single lexical
environment for all variables, and to evaluate the operator position
of a procedure call in the same way as an operand position.  By relying
entirely on procedure calls to express iteration, Scheme emphasized the
fact that tail-recursive procedure calls are essentially gotos that
pass arguments.  Scheme was the first widely used programming language to
embrace first class escape procedures, from which all previously known
sequential control structures can be synthesized.  A subsequent
version of Scheme introduced the concept of exact and inexact numbers,
an extension of Common Lisp's generic arithmetic.
More recently, Scheme became the first programming language to support
hygienic macros, which permit the syntax of a block-structured language
to be extended in a consistent and reliable manner.

\todo{Ramsdell:
I would like to make a few comments on presentation.  The most
important comment is about section organization.  Newspaper writers
spend most of their time writing the first three paragraphs of any
article.  This part of the article is often the only part read by
readers, and is important in enticing readers to continue.  In the
same way, The first page is most likely to be the only page read by
many SIGPLAN readers.  If I had my choice of what I would ask them to
read, it would be the material in section 1.1, the Semantics section
that notes that scheme is lexically scoped, tail recursive, weakly
typed, ... etc.  I would expand on the discussion on continuations,
as they represent one important difference between Scheme and other
languages.  The introduction, with its history of scheme, its history
of scheme reports and meetings, and acknowledgements giving names of
people that the reader will not likely know, is not that one page I
would like all to read.  I suggest moving the history to the back of
the report, and use the first couple of pages to convince the reader
that the language documented in this report is worth studying.
}

\subsection*{Background}

\vest The first description of Scheme was written by Gerald Jay
Sussman and Guy Lewis Steele Jr.\ in
1975~\cite{Scheme75}.  A revised report by Steele and
Sussman~\cite{Scheme78}
appeared in 1978 and described the evolution
of the language as its MIT implementation was upgraded to support an
innovative compiler~\cite{Rabbit}.  Three distinct projects began in
1981 and 1982 to use variants of Scheme for courses at MIT, Yale, and
Indiana University~\cite{Rees82,MITScheme,Scheme311}.  An introductory
computer science textbook using Scheme was published in
1984~\cite{SICP}.  A number of textbooks describing and using Scheme
have been published since~\cite{tspl3}.

\vest As Scheme became more widespread,
local dialects began to diverge until students and researchers
occasionally found it difficult to understand code written at other
sites.
Fifteen representatives of the major implementations of Scheme therefore
met in October 1984 to work toward a better and more widely accepted
standard for Scheme.
Participating in this workshop were Hal Abelson, Norman Adams, David
Bartley, Gary Brooks, William Clinger, Daniel Friedman, Robert Halstead,
Chris Hanson, Christopher Haynes, Eugene Kohlbecker, Don Oxley, Jonathan Rees,
Guillermo Rozas, Gerald Jay Sussman, and Mitchell Wand.  Kent Pitman
made valuable contributions to the agenda for the workshop but was
unable to attend the sessions.
%
%Subsequent electronic mail discussions and committee work completed the
%definition of the language.
%Gerry Sussman drafted the section on numbers, Chris Hanson drafted the
%sections on characters and strings, and Gary Brooks and William Clinger
%drafted the sections on input and output.
%William Clinger recorded the decisions of the workshop and
%compiled the pieces into a coherent document.
%The ``Revised revised report on Scheme''~\cite{RRRS}
Their report~\cite{RRRS}, edited by Will Clinger,
was published at MIT and Indiana University in the summer of 1985.
Further revision took place in the spring of 1986~\cite{R3RS} (edited
by Jonathan Rees and Will Clinger),
and in the spring of 1988~\cite{R4RS} (also edited by Will Clinger and
Jonathan Rees).  Another revision published in 1998, edited
by Richard Kelsey, Will Clinger and Jonathan Rees,
reflected further revisions agreed upon in a meeting at Xerox PARC in
June 1992~\cite{R5RS}.

Attendees of the Scheme Workshop in Pittsburgh in October 2002 formed
a Strategy Committee to discuss a process for producing new revisions
of the report.  The strategy committee drafted a charter for Scheme
standardization.  This charter, together with a process for selecting
editorial committees for producing new revisions for the report, was
confirmed by the attendees of the Scheme Workshop in Boston in
November 2003.  Subsequently, a Steering Committee according to the
charter was selected, consisting of Alan Bawden, Guy L.\ Steele Jr.,
and Mitch Wand.  An editors' committee charged with producing this report was
also formed at the end of 2003, consisting of Will Clinger,
R.\ Kent Dybvig, Marc Feeley, Matthew Flatt, Richard Kelsey, Manuel
Serrano, and Mike Sperber, with Marc Feeley acting as Editor-in-Chief.
Richard Kelsey resigned from the committee in April 2005, and was
replaced by Anton van Straaten.  
Marc Feeley and Manuel Serrano
resigned from the committee in January 2006.  Subsequently, the charter
was revised to reduce the size of the editors' committee to five and
to replace the office of Editor-in-Chief by a Chair and a Project
Editor~\cite{SchemeCharter2006}.  R.\ Kent Dybvig served as Chair, and
Mike Sperber served as Project Editor.
Parts of the report were posted as Scheme Requests for Implementation
(SRFIs, see \url{http://srfi.schemers.org/})
and discussed by the community before being revised and finalized for
the report~\cite{srfi75,srfi76,srfi77,srfi83,srfi93}.
Jacob Matthews and Robby
Findler wrote the operational semantics for the language core.

\subsection*{Guiding principles}

To help guide the standardization effort, the editors have adopted a
set of principles, presented below.
Like the Scheme language defined in \rrs{5}~\cite{R5RS}, the language described
in this report is intended to:

\begin{itemize}
\item allow programmers to read each other's code, and allow
  development of portable programs that can be executed in any
  conforming implementation of Scheme;

\item derive its power from simplicity, a small number of generally
  useful core syntactic forms and procedures, and no unnecessary
  restrictions on how they are composed;
  
\item allow programs to define new procedures and new hygienic
  syntactic forms;
  
\item support the representation of program source code as data;
  
\item make procedure calls powerful enough to express any form of
  sequential control, and allow programs to perform non-local control
  operations without the use of global program transformations;
  
\item allow interesting, purely functional programs to run indefinitely
  without terminating or running out of memory on finite-memory
  machines;
  
\item allow educators to use the language to teach programming
  effectively, at various levels and with a variety of pedagogical
  approaches; and

\item allow researchers to use the language to explore the design,
  implementation, and semantics of programming languages.
\end{itemize}

In addition, this report is intended to:

\begin{itemize}
\item allow programmers to create and distribute substantial programs
  and libraries, e.g., implementations of Scheme Requests for
  Implementation, that run without
  modification in a variety of Scheme implementations;
  
\item support procedural, syntactic, and data abstraction more fully
  by allowing programs to define hygiene-bending and hygiene-breaking
  syntactic abstractions and new unique datatypes along with
  procedures and hygienic macros in any scope;
  
\item allow programmers to rely on a level of automatic run-time type
  and bounds checking sufficient to ensure type safety; and

\item allow implementations to generate efficient code, without
  requiring programmers to use implementation-specific operators or
  declarations.
\end{itemize}

While it was possible to write portable programs in Scheme as
described in \rrs{5}, and indeed portable Scheme programs were written
prior to this report, many Scheme programs were not, primarily because
of the lack of substantial standardized libraries and the
proliferation of implementation-specific language additions.

In general, Scheme should include building blocks that allow a wide
variety of libraries to be written, include commonly used user-level
features to enhance portability and readability of library and
application code, and exclude features that are less commonly used and
easily implemented in separate libraries.

The language described in this report is intended to also be backward
compatible with programs written in Scheme as described in \rrs{5} to
the extent possible without compromising the above principles and
future viability of the language.  With respect to future viability,
the editors have operated under the assumption that many more Scheme
programs will be written in the future than exist in the present, so
the future programs are those with which we should be most concerned.

\subsection*{Acknowledgements}

We would like to thank the following people for their help: Lauri
Alanko, Eli
Barzilay, Alan Bawden, Michael
Blair, Per Bothner, Trent Buck, Thomas Bushnell, Taylor Campbell, 
Ludovic Court�s, Pascal Costanza,
John Cowan, George Carrette, Andy Cromarty, David Cuthbert, Pavel Curtis, Jeff Dalton, Olivier Danvy,
Ken Dickey, Ray Dillinger, Blake Coverett, Jed Davis, Bruce Duba, Carl Eastlund,
Sebastian Egner, Tom Emerson, Marc Feeley,
Andy Freeman, Ken Friedenbach, Richard Gabriel, Martin Gasbichler, Peter Gavin, Arthur A.\ Gleckler,
Aziz Ghuloum, Yekta G\"ursel, Ken Haase, Lars T Hansen, Ben Harris,
Dave Herman, Robert Hieb, Nils M.\ Holm, Paul Hudak, Stanislav Ievlev,
James Jackson, Aubrey Jaffer, Shiro Kawai,
Alexander Kjeldaas, Michael Lenaghan, Morry Katz, Felix Klock, Donovan Kolbly,
Marcin Kowalczyk, Chris Lindblad, Thomas Lord, Bradley
Lucier, Mark Meyer, Jim Miller, Dan Muresan, Jason Orendorff, Jim Philbin,
John Ramsdell, Jeff Read, Jorgen Schaefer, Paul Schlie, Manuel Serrano,
Mike Shaff, Olin Shivers, Jonathan Shapiro, Jens Axel S\o{}gaard,
Pinku Surana, Julie Sussman, Mikael Tillenius, Sam Tobin-Hochstadt,
David Van Horn, Andre van Tonder, Reinder Verlinde, Oscar Waddell,
Perry Wagle, Alan Watson, Daniel Weise, Andrew Wilcox, Jon Wilson,
Henry Wu, Ozan Yigit,
and Chongkai Zhu.
We thank Carol Fessenden, Daniel
Friedman, and Christopher Haynes for permission to use text from the Scheme 311
version 4 reference manual.  We thank Texas Instruments, Inc.~for permission to
use text from the {\em TI Scheme Language Reference Manual}~\cite{TImanual85}.
We gladly acknowledge the influence of manuals for MIT Scheme~\cite{MITScheme},
T~\cite{Rees84}, Scheme 84~\cite{Scheme84}, Common Lisp~\cite{CLtL},
Chez Scheme~\cite{csug7}, PLT~Scheme~\cite{mzscheme352},
and Algol 60~\cite{Naur63}.

\vest We also thank Betty Dexter for the extreme effort she put into
setting this report in \TeX, and Donald Knuth for designing the program
that caused her troubles.

\vest The Artificial Intelligence Laboratory of the
Massachusetts Institute of Technology, the Computer Science
Department of Indiana University, the Computer and Information
Sciences Department of the University of Oregon, and the NEC Research
Institute supported the preparation of this report.  Support for the MIT
work was provided in part by
the Advanced Research Projects Agency of the Department of Defense under Office
of Naval Research contract N00014-80-C-0505.  Support for the Indiana
University work was provided by NSF grants NCS 83-04567 and NCS
83-03325.


%%% Local Variables: 
%%% mode: latex
%%% TeX-master: "r6rs"
%%% End: 
   \par
\vskip 2ex
\clearchaptergroupstar{Description of the language} %\unskip\vskip -2ex
\chapter{Overview of Scheme}
\label{semanticchapter}

This chapter gives an overview of Scheme's semantics.
The purpose of this overview is to explain
enough about the basic concepts of the language to facilitate
understanding of the subsequent chapters of the report, which are
organized as a reference manual.  Consequently, this overview is
not a complete introduction to the language, nor is it precise
in all respects or normative in any way.

\vest Following Algol, Scheme is a statically scoped programming
language.  Each use of a variable is associated with a lexically
apparent binding of that variable.

\vest Scheme has latent as opposed to manifest types
\cite{WaiteGoos}.  Types
are associated with objects\mainindex{object} (also called values) rather than
with variables.  (Some authors refer to languages with latent types as
untyped, weakly typed or dynamically typed languages.)  Other languages with
latent types are Python, Ruby, Smalltalk, and other dialects of Lisp.  Languages
with manifest types (sometimes referred to as strongly typed or
statically typed languages) include Algol 60, C, C\#, Java, Haskell, and ML.

\vest All objects created in the course of a Scheme computation, including
procedures and continuations, have unlimited extent.
No Scheme object is ever destroyed.  The reason that
implementations of Scheme do not (usually!)\ run out of storage is that
they are permitted to reclaim the storage occupied by an object if
they can prove that the object cannot possibly matter to any future
computation.  Other languages in which most objects have unlimited
extent include C\#, Java, Haskell, most Lisp dialects, ML, Python,
Ruby, and Smalltalk.

Implementations of Scheme must be properly tail-recursive.
This allows the execution of an iterative computation in constant space,
even if the iterative computation is described by a syntactically
recursive procedure.  Thus with a properly tail-recursive implementation,
iteration can be expressed using the ordinary procedure-call
mechanics, so that special iteration constructs are useful only as
syntactic sugar.

\vest Scheme was one of the first languages to support procedures as
objects in their own right.  Procedures can be created dynamically,
stored in data structures, returned as results of procedures, and so
on.  Other languages with these properties include Common Lisp,
Haskell, ML, Ruby, and Smalltalk.

\vest One distinguishing feature of Scheme is that continuations, which
in most other languages only operate behind the scenes, also have
``first-class'' status.  First-class continuations are useful for implementing a
wide variety of advanced control constructs, including non-local exits,
backtracking, and coroutines.

In Scheme, the argument expressions of a procedure call are evaluated
before the procedure gains control, whether the procedure needs the
result of the evaluation or not.  C, C\#, Common Lisp, Python,
Ruby, and Smalltalk are other languages that always evaluate argument
expressions before invoking a procedure.  This is distinct from the
lazy-evaluation semantics of Haskell, or the call-by-name semantics of
Algol 60, where an argument expression is not evaluated unless its
value is needed by the procedure.

Scheme's model of arithmetic provides a rich set of numerical types
and operations on them.  Furthermore, it distinguishes \textit{exact}
and \textit{inexact} number objects: Essentially, an exact number
object corresponds to a number exactly, and an inexact number object
is the result of a computation that involved rounding or other errors.

\section{Basic types}

Scheme programs manipulate \textit{objects}, which are also referred
to as \textit{values}.
Scheme objects are organized into sets of values called \textit{types}.
This section gives an overview of the fundamentally important types of the
Scheme language.  More types are described in later chapters.

\begin{note}
  As Scheme is latently typed, the use of the term \textit{type} in
  this report differs from the use of the term in the context of other
  languages, particularly those with manifest typing.
\end{note}

\paragraph{Booleans}

\mainindex{boolean}A boolean is a truth value, and can be either
true or false.  In Scheme, the object for ``false'' is written
\schfalse{}.  The object for ``true'' is written \schtrue{}.  In
most places where a truth value is expected, however, any object different from
\schfalse{} counts as true.

\paragraph{Numbers}

\mainindex{number}Scheme supports a rich variety of numerical data types, including
objects representing integers of arbitrary precision, rational numbers, complex numbers, and
inexact numbers of various kinds.  Chapter~\ref{numbertypeschapter} gives an
overview of the structure of Scheme's numerical tower.

\paragraph{Characters}

\mainindex{character}Scheme characters mostly correspond to textual characters.
More precisely, they are isomorphic to the \textit{scalar values} of
the Unicode standard.

\paragraph{Strings}

\mainindex{string}Strings are finite sequences of characters with fixed length and thus
represent arbitrary Unicode texts.

\paragraph{Symbols}

\mainindex{symbol}A symbol is an object representing a string,
the symbol's \textit{name}.
Unlike strings, two symbols whose names are spelled the same
way are never distinguishable.  Symbols are useful for many applications;
for instance, they may be used the way enumerated values are used in
other languages.

\paragraph{Pairs and lists}

\mainindex{pair}\mainindex{list}
A pair is a data structure with two components.  The most common use
of pairs is to represent (singly linked) lists, where the first
component (the ``car'') represents the first element of the list, and
the second component (the ``cdr'') the rest of the list.  Scheme also
has a distinguished empty list, which is the last cdr in a chain of
pairs that form a list.

\paragraph{Vectors}

\mainindex{vector}Vectors, like lists, are linear data structures
representing finite sequences of arbitrary objects.
Whereas the elements of a list are accessed
sequentially through the chain of pairs representing it,
the elements of a vector are addressed by integer indices.
Thus, vectors are more appropriate than
lists for random access to elements.

\paragraph{Procedures}

\mainindex{procedure}Procedures are values in Scheme.

\section{Expressions}

The most important elements of Scheme code are
\mainindex{expression}\textit{expressions}.  Expressions can be
\textit{evaluated}, producing a \textit{value}.  (Actually, any number
of values---see section~\ref{multiplereturnvaluessection}.)  The most
fundamental expressions are literal expressions:

\begin{scheme}
\schtrue{} \ev \schtrue
23 \ev 23%
\end{scheme}

This notation means that the expression \schtrue{} evaluates to
\schtrue{}, that is, the value for ``true'',  and that the expression
{\cf 23} evaluates to a number object representing the number 23.

Compound expressions are formed by placing parentheses around their
subexpressions.  The first subexpression identifies an operation; the
remaining subexpressions are operands to the operation:
%
\begin{scheme}
(+ 23 42) \ev 65
(+ 14 (* 23 42)) \ev 980%
\end{scheme}
%
In the first of these examples, {\cf +} is the name of
the built-in operation for addition, and {\cf 23} and {\cf 42} are the
operands.  The expression {\cf (+ 23 42)} reads as ``the sum of 23 and
42''.  Compound expressions can be nested---the second example reads
as ``the sum of 14 and the product of 23 and 42''.

As these examples indicate, compound expressions in Scheme are always
written using the same prefix notation\mainindex{prefix notation}.  As
a consequence, the parentheses are needed to indicate structure.
Consequently, ``superfluous'' parentheses, which are often permissible in
mathematical notation and also in many programming languages, are not
allowed in Scheme.

As in many other languages, whitespace (including line endings) is not
significant when it separates subexpressions of an expression, and
can be used to indicate structure.

\section{Variables and binding}

\mainindex{variable}\mainindex{binding}\mainindex{identifier}Scheme
allows identifiers to stand for locations containing values.
These identifiers are called variables.  In many cases, specifically
when the location's value is never modified after its creation, it is
useful to think of the variable as standing for the value directly.

\begin{scheme}
(let ((x 23)
      (y 42))
  (+ x y)) \ev 65%
\end{scheme}

In this case, the expression starting with {\cf let} is a binding
construct.  The parenthesized structure following the {\cf let} lists
variables alongside expressions: the variable {\cf x} alongside {\cf
  23}, and the variable {\cf y} alongside {\cf 42}.  The {\cf let}
expression binds {\cf x} to 23, and {\cf y} to 42.  These bindings are
available in the \textit{body} of the {\cf let} expression, {\cf (+ x
  y)}, and only there.

\section{Definitions}

\index{definition}The variables bound by a {\cf let} expression
are \textit{local}, because their bindings are visible only in the
{\cf let}'s body.  Scheme also allows creating top-level bindings for
identifiers as follows:

\begin{scheme}
(define x 23)
(define y 42)
(+ x y) \ev 65%
\end{scheme}

(These are actually ``top-level'' in the body of a top-level program or library;
see section~\ref{librariesintrosection} below.)

The first two parenthesized structures are \textit{definitions}; they
create top-level bindings, binding {\cf x} to 23 and {\cf y} to 42.
Definitions are not expressions, and cannot appear in all places
where an expression can occur.  Moreover, a definition has no value.

Bindings follow the lexical structure of the program:  When several
bindings with the same name exist, a variable refers to the binding
that is closest to it, starting with its occurrence in the program
and going from inside to outside, and referring to a top-level
binding if no
local binding can be found along the way:

\begin{scheme}
(define x 23)
(define y 42)
(let ((y 43))
  (+ x y)) \ev 66

(let ((y 43))
  (let ((y 44))
    (+ x y))) \ev 67%
\end{scheme}

\section{Forms}

While definitions are not expressions, compound expressions and
definitions exhibit similar syntactic structure:
%
\begin{scheme}
(define x 23)
(* x 2)%
\end{scheme}
%
While the first line contains a definition, and the second an
expression, this distinction depends on the bindings for {\cf define}
and {\cf *}.  At the purely syntactical level, both are
\textit{forms}\index{form}, and \textit{form} is the general name for
a syntactic part of a Scheme program.  In particular, {\cf 23} is a
\textit{subform}\index{subform} of the form {\cf (define x 23)}.

\section{Procedures}
\label{proceduressection}

\index{procedure}Definitions can also be used to define
procedures:

\begin{scheme}
(define (f x)
  (+ x 42))

(f 23) \ev 65%
\end{scheme}

A procedure is, slightly simplified, an abstraction of an
expression over objects.  In the example, the first definition defines a procedure
called {\cf f}.  (Note the parentheses around {\cf f x}, which
indicate that this is a procedure definition.)  The expression {\cf (f
  23)} is a \index{procedure call}procedure call, meaning,
roughly, ``evaluate {\cf (+ x 42)} (the body of the procedure) with
{\cf x} bound to 23''.

As procedures are objects, they can be passed to other
procedures:
%
\begin{scheme}
(define (f x)
  (+ x 42))

(define (g p x)
  (p x))

(g f 23) \ev 65%
\end{scheme}

In this example, the body of {\cf g} is evaluated with {\cf p}
bound to {\cf f} and {\cf x} bound to 23, which is equivalent
to {\cf (f 23)}, which evaluates to 65.

In fact, many predefined operations of Scheme are provided not by
syntax, but by variables whose values are procedures.
The {\cf +} operation, for example, which receives
special syntactic treatment in many other languages, is just a regular
identifier in Scheme, bound to a procedure that adds number objects.  The
same holds for {\cf *} and many others:

\begin{scheme}
(define (h op x y)
  (op x y))

(h + 23 42) \ev 65
(h * 23 42) \ev 966%
\end{scheme}

Procedure definitions are not the only way to create procedures.  A
{\cf lambda} expression creates a new procedure as an object, with no
need to specify a name:

\begin{scheme}
((lambda (x) (+ x 42)) 23) \ev 65%
\end{scheme}

The entire expression in this example is a procedure call; {\cf
  (lambda (x) (+ x 42))}, evaluates to a procedure that takes a single
number object and adds 42 to it.

\section{Procedure calls and syntactic keywords}

Whereas {\cf (+ 23 42)}, {\cf (f 23)}, and {\cf ((lambda (x) (+ x 42))
  23)} are all examples of procedure calls, {\cf lambda} and {\cf
  let} expressions are not.  This is because {\cf let}, even though
it is an identifier, is not a variable, but is instead a \textit{syntactic
  keyword}\index{syntactic keyword}.  A form that has a
syntactic keyword as its first subexpression obeys special rules determined by
the keyword.  The {\cf define} identifier in a definition is also a
syntactic keyword.  Hence, definitions are also not procedure calls.

The rules for the {\cf lambda} keyword specify that the first
subform is a list of parameters, and the remaining subforms are the body of
the procedure.  In {\cf let} expressions, the first subform is a list
of binding specifications, and the remaining subforms constitute a body of
expressions.

Procedure calls can generally be distinguished from these
\textit{special forms}\mainindex{special form} by
looking for a syntactic keyword in the first position of an
form: if the first position does not contain a syntactic keyword, the expression
is a procedure call.  
(So-called \textit{identifier macros} allow creating other kinds of
special forms, but are comparatively rare.)
The set of syntactic keywords of Scheme is
fairly small, which usually makes this task fairly simple.
It is possible, however, to create new bindings for syntactic keywords; see
section~\ref{macrosintrosection} below.

\section{Assignment}

Scheme variables bound by definitions or {\cf let} or {\cf lambda}
expressions are not actually bound directly to the objects specified in the
respective bindings, but to locations containing these objects.  The
contents of these locations can subsequently be modified destructively
via \textit{assignment}\index{assignment}:
%
\begin{scheme}
(let ((x 23))
  (set! x 42)
  x) \ev 42%
\end{scheme}

In this case, the body of the {\cf let} expression consists of two
expressions which are evaluated sequentially, with the value of the
final expression becoming the value of the entire {\cf let}
expression.  The expression {\cf (set! x 42)} is an assignment, saying
``replace the object in the location referenced by {\cf x} with 42''.
Thus, the previous value of {\cf x}, 23, is replaced by 42.

\section{Derived forms and macros}
\label{macrosintrosection}

Many of the special forms specified in this report
can be translated into more basic special forms.
For example, a {\cf let} expression can be translated
into a procedure call and a {\cf lambda} expression.  The following two
expressions are equivalent:
%
\begin{scheme}
(let ((x 23)
      (y 42))
  (+ x y)) \ev 65

((lambda (x y) (+ x y)) 23 42) \lev 65%
\end{scheme}

Special forms like {\cf let} expressions are called \textit{derived
  forms}\index{derived form} because their semantics can be
derived from that of other kinds of forms by a syntactic
transformation.  Some procedure definitions are also derived forms.  The
following two definitions are equivalent:

\begin{scheme}
(define (f x)
  (+ x 42))

(define f
  (lambda (x)
    (+ x 42)))%
\end{scheme}

In Scheme, it is possible for a program to create its own derived
forms by binding syntactic keywords to macros\index{macro}:

\begin{scheme}
(define-syntax def
  (syntax-rules ()
    ((def f (p ...) body)
     (define (f p ...)
       body))))

(def f (x)
  (+ x 42))%
\end{scheme}

The {\cf define-syntax} construct specifies that a parenthesized
structure matching the pattern {\cf (def f (p ...) body)}, where {\cf
  f}, {\cf p}, and {\cf body} are pattern variables, is translated to
{\cf (define (f p ...) body)}.  Thus, the {\cf def} form appearing in
the example gets translated to:

\begin{scheme}
(define (f x)
  (+ x 42))%
\end{scheme}

The ability to create new syntactic keywords makes Scheme extremely
flexible and expressive, allowing many of the features
built into other languages to be derived forms in Scheme.

\section{Syntactic data and datum values}

A subset of the Scheme objects is called \textit{datum
  values}\index{datum value}. 
These include booleans, number objects, characters, symbols,
and strings as well as lists and vectors whose elements are data.  Each
datum value may be represented in textual form as a
\textit{syntactic datum}\index{syntactic datum}, which can be written out
and read back in without loss of information.
A datum value may be represented by several different syntactic data.
Moreover, each datum value
can be trivially translated to a literal expression in a program by
prepending a {\cf\singlequote} to a corresponding syntactic datum:

\begin{scheme}
'23 \ev 23
'\schtrue{} \ev \schtrue{}
'foo \ev foo
'(1 2 3) \ev (1 2 3)
'\#(1 2 3) \ev \#(1 2 3)%
\end{scheme}

The {\cf\singlequote} shown in the previous examples
is not needed for representations of number objects or booleans.
The syntactic datum {\cf foo} represents a
symbol with name ``foo'', and {\cf 'foo} is a literal expression with
that symbol as its value.  {\cf (1 2 3)} is a syntactic datum that 
represents a list with elements 1, 2, and 3, and {\cf '(1 2 3)} is a literal
expression with this list as its value.  Likewise, {\cf \#(1 2 3)}
is a syntactic datum that represents a vector with elements 1, 2 and 3, and
{\cf '\#(1 2 3)} is the corresponding literal.

The syntactic data are a superset of the Scheme forms.  Thus, data
can be used to represent Scheme forms as data objects.  In
particular, symbols can be used to represent identifiers.

\begin{scheme}
'(+ 23 42) \ev (+ 23 42)
'(define (f x) (+ x 42)) \lev (define (f x) (+ x 42))%
\end{scheme}

This facilitates writing programs that operate on Scheme source code,
in particular interpreters and program transformers.

\section{Continuations}

Whenever a Scheme expression is evaluated there is a
\textit{continuation}\index{continuation} wanting the result of the
expression.  The continuation represents an entire (default) future
for the computation.  For example, informally the continuation of {\cf 3}
in the expression
%
\begin{scheme}
(+ 1 3)%
\end{scheme}
%
adds 1 to it.  Normally these ubiquitous continuations are hidden
behind the scenes and programmers do not think much about them.  On
rare occasions, however, a programmer may need to deal with
continuations explicitly.  The {\cf call-with-current-continuation}
procedure (see section~\ref{call-with-current-continuation}) allows
Scheme programmers to do that by creating a procedure that reinstates
the current continuation.  The {\cf call-with-current-continuation}
procedure accepts a procedure, calls it immediately with an argument
that is an \textit{escape procedure}\index{escape procedure}.  This
escape procedure can then be called with an argument that becomes the
result of the call to {\cf call-with-current-continuation}.  That is,
the escape procedure abandons its own continuation, and reinstates the
continuation of the call to {\cf call-with-current-continuation}.

In the following example, an escape procedure representing the
continuation that adds 1 to its argument is bound to {\cf escape}, and
then called with 3 as an argument.  The continuation of the call to
{\cf escape} is abandoned, and instead the 3 is passed to the
continuation that adds 1:
%
\begin{scheme}
(+ 1 (call-with-current-continuation
       (lambda (escape)
         (+ 2 (escape 3))))) \lev 4%
\end{scheme}
%
An escape procedure has unlimited extent: It can be called after the
continuation it captured has been invoked, and it can be called
multiple times.  This makes {\cf call-with-current-continuation}
significantly more powerful than typical non-local control constructs
such as exceptions in other languages.

\section{Libraries}
\label{librariesintrosection}

Scheme code can be organized in components called
\textit{libraries}\index{library}.  Each library contains 
definitions and expressions.  It can import definitions
from other libraries and export definitions to other libraries.

The following library called {\cf (hello)} exports a definition called
{\cf hello-world},  and imports the base library (see
chapter~\ref{baselibrarychapter}) and the simple I/O library (see
library section~\extref{lib:simpleiosection}{Simple I/O}).  The {\cf
  hello-world} export is a procedure that displays {\cf Hello World}
on a separate line:
%
\begin{scheme}
(library (hello)
  (export hello-world)
  (import (rnrs base)
          (rnrs io simple))
  (define (hello-world)
    (display "Hello World")
    (newline)))%
\end{scheme}

\section{Top-level programs}

A Scheme program is invoked via a \textit{top-level
  program}\index{top-level program}.  Like a library, a top-level
program contains imports, definitions and expressions, and specifies
an entry point for execution.  Thus a top-level program defines, via
the transitive closure of the libraries it imports, a Scheme program.

The following top-level program obtains the first argument from the command line
via the {\cf command-line} procedure from the \rsixlibrary{programs}
library (see library chapter~\extref{lib:programlibchapter}{Command-line
  access and exit values}).  It then opens the file using {\cf
  open-file-input-port} (see library section~\extref{lib:portsiosection}{Port I/O}),
yielding a \textit{port}, i.e.\ a connection to the file as a data
source, and calls the {\cf get-bytes-all} procedure to obtain the
contents of the file as binary data.  It then uses {\cf put-bytes} to
output the contents of the file to standard output:
%
\begin{scheme}
\#!r6rs
(import (rnrs base)
        (rnrs io ports)
        (rnrs programs))
(let ((p (standard-output-port)))
  (put-bytevector p
                  (call-with-port
                      (open-file-input-port
                        (cadr (command-line)))
                    get-bytevector-all))
  (close-port p))%
\end{scheme}

%%% Local Variables: 
%%% mode: latex
%%% TeX-master: "r6rs"
%%% End: 
  \par
\chapter{Numbers}
\label{numbertypeschapter}
\mainindex{number}

This chapter describes Scheme's representations for numbers.
It is important to distinguish between the mathematical numbers, the
Scheme numbers that attempt to model them, the machine representations
used to implement the Scheme numbers, and notations used to write numbers.
This report uses the types \type{number}, \type{complex}, \type{real},
\type{rational}, and \type{integer} to refer to both mathematical numbers
and Scheme numbers.
The \type{fixnum} and \type{flonum} types refer to certain
subtypes of the Scheme numbers, as explained below.

\section{Numerical types}
\label{numericaltypes}
\index{numerical types}

\vest Mathematically, numbers may be arranged into a tower of subtypes
in which each level is a subset of the level above it:
\begin{tabbing}
\ \ \ \ \ \ \ \ \ \=\tupe{number} \\
\> \tupe{complex} \\
\> \tupe{real} \\
\> \tupe{rational} \\
\> \tupe{integer} 
\end{tabbing}

For example, 5 is an integer.  Therefore 5 is also a rational,
a real, and a complex.  The same is true of the Scheme numbers
that model 5.  For Scheme numbers, these types are defined by the
predicates \ide{number?}, \ide{complex?}, \ide{real?}, \ide{rational?},
and \ide{integer?}.

There is no simple relationship between a number's type and its
representation inside a computer.  Although most implementations of
Scheme offer at least three different representations of 5, these
different representations denote the same integer.

Scheme's numerical operations treat numbers as abstract data, as
independent of their representation as possible.  Although an implementation
of Scheme may use many different representations for
numbers, this should not be apparent to a casual programmer writing
simple programs.

It is necessary, however, to distinguish between numbers that are
represented exactly and those that may not be.  For example, indexes
into data structures must be known exactly, as must some polynomial
coefficients in a symbolic algebra system.  On the other hand, the
results of measurements are inherently inexact, and irrational numbers
may be approximated by rational and therefore inexact approximations.
In order to catch uses of inexact numbers where exact numbers are
required, Scheme explicitly distinguishes exact from inexact numbers.
This distinction is orthogonal to the dimension of type.

A \defining{fixnum} is an exact integer whose value lies
within a certain implementation-dependent subrange of the
exact integers (section \ref{fixnumssection}).
Likewise, every implementation is required
to designate a subset of its inexact reals as \defining{flonum}s, and
to convert certain external representations into flonums.  Note that
this does not imply that an implementation is required to use
floating point representations.

\section{Exactness}
\label{exactly}

\mainindex{exactness} Scheme numbers are either \type{exact} or
\type{inexact}.  A number is exact if it is written as an exact
constant or was derived from exact numbers using only exact
operations.  A number is inexact if it is written as an inexact
constant or was derived from inexact numbers.  Thus inexactness is
contagious.  

Exact arithmetic is reliable in the following sense:
If exact numbers are passed to any of the arithmetic procedures
described in section~\ref{genericarithmeticsection}, and an
exact number is returned, then the result is mathematically
correct.
This is generally not true of
computations involving inexact numbers because approximate methods
such as floating point arithmetic may be used, but it is the duty of
each implementation to make the result as close as practical to the
mathematically ideal result.

\section{Implementation restrictions}

\index{implementation restriction}\label{restrictions}

\vest Implementations of Scheme are required to implement the whole
tower of subtypes given in section~\ref{numericaltypes}.

\vest Implementations are required to support
exact integers and exact rationals of
practically unlimited size and precision, and to implement
certain procedures (listed in \ref{propagationsection})
so they always return exact results when given exact
arguments.

\vest Implementations may also support only a limited range of
inexact numbers of
any type, subject to the requirements of this section.  For example,
an implementation may
limit the range of inexact reals (and therefore
the range of inexact integers and rationals)
to the dynamic range of the flonum format.
Furthermore
the gaps between the representable inexact integers and
rationals are
likely to be very large in such an implementation as the limits of this
range are approached.

\vest An implementation may use floating point and other approximate 
representation strategies for \tupe{inexact} numbers.
This report recommends, but does not require, that the IEEE 
floating point standards be followed by implementations that use
floating point representations, and that implementations using
other representations should match or exceed the precision achievable
using these floating point standards~\cite{IEEE}.

\vest In particular, implementations that use floating point
representations must follow these rules: A floating point result
must be represented with at least as much precision as is
used to express any of the inexact arguments to that operation.
It is desirable (but not required) for
potentially inexact operations such as {\cf sqrt}, when applied to exact
arguments, to produce exact answers whenever possible (for example the
square root of an exact 4 ought to be an exact 2).
If, however, an
exact number is operated upon so as to produce an inexact result
(as by {\cf sqrt}), and if the result is represented in floating
point, then the most precise floating point format available
must be used; but if the result
is represented in some other way then the representation must have
at least as much precision as the most precise
floating point format available.

It is the programmer's responsibility to avoid using inexact numbers
with magnitude or significand too large to be represented in the
implementation.

\section{Infinities and NaNs}

Positive infinity is regarded as a real (but not rational) number,
whose value is indeterminate but greater than all rational numbers.
Negative infinity is regarded as a real (but not rational) number,
whose value is indeterminate but less than all rational numbers.

A NaN is regarded as a real (but not rational) number whose value is
so indeterminate that it might represent any real number, including
positive or negative infinity, and might even be greater than positive
infinity or less than negative infinity.

%%% Local Variables: 
%%% mode: latex
%%% TeX-master: "r6rs"
%%% End: 
 \par
% Lexical structure
\hyphenation{white-space}
%%\vfill\eject
\chapter{Lexical syntax and read syntax}
\label{readsyntaxchapter}

The syntax of Scheme programs is organized in three levels:
%
\begin{enumerate}
\item the \textit{lexical syntax} that describes how a program text is split
  into a sequence of lexemes,
\item the \textit{read syntax}, formulated in terms of the lexical
  syntax, that structures the lexeme sequence as a sequence of
  \textit{syntactic datums\mainindex{datum}\mainindex{syntactic
      datum}}, where a syntactic datum is
    a recursively structured entity,
\item the \textit{program syntax} formulated in terms of the read
  syntax, imposing further structure and assigning meaning to
  syntactic datums.
\end{enumerate}
%
Syntactic datums (also called \textit{external
  representations\index{external representation}}) double
as a notation for data, and Scheme's \library{r6rs i/o ports} library
(section~\ref{portsiosection})
provides the {\cf get-datum} and {\cf put-datum} procedures
for reading and writing syntactic datums, converting between their
textual representation and the corresponding values. 
A
syntactic datum can be used in a program to obtain the corresponding
value using {\cf quote} (see section~\ref{quote}).

Moreover, valid Scheme expressions form a subset of the syntactic datums.
Consequently, Scheme's syntax has the property that any sequence of
characters that is an expression is also a syntactic datum representing
some object.  This can lead to confusion, since it may not be obvious
out of context whether a given sequence of characters is intended to
denote data or program. It is also a source of power, since it
facilitates writing programs such as interpreters and compilers that
treat programs as data (or vice versa).
A syntactic datum occurring in program text is often called a \defining{form}.

Note that several syntactic datums may represent the same object, a
so-called \defining{datum value}.
For example, both``{\tt \#e28.000}'' and
``{\tt\#x1c}'' are syntactic datums representing the exact integer 28;
The syntactic datums ``{\tt(8 13)}'', ``{\tt( 08 13 )}'', ``{\tt(8 .\
  (13 .\ ()))}'' (and more)
all represent a list containing the integers 8 and 13. 
Syntactic datums that denote equal objects are always equivalent 
as forms of a program.

Because of the close correspondence between syntactic datums and datum
values, this report sometimes uses the term \defining{datum} to denote
either a syntactic datum or a datum value when the exact meaning
is apparent from the context.

An implementation is not permitted to extend the lexical or read syntax in
any way, with one exception: it need not treat the syntax
{\cf \sharpsign{}!\meta{identifier}}, for any \meta{identifier} (see
section~\ref{identifiersection}) that is not {\cf r6rs}, as a syntax
violation, and it may use specific {\cf \sharpsign{}!}-prefixed
identifiers as flags indicating that subsequent input contains extensions
to the standard lexical syntax. 
(The comment syntax {\cf \sharpsign{}!r6rs} may be used to signify that
the input which follows is written purely in the language described by
this report; see section~\ref{whitespaceandcomments}.)

This chapter overviews and provides formal accounts of the lexical
syntax and the read syntax.

\section{Notation}
\label{BNF}

The formal syntax for Scheme is written in an extended BNF.
Non-terminals are written using angle brackets; case is insignificant
for non-terminal names.

All spaces in the grammar are for legibility.
\meta{Empty} stands for the empty string.

The following extensions to BNF are used to make the description more
concise:  \arbno{\meta{thing}} means zero or more occurrences of
\meta{thing}; and \atleastone{\meta{thing}} means at least one
\meta{thing}.

Some non-terminal names refer to the Unicode scalar values of the
same name: \meta{character tabulation} (U+0009), \meta{linefeed}
(U+000A), \meta{line tabulation} (U+000B), \meta{form feed} (U+000C),
\meta{carriage return} (U+000D), and \meta{space} (U+0020).

\section{Lexical syntax}
\label{lexicalsyntaxsection}

The lexical syntax describes how a character sequence is split into a
sequence of lexemes\index{lexeme}, omitting non-significant portions
such as comments and whitespace.  The character sequence is assumed to
be text according to the Unicode standard~\cite{Unicode}.  Some of
the lexemes, such as numbers, identifiers, strings etc.\ of the lexical
syntax are syntactic datums in the read syntax, and thus represent data.
Besides the formal account of the syntax, this section also describes
what datum values are denoted by these syntactic datums.

Note that the lexical syntax, in the description of comments, contains
a forward reference to \meta{datum}, which is described as part of the
read syntax.  However, being comments, these \meta{datum}s do not play
a significant role in the syntax.

Case is significant except in boolean datums, number datums, and
hexadecimal numbers denoting Unicode scalar values.  For example, {\cf \#x1A}
and {\cf \#X1a} are equivalent.  The identifier {\cf Foo} is, however,
distinct from the identifier {\cf FOO}.

\subsection{Formal account}
\label{lexicalgrammarsection}

\meta{Interlexeme space} may occur on either side of any lexeme, but not
within a lexeme.

\vest Lexemes that require implicit termination (identifiers, numbers,
characters, booleans, and dot) are terminated by any \meta{delimiter}
or by the end of the input, but not necessarily by anything else.

The following two characters are reserved for future extensions to the
language: {\tt \verb"{" \verb"}"}

\begin{grammar}%
\meta{lexeme} \: \meta{identifier} \| \meta{boolean} \| \meta{number}\index{identifier}
\>  \| \meta{character} \| \meta{string}
\>  \| ( \| ) \| \openbracket{} \| \closedbracket{} \| \sharpsign( \| \singlequote{} \| \backquote{} \| , \| ,@ \| {\bf.}
\meta{delimiter} \: \meta{whitespace} \| ( \| ) \| \openbracket{} \| \closedbracket{} \| " \| ;
\meta{whitespace} \: \meta{character tabulation} \| \meta{linefeed}
\> \| \meta{line tabulation} \| \meta{form feed} \meta{carriage return}
\> \| \meta{any character whose category is Zs, Zl, or Zp}
\meta{intra-line whitespace} \: $\langle${\rm any \meta{whitespace}}
\> \quad {\rm that is not \meta{linefeed}}$\rangle$
\meta{comment} \: ; \= $\langle$\rm all subsequent characters up to a
                    \>\ \rm linefeed$\rangle$\index{comment}
\qquad \= \| \meta{nested comment}
\> \| \#; \meta{datum}
\> \| \#!r6rs
\meta{nested comment} \: \#| \= \meta{comment text}
\> \arbno{\meta{comment cont}} |\#
\meta{comment text} \: \= $\langle$\rm character sequence not containing
\>\ \rm {\tt \#|} or {\tt |\#}$\rangle$
\meta{comment cont} \: \meta{nested comment} \meta{comment text}
\meta{atmosphere} \: \meta{whitespace} \| \meta{comment}
\meta{interlexeme space} \: \arbno{\meta{atmosphere}}%
\end{grammar}

\label{extendedalphas}
\label{identifiersyntax}

% This is a kludge, but \multicolumn doesn't work in tabbing environments.
\setbox0\hbox{\cf\meta{variable} \goesto{} $\langle$}

\begin{grammar}%
\meta{identifier} \: \meta{initial} \arbno{\meta{subsequent}}
 \>  \| \meta{peculiar identifier}
\meta{initial} \: \meta{constituent} \| \meta{special initial}
 \> \| \meta{inline hex escape}
\meta{letter} \:  a \| b \| c \| ... \| z
\> \| A \| B \| C \| ... \| Z
\meta{constituent} \: \meta{letter}
 \> \| $\langle${\rm any character whose Unicode scalar value is greater than}
 \> \quad {\rm 127, and whose category is Lu, Ll, Lt, Lm, Lo, Mn, Mc,}
 \> \quad {\rm Me, Nd, Nl, No, Pd, Pc, Po, Sc, Sm, Sk, So, or Co}$\rangle$

\meta{special initial} \: ! \| \$ \| \% \| \verb"&" \| * \| / \| : \| < \| =
 \>  \| > \| ? \| \verb"^" \| \verb"_" \| \verb"~"
\meta{subsequent} \: \meta{initial} \| \meta{digit}
 \>  \| \meta{special subsequent}
\meta{digit} \: 0 \| 1 \| 2 \| 3 \| 4 \| 5 \| 6 \| 7 \| 8 \| 9
\meta{hex digit} \: \meta{digit}
 \> \| a \| A \| b \| B \| c \| C \| d \| D \| e \| E \| f \| F
\meta{special subsequent} \: + \| - \| .\ \| @
\meta{inline hex escape} \: \backwhack{}x\meta{hex scalar value};
\meta{hex scalar value} \: \atleastone{\meta{hex digit}}
 \> {\rm with at most 8 digits}
 \meta{peculiar identifier} \: + \| - \| ... \| -> \arbno{\meta{subsequent}}
%\| 1+ \| -1+
\meta{boolean} \: \schtrue{} \| \#T \| \schfalse{} \| \#F
\meta{character} \: \#\backwhack{}\meta{any character}
 \>  \| \#\backwhack{}\meta{character name}
 \>  \| \#\backwhack{}x\meta{hex scalar value}
\meta{character name} \: nul \| alarm \| backspace \| tab
\> \| linefeed \| vtab \| page \| return \| esc
\> \| space \| delete
\todo{Explain what happens in the ambiguous case.}
\meta{string} \: " \arbno{\meta{string element}} "
\meta{string element} \: \meta{any character other than \doublequote{} or \backwhack}
 \> \| \backwhack{}a \| \backwhack{}b \| \backwhack{}t \| \backwhack{}n \| \backwhack{}v \| \backwhack{}f \| \backwhack{}r
 \>  \| \backwhack\doublequote{} \| \backwhack\backwhack 
 \>  \| \backwhack\meta{linefeed} \| \backwhack\meta{space}
 \>  \| \meta{inline hex escape}
\end{grammar}


\label{numbersyntax}

\begin{grammar}%
\meta{number} \: \meta{num $2$} \| \meta{num $8$}
   \>  \| \meta{num $10$} \| \meta{num $16$}
\end{grammar}

The following rules for \meta{num $R$}, \meta{complex $R$}, \meta{real
$R$}, \meta{ureal $R$}, \meta{uinteger $R$}, and \meta{prefix $R$}
should be replicated for \hbox{$R = 2, 8, 10,$}
and $16$.  There are no rules for \meta{decimal $2$}, \meta{decimal
$8$}, and \meta{decimal $16$}, which means that numbers containing
decimal points or exponents must be in decimal radix.
\todo{Mark Meyer and David Bartley want to fix this.  (What? -- Will)}

\begin{grammar}%
\meta{num $R$} \: \meta{prefix $R$} \meta{complex $R$}
\meta{complex $R$} \: %
         \meta{real $R$} %
      \| \meta{real $R$} @ \meta{real $R$}
   \> \| \meta{real $R$} + \meta{ureal $R$} i %
      \| \meta{real $R$} - \meta{ureal $R$} i
   \> \| \meta{real $R$} + i %
      \| \meta{real $R$} - i
   \> \| + \meta{ureal $R$} i %
      \| - \meta{ureal $R$} i %
      \| + i %
      \| - i
\meta{real $R$} \: \meta{sign} \meta{ureal $R$}
\meta{ureal $R$} \: %
         \meta{uinteger $R$}
   \> \| \meta{uinteger $R$} / \meta{uinteger $R$}
   \> \| \meta{decimal $R$} \meta{mantissa width}
   \> \| inf.0 \| nan.0
\meta{decimal $10$} \: %
         \meta{uinteger $10$} \meta{suffix}
   \> \| . \atleastone{\meta{digit $10$}} \arbno{\#} \meta{suffix}
   \> \| \atleastone{\meta{digit $10$}} . \arbno{\meta{digit $10$}} \arbno{\#} \meta{suffix}
   \> \| \atleastone{\meta{digit $10$}} \atleastone{\#} . \arbno{\#} \meta{suffix}
\meta{uinteger $R$} \: \atleastone{\meta{digit $R$}} \arbno{\#}
\meta{prefix $R$} \: %
         \meta{radix $R$} \meta{exactness}
   \> \| \meta{exactness} \meta{radix $R$}
\end{grammar}

\begin{grammar}%
\meta{suffix} \: \meta{empty} 
   \> \| \meta{exponent marker} \meta{sign} \atleastone{\meta{digit $10$}}
\meta{exponent marker} \: e \| E \| s \| S \| f \| F
   \> \| d \| D \| l \| L
\meta{mantissa width} \: \meta{empty}
   \> \| | \atleastone{\meta{digit 10}}
\meta{sign} \: \meta{empty}  \| + \|  -
\meta{exactness} \: \meta{empty}
   \> \| \#i\sharpindex{i} \| \#I \| \#e\sharpindex{e} \| \#E
\meta{radix 2} \: \#b\sharpindex{b} \| \#B
\meta{radix 8} \: \#o\sharpindex{o} \| \#O
\meta{radix 10} \: \meta{empty} \| \#d \| \#D
\meta{radix 16} \: \#x\sharpindex{x} \| \#X
\meta{digit 2} \: 0 \| 1
\meta{digit 8} \: 0 \| 1 \| 2 \| 3 \| 4 \| 5 \| 6 \| 7
\meta{digit 10} \: \meta{digit}
\meta{digit 16} \: \meta{hex digit}
\end{grammar}

\todo{Mark Meyer of TI sez, shouldn't we allow {\tt 1e3/2}?}

\subsection{Whitespace and comments}
\label{whitespaceandcomments}

\defining{Whitespace} characters are spaces, linefeeds,
carriage returns, character tabulations, form feeds, line tabulations,
and any other character whose category is Zs, Zl, or Zp.
Whitespace is used for improved readability and
as necessary to separate lexemes from each other.  Whitespace may
occur between any two lexemes,
but not within a lexeme.  Whitespace may also occur inside a string,
where it is significant.

The lexical syntax includes several comment forms. In all cases,
comments are invisible to Scheme, except that they act as delimiters,
so a comment cannot appear in the middle of an identifier or number.

A semicolon ({\tt;}) indicates the start of a line
comment.\mainindex{comment}\mainschindex{;} The comment continues to
the end of the line on which the semicolon appears (i.e., it is
terminated by a linefeed character).

Another way to indicate a comment is to prefix a \hyper{datum}
(cf.\ Section~\ref{datumsyntax}) with {\tt \#;}, possibly with
whitespace before the \hyper{datum}.  The comment consists of
the comment prefix {\tt \#;} and the \hyper{datum} together.  (This
notation is useful for ``commenting out'' sections of code.)

Block comments may be indicated with properly nested {\tt
  \#|} and {\tt |\#} pairs.

\begin{scheme}
\#|
   The FACT procedure computes the factorial
   of a non-negative integer.
|\#
(define fact
  (lambda (n)
    ;; base case
    (if (= n 0)
        \#;(= n 1)
        1       ; identity of *
        (* n (fact (- n 1))))))%
\end{scheme}

The lexeme {\cf \sharpsign{}!r6rs} is also a comment.  When it occurs in
program text, it signifies that the program text is written purely in the
language described by this report (see
section~\ref{librarysyntaxsection}).

\subsection{Identifiers}
\label{identifiersection}

Most identifiers\mainindex{identifier} allowed by other programming
languages are also acceptable to Scheme.  In particular,
a sequence of letters, digits, and ``extended alphabetic
characters'' that begins with a character that cannot begin a number is
an identifier.  In addition, \ide{+}, \ide{-}, and \ide{...} are identifiers. 
Here are some examples of identifiers:

\begin{scheme}
lambda                   q
list->vector             soup
{+}                        V17a
<=?                      a34kTMNs
the-word-recursion-has-many-meanings%
\end{scheme}

Extended alphabetic characters may be used within identifiers as if
they were letters.  The following are extended alphabetic characters:

\begin{scheme}
!\ \$ \% \verb"&" * + - . / :\ < = > ? @ \verb"^" \verb"_" \verb"~" %
\end{scheme}

Moreover, all characters whose Unicode scalar values are greater than 127 and
whose Unicode category is Lu, Lt, Lm, Lo, Mn, Mc, Me, Nd, Nl, No, Pd,
Pc, Po, Sc, Sm, Sk, So, or Co can be used within identifiers.
Moreover, any character can appear as the constituent of an identifier
when denoted via a hexadecimal escape sequence.  For example, the
identifier \verb|H\x65;llo| is the same as the identifier
\verb|Hello|, and the identifier \verb|\x3BB;| is the same as the
identifier $\lambda$.

Any identifier may be used as a variable\index{variable} or as a
syntactic keyword\index{syntactic keyword} (see
sections~\ref{variablesection} and~\ref{macrosection}) in a Scheme
program.

Moreover, when viewed as a datum value, an identifier denotes a \textit{symbol}\index{symbol}
(see section~\ref{symbolsection}).

\subsection{Booleans}

The standard boolean objects for true and false are written as
\schtrue{} and \schfalse.\sharpindex{t}\sharpindex{f}  The character
after a boolean literal must be a delimiter character, such as a
space or parenthesis.

\subsection{Characters}

Characters are written using the notation
\sharpsign\backwhack\hyper{character} or
\sharpsign\backwhack\hyper{character name} or
\sharpsign\backwhack{}x\atleastone{\hyper{digit 16}}, where the last
specifies the Unicode scalar value of a character with a hexadecimal number of
no more than eight digits.

For example:

\begin{schemenoindent}
\#\backwhack{}a          \ev \textrm{lower case letter a}
\#\backwhack{}A          \ev \textrm{upper case letter A}
\#\backwhack{}(          \ev \textrm{left parenthesis}
\#\backwhack{}           \ev \textrm{space character}
\#\backwhack{}nul        \ev \textrm{U+0000}
\#\backwhack{}alarm      \ev \textrm{U+0007}
\#\backwhack{}backspace  \ev \textrm{U+0008}
\#\backwhack{}tab        \ev \textrm{U+0009}
\#\backwhack{}linefeed   \ev \textrm{U+000A}
\#\backwhack{}vtab       \ev \textrm{U+000B}
\#\backwhack{}page       \ev \textrm{U+000C}
\#\backwhack{}return     \ev \textrm{U+000D}
\#\backwhack{}esc        \ev \textrm{U+001B}
\#\backwhack{}space      \ev \textrm{U+0020}
\>\>; \textrm{preferred way to write a space}
\#\backwhack{}delete     \ev \textrm{U+007F}

\#\backwhack{}xFF        \ev \textrm{U+00FF}
\#\backwhack{}x03BB      \ev \textrm{U+03BB}
\#\backwhack{}x00006587  \ev \textrm{U+6587}
\#\backwhack{}\(\lambda\) \ev \textrm{U+03BB}

\#\backwhack{}x0001z     \ev \exception{\&lexical}
\#\backwhack{}\(\lambda\)x         \ev \exception{\&lexical}
\#\backwhack{}alarmx     \ev \exception{\&lexical}
\#\backwhack{}alarm x    \ev \textrm{U+0007}
\>\>; \textrm{followed by {\cf x}}
\#\backwhack{}Alarm      \ev \exception{\&lexical}
\#\backwhack{}alert      \ev \exception{\&lexical}
\#\backwhack{}xA         \ev \textrm{U+000A}
\#\backwhack{}xFF        \ev \textrm{U+00FF}
\#\backwhack{}xff        \ev \textrm{U+00FF}
\#\backwhack{}x ff       \ev \textrm{U+0078}
\>\>; \textrm{followed by another datum, {\cf ff}}
\#\backwhack{}x(ff)      \ev \textrm{U+0078}
\>\>; \textrm{followed by another datum,}
\>\>; \textrm{a parenthesized {\cf ff}}
\#\backwhack{}(x)        \ev \exception{\&lexical}
\#\backwhack{}(x         \ev \exception{\&lexical}
\#\backwhack{}((x)       \ev \textrm{U+0028}
\>\>; \textrm{followed by another datum,}
\>\>; \texttt{parenthesized {\cf x}}
\#\backwhack{}x00110000  \ev \exception{\&lexical}
\>\>; \textrm{out of range}
\#\backwhack{}x000000001 \ev \exception{\&lexical}
\>\>; \textrm{too many digits}  
\#\backwhack{}xD800      \ev \exception{\&lexical}
\>\>; \textrm{in excluded range}
\end{schemenoindent}

(The notation \exception{\&lexical} means that the line in question is
a lexical syntax violation.)

Case is significant in \sharpsign\backwhack\hyper{character}, and in in
\sharpsign\backwhack{\rm$\langle$character name$\rangle$}, % \hyper doesn't allow a linebreak
but not in \sharpsign\backwhack{}x\atleastone{\hyper{digit 16}}.  
The character after a \meta{character}
must be a delimiter character such as a
space or parenthesis.  This rule resolves various ambiguous cases, for
example, the sequence of characters ``{\tt\sharpsign\backwhack space}''
could be taken to be either a representation of the space character or a
representation of the character ``{\tt\sharpsign\backwhack s}'' followed
by a representation of the symbol ``{\tt pace}.''

\subsection{Strings}

\vest String are written as sequences of characters enclosed within doublequotes
({\cf "}).  Within a string literal, various escape
sequences\mainindex{escape sequence} denote characters other than
themselves.  Escape sequences always start with a backslash (\backwhack{}):

\begin{itemize}
\item{\tt \backwhack{}a} : alarm, U+0007
\item{\tt \backwhack{}b} : backspace, U+0008 
\item{\tt \backwhack{}t} : character tabulation, U+0009 
\item{\tt \backwhack{}n} : linefeed, U+000A 
\item{\tt \backwhack{}v} : line tabulation, U+000B 
\item{\tt \backwhack{}f} : formfeed, U+000C 
\item{\tt \backwhack{}r} : return, U+000D 
\item{\tt \backwhack{}}\verb|"| : doublequote, U+0022 
\item{\tt \backwhack{}\backwhack{}} : backslash, U+005C 
\item{\tt \backwhack{}\hyper{linefeed}\hyper{intraline whitespace}} : nothing
\item{\tt \backwhack{}\hyper{space}} : space, U+0020 (useful for terminating the
  previous escape sequence before continuing with whitespace)
\item{\tt \backwhack{}x\atleastone{\hyper{digit 16}};} : (note the
  terminating semi-colon) where no more than eight \hyper{digit 16}s
  are provided, and the sequence of \hyper{digit 16}s forms a
  hexadecimal number between 0 and \sharpsign{}x10FFFF excluding the
  range $\left[\sharpsign{}x\textrm{D800},
    \sharpsign{}x\textrm{DFFF}\right]$.
\end{itemize}

These escape sequences are case-sensitive, except that \hyper{digit
  16} can be an uppercase or lowercase hexadecimal digit.

Any other character in a string after a backslash is an error. Any
character outside of an escape sequence and not a doublequote stands
for itself in the string literal. For example the single-character
string {\tt "$\lambda$"} (double quote, a lower case lambda, double
quote) denotes the same string literal as {\tt "\backwhack{}x03bb;"}.

Examples:

\begin{schemenoindent}
"abc" \ev  \textrm{U+0061, U+0062, U+0063}
"\backwhack{}x41;bc" \ev  "Abc" ; \textrm{U+0041, U+0062, U+0063}
"\backwhack{}x41; bc" \ev "A bc"
\>\>; \textrm{U+0041, U+0020, U+0062, U+0063}
"\backwhack{}x41bc;" \ev  \textrm{U+41BC}
"\backwhack{}x41" \ev \exception{\&lexical}
"\backwhack{}x;" \ev \exception{\&lexical}
"\backwhack{}x41bx;" \ev \exception{\&lexical}
"\backwhack{}x00000041;" \ev  "A" ; \textrm{U+0041}
"\backwhack{}x0010FFFF;" \ev \textrm{U+10FFFF}
"\backwhack{}x00110000;" \ev  \exception{\&lexical}
\>\>; \textrm{out of range}
"\backwhack{}x000000001;" \ev \exception{\&lexical}
\>\>; \textrm{too many digits}
"\backwhack{}xD800;" \ev \exception{\&lexical}
\>\>; \textrm{in excluded range}
\end{schemenoindent}
  
\subsection{Numbers}
\label{numbernotations}

The syntax of written representations for numbers is described
formally by the \meta{number} rule in the formal grammar.  Note that
case is not significant in numerical constants.

A number may be written in binary, octal, decimal, or
hexadecimal by the use of a radix prefix.  The radix prefixes are {\cf
\#b}\sharpindex{b} (binary), {\cf \#o}\sharpindex{o} (octal), {\cf
\#d}\sharpindex{d} (decimal), and {\cf \#x}\sharpindex{x} (hexadecimal).  With
no radix prefix, a number is assumed to be expressed in decimal.

A
numerical constant may be specified to be either exact or
inexact by a prefix.  The prefixes are {\cf \#e}\sharpindex{e}
for exact, and {\cf \#i}\sharpindex{i} for inexact.  An exactness
prefix may appear before or after any radix prefix that is used.  If
the written representation of a number has no exactness prefix, the
constant may be either inexact or exact.  It is
inexact if it contains a decimal point, an
exponent, or a ``\sharpsign'' character in the place of a digit;
otherwise it is exact.

In systems with inexact numbers
of varying precisions, it may be useful to specify
the precision of a constant.  For this purpose, numerical constants
may be written with an exponent marker that indicates the
desired precision of the inexact
representation.  The letters {\cf s}, {\cf f},
{\cf d}, and {\cf l} specify the use of \var{short}, \var{single},
\var{double}, and \var{long} precision, respectively.  (When fewer
than four internal
inexact
representations exist, the four size
specifications are mapped onto those available.  For example, an
implementation with two internal representations may map short and
single together and long and double together.)  In addition, the
exponent marker {\cf e} specifies the default precision for the
implementation.  The default precision has at least as much precision
as \var{double}, but
implementations may wish to allow this default to be set by the user.

\begin{scheme}
3.14159265358979F0
       {\rm Round to single ---} 3.141593
0.6L0
       {\rm Extend to long ---} .600000000000000%
\end{scheme}

If \var{x} is an external representation of an inexact real number
that contains no vertical bar,
and \var{p} is a sequence of 1 or more decimal
digits, then {\cf \var{x}|\var{p}} is an external representation that
denotes the best binary floating point approximation to \var{x} using
a \var{p}-bit significand.  For example, {\cf 1.1|53} is an external
representation for the best approximation to 1.1 in IEEE double
precision.

If \var{x} is an external representation of an inexact real number
that contains no vertical bar,
then \var{x} by itself should be regarded as
equivalent to {\cf \var{x}|53}.

Implementations that use binary floating point representations
of real numbers should represent {\cf \var{x}|\var{p}}
using a \var{p}-bit significand if practical, or by a greater
precision if a \var{p}-bit significand is not practical, or
by the largest available precision if \var{p} or more bits
of significand are not practical within the implementation.

\begin{note}
The precision of a significand should not be confused with the
number of bits used to represent the significand.  In the IEEE
floating point standards, for example, the significand's most
significant bit is implicit in single and double precision but
is explicit in extended precision.  Whether that bit is implicit
or explicit does not affect the mathematical precision.
In implementations that use binary floating point, the default
precision can be calculated by calling the following procedure:

\begin{scheme}
(define (precision)
  (do ((n 0 (+ n 1))
       (x 1.0 (/ x 2.0)))
    ((= 1.0 (+ 1.0 x)) n)))
\end{scheme}
\end{note}      

\begin{note}
When the underlying floating-point representation is IEEE double
precision, the {\cf |\var{p}} suffix should not always be omitted:
Denormalized numbers have diminished precision, and therefore should
carry a {\cf |\var{p}} suffix with the actual width of the
significand.
\end{note}

The literals {\cf +inf.0} and {\cf -inf.0} represent positive and
negative infinity, respectively.  The {\cf +nan.0}
literal represents the NaN that is the result of {\cf (/ 0.0 0.0)},
and may represent other NaNs as well.

If a \meta{decimal 10} contains no vertical bar
and does not contain one of the exponent markers {\cf s},
{\cf f}, {\cf d}, or {\cf l}, but does contain a decimal point or the
exponent marker {\cf e}, then it is an external representation for a
flonum.  Furthermore {\cf inf.0}, {\cf +inf.0}, {\cf -inf.0}, {\cf
  nan.0}, {\cf +nan.0}, and {\cf -nan.0} are external representations
for flonums.  Some or all of the other external representations for
inexact reals may also represent flonums, but that is not required by
this report.

If a \meta{decimal 10} contains a non-empty \meta{mantissa width} or
one of the exponent markers {\cf s}, {\cf f}, {\cf d}, or {\cf l},
then it represents an inexact number, but does not necessarily
represent a flonum.

\section{Read syntax}
\label{readsyntaxsection}

The read syntax describes the syntax of
syntactic datums\mainindex{syntactic datum} in terms of a sequence of
\meta{lexeme}s, as defined in the lexical syntax.

Syntactic datums include the lexeme datums described in the
previous section as well as the following constructs for forming
compound structure:
%
\begin{itemize}
\item pairs and lists, enclosed by \verb|( )| or \verb|[ ]| (see
  section~\ref{pairlistsyntax})
\item  vectors (see section~\ref{vectorsyntax})
\end{itemize}

Note that the sequence of characters ``{\tt(+ 2 6)}'' is {\em not} a
syntactic datum representing the integer 8, even though it {\em is} a
base-library expression evaluating to the integer 8; rather, it is a
datum representing a three-element list, the elements of which
are the symbol {\tt +} and the integers 2 and 6.

\subsection{Formal account}
\label{datumsyntax}

The following grammar describes the syntax of syntactic datums in terms
of various kinds of lexemes defined in the grammar in
section~\ref{lexicalsyntaxsection}:

\begin{grammar}%
\meta{datum} \: \meta{simple datum}
\>  \| \meta{compound datum}
\meta{simple datum} \: \meta{boolean} \| \meta{number}
\>  \| \meta{character} \| \meta{string} \|  \meta{symbol}
\meta{symbol} \: \meta{identifier}
\meta{compound datum} \: \meta{list} \| \meta{vector}
\meta{list} \: (\arbno{\meta{datum}})
\>    \| [\arbno{\meta{datum}}]
\>    \| (\atleastone{\meta{datum}} .\ \meta{datum})
\>    \| [\atleastone{\meta{datum}} .\ \meta{datum}]
\>    \| \meta{abbreviation}
\meta{abbreviation} \: \meta{abbrev prefix} \meta{datum}
\meta{abbrev prefix} \: ' \| ` \| , \| ,@ \| \#' | \#` | \#, | \#,@
\meta{vector} \: \#(\arbno{\meta{datum}})
\meta{bytes} \: \#vu8(\arbno{\meta{u8}})
\meta{u8} \: $\langle${\rm any \meta{number} denoting an exact}
 \>\>\quad\quad {\rm integer in $\{0, \ldots, 255\}$}$\rangle$%
\end{grammar}

\subsection{Vectors}
\label{vectorsyntax}

Vector datums, denoting vectors of values (see
section~\ref{vectorsection}, are written using the notation
{\tt\#(\hyper{datum} \dotsfoo)}.  For example, a vector of length 3
containing the number zero in element 0, the list {\cf(2 2 2 2)} in
element 1, and the string {\cf "Anna"} in element 2 can be written as
following:

\begin{scheme}
\#(0 (2 2 2 2) "Anna")%
\end{scheme}

Note that this is the external representation of a vector,
and is not a
base-library expression that evaluates to a vector.

\subsection{Pairs and lists}
\label{pairlistsyntax}

List and pair datums, denoting pairs and lists of values
(see section~\ref{listsection}) are written using parentheses or brackets.
Matching pairs of parentheses that occur in the rules of \meta{list} are
equivalent to matching pairs of brackets.

The most general notation for Scheme pairs as syntactic datums is
the ``dotted'' notation \hbox{\cf (\hyperi{datum} .\ \hyperii{datum})} where
\hyperi{datum} is the representation of the value of the car field and
\hyperii{datum} is the representation of the value of the
cdr field.  For example {\cf (4 .\ 5)} is a pair whose car is 4 and whose
cdr is 5.  Note that {\cf (4 .\ 5)} is the external representation of a
pair, not an expression that evaluates to a pair.

A more streamlined notation can be used for lists: the elements of the
list are simply enclosed in parentheses and separated by spaces.  The
empty list\index{empty list} is written {\tt()} .  For example,

\begin{scheme}
(a b c d e)%
\end{scheme}

and

\begin{scheme}
(a . (b . (c . (d . (e . ())))))%
\end{scheme}

are equivalent notations for a list of symbols.

The general rule is that, if a dot is followed by an open parenthesis,
the dot, the open parenthesis, and the matching closing parenthesis
can be omitted in the external representation.

\subsection{Bytes objects}

Bytes datums, denoting bytes objects (see
section~\ref{byteschapter}), are written using the notation
{\tt\#vu8(\hyper{u8} \dotsfoo)}, where the \hyper{u8}s represent the octets of
the bytes object.  For example, a bytes object of length 3 containing the
octets 2, 24, and 123 can be written as follows:

\begin{scheme}
\#vu8(2 24 123)%
\end{scheme}

Note that this is the external representation of a bytes object,
and is not an
expression that evaluates to a bytes object.

\subsection{Abbreviations}\unsection
\label{quotesection}

\begin{entry}{%
\pproto{\singlequote\hyper{datum}}{}
\pproto{\backquote\hyper{datum}}{}
\pproto{,\hyper{datum}}{}
\pproto{,\atsign\hyper{datum}}{}
\pproto{\#'\hyper{datum}}{}
\pproto{\#\backquote\hyper{datum}}{}
\pproto{\#,\hyper{datum}}{}
\pproto{\#,@\hyper{datum}}{}
}

Each of these is an abbreviation:
\\\quad\mainschindex{'}\singlequote\hyper{datum}
for {\cf (quote \hyper{datum})},
\\\quad\mainschindex{`}\backquote\hyper{datum}
for {\cf (quasiquote \hyper{datum})},
\\\quad\mainschindex{,}{\cf,}\hyper{datum}
for {\cf (unquote \hyper{datum})},
\\\quad\mainschindex{,@}{\cf,}\atsign\hyper{datum}
for {\cf (unquote-splicing \hyper{datum})},
\\\quad\sharpindex{'}{\cf\#'}\hyper{datum}
for {\cf (syntax \hyper{datum})},
\\\quad\sharpindex{`}{\cf\#`}\hyper{datum}
for {\cf (quasisyntax \hyper{datum})},
\\\quad\sharpindex{,}{\cf\#,}\hyper{datum}
for {\cf (unsyntax \hyper{datum})}, and
\\\quad\sharpindex{,}{\cf\#,@}\hyper{datum}
for {\cf (unsyntax-splicing \hyper{datum})}.
\end{entry}

% was:
% \mainschindex{'}\singlequote\hyper{datum} is an abbreviation
% for {\cf (quote \hyper{datum})}.
% \mainschindex{`}\backquote\hyper{datum} is an abbreviation
% for {\cf (quasiquote \hyper{datum})}.
% \mainschindex{,}{\cf,}\hyper{datum} is an abbreviation
% for {\cf (unquote \hyper{datum})}.
% \mainschindex{,@}{\cf,}\atsign\hyper{datum} is an abbreviation
% for {\cf (unquote-splicing \hyper{datum})}.
% \sharpindex{'}{\cf\#'}\hyper{datum} is an abbreviation
% for {\cf (syntax \hyper{datum})}.
% \sharpindex{`}{\cf\#`}\hyper{datum} is an abbreviation
% for {\cf (quasisyntax \hyper{datum})}.
% \sharpindex{,}{\cf\#,}\hyper{datum} is an abbreviation
% for {\cf (unsyntax \hyper{datum})}.
% \sharpindex{,}{\cf\#,@}\hyper{datum} is an abbreviation
% for {\cf (unsyntax-splicing \hyper{datum})}.

%%% Local Variables: 
%%% mode: latex
%%% TeX-master: "r6rs"
%%% End: 
     \par
%\vfill\eject
\chapter{Basic concepts and terminology}
\label{basicchapter}

\section{Entry format}

The chapters describing bindings in the base library and the standard
libraries are organized
into entries.  Each entry describes one language feature or a group of
related features, where a feature is either a syntactic construct or a
built-in procedure.  An entry begins with one or more header lines of the form

\noindent\pproto{\var{template}}{\var{category}}\unpenalty

If \var{category} is ``\exprtype'', the entry describes an expression
type, and the template gives the syntax of the expression type.  Even
though the template is written in a notation similar to a right-hand
side of the BNF rules in chapter~\ref{readsyntaxchapter}, it describes
the set of S-expressions equivalent to the S-expressions matching the
template.

Components of the S-expressions described by a template are designated
by syntactic variables, which are written using angle brackets, for
example, \hyper{expression}, \hyper{variable}.  Syntactic variables
should be understood to denote other S-expressions, or, in some cases,
sequences of them.  A syntactic variable may refer to a non-terminal
in the S-expression grammar, in which case only S-expressions matching
that non-terminal are permissible in that position.  For example,
\hyper{expression} stands for any S-expression which is a
syntactically valid expression.  Other non-terminals that are used in
templates will be defined as part of the specification.

The notation
\begin{tabbing}
\qquad \hyperi{thing} $\ldots$
\end{tabbing}
indicates zero or more occurrences of a \hyper{thing}, and
\begin{tabbing}
\qquad \hyperi{thing} \hyperii{thing} $\ldots$
\end{tabbing}
indicates one or more occurrences of a \hyper{thing}.

If \var{category} is ``procedure'', then the entry describes a procedure, and
the header line gives a template for a call to the procedure.  Parameter
names in the template are \var{italicized}.  Thus the header line

\noindent\pproto{(vector-ref \var{vector} \var{k})}{procedure}\unpenalty

indicates that the built-in procedure {\tt vector-ref} takes
two arguments, a vector \var{vector} and an exact non-negative integer
\var{k} (see below).  The header lines

\noindent%
\pproto{(make-vector \var{k})}{procedure}
\pproto{(make-vector \var{k} \var{fill})}{procedure}\unpenalty

indicate that the {\tt make-vector} procedure must be defined to take
either one or two arguments.

\label{typeconventions}
An operation that is presented with an argument that it
is not specified to handle will raise an exception with condition type
{\cf\&contract}.  For succinctness, we follow the convention
that if an parameter name is also the name of a type, then the corresponding argument must be of the named type.
For example, the header line for {\tt vector-ref} given above dictates that the
first argument to {\tt vector-ref} must be a vector.  The following naming
conventions imply type restrictions:
\newcommand{\foo}[1]{\vr{#1}, \vri{#1}, $\ldots$ \vrj{#1}, $\ldots$}
$$
\begin{tabular}{ll}
\var{obj}&any object\\
\var{z}&complex number\\
\var{x}&real number\\
\var{y}&real number\\
\var{q}&rational number\\
\var{n}&integer\\
\var{k}&exact non-negative integer\\
\var{char}&character (see section~\ref{charactersection})\\
\var{pair}&pair (see section~\ref{listsection})\\
\var{list}&list (see section~\ref{listsection})\\
\var{vector}&vector (see section~\ref{vectorsection})\\
\var{datum}&datum (see section~\ref{readsyntaxsection})\\
\var{string}&string (see section~\ref{stringsection})\\
\var{condition}&condition (see section~\ref{conditionssection})\\
\var{bytes}&bytes object (see section~\ref{bytessection})\\
\var{hash-table}&hash table (see section~\ref{hashtablesection})\\
\end{tabular}
$$

Other type restrictions are defined used in specific chapters.  For
example, chapter~\ref{numberchapter} uses a number of special parameter
variables for the various subsets of the numbers.

\todo{Provide an example entry??}

If \var{category} is something other than ``syntax'' and
``procedure,'' then the entry describes a non-procedural value, and
the \var{category} describes the type of that value.  The header line

\noindent\rvproto{\&who}{condition type}

indicates that {\cf\&who} is a condition type.

\section{List arguments}

List arguments are immutable in programs that do not make use of the
\library{r6rs mutable-pairs} library.  In such programs, a procedure accepting a list as an
argument can trivially check that the argument is indeed a list.

In programs that mutate pairs through use of the \library{r6rs
  mutable-pairs} library, a pair
that is the head of a list at one moment may not always be the head of
a list.  This complicates the description of how such procedures must
verify that their arguments are valid.  For that reason, the
specifications of procedures accepting lists will mostly assume that
lists are not mutated.  Section~\ref{mutablelistargumentsection}
extends the notion of lists and defines more precise restrictions on
the arguments to these procedures.

\section{Evaluation examples}

The symbol ``\evalsto'' used in program examples should be read
``evaluates to.''  For example,

\begin{scheme}
(* 5 8)      \ev  40%
\end{scheme}

means that the expression {\tt(* 5 8)} evaluates to the object {\tt 40}.
Or, more precisely:  the expression given by the sequence of characters
``{\tt(* 5 8)}'' evaluates, in the initial environment, to an object
that may be represented externally by the sequence of characters ``{\tt
40}''.  See chapter~\ref{readsyntaxchapter} for a discussion of external
representations of objects.

\section{Unspecified behavior}

\vest If the value of an expression is said to be ``unspecified,''
then the expression must evaluate without raising an exception, but
the values returned depends on the implementation; this report
explicitly does not say what values should be returned.
\mainindex{unspecified behavior}

Some expressions are specified to return \emph{the} unspecified value,
which is a special value returned by the \texttt{unspecified}
procedure.  (See section~\ref{unspecifiedvalue}.)  In this case, the
return value is meaningless, and programmers are discouraged from
relying on its specific nature.

\section{Multiple return values}

A Scheme expression can evaluate to an arbitrary finite number of
values.  These values are passed to the expression's continuation.

Not all continuations accept any number of values: A continuation that
accepts the argument to a procedure call is guaranteed to accept
exactly one value.  The effect of passing such a continuation a
different number of values is unspecified.  The {\cf call-with-values}
described in section~\ref{controlsection} allows creating
continuations that specified numbers of return values.  If a number of
return values is passed to a continuation created by {\cf
  call-with-values} not accepted by its \var{consumer} an exception is
raised.

A number of forms in the base library have sequences of expressions
as subforms that are evaluated sequentially, with the return values of
all but the last expression being discarded.  The continuations
discarding these values accept any number of values.

\section{Exceptional situations}

\mainindex{exceptional situation} When speaking of an exceptional situation, this
report uses the phrase ``an exception is raised'' to indicate
that implementations must detect the situation and report it to the
program through the exception system described in
chapter~\ref{exceptionsconditionschapter}.

Several variations on ``an exception is raised'' are possible:

\begin{itemize}
\item ``An exception should be raised'' means that implementations
  are encouraged, but not required, to detect the situation
  and to raise an exception.

\item ``An exception may be raised'' means that implementations
are allowed, but not required or encouraged, to detect
the situation and to raise an exception.

\item ``An exception might be  means that implementations
are allowed, but discouraged, to detect the situation
and to raise an exception.
\end{itemize}

A variety of exceptional situations are distinguished in this report,
among them violations of program syntax, violations of a procedure's
specification, violations of implementation restrictions, and
exceptional situations in the environment.  When an exception is
raised, an object is provided that describes the nature of the
exceptional siutation.  The report uses the condition system described
in section~\ref{conditionssection} to describe exceptional situations,
classifying them by condition types.  This report uses the phrase
``an exception with condition type \meta{condition-type}'' to indicate
that the object provided with the exception is a condition object of
the specified type.

For most exceptional situations where an exception is raised, the
program cannot continue at the place the situation was detected.  In
that case, the exception handler invoked by the exception must not
return.  In some cases, however, continuing is permissible; the
handler may return.  The phrase ``a continuable
exception is raised'' indicates such a situation.
See~\ref{exceptionssection}.

\vest For example, an exception with condition type {\cf\&contract}
is raised if a procedure is passed an argument that the procedure
is not explicitly specified to handle, even though such domain
exceptions are not always mentioned in this report.

The above requirements for violations and implementation restrictions
only apply in \textit{safe mode}.  Implementations may not raise
exceptions in those situations in \textit{unsafe mode}.  The
distinction is explained in section~\ref{safeunsafemodesection}.

Implementation restrictions indicate circumstances under which an
implementation is permitted to raise an exception if it is unable to
continue execution of a correct program because of some restriction
imposed by the implementation.

Some possible implementation restrictions
such as the lack of representations for NaNs and infinities (see
section~\ref{infinitiesnanssection}) are covered by this report, and
implementations must raise an exception of the appropriate condition
type if it encounters such a situation.

Implementation restrictions not explicitly covered in this report are
of course discouraged, but implementations are encouraged to report
violations of implementation restrictions.\mainindex{implementation
  restriction} For example, an implementation may raise an exception
with condition type {\cf\&implementation-restriction} if it does not
have enough storage to run a program.


\section{Safe and unsafe mode}
\label{safeunsafemodesection}

FIXME

\section{Safety}

The standard libraries whose exports are described by this
document are said to be \defining{safe libraries}.  Libraries
that import only from safe libraries, and do not contain any
{\cf (safe 0)} or {\cf unsafe} declarations, are also said
to be safe libraries.  A script is said to be safe if and
only if its library part is a safe library.

As defined by this document, the Scheme programming language
is safe in the following sense:
If a Scheme script is said to be safe, then its execution
cannot go so badly wrong as to behave in ways that are
inconsistent with the semantics described in this document,
unless said execution first encounters some implementation
restriction or other defect in the implementation of Scheme
that is executing the script.

Violations of an implementation restriction must cause an exception
with condition type {\cf\&implementation-restriction} to be raised, as must all
violations and errors that would otherwise threaten system
integrity in ways that might result in execution that is
inconsistent with the semantics described in this document.

The above safety properties are guaranteed only for scripts
and libraries that are said to be safe.  Implementations
may provide access to unsafe libraries, and may interpret
{\cf (safe 0)} and {\cf unsafe} declarations in ways that
cannot guarantee safety.

\section{Naming conventions}

By convention, the names of procedures that always return a boolean
value usually end
in ``\ide{?}''.  Such procedures are called predicates.

By convention, the names of procedures that store values into previously
allocated locations (see section~\ref{storagemodel}) usually end in
``\ide{!}''.
Such procedures are called mutation procedures.
By convention, the value returned by a mutation procedure is
\unspecifiedreturn{} (see section~\ref{unspecifiedvalue}).

By convention, ``\ide{->}'' appears within the names of procedures that
take an object of one type and return an analogous object of another type.
For example, {\cf list->vector} takes a list and returns a vector whose
elements are the same as those of the list.


	
\todo{Terms that need defining: thunk, command (what else?).}

\section{Programs and libraries}

A Scheme consists of a set of \textit{libraries\index{library}}, each
of which defines a part of the program connected to the others through
explicitly specified exports and imports.  A library consists of a set
of import and export FIXME specifications and a body, which contains
the code defining the library.  Chapter~\ref{librarychapter} describes
the syntax and semantics for libraries.  The subsequent chapters
describe various standard libraries provided by a Scheme system.  In
particular, chapter~\ref{baselibrarychapter} describes a base
library defining most of the constructs traditionally associated with
Scheme programs.

The division between the base library and other standard libraries is
based on use, not on construction.  In particular, some facilities
that are typically implemented as ``primitives'' by a compiler or
run-time libraries rather than in terms of other standard procedures
 or syntactic forms are not part of the base library, but defined in
separate libraries.  Examples include the fixnum and flonum libraries,
the exceptions and conditions libraries, and the libraries for
records.

\section{Variables, syntactic keywords, and regions}
\label{specialformsection}
\label{variablesection}

In a library body,
an identifier\index{identifier} may name a type of syntax, or it may name
a location where a value can be stored.  An identifier that names a type
of syntax is called a {\em syntactic keyword}\mainindex{syntactic keyword}
and is said to be {\em bound} to that syntax.  An identifier that names a
location is called a {\em variable}\mainindex{variable} and is said to be
{\em bound} to that location.  The set of all visible
bindings\mainindex{binding} in effect at some point in a program is
known as the {\em environment} in effect at that point.  The value
stored in the location to which a variable is bound is called the
variable's value.  By abuse of terminology, the variable is sometimes
said to name the value or to be bound to the value.  This is not quite
accurate, but confusion rarely results from this practice.

\todo{Define ``assigned'' and ``unassigned'' perhaps?}

\todo{In programs without side effects, one can safely pretend that the
variables are bound directly to the arguments.  Or:
In programs without \ide{set!}, one can safely pretend that the
variable is bound directly to the value. }

\vest Certain expression types are used to create new kinds of syntax
and bind syntactic keywords to those new syntaxes, while other
expression types create new locations and bind variables to those
locations.  These expression types are called {\em binding constructs}.
\mainindex{binding construct}
The constructs in the base library that bind syntactic keywords are listed in section~\ref{macrosection}.
The most fundamental of the variable binding constructs is the
{\cf lambda} expression, because all other variable binding constructs
can be explained in terms of {\cf lambda} expressions.  The other
variable binding constructs are {\cf let}, {\cf let*}, {\cf letrec*},
{\cf letrec}, {\cf let-values}, {\cf let*-values}, {\cf do}, and {\cf
  case-lambda} expressions (see sections~\ref{lambda}, \ref{letrec}, 
\ref{do}, and \ref{case-lambda}).

%Note: internal definitions not mentioned here.

\vest Like Algol and Pascal, and unlike most other dialects of Lisp
except for Common Lisp, Scheme is a statically scoped language with
block structure.  To each place where an identifier is bound in a program
there corresponds a \defining{region} of the program text within which
the binding is visible.  The region is determined by the particular
binding construct that establishes the binding; if the binding is
established by a {\cf lambda} expression, for example, then its region
is the entire {\cf lambda} expression.  Every mention of an identifier
refers to the binding of the identifier that established the
innermost of the regions containing the use.  If there is no binding of
the identifier whose region contains the use, then the use refers to the
binding for the variable in the top level environment, if any
(FIXME chapters~\ref{modulelibraries}); if there is no
binding for the identifier,
it is said to be \defining{unbound}.\mainindex{bound}\index{top level
environment}

\todo{Mention that some implementations have multiple top level environments?}

\todo{Pitman sez: needs elaboration in case of {\tt(let ...)}}

\todo{Pitman asks: say something about vars created after scheme starts?
{\tt (define x 3) (define (f) x) (define (g) y) (define y 4)}
Clinger replies: The language was explicitly
designed to permit a view in which no variables are created after
Scheme starts.  In files, you can scan out the definitions beforehand.
I think we're agreed on the principle that interactive use should
approximate that behavior as closely as possible, though we don't yet
agree on which programming environment provides the best approximation.}

\section{Syntax defects}

Scheme implementations conformant with this report must detect
defects in the syntax.  A \defining{syntax defect} may be an error with
respect to the lexical syntax specified in
section~\ref{lexicalsyntaxsection}, the read syntax specified in
section~\ref{readsyntaxsection}, or the ``\exprtype'' entries in the
specification of the base library or the standard libraries.
Moreover, attempting to assign to an immutable variable is also
considered a syntax defect.  (The locations bound to variables
imported from other libraries are immutable.  See
section~\ref{importsareimmutablesection}.)

If a script or library form is not syntactically correct, then the
execution of that script or library must not be allowed to begin.

\section{Storage model}
\label{storagemodel}

Variables and objects such as pairs, vectors, and strings implicitly
denote locations\mainindex{location} or sequences of locations.  A string, for
example, denotes as many locations as there are characters in the string. 
(These locations need not correspond to a full machine word.) A new value may be
stored into one of these locations using the {\tt string-set!} procedure, but
the string continues to denote the same locations as before.

An object fetched from a location, by a variable reference or by
a procedure such as {\cf car}, {\cf vector-ref}, or {\cf string-ref}, is
equivalent in the sense of \ide{eqv?} % and \ide{eq?} ??
(section~\ref{equivalencesection})
to the object last stored in the location before the fetch.

Every location is marked to show whether it is in use.
No variable or object ever refers to a location that is not in use.
Whenever this report speaks of storage being allocated for a variable
or object, what is meant is that an appropriate number of locations are
chosen from the set of locations that are not in use, and the chosen
locations are marked to indicate that they are now in use before the variable
or object is made to denote them.

In many systems it is desirable for constants\index{constant} (i.e. the values of
literal expressions) to reside in read-only-memory.  To express this, it is
convenient to imagine that every object that denotes locations is associated
with a flag telling whether that object is mutable\index{mutable} or
immutable\index{immutable}.  In such systems literal constants and the strings
returned by \ide{symbol->string} are immutable objects, while all objects
created by the other procedures listed in this report are mutable.  An
attempt to store a new value into a location that is denoted by an
immutable object will raise an exception.

\section{Proper tail recursion}
\label{proper tail recursion}

Implementations of Scheme are required to be
{\em properly tail-recursive}\mainindex{proper tail recursion}.
Procedure calls that occur in certain syntactic
contexts defined below are `tail calls'.  A Scheme implementation is
properly tail-recursive if it supports an unbounded number of active
tail calls.  A call is {\em active} if the called procedure may still
return.  Note that this includes calls that may be returned from either
by the current continuation or by continuations captured earlier by
{\cf call-with-current-continuation} that are later invoked.
In the absence of captured continuations, calls could
return at most once and the active calls would be those that had not
yet returned.
A formal definition of proper tail recursion can be found
in~\cite{propertailrecursion}.  The rules for identifying tail calls
in base-library constructs are described in
section~\ref{basetailcontextsection}.

\begin{rationale}

Intuitively, no space is needed for an active tail call because the
continuation that is used in the tail call has the same semantics as the
continuation passed to the procedure containing the call.  Although an improper
implementation might use a new continuation in the call, a return
to this new continuation would be followed immediately by a return
to the continuation passed to the procedure.  A properly tail-recursive
implementation returns to that continuation directly.

Proper tail recursion was one of the central ideas in Steele and
Sussman's original version of Scheme.  Their first Scheme interpreter
implemented both functions and actors.  Control flow was expressed using
actors, which differed from functions in that they passed their results
on to another actor instead of returning to a caller.  In the terminology
of this section, each actor finished with a tail call to another actor.

Steele and Sussman later observed that in their interpreter the code
for dealing with actors was identical to that for functions and thus
there was no need to include both in the language.

\end{rationale}

%%% Local Variables: 
%%% mode: latex
%%% TeX-master: "r6rs"
%%% End: 
   \par
%\vfill\eject
\chapter{Notation and terminology}
\label{terminologychapter}

\section{Entry format}

The chapters describing bindings in the base library and the standard
libraries are organized
into entries.  Each entry describes one language feature or a group of
related features, where a feature is either a syntactic construct or a
built-in procedure.  An entry begins with one or more header lines of the form

\noindent\pproto{\var{template}}{\var{category}}\unpenalty

If \var{category} is ``\exprtype'', the entry describes a 
special syntactic form, and the template gives the syntax of the form.  Even
though the template is written in a notation similar to a right-hand
side of the BNF rules in chapter~\ref{readsyntaxchapter}, it describes
the set of forms equivalent to the forms matching the
template as syntactic datums.

Components of the form described by a template are designated
by syntactic variables, which are written using angle brackets, for
example, \hyper{expression}, \hyper{variable}.  Case is insignificant
in syntactic variables.  Syntactic variables
should be understood to denote other forms, or, in some cases,
sequences of them.  A syntactic variable may refer to a non-terminal
in the grammar for syntactic datums, in which case only forms matching
that non-terminal are permissible in that position.  For example,
\hyper{expression} stands for any form which is a
syntactically valid expression.  Other non-terminals that are used in
templates will be defined as part of the specification.

The notation
\begin{tabbing}
\qquad \hyperi{thing} $\ldots$
\end{tabbing}
indicates zero or more occurrences of a \hyper{thing}, and
\begin{tabbing}
\qquad \hyperi{thing} \hyperii{thing} $\ldots$
\end{tabbing}
indicates one or more occurrences of a \hyper{thing}.

It is a syntax violation if a component of a form does not have the
shape specified by a template---an exception with condition type
{\cf\&syntax} is raised at expansion time.

Descriptions of syntax may express other restrictions on the
components of a form.  Typically, such a restriction is formulated
as a phrase of the form ``\hyper{x} must be a \ldots'' (or otherwise
using the word ``must''.)  As with
implicit restrictions, such a phrase means that an exception with
condition type {\cf\&syntax} is raised if the component does not
meet the restriction.


If \var{category} is ``procedure'', then the entry describes a procedure, and
the header line gives a template for a call to the procedure.  Parameter
names in the template are \var{italicized}.  Thus the header line

\noindent\pproto{(vector-ref \var{vector} \var{k})}{procedure}\unpenalty

indicates that the built-in procedure {\tt vector-ref} takes
two arguments, a vector \var{vector} and an exact non-negative integer
\var{k} (see below).  The header lines

\noindent%
\pproto{(make-vector \var{k})}{procedure}
\pproto{(make-vector \var{k} \var{fill})}{procedure}\unpenalty

indicate that the {\tt make-vector} procedure takes
either one or two arguments.  The parameter names are
case-insensitive: \var{Vector} is the same as \var{vector}.

As with syntax templates, an ellipsis \dotsfoo{} at the end of a header
line, as in

\pproto{=}{ \vari{z} \varii{z} \variii{z} \dotsfoo}{procedure}

indicates that the procedure takes arbitrarily many arguments of the
same type as specified for the last parameter name.  In this case,
{\cf =} accepts two or more arguments that must all be complex
numbers.

\label{typeconventions}
A procedure that is called with an argument that it is not
specified to handle raises an exception with condition type
{\cf\&contract}.  Also, if the number of arguments provided in
a procedure call does not match any argument count specified for the
called procedure, an exception with condition type {\cf\&contract}
must be raised.

For succinctness, we follow the convention
that if a parameter name is also the name of a type, then the corresponding argument must be of the named type.
For example, the header line for {\tt vector-ref} given above dictates that the
first argument to {\tt vector-ref} must be a vector.  The following naming
conventions imply type restrictions:
\newcommand{\foo}[1]{\vr{#1}, \vri{#1}, $\ldots$ \vrj{#1}, $\ldots$}
\begin{displaymath}
\begin{tabular}{ll}
\var{obj}&any object\\
\var{z}&complex number\\
\var{x}&real number\\
\var{y}&real number\\
\var{q}&rational number\\
\var{n}&integer\\
\var{k}&exact non-negative integer\\
\var{octet}&exact integer in $$\{0, \ldots, 255\}\\
\var{byte}&exact integer in $$\{-128, \ldots, 127\}\\
\var{char}&character (see section~\ref{charactersection})\\
\var{pair}&pair (see section~\ref{listsection})\\
\var{list}&list (see section~\ref{listargumentssection})\\
\var{vector}&vector (see section~\ref{vectorsection})\\
\var{string}&string (see section~\ref{stringsection})\\
\var{condition}&condition (see section~\ref{conditionssection})\\
\var{bytes}&bytes object (see chapter~\ref{byteschapter})\\
\var{proc}&procedure (see section~\ref{proceduressection})
\end{tabular}
\end{displaymath}

Other type restrictions are expressed through parameter naming
conventions that are described in specific chapters.  For
example, chapter~\ref{numberchapter} uses a number of special parameter
variables for the various subsets of the numbers.

Descriptions of procedures may express other restrictions on the
arguments of a procedure.  Typically, such a restriction is formulated
as a phrase of the form ``\var{x} must be a \ldots'' (or otherwise
using the word ``must''.)  As with
implicit restrictions, such a phrase means that an exception with
condition type {\cf\&contract} is raised if the argument does not
meet the restriction.

If \var{category} is something other than ``syntax'' and
``procedure'', then the entry describes a non-procedural value, and
the \var{category} describes the type of that value.  The header line

\noindent\rvproto{\&who}{condition type}

indicates that {\cf\&who} is a condition type.

The description of an entry occasionally states that it is \textit{the
  same} as another entry.  This means that both entries are
equivalent.  Specifically, it means that if both entries have the same
name and are thus exported from different libraries, the entries from
both libraries can be imported under the same name without conflict.

\section{Boolean values}
\label{booleanvaluessection}

Although there is a separate boolean type, any Scheme value can be
used as a boolean value for the purpose of a conditional test.  In a
conditional test, all values count as true in such a test except for
\schfalse{}.  This report uses the word ``true'' to refer to any
Scheme value except \schfalse{}, and the word ``false'' to refer to
\schfalse{}. \mainindex{true} \mainindex{false}

\section{List arguments}
\label{listargumentssection}

List arguments are immutable in programs that do not make use of the
\library{r6rs mutable-pairs} library.  In such programs, a procedure accepting a list as an
argument can check whether the argument is a list by traversing it.

In programs that mutate pairs through use of the \library{r6rs
  mutable-pairs} library, a pair
that is the head of a list at one moment may not always be the head of
a list.  Thus a traversal of the structure cannot by itself guarantee
that the structure is a list; one must also know that no concurrent or
interleaved computation can mutate the pairs of the structure.
This greatly complicates the description of how certain procedures
must verify that their arguments are valid.

For that reason, the specifications of procedures that accept lists
generally assume that those lists are not mutated.
Section~\ref{mutablelistargumentsection} relaxes that assumption
and states more precise restrictions on the arguments to these
procedures.

\section{Evaluation examples}

The symbol ``\evalsto'' used in program examples should be read
``evaluates to''.  For example,

\begin{scheme}
(* 5 8)      \ev  40%
\end{scheme}

means that the expression {\tt(* 5 8)} evaluates to the object {\tt 40}.
Or, more precisely:  the expression given by the sequence of characters
``{\tt(* 5 8)}'' evaluates, in the initial environment, to an object
that may be represented externally by the sequence of characters ``{\tt
40}''.  See section~\ref{readsyntaxsection} for a discussion of external
representations of objects.

The ``\evalsto'' symbol is also used when the evaluation of an
expression raises an exception.  For example,

\begin{scheme}
(integer->char \sharpsign{}xD800) \ev \exception{\&contract}
\end{scheme}

means that the evaluation of the expression {\cf (integer->char
  \sharpsign{}xD800)} causes an exception with condition type
{\cf\&contract} to be raised.

\section{Unspecified behavior}

\vest If the value of an expression is said to be ``unspecified'',
then the expression must evaluate without raising an exception, but
the values returned depend on the implementation; this report
explicitly does not say what values should be returned.
\mainindex{unspecified behavior}

Some expressions are specified to return \emph{the} unspecified value,
which is a special value returned by the \texttt{unspecified}
procedure.  (See section~\ref{unspecifiedvalue}.)  In this case, the
return value is meaningless, and programmers are discouraged from
relying on its specific nature.

\section{Exceptional situations}

When speaking of an exceptional situation (see section~\ref{exceptionalsituationsection}), this
report uses the phrase ``an exception is raised'' to indicate
that implementations must detect the situation and report it to the
program through the exception system described in
chapter~\ref{exceptionsconditionschapter}.

Several variations on ``an exception is raised'' are possible:

\begin{itemize}
\item ``An exception should be raised'' means that implementations
  are encouraged, but not required, to detect the situation
  and to raise an exception.

\item ``An exception may be raised'' means that implementations
are allowed, but not required or encouraged, to detect
the situation and to raise an exception.

\item ``An exception might be raised'' means that implementations
are allowed, but discouraged, to detect the situation
and to raise an exception.
\end{itemize}

This report uses the phrase ``an exception with condition type \var{t}''
to indicate that the object provided with the
exception is a condition object of the specified type.

The phrase ``a continuable exception is raised'' indicates
an exceptional situation that permits the exception handler to return,
thereby allowing program execution to continue at the place where the
original exception occurred.  See section~\ref{exceptionssection}.

\vest For example, an exception with condition type {\cf\&contract}
is raised if a procedure is passed an argument that the procedure
is not explicitly specified to handle, even though such domain
exceptions are not always mentioned in this report.

\section{Naming conventions}

By convention, the names of procedures that always return a boolean
value usually end
in ``\ide{?}''.  Such procedures are called predicates.

By convention, the names of procedures that store values into previously
allocated locations (see section~\ref{storagemodel}) usually end in
``\ide{!}''.
Such procedures are called mutation procedures.
By convention, the value returned by a mutation procedure is
\unspecifiedreturn{} (see section~\ref{unspecifiedvalue}),
but this convention is not always followed.

By convention, ``\ide{->}'' appears within the names of procedures that
take an object of one type and return an analogous object of another type.
For example, {\cf list->vector} takes a list and returns a vector whose
elements are the same as those of the list.

By convention, the names of condition types usually end in
``{\cf\&}''\index{&@\texttt{\&}}.

\section{Syntax violations}

Scheme implementations conformant with this report must detect
violations of the syntax.  A \defining{syntax violation} is an error
with respect to the syntax of library bodies, script bodies,
or the ``\exprtype'' entries in the
specification of the base library or the standard libraries.
Moreover, attempting to assign to an immutable variable (i.e., the
variables exported by a library; see
section~\ref{importsareimmutablesection}) is also
considered a syntax violation.

If a script or library form is not syntactically correct, then the
execution of that script or library must not be allowed to begin.

%%% Local Variables: 
%%% mode: latex
%%% TeX-master: "r6rs"
%%% End: 
 \par
\chapter{Libraries}
\label{librarychapter}
\mainindex{library}
Libraries are pieces of
code that can be incorporated into larger
programs, and especially into programs that use library code from multiple
sources.  The library system supports macro definitions within libraries,
allows macro exports, and distinguishes the phases in which definitions
and imports are needed.

Libraries address the following specific goals:

\begin{description}
\item[Separate compilation and analysis] No two libraries have to be
  compiled at the same time (i.e., the meanings of two libraries
  cannot depend on each other cyclically, and compilation of two
  different libraries cannot rely on state shared across
  compilations), and significant program analysis
  can be performed without examining a whole program.
\item[Independent compilation/analysis of unrelated libraries] 
  ``Unrelated'' means that neither depends on the other through a
  transitive closure of imports.
\item[Explicit declaration of dependencies] The meaning of
  each identifier is clear at compile time.  Hence, there is no
  ambiguity about whether a library needs to be executed for another
  library's compile time and/or run time.
\item[Namespace management] This helps prevent name conflicts.
\end{description}

This chapter defines the notation for
libraries and a semantics for library expansion and execution.

\section{Library form}
\label{librarysyntaxsection}

A library definition must have the following form:\mainschindex{library}\mainschindex{import}\mainschindex{export}

\begin{scheme}
(library \hyper{library~name}
  (export \hyper{export~spec} \ldots)
  (import \hyper{import~spec} \ldots)
  \hyper{library~body})%
\end{scheme}

A library declaration contains the following elements:

\begin{itemize}
\item The \hyper{library~name} specifies the name of the library
  (possibly with versioning).
\item The {\cf export} subform specifies a list of exports, which name
  a subset of the bindings defined within or imported into the
  library.
\item The {\cf import} subform specifies the imported bindings as a
  list of import dependencies, where each dependency specifies:
\begin{itemize}
\item the imported library's name,
\item the relevant levels, e.g., expand or run time, and
\item the subset of the library's exports to make available within the
      importing library, and the local names to use within the importing
      library for each of the library's exports, and
\end{itemize}
\item The \hyper{library body} is the library body, consisting of a
  sequence of definitions followed by a sequence of expressions.  The
  definitions may be both for local (unexported) and exported
  bindings, and the set of initialization expressions to be evaluated
  for their effects.
\end{itemize}

An identifier can be imported with the same local name from two or
more libraries or for two levels from the same library only if the
binding exported by each library is the same (i.e., the binding is
defined in one library, and it arrives through the imports only by
exporting and re-exporting).  Otherwise, no identifier can be imported
multiple times, defined multiple times, or both defined and imported.
No identifiers are visible within a library except for those
explicitly imported into the library or defined within the library.

A \hyper{library name} has the following form:

\begin{scheme}
(\hyperi{identifier} \hyperii{identifier} \ldots \hyper{version})%
\end{scheme}

where \hyper{version} is empty or has the following form:
%
\begin{scheme}
(\hyper{sub-version} \ldots)%
\end{scheme}

Each \hyper{sub-version} must be an exact nonnegative integer.
An empty \hyper{version} is equivalent to {\cf ()}.

An \hyper{export~spec} names a set of imported and locally defined bindings to
be exported, possibly with different
external names.  An \hyper{export~spec} must have one of the
following forms:

\begin{scheme}
\hyper{identifier}
(rename (\hyperi{identifier} \hyperii{identifier}) \ldots)%
\end{scheme}

In an \hyper{export~spec}, an \hyper{identifier} names a single binding defined
within or imported into the library, where the external name for the export is
the same as the name of the binding within the library. 
A {\cf rename} spec exports the binding named by the first
\hyper{identifier} in each {\cf (\hyper{identifier}
  \hyper{identifier})} pairing, using the second \hyper{identifier} as the
external name.

Each \hyper{import~spec} specifies a set of bindings to be imported into
the library, the levels at which they are to be available, and the local
names by which they are to be known.  An \hyper{import spec} must
be one of the following:
%
\begin{scheme}
\hyper{import set}
(for \hyper{import~set} \hyper{import~level} \ldots)%
\end{scheme}

An \hyper{import level}  is one of the following:
\begin{scheme}
run
expand
(meta \hyper{level})%
\end{scheme}

where \hyper{level} is an exact integer.

As an \hyper{import level}, {\cf run} is an abbreviation for {\cf
  (meta 0)}, and {\cf expand} is an abbreviation for {\cf (meta 1)}.
Levels and phases are discussed in section~\ref{phasessection}.

An \hyper{import~set} names a set of bindings from another library and
possibly specifies local names for the imported bindings.  It must be
one of the following:

\begin{scheme}
\hyper{library~reference}
(only \hyper{import~set} \hyper{identifier} \ldots)
(except \hyper{import~set} \hyper{identifier} \ldots)
(prefix \hyper{import~set} \hyper{identifier})
(rename \hyper{import~set} (\hyper{identifier} \hyper{identifier}) \ldots)%
\end{scheme}

A \hyper{library~reference} identifies a library by its 
name and optionally by its version.  It has the following form:

\begin{scheme}
(\hyperi{identifier} \hyperii{identifier} \ldots \hyper{version~reference})%
\end{scheme}

A \hyper{version~reference} specifies a set of \hyper{version}s that
it matches.  The \hyper{library~reference} identifies all libraries of
the same name and whose version is matched by the
\hyper{version~reference}.  A \hyper{version~reference} is empty or has
the following form:
%
\begin{scheme}
(\hyperi{sub-version reference} \ldots \hypern{sub-version reference})
(and \hyper{version reference} \ldots)
(or \hyper{version reference} \ldots)
(not \hyper{version reference})%
\end{scheme}
%
An empty \hyper{version reference} is equivalent to {\cf ()}.  A
\hyper{version reference} of the first form matches a \hyper{version}
with at least $n$ elements, whose \hyper{sub-version reference}s match
the corresponding \hyper{sub-version}s.  An {\cf and} \hyper{version
  reference} matches a version if all \hyper{version references}
following the {\cf and} match it.  Correspondingly, an {\cf
  or} \hyper{version reference} matches a version if one of
\hyper{version references} following the {\cf or} matches it,
and a {\cf not} \hyper{version reference} matches a version if the
\hyper{version reference} following it does not match it.

A \hyper{sub-version reference} has one of the following forms:

\begin{scheme}
\hyper{sub-version}
(>= \hyper{sub-version})
(<= \hyper{sub-version})
(and \hyper{sub-version~reference} \ldots)
(or \hyper{sub-version~reference} \ldots)
(not \hyper{sub-version~reference})%
\end{scheme}

A \hyper{sub-version reference} of the first form matches a
\hyper{sub-version} if it is equal to it.  A {\cf >=}
\hyper{sub-version reference} of the first form matches a sub-version
if it is greater or equal to the \hyper{sub-version} following it;
analogously for {\cf <=}.  An {\cf and} \hyper{sub-version reference}
matches a sub-version if all of the subsequent \hyper{sub-version
  reference}s match it.  Correspondingly, an {\cf or}
\hyper{sub-version reference} matches a sub-version if one of the
subsequent \hyper{sub-version reference}s matches it, and a {\cf not}
\hyper{sub-version reference} matches a sub-version if the subsequent
\hyper{sub-version reference} does not match it.

Examples:

\texonly\begin{center}\endtexonly
  \begin{tabular}{lll}
    version reference & version & match?
    \\
    {\cf ()} & {\cf (1)} & yes\\
    {\cf (1)} & {\cf (1)} & yes\\
    {\cf (1)} & {\cf (2)} & no\\
    {\cf (2 3)} & {\cf (2)} & no\\
    {\cf (2 3)} & {\cf (2 3)} & yes\\
    {\cf (2 3)} & {\cf (2 3 5)} & yes\\
    {\cf (or (1 (>= 1)) (2))} & {\cf (2)} & yes\\
    {\cf (or (1 (>= 1)) (2))} & {\cf (1 1)} & yes\\
    {\cf (or (1 (>= 1)) (2))} & {\cf (1 0)} & no\\
    {\cf ((or 1 2 3))} & {\cf (1)} & yes\\
    {\cf ((or 1 2 3))} & {\cf (2)} & yes\\
    {\cf ((or 1 2 3))} & {\cf (3)} & yes\\
    {\cf ((or 1 2 3))} & {\cf (4)} & no
  \end{tabular}
\texonly\end{center}\endtexonly

When more than one library is identified by a library reference, the
choice of libraries is determined in some implementation-dependent manner.

To avoid problems such as incompatible types and replicated state, two
libraries whose library names consist of the same sequence of identifiers but
whose versions do not match cannot co-exist in the same program.

By default, all of an imported library's exported bindings are made
visible within an importing library using the names given to the bindings
by the imported library.
The precise set of bindings to be imported and the names of those
bindings can be adjusted with the {\cf only}, {\cf except},
{\cf prefix}, and {\cf rename} forms as described below.

\begin{itemize}
\item An {\cf only} form produces a subset of the bindings from another
\hyper{import~set}, including only the listed
\hyper{identifier}s.
The included \hyper{identifier}s must be in
the original \hyper{import~set}.
\item An {\cf except} form produces a subset of the bindings from another
\hyper{import~set}, including all but the listed
\hyper{identifier}s.
All of the excluded \hyper{identifier}s must be in
the original \hyper{import~set}.
\item A {\cf prefix} form adds the \hyper{identifier} prefix to each
name from another \hyper{import~set}.
\item A {\cf rename} form, {\cf (rename (\hyperi{identifier} \hyperii{identifier}) \ldots)},
removes the bindings for {\cf \hyperi{identifier} \ldots} to form an
intermediate \hyper{import~set}, then adds the bindings back for the
corresponding {\cf \hyperii{identifier} \ldots} to form the final
\hyper{import~set}.
Each \hyperi{identifier} must be in the original \hyper{import~set},
each \hyperii{identifier} must not be in the intermediate \hyper{import~set},
and the \hyperii{identifier}s must be distinct.
\end{itemize}
It is a syntax violation if a constraint given above is not met.

\label{librarybodysection}
The \hyper{library~body} of a {\cf library} form consists of forms
that are classified as 
\textit{definitions}\mainindex{definition} or
\textit{expressions}\mainindex{expression}.  Which forms belong to
which class depends on the imported libraries and the result of
expansion---see chapter~\ref{expansionchapter}.  Generally, forms that
are not 
definitions (see section~\ref{defines} for definitions available
through the base library) are expressions.

A \hyper{library~body} is like a \hyper{body} (see section~\ref{bodiessection}) except that
a \hyper{library~body}s need not include any expressions.  It must
have the following form:

\begin{scheme}
\hyper{definition} \ldots \hyper{expression} \ldots%
\end{scheme}

When base-library {\cf begin}, {\cf let-syntax}, or {\cf letrec-syntax} forms
occur in a top-level body prior to the first
expression, they are spliced into the body; see section~\ref{begin}.
Some or all of the body, including portions wrapped in {\cf begin},
{\cf let-syntax}, or {\cf letrec-syntax}
forms, may be specified by a syntactic abstraction
(see section~\ref{macrosection}).

The transformer expressions and bindings are evaluated and created
from left to right, as described in chapter~\ref{expansionchapter}.
The variable-definition right-hand-side expressions are evaluated
from left to right, as if in an implicit {\cf letrec*},
and the body expressions are also evaluated from left to right
after the variable-definition right-hand-side expressions.
A fresh location is created for each exported variable and initialized
to the value of its local counterpart.
The effect of returning twice to the continuation of the last body
expression is unspecified.

The names {\cf library}, {\cf export}, {\cf import},
{\cf for}, {\cf run}, {\cf expand}, {\cf meta},
{\cf import}, {\cf export}, {\cf only}, {\cf except}, {\cf
  prefix}, {\cf rename}, {\cf and}, {\cf or}, {\cf >=}, and {\cf <=}
appearing in the library syntax are part of the
syntax and are not reserved, i.e., the same names can be used for other
purposes within the library or even exported from or imported 
into a library with different meanings, without affecting their
use in the {\cf library} form.

Bindings defined with a library are not visible in code
outside of the library, unless the bindings are explicitly exported from the
library. 
An exported macro may, however, \emph{implicitly export} an otherwise
unexported identifier defined within or imported into the library.
That is, it may insert a reference to that identifier into the output code
it produces.

\label{importsareimmutablesection} 
All explicitly exported variables are immutable in both the
exporting and importing libraries. 
It is thus a syntax violation if an
explicitly exported variable appears on the left-hand side of a {\cf set!}
expression, either in the exporting or importing libraries.

All implicitly exported variables are also immutable in both the
exporting and importing libraries.
It is thus a syntax violation if a
variable appears on the left-hand side of a {\cf set!}
expression in any code produced by an exported macro outside of the
library in which the variable is defined.
It is also a syntax violation if a
reference to an assigned variable appears in any code produced by
an exported macro outside of the library in which the variable is defined,
where an assigned variable is one that appears on the left-hand
side of a {\cf set!} expression in the exporting library.

All other variables defined within a library are mutable.

\section{Import and export levels}
\label{phasessection}

Every library can be characterized by expand-time information (minimally,
its imported libraries, a list of the exported keywords, a list of the
exported variables, and code to evaluate the transformer expressions) and
run-time information (minimally, code to evaluate the variable definition
right-hand-side expressions, and code to evaluate the body expressions).
The expand-time information must be available to expand references to
any exported binding, and the run-time information must be available to
evaluate references to any exported variable binding.

Expanding a library may require run-time information from another
library. For example, if a library provides procedures that are called
by another library's macros during expansion, then the former library
must be run when expanding the latter. The former may not be needed
when the latter is eventually run as part of a program, or it may be
needed for the latter's run time, too.

\mainindex{phase}
%
A \emph{phase} is a time at which the expressions within a library are
evaluated.
Within a library body, top-level expressions and
the right-hand sides of {\cf define} forms are evaluated at run time,
i.e., phase $0$, and the right-hand
sides of {\cf define-syntax} forms are evaluated at expand time, i.e.,
phase $1$.
When {\cf define-syntax},
{\cf let-syntax}, or {\cf letrec-syntax}
forms appear within code evaluated at phase $n$, the right-hand sides
are evaluated as phase $n+1$ expressions.

These phases are relative to the phase in which the library itself is
used.
An \defining{instance} of a library corresponds to an evaluation of its
variable definitions and expressions in a particular phase relative to another
library---a process called \defining{instantiation}.
For example, if a top-level expression in a library $L_1$ refers to
a variable export from another library $L_0$, then it refers to the export from an
instance of $L_0$ at phase $0$ (relative to the phase of $L_1$).
But if a phase $1$ expression within $L_1$ refers to the same binding from
$L_0$, then it refers to the export from an instance of $L_0$ at phase $1$
(relative to the phase of $L_1$).

A \defining{visit} of a library corresponds to the evaluation of its syntax
definitions in a particular phase relative to another
library---a process called \defining{visiting}. Evaluating a syntax definition
at phase $n$ means that its right-hand side is evaluated at phase $n+1$.
For example, if a top-level expression in a library $L_1$ refers to
a macro export from another library $L_0$, then it refers to the export from an
visit of $L_0$ at phase $0$ (relative to the phase of $L_1$), which corresponds
to the evaluation of the macro's transformer expression at phase $1$.


\mainindex{level}\mainindex{import level} 
%
A \emph{level} is a lexical property of an identifier that determines
in which phases it can be referenced. The level for each identifier
bound by a definition within a library is $0$; that is, the identifier
can be referenced only by phase $0$ expressions within the library.
The level for each imported binding is determined by the enclosing {\cf
  for} form of the {\cf import} in the importing library, in
addition to the levels of the identifier in the exporting
library. Import and export levels are combined by pairwise addition of
all level combinations.  For example, references to an imported
identifier exported for levels $p_a$ and $p_b$ and imported for levels
$q_a$, $q_b$, and $q_c$ are valid at levels $p_a+q_a$, $p_a+q_b$,
$p_a+q_c$, $p_b+q_a$, $p_b+q_b$, and $p_b+q_c$. An \hyper{import~set}
without an enclosing {\cf for} is equivalent to {\cf (for
  \hyper{import~set} run)}, which is the same as {\cf (for
  \hyper{import~set} (meta 0))}.

The export level of an exported binding is $0$ for all bindings
that are defined within the exporting library. The export levels of a
reexported binding, i.e., an export imported from another library, are the
same as the effective import levels of that binding within the reexporting
library.

For the libraries defined in the library report, the export level is
$0$ for nearly all bindings. The exceptions are {\cf syntax-rules},
{\cf identifier-syntax}, {\cf ...}, and {\cf \_} from the
\rsixlibrary{base} library, which are exported with level $1$, {\cf
  set!} from the \rsixlibrary{base} library, which is exported with
levels $0$ and $1$, and all bindings from the composite
\thersixlibrary{} library (see library
chapter~\extref{lib:complibchapter}{Composite library}), which are
exported with levels $0$ and $1$.

Macro expansion within a library can introduce a reference to an
identifier that is not explicitly imported into the library. In that
case, the phase of the reference must match the identifier's level as
shifted by the difference between the phase of the source library
(i.e., the library that supplied the identifier's lexical context) and
the library that encloses the reference. For example, suppose that
expanding a library invokes a macro transformer, and the evaluation of
the macro transformer refers to an identifier that is exported from
another library (so the phase $1$ instance of the library is used);
suppose further that the value of the binding is a syntax object
representing an identifier with only a level-$n$ binding; then, the
identifier must be used only in a phase $n+1$ expression in the
library being expanded. This combination of levels and phases is why
negative levels on identifiers can be useful, even though libraries
exist only at non-negative phases.

If any of a library's definitions are referenced at phase $0$ in the
expanded form of a program, then an instance of the referenced library
is created for phase $0$ before the program's definitions and
expressions are evaluated. This rule applies transitively: if the
expanded form of one library references at phase $0$ an identifier
from another library, then before the referencing library is
instantiated at phase $n$, the referenced library must be instantiated
at phase $n$. When an identifier is referenced at any phase $n$
greater than $0$, in contrast, then the defining library is
instantiated at phase $n$ at some unspecified time before the
reference is evaluated. Similarly, when a macro keyword is referenced at
phase $n$ during the expansion of a library, then the
defining library is visited at phase $n$ at some unspecified time
before the reference is evaluated.

An implementation is allowed to distinguish instances/visits of a library for
different phases or to use an instance/visit at any phase as an instance/visit at
any other phase. An implementation is further allowed to start each
expansion of a {\cf library} form by removing
visits of libraries in any phase and/or instances of
libraries in phases above $0$. An implementation is allowed to
create instances/visits of more libraries at more phases than required to
satisfy references. When an identifier appears as an expression in a
phase that is inconsistent with the identifier's level, then an
implementation may raise an exception either at expand time or run
time, or it may allow the reference. Thus, a library is portable only
when it references identifiers in phases consistent with the declared
levels, and a library whose meaning depends on whether the
instances of a library are distinguished or shared across phases or
{\cf library} expansions may be unportable.

\begin{note}
If a program and its libraries avoid the \thersixlibrary{}
and \rsixlibrary{syntax-case} libraries, and if the program and libraries
never use the {\cf for} import form, then the program does not depend
on whether instances are distinguished across phases, and the phase of
an identifier's use cannot be inconsistent with the identifier's level.
\end{note} 

\section{Primitive syntax}

After the {\cf import} form within a {\cf library} form, the forms
that constitute a library body depend on the libraries that are
imported. In particular, imported syntactic keywords determine most
of the available forms and whether each form is a 
definition or expression. A few form types are
always available independent of imported libraries, however,
including constant literals, variable references, procedure calls,
 and macro uses.

\subsection{Primitive expression types}
\label{primitiveexpressionsection}

The entries in this section all describe expressions, which may occur
in the place of \hyper{expression} syntactic variables.  See
also section~\ref{expressionsection}.

\subsubsection*{Constant literals}\unsection

\begin{entry}{%
\pproto{\hyper{number}}{\exprtype}
\pproto{\hyper{boolean}}{\exprtype}
\pproto{\hyper{character}}{\exprtype}
\pproto{\hyper{string}}{\exprtype}
\pproto{\hyper{bytevector}}{\exprtype}}\mainindex{literal}

An expression consisting of a number, a boolean, a character, a
string, or a bytevector, evaluates ``to itself.

\begin{scheme}
145932     \ev  145932
\schtrue   \ev  \schtrue
"abc"      \ev  "abc"
\#vu8(2 24 123) \ev \#vu8(2 24 123)%
\end{scheme}

As noted in section~\ref{storagemodel}, the value of a literal
expression is immutable.
\end{entry}

\subsubsection*{Variable references}\unsection
\begin{entry}{%
\pproto{\hyper{variable}}{\exprtype}}

An expression consisting of a variable\index{variable}
(section~\ref{variablesection}) is a variable reference.  The value of
the variable reference is the value stored in the location to which the
variable is bound.  It is a syntax violation to reference
an unbound\index{unbound} variable.

\begin{scheme}
; These examples assume the base library
; has been imported.
(define x 28)
x   \ev  28%
\end{scheme}
\end{entry}

\subsubsection*{Procedure calls}\unsection

\begin{entry}{%
\pproto{(\hyper{operator} \hyperi{operand} \dotsfoo)}{\exprtype}}

A procedure call is written by simply enclosing in parentheses
expressions for the procedure to be called and the arguments to be
passed to it.  A form in an expression context is a procedure
call if \hyper{operator} is not an identifier bound as a syntactic keyword.

When a procedure call is evaluated, the operator and operand
expressions are evaluated (in an unspecified order) and the resulting
procedure is passed the resulting
arguments.\mainindex{call}\mainindex{procedure call}
\begin{scheme}%
; These examples assume the base library
; has been imported.
(+ 3 4)                          \ev  7
((if \schfalse + *) 3 4)         \ev  12%
\end{scheme}

If the value of \hyper{operator} is not a procedure, an exception with
condition type {\cf\&assertion} is raised.  Also, if \hyper{operator}
does not accept as many arguments as there are \hyper{operand}s, an
exception with condition type {\cf\&assertion} is raised.

\begin{note} In contrast to other dialects of Lisp, the order of
evaluation is unspecified, and the operator expression and the operand
expressions are always evaluated with the same evaluation rules.
\end{note}

\begin{note}
Although the order of evaluation is otherwise unspecified, the effect of
any concurrent evaluation of the operator and operand expressions is
constrained to be consistent with some sequential order of evaluation.
The order of evaluation may be chosen differently for each procedure call.
\end{note}

\begin{note} In many dialects of Lisp, the form {\tt
()} is a legitimate expression.  In Scheme, expressions written as
list/pair forms must have at
least one subexpression, so {\tt ()} is not a syntactically valid
expression.
\end{note}

\todo{Freeman:
I think an explanation as to why evaluation order is not specified
should be included.  It should not include any reference to parallel
evaluation.  Does any existing compiler generate better code because
the evaluation order is unspecified?  Clinger: yes: T3, MacScheme v2,
probably MIT Scheme and Chez Scheme.  But that's not the main reason
for leaving the order unspecified.}

\end{entry}

\subsection{Macros}
\label{macrosection}

Scheme libraries can define and use new derived expressions and
definitions called {\em syntactic abstractions} or
{\em macros}.\mainindex{syntactic abstraction}\mainindex{macro}
A syntactic abstraction is created by binding a keyword to a
{\em macro transformer} or, simply, {\em transformer}.
\index{macro transformer}\index{transformer}
The transformer determines
how a use of the macro is transcribed into a more primitive
form.

Most macro uses have the form:
\begin{scheme}
(\hyper{keyword} \hyper{datum} \dotsfoo)%
\end{scheme}%
where \hyper{keyword} is an identifier that uniquely determines the
type of form.  This identifier is called the {\em syntactic
keyword}\index{syntactic keyword}, or simply {\em
keyword}\index{keyword}, of the macro\index{macro keyword}.
The number of \hyper{datum}s and the syntax
of each depends on the syntactic abstraction.

Macro uses can also take the form of improper lists, singleton
identifiers, or {\cf set!} forms, where the second subform of the
{\cf set!} is the keyword (see section~\ref{identifier-syntax})
library section~\extref{lib:make-variable-transformer}{{\cf make-variable-transformer}}):
\begin{scheme}
(\hyper{keyword} \hyper{datum} \dotsfoo . \hyper{datum})
\hyper{keyword}
(set! \hyper{keyword} \hyper{datum})%
\end{scheme}

The {\cf define-syntax}, {\cf let-syntax} and {\cf letrec-syntax}
forms, described in sections~\ref{define-syntax} and \ref{let-syntax},
create bindings for keywords, associate them with macro transformers,
and control the scope within which they are visible.

The {\cf syntax-rules} and {\cf identifier-syntax} forms, described in
section~\ref{syntaxrulessection}, create transformers via a pattern
language.  Moreover, the {\cf syntax-case} form, described in library
chapter~\extref{lib:syntaxcasechapter}{{\cf syntax-case}}, 
creates transformers via a pattern language that permits the use of
arbitrary Scheme code.

Keywords occupy the same name space as variables.
That is, within the same
scope, an identifier can be bound as a variable or keyword, or neither, but
not both, and local bindings of either kind may shadow other bindings of
either kind.

Macros defined using {\cf syntax-rules} and {\cf identifier-syntax}
are ``hygienic'' and ``referentially transparent'' and thus preserve
Scheme's lexical scoping~\cite{Kohlbecker86,
  hygienic,Bawden88,macrosthatwork,syntacticabstraction}:
\mainindex{hygienic} \mainindex{referentially transparent}

\begin{itemize}
\item If a macro transformer inserts a binding for an identifier
(variable or keyword), the identifier is in effect renamed
throughout its scope to avoid conflicts with other identifiers.

\item If a macro transformer inserts a free reference to an
identifier, the reference refers to the binding that was visible
where the transformer was specified, regardless of any local
bindings that may surround the use of the macro.
\end{itemize}

Macros defined using the {\cf syntax-case} facility are also
hygienic unless {\cf datum\coerce{}syntax}
(see library section~\extref{lib:conversionssection}{Syntax-object and datum conversions}) is used.

\section{Examples}

Examples for various \hyper{import~spec}s and \hyper{export~spec}s:

\begin{scheme}
(library (stack)
  (export make push! pop! empty!)
  (import (rnrs (6)))

  (define (make) (list '()))
  (define (push! s v) (set-car! s (cons v (car s))))
  (define (pop! s) (let ([v (caar s)])
                     (set-car! s (cdar s))
                     v))
  (define (empty! s) (set-car! s '())))

(library (balloons)
  (export make push pop)
  (import (rnrs (6)))

  (define (make w h) (cons w h))
  (define (push b amt)
    (cons (- (car b) amt) (+ (cdr b) amt)))
  (define (pop b) (display "Boom! ") 
                  (display (* (car b) (cdr b))) 
                  (newline)))

(library (party)
  ;; Total exports:
  ;; make, push, push!, make-party, pop!
  (export (rename (balloon:make make)
                  (balloon:push push))
          push!
          make-party
          (rename (party-pop! pop!)))
  (import (rnrs (6))
          (only (stack) make push! pop!) ; not empty!
          (prefix (balloons) balloon:))

  ;; Creates a party as a stack of balloons,
  ;; starting with two balloons
  (define (make-party)
    (let ([s (make)]) ; from stack
      (push! s (balloon:make 10 10))
      (push! s (balloon:make 12 9))
      s))
  (define (party-pop! p)
    (balloon:pop (pop! p))))


(library (main)
  (export)
  (import (rnrs (6)) (party))

  (define p (make-party))
  (pop! p)        ; displays "Boom! 108"
  (push! p (push (make 5 5) 1))
  (pop! p))       ; displays "Boom! 24"%
\end{scheme}

Examples for macros and phases:

\begin{schemenoindent}
(library (my-helpers id-stuff)
  (export find-dup)
  (import (rnrs (6)))

  (define (find-dup l)
    (and (pair? l)
         (let loop ((rest (cdr l)))
           (cond
            [(null? rest) (find-dup (cdr l))]
            [(bound-identifier=? (car l) (car rest)) 
             (car rest)]
            [else (loop (cdr rest))])))))

(library (my-helpers values-stuff)
  (export mvlet)
  (import (rnrs (6)) (for (my-helpers id-stuff) expand))

  (define-syntax mvlet
    (lambda (stx)
      (syntax-case stx ()
        [(\_ [(id ...) expr] body0 body ...)
         (not (find-dup (syntax (id ...))))
         (syntax
           (call-with-values
               (lambda () expr) 
             (lambda (id ...) body0 body ...)))]))))

(library (let-div)
  (export let-div)
  (import (rnrs (6))
          (my-helpers values-stuff)
          (rnrs r5rs (6)))

  (define (quotient+remainder n d)
    (let ([q (quotient n d)])
      (values q (- n (* q d)))))
  (define-syntax let-div
    (syntax-rules ()
     [(\_ n d (q r) body0 body ...)
      (mvlet [(q r) (quotient+remainder n d)]
        body0 body ...)])))%
\end{schemenoindent}


%%% Local Variables: 
%%% mode: latex
%%% TeX-master: "r6rs"
%%% End: 
 \par
\chapter{Top-level programs}
\label{programchapter}

A \defining{top-level program} specifies an entry point for defining and running
a Scheme program.  A top-level program specifies a set of libraries to import and
code to run.  Through the imported libraries, whether directly or through the
transitive closure of importing, a top-level program defines a complete Scheme
program.

Top-level programs follow the convention of many common platforms of accepting 
a list of string \defining{command-line arguments} that may be used to
pass data to the script.

\section{Top-level program syntax}
\label{programsyntaxsection}

A top-level program is a delimited piece of text, typically a file, that follows
the following syntax:
%
\begin{grammar}
\meta{toplevel program} \: \meta{import form} \meta{toplevel body}
\meta{import form} \: (import \arbno{\meta{import spec}})
\meta{toplevel body} \: \arbno{\meta{toplevel body form}}
\meta{toplevel body form} \: \meta{definition} \| \meta{expression}
\end{grammar}
%
The rules for \meta{toplevel program} specify syntax at the form level.

The \meta{import form} is identical to the import clause in
libraries (see section~\ref{librarysyntaxsection}), 
and specifies a set of libraries to import.  A \meta{toplevel 
 body} is like a \meta{library body} (see
section~\ref{librarybodysection}), except that 
definitions and expressions may occur in any order.  Thus, the syntax
specified by \meta{toplevel body form} refers to the result of macro
expansion.

\begin{rationale}
By allowing the interleaving of definitions and expressions, top-level 
programs support exploratory and interactive development, without 
imposing unnecessary organizational overhead on code which may not be 
intended for reuse.
\end{rationale}

When base-library {\cf begin} forms occur anywhere within a top-level body,
they are spliced into the body; see section~\ref{begin}.
Some or all of the top-level body, including portions wrapped in {\cf begin}
forms, may be specified by a syntactic abstraction
(see section~\ref{macrosection}).

\section{Top-level program semantics}

A top-level program is executed by treating the program similarly to a library, and
evaluating its definitions and expressions.
The semantics of a top-level body may be roughly explained by
a simple translation into a library body: 
Each \hyper{expression} that appears before a
definition in
the top-level body is converted into a dummy definition 
{\cf (define \hyper{variable} (begin \hyper{expression} \hyper{unspecified}))},
where \hyper{variable} is a fresh identifier and \hyper{unspecified}
is a side-effect-free expression returning \unspecifiedreturn.
(It is generally impossible to determine which forms are 
definitions and expressions without concurrently expanding the body, so
the actual translation is somewhat more complicated; see
chapter~\ref{expansionchapter}.)

On platforms that support it, a top-level program may access its command-line 
arguments by calling the {\cf command-line} procedure (see library 
section~\extref{lib:command-line}{Command-line access and exit values}).

%%% Local Variables: 
%%% mode: latex
%%% TeX-master: "r6rs"
%%% End: 
 \par
\chapter{Expansion process}
\label{expansionchapter}

Macro uses (see section~\ref{macrosection}) are expanded into \textit{core
forms}\mainindex{core form} at the start of evaluation (before compilation or interpretation)
by a syntax \emph{expander}.
(The set of core forms is implementation-dependent, as is the
representation of these forms in the expander's output.)
If the expander encounters a syntactic abstraction, it invokes
the associated transformer to expand the syntactic abstraction, then
repeats the expansion process for the form returned by the transformer.
If the expander encounters a core form, it recursively
processes the subforms, if any, and reconstructs the form from the
expanded subforms.
Information about identifier bindings is maintained during expansion
to enforce lexical scoping for variables and keywords.

To handle internal definitions, the expander processes the initial
forms in a \hyper{body} (see section~\ref{bodiessection}) or
\hyper{library body} (see section~\ref{librarybodysection})
from left to
right.  How the expander processes each form encountered as it does so
depends upon the kind of form.

\begin{description}
\item[macro use]
The expander invokes the associated transformer to transform the macro
use, then recursively performs whichever of these actions are appropriate
for the resulting form.

\item[{\cf define-syntax} form]
The expander expands and evaluates the right-hand-side expression and binds the
keyword to the resulting transformer.

\item[{\cf define} form]
The expander records the fact that the defined identifier is a variable but defers
expansion of the right-hand-side expression until after all of the
definitions have been processed.

\item[{\cf begin} form]
The expander splices the subforms into the list of body forms it is
processing.  (See section~\ref{begin}.)

\item[{\cf let-syntax} or {\cf letrec-syntax} form]
The expander splices the inner body forms into the list of (outer) body forms it is
processing, arranging for the keywords bound by the {\cf let-syntax}
and {\cf letrec-syntax} to be visible only in the inner body forms.

\item[expression, i.e., nondefinition]
The expander completes the expansion of the deferred right-hand-side forms
and the current and remaining expressions in the body, then
constructs a residual {\cf letrec*} form from the defined variables,
expanded right-hand-side expressions, and expanded body expressions.
\end{description}

It is a syntax violation
if the keyword that identifies one of the body forms
as a definition (derived or core) is redefined by the same definition or a
later definition in the same body.
To detect this error, the expander records the identifying keyword for each
macro use, {\cf define-syntax} form, {\cf define}
form, {\cf begin} form, {\cf let-syntax} form, and {\cf letrec-syntax}
form it encounters while processing the definitions and checks each
newly defined identifier ({\cf define} or {\cf define-syntax}
left-hand side) against the recorded keywords, as with
{\cf bound-identifier=?} (see library section~\ref{lib:identifierpredicatessection}).
For example, the following forms are syntax violations.

\begin{scheme}
(let ()
  (define define 17)
  define)

(let-syntax ([def0 (syntax-rules ()
                     [(\_ x) (define x 0)])])
  (let ()
    (def0 z)
    (define def0 '(def 0))
    (list z def0)))
\end{scheme}

Expansion of each variable definition right-hand side is deferred until
after all of the definitions have been seen so that each keyword and
variable reference within the right-hand side resolves to the local
binding, if any.

Note that this algorithm does not directly reprocess any form.
It requires a single left-to-right pass over the definitions followed by a
single pass (in any order) over the body expressions and deferred
right-hand sides.

For example, in

\begin{scheme}
(lambda (x)
  (define-syntax defun
    (syntax-rules ()
      [(\_ (x . a) e) (define x (lambda a e))]))
  (defun (even? n) (or (= n 0) (odd? (- n 1))))
  (define-syntax odd?
    (syntax-rules () [(\_ n) (not (even? n))]))
  (odd? (if (odd? x) (* x x) x)))
\end{scheme}

The definition of {\cf defun} is encountered first, and the keyword
{\cf defun} is associated with the transformer resulting from
the expansion and evaluation of the corresponding right-hand side.
A use of {\cf defun} is encountered next and expands into a
{\cf define} form.
Expansion of the right-hand side of this define form is deferred.
The definition of {\cf odd?} is next and results in the association
of the keyword {\cf odd?} with the transformer resulting from
expanding and evaluating the corresponding right-hand side.
A use of {\cf odd?} appears next and is expanded; the resulting
call to {\cf not} is recognized as an expression
because {\cf not} is bound as a variable.
At this point, the expander completes the expansion of the current
expression (the {\cf not} call) and the deferred right-hand side of the
{\cf even?} definition;
the uses of {\cf odd?} appearing in these expressions are expanded
using the transformer associated with the keyword {\cf odd?}.
The final output is the equivalent of

\begin{scheme}
(lambda (x)
  (letrec* ([even?
              (lambda (n)
                (or (= n 0)
                    (not (even? (- n 1)))))])
    (not (even? (if (not (even? x)) (* x x) x)))))
\end{scheme}

although the structure of the output is implementation dependent.

Because definitions and expressions can be interleaved in a
\hyper{toplevel body} (see chapter~\ref{programchapter}),
the expander's processing of a \hyper{toplevel body} is somewhat
more complicated.
It behaves as described above for a \hyper{body} or
\hyper{library body} with the following exceptions.
When the expander finds a nondefinition,
it defers its expansion and continues scanning for definitions.
Once it reaches the end of set of forms, it processes the
deferred right-hand-side and body expressions, then
constructs a residual {\cf letrec*} form from the defined variables,
expanded right-hand-side expressions, and expanded body expressions.
For each body expression that appears before a variable definition
in the body, a dummy binding is created at the corresponding place within
the set of {\cf letrec*} bindings, with a fresh temporary variable on the
left-hand side and the expression on the right-hand side, so that
left-to-right evaluation order is preserved.

%%% Local Variables: 
%%% mode: latex
%%% TeX-master: "r6rs"
%%% End: 
 \par
%\vfill\eject
\chapter{Base library}
\label{baselibrarychapter}

This chapter describes Scheme's \deflibrary{r6rs base} library, which exports many of
the procedure and syntax bindings that are traditionally associated
with Scheme.

Section~\ref{basetailcontextsection} defines the rules that identify
tail calls and tail contexts in base-library constructs.

\section{Base types}
\label{disjointness}

No object satisfies more than one of the following predicates:

\begin{scheme}
boolean?          pair?
symbol?           number?
char?             string?
vector?           procedure?
unspecified?      null?%
\end{scheme}

These predicates define the base types {\em boolean}, {\em pair}, {\em
symbol}, {\em number}, {\em char} (or {\em character}), {\em string}, {\em
vector}, and {\em procedure}.  Moreover, the empty list is a special
object of its own type, as is the unspecified value.
\mainindex{type}\schindex{boolean?}\schindex{pair?}\schindex{symbol?}
\schindex{number?}\schindex{char?}\schindex{string?}\schindex{vector?}
\schindex{procedure?}\index{empty list}\index{unspecified value}
\schindex{unspecified?}\schindex{null?}

Note that, although there is a separate boolean type, any Scheme value
can be used as a boolean value for the purpose of a conditional test;
see section~\ref{booleanvaluessection}.

\section{Definitions}
\label{defines}

The {\cf define} forms described in this section are definitions for
value bindings and may appear anywhere other definitions may appear.
See section~\ref{librarybodysection}.

A \hyper{definition} must have one of the following forms:\mainschindex{define}\mainindex{definition}

\begin{itemize}

\item{\tt(define \hyper{variable} \hyper{expression})}
  This binds \hyper{variable} to a new
  location before assigning the value of \hyper{expression} to it.
\begin{scheme}
(define add3
  (lambda (x) (+ x 3)))
(add3 3)                            \ev  6
(define first car)
(first '(1 2))                      \ev  1%
\end{scheme}

\item{\tt(define \hyper{variable})}

This form is equivalent to
\begin{scheme}
(define \hyper{variable} (unspecified))%
\end{scheme}

\item{\tt(define (\hyper{variable} \hyper{formals}) \hyper{body})}

\hyper{Formals} must be either a
sequence of zero or more variables, or a sequence of one or more
variables followed by a space-delimited period and another variable (as
in a lambda expression, see section~\ref{lambda}).  This form is equivalent to
\begin{scheme}
(define \hyper{variable}
  (lambda (\hyper{formals}) \hyper{body}))\rm.%
\end{scheme}

\item{\tt(define (\hyper{variable} .\ \hyper{formal}) \hyper{body})}

\hyper{Formal} must be a single
variable.  This form is equivalent to
\begin{scheme}
(define \hyper{variable}
  (lambda \hyper{formal} \hyper{body}))\rm.%
\end{scheme}

\item a syntax definition (see section~\ref{define-syntax})
\end{itemize}

\section{Syntax definitions}
\label{syntaxdefinitionsection}

Syntax definitions are established with {\cf define-syntax}.
A {\cf define-syntax} form is a \hyper{definition} and may appear
anywhere other definitions may appear.

\begin{entry}{%
\proto{define-syntax}{ \hyper{variable} \hyper{expression}}{\exprtype}}

This binds the keyword \hyper{variable} to the value of
\hyper{expression}, which must evaluate, at macro-expansion
time, to a transformer.  (See library section~\extref{lib:transformerssection}{Transformers}).

Keyword bindings established by {\cf define-syntax} are visible
throughout the body in which they appear, except where shadowed by
other bindings, and nowhere else, just like variable bindings established
by {\cf define}.
All bindings established by a set of internal definitions, whether
keyword or variable definitions, are visible within the definitions
themselves.
For example:

\begin{scheme}
(let ()
  (define even?
    (lambda (x)
      (or (= x 0) (odd? (- x 1)))))
  (define-syntax odd?
    (syntax-rules ()
      ((odd?  x) (not (even? x)))))
  (even? 10))                       \ev \schtrue{}%
\end{scheme}

An implication of the left-to-right processing order
(section~\ref{expansionchapter}) is that one internal definition can
affect whether a subsequent form is also a definition.  For example,
the expression

\begin{scheme}
(let ()
  (define-syntax bind-to-zero
    (syntax-rules ()
      ((bind-to-zero id) (define id 0))))
  (bind-to-zero x)
  x) \ev 0%
\end{scheme}

This behavior is irrespective of any binding for
{\cf bind-to-zero} that might appear outside of the {\cf let}
expression.
\end{entry}

\section{Bodies and sequences}
\label{bodiessection}

\index{body}The body \hyper{body} of a \ide{lambda}, \ide{let}, \ide{let*},
\ide{let-values}, \ide{let*-values}, \ide{letrec*}, \ide{letrec}
expression or that of a
definition with a body has the following form:

{\cf \hyper{definition} \ldots{} \hyper{sequence}}

\hyper{Sequence} has the following form:

{\cf \hyperi{expression} \hyperii{expression} \ldots}

Definitions may occur in a \hyper{body}.
Such definitions are known as {\em internal definitions}
\mainindex{internal definition} as opposed to library body
definitions. 

With \ide{lambda}, \ide{let}, \ide{let*}, \ide{let-values},
\ide{let*-values}, \ide{letrec*}, and \ide{letrec},
the identifier defined by an internal
definition is local to the \hyper{body}.  That is, the identifier is
bound, and the region of the binding is the
entire \hyper{body}.  For example,

\begin{scheme}
(let ((x 5))
  (define foo (lambda (y) (bar x y)))
  (define bar (lambda (a b) (+ (* a b) a)))
  (foo (+ x 3)))                \ev  45%
\end{scheme}

When base-library {\cf begin} forms occur in a body prior to the first
expression, they are spliced into the body; see section~\ref{begin}.
Some or all of the body, including portions wrapped in {\cf begin}
forms, may be specified by a syntactic abstraction
(see section~\ref{macrosection}).

An expanded \hyper{body} (see chapter~\ref{expansionchapter})
containing internal definitions can
always be converted into an equivalent {\cf letrec*}
expression.  For example, the {\cf let} expression in the above
example is equivalent to

\begin{scheme}
(let ((x 5))
  (letrec* ((foo (lambda (y) (bar x y)))
            (bar (lambda (a b) (+ (* a b) a))))
    (foo (+ x 3))))%
\end{scheme}

\section{Expressions}
\label{expressionsection}

The entries in this section describe the expressions of the base
language, which may occur in the position of the \hyper{expression}
syntactic variable.  The expressions also include constant literals,
variable references and procedure calls as described in
section~\ref{primitiveexpressionsection}.

\subsection{Literal expressions}\unsection
\label{literalsection}

\begin{entry}{%
\proto{quote}{ \hyper{datum}}{\exprtype}}

\syntax \hyper{Datum} should be a datum value.

\semantics
{\cf (quote \hyper{datum})} evaluates to the datum
denoted by \hyper{datum}.
(See
section~\ref{readsyntaxsection}.).  This notation is used to include literal
constants in Scheme code.

\begin{scheme}%
(quote a)                     \ev  a
(quote \sharpsign(a b c))     \ev  \#(a b c)
(quote (+ 1 2))               \ev  (+ 1 2)%
\end{scheme}

As noted in section~\ref{quotesection}, {\cf (quote \hyper{datum})}
may be abbreviated as \singlequote\hyper{datum}:

\begin{scheme}
'"abc"               \ev  "abc"
'145932              \ev  145932
'a                   \ev  a
'\#(a b c)           \ev  \#(a b c)
'()                  \ev  ()
'(+ 1 2)             \ev  (+ 1 2)
'(quote a)           \ev  (quote a)
''a                  \ev  (quote a)%
\end{scheme}

As noted in section~\ref{storagemodel}, the value of a literal
expression may be immutable.
\end{entry}

\subsection{Procedures}\unsection
\label{lamba}

\begin{entry}{%
\proto{lambda}{ \hyper{formals} \hyper{body}}{\exprtype}}

\syntax
\hyper{Formals} must be a formal arguments list as described below,
and \hyper{body} must be according to section~\ref{bodiessection}.

\semantics
\vest A \lambdaexp{} evaluates to a procedure.  The environment in
effect when the \lambdaexp{} is evaluated is remembered as part of the
procedure.  When the procedure is later called with some actual
arguments, the environment in which the \lambdaexp{} was evaluated is
extended by binding the variables in the formal argument list to
fresh locations, and the resulting actual argument values are stored
in those locations.  Then, the expressions in the body of the \lambdaexp{}
(which may contain internal definitions and thus represent a {\cf
  letrec*} form, see section~\ref{bodiessection}) are evaluated
sequentially in the extended environment.
The results of the last expression in the body are returned as
the results of the procedure call.

\begin{scheme}
(lambda (x) (+ x x))      \ev  {\em{}a procedure}
((lambda (x) (+ x x)) 4)  \ev  8

((lambda (x)
   (define (p y)
     (+ y 1))
   (+ (p x) x))
 5) \ev 11

(define reverse-subtract
  (lambda (x y) (- y x)))
(reverse-subtract 7 10)         \ev  3

(define add4
  (let ((x 4))
    (lambda (y) (+ x y))))
(add4 6)                        \ev  10%
\end{scheme}

\hyper{Formals} must have one of the following forms:

\begin{itemize}
\item {\tt(\hyperi{variable} \dotsfoo)}:
The procedure takes a fixed number of arguments; when the procedure is
called, the arguments are stored in the bindings of the
corresponding variables.

\item \hyper{variable}:
The procedure takes any number of arguments; when the procedure is
called, the sequence of actual arguments is converted into a newly
allocated list, and the list is stored in the binding of the
\hyper{variable}.

\item {\tt(\hyperi{variable} \dotsfoo{} \hyper{variable$_{n}$}\ {\bf.}\
\hyper{variable$_{n+1}$})}:
If a space-delimited period precedes the last variable, then
the procedure takes $n$ or more arguments, where $n$ is the
number of formal arguments before the period (there must
be at least one).
The value stored in the binding of the last variable is a
newly allocated
list of the actual arguments left over after all the other actual
arguments have been matched up against the other formal arguments.
\end{itemize}

\begin{scheme}
((lambda x x) 3 4 5 6)          \ev  (3 4 5 6)
((lambda (x y . z) z)
 3 4 5 6)                       \ev  (5 6)%
\end{scheme}

It is a syntax violation for a \hyper{variable} to appear more than once in
\hyper{formals}.

Each procedure created as the result of evaluating a \lambdaexp{} is
(conceptually) tagged
with a storage location, in order to make \ide{eqv?} and
\ide{eq?} work on procedures (see section~\ref{equivalencesection}).

\end{entry}


\subsection{Conditionals}\unsection

\begin{entry}{%
\proto{if}{ \hyper{test} \hyper{consequent} \hyper{alternate}}{\exprtype}
\rproto{if}{ \hyper{test} \hyper{consequent}}{\exprtype}}  %\/ if hyper = italic

\syntax
\hyper{Test}, \hyper{consequent}, and \hyper{alternate} must be 
expressions.

\semantics
An {\cf if} expression is evaluated as follows: first,
\hyper{test} is evaluated.  If it yields a true value\index{true} (see
section~\ref{booleanvaluessection}), then \hyper{consequent} is evaluated and
its value(s) is(are) returned.  Otherwise \hyper{alternate} is evaluated and its
value(s) is(are) returned.  If \hyper{test} yields a false value and no
\hyper{alternate} is specified, then the result of the expression is
the unspecified value.

\begin{scheme}
(if (> 3 2) 'yes 'no)           \ev  yes
(if (> 2 3) 'yes 'no)           \ev  no
(if (> 3 2)
    (- 3 2)
    (+ 3 2))                    \ev  1
(if \#f \#f)                    \ev \theunspecified%
\end{scheme}

\end{entry}


\subsection{Assignments}\unsection
\label{assignment}

\begin{entry}{%
\proto{set!}{ \hyper{variable} \hyper{expression}}{\exprtype}}

\hyper{Expression} is evaluated, and the resulting value is stored in
the location to which \hyper{variable} is bound.  \hyper{Variable} must
be bound either in some region\index{region} enclosing the {\cf set!}\ expression
or at the top level of a library body.  The result of the {\cf set!} expression is
the unspecified value.

\begin{scheme}
(let ((x 2))
  (+ x 1)
  (set! x 4)
  (+ x 1)) \ev  5%
\end{scheme}

It is a syntax violation if \hyper{variable} refers to an
immutable binding.
\end{entry}

\subsection{Derived conditionals}\unsection

\begin{entry}{%
\proto{cond}{ \hyperi{clause} \hyperii{clause} \dotsfoo}{\exprtype}}

\syntax
Each \hyper{clause} must be of the form
\begin{scheme}
(\hyper{test} \hyperi{expression} \dotsfoo)%
\end{scheme}
where \hyper{test} is any expression.  Alternatively, a \hyper{clause} may be
of the form
\begin{scheme}
(\hyper{test} => \hyper{expression})%
\end{scheme}
The last \hyper{clause} may be
an ``else clause'', which has the form
\begin{scheme}
(else \hyperi{expression} \hyperii{expression} \dotsfoo)\rm.%
\end{scheme}
\mainschindex{else}
\mainschindex{=>}

\semantics
A {\cf cond} expression is evaluated by evaluating the \hyper{test}
expressions of successive \hyper{clause}s in order until one of them
evaluates to a true value\index{true} (see
section~\ref{booleanvaluessection}).  When a \hyper{test} evaluates to a true
value, then the remaining \hyper{expression}s in its \hyper{clause} are
evaluated in order, and the result(s) of the last \hyper{expression} in the
\hyper{clause} is(are) returned as the result(s) of the entire {\cf cond}
expression.  If the selected \hyper{clause} contains only the
\hyper{test} and no \hyper{expression}s, then the value of the
\hyper{test} is returned as the result.  If the selected \hyper{clause} uses the
\ide{=>} alternate form, then the \hyper{expression} is evaluated.
Its value must be a procedure that accepts one argument; this procedure is then
called on the value of the \hyper{test} and the value(s) returned by this
procedure is(are) returned by the {\cf cond} expression.
If all \hyper{test}s evaluate
to false values, and there is no else clause, then the result of
the conditional expression is the unspecified value; if there is an else
clause, then its \hyper{expression}s are evaluated, and the value(s) of
the last one is(are) returned.

\begin{scheme}
(cond ((> 3 2) 'greater)
      ((< 3 2) 'less))         \ev  greater%

(cond ((> 3 3) 'greater)
      ((< 3 3) 'less)
      (else 'equal))            \ev  equal%

(cond ('(1 2 3) => cadr)
      (else \schfalse{}))         \ev  2%
\end{scheme}

A sample definition of {\cf cond} in terms of simpler forms is in
appendix~\ref{derivedformsappendix}.
\end{entry}


\begin{entry}{%
\proto{case}{ \hyper{key} \hyperi{clause} \hyperii{clause} \dotsfoo}{\exprtype}}

\syntax
\hyper{Key} must be any expression.  Each \hyper{clause} has one of
the following forms:
\begin{scheme}
((\hyperi{datum} \dotsfoo) \hyperi{expression} \hyperii{expression} \dotsfoo)
(else \hyperi{expression} \hyperii{expression} \dotsfoo)%
\end{scheme}
\schindex{else}
The second form, which specifies an ``else clause'',
may only appear as the last \hyper{clause}.
Each \hyper{datum} is an external representation of some object.
The datums denoted by the \hyper{datum}s need not be distinct.

\semantics
A {\cf case} expression is evaluated as follows.  \hyper{Key} is
evaluated and its result is compared against the datums
denoted by the \hyper{datum}s of each \hyper{clause} in turn, proceding
in order from left to right through the set of clauses.  If the
result of evaluating \hyper{key} is equivalent (in the sense of
{\cf eqv?}; see section~\ref{eqv?}) to a datum of a \hyper{clause}, the
corresponding \hyper{expression}s are evaluated from left
to right and the results of the last expression in the \hyper{clause} are
returned as the results of the {\cf case} expression.  Otherwise, the
comparison process continues.  If the result of
evaluating \hyper{key} is different from every datum in each set, then if
there is an else clause its expressions are evaluated and the
results of the last are the results of the {\cf case} expression;
otherwise the result of the {\cf case} expression is the unspecified value.

\begin{scheme}
(case (* 2 3)
  ((2 3 5 7) 'prime)
  ((1 4 6 8 9) 'composite))     \ev  composite
(case (car '(c d))
  ((a) 'a)
  ((b) 'b))                     \ev  \theunspecified
(case (car '(c d))
  ((a e i o u) 'vowel)
  ((w y) 'semivowel)
  (else 'consonant))            \ev  consonant%
\end{scheme}

% A sample definition of {\cf case} in terms of simpler forms is in
% appendix~\ref{derivedformsappendix}.
\end{entry}


\begin{entry}{%
\proto{and}{ \hyperi{test} \dotsfoo}{\exprtype}}

\syntax The \hyper{test}s must be expressions.

\semantics
The \hyper{test} expressions are evaluated from left to right, and the
value of the first expression that evaluates to a false value (see
section~\ref{booleanvaluessection}) is returned.  Any remaining expressions
are not evaluated.  If all the expressions evaluate to true values, the
value of the last expression is returned.  If there are no expressions
then \schtrue{} is returned.

\begin{scheme}
(and (= 2 2) (> 2 1))           \ev  \schtrue
(and (= 2 2) (< 2 1))           \ev  \schfalse
(and 1 2 'c '(f g))             \ev  (f g)
(and)                           \ev  \schtrue%
\end{scheme}

The {\cf and} keyword could be defined in terms of {\cf if} using {\cf
  syntax-rules} (see section~\ref{syntaxrulessection}) as follows:

\begin{scheme}
(define-syntax \ide{and}
  (syntax-rules ()
    ((and) \sharpfoo{t})
    ((and test) test)
    ((and test1 test2 ...)
     (if test1 (and test2 ...) \sharpfoo{f}))))%
\end{scheme}
\end{entry}


\begin{entry}{%
\proto{or}{ \hyperi{test} \dotsfoo}{\exprtype}}

\syntax The \hyper{test}s must be expressions.

\semantics
The \hyper{test} expressions are evaluated from left to right, and the value of the
first expression that evaluates to a true value (see
section~\ref{booleanvaluessection}) is returned.  Any remaining expressions
are not evaluated.  If all expressions evaluate to false values, the
value of the last expression is returned.  If there are no
expressions then \schfalse{} is returned.

\begin{scheme}
(or (= 2 2) (> 2 1))            \ev  \schtrue
(or (= 2 2) (< 2 1))            \ev  \schtrue
(or \schfalse \schfalse \schfalse) \ev  \schfalse
(or '(b c) (/ 3 0))             \ev  (b c)%
\end{scheme}

The {\cf or} keyword could be defined in terms of {\cf if} using {\cf
  syntax-rules} (see section~\ref{syntaxrulessection}) as follows:

\begin{scheme}
(define-syntax \ide{or}
  (syntax-rules ()
    ((or) \sharpfoo{f})
    ((or test) test)
    ((or test1 test2 ...)
     (let ((x test1))
       (if x x (or test2 ...))))))%
\end{scheme}
\end{entry}


\subsection{Binding constructs}

The four binding constructs {\cf let}, {\cf let*}, {\cf letrec}, and {\cf letrec*}
give Scheme a block structure, like Algol 60.  The syntax of the four
constructs is identical, but they differ in the regions\index{region} they establish
for their variable bindings.  In a {\cf let} expression, the initial
values are computed before any of the variables become bound; in a
{\cf let*} expression, the bindings and evaluations are performed
sequentially.  In a {\cf letrec} or {\cf letrec*}
expression, all the bindings are in
effect while their initial values are being computed, thus allowing
mutually recursive definitions.  In a {\cf letrec} expression, the
initial values are computed before being assigned to the variables;
in a {\cf letrec*}, the evaluations and assignments are performed
sequentially.

In addition, the binding constructs {\cf let-values} and {\cf
  let*-values} allow the binding of results of expression returning
multiple values.  They are analogous to {\cf let} and {\cf let*} in the
way they establish regions: in a {\cf let-values} expression, the
initial values are computed before any of the variables become bound;
in a {\cf let*-values} expression, the bindings are performed
sequentially. 

\begin{note}
  These forms are compatible with SRFI~11~\cite{srfi11}.
\end{note}

\begin{entry}{%
\proto{let}{ \hyper{bindings} \hyper{body}}{\exprtype}}

\syntax
\hyper{Bindings} must have the form
\begin{scheme}
((\hyperi{variable} \hyperi{init}) \dotsfoo)\rm,%
\end{scheme}
where each \hyper{init} is an expression, and \hyper{body} 
is as described in section~\ref{bodiessection}.  It is a
syntax violation for a \hyper{variable} to appear more than once in the list of variables
being bound.

\semantics
The \hyper{init}s are evaluated in the current environment (in some
unspecified order), the \hyper{variable}s are bound to fresh locations
holding the results, the \hyper{body} is evaluated in the extended
environment, and the value(s) of the last expression of \hyper{body}
is(are) returned.  Each binding of a \hyper{variable} has \hyper{body} as its
region.\index{region}

\begin{scheme}
(let ((x 2) (y 3))
  (* x y))                      \ev  6

(let ((x 2) (y 3))
  (let ((x 7)
        (z (+ x y)))
    (* z x)))                   \ev  35%
\end{scheme}

See also named {\cf let}, section \ref{namedlet}.

\end{entry}


\begin{entry}{%
\proto{let*}{ \hyper{bindings} \hyper{body}}{\exprtype}}\nobreak

\nobreak
\syntax
\hyper{Bindings} must have the form
\begin{scheme}
((\hyperi{variable} \hyperi{init}) \dotsfoo)\rm,%
\end{scheme}
and \hyper{body} must be a sequence of
one or more expressions.

\semantics
The {\cf let*} form is similar to {\cf let}, but the bindings are performed
sequentially from left to right, and the region\index{region} of a binding indicated
by {\cf(\hyper{variable} \hyper{init})} is that part of the {\cf let*}
expression to the right of the binding.  Thus the second binding is done
in an environment in which the first binding is visible, and so on.

\begin{scheme}
(let ((x 2) (y 3))
  (let* ((x 7)
         (z (+ x y)))
    (* z x)))             \ev  70%
\end{scheme}

\begin{note}
  While a {\cf let} expression must not contain duplicate
  variables, a {\cf let*} expression can.
\end{note}

The {\cf let*} keyword could be defined in terms of {\cf let} using {\cf
  syntax-rules} (see section~\ref{syntaxrulessection}) as follows:

\begin{scheme}
(define-syntax \ide{let*}
  (syntax-rules ()
    ((let* () body1 body2 ...)
     (let () body1 body2 ...))
    ((let* ((name1 expr1) (name2 expr2) ...)
       body1 body2 ...)
     (let ((name1 expr1))
       (let* ((name2 expr2) ...)
         body1 body2 ...)))))%
\end{scheme}

\end{entry}

\begin{entry}{%
\proto{letrec}{ \hyper{bindings} \hyper{body}}{\exprtype}}

\syntax
\hyper{Bindings} must have the form
\begin{scheme}
((\hyperi{variable} \hyperi{init}) \dotsfoo)\rm,%
\end{scheme}
and \hyper{body} must be a sequence of
one or more expressions. It is a syntax violation for a \hyper{variable} to appear more
than once in the list of variables being bound.

\semantics
The \hyper{variable}s are bound to fresh locations, the \hyper{init}s
are evaluated in the resulting environment (in
some unspecified order), each \hyper{variable} is assigned to the result
of the corresponding \hyper{init}, the \hyper{body} is evaluated in the
resulting environment, and the value(s) of the last expression in
\hyper{body} is(are) returned.  Each binding of a \hyper{variable} has the
entire {\cf letrec} expression as its region\index{region}, making it possible to
define mutually recursive procedures.

\begin{scheme}
%(letrec ((x 2) (y 3))
%  (letrec ((foo (lambda (z) (+ x y z))) (x 7))
%    (foo 4)))                   \ev  14
%
(letrec ((even?
          (lambda (n)
            (if (zero? n)
                \schtrue
                (odd? (- n 1)))))
         (odd?
          (lambda (n)
            (if (zero? n)
                \schfalse
                (even? (- n 1))))))
  (even? 88))   
                \ev  \schtrue%
\end{scheme}

One restriction on {\cf letrec} is very important: it must be possible
to evaluate each \hyper{init} without assigning or referring to the value of any
\hyper{variable}.  If this restriction is violated, an exception  with
condition type {\cf\&assertion} is
raised.  The restriction is necessary because Scheme passes arguments by value rather than by
name.
In the most common uses of {\cf letrec}, all the \hyper{init}s are
\lambdaexp{}s and the restriction is satisfied automatically.

A sample definition of {\cf letrec} in terms of simpler forms is in
appendix~\ref{derivedformsappendix}.
\end{entry}

\begin{entry}{%
\proto{letrec*}{ \hyper{bindings} \hyper{body}}{\exprtype}}

\syntax
\hyper{Bindings} must have the form
\begin{scheme}
((\hyperi{variable} \hyperi{init}) \dotsfoo)\rm,%
\end{scheme}
and \hyper{body} must be a sequence of
one or more expressions. It is a syntax violation for a \hyper{variable} to appear more
than once in the list of variables being bound.

\semantics
The \hyper{variable}s are bound to fresh locations  undefined,
each \hyper{variable} is assigned in left-to-right order to the
result of evaluating the corresponding \hyper{init}, the \hyper{body} is
evaluated in the resulting environment, and the value(s) of the last
expression in \hyper{body} is(are) returned. 
Despite the left-to-right evaluation and assignment order, each binding of
a \hyper{variable} has the entire {\cf letrec*} expression as its
region\index{region}, making it possible to define mutually recursive
procedures.

\begin{scheme}
(letrec* ((p
           (lambda (x)
             (+ 1 (q (- x 1)))))
          (q
           (lambda (y)
             (if (zero? y)
                 0
                 (+ 1 (p (- y 1))))))
          (x (p 5))
          (y x))
  y)
                \ev  5%
\end{scheme}

One restriction on {\cf letrec*} is very important: it must be possible
to evaluate each \hyper{init} without assigning or referring to the value
the corresponding \hyper{variable} or the \hyper{variable} of any of
the bindings that follow it in \hyper{bindings}.
If this restriction is violated, an exception with condition type
{\cf\&assertion} is raised.
The restriction is necessary because Scheme passes arguments by value
rather than by name. 

The {\cf letrec*} keyword could be defined approximately in terms of {\cf let}
and {\cf set!}
using {\cf syntax-rules} (see section~\ref{syntaxrulessection})
as follows:

\begin{scheme}
(define-syntax \ide{letrec*}
  (syntax-rules ()
    ((letrec* ((var1 init1) ...) body1 body2 ...)
     (let ((var1 <undefined>) ...)
       (set! var1 init1)
       ...
       (let () body1 body2 ...)))))%
\end{scheme}

The syntax {\cf <undefined>} represents an expression that
returns something that, when stored in a location, causes an exception
with condition type {\cf\&assertion} to
be raised if an attempt to read from or write to the location occurs before the
assignments generated by the {\cf letrec*} transformation take place.
(No such expression is defined in Scheme.)
\end{entry}

\begin{entry}{%
\proto{let-values}{ \hyper{mv-bindings} \hyper{body}}{\exprtype}}

\syntax
\hyper{Mv-bindings} must have the form
\begin{scheme}
((\hyperi{formals} \hyperi{init}) \dotsfoo)\rm,%
\end{scheme}
and \hyper{body} is as described in section~\ref{bodiessection}. It is
a syntax violation for a variable to appear more
than once in the list of variables that appear as part of the formals.

\semantics The \hyper{init}s are evaluated in the current environment
(in some unspecified order), and the variables occurring in the
\hyper{formals} are bound to fresh locations containing the values
returned by the \hyper{init}s, where the \hyper{formals} are matched
to the return values in the same way that the \hyper{formals} in a
\lambdaexp{} are matched to the actual arguments in a procedure call.
Then, the \hyper{body} is evaluated in the extended environment, and the
value(s) of the last expression of \hyper{body} is(are) returned.
Each binding of a variable has \hyper{body} as its
region.\index{region}
If the \hyper{formals} do not match, an exception with condition type
{\cf\&assertion} is raised.

\begin{scheme}
(let-values (((a b) (values 1 2))
             ((c d) (values 3 4)))
  (list a b c d)) \ev (1 2 3 4)

(let-values (((a b . c) (values 1 2 3 4)))
  (list a b c))            \ev (1 2 (3 4))

(let ((a 'a) (b 'b) (x 'x) (y 'y))
  (let-values (((a b) (values x y))
               ((x y) (values a b)))
    (list a b x y)))       \ev (x y a b)%
\end{scheme}

A sample definition of {\cf let-values} in terms of simpler forms is in
appendix~\ref{derivedformsappendix}.
\end{entry}

\begin{entry}{%
\proto{let*-values}{ \hyper{mv-bindings} \hyper{body}}{\exprtype}}

The {\cf let*-values} form is the same as with {\cf let-values}, but the bindings are
processed sequentially from left to right, and the
region\index{region} of the bindings indicated by {\cf(\hyper{formals}
  \hyper{init})} is that part of the {\cf let*-values} expression to
the right of the bindings.  Thus, the second set of bindings is evaluated in
an environment in which the first set of bindings is visible, and so
on.

\begin{scheme}
(let ((a 'a) (b 'b) (x 'x) (y 'y))
  (let*-values (((a b) (values x y))
                ((x y) (values a b)))
    (list a b x y)))  \ev (x y x y)%
\end{scheme}

The following macro defines {\cf let*-values} in terms of {\cf let}
and {\cf let-values}:

\begin{scheme}
(define-syntax let*-values
  (syntax-rules ()
    ((let*-values () body1 body2 ...)
     (let () body1 body2 ...))
    ((let*-values (binding1 binding2 ...)
       body1 body2 ...)
     (let-values (binding1)
       (let*-values (binding2 ...)
         body1 body2 ...)))))%
\end{scheme}

\end{entry}

\subsection{Sequencing}\unsection

\begin{entry}{%
\proto{begin}{ \hyper{form} \dotsfoo}{\exprtype}
\rproto{begin}{ \hyper{expression} \hyper{expression} \dotsfoo}{\exprtype}}

The \hyper{begin} keyword has two different roles, depending on its
context:
\begin{itemize}
\item It may appear as a form in a \hyper{body} (see
  section~\ref{bodiessection}), \hyper{library body} (see
  section~\ref{librarybodysection}), or \hyper{toplevel body} (see
  chapter~\ref{programchapter}), or directly nested in a {\cf begin}
  form that appears in a body.  In this case, the {\cf begin} form
  must have the shape specified in the first header line.  This use of
  {\cf begin} acts as a \defining{splicing} form---the forms inside
  the \hyper{body} are spliced into the surrounding body, as if the
  {\cf begin} wrapper were not actually present.
  
  A {\cf begin} form in a \hyper{body} or \hyper{library body} must
  be non-empty if it appears after the first \hyper{expression}
  within the body.
\item It may appear as an ordinary expression and must have the shape
  specified in the second header line.  In this case, the
  \hyper{expression}s are evaluated sequentially from left to right,
  and the value(s) of the last \hyper{expression} is(are) returned.
  This expression type is used to sequence side effects such as
  assignments or input
  and output.
\end{itemize}

\begin{scheme}
(define x 0)

(begin (set! x 5)
       (+ x 1))                  \ev  6

(begin (display "4 plus 1 equals ")
       (display (+ 4 1)))      \ev  \unspecified
 \>{\em and prints}  4 plus 1 equals 5%
\end{scheme}

The following macro, which uses {\cf syntax-rules} (see
section~\ref{syntaxrulessection}), defines {\cf begin} in terms of {\cf
  lambda}.  Note that it only covers the expression case of {\cf begin}.
%
\begin{scheme}
(define-syntax \ide{begin}
  (syntax-rules ()
    ((begin exp ...)
     ((lambda () exp ...)))))%
\end{scheme}

The following alternative expansion for {\cf begin} does not make use of
the ability to write more than one expression in the body of a lambda
expression.  It, too, only covers the expression case of {\cf begin}.

\begin{scheme}
(define-syntax begin
  (syntax-rules ()
    ((begin exp)
     exp)
    ((begin exp1 exp2 ...)
     (call-with-values
         (lambda () exp1)
       (lambda ignored
         (begin exp2 ...))))))%
\end{scheme}

\end{entry}

\section{Equivalence predicates}
\label{equivalencesection}

A \defining{predicate} is a procedure that always returns a boolean
value (\schtrue{} or \schfalse).  An \defining{equivalence predicate} is
the computational analogue of a mathematical equivalence relation (it is
symmetric, reflexive, and transitive).  Of the equivalence predicates
described in this section, {\cf eq?}\ is the finest or most
discriminating, and {\cf equal?}\ is the coarsest.  The {\cf eqv?} predicate is
slightly less discriminating than {\cf eq?}.  \todo{Pitman doesn't like
this paragraph.  Lift the discussion from the Maclisp manual.  Explain
why there's more than one predicate.}


\begin{entry}{%
\proto{eqv?}{ \vari{obj} \varii{obj}}{procedure}}

The {\cf eqv?} procedure defines a useful equivalence relation on objects.
Briefly, it returns \schtrue{} if \vari{obj} and \varii{obj} should
normally be regarded as the same object.  This relation is left slightly
open to interpretation, but the following partial specification of
{\cf eqv?} holds for all implementations of Scheme.

The {\cf eqv?} procedure returns \schtrue{} if one of the following holds:

\begin{itemize}
\item \vari{Obj} and \varii{obj} are both \schtrue{} or both \schfalse.

\item \vari{Obj} and \varii{obj} are both symbols and

\begin{scheme}
(string=? (symbol->string obj1)
          (symbol->string obj2))
    \ev  \schtrue%
\end{scheme}

\item \vari{Obj} and \varii{obj} are both exact\index{exact} numbers,
  and are numerically equal (see {\cf =}, 
  section~\ref{genericarithmeticsection}).

\item \vari{Obj} and \varii{obj} are both inexact\index{inexact} numbers, are numerically
  equal (see {\cf =}, section~\ref{genericarithmeticsection}, and
  yield the same results (in the sense of {\cf eqv?}) when passed
  as arguments to any other procedure that can be defined
  as a finite composition of Scheme's standard arithmetic
  procedures.

\item \vari{Obj} and \varii{obj} are both characters and are the same
character according to the {\cf char=?} procedure
(section~\ref{charactersection}).

\item Both \vari{obj} and \varii{obj} are the empty list, or the unspecified value,
respectively.

\item \vari{Obj} and \varii{obj} are pairs, vectors, or strings that denote the
same locations in the store (section~\ref{storagemodel}).

\item \vari{Obj} and \varii{obj} are procedures whose location tags are
equal (section~\ref{lambda}).
\end{itemize}

The {\cf eqv?} procedure returns \schfalse{} if one of the following holds:

\begin{itemize}
\item \vari{Obj} and \varii{obj} are of different types
(section~\ref{disjointness}).

\item One of \vari{obj} and \varii{obj} is \schtrue{} but the other is
\schfalse{}.

\item \vari{Obj} and \varii{obj} are symbols but

\begin{scheme}
(string=? (symbol->string \vari{obj})
          (symbol->string \varii{obj}))
    \ev  \schfalse%
\end{scheme}

\item One of \vari{obj} and \varii{obj} is an exact number but the other is
        an inexact number.

\item \vari{Obj} and \varii{obj} are rational numbers for which the {\cf =} procedure
  returns \schfalse{}.

\item \vari{Obj} and \varii{obj} yield different results (in the sense of
  {\cf eqv?}) when passed as arguments to any other procedure
  that can be defined as a finite composition of Scheme's
  standard arithmetic procedures.

\item \vari{Obj} and \varii{obj} are characters for which the {\cf char=?}
  procedure returns \schfalse{}.

\item One of \vari{obj} and \varii{obj} is the empty list, 
  or the unspecified value, but the other is not.

\item \vari{Obj} and \varii{obj} are pairs, vectors, or strings that denote
distinct locations.

\item \vari{Obj} and \varii{obj} are procedures that would behave differently
(return different value(s) or have different side effects) for some arguments.

\end{itemize}

\begin{note}
  The {\cf eqv?} procedure returning \schtrue{} when \vari{obj} and
  \varii{obj} are numbers does not imply that {\cf =} would also
  return \schtrue{} when called with \vari{obj} and \varii{obj} as
  arguments.
\end{note}


\begin{scheme}
(eqv? 'a 'a)                     \ev  \schtrue
(eqv? 'a 'b)                     \ev  \schfalse
(eqv? 2 2)                       \ev  \schtrue
(eqv? '() '())                   \ev  \schtrue
(eqv? (unspecified) (unspecified)) \lev  \schtrue
(eqv? 100000000 100000000)       \ev  \schtrue
(eqv? (cons 1 2) (cons 1 2))     \ev  \schfalse
(eqv? (lambda () 1)
      (lambda () 2))             \ev  \schfalse
(eqv? \#f 'nil)                  \ev  \schfalse
(let ((p (lambda (x) x)))
  (eqv? p p))                    \ev  \schtrue%
\end{scheme}

The following examples illustrate cases in which the above rules do
not fully specify the behavior of {\cf eqv?}.  All that can be said
about such cases is that the value returned by {\cf eqv?} must be a
boolean.

\begin{scheme}
(eqv? "" "")             \ev  \unspecified
(eqv? '\#() '\#())         \ev  \unspecified
(eqv? (lambda (x) x)
      (lambda (x) x))    \ev  \unspecified
(eqv? (lambda (x) x)
      (lambda (y) y))    \ev  \unspecified
(eqv? +nan.0 +nan.0)             \ev \unspecified%
\end{scheme}

The next set of examples shows the use of {\cf eqv?}\ with procedures
that have local state.  Calls to {\cf gen-counter} must return a distinct
procedure every time, since each procedure has its own internal counter.
The {\cf gen-loser} procedure, however, returns equivalent procedures each time, since
the local state does not affect the value or side effects of the
procedures.

\begin{scheme}
(define gen-counter
  (lambda ()
    (let ((n 0))
      (lambda () (set! n (+ n 1)) n))))
(let ((g (gen-counter)))
  (eqv? g g))           \ev  \schtrue
(eqv? (gen-counter) (gen-counter))
                        \ev  \schfalse
(define gen-loser
  (lambda ()
    (let ((n 0))
      (lambda () (set! n (+ n 1)) 27))))
(let ((g (gen-loser)))
  (eqv? g g))           \ev  \schtrue
(eqv? (gen-loser) (gen-loser))
                        \ev  \unspecified

(letrec ((f (lambda () (if (eqv? f g) 'both 'f)))
         (g (lambda () (if (eqv? f g) 'both 'g))))
  (eqv? f g))
                        \ev  \unspecified

(letrec ((f (lambda () (if (eqv? f g) 'f 'both)))
         (g (lambda () (if (eqv? f g) 'g 'both))))
  (eqv? f g))
                        \ev  \schfalse%
\end{scheme}

Since the effect of trying to modify constant objects (those returned by
literal expressions) is unspecified, implementations are permitted, though not
required, to share structure between constants where appropriate.  Thus
the value of {\cf eqv?} on constants is sometimes
implementation-dependent.

\begin{scheme}
(eqv? '(a) '(a))                 \ev  \unspecified
(eqv? "a" "a")                   \ev  \unspecified
(eqv? '(b) (cdr '(a b)))         \ev  \unspecified
(let ((x '(a)))
  (eqv? x x))                    \ev  \schtrue%
\end{scheme}

\begin{rationale} 
The above definition of {\cf eqv?} allows implementations latitude in
their treatment of procedures and literals:  implementations are free
either to detect or to fail to detect that two procedures or two literals
are equivalent to each other, and can decide whether or not to
merge representations of equivalent objects by using the same pointer or
bit pattern to represent both.
\end{rationale}

\end{entry}


\begin{entry}{%
\proto{eq?}{ \vari{obj} \varii{obj}}{procedure}}

The {\cf eq?} predicate is similar to {\cf eqv?}\ except that in some cases it is
capable of discerning distinctions finer than those detectable by
{\cf eqv?}.

The {\cf eq?}\ and {\cf eqv?} predicates are guaranteed to have the same
behavior on symbols, booleans, the empty list, 
the unspecified value, pairs, procedures,
and non-empty
strings and vectors.  The behavior of {\cf eq?} on numbers and characters is
implementation-dependent, but it always returns either true or
false, and returns true only when {\cf eqv?}\ would also return
true.  The {\cf eq?} predicate may also behave differently from {\cf eqv?} on empty
vectors and empty strings.

\begin{scheme}
(eq? 'a 'a)                     \ev  \schtrue
(eq? '(a) '(a))                 \ev  \unspecified
(eq? (list 'a) (list 'a))       \ev  \schfalse
(eq? "a" "a")                   \ev  \unspecified
(eq? "" "")                     \ev  \unspecified
(eq? '() '())                   \ev  \schtrue
(eq? (unspecified) (unspecified)) \lev  \schtrue
(eq? 2 2)                       \ev  \unspecified
(eq? \#\backwhack{}A \#\backwhack{}A) \ev  \unspecified
(eq? car car)                   \ev  \schtrue
(let ((n (+ 2 3)))
  (eq? n n))      \ev  \unspecified
(let ((x '(a)))
  (eq? x x))      \ev  \schtrue
(let ((x '\#()))
  (eq? x x))      \ev  \schtrue
(let ((p (lambda (x) x)))
  (eq? p p))      \ev  \schtrue%
\end{scheme}

\todo{Needs to be explained better above.  How can this be made to be
not confusing?  A table maybe?}

\begin{rationale} It is usually possible to implement {\cf eq?}\ much
more efficiently than {\cf eqv?}, for example, as a simple pointer
comparison instead of as some more complicated operation.  One reason is
that it may not be possible to compute {\cf eqv?}\ of two numbers in
constant time, whereas {\cf eq?}\ implemented as pointer comparison will
always finish in constant time.  The {\cf eq?} predicate may be used like {\cf eqv?}\
in applications using procedures to implement objects with state since
it obeys the same constraints as {\cf eqv?}.
\end{rationale}

\end{entry}


\begin{entry}{%
\proto{equal?}{ \vari{obj} \varii{obj}}{procedure}}

The {\cf equal?}  predicate returns \schtrue{} if and only if the
(possibly infinite) unfoldings of its arguments into regular trees are
equal as ordered trees.

The {\cf equal?} predicate treats pairs and vectors
as nodes with outgoing edges, uses {\cf
  string=?} to compare strings, uses {\cf
  bytesvector=?} to compare bytevectors (see library chapter~\extref{lib:bytevectorschapter}{Bytevectors}),
  and uses {\cf eqv?} to compare other nodes.

\begin{scheme}
(equal? 'a 'a)                  \ev  \schtrue
(equal? '(a) '(a))              \ev  \schtrue
(equal? '(a (b) c)
        '(a (b) c))             \ev  \schtrue
(equal? "abc" "abc")            \ev  \schtrue
(equal? 2 2)                    \ev  \schtrue
(equal? (make-vector 5 'a)
        (make-vector 5 'a))     \ev  \schtrue
(equal? '\#vu8(1 2 3 4 5)
        (u8-list->bytevector
         '(1 2 3 4 5))          \ev  \schtrue
(equal? (lambda (x) x)
        (lambda (y) y))  \ev  \unspecified

(let* ((x (list 'a))
       (y (list 'a))
       (z (list x y)))
  (list (equal? z (list y x))
        (equal? z (list x x))))             \lev  (\schtrue{} \schtrue{})%
\end{scheme}

\end{entry}

\section{Procedure predicate}

\begin{entry}{%
\proto{procedure?}{ obj}{procedure}}

Returns \schtrue{} if \var{obj} is a procedure, otherwise returns \schfalse.

\begin{scheme}
(procedure? car)            \ev  \schtrue
(procedure? 'car)           \ev  \schfalse
(procedure? (lambda (x) (* x x)))   
                            \ev  \schtrue
(procedure? '(lambda (x) (* x x)))  
                            \ev  \schfalse%
\end{scheme}

\end{entry}

\section{Unspecified value}
\label{unspecifiedvalue}

\begin{entry}{%
\proto{unspecified}{}{procedure}}

Returns the unspecified value.\index{unspecified value} (See section
\ref{disjointness}.)
\end{entry}

\begin{note}
  The unspecified value is not a datum value, and thus has no external
  representation.
\end{note}

\begin{entry}{%
\proto{unspecified?}{ obj}{procedure}}

Returns \schtrue{} if \var{obj} is the unspecified value, otherwise
returns \schfalse.
\end{entry}

\section{Generic arithmetic}
\label{genericarithmeticsection}

The procedures described here implement arithmetic that is
generic over
the numerical tower described in chapter~\ref{numbertypeschapter}.
The generic procedures described in this section
accept both exact and inexact numbers as arguments,
performing coercions and selecting the appropriate operations
as determined by the numeric subtypes of their arguments.

Library chapter~\extref{lib:numberchapter}{Arithmetic} describes
libraries that define other numerical procedures.

\subsection{Propagation of exactness and inexactness}
\label{propagationsection}

The procedures listed below must return the correct exact result
provided all their arguments are exact:

\begin{scheme}
+            -            *
max          min          abs
numerator    denominator  gcd
lcm          floor        ceiling
truncate     round        rationalize
expt         real-part    imag-part
make-rectangular%
\end{scheme}

The procedures listed below must return the correct exact result
provided all their arguments are exact, and no divisors are zero:

\begin{scheme}
/
div          mod           div-and-mod
div0         mod0          div0-and-mod0%
\end{scheme}

The general rule is that the generic operations return the correct
exact result when all of their arguments are exact and the result is
mathematically well-defined, but return an inexact result when any
argument is inexact.  Exceptions to this rule include
{\cf sqrt}, {\cf exp}, {\cf log},
{\cf sin}, {\cf cos}, {\cf tan},
{\cf asin}, {\cf acos}, {\cf atan},
{\cf expt}, {\cf make-polar}, {\cf magnitude}, and {\cf angle}, which
are allowed (but not required) to return inexact results even when
given exact arguments, as indicated in the specification of these
procedures.

One general exception to the rule above is that an implementation may
return an exact result despite inexact arguments if that exact result
would be the correct result for all possible substitutions of exact
arguments for the inexact ones.  An example is {\cf (* 1.0 0)} which
may return either {\cf 0} (exact) or {\cf 0.0} (inexact).

\subsection{Representability of infinities and NaNs}
\label{infinitiesnanssection}

The specification of the numerical operations is written as though
infinities and NaNs are representable, and specifies many operations
with respect to these numbers in ways that are consistent with the
IEEE 754 standard for binary floating point arithmetic.  
An implementation of Scheme is not required to represent infinities and
NaNs, however;
an implementation must raise a continuable exception with
condition type {\cf\&no-infinities} or {\cf\&no-nans} (respectively;
see library section~\extref{lib:flonumssection}{Flonums})
whenever it is unable to represent an infinity or NaN as required by
the specification.  In this case, the continuation of the exception
handler is the continuation that otherwise would have received
the infinity or NaN value.  This requirement also applies to
conversions between numbers and external representations, including
the reading of program source code.

\subsection{Semantics of common operations}

Some operations are the semantic basis for several arithmetic
procedures.  The behavior of these operations is described in this
section for later reference.

\subsubsection{Integer division}
\label{integerdivision}

For various kinds of arithmetic (fixnum, flonum, exact, inexact, and
generic), Scheme provides operations for performing integer
division.  They rely on mathematical operations $\mathrm{div}$,
$\mathrm{mod}$, $\mathrm{div}_0$, and
$\mathrm{mod}_0$, that are defined as follows:

$\mathrm{div}$, $\mathrm{mod}$, $\mathrm{div}_0$, and $\mathrm{mod}_0$
each accept two real numbers $x_1$ and $x_2$ as operands, where
$x_2$ must be nonzero.

$\mathrm{div}$ returns an integer, and $\mathrm{mod}$ returns a real.
Their results are specified by
%
\begin{eqnarray*}
x_1~\mathrm{div}~x_2 &=& n_d\\
x_1~\mathrm{mod}~x_2 &=& x_m
\end{eqnarray*}
%
where
%
\begin{displaymath}
\begin{array}{c}
x_1 = n_d * x_2 + x_m\\
0 \leq x_m < |x_2|
\end{array}
\end{displaymath}
%
Examples:
\begin{eqnarray*}
5~\mathrm{div}~3    &=&  1\\
5~\mathrm{div}~-3   &=&  -1\\
5~\mathrm{mod}~3    &=&  2\\
5~\mathrm{mod}~-3   &=&  2
\end{eqnarray*}
%
$\mathrm{div}_0$ and $\mathrm{mod}_0$ are like $\mathrm{div}$ and
$\mathrm{mod}$, except the result of $\mathrm{mod}_0$ lies within a
half-open interval centered on zero.  The results are specified by
%
\begin{eqnarray*}
x_1~\mathrm{div}_0~x_2 &=& n_d\\
x_1~\mathrm{mod}_0~x_2 &=& x_m
\end{eqnarray*}
%
where:
%
\begin{displaymath}
\begin{array}{c}
x_1 = n_d * x_2 + x_m\\
-|\frac{x_2}{2}| \leq x_m < |\frac{x_2}{2}|
\end{array}
\end{displaymath}
%
Examples:
%
\begin{eqnarray*}
5~\mathrm{div}_0~3    &=&  2\\
5~\mathrm{div}_0~-3   &=&  -2\\
5~\mathrm{mod}_0~3    &=&  -1\\
5~\mathrm{mod}_0~-3   &=&  -1
\end{eqnarray*}

\begin{rationale}
The half-open symmetry about zero is convenient for some purposes.
\end{rationale}

\subsubsection{Transcendental functions}
\label{transcendentalfunctions}

In general, the transcendental functions $\log$, $\sin^{-1}$
(arcsine), $\cos^{-1}$ (arccosine), and $\tan^{-1}$ are multiply
defined.  The value of $\log z$ is defined to be the one whose
imaginary part lies in the range from $-\pi$ (inclusive if $-0.0$ is
distinguished, exclusive otherwise) to $\pi$ (inclusive).  $\log 0$ is
undefined.

The value of $\log z$ for non-real $z$ is defined in terms of log on real numbers as 

\begin{displaymath}
\log z = \log |z| + \mathrm{angle}~z
\end{displaymath}
%
where $\mathrm{angle}~z$ is the angle of $z = a\cdot e^{ib}$ specified
as:
$$\mathrm{angle}~z = b+2\pi n$$
with $-\pi \leq \mathrm{angle}~z\leq \pi$ and $\mathrm{angle}~z =
b+2\pi n$ for some integer $n$.

With the one-argument version of $\log$ defined this way, the values
of the two-argument-version of $\log$, $\sin^{-1} z$, $\cos^{-1} z$,
$\tan^{-1} z$, and the two-argument version of $\tan^{-1}$ are
according to the following formul\ae:
\begin{eqnarray*}
\log z~b &=& \frac{\log z}{\log b}\\
\sin^{-1} z &=& -i \log (i z + \sqrt{1 - z^2})\\
\cos^{-1} z &=& \pi / 2 - \sin^{-1} z\\
\tan^{-1} z &=& (\log (1 + i z) - \log (1 - i z)) / (2 i)\\
\tan^{-1} x~y &=& \mathrm{angle}(x+ yi)
\end{eqnarray*}

The range of $\tan^{-1} x~y$ is as in the following table. The
asterisk (*) indicates that the entry applies to implementations that
distinguish minus zero.

\begin{center}
\begin{tabular}{clll}
& $y$ condition & $x$ condition & range of result $r$\\\hline
& $y = 0.0$ & $x > 0.0$ & $0.0$\\
$\ast$ & $y = +0.0$  & $x > 0.0$ & $+0.0$\\     
$\ast$ & $y = -0.0$ & $x > 0.0$ & $-0.0$\\
& $y > 0.0$ & $x > 0.0$ & $0.0 < r < \frac{\pi}{2}$\\
& $y > 0.0$ & $x = 0.0$ & $\frac{\pi}{2}$\\
& $y > 0.0$ & $x < 0.0$ & $\frac{\pi}{2} < r < \pi$\\
& $y = 0.0$ & $x < 0$ & $\pi$\\
$\ast$ & $y = +0.0$ & $x < 0.0$ & $\pi$\\
$\ast$ & $y = -0.0$ & $x < 0.0$ & $-\pi$\\      
&$y < 0.0$ & $x < 0.0$ & $-\pi< r< -\frac{\pi}{2}$\\
&$y < 0.0$ & $x = 0.0$ & $-\frac{\pi}{2}$\\
&$y < 0.0$ & $x > 0.0$ & $-\frac{\pi}{2} < r< 0.0$\\    
&$y = 0.0$ & $x = 0.0$ & undefined\\
$\ast$& $y = +0.0$ & $x = +0.0$ & $+0.0$\\
$\ast$& $y = -0.0$ & $x = +0.0$& $-0.0$\\
$\ast$& $y = +0.0$ & $x = -0.0$ & $\pi$\\
$\ast$& $y = -0.0$ & $x = -0.0$ & $-\pi$\\
$\ast$& $y = +0.0$ & $x = 0$ & $\frac{\pi}{2}$\\
$\ast$& $y = -0.0$ & $x = 0$    & $-\frac{\pi}{2}$
\end{tabular}
\end{center}

The above specification follows Steele~\cite{CLtL}, which in turn
cites Penfield~\cite{Penfield81}; refer to these sources for more detailed
discussion of branch cuts, boundary conditions, and implementation of
these functions.

\subsection{Numerical operations}

\subsubsection{Numerical type predicates}

\begin{entry}{%
\proto{number?}{ obj}{procedure}
\proto{complex?}{ obj}{procedure}
\proto{real?}{ obj}{procedure}
\proto{rational?}{ obj}{procedure}
\proto{integer?}{ obj}{procedure}}

These numerical type predicates can be applied to any kind of
argument, including non-numbers.  They return \schtrue{} if the object is
of the named type, and otherwise they return \schfalse{}.
In general, if a type predicate is true of a number then all higher
type predicates are also true of that number.  Consequently, if a type
predicate is false of a number, then all lower type predicates are
also false of that number.

If \var{z} is a complex number, then {\cf (real? \var{z})} is true if
and only if {\cf (zero? (imag-part \var{z}))} and {\cf (exact?
  (imag-part \var{z}))} are both true.

If \var{x} is a real number, then {\cf (rational? \var{x})} is true if
and only if there exist exact integers \vari{k} and \varii{k} such that
{\cf (= \var{x} (/ \vari{k} \varii{k}))} and {\cf (= (numerator
  \var{x}) \vari{k})} and {\cf (= (denominator \var{x}) \varii{k})} are
all true.  Thus infinities and NaNs are not rational numbers.

If \var{q} is a rational number, then {\cf (integer?
\var{q})} is true if and only if {\cf (= (denominator
\var{q}) 1)} is true.  If \var{q} is not a rational number,
then {\cf (integer? \var{q})} is false.

\begin{scheme}
(complex? 3+4i)                        \ev  \schtrue{}
(complex? 3)                           \ev  \schtrue{}
(real? 3)                              \ev  \schtrue{}
(real? -2.5+0.0i)                      \ev  \schfalse{}
(real? -2.5+0i)                        \ev  \schtrue{}
(real? -2.5)                           \ev  \schtrue{}
(real? \sharpsign{}e1e10)                         \ev  \schtrue{}
(rational? 6/10)                       \ev  \schtrue{}
(rational? 6/3)                        \ev  \schtrue{}
(rational? 2)                          \ev  \schtrue{}
(integer? 3+0i)                        \ev  \schtrue{}
(integer? 3.0)                         \ev  \schtrue{}
(integer? 8/4)                         \ev  \schtrue{}

(number? +nan.0)                       \ev  \schtrue{}
(complex? +nan.0)                      \ev  \schtrue{}
(real? +nan.0)                         \ev  \schtrue{}
(rational? +nan.0)                     \ev  \schfalse{}
(complex? +inf.0)                      \ev  \schtrue{}
(real? -inf.0)                         \ev  \schtrue{}
(rational? -inf.0)                     \ev  \schfalse{}
(integer? -inf.0)                      \ev  \schfalse{}%
\end{scheme}

\begin{note}
The behavior of these type predicates on inexact numbers is
unreliable, because any inaccuracy may
affect the result.
\end{note}
\end{entry}

\begin{entry}{%
\proto{real-valued?}{ obj}{procedure}
\proto{rational-valued?}{ obj}{procedure}
\proto{integer-valued?}{ obj}{procedure}}

These numerical type predicates can be applied to any kind of
argument, including non-numbers.  The {\cf real-valued?} procedure
They return \schtrue{} if the object is a number and is equal in the
sense of {\cf =} to some real number, or if the object is a NaN, or a
complex number whose real part is a NaN and whose imaginary part zero
in the sense of {\cf zero?}.  The {\cf rational-valued?} and {\cf
  integer-valued?} procedures return \schtrue{} if the object is a
number and is equal in the sense of {\cf =} to some object of the
named type, and otherwise they return \schfalse{}.

\begin{scheme}
(real-valued? +nan.0)                  \ev  \schtrue{}
(real-valued? +nan.0+0i)                  \ev  \schtrue{}
(real-valued? -inf.0)                  \ev  \schtrue{}
(real-valued? 3)                       \ev  \schtrue{}
(real-valued? -2.5+0.0i)               \ev  \schtrue{}
(real-valued? -2.5+0i)                 \ev  \schtrue{}
(real-valued? -2.5)                    \ev  \schtrue{}
(real-valued? \sharpsign{}e1e10)                  \ev  \schtrue{}

(rational-valued? +nan.0)              \ev  \schfalse{}
(rational-valued? -inf.0)              \ev  \schfalse{}
(rational-valued? 6/10)                \ev  \schtrue{}
(rational-valued? 6/10+0.0i)           \ev  \schtrue{}
(rational-valued? 6/10+0i)             \ev  \schtrue{}
(rational-valued? 6/3)                 \ev  \schtrue{}

(integer-valued? 3+0i)                 \ev  \schtrue{}
(integer-valued? 3+0.0i)               \ev  \schtrue{}
(integer-valued? 3.0)                  \ev  \schtrue{}
(integer-valued? 3.0+0.0i)             \ev  \schtrue{}
(integer-valued? 8/4)                  \ev  \schtrue{}%
\end{scheme}

\begin{rationale}
  These procedures test whether a given number can be coerced to the
  specified type without loss of numerical accuracy.  Their behavior
  is different from the numerical type predicates in the previous
  entry, whose behavior is motivated by closure properties designed to
  enable statically predictable semantics and efficient implementation.
\end{rationale}

\begin{note}
The behavior of these type predicates on inexact numbers is
unreliable, because any inaccuracy may
affect the result.
\end{note}
\end{entry}

\begin{entry}{%
\proto{exact?}{ z}{procedure}
\proto{inexact?}{ z}{procedure}}

These numerical predicates provide tests for the exactness of a
quantity.  For any Scheme number, precisely one of these predicates is
true.

\begin{scheme}
(exact? 5)                   \ev  \schtrue{}
(inexact? +inf.0)            \ev  \schtrue{}%
\end{scheme}
\end{entry}

\subsubsection{Generic conversions}

\begin{entry}{%
\proto{->inexact}{ z}{procedure}
\proto{->exact}{ z}{procedure}}

{\cf ->inexact} returns an inexact representation of \var{z}.  If
inexact numbers of the appropriate type have bounded precision, then
the value returned is an inexact number that is nearest to the
argument.  If an exact argument has no reasonably close inexact
equivalent, an exception with condition type
{\cf\&implementation-violation} may be
raised.

{\cf ->exact} returns an exact representation of \var{z}.  The value
returned is the exact number that is numerically closest to the
argument; in most cases, the result of this procedure should be
numerically equal to its argument.  If an inexact argument has no
reasonably close exact equivalent, an exception with condition type
{\cf\&implementation-violation} may be
raised.

These procedures implement the natural one-to-one correspondence
between exact and inexact integers throughout an
implementation-dependent range.

{\cf ->inexact} and {\cf ->exact} are idempotent.
\end{entry}

\begin{entry}{%
\proto{real->flonum}{ x}{procedure}}

Returns the best flonum representation of
\var{x}.

The value returned is a flonum that is numerically closest to the
argument.

\begin{rationale}
  Not all reals are inexact, and some inexact reals may
  not be flonums.
\end{rationale}

\begin{note}
  If flonums are represented in binary floating point, then
  implementations are strongly encouraged to break ties by preferring
  the floating point representation whose least significant bit is
  zero.
\end{note}
\end{entry}

\begin{entry}{%
\proto{real->single}{ x}{procedure}
\proto{real->double}{ x}{procedure}}

Given a real number \var{x}, these procedures compute the best
IEEE-754 single or double precision approximation to \var{x} and
return that approximation as an inexact real.

\begin{note}
  Both of the two conversions performed by these procedures (to
  IEEE-754 single or double, and then to an inexact real) may lose
  precision, introduce error, or may underflow or overflow.
\end{note}

\begin{rationale}
  The ability to round to IEEE-754 single or double precision is
  occasionally needed for control of precision or for
  interoperability.
\end{rationale}

\end{entry}
\subsubsection{Arithmetic operations}

\begin{entry}{%
\proto{=}{ \vari{z} \varii{z} \variii{z} \dotsfoo}{procedure}
\proto{<}{ \vari{x} \varii{x} \variii{x} \dotsfoo}{procedure}
\proto{>}{ \vari{x} \varii{x} \variii{x} \dotsfoo}{procedure}
\proto{<=}{ \vari{x} \varii{x} \variii{x} \dotsfoo}{procedure}
\proto{>=}{ \vari{x} \varii{x} \variii{x} \dotsfoo}{procedure}}

These procedures return \schtrue{} if their arguments are
(respectively): equal, monotonically increasing, monotonically
decreasing, monotonically nondecreasing, or monotonically
nonincreasing, and \schfalse{} otherwise.

\begin{scheme}
(= +inf.0 +inf.0)           \ev  \schtrue{}
(= -inf.0 +inf.0)           \ev  \schfalse{}
(= -inf.0 -inf.0)           \ev  \schtrue{}%
\end{scheme}

For any real number \var{x} that is neither infinite nor NaN:

\begin{scheme}
(< -inf.0 \var{x} +inf.0))        \ev  \schtrue{}
(> +inf.0 \var{x} -inf.0))        \ev  \schtrue{}%
\end{scheme}

For any number \var{z}:
%
\begin{scheme}
(= +nan.0 \var{z})               \ev  \schfalse{}%
\end{scheme}
%
For any real number \var{x}:
%
\begin{scheme}
(< +nan.0 \var{x})               \ev  \schfalse{}
(> +nan.0 \var{x})               \ev  \schfalse{}%
\end{scheme}

These predicates are required to be transitive.

\begin{note}
The traditional implementations of these predicates in Lisp-like
languages are not transitive.
\end{note}

\begin{note}
While it is possible to compare inexact numbers using these
predicates, the results may be unreliable because a small inaccuracy
may affect the result; this is especially true of {\cf =} and {\cf zero?}.

When in doubt, consult a numerical analyst.
\end{note}
\end{entry}

\begin{entry}{%
\proto{zero?}{ z}{procedure}
\proto{positive?}{ x}{procedure}
\proto{negative?}{ x}{procedure}
\proto{odd?}{ n}{procedure}
\proto{even?}{ n}{procedure}
\proto{finite?}{ x}{procedure}
\proto{infinite?}{ x}{procedure}
\proto{nan?}{ x}{procedure}}

These numerical predicates test a number for a particular property,
returning \schtrue{} or \schfalse{}.  See note above.  The {\cf zero?}
procedure
tests if the number is {\cf =} to zero, {\cf positive?} tests whether it is
greater than zero, {\cf negative?} tests whether it is less than zero, {\cf
  odd?} tests whether it is odd, {\cf even?} tests whether it is even, {\cf
  finite?} tests whether it is not an infinity and not a NaN, {\cf
  infinite?} tests whether it is an infinity, {\cf nan?} tests whether it is a
NaN.

\begin{scheme}
(positive? +inf.0)            \ev  \schtrue{}
(negative? -inf.0)            \ev  \schtrue{}
(finite? +inf.0)              \ev  \schfalse{}
(finite? 5)                   \ev  \schtrue{}
(finite? 5.0)                 \ev  \schtrue{}
(infinite? 5.0)               \ev  \schfalse{}
(infinite? +inf.0)            \ev  \schtrue{}%
\end{scheme}
\end{entry}

\begin{entry}{%
\proto{max}{ \vari{x} \varii{x} \dotsfoo}{procedure}
\proto{min}{ \vari{x} \varii{x} \dotsfoo}{procedure}}

These procedures return the maximum or minimum of their arguments.

\begin{scheme}
(max 3 4)                              \ev  4    ; exact
(max 3.9 4)                            \ev  4.0  ; inexact%
\end{scheme}

For any real number \var{x}:

\begin{scheme}
(max +inf.0 \var{x})                         \ev  +inf.0
(min -inf.0 \var{x})                         \ev  -inf.0%
\end{scheme}

\begin{note}
If any argument is inexact, then the result is also inexact (unless
the procedure can prove that the inaccuracy is not large enough to affect the
result, which is possible only in unusual implementations).  If {\cf min} or
{\cf max} is used to compare numbers of mixed exactness, and the numerical
value of the result cannot be represented as an inexact number without loss of
accuracy, then the procedure may raise an exception with condition
type {\cf\&implementation-restriction}.
\end{note}

\end{entry}

\begin{entry}{%
\proto{+}{ \vari{z} \dotsfoo}{procedure}
\proto{*}{ \vari{z} \dotsfoo}{procedure}}

These procedures return the sum or product of their arguments.

\begin{scheme}
(+ 3 4)                                \ev  7
(+ 3)                                  \ev  3
(+)                                    \ev  0
(+ +inf.0 +inf.0)                      \ev  +inf.0
(+ +inf.0 -inf.0)                      \ev  +nan.0

(* 4)                                  \ev  4
(*)                                    \ev  1
(* 5 +inf.0)                           \ev  +inf.0
(* -5 +inf.0)                          \ev  -inf.0
(* +inf.0 +inf.0)                      \ev  +inf.0
(* +inf.0 -inf.0)                      \ev  -inf.0
(* 0 +inf.0)                           \ev  0 \textit{or} +nan.0
(* 0 +nan.0)                           \ev  0 \textit{or} +nan.0
(* 1.0 0)                              \ev  0 \textit{or} 0.0%
\end{scheme}

For any real number \var{x} that is neither infinite nor NaN:

\begin{scheme}
(+ +inf.0 \var{x})                           \ev  +inf.0
(+ -inf.0 \var{x})                           \ev  -inf.0
(+ +nan.0 \var{x})                           \ev  +nan.0%
\end{scheme}

For any real number \var{x} that is neither
infinite nor NaN nor an exact 0:

\begin{scheme}
(* +nan.0 \var{x})                           \ev  +nan.0%
\end{scheme}

If any of these procedures are applied to mixed non-rational real and
non-real complex arguments, they either raise an exception with
condition type {\cf\&implementation-restriction} or return an unspecified number.
\end{entry}

\begin{entry}{%
\proto{-}{ z}{procedure}
\rproto{-}{ \vari{z} \varii{z} \dotsfoo}{procedure}
\proto{/}{ z}{procedure}
\rproto{/}{ \vari{z} \varii{z} \dotsfoo}{procedure}}

With two or more arguments, these procedures return the difference or
quotient of their arguments, associating to the left.  With one
argument, however, they return the additive or multiplicative inverse
of their argument.

\begin{scheme}
(- 3 4)                                \ev  -1
(- 3 4 5)                              \ev  -6
(- 3)                                  \ev  -3
(- +inf.0 +inf.0)                      \ev  +nan.0

(/ 3 4 5)                              \ev  3/20
(/ 3)                                  \ev  1/3
(/ 0.0)                                \ev  +inf.0
(/ 1.0 0)                              \ev  +inf.0
(/ -1 0.0)                             \ev  -inf.0
(/ +inf.0)                             \ev  0.0
(/ 0 0)                                \lev \exception{\&assertion} \textit{or} +nan.0
(/ 0 3.5)                              \ev  0.0 ; inexact
(/ 0 0.0)                              \ev  +nan.0
(/ 0.0 0)                              \ev  +nan.0
(/ 0.0 0.0)                            \ev  +nan.0%
\end{scheme}

If any of these procedures are applied to mixed non-rational real and
non-real complex arguments, they either raise an exception with
condition type {\cf\&implementation-restriction} or return an
unspecified number.
\end{entry}

\begin{entry}{%
\proto{abs}{ x}{procedure}}

Returns the absolute value of its argument.

\begin{scheme}
(abs -7)                               \ev  7
(abs -inf.0)                           \ev  +inf.0%
\end{scheme}

\end{entry}

\begin{entry}{%
\proto{div-and-mod}{ \vari{x} \varii{x}}{procedure}
\proto{div}{ \vari{x} \varii{x}}{procedure}
\proto{mod}{ \vari{x} \varii{x}}{procedure}
\proto{div0-and-mod0}{ \vari{x} \varii{x}}{procedure}
\proto{div0}{ \vari{x} \varii{x}}{procedure}
\proto{mod0}{ \vari{x} \varii{x}}{procedure}}

These procedures implement number-theoretic integer division and
return the results of the corresponding mathematical operations
specified in section~\ref{integerdivision}.  In each case, \vari{x}
must be neither infinite nor a NaN, and \varii{x} must be nonzero;
otherwise, an exception with condition type {\cf\&assertion} is raised.

\begin{scheme}
(div \vari{x} \varii{x})         \ev \(\vari{x}~\mathrm{div}~\varii{x}\)
(mod \vari{x} \varii{x})         \ev \(\vari{x}~\mathrm{mod}~\varii{x}\)
(div-and-mod \vari{x} \varii{x})     \ev \(\vari{x}~\mathrm{div}~\varii{x}, \vari{x}~\mathrm{mod}~\varii{x}\)\\\>\>\>; two return values
(div0 \vari{x} \varii{x})        \ev \(\vari{x}~\mathrm{div}_0~\varii{x}\)
(mod0 \vari{x} \varii{x})        \ev \(\vari{x}~\mathrm{mod}_0~\varii{x}\)
(div0-and-mod0 \vari{x} \varii{x})   \lev \(\vari{x}~\mathrm{div}_0~\varii{x}, \vari{x}~\mathrm{mod}_0~\varii{x}\)\\\>\>; two return values%
\end{scheme}

\begin{entry}{%
\proto{gcd}{ \vari{n} \dotsfoo}{procedure}
\proto{lcm}{ \vari{n} \dotsfoo}{procedure}}

These procedures return the greatest common divisor or least common
multiple of their arguments.  The result is always non-negative.

\begin{scheme}
(gcd 32 -36)                           \ev  4
(gcd)                                  \ev  0
(lcm 32 -36)                           \ev  288
(lcm 32.0 -36)                         \ev  288.0 ; inexact
(lcm)                                  \ev  1%
\end{scheme}
\end{entry}

\begin{entry}{%
\proto{numerator}{ q}{procedure}
\proto{denominator}{ q}{procedure}}

These procedures return the numerator or denominator of their
argument; the result is computed as if the argument was represented as
a fraction in lowest terms.  The denominator is always positive.  The
denominator of $0$ is defined to be $1$.

\begin{scheme}
(numerator (/ 6 4))                    \ev  3
(denominator (/ 6 4))                  \ev  2
(denominator
 (->inexact (/ 6 4)))                  \ev  2.0%
\end{scheme}
\end{entry}

\begin{entry}{%
\proto{floor}{ x}{procedure}
\proto{ceiling}{ x}{procedure}
\proto{truncate}{ x}{procedure}
\proto{round}{ x}{procedure}}

These procedures return inexact integers on inexact arguments that are
not infinities or NaNs, and exact integers on exact rational
arguments.  For such arguments, {\cf floor} returns the largest
integer not larger than \var{x}.  The {\cf ceiling} procedure returns the smallest
integer not smaller than \var{x}.  The {\cf truncate} procedure returns the integer
closest to \var{x} whose absolute value is not larger than the
absolute value of \var{x}.  The {\cf round} procedure returns the closest integer to
\var{x}, rounding to even when \var{x} is halfway between two
integers.

\begin{rationale}
The {\cf round} procedure rounds to even for consistency with the default rounding
mode specified by the IEEE floating point standard.
\end{rationale}

\begin{note}
If the argument to one of these procedures is inexact, then the result
is also inexact.  If an exact value is needed, the
result should be passed to the {\cf ->exact} procedure.
\end{note}

Although infinities and NaNs are not integers, these procedures return
an infinity when given an infinity as an argument, and a NaN when
given a NaN.

\begin{scheme}
(floor -4.3)                           \ev  -5.0
(ceiling -4.3)                         \ev  -4.0
(truncate -4.3)                        \ev  -4.0
(round -4.3)                           \ev  -4.0

(floor 3.5)                            \ev  3.0
(ceiling 3.5)                          \ev  4.0
(truncate 3.5)                         \ev  3.0
(round 3.5)                            \ev  4.0  ; inexact

(round 7/2)                            \ev  4    ; exact
(round 7)                              \ev  7

(floor +inf.0)                         \ev  +inf.0
(ceiling -inf.0)                       \ev  -inf.0
(round +nan.0)                         \ev  +nan.0%
\end{scheme}

\end{entry}

\begin{entry}{%
\proto{rationalize}{ \vari{x} \varii{x}}{procedure}}

The {\cf rationalize} procedure returns the {\em simplest} rational number
differing from \vari{x} by no more than \varii{x}.    A rational number $r_1$ is
{\em simpler} \mainindex{simplest rational} than another rational number
$r_2$ if $r_1 = p_1/q_1$ and $r_2 = p_2/q_2$ (in lowest terms) and $|p_1|
\leq |p_2|$ and $|q_1| \leq |q_2|$.  Thus $3/5$ is simpler than $4/7$.
Although not all rationals are comparable in this ordering (consider $2/7$
and $3/5$) any interval contains a rational number that is simpler than
every other rational number in that interval (the simpler $2/5$ lies
between $2/7$ and $3/5$).  Note that $0 = 0/1$ is the simplest rational of
all.

\begin{scheme}
(rationalize
  (->exact .3) 1/10)                   \ev 1/3    ; exact
(rationalize .3 1/10)                  \ev \sharpsign{}i1/3  ; inexact

(rationalize +inf.0 3)                 \ev  +inf.0
(rationalize +inf.0 +inf.0)            \ev  +nan.0
(rationalize 3 +inf.0)                 \ev  0.0%
\end{scheme}

\end{entry}

\begin{entry}{%
\proto{exp}{ z}{procedure}
\proto{log}{ z}{procedure}
\rproto{log}{ \vari{z} \varii{z}}{procedure}
\proto{sin}{ z}{procedure}
\proto{cos}{ z}{procedure}
\proto{tan}{ z}{procedure}
\proto{asin}{ z}{procedure}
\proto{acos}{ z}{procedure}
\proto{atan}{ z}{procedure}
\rproto{atan}{ \vari{x} \varii{x}}{procedure}}

These procedures compute the usual transcendental functions.  The {\cf
  exp} procedure computes the base-$e$ exponential of \var{z}. 
The {\cf log} procedure with a single argument computes the natural logarithm of
\var{z} (not the base ten logarithm); {\cf (log \vari{z}
  \varii{z})} computes the base-\varii{z} logarithm of \vari{z}.
The {\cf asin}, {\cf acos}, and {\cf atan} procedures compute arcsine,
arccosine, and arctangent, respectively.  The two-argument variant of
{\cf atan} computes {\cf (angle (make-rectangular \varii{x}
\vari{x}))}.

See section~\ref{transcendentalfunctions} for the underlying
mathematical operations. These procedures may return inexact results
even when given exact arguments.

\begin{scheme}
(exp +inf.0)                   \ev +inf.0
(exp -inf.0)                   \ev 0.0
(log +inf.0)                   \ev +inf.0
(log 0.0)                      \ev -inf.0
(log 0)                        \lev \exception{\&assertion}
(log -inf.0)                   \ev +inf.0+\(\pi\)i
(atan -inf.0)                  \lev -1.5707963267948965 ; approximately
(atan +inf.0)                  \lev 1.5707963267948965 ; approximately
(log -1.0+0.0i)                \ev 0.0+\(\pi\)i
(log -1.0-0.0i)                \ev 0.0-\(\pi\)i\\\>; if -0.0 is distinguished%
\end{scheme}
\end{entry}

\begin{entry}{%
\proto{sqrt}{ z}{procedure}}

Returns the principal square root of \var{z}.  For rational \var{z},
the result has either positive real part, or zero real part and
non-negative imaginary part.  With $\log$ defined as in
section~\ref{transcendentalfunctions}, the value of {\cf (sqrt
  \var{z})} could be expressed as
%
\begin{displaymath}
e^{\frac{\log \var{z}}{2}}.
\end{displaymath}

The {\cf sqrt} procedure may return an inexact result even when given an exact
argument.

\begin{scheme}
(sqrt -5)                   \lev  0.0+2.23606797749979i ; approximately
(sqrt +inf.0)               \ev  +inf.0
(sqrt -inf.0)               \ev  +inf.0i%
\end{scheme}
\end{entry}

\begin{entry}{%
\proto{exact-integer-sqrt}{ k}{procedure}}

The {\cf exact-integer-sqrt} procedure returns two non-negative exact
integers $s$ and $r$ where $\var{ei} = s^2 +
r$ and $\var{ei} < (s+1)^2$.
\end{entry}

\begin{entry}{%
\proto{expt}{ \vari{z} \varii{z}}{procedure}}

Returns \vari{z} raised to the power \varii{z}.  For nonzero \vari{z},
%
\begin{displaymath}
  \vari{z}^{\varii{z}} = e^{\varii{z} \log \vari{z}}
\end{displaymath}

$0.0^{\var{z}}$ is $1.0$ if $\var{z} = 0.0$, and $0.0$ if {\cf
  (real-part \var{z})} is positive.  For other cases in which
the first argument is zero, an exception is raised with
condition type {\cf\&implementation-restriction} or an unspecified
number is returned.

For an exact real \vari{z} and an exact
integer \varii{z}, {\cf (expt \vari{z}
\varii{z})} must return an exact result.  For all other
values of \vari{z} and \varii{z}, {\cf (expt \vari{z}
\varii{z})} may return an inexact result, even when both
\vari{z} and \varii{z} are exact.

\begin{scheme}
(expt 5 3)                  \ev  125
(expt 5 -3)                 \ev  1/125
(expt 5 0)                  \ev  1
(expt 0 5)                  \ev  0
(expt 0 5+.0000312i)        \ev  0
(expt 0 -5)                 \ev  \unspecified
(expt 0 -5+.0000312i)       \ev  \unspecified
(expt 0 0)                  \ev  1
(expt 0.0 0.0)              \ev  1.0%
\end{scheme}
\end{entry}

\begin{entry}{%
\proto{make-rectangular}{ \vari{x} \varii{x}}{procedure}
\proto{make-polar}{ \variii{x} \variv{x}}{procedure}
\proto{real-part}{ z}{procedure}
\proto{imag-part}{ z}{procedure}
\proto{magnitude}{ z}{procedure}
\proto{angle}{ z}{procedure}}

Suppose \vari{x}, \varii{x}, \variii{x}, and \variv{x} are real
numbers and \var{z} is a complex number such that
%
\begin{displaymath}
\var{z} = \vari{x} + \varii{x}i = \variii{x} e^{i\variv{x}}.
\end{displaymath}

Then:
%
\begin{scheme}
(make-rectangular \vari{x} \varii{x}) \ev \var{z}
(make-polar \variii{x} \variv{x}) \ev \var{z}
(real-part \var{z})              \ev \vari{x}
(imag-part \var{z})              \ev \varii{x}
(magnitude \var{z})              \ev |\variii{x}|
(angle \var{z})                  \ev \var{x}\(_{\mathrm{angle}}\)%
\end{scheme}
%
where $-\pi \leq \var{x}_{\mathrm{angle}} \leq \pi$ with
$\var{x}_{\mathrm{angle}} = \variv{x} + 2\pi n$ for
some integer $n$.

\begin{scheme}
(angle -1.0)         \ev \(\pi\)
(angle -1.0+0.0)     \ev \(\pi\)
(angle -1.0-0.0)     \ev -\(\pi\)\\\>; if -0.0 is distinguished%
\end{scheme}

Moreover, suppose \vari{x}, \varii{x} are such that either \vari{x}
or \varii{x} is an infinity, then
%
\begin{scheme}
(make-rectangular \vari{x} \varii{x}) \ev \var{z}
(magnitude \var{z})              \ev +inf.0%
\end{scheme}
\end{entry}

The {\cf make-polar}, {\cf magnitude}, and
{\cf angle} procedures may return inexact results even when given exact
arguments.

\begin{scheme}
(angle -1)                    \ev \(\pi\)
(angle +inf.0)                \ev 0.0
(angle -inf.0)                \ev \(\pi\)
(angle -1.0+0.0)              \ev \(\pi\)
(angle -1.0-0.0)              \ev \(-\pi\)\\\>; if -0.0 is distinguished%
\end{scheme}
\end{entry}

\subsubsection{Numerical Input and Output}

\begin{entry}{%
\proto{number->string}{ z}{procedure}
\rproto{number->string}{ z radix}{procedure}
\rproto{number->string}{ z radix precision}{procedure}}

\var{Radix} must be an exact integer, either 2, 8, 10, or 16.  If
omitted, \var{radix} defaults to 10.  If a \var{precision} is
specified, then \var{z} must be an inexact complex number,
\var{precision} must be an exact positive integer, and \var{radix}
must be 10.  The {\cf number->string} procedure takes a number and a
radix and returns as a string an external representation of the given
number in the given radix such that
%
\begin{scheme}
(let ((number \var{number})
      (radix \var{radix}))
  (eqv? number
        (string->number (number->string number
                                        radix)
                        radix)))%
\end{scheme}
%
is true.  If no possible result makes this expression
true, an exception with condition type
{\cf\&implementation-restriction} is raised.

If a \var{precision} is specified, then the representations of the
inexact real components of the result, unless they are infinite or
NaN, specify an explicit \meta{mantissa width} \var{p}, and \var{p} is the
least $\var{p} \geq \var{precision}$ for which the above expression is
true.

If \var{z} is inexact, the radix is 10, and the above expression and
condition can be satisfied by a result that contains a decimal point,
then the result contains a decimal point and is expressed using the
minimum number of digits (exclusive of exponent, trailing zeroes, and
mantissa width) needed to make the above expression and condition
true~\cite{howtoprint,howtoread}; otherwise the format of the result
is unspecified.

The result returned by {\cf number->string} never contains an explicit
radix prefix.

\begin{note}
The error case can occur only when \var{z} is not a complex number
or is a complex number with a non-rational real or imaginary part.
\end{note}

\begin{rationale}
If \var{z} is an inexact number represented using binary floating
point, and the radix is 10, then the above expression is normally satisfied by
a result containing a decimal point.  The unspecified case
allows for infinities, NaNs, and representations other than binary
floating point.
\end{rationale}
\end{entry}

\begin{entry}{%
\proto{string->number}{ string}{procedure}
\rproto{string->number}{ string radix}{procedure}}

Returns a number of the maximally precise representation expressed by the
given \var{string}.  \var{Radix} must be an exact integer, either 2, 8, 10,
or 16.  If supplied, \var{radix} is a default radix that may be overridden
by an explicit radix prefix in \var{string} (e.g. {\tt "\#o177"}).  If \var{radix}
is not supplied, then the default radix is 10.  If \var{string} is not
a syntactically valid notation for a number, then {\cf string->number}
returns \schfalse{}.
%
\begin{scheme}
(string->number "100")                 \ev  100
(string->number "100" 16)              \ev  256
(string->number "1e2")                 \ev  100.0
(string->number "15\sharpsign\sharpsign")                \ev  1500.0
(string->number "+inf.0")              \ev  +inf.0
(string->number "-inf.0")              \ev  -inf.0
(string->number "+nan.0")              \ev  +nan.0%
\end{scheme}
\end{entry}


\section{Booleans}
\label{booleansection}

The standard boolean objects for true and false are written as
\schtrue{} and \schfalse.\sharpindex{t}\sharpindex{f} However, of all
the standard Scheme values, only \schfalse{} counts as false in
conditional expressions.  See section~\ref{booleanvaluessection}.

\begin{note}
Programmers accustomed to other dialects of Lisp should be aware that
Scheme distinguishes both \schfalse{} and the empty list \index{empty list}
from the symbol \ide{nil}.
\end{note}

\begin{entry}{%
\proto{not}{ obj}{procedure}}

Returns \schtrue{} if \var{obj} is false, and returns
\schfalse{} otherwise.

\begin{scheme}
(not \schtrue)   \ev  \schfalse
(not 3)          \ev  \schfalse
(not (list 3))   \ev  \schfalse
(not \schfalse)  \ev  \schtrue
(not '())        \ev  \schfalse
(not (list))     \ev  \schfalse
(not 'nil)       \ev  \schfalse%
\end{scheme}

\end{entry}


\begin{entry}{%
\proto{boolean?}{ obj}{procedure}}

Returns \schtrue{} if \var{obj} is either \schtrue{} or
\schfalse{} and returns \schfalse{} otherwise.

\begin{scheme}
(boolean? \schfalse)  \ev  \schtrue
(boolean? 0)          \ev  \schfalse
(boolean? '())        \ev  \schfalse%
\end{scheme}

\end{entry}

 
\section{Pairs and lists}
\label{listsection}

A \defining{pair} (sometimes called a \defining{dotted pair}) is a
record structure with two fields called the car and cdr fields (for
historical reasons).  Pairs are created by the procedure {\cf cons}.
The car and cdr fields are accessed by the procedures {\cf car} and
{\cf cdr}.

Pairs are used primarily to represent lists.  A list can
be defined recursively as either the empty list\index{empty list} or a pair whose
cdr is a list.  More precisely, the set of lists is defined as the smallest
set \var{X} such that

\begin{itemize}
\item The empty list is in \var{X}.
\item If \var{list} is in \var{X}, then any pair whose cdr field contains
      \var{list} is also in \var{X}.
\end{itemize}

The objects in the car fields of successive pairs of a list are the
elements of the list.  For example, a two-element list is a pair whose car
is the first element and whose cdr is a pair whose car is the second element
and whose cdr is the empty list.  The length of a list is the number of
elements, which is the same as the number of pairs.

The empty list\mainindex{empty list} is a special object of its own type.
It is not a pair.  It has no elements and its length is zero.

\begin{note}
The above definitions imply that all lists have finite length and are
terminated by the empty list.
\end{note}

A chain of pairs not ending in the empty list is called an
\defining{improper list}.  Note that an improper list is not a list.
The list and dotted notations can be combined to represent
improper lists:

\begin{scheme}
(a b c . d)%
\end{scheme}

is equivalent to

\begin{scheme}
(a . (b . (c . d)))%
\end{scheme}

Whether a given pair is a list depends upon what is stored in the cdr
field.

\begin{entry}{%
\proto{pair?}{ obj}{procedure}}

Returns \schtrue{} if \var{obj} is a pair, and otherwise
returns \schfalse.

\begin{scheme}
(pair? '(a . b))        \ev  \schtrue
(pair? '(a b c))        \ev  \schtrue
(pair? '())             \ev  \schfalse
(pair? '\#(a b))         \ev  \schfalse%
\end{scheme}
\end{entry}


\begin{entry}{%
\proto{cons}{ \vari{obj} \varii{obj}}{procedure}}

Returns a newly allocated pair whose car is \vari{obj} and whose cdr is
\varii{obj}.  The pair is guaranteed to be different (in the sense of
{\cf eqv?}) from every existing object.

\begin{scheme}
(cons 'a '())           \ev  (a)
(cons '(a) '(b c d))    \ev  ((a) b c d)
(cons "a" '(b c))       \ev  ("a" b c)
(cons 'a 3)             \ev  (a . 3)
(cons '(a b) 'c)        \ev  ((a b) . c)%
\end{scheme}
\end{entry}


\begin{entry}{%
\proto{car}{ pair}{procedure}}

Returns the contents of the car field of \var{pair}.

\begin{scheme}
(car '(a b c))          \ev  a
(car '((a) b c d))      \ev  (a)
(car '(1 . 2))          \ev  1
(car '())               \ev  \exception{\&assertion}%
\end{scheme}
 
\end{entry}


\begin{entry}{%
\proto{cdr}{ pair}{procedure}}

Returns the contents of the cdr field of \var{pair}.

\begin{scheme}
(cdr '((a) b c d))      \ev  (b c d)
(cdr '(1 . 2))          \ev  2
(cdr '())               \ev  \exception{\&assertion}%
\end{scheme}
 
\end{entry}



\setbox0\hbox{\tt(cadr \var{pair})}
\setbox1\hbox{procedure}


\begin{entry}{%
\proto{caar}{ pair}{procedure}
\proto{cadr}{ pair}{procedure}
\texonly
\pproto{\hbox to 1\wd0 {\hfil$\vdots$\hfil}}{\hbox to 1\wd1 {\hfil$\vdots$\hfil}}
\endtexonly
\htmlonly $\vdots$ \endhtmlonly
\proto{cdddar}{ pair}{procedure}
\proto{cddddr}{ pair}{procedure}}

These procedures are compositions of {\cf car} and {\cf cdr}, where
for example {\cf caddr} could be defined by

\begin{scheme}
(define caddr (lambda (x) (car (cdr (cdr x))))){\rm.}%
\end{scheme}

Arbitrary compositions, up to four deep, are provided.  There are
twenty-eight of these procedures in all.

\end{entry}


\begin{entry}{%
\proto{null?}{ obj}{procedure}}

Returns \schtrue{} if \var{obj} is the empty list\index{empty list}.
Otherwise, returns \schfalse.

\end{entry}

\begin{entry}{%
\proto{list?}{ obj}{procedure}}

Returns \schtrue{} if \var{obj} is a list.  Otherwise, returns \schfalse{}.
By definition, all lists are chains of pairs that have finite length and are terminated by
the empty list.

\begin{scheme}
(list? '(a b c))     \ev  \schtrue
(list? '())          \ev  \schtrue
(list? '(a . b))     \ev  \schfalse%
\end{scheme}
\end{entry}


\begin{entry}{%
\proto{list}{ \var{obj} \dotsfoo}{procedure}}

Returns a newly allocated list of its arguments.

\begin{scheme}
(list 'a (+ 3 4) 'c)            \ev  (a 7 c)
(list)                          \ev  ()%
\end{scheme}
\end{entry}


\begin{entry}{%
\proto{length}{ list}{procedure}}

\nodomain{\var{List} must be a list.}
Returns the length of \var{list}.

\begin{scheme}
(length '(a b c))               \ev  3
(length '(a (b) (c d e)))       \ev  3
(length '())                    \ev  0%
\end{scheme}
\end{entry}


\begin{entry}{%
\proto{append}{ list \dotsfoo{} obj}{procedure}}

\nodomain{All \var{list}s must be lists.}
Returns a possibly improper list consisting of the elements of the first \var{list}
followed by the elements of the other \var{list}s, with \var{obj} as
the cdr of the final pair.
An improper list results if \var{obj} is not a
proper list.

\begin{scheme}
(append '(x) '(y))              \ev  (x y)
(append '(a) '(b c d))          \ev  (a b c d)
(append '(a (b)) '((c)))        \ev  (a (b) (c))
(append '(a b) '(c . d))        \ev  (a b c . d)
(append '() 'a)                 \ev  a%
\end{scheme}

The resulting chain of pairs is always newly allocated, except that it shares
structure with the \var{obj} argument.
\end{entry}


\begin{entry}{%
\proto{reverse}{ list}{procedure}}

\nodomain{\var{List} must be a list.}
Returns a newly allocated list consisting of the elements of \var{list}
in reverse order.

\begin{scheme}
(reverse '(a b c))              \ev  (c b a)
(reverse '(a (b c) d (e (f))))  \lev  ((e (f)) d (b c) a)%
\end{scheme}
\end{entry}


\begin{entry}{%
\proto{list-tail}{ list k}{procedure}}

\domain{\var{List} must be a list of size at least \var{k}.}

The {\cf list-tail} procedure returns the subchain of pairs of \var{list}
obtained by omitting the first \var{k} elements.

\begin{scheme}
(list-tail '(a b c d) 2)                 \ev  (c d)%
\end{scheme}

\implresp The implementation must check that \var{list} is a chain of
pairs of size at least \var{k}.  It should not check that it is a chain
of pairs beyond this size.

{\cf List-tail} could be defined by

\begin{scheme}
(define (list-tail l k)
  (if (and (not (null? l))
           (not (pair? l)))
      (assertion-violation
       'list-tail
       "not a list"
       l))
  (if (or (not (exact? k))
          (not (integer? k))
          (negative? k))
      (assertion-violation
       'list-tail
       "not an exact non-negative integer"
       l))
  
  (let loop ((l l) (k k))
    (if (zero? k)
        l
        (loop (cdr l) (- k 1)))))%
\end{scheme} 

\end{entry}


\begin{entry}{%
\proto{list-ref}{ list k}{procedure}}

\domain{\var{List} must be a list of size at least $\var{k}+1$.}

Returns the \var{k}th element of \var{list}.

\begin{scheme}
(list-ref '(a b c d) 2)                 \ev c%
\end{scheme}

\implresp The implementation must check that \var{list} is a chain of
pairs of size at least $\var{k}+1$.  It should not check that it is a list
of pairs beyond this size.
\end{entry}


\begin{entry}{%
\proto{map}{ proc \vari{list} \varii{list} \dotsfoo}{procedure}}

\domain{The \var{list}s must all have the same length.  \var{Proc}
  must be a procedure that takes as many arguments as there are
  \var{list}s and returns a single value.  \var{Proc} must not mutate
  any of the \var{list}s.}

The {\cf map} procedure applies \var{proc} element-wise to the elements of the
\var{list}s and returns a list of the results, in order.
\var{Proc} is always called in same dynamic environment 
as {\cf map} itself.
The dynamic order in which \var{proc} is applied to the elements of the
\var{list}s is unspecified.

\begin{scheme}
(map cadr '((a b) (d e) (g h)))   \lev  (b e h)

(map (lambda (n) (expt n n))
     '(1 2 3 4 5))                \lev  (1 4 27 256 3125)

(map + '(1 2 3) '(4 5 6))         \ev  (5 7 9)

(let ((count 0))
  (map (lambda (ignored)
         (set! count (+ count 1))
         count)
       '(a b)))                 \ev  (1 2) \var{or} (2 1)%
\end{scheme}

\implresp The implementation should check that the \var{list}s all
have the same length.  The implementation must check the restrictions
on \var{proc} to the extent performed by applying it as described.
\end{entry}


\begin{entry}{%
\proto{for-each}{ proc \vari{list} \varii{list} \dotsfoo}{procedure}}

\domain{The \var{list}s must all have the same length.  \var{Proc}
  must be a procedure that takes as many arguments as there are
  \var{list}s.  \var{Proc} must not mutate
  any of the \var{list}s.}

The {\cf for-each} procedure applies \var{proc}
element-wise to the elements of the
\var{list}s for its side effects,  in order from the first element(s) to the
last.
\var{Proc} is always called in same dynamic environment 
as {\cf for-each} itself.
The return values of {\cf for-each} are 
unspecified.

\begin{scheme}
(let ((v (make-vector 5)))
  (for-each (lambda (i)
              (vector-set! v i (* i i)))
            '(0 1 2 3 4))
  v)                                \ev  \#(0 1 4 9 16)

(for-each (lambda (x) x) '(1 2 3 4)) \lev 4

(for-each even? '()) \ev \theunspecified%
\end{scheme}

\implresp The implementation should check that the \var{list}s all
have the same length.  The implementation must check the restrictions
on \var{proc} to the extent performed by applying it as described.

\begin{rationale}
  The return values are unspecified to allow implementations of {\cf
    for-each} to tail-call \var{proc} on the last element(s).
\end{rationale}

\end{entry}


\section{Symbols}
\label{symbolsection}

Symbols are objects whose usefulness rests on the fact that two
symbols are identical (in the sense of {\cf eq?}, {\cf eqv?} and {\cf equal?}) if and only if their
names are spelled the same way.  This is exactly the property needed to
represent identifiers\index{identifier} in programs, and so most
implementations of Scheme use them internally for that purpose.  Symbols
are useful for many other applications; for instance, they may be used
the way enumerated values are used in C and Pascal.

A symbol literal is formed using {\cf quote}.

\begin{scheme}
Hello \ev Hello
'H\backwhack{}x65;llo \ev Hello
'$\lambda$ \ev $\lambda$
'\backwhack{}x3BB; \ev $\lambda$
(string->symbol "a b") \ev a\backwhack{}x20;b
(string->symbol "a\backwhack{}\backwhack{}b") \ev a\backwhack{}x5C;b
'a\backwhack{}x20;b \ev a\backwhack{}x20;b
'|a b| \>; \emph{syntax violation}
\>; \textrm{(illegal character}
\>; \textrm{vertical bar)}
'a\backwhack{}nb  \>; \emph{syntax violation}
\>; \textrm{(illegal use of backslash)}
'a\backwhack{}x20 \>; \emph{syntax violation}
\>; \textrm{(missing semi-colon to}
\>; \textrm{terminate \backwhack{}x escape)}%
\end{scheme}

\begin{entry}{%
\proto{symbol?}{ obj}{procedure}}

Returns \schtrue{} if \var{obj} is a symbol, otherwise returns \schfalse.

\begin{scheme}
(symbol? 'foo)          \ev  \schtrue
(symbol? (car '(a b)))  \ev  \schtrue
(symbol? "bar")         \ev  \schfalse
(symbol? 'nil)          \ev  \schtrue
(symbol? '())           \ev  \schfalse
(symbol? \schfalse)     \ev  \schfalse%
\end{scheme}
\end{entry}


\begin{entry}{%
\proto{symbol->string}{ symbol}{procedure}}

Returns the name of \var{symbol} as a string.  
The returned string may be immutable.

\begin{scheme}
(symbol->string 'flying-fish)     
                                  \ev  "flying-fish"
(symbol->string 'Martin)          \ev  "Martin"
(symbol->string
   (string->symbol "Malvina"))     
                                  \ev  "Malvina"%
\end{scheme}
\end{entry}


\begin{entry}{%
\proto{string->symbol}{ string}{procedure}}

Returns the symbol whose name is \var{string}. 

\begin{scheme}
(eq? 'mISSISSIppi 'mississippi)  \lev  \schfalse
(string->symbol "mISSISSIppi")  \lev%
  {\rm{}the symbol with name} "mISSISSIppi"
(eq? 'bitBlt (string->symbol "bitBlt"))     \lev  \schtrue
(eq? 'JollyWog
     (string->symbol
       (symbol->string 'JollyWog)))  \lev  \schtrue
(string=? "K. Harper, M.D."
          (symbol->string
            (string->symbol "K. Harper, M.D.")))  \lev  \schtrue%
\end{scheme}

\end{entry}


\section{Characters}
\label{charactersection}

\mainindex{Unicode}
\mainindex{scalar value}

\defining{Characters} are objects that represent Unicode scalar
values~\cite{Unicode}.

\begin{note}
  Unicode defines a standard mapping between sequences of {\em code
  points}\mainindex{code point} (integers in the range 0 to \#x10FFFF
  in the latest version of the standard) and human-readable
  ``characters''. More precisely, Unicode distinguishes between
  glyphs, which are printed for humans to read, and characters, which
  are abstract entities that map to glyphs (sometimes in a way that's
  sensitive to surrounding characters).  Furthermore, different
  sequences of code points sometimes correspond to the same character.
  The relationships among code points, characters, and glyphs are
  subtle and complex.

  Despite this complexity, most things that a literate human would
  call a ``character'' can be represented by a single code point in
  Unicode (though there may exist code-point sequences that represent
  that same character). For example, Roman letters, Cyrillic letters,
  Hebrew consonants, and most Chinese characters fall into this
  category. Thus, the ``code point'' approximation of ``character''
  works well for many purposes. More specifically, Scheme characters
  correspond to Unicode {\em scalar values}\mainindex{scalar
    value}, which includes all code points except those designated as
  surrogates. A \defining{surrogate} is a code point in the range
  \#xD800 to \#xDFFF that is used in pairs in the UTF-16 encoding to
  encode a supplementary character (whose code is in the range
  \#x10000 to \#x10FFFF).
\end{note}

\begin{entry}{%
\proto{char?}{ obj}{procedure}}

Returns \schtrue{} if \var{obj} is a character, otherwise returns \schfalse.

\end{entry}

\begin{entry}{%
\proto{char->integer}{ char}{procedure}
\proto{integer->char}{ \vr{sv}}{procedure}}

\domain{\var{Sv} must be a Unicode scalar value, i.e.\ a non-negative exact
  integer in $\left[0, \#x\textrm{D7FF}\right] \cup
  \left[\#x\textrm{E000}, \#x\textrm{10FFFF}\right]$.}

Given a character, {\cf char\coerce{}integer} returns its Unicode scalar value
as an exact integer.  
For a Unicode scalar value \var{sv}, {\cf integer\coerce{}char}
returns its associated character.

\begin{scheme}
(integer->char 32) \ev \sharpsign\backwhack{}space
(char->integer (integer->char 5000))
\ev 5000
(integer->char \sharpsign{}xD800) \ev \exception{\&assertion}%
\end{scheme}
\end{entry}


\begin{entry}{%
\proto{char=?}{ \vari{char} \varii{char} \variii{char} \dotsfoo}{procedure}
\proto{char<?}{ \vari{char} \varii{char} \variii{char} \dotsfoo}{procedure}
\proto{char>?}{ \vari{char} \varii{char} \variii{char} \dotsfoo}{procedure}
\proto{char<=?}{ \vari{char} \varii{char} \variii{char} \dotsfoo}{procedure}
\proto{char>=?}{ \vari{char} \varii{char} \variii{char} \dotsfoo}{procedure}}

\label{characterequality}
These procedures impose a total ordering on the set of characters
according to their Unicode scalar values.

\begin{scheme}
(char<? \sharpsign\backwhack{}z \sharpsign\backwhack{}\ss) \ev \schtrue
(char<? \sharpsign\backwhack{}z \sharpsign\backwhack{}Z) \ev \schfalse%
\end{scheme}

\end{entry}

\section{Strings}
\label{stringsection}

Strings are sequences of characters.  

\vest The {\em length} of a string is the number of characters that it
contains.  This number is an exact, non-negative integer that is fixed when the
string is created.  The \defining{valid indexes} of a string are the
exact non-negative integers less than the length of the string.  The first
character of a string has index 0, the second has index 1, and so on.

\vest In phrases such as ``the characters of \var{string} beginning with
index \var{start} and ending with index \var{end}'', it is understood
that the index \var{start} is inclusive and the index \var{end} is
exclusive.  Thus if \var{start} and \var{end} are the same index, a null
substring is referred to, and if \var{start} is zero and \var{end} is
the length of \var{string}, then the entire string is referred to.

\begin{entry}{%
\proto{string?}{ obj}{procedure}}

Returns \schtrue{} if \var{obj} is a string, otherwise returns \schfalse.
\end{entry}


\begin{entry}{%
\proto{make-string}{ k}{procedure}
\rproto{make-string}{ k char}{procedure}}

Returns a newly allocated string of
length \var{k}.  If \var{char} is given, then all elements of the string
are initialized to \var{char}, otherwise the contents of the
\var{string} are unspecified.

\end{entry}

\begin{entry}{%
\proto{string}{ char \dotsfoo}{procedure}}

Returns a newly allocated string composed of the arguments.

\end{entry}

\begin{entry}{%
\proto{string-length}{ string}{procedure}}

Returns the number of characters in the given \var{string}.
\end{entry}


\begin{entry}{%
\proto{string-ref}{ string k}{procedure}}

\domain{\var{K} must be a valid index of \var{string}.}
The {\cf string-ref} procedure returns character \vr{k} of \var{string} using zero-origin indexing.
\end{entry}


\begin{entry}{%
\proto{string-set!}{ string k char}{procedure}}

\domain{%\var{string} must be a string, 
\vr{k} must be a valid index of \var{string}%, and \var{char} must be a character
.}
The {\cf string-set!} procedure stores \var{char} in element \vr{k} of \var{string}
and returns the unspecified value.  % <!>

Passing an immutable string to {\cf string-set!} should cause an exception
with condition type {\cf\&assertion} to be raised.
\begin{scheme}
(define (f) (make-string 3 \sharpsign\backwhack{}*))
(define (g) "***")
(string-set! (f) 0 \sharpsign\backwhack{}?)  \ev  \theunspecified
(string-set! (g) 0 \sharpsign\backwhack{}?)  \ev  \unspecified
             ; \textrm{should raise \exception{\&assertion}}
(string-set! (symbol->string 'immutable)
             0
             \sharpsign\backwhack{}?)  \ev  \unspecified
             ; \textrm{should raise \exception{\&assertion}}%
\end{scheme}

\end{entry}


\begin{entry}{%
\proto{string=?}{ \vari{string} \varii{string} \variii{string} \dotsfoo}{procedure}}

Returns \schtrue{} if the strings are the same length and contain the same
characters in the same positions.  Otherwise, returns \schfalse.

\begin{scheme}
(string=? "Stra\ss{}e" "Strasse") \ev \schfalse%
\end{scheme}
\end{entry}


\begin{entry}{%
\proto{string<?}{ \vari{string} \varii{string} \variii{string} \dotsfoo}{procedure}
\proto{string>?}{ \vari{string} \varii{string} \variii{string} \dotsfoo}{procedure}
\proto{string<=?}{ \vari{string} \varii{string} \variii{string} \dotsfoo}{procedure}
\proto{string>=?}{ \vari{string} \varii{string} \variii{string} \dotsfoo}{procedure}}

These procedures are the lexicographic extensions to strings of the
corresponding orderings on characters.  For example, {\cf string<?}\ is
the lexicographic ordering on strings induced by the ordering
{\cf char<?}\ on characters.  If two strings differ in length but
are the same up to the length of the shorter string, the shorter string
is considered to be lexicographically less than the longer string.

\begin{scheme}
(string<? "z" "\ss") \ev \schtrue
(string<? "z" "zz") \ev \schtrue
(string<? "z" "Z") \ev \schfalse%
\end{scheme}
\end{entry}


\begin{entry}{%
\proto{substring}{ string start end}{procedure}}

\domain{\var{String} must be a string, and \var{start} and \var{end}
must be exact integers satisfying
$$0 \leq \var{start} \leq \var{end} \leq \hbox{\tt(string-length \var{string})\rm.}$$}
The {\cf substring} procedure returns a newly allocated string formed from the characters of
\var{string} beginning with index \var{start} (inclusive) and ending with index
\var{end} (exclusive).
\end{entry}


\begin{entry}{%
\proto{string-append}{ \var{string} \dotsfoo}{procedure}}

Returns a newly allocated string whose characters form the concatenation of the
given strings.

\end{entry}


\begin{entry}{%
\proto{string->list}{ string}{procedure}
\proto{list->string}{ list}{procedure}}

\domain{\var{List} must be a list of characters.}
The {\cf string\coerce{}list} procedure returns a newly allocated list of the
characters that make up the given string.  The {\cf
  list\coerce{}string} procedure
returns a newly allocated string formed from the characters in 
\var{list}. The {\cf string\coerce{}list}
and {\cf list\coerce{}string} procedures are
inverses so far as {\cf equal?}\ is concerned.  
\end{entry}


\begin{entry}{%
\proto{string-copy}{ string}{procedure}}

Returns a newly allocated copy of the given \var{string}.

\end{entry}


\begin{entry}{%
\proto{string-fill!}{ string char}{procedure}}

Stores \var{char} in every element of the given \var{string} and returns the
unspecified value.  % <!>

\end{entry}

\section{Vectors}
\label{vectorsection}

Vectors are heterogeneous structures whose elements are indexed
by integers.  A vector typically occupies less space than a list
of the same length, and the average time required to access a randomly
chosen element is typically less for the vector than for the list.

\vest The {\em length} of a vector is the number of elements that it
contains.  This number is a non-negative integer that is fixed when the
vector is created.  The {\em valid indexes}\index{valid indexes} of a
vector are the exact non-negative integers less than the length of the
vector.  The first element in a vector is indexed by zero, and the last
element is indexed by one less than the length of the vector.

Like list constants, vector constants must be quoted:

\begin{scheme}
'\#(0 (2 2 2 2) "Anna")  \lev  \#(0 (2 2 2 2) "Anna")%
\end{scheme}

\begin{entry}{%
\proto{vector?}{ obj}{procedure}}
 
Returns \schtrue{} if \var{obj} is a vector.  Otherwise, returns \schfalse.
\end{entry}


\begin{entry}{%
\proto{make-vector}{ k}{procedure}
\rproto{make-vector}{ k fill}{procedure}}

Returns a newly allocated vector of \var{k} elements.  If a second
argument is given, then each element is initialized to \var{fill}.
Otherwise the initial contents of each element is unspecified.

\end{entry}


\begin{entry}{%
\proto{vector}{ obj \dotsfoo}{procedure}}

Returns a newly allocated vector whose elements contain the given
arguments.  Analogous to {\cf list}.

\begin{scheme}
(vector 'a 'b 'c)               \ev  \#(a b c)%
\end{scheme}
\end{entry}


\begin{entry}{%
\proto{vector-length}{ vector}{procedure}}

Returns the number of elements in \var{vector} as an exact integer.
\end{entry}


\begin{entry}{%
\proto{vector-ref}{ vector k}{procedure}}

\domain{\var{K} must be a valid index of \var{vector}.}
The {\cf vector-ref} procedure returns the contents of element \vr{k} of
\var{vector}.

\begin{scheme}
(vector-ref '\#(1 1 2 3 5 8 13 21)
            5)  \lev  8
(vector-ref '\#(1 1 2 3 5 8 13 21)
            (->exact (round (* 2 (acos -1))))) \lev 13%
\end{scheme}
\end{entry}


\begin{entry}{%
\proto{vector-set!}{ vector k obj}{procedure}}

\domain{\var{K} must be a valid index of \var{vector}.}
The {\cf vector-set!} procedure stores \var{obj} in element \vr{k} of \var{vector}.
The value returned by {\cf vector-set!}\ is the unspecified value.  % <!>

Passing an immutable vector to {\cf vector-set!} should cause an exception
with condition type {\cf\&assertion} to be raised.

\begin{scheme}
(let ((vec (vector 0 '(2 2 2 2) "Anna")))
  (vector-set! vec 1 '("Sue" "Sue"))
  vec)      \lev  \#(0 ("Sue" "Sue") "Anna")

(vector-set! '\#(0 1 2) 1 "doe")  \lev  \unspecified
             ; constant vector
             ; may raise \exception{\&assertion}%
\end{scheme}

\end{entry}


\begin{entry}{%
\proto{vector->list}{ vector}{procedure}
\proto{list->vector}{ list}{procedure}}

The {\cf vector->list} procedure returns a newly allocated list of the objects contained
in the elements of \var{vector}.  The {\cf list->vector} procedure returns a newly
created vector initialized to the elements of the list \var{list}.

\begin{scheme}
(vector->list '\#(dah dah didah))  \lev  (dah dah didah)
(list->vector '(dididit dah))   \lev  \#(dididit dah)%
\end{scheme}
\end{entry}


\begin{entry}{%
\proto{vector-fill!}{ vector fill}{procedure}}

Stores \var{fill} in every element of \var{vector}
and returns the unspecified value.  % <!>

\end{entry}

\begin{entry}{%
\proto{vector-map}{ proc \vari{vector} \varii{vector} \dotsfoo}{procedure}}

\domain{The \var{vector}s must all have the same length.  \var{Proc}
  must be a procedure.  If the \var{vector}s are non-empty,
  \var{proc} must
  take as many arguments as there are {\it vector}s and must return a
  single value.}

The {\cf vector-map} procedure applies \var{proc} element-wise to the elements of the
\var{vector}s and returns a vector of the results, in order.
\var{Proc} is always called in same dynamic environment 
as {\cf vector-map} itself.
The dynamic order in which \var{proc} is applied to the elements of the
\var{vector}s is unspecified.

Analogous to {\cf map}.
\end{entry}


\begin{entry}{%
\proto{vector-for-each}{ proc \vari{vector} \varii{vector} \dotsfoo}{procedure}}

\domain{The \var{vector}s must all have the same length.  \var{Proc}
  must be a procedure.  If the \var{vector}s are non-empty,
  \var{proc} must
  take as many arguments as there are {\it vector}s.}
The {\cf vector-for-each} procedure applies \var{proc}
element-wise to the elements of the
\var{vector}s for its side effects,  in order from the first element(s) to the
last.
\var{Proc} is always called in same dynamic environment 
as {\cf vector-for-each} itself.
The return values of {\cf vector-for-each} are 
unspecified.

Analogous to {\cf for-each}.
\end{entry}

\section{Errors and violations}
\label{errorviolation}

\begin{entry}{%
\proto{error}{ who message \vari{irritant} \dotsfoo}{procedure}
\proto{assertion-violation}{ who message \vari{irritant} \dotsfoo}{procedure}}

\domain{\var{Who} must be a string or a symbol or \schfalse{}.
  \var{message} must be a string.
  The \var{irritant}s are arbitrary objects.}

These procedures raise an exception.  Calling the {\cf error}
procedure means that an error has occurred, typically caused by
something that has gone wrong in the interaction of the program with the
external world or the user.  Calling the {\cf assertion-violation} procedure
means that an invalid call to a procedure was made, either passing an
invalid number of arguments, or passing an argument that it is not
specified to handle.

The \var{who} argument should describe the procedure or operation that
detected the exception.  The \var{message} argument should describe
the exceptional situation.  The \var{irritant}s should be the arguments
to the operation that detected the operation.

The condition object provided with the exception (see
library chapter~\extref{lib:exceptionsconditionschapter}{Exceptions
  and conditions}) has the following condition types:
%
\begin{itemize}
\item If \var{who} is not \schfalse, the condition has condition type
  {\cf \&who}, with \var{who} as the value of the {\cf who} field.  In
  that case, \var{who} should identify the procedure or entity that
  detected the exception.  If it is \schfalse, the condition does not
  have condition type {\cf \&who}.
\item The condition has condition type {\cf \&message}, with
  \var{message} as the value of the {\cf message} field.
\item The condition has condition type {\cf \&irritants}, and the {\cf
    irritants} field has as its value a list of the \var{irritant}s.
\end{itemize}
%
Moreover, the condition created by {\cf error} has condition type 
{\cf \&error}, and the condition created by {\cf assertion-violation} has
condition type {\cf \&assertion}.

\begin{scheme}
(define (fac n)
  (if (not (integer-valued? n))
      (assertion-violation
       'fac "non-integral argument" n))
  (if (negative? n)
      (assertion-violation
       'fac "negative argument" n))
  (letrec
    ((loop (lambda (n r)
             (if (zero? n)
                 r
                 (loop (- n 1) (* r n))))))
      (loop n 1)))

(fac 5) \ev 120
(fac 4.5) \ev \exception{\&assertion}
(fac -3) \ev \exception{\&assertion}%
\end{scheme}

\begin{rationale}
  The procedures encode a common pattern of raising exceptions.
\end{rationale}
\end{entry}

\section{Control features}
\label{controlsection}
\label{valuessection}
 
This chapter describes various primitive procedures which control the
flow of program execution in special ways.

\begin{entry}{%
\proto{apply}{ proc \vari{arg} $\ldots$ args}{procedure}}

\domain{\var{Proc} must be a procedure and \var{args} must be a
  list.}
Calls \var{proc} with the elements of the list
{\cf(append (list \vari{arg} \dotsfoo) \var{args})} as the actual
arguments.

\begin{scheme}
(apply + (list 3 4))              \ev  7

(define compose
  (lambda (f g)
    (lambda args
      (f (apply g args)))))

((compose sqrt *) 12 75)              \ev  30%
\end{scheme}
\end{entry}


\begin{entry}{%
\proto{call-with-current-continuation}{ proc}{procedure}
\proto{call/cc}{ proc}{procedure}}

\label{continuations} \domain{\var{Proc} must be a procedure of one
argument.} The procedure {\cf call-with-current-continuation} 
(which is the same as the procedure {\cf call/cc}) packages
the current continuation (see the rationale below) as an ``escape
procedure''\mainindex{escape procedure} and passes it as an argument to
\var{proc}.  The escape procedure is a Scheme procedure that, if it is
later called, will abandon whatever continuation is in effect at that later
time and will instead use the continuation that was in effect
when the escape procedure was created.  Calling the escape procedure
may cause the invocation of \var{before} and \var{after} thunks installed using
\ide{dynamic-wind}.

The escape procedure accepts the same number of arguments as the
continuation of the original call to \callcc.

\vest The escape procedure that is passed to \var{proc} has
unlimited extent just like any other procedure in Scheme.  It may be stored
in variables or data structures and may be called as many times as desired.

\vest The following examples show only the most common ways in which
{\cf call-with-current-continuation} is used.  If all real uses were as
simple as these examples, there would be no need for a procedure with
the power of {\cf call-with-current-continuation}.

\begin{scheme}
(call-with-current-continuation
  (lambda (exit)
    (for-each (lambda (x)
                (if (negative? x)
                    (exit x)))
              '(54 0 37 -3 245 19))
    \schtrue))                        \ev  -3

(define list-length
  (lambda (obj)
    (call-with-current-continuation
      (lambda (return)
        (letrec ((r
                  (lambda (obj)
                    (cond ((null? obj) 0)
                          ((pair? obj)
                           (+ (r (cdr obj)) 1))
                          (else (return \schfalse))))))
          (r obj))))))

(list-length '(1 2 3 4))            \ev  4

(list-length '(a b . c))            \ev  \schfalse%

(call-with-current-continuation procedure?)
                            \ev  \schtrue%
\end{scheme}

\begin{rationale}

\vest A common use of {\cf call-with-current-continuation} is for
structured, non-local exits from loops or procedure bodies, but in fact
{\cf call-with-current-continuation} is useful for implementing a
wide variety of advanced control structures.

\vest Whenever a Scheme expression is evaluated there is a
\defining{continuation} wanting the result of the expression.  The continuation
represents an entire (default) future for the computation.
Most of the time the continuation includes actions
specified by user code, as in a continuation that will take the result,
multiply it by the value stored in a local variable, add seven, and store
the result in some other variable.  Normally these
ubiquitous continuations are hidden behind the scenes and programmers do not
think much about them.  On rare occasions, however, a programmer may
need to deal with continuations explicitly.
The {\cf call-with-current-continuation} procedure allows Scheme programmers to do
that by creating a procedure that acts just like the current
continuation.

\vest Most programming languages incorporate one or more special-purpose
escape constructs with names like {\tt exit}, \hbox{{\cf return}}, or
even {\tt goto}.  In 1965, however, Peter Landin~\cite{Landin65}
invented a general purpose escape operator called the J-operator.  John
Reynolds~\cite{Reynolds72} described a simpler but equally powerful
construct in 1972.  The {\cf catch} special form described by Sussman
and Steele in the 1975 report on Scheme is exactly the same as
Reynolds's construct, though its name came from a less general construct
in MacLisp.  Several Scheme implementors noticed that the full power of the
\ide{catch} construct could be provided by a procedure instead of by a
special syntactic construct, and the name
{\cf call-with-current-continuation} was coined in 1982.  This name is
descriptive, but opinions differ on the merits of such a long name, and
some people use the name \ide{call/cc} instead.
\end{rationale}

\end{entry}

\begin{entry}{%
\proto{values}{ obj $\ldots$}{procedure}}

Delivers all of its arguments to its continuation.
The {\cf values} procedure might be defined as follows:
\begin{scheme}
(define (values . things)
  (call-with-current-continuation 
    (lambda (cont) (apply cont things))))%
\end{scheme}

The continuations of all non-final expressions within a sequence of
expressions in {\cf lambda}, {\cf begin}, {\cf let}, {\cf let*}, {\cf
  letrec}, {\cf letrec*}, {\cf let-values}, {\cf let*-values}, {\cf
  case}, {\cf cond}, and {\cf do} forms as well as the continuations
of the \var{before} and \var{after} arguments to {\cf dynamic-wind}
take an arbitrary number of values.

Except for these and the continuations created by {\cf
  call-with-values}, {\cf let-values}, and {\cf let*-values}, all
other continuations take exactly one value.  The effect of passing an
inappropriate number of values to a continuation not created by {\cf
  call-with-values}, {\cf let-values}, or {\cf let*-values} is
undefined.
\end{entry}

\begin{entry}{%
\proto{call-with-values}{ producer consumer}{procedure}}

Calls \var{producer} with no arguments and
a continuation that, when passed some values, calls the
\var{consumer} procedure with those values as arguments.
The continuation for the call to \var{consumer} is the
continuation of the call to {\tt call-with-values}.

\begin{scheme}
(call-with-values (lambda () (values 4 5))
                  (lambda (a b) b))
                                                   \ev  5

(call-with-values * -)                             \ev  -1%
\end{scheme}

If an inappropriate number of values is passed to a continuation
created by {\cf call-with-values}, an exception with condition type
{\cf\&assertion} is raised.
\end{entry}

\begin{entry}{%
\proto{dynamic-wind}{ before thunk after}{procedure}}

\domain{\var{Before}, \var{thunk}, and \var{after} must be procedures
accepting zero arguments and returning any number of values.}

In the absence of any calls to escape procedures
(see \ide{call-with-current-continuation}),
{\cf dynamic-wind} behaves as if defined as follows.

\begin{scheme}
(define dynamic-wind
  (lambda (before thunk after)
    (before)
    (call-with-values
      (lambda () (thunk))
      (lambda vals
        (after)
        (apply values vals)))))%
\end{scheme}

That is, \var{before} is called without arguments.
If \var{before} returns, \var{thunk} is called without arguments.
If \var{thunk} returns, \var{after} is called without arguments.
Finally, if \var{after} returns, the values resulting from the
call to \var{thunk} are returned.

Invoking an escape procedure to transfer control into or out of the
dynamic extent of the call to \var{thunk} can cause additional calls to
\var{before} and \var{after}.
When an escape procedure created outside the dynamic extent of the call to
\var{thunk} is invoked from within the dynamic extent, \var{after} is
called just after control leaves the dynamic extent.
Similarly, when an escape procedure created within the dynamic extent of
the call to \var{thunk} is invoked from outside the dynamic extent,
\var{before} is called just before control reenters the dynamic extent.
In the latter case, if \var{thunk} returns, \var{after} is called even
if \var{thunk} has returned previously.
While the calls to \var{before} and \var{after} are not considered to be
within the dynamic extent of the call to \var{thunk}, calls to the before
and after thunks of any other calls to {\cf dynamic-wind} that occur
within the dynamic extent of the call to \var{thunk} are considered to be
within the dynamic extent of the call to \var{thunk}.

More precisely, an escape procedure used to transfer control out of the
dynamic extent of a set of zero or more active {\cf dynamic-wind}
\var{thunk} calls $x\ \dots$ and transfer control into the dynamic extent
of a set of zero or more active {\cf dynamic-wind} \var{thunk} calls
$y\ \dots$ proceeds as follows.
It leaves the dynamic extent of the most recent $x$ and calls without
arguments the corresponding \var{after} thunk.
If the \var{after} thunk returns, the escape procedure proceeds to
the next most recent $x$, and so on.
Once each $x$ has been handled in this manner,
the escape procedure calls without arguments the \var{before} thunk
corresponding to the least recent $y$.
If the \var{before} thunk returns, the escape procedure reenters the
dynamic extent of the least recent $y$ and proceeds with the next least
recent $y$, and so on.
Once each $y$ has been handled in this manner, control is transfered to
the continuation packaged in the escape procedure.

\begin{scheme}
(let ((path '())
      (c \#f))
  (let ((add (lambda (s)
               (set! path (cons s path)))))
    (dynamic-wind
      (lambda () (add 'connect))
      (lambda ()
        (add (call-with-current-continuation
               (lambda (c0)
                 (set! c c0)
                 'talk1))))
      (lambda () (add 'disconnect)))
    (if (< (length path) 4)
        (c 'talk2)
        (reverse path))))
    \lev (connect talk1 disconnect
               connect talk2 disconnect)

(let ((n 0))
  (call-with-current-continuation
    (lambda (k)
      (dynamic-wind
        (lambda ()
          (set! n (+ n 1))
          (k))
        (lambda ()
          (set! n (+ n 2)))
        (lambda ()
          (set! n (+ n 4))))))
  n) \ev 1

(let ((n 0))
  (call-with-current-continuation
    (lambda (k)
      (dynamic-wind
        values
        (lambda ()
          (dynamic-wind
            values
            (lambda ()
              (set! n (+ n 1))
              (k))
            (lambda ()
              (set! n (+ n 2))
              (k))))
        (lambda ()
          (set! n (+ n 4))))))
  n) \ev 7%
\end{scheme}
\end{entry}

\section{Iteration}%\unsection

\begin{entry}{%
\rproto{let}{ \hyper{variable} \hyper{bindings} \hyper{body}}{\exprtype}}

\label{namedlet}
``Named {\cf let}'' is a variant on the syntax of \ide{let} which provides
a more general looping construct than {\cf do} and may also be used to express
recursions.
It has the same syntax and semantics as ordinary {\cf let}
except that \hyper{variable} is bound within \hyper{body} to a procedure
whose formal arguments are the bound variables and whose body is
\hyper{body}.  Thus the execution of \hyper{body} may be repeated by
invoking the procedure named by \hyper{variable}.

%                                              |  <-- right margin
\begin{scheme}
(let loop ((numbers '(3 -2 1 6 -5))
           (nonneg '())
           (neg '()))
  (cond ((null? numbers) (list nonneg neg))
        ((>= (car numbers) 0)
         (loop (cdr numbers)
               (cons (car numbers) nonneg)
               neg))
        ((< (car numbers) 0)
         (loop (cdr numbers)
               nonneg
               (cons (car numbers) neg))))) %
  \lev  ((6 1 3) (-5 -2))%
\end{scheme}

The {\cf let} keyword could be defined in terms of {\cf lambda} and {\cf letrec}
using {\cf syntax-rules} (see section~\ref{syntaxrulessection}) as
follows:

\begin{scheme}
(define-syntax \ide{let}
  (syntax-rules ()
    ((let ((name val) ...) body1 body2 ...)
     ((lambda (name ...) body1 body2 ...)
      val ...))
    ((let tag ((name val) ...) body1 body2 ...)
     ((letrec ((tag (lambda (name ...)
                      body1 body2 ...)))
        tag)
      val ...))))%
\end{scheme}

\end{entry}

\noindent%
\pproto{(do ((\hyperi{variable} \hyperi{init} \hyperi{step})}{\exprtype}
\mainschindex{do}{\tt\obeyspaces%
     \dotsfoo)\\
    (\hyper{test} \hyper{expression} \dotsfoo)\\
  \hyper{expression} \dotsfoo)}

\syntax
The \hyper{init}s, \hyper{step}s, and \hyper{test}s must be
expressions.  The \hyper{variable}s must be pairwise distinct variables.

\semantics
The {\cf do} expression is an iteration construct.  It specifies a set of variables to
be bound, how they are to be initialized at the start, and how they are
to be updated on each iteration.  When a termination condition is met,
the loop exits after evaluating the \hyper{expression}s.

A {\cf do} expression is evaluated as follows:
The \hyper{init} expressions are evaluated (in some unspecified order),
the \hyper{variable}s are bound to fresh locations, the results of the
\hyper{init} expressions are stored in the bindings of the
\hyper{variable}s, and then the iteration phase begins.

\vest Each iteration begins by evaluating \hyper{test}; if the result is
false (see section~\ref{booleanvaluessection}), then the \hyper{expression}s
are evaluated in order for effect, the \hyper{step}
expressions are evaluated in some unspecified order, the
\hyper{variable}s are bound to fresh locations, the results of the
\hyper{step}s are stored in the bindings of the
\hyper{variable}s, and the next iteration begins.

\vest If \hyper{test} evaluates to a true value, then the
\hyper{expression}s are evaluated from left to right and the value(s) of
the last \hyper{expression} is(are) returned.  If no \hyper{expression}s
are present, then the value of the {\cf do} expression is the
unspecified value.

\vest The region\index{region} of the binding of a \hyper{variable}
consists of the entire {\cf do} expression except for the \hyper{init}s.
It is a syntax violation for a \hyper{variable} to appear more than once in the
list of {\cf do} variables.

\vest A \hyper{step} may be omitted, in which case the effect is the
same as if {\cf(\hyper{variable} \hyper{init} \hyper{variable})} had
been written instead of {\cf(\hyper{variable} \hyper{init})}.

\begin{scheme}
(do ((vec (make-vector 5))
     (i 0 (+ i 1)))
    ((= i 5) vec)
  (vector-set! vec i i))          \ev  \#(0 1 2 3 4)

(let ((x '(1 3 5 7 9)))
  (do ((x x (cdr x))
       (sum 0 (+ sum (car x))))
      ((null? x) sum)))             \ev  25%
\end{scheme}

The following definition
of {\cf do} uses a trick to expand the variable clauses.

\begin{scheme}
(define-syntax \ide{do}
  (syntax-rules ()
    ((do ((var init step ...) ...)
         (test expr ...)
         command ...)
     (letrec
       ((loop
         (lambda (var ...)
           (if test
               (begin
                 (unspecified)
                 expr ...)
               (begin
                 command
                 ...
                 (loop (do "step" var step ...)
                       ...))))))
       (loop init ...)))
    ((do "step" x)
     x)
    ((do "step" x y)
     y)))%
\end{scheme}

%\end{entry}


\section{Quasiquotation}\unsection
\label{quasiquotesection}

\begin{entry}{%
\proto{quasiquote}{ \hyper{qq template}}{\exprtype}}

``Backquote'' or ``quasiquote''\index{backquote} expressions are useful
for constructing a list or vector structure when some but not all of the
desired structure is known in advance.  If no
{\cf unquote} or {\cf unquote-splicing} forms
appear within the \hyper{qq template}, the result of
evaluating
{\cf (quasiquote \hyper{qq template})} is equivalent to the result of evaluating
{\cf (quote \hyper{qq template})}.

If an {\cf (unquote \hyper{expression} \dotsfoo)} form appears inside a
\hyper{qq template}, however, the \hyper{expression}s are evaluated
(``unquoted'') and their results are inserted into the structure instead
of the {\cf unquote} form.

If an {\cf (unquote-splicing \hyper{expression} \dotsfoo)} form
appears inside a \hyper{qq template}, then the \hyper{expression}s must
evaluate to lists; the opening and closing parentheses of the lists are
then ``stripped away'' and the elements of the lists are inserted in
place of the {\cf unquote-splicing} form.

{\cf unquote-splicing} and multi-operand {\cf unquote} forms must
appear only within a list or vector \hyper{qq template}.

As noted in section~\ref{quotesection},
{\cf (quasiquote \hyper{qq template})} may be abbreviated
\backquote\hyper{qq template},
{\cf (unquote \hyper{expression})} may be abbreviated
{\cf,}\hyper{expression}, and
{\cf (unquote-splicing \hyper{expression})} may be abbreviated
{\cf,}\atsign\hyper{expression}.

\begin{scheme}
`(list ,(+ 1 2) 4)  \ev  (list 3 4)
(let ((name 'a)) `(list ,name ',name)) %
          \lev  (list a (quote a))
`(a ,(+ 1 2) ,@(map abs '(4 -5 6)) b) %
          \lev  (a 3 4 5 6 b)
`(({\cf foo} ,(- 10 3)) ,@(cdr '(c)) . ,(car '(cons))) %
          \lev  ((foo 7) . cons)
`\#(10 5 ,(sqrt 4) ,@(map sqrt '(16 9)) 8) %
          \lev  \#(10 5 2 4 3 8)
(let ((name 'foo))
  `((unquote name name name)))%
          \lev (foo foo foo)
(let ((name '(foo)))
  `((unquote-splicing name name name)))%
          \lev (foo foo foo)
(let ((q '((append x y) (sqrt 9))))
  ``(foo ,,@q)) \lev `(foo (unquote (append x y) (sqrt 9)))
(let ((x '(2 3))
      (y '(4 5)))
  `(foo (unquote (append x y) (sqrt 9)))) \lev (foo (2 3 4 5) 3)%
\end{scheme}

Quasiquote forms may be nested.  Substitutions are made only for
unquoted components appearing at the same nesting level
as the outermost {\cf quasiquote}.  The nesting level increases by one inside
each successive quasiquotation, and decreases by one inside each
unquotation.

\begin{scheme}
`(a `(b ,(+ 1 2) ,(foo ,(+ 1 3) d) e) f) %
          \lev  (a `(b ,(+ 1 2) ,(foo 4 d) e) f)
(let ((name1 'x)
      (name2 'y))
  `(a `(b ,,name1 ,',name2 d) e)) %
          \lev  (a `(b ,x ,'y d) e)%
\end{scheme}

It is a syntax violation if any of the identifiers
\ide{quasiquote}, \ide{unquote}, or \ide{unquote-splicing} appear in
positions within a \hyper{qq template} otherwise than as described above.

The following grammar for quasiquote expressions is not context-free.
It is presented as a recipe for generating an infinite number of
production rules.  Imagine a copy of the following rules for $D = 1, 2,
3, \ldots$.  $D$ keeps track of the nesting depth.

\begin{grammar}%
\meta{quasiquotation} \: \meta{quasiquotation 1}
\meta{qq template 0} \: \meta{expression}
\meta{quasiquotation $D$} \: (quasiquote \meta{qq template $D$})
\meta{qq template $D$} \: \meta{simple datum}
\>    \| \meta{list qq template $D$}
\>    \| \meta{vector qq template $D$}
\>    \| \meta{unquotation $D$}
\meta{list qq template $D$} \: (\arbno{\meta{qq template or splice $D$}})
\>    \| (\atleastone{\meta{qq template or splice $D$}} .\ \meta{qq template $D$})
\>    \| \meta{quasiquotation $D+1$}
\meta{vector qq template $D$} \: \#(\arbno{\meta{qq template or splice $D$}})
\meta{unquotation $D$} \: (unquote \meta{qq template $D-1$})
\meta{qq template or splice $D$} \: \meta{qq template $D$}
\>    \| \meta{splicing unquotation $D$}
\meta{splicing unquotation $D$} \:
\>\> (unquote-splicing \arbno{\meta{qq template $D-1$}})
\>    \| (unquote \arbno{\meta{qq template $D-1$}}) %
\end{grammar}

In \meta{quasiquotation}s, a \meta{list qq template $D$} can sometimes
be confused with either an \meta{un\-quota\-tion $D$} or a \meta{splicing
un\-quo\-ta\-tion $D$}.  The interpretation as an
\meta{un\-quo\-ta\-tion} or \meta{splicing
un\-quo\-ta\-tion $D$} takes precedence.

\end{entry}

\section{Binding constructs for syntactic keywords}
\label{bindsyntax}

The {\cf let-syntax} and {\cf letrec-syntax} forms are analogous to {\cf let}
and {\cf letrec} but bind keywords rather than variables.
Like a {\cf begin} form, a {\cf let-syntax} or {\cf letrec-syntax} form
may appear in a definition context, in which case it is treated as a
definition, and the forms in the body of the form must also be
definitions.
A {\cf let-syntax} or {\cf letrec-syntax} form may also appear in an
expression context, in which case the forms within their bodies must be
expressions.

\begin{entry}{%
\proto{let-syntax}{ \hyper{bindings} \hyper{form} \dotsfoo}{\exprtype}}

\syntax
\hyper{Bindings} must have the form
\begin{scheme}
((\hyper{keyword} \hyper{expression}) \dotsfoo)%
\end{scheme}
Each \hyper{keyword} is an identifier,
each \hyper{expression} is 
an expression that evaluates, at macro-expansion
time, to a transformer (see
library chapter~\extref{lib:syntaxcasechapter}{{\cf syntax-case}}).  It is a
syntax violation for \hyper{keyword} to appear more than once in the list of keywords
being bound.
The \hyper{form}s are arbitrary forms.

\semantics
The \hyper{form}s are expanded in the syntactic environment
obtained by extending the syntactic environment of the
{\cf let-syntax} form with macros whose keywords are
the \hyper{keyword}s, bound to the specified transformers.
Each binding of a \hyper{keyword} has the \hyper{form}s as its region.

The \hyper{form}s of a {\cf let-syntax}
form are treated, whether in definition or expression context, as if
wrapped in an implicit {\cf begin}.  See section~\ref{begin}.
Thus, internal definitions in the result of expanding the \hyper{form}s have
the same region as any definition appearing in place of the {\cf
  let-syntax} form would have.

\begin{scheme}
(let-syntax ((when (syntax-rules ()
                     ((when test stmt1 stmt2 ...)
                      (if test
                          (begin stmt1
                                 stmt2 ...))))))
  (let ((if \schtrue))
    (when if (set! if 'now))
    if))                           \ev  now

(let ((x 'outer))
  (let-syntax ((m (syntax-rules () ((m) x))))
    (let ((x 'inner))
      (m))))                       \ev  outer%

(let ()
  (let-syntax
    ((def (syntax-rules ()
            ((def stuff ...) (define stuff ...)))))
    (def foo 42))
  foo) \ev 42

(let ()
  (let-syntax ())
  5) \ev 5%
\end{scheme}

\end{entry}

\begin{entry}{%
\proto{letrec-syntax}{ \hyper{bindings} \hyper{form} \dotsfoo}{\exprtype}}

\syntax
Same as for {\cf let-syntax}.

\semantics
The \hyper{form}s are
expanded in the syntactic environment obtained by
extending the syntactic environment of the {\cf letrec-syntax}
form with macros whose keywords are the
\hyper{keyword}s, bound to the specified transformers.
Each binding of a \hyper{keyword} has the \hyper{bindings}
as well as the \hyper{form}s within its region,
so the transformers can
transcribe forms into uses of the macros
introduced by the {\cf letrec-syntax} form.

The \hyper{form}s of a {\cf letrec-syntax}
form are treated, whether in definition or expression context, as if
wrapped in an implicit {\cf begin}, see section~\ref{begin}.
Thus, internal definitions in the result of expanding the \hyper{form}s have
the same region as any definition appearing in place of the {\cf
  letrec-syntax} form would have.

\begin{scheme}
(letrec-syntax
  ((my-or (syntax-rules ()
            ((my-or) \schfalse)
            ((my-or e) e)
            ((my-or e1 e2 ...)
             (let ((temp e1))
               (if temp
                   temp
                   (my-or e2 ...)))))))
  (let ((x \schfalse)
        (y 7)
        (temp 8)
        (let odd?)
        (if even?))
    (my-or x
           (let temp)
           (if y)
           y)))        \ev  7%
\end{scheme}

The following example highlights how {\cf let-syntax}
and {\cf letrec-syntax} differ.

\begin{scheme}
(let ((f (lambda (x) (+ x 1))))
  (let-syntax ((f (syntax-rules ()
                    ((f x) x)))
               (g (syntax-rules ()
                    ((g x) (f x)))))
    (list (f 1) (g 1)))) \lev (1 2)

(let ((f (lambda (x) (+ x 1))))
  (letrec-syntax ((f (syntax-rules ()
                       ((f x) x)))
                  (g (syntax-rules ()
                       ((g x) (f x)))))
    (list (f 1) (g 1)))) \lev (1 1)%
\end{scheme}

The two expressions are identical except that the {\cf let-syntax} form
in the first expression is a {\cf letrec-syntax} form in the second.
In the first expression, the {\cf f} occurring in {\cf g} refers to
the {\cf let}-bound variable {\cf f}, whereas in the second it refers
to the keyword {\cf f} whose binding is established by the
{\cf letrec-syntax} form.
\end{entry}

\section{Macro transformers}
\label{syntaxrulessection}

\begin{entry}{%
\pproto{(syntax-rules (\hyper{literal} \dots) \hyper{syntax rule} \dots)}{\exprtype}}
\mainschindex{syntax-rules}

\syntax Each \hyper{literal} must be an identifier.
Each \hyper{syntax rule} must have the following form:

\begin{scheme}
(\hyper{srpattern} \hyper{template})
\end{scheme}

An \hyper{srpattern} is a restricted form of \hyper{pattern},
namely, a nonempty \hyper{pattern} in one of four parenthesized forms below
whose first subform is an identifier or an underscore {\cf \_}.
A \hyper{pattern} is an identifier, constant, or one of the following.

\begin{schemenoindent}
(\hyper{pattern} \ldots)
(\hyper{pattern} \hyper{pattern} \ldots . \hyper{pattern})
(\hyper{pattern} \ldots \hyper{pattern} \hyper{ellipsis} \hyper{pattern} \ldots)
(\hyper{pattern} \ldots \hyper{pattern} \hyper{ellipsis} \hyper{pattern} \ldots . \hyper{pattern})
\#(\hyper{pattern} \ldots)
\#(\hyper{pattern} \ldots \hyper{pattern} \hyper{ellipsis} \hyper{pattern} \ldots)%
\end{schemenoindent}

An \hyper{ellipsis} is the identifier ``{\cf ...}'' (three periods).\schindex{...}

A \hyper{template} is a pattern variable, an identifier that
is not a pattern
variable, a pattern datum, or one of the following.

\begin{scheme}
(\hyper{subtemplate} \ldots)
(\hyper{subtemplate} \ldots . \hyper{template})
\#(\hyper{subtemplate} \ldots)%
\end{scheme}

A \hyper{subtemplate} is a \hyper{template} followed by zero or more ellipses.

\semantics An instance of {\cf syntax-rules} evaluates, at
macro-expansion time, to a new macro
transformer by specifying a sequence of hygienic rewrite rules.  A use
of a macro whose keyword is associated with a transformer specified by
{\cf syntax-rules} is matched against the patterns contained in the
\hyper{syntax rule}s, beginning with the leftmost \hyper{syntax rule}.
When a match is found, the macro use is transcribed hygienically
according to the template.  It is a syntax violation when no match is found.

An identifier appearing within a \hyper{pattern} may be an underscore
(~{\cf \_}~), a literal identifier listed in the list of literals
{\cf (\hyper{literal} \dots)}, or an ellipsis (~{\cf ...}~).
All other identifiers appearing within a \hyper{pattern} are
\textit{pattern variables\mainindex{pattern variable}}.
It is a syntax violation if an ellipsis or underscore appears in {\cf (\hyper{literal} \dots)}.

While the first subform of \hyper{srpattern} may be an identifier, the
identifier is not involved in the matching and
is not considered a pattern variable or literal identifier.

\begin{rationale}
The identifier is most often the keyword used to identify the macro.
The scope of the keyword is determined by the binding form or syntax
definition that binds it to the associated macro transformer.
If the keyword were a pattern variable or literal
identifier, then
the template that follows the pattern would be within its scope
regardless of whether the keyword were bound by {\cf let-syntax},
{\cf letrec-syntax}, or {\cf define-syntax}.
\end{rationale}

Pattern variables match arbitrary input subforms and
are used to refer to elements of the input.
It is a syntax violation if the same pattern variable appears more than once in a
\hyper{pattern}.

Underscores also match arbitrary input subforms but are not pattern variables
and so cannot be used to refer to those elements.
Multiple underscores may appear in a \hyper{pattern}.

A literal identifier matches an input subform if and only if the input
subform is an identifier and either both its occurrence in the input
expression and its occurrence in the list of literals have the same
lexical binding, or the two identifiers have the same name and both have
no lexical binding.

A subpattern followed by an ellipsis can match zero or more elements of
the input.

More formally, an input form $F$ matches a pattern $P$ if and only if
one of the following holds:

\begin{itemize}
\item $P$ is an underscore (~{\cf \_}~).

\item $P$ is a pattern variable.

\item $P$ is a literal identifier
and $F$ is an identifier such that both $P$ and $F$ would refer to the
same binding if both were to appear in the output of the macro outside
of any bindings inserted into the output of the macro.
(If neither of two like-named identifiers refers to any binding, i.e., both
are undefined, they are considered to refer to the same binding.)

\item $P$ is of the form
{\cf ($P_1$ \dots{} $P_n$)}
and $F$ is a list of $n$ elements that match $P_1$ through
$P_n$.

\item $P$ is of the form
{\cf ($P_1$ \dots{} $P_n$ . $P_x$)}
and $F$ is a list or improper list of $n$ or more elements
whose first $n$ elements match $P_1$ through $P_n$
and
whose $n$th cdr matches $P_x$.

\item $P$ is of the form
{\cf ($P_1$ \dots{} $P_k$ $P_e$ \hyper{ellipsis} $P_{m+1}$ \dots{} $P_n$)},
where \hyper{ellipsis} is the identifier {\cf ...}
and $F$ is a proper list of $n$
elements whose first $k$ elements match $P_1$ through $P_k$,
whose next $m-k$ elements each match $P_e$,
and
whose remaining $n-m$ elements match $P_{m+1}$ through $P_n$.

\item $P$ is of the form
{\cf ($P_1$ \dots{} $P_k$ $P_e$ \hyper{ellipsis} $P_{m+1}$ \dots{} $P_n$ . $P_x$)},
where \hyper{ellipsis} is the identifier {\cf ...}
and $F$ is a list or improper list of $n$
elements whose first $k$ elements match $P_1$ through $P_k$,
whose next $m-k$ elements each match $P_e$,
whose next $n-m$ elements match $P_{m+1}$ through $P_n$,
and 
whose $n$th and final cdr matches $P_x$.

\item $P$ is of the form
{\cf \#($P_1$ \dots{} $P_n$)}
and $F$ is a vector of $n$ elements that match $P_1$ through
$P_n$.

\item $P$ is of the form
{\cf \#($P_1$ \dots{} $P_k$ $P_e$ \hyper{ellipsis} $P_{m+1}$ \dots{} $P_n$)},
where \hyper{ellipsis} is the identifier {\cf ...}
and $F$ is a vector of $n$ or more elements
whose first $k$ elements match $P_1$ through $P_k$,
whose next $m-k$ elements each match $P_e$,
and
whose remaining $n-m$ elements match $P_{m+1}$ through $P_n$.

\item $P$ is a pattern datum (any nonlist, nonvector, nonsymbol
datum) and $F$ is equal to $P$ in the sense of the
{\cf equal?} procedure.
\end{itemize}

When a macro use is transcribed according to the template of the
matching \hyper{syntax rule}, pattern variables that occur in the
template are replaced by the subforms they match in the input.

Pattern data and identifiers that are not pattern variables
or ellipses are copied directly into the output.
A subtemplate followed by an ellipsis expands
into zero or more occurrences of the subtemplate.
Pattern variables that occur in subpatterns followed by one or more
ellipses may occur only in subtemplates that are
followed by (at least) as many ellipses.
These pattern variables are replaced in the output by the input
subforms to which they are bound, distributed as specified.
If a pattern variable is followed by more ellipses in the subtemplate
than in the associated subpattern, the input form is replicated as
necessary.
The subtemplate must contain at least one pattern variable from a
subpattern followed by an ellipsis, and for at least one such pattern
variable, the subtemplate must be followed by exactly as many ellipses as
the subpattern in which the pattern variable appears.
(Otherwise, the expander would not be able to determine how many times the
subform should be repeated in the output.)
It is a syntax violation if the constraints of this paragraph are not met.

A template of the form
{\cf (\hyper{ellipsis} \hyper{template})} is identical to \hyper{template}, except that
ellipses within the template have no special meaning.
That is, any ellipses contained within \hyper{template} are
treated as ordinary identifiers.
In particular, the template {\cf (... ...)} produces a single
ellipsis, {\cf ...}.
This allows syntactic abstractions to expand into forms containing
ellipses.

As an example, if \ide{let} and \ide{cond} are defined as in
section~\ref{let} and appendix~\ref{derivedformsappendix} then they
are hygienic (as required) and the following is not an error.

\begin{scheme}
(let ((=> \schfalse))
  (cond (\schtrue => 'ok)))           \ev ok%
\end{scheme}

The macro transformer for {\cf cond} recognizes {\cf =>}
as a local variable, and hence an expression, and not as the
top-level identifier {\cf =>}, which the macro transformer treats
as a syntactic keyword.  Thus the example expands into

\begin{scheme}
(let ((=> \schfalse))
  (if \schtrue (begin => 'ok)))%
\end{scheme}

instead of

\begin{scheme}
(let ((=> \schfalse))
  (let ((temp \schtrue))
    (if temp ('ok temp))))%
\end{scheme}

which would result in an assertion violation.

\end{entry}

\begin{entry}{%
\proto{identifier-syntax}{ \hyper{template}}{\exprtype}
\pproto{(identifier-syntax (\hyperi{id} \hyperi{template})}{\exprtype}}\\
{\tt\obeyspaces
                   ((set! \hyperii{id} \hyper{pattern})
                    \hyperii{template}))}

\syntax The \hyper{id}s must be identifiers.

\semantics
When a keyword is bound to a transformer produced by the first form of
{\cf identifier-syntax}, references to the keyword within the scope
of the binding are replaced by \hyper{template}.

\begin{scheme}
(define p (cons 4 5))
(define-syntax p.car (identifier-syntax (car p)))
p.car \ev 4
(set! p.car 15) \ev \exception{\&syntax}%
\end{scheme}

The second, more general, form of {\cf identifier-syntax} permits
the transformer to determine what happens when {\cf set!} is used.
In this case, uses of the identifier by itself are replaced by
\hyperi{template}, and uses of {\cf set!} with the identifier are
replaced by \hyperii{template}.

\begin{scheme}
(define p (cons 4 5))
(define-syntax p.car
  (identifier-syntax
    (\_ (car p))
    ((set! \_ e) (set-car! p e))))
(set! p.car 15)
p.car           \ev 15
p               \ev (15 5)%
\end{scheme}

\end{entry}

\section{Tail calls and tail contexts}
\label{basetailcontextsection}

A {\em tail call}\mainindex{tail call} is a procedure call that occurs
in a {\em tail context}.  Tail contexts are defined inductively.  Note
that a tail context is always determined with respect to a particular lambda
expression.

\begin{itemize}
\item The last expression within the body of a lambda expression,
  shown as \hyper{tail expression} below, occurs in a tail context.
\begin{grammar}%
(l\=ambda \meta{formals}
  \>\arbno{\hyper{definition}} 
  \>\arbno{\meta{expression}} \meta{tail expression})
\end{grammar}%

\item If one of the following expressions is in a tail context,
then the subexpressions shown as \meta{tail expression} are in a tail context.
These were derived from rules for the syntax of the forms described in
this chapter by replacing some occurrences of \meta{expression}
with \meta{tail expression}.  Only those rules that contain tail contexts
are shown here.

\begin{grammar}%
(if \meta{expression} \meta{tail expression} \meta{tail expression})
(if \meta{expression} \meta{tail expression})

(cond \atleastone{\meta{cond clause}})
(cond \arbno{\meta{cond clause}} (else \meta{tail sequence}))

(c\=ase \meta{expression}
  \>\atleastone{\meta{case clause}})
(c\=ase \meta{expression}
  \>\arbno{\meta{case clause}}
  \>(else \meta{tail sequence}))

(and \arbno{\meta{expression}} \meta{tail expression})
(or \arbno{\meta{expression}} \meta{tail expression})

(let (\arbno{\meta{binding spec}}) \meta{tail body})
(let \meta{variable} (\arbno{\meta{binding spec}}) \meta{tail body})
(let* (\arbno{\meta{binding spec}}) \meta{tail body})
(letrec* (\arbno{\meta{binding spec}}) \meta{tail body})
(letrec (\arbno{\meta{binding spec}}) \meta{tail body})
(let-values (\arbno{\meta{mv binding spec}}) \meta{tail body})
(let*-values (\arbno{\meta{mv binding spec}}) \meta{tail body})

(let-syntax (\arbno{\meta{syntax spec}}) \meta{tail body})
(letrec-syntax (\arbno{\meta{syntax spec}}) \meta{tail body})

(begin \meta{tail sequence})

(d\=o \=(\arbno{\meta{iteration spec}})
  \>  \>(\meta{test} \meta{tail sequence})
  \>\arbno{\meta{expression}})

{\rm where}

\meta{cond clause} \: (\meta{test} \meta{tail sequence})
\meta{case clause} \: ((\arbno{\meta{datum}}) \meta{tail sequence})

\meta{tail body} \: \arbno{\meta{definition}}
\>\> \meta{tail sequence}
\meta{tail sequence} \: \arbno{\meta{expression}} \meta{tail expression}
\end{grammar}%

\item
If a {\cf cond} expression is in a tail context, and has a clause of
the form {\cf (\hyperi{expression} => \hyperii{expression})}
then the (implied) call to
the procedure that results from the evaluation of \hyperii{expression} is in a
tail context.  \hyperii{expression} itself is not in a tail context.

\end{itemize}

Certain built-in procedures are also required to perform tail calls.
The first argument passed to \ide{apply} and to
\ide{call-with-current-continuation}, and the second argument passed to
\ide{call-with-values}, must be called via a tail call.

In the following example the only tail call is the call to {\cf f}.
None of the calls to {\cf g} or {\cf h} are tail calls.  The reference to
{\cf x} is in a tail context, but it is not a call and thus is not a
tail call.
\begin{scheme}%
(lambda ()
  (if (g)
      (let ((x (h)))
        x)
      (and (g) (f))))%
\end{scheme}%

\begin{note}
Implementations are allowed, but not required, to
recognize that some non-tail calls, such as the call to {\cf h}
above, can be evaluated as though they were tail calls.
In the example above, the {\cf let} expression could be compiled
as a tail call to {\cf h}. (The possibility of {\cf h} returning
an unexpected number of values can be ignored, because in that
case the effect of the {\cf let} is explicitly unspecified and
implementation-dependent.)
\end{note}

%%% Local Variables: 
%%% mode: latex
%%% TeX-master: "r6rs"
%%% End: 

    \par
\clearchaptergroupstar{Appendices}
\appendix
\chapter{Formal semantics}
\label{formalsemanticschapter}
%!TEX root = r6rs.tex

This appendix presents a non-normative, formal, operational semantics for Scheme. It does not cover the entire language. The notable missing features are the macro system, I/O, and the numeric tower. The precise list of features included is given in section~\ref{sec:semantics:grammar}.

The core of the specification is a single-step term rewriting relation that indicates how an (abstract) machine behaves. In general, the report is not a complete specification, giving implementations freedom to behave differently, typically to allow optimizations. This underspecification shows up in two ways in the semantics. 

The first is reduction rules that reduce to special ``\textbf{unknown:} \textit{string}'' states (where the string provides a description of the unknown state). The intention is that rules that reduce to such states can be replaced with arbitrary reduction rules. The precise specification of how to replace those rules is given in section~\ref{sec:semantics:underspecification}.

The other is that the single-step relation relates one program to
multiple different programs, each corresponding to a legal transition
that an abstract machine might take. Accordingly we use the transitive
closure of the single step relation $\rightarrow^*$ to define the
semantics, \calS, as a function from programs (\calP)
to sets of observable results (\calR):
\begin{center}
\begin{tabular}{l}
$\calS : \calP \longrightarrow 2^{\calR}$ \\
$\calS(\calP) = \{ \scrO(\calA) \mid \calP \rightarrow^* \calA \}$
\end{tabular}
\end{center}
where the function $\scrO$ turns an answer ($\calA$) from the semantics into an observable result. Roughly, $\scrO$ is the identity function on simple base values, and returns a special tag for more complex values, like procedure and pairs.

So, an implementation conforms to the semantics if, for every program $\calP$, the implementation produces one of the results in $\calS(\calP)$ or, if the implementation loops forever, then there is an infinite reduction sequence starting at $\calP$, assuming that the reduction relation $\rightarrow$ has been adjusted to replace the \textbf{unknown:} states.

The precise definitions of $\calP$, $\calA$, $\calR$, and $\scrO$ are also given in section~\ref{sec:semantics:grammar}.

To help understand the semantics and how it behaves, we have
implemented it in PLT Redex. The implementation is available at the
report's website: \url{http://www.r6rs.org/}. All of the reduction
rules and the metafunctions shown in the figures in this semantics
were generated automatically from the source code.

\section{Background}

We assume the reader has a basic familiarity with context-sensitive
reduction semantics. Readers unfamiliar with this system may wish to
consult Felleisen and Flatt's monograph~\cite{ff:monograph} or Wright
and Felleisen~\cite{wf:type-soundness} for a thorough introduction,
including the relevant technical background, or an introduction to PLT
Redex~\cite{mfff:plt-redex} for a somewhat lighter one.

As a rough guide, we define the operational semantics of a language
via a relation on program terms, where the relation corresponds to a
single step of an abstract machine. The relation is defined using
evaluation contexts, namely terms with a distinguished place in them,
called \emph{holes}\index{hole}, where the next step of evaluation
occurs. We say that a term $e$ decomposes into an evaluation
context $E$ and another term $e'$ if $e$ is the
same as $E$ but with the hole replaced by $e'$. We write
$E[e']$ to indicate the term obtained by replacing the hole in
$E$ with $e'$.

For example, assuming that we have defined a grammar containing
non-terminals for evaluation contexts ($E$), expressions
($e$), variables ($x$), and values ($v$), we
would write:
%
\begin{displaymath}
  \begin{array}{l}
    E_1[\texttt{((}\sy{lambda}~\texttt{(}x_1 \cdots{}\texttt{)}~e_1\texttt{)}~v_1~\cdots\texttt{)}] \rightarrow
    \\
    E_1[\{ x_1 \cdots \mapsto v_1 \cdots \} e_1] ~~~~~ (\#x_1 = \#v_1)
  \end{array}
\end{displaymath}
%
to define the $\beta_v$ rewriting rule (as a part of the $\rightarrow$
single step relation). We use the names of the non-terminals (possibly
with subscripts) in a rewriting rule to restrict the application of
the rule, so it applies only when some term produced by that grammar
appears in the corresponding position in the term. If the same
non-terminal with an identical subscript appears multiple times, the
rule only applies when the corresponding terms are structurally
identical (nonterminals without subscripts are not constrained to
match each other). Thus, the occurrence of $E_1$ on both the
left-hand and right-hand side of the rule above means that the context
of the application expression does not change when using this rule.
The ellipses are a form of Kleene star, meaning that zero or more
occurrences of terms matching the pattern proceeding the ellipsis may
appear in place of the the ellipsis and the pattern preceding it. We
use the notation $\{ x_1 \cdots \mapsto v_1 \cdots \} e_1$ for
capture-avoiding substitution; in this case it means that each
$x_1$ is replaced with the corresponding $v_1$ in
$e_1$. Finally, we write side-conditions in parentheses beside
a rule; the side-condition in the above rule indicates that the number
of $x_1$s must be the same as the number of $v_1$s.
Sometimes we use equality in the side-conditions; when we do it merely
means simple term equality, i.e., the two terms must have the
same syntactic shape.


\addtocounter{figure}{1} % get the figure counter in sync with the section counter
\subfigurestart{}
\beginfig
\input{r6-fig-grammar-parti.tex}
\caption{Grammar for programs and observables}\label{fig:grammar}
\endfig

Making the evaluation context $E$ explicit in the rule allows
us to define relations that manipulate their context. As a simple
example, we can add another rule that signals an error when a
procedure is applied to the wrong number of arguments by discarding
the evaluation context on the right-hand side of a rule:
%
\begin{displaymath}
  \begin{array}{l}
    E[\texttt{((}\sy{lambda}~\texttt{(}x_1 \cdots\texttt{)}~e\texttt{)}~v_1~\cdots\texttt{)}] \rightarrow
    \\
    \textrm{\textbf{error:} wrong argument count} ~~~~~ (\#x_1 \neq \#v_1)
  \end{array}
\end{displaymath}
%
Later we take advantage of the explicit evaluation context in more
sophisticated ways.



\section{Grammar}\label{sec:semantics:grammar}

\beginfig
\subfigureadjust{}
\input{r6-fig-grammar-partii.tex}
\caption{Grammar for evaluation contexts}\label{fig:ec-grammar}
\endfig
\subfigurestop{}

Figure~\ref{fig:grammar} shows the grammar for the subset of the
report this semantics models. Non-terminals are written in
\textit{italics} or in a calligraphic font ($\calP$
$\calA$, $\calR$, and $\calRv$) and literals are 
written in a \texttt{monospaced} font.

The $\calP$ non-terminal represents possible program states. The
first alternative is a program with a store and an expression. The second alternative is an error, and the third is
used to indicate a place where the model does not completely specify
the behavior of the primitives it models (see section~\ref{sec:semantics:underspecification} for details of those situations). 
The $\calA$ non-terminal
represents a final result of a program. It is just like $\calP$
except that expression has been reduced to some sequence of values.

The $\calR$ and $\calRv$ non-terminals specify the observable results of a program. Each $\calR$ is either a sequence of values that correspond to the values produced by the program that terminates normally, or a tag indicating an uncaught exception was raised, or \sy{unknown} if the program encounters a situation the semantics does not cover. The $\calRv$ non-terminal specifies what the observable results are for a particular value: the unspecified value, a pair, the empty list, a symbol, a self-quoting value (true, false, and numbers), a condition, or a procedure.

The \nt{sf} non-terminal generates individual elements of the
store. The store holds all of the mutable state of a program. It is
explained in more detail along with the rules that manipulate it.

Expressions ($\mathit{es}$) include quoted data, \sy{begin} expressions, \sy{begin0} expressions%
\footnote{ \sy{begin0} is not part of the standard, but we include it
  to make the rules for \va{dynamic-wind} and \va{letrec} easier to read. Although
  we model it directly, it can be defined in terms of other forms we
  model here that do come from the standard:
\begin{displaymath}
  \begin{array}{rcl}\tt
    \texttt{(}\sy{begin0}~e_1~e_2~\cdots\texttt{)} &=&
    \begin{array}{l}
      \texttt{(}\va{call\mbox{-}with\mbox{-}values}\\
      ~\texttt{(}\sy{lambda}~\texttt{()}~e_1\texttt{)}\\
      ~\texttt{(}\sy{lambda}~x\\
      ~~~e_2~\cdots\\
      ~~~\texttt{(}\va{apply}~\va{values}~x\texttt{)))}
    \end{array}
  \end{array}
\end{displaymath}
}, application expressions, \sy{if} expressions, \sy{set!}
expressions, variables, non-procedure values (\nt{nonproc}), primitive
procedures (\nt{pproc}), lambda expressions, \sy{letrec} and \sy{letrec*} expressions. 

The last few expression forms are only generated for intermediate states (\sy{dw} for \sy{dynamic-wind}, \sy{throw} for continuations, \sy{unspecified} for the result of the assignment operators, \sy{handlers} for exception handlers, and \sy{l!} and \sy{reinit} for \sy{letrec}), and should not appear in an initial program. Their use is described in the relevant sections of this appendix.

The \nt{f} describes the arguments for \sy{lambda} expressions. (The \sy{dot} is used instead of a period for procedures that accept an arbitrary number of arguments, in order to avoid meta-circular confusion in our PLT Redex model.) 

The \nt{s} non-terminal covers all s-expressions, which can be either non-empty sequences (\nt{seq}), the empty sequence, self-quoting values (\nt{sqv}), or symbols. Non-empty sequences are either just a sequence of s-expressions, or they are terminated with a dot followed by either a symbol or a self-quoting value. Finally the self-quoting values are numbers and the booleans \semtrue{} and \semfalse{}.

The \nt{p} non-terminal represents programs that have no quoted
data. Most of the reduction rules rewrite \nt{p} to \nt{p},
rather than $\calP$ to $\calP$, since quoted data is first
rewritten into calls to the list construction functions before
ordinary evaluation proceeds. In parallel to \nt{es}, \nt{e} represents
expressions that have no quoted expressions.

\beginfig
\begin{center}
\input{r6-fig-Quote.tex}

\input{r6-fig-QtocQtoic.tex}
\end{center}
\caption{Quote}\label{fig:quote}
\endfig

The values ($v$) are divided into four categories:
%
\begin{itemize}
\item Non-procedures (\nt{nonproc}) include pair pointers
  (\va{pp}), \va{null}, symbols, self-quoting values
  (\nt{sqv}), and conditions. Conditions represent
  the report's condition values, but here just contain a message and
  are otherwise inert.
\item User procedures (\texttt{(}\sy{lambda} \nt{f} \nt{e} \nt{e} $\cdots$\texttt{)}) include multi-arity lambda expressions and lambda expressions with dotted argument lists,
\item Primitive procedures (\nt{pproc}) include

\begin{itemize}
\item
 arithmetic procedures
  (\nt{aproc}): \va{+}, \va{-}, \va{/}, and \va{*}, 
\item 
  procedures of one
  argument (\nt{proc1}): \va{null?}, \va{pair?}, \va{car}, \va{cdr},
  \va{call/cc}, \va{procedure?}, \va{condition?}, \va{unspecified?}, \va{raise}, and \va{raise-continuable}, 
  \item
  procedures of
  two arguments (\nt{proc2}): \va{cons}, \va{set-car!}, \va{set-cdr!}, \va{eqv?},
  and \va{call-with-values}, 
  \item as well as \va{list}, \va{dynamic-wind},
  \va{apply}, \va{values}, and \va{with-exception-handler}.
\end{itemize}
\item Finally, continuations are represented as \sy{throw} expressions
  whose body consists of the context where the continuation was
  grabbed.
\end{itemize}
%
The next three set of non-terminals in figure~\ref{fig:grammar} represent pairs (\nt{pp}), which are divided into immutable pairs (\nt{ip}) and mutable pairs (\nt{mp}). The final set of non-terminals in figure~\ref{fig:grammar}, \nt{sym},
\nt{x}, and $n$ represent symbols, variables, and
numbers respectively. The non-terminals \nt{ip}, \nt{mp}, and \nt{sym} are all assumed to all be disjoint. Additionally, the variables $x$ are assumed not to include any keywords or primitive operations, so any program variables whose names coincide with them must be renamed before the semantics can give the meaning of that program.

\beginfig
\begin{center}
\input{r6-fig-Multiple--values--and--call-with-values.tex}
\end{center}
\caption{Multiple values and call-with-values}\label{fig:Multiple--values--and--call-with-values}
\endfig

The set of non-terminals for evaluation contexts is shown in
figure~\ref{fig:ec-grammar}. The \nt{P} non-terminal controls where
evaluation happens in a program that does not contain any quoted data.
The $E$ and $F$ evaluation contexts are for expressions.  They are factored in
that manner so that the \nt{PG}, \nt{G}, and \nt{H} evaluation contexts can
re-use \nt{F} and have fine-grained control over the context to support
exceptions and \va{dynamic-wind}. The starred and circled variants,
\Estar{}, \Eo{}, \Fstar{}, and \Fo{} dictate where a single value is
promoted to multiple values and where multiple values are demoted to a
single value. The \nt{U} context is used to manage the report's underspecification of the results of \sy{set!}, \va{set-car!}, and \va{set-cdr!} (see section~\ref{sec:semantics:underspecification} for details). Finally, the \nt{S} context is where quoted expressions can be simplified. The precise use of the evaluation contexts is explained along with the relevant rules.

To convert the answers ($\calA$)  of the semantics into observable results, we uses these two functions:
\input{r6-fig-observable}
\input{r6-fig-observable-value}
They eliminate the store, and replace complex values with simple tags that indicate only the kind of value that was produced or, if no values were produced, indicates that either an uncaught exception was raised, or that the program reached a state that is not specified by the semantics.

\section{Quote}\label{sec:semantics:quote}

The first reduction rules that apply to any program is the 
rules in figure~\ref{fig:quote} that eliminate quoted expressions. 
The first two rules erase the quote for quoted expressions that do not introduce any cons pairs.
The last two rules lift quoted s-expressions to the top of the expression so they are evaluated first, and turn the s-expressions into calls to either \va{cons} or \va{consi}, via the metafunctions $\mathscr{Q}_i$ and $\mathscr{Q}_m$.

Note that the left-hand side of the \rulename{6qcons} and \rulename{6qconsi} rules are identical, meaning that if one rule applies to a term, so does the other rule. 
Accordingly, a quoted expression may be lifted out into a sequence of \va{cons} expressions, which create mutable pairs, or into a sequence of \va{consi} expressions, which create immutable pairs (see section~\ref{sec:semantics:lists} for the rules on how that happens).

These rules apply before any other because of the contexts in which they, and all of the other rules, apply. In particular, these rule applies in the
\nt{S} context. Figure~\ref{fig:ec-grammar} shows that the
\nt{S} context allows this reduction to apply in
any subexpression of an \nt{e}, as long as all of the
subexpressions to the left have no quoted expressions in them,
although expressions to the right may have quoted expressions.
Accordingly, this rule applies once for each quoted expression in the
program, moving out to the beginning of the program.
The rest of the rules apply in contexts that do not contain any quoted
expressions, ensuring that these rules convert all quoted data
into lists before those rules apply.

Although the identifier \nt{qp} does not have a subscript, the semantics of PLT Redex's ``fresh'' declaration takes special care to ensures that the \nt{qp} on the right-hand side of the rule is indeed the same as the one in the side-condition.

\beginfig
\begin{center}
\input{r6-fig-Exceptions}
\end{center}
\caption{Exceptions}\label{fig:Exceptions}
\endfig

\section{Multiple values}

The basic strategy for multiple values is to add a rule that demotes
$(\va{values}~v)$ to $v$ and another rule that promotes
$v$ to $(\va{values}~v)$. If we allowed these rules to apply
in an arbitrary evaluation context, however, we would get infinite
reduction sequences of endless alternation between promotion and
demotion. So, the semantics allows demotion only in a context
expecting a single value and allows promotion only in a context
expecting multiple values. We obtain this behavior with a small
extension to the Felleisen-Hieb framework (also present in the
operational model for R$^5$RS~\cite{mf:op-r5rs}).
We extend the notation so that
holes have names (written with a subscript), and the context-matching
syntax may also demand a hole of a particular name (also written with
a subscript, for instance $E[e]_{\star}$).  The extension
allows us to give different names to the holes in which multiple
values are expected and those in which single values are expected, and
structure the grammar of contexts accordingly.

To exploit this extension, we use three kinds of holes in the
evaluation context grammar in figure~\ref{fig:ec-grammar}. The
ordinary hole \hole{} appears where the usual kinds of
evaluation can occur. The hole \holes{} appears in contexts that
allow multiple values and the hole \holeone{} appears in
contexts that expect a single value. Accordingly, the rules
\rulename{6promote} only applies in \holes{} contexts, and the
rule \rulename{6demote} only applies in \holeone{} contexts.

To see how the evaluation contexts are organized to ensure that
promotion and demotion occur in the right places, consider the \nt{F},
\Fstar{} and \Fo{} evaluation contexts. The \Fstar{} and \Fo{}
evaluation contexts are just the same as \nt{F}, except that they allow
promotion to multiple values and demotion to a single value,
respectively. So, the \nt{F} evaluation context, rather than being
defined in terms of itself, exploits \Fstar{} and \Fo{} to dictate
where promotion and demotion can occur. For example, \nt{F} can be
$\texttt{(}\sy{if}~\Fo{}~e~e\texttt{)}$ meaning that demotion from
$\texttt{(}\va{values}~v\texttt{)}$ to
$v$ can occur in the first argument to an \sy{if} expression.
Similarly, $F$ can be $\texttt{(}\sy{begin}~\Fstar{}~e~e~\cdots\texttt{)}$ meaning that
$v$ can be promoted to $\texttt{(}\va{values}~v\texttt{)}$ in the first argument of a \sy{begin}.

In general, the promotion and demotion rules simplify the definitions
of the other rules. For instance, the rule for \sy{if} does not
need to consider multiple values in its first subexpression.
Similarly, the rule for \sy{begin} does not need to consider the
case of a single value as its first subexpression.

\beginfig
\begin{center}
\input{r6-fig-Arithmetic.tex}
\input{r6-fig-Basic--syntactic--forms.tex}
\end{center}
\caption{Arithmetic and basic forms}\label{fig:Arithmetic}
\endfig

The other two rules in
figure~\ref{fig:Multiple--values--and--call-with-values} handle
\va{call-with-values}. The evaluation contexts for
\va{call-with-values} (in the $F$ non-terminal) allow
evaluation in the body of a procedure that has been passed as the first
argument to \va{call-with-values}, as long as the second argument
has been reduced to a value. Once evaluation inside that procedure
completes, it will produce multiple values (since it is an \Fstar{}
position), and the entire \va{call-with-values} expression reduces
to an application of its second argument to those values, via the rule
\rulename{6cwvd}. Finally, in the
case that the first argument to \va{call-with-values} is a value,
but is not of the form $\texttt{(}\sy{lambda}~\texttt{()}~e\texttt{)}$, the rule
\rulename{6cwvw} wraps it in a thunk to trigger evaluation.

\beginfig
\begin{center}
\input{r6-fig-Cons.tex}
\end{center}
\caption{Lists}\label{fig:Cons}
\endfig

\section{Exceptions}

The workhorses for the exception system are $$\texttt{(}\sy{handlers}~\nt{proc}~\cdots{}~\nt{e}\texttt{)}$$ expressions and the \nt{G} and \nt{PG} evaluation contexts (shown in figure~\ref{fig:ec-grammar}). 
The \sy{handlers} expression records the
active exception handlers (\nt{proc} $\cdots$) in some expression (\nt{e}). The
intention is that only the nearest enclosing \sy{handlers} expression
is relevant to raised exceptions, and the $G$ and \nt{PG} evaluation
contexts help achieve that goal. They are just like their counterparts
\nt{E} and \nt{P}, except that \sy{handlers} expressions cannot occur on the
path to the hole, and the exception system rules take advantage of
that context to find the closest enclosing handler.

To see how the contexts work together with \sy{handler}
expressions, consider the left-hand side of the \rulename{6xunee}
rule in figure~\ref{fig:Exceptions}.
It matches expressions that have a call to \va{raise} or
\va{raise-continuable} (the non-terminal \nt{raise*} matches
both exception-raising procedures) in a \nt{PG}
evaluation context. Since the \nt{PG} context does not contain any
\sy{handlers} expressions, this exception cannot be caught, so
this expression reduces to a final state indicating the uncaught
exception. The rule \rulename{6xuneh} also signals an uncaught
exception, but it covers the case where a \sy{handlers} expression
has exhausted all of the handlers available to it. The rule applies to
expressions that have a \sy{handlers} expression (with no
exception handlers) in an arbitrary evaluation context where a call to
one of the exception-raising functions is nested in the
\sy{handlers} expression. The use of the \nt{G} evaluation
context ensures that there are no other \sy{handler} expressions
between this one and the raise.

The next two rules cover call to the procedure \va{with-exception-handler}.
The \rulename{6xwh1} rule applies when there are no \sy{handler}
expressions. It constructs a new one and applies $\nt{v}_2$ as a
thunk in the \sy{handler} body. If there already is a handler
expression, the \rulename{6xwhn} applies. It collects the current
handlers and adds the new one into a new \sy{handlers} expression
and, as with the previous rule, invokes the second argument to
\va{with-exception-handlers}.

The next two rules cover exceptions that are raised in the context of
a \sy{handlers} expression. If a continuable exception is raised,
\rulename{6xrc} applies. It takes the most recently installed
handler from the nearest enclosing \sy{handlers} expression and
applies it to the argument to \va{raise-continuable}, but in a
context where the exception handlers do not include that latest
handler. The \rulename{6xr} rule behaves similarly, except it
raises a new exception if the handler returns. The new exception is
created with the \sy{condition} special form.

\beginfig
\begin{center}
\input{r6-fig-Eqv.tex}
\end{center}
\caption{Eqv}\label{fig:Eqv}
\endfig

The \sy{make-cond} special form is a stand-in for the report's
conditions. It does not evaluate its argument (note its absence from
the $E$ grammar in figure~\ref{fig:ec-grammar}). That argument
is just a literal string describing the context in which the exception
was raised. The only operation on conditions is \va{condition?},
whose semantics are given by the two rules \rulename{6ct} and
\rulename{6cf}.

Finally, the rule \rulename{6xdone} drops a \sy{handlers} expression
when its body is fully evaluated, and the rule \rulename{6weherr}
raises an exception when \va{with-exception-handler} is supplied with
incorrect arguments.

\section{Arithmetic and basic forms}

This model does not include the report's arithmetic, but does include
an idealized form in order to make experimentation with other features
and writing test suites for the model simpler.
Figure~\ref{fig:Arithmetic} shows the reduction rules for the
primitive procedures that implement addition, subtraction,
multiplication, and division. They defer to their mathematical
analogues. In addition, when the subtraction or divison operator are
applied to no arguments, or when division receives a zero as a
divisor, or when any of the arithmetic operations receive a
non-number, an exception is raised.

The bottom half of figure~\ref{fig:Arithmetic} shows the rules for
\sy{if}, \sy{begin}, and \sy{begin0}. The relevant
evaluation contexts are given by the $F$ non-terminal.

The evaluation contexts for \sy{if} only allow evaluation in its
first argument. Once that is a value, the rules for \sy{if} reduce
an \sy{if} expression to its second argument if the test is not
\semfalse{}, and to its third subexpression if it is.

The \sy{begin} evaluation contexts allow evaluation in the first
subexpression of a begin, but only if there are two or more
subexpressions. In that case, once the first expression has been fully
simplified, the reduction rules drop its value. If there is only a
single subexpression, the \sy{begin} itself is dropped.

\subfigurestart{}
\beginfig
\begin{center}
\input{r6-fig-Procedure--application.tex}
\end{center}
\caption{Procedures \& application}\label{fig:Procedure--application}
\endfig

Like the \sy{begin} evaluation contexts, the \sy{begin0}
evaluation contexts allow evaluation of the first argument of a
\sy{begin0} expression when there are two or more subexpressions.
The \sy{begin0} evaluation contexts also allow evaluation in the
second argument of a \sy{begin0} expression, as long as the first
argument has been fully simplified. The \rulename{6begin0n} rule for
\sy{begin0} then drops a fully simplified second argument.
Eventually, there is only a single expression in the \sy{begin0},
at which point the \rulename{begin01} rule fires, and removes the
\sy{begin0} expression.

\section{Lists}\label{sec:semantics:lists}

The rules in figure~\ref{fig:Cons} handle lists. The first two rules handle \va{list} by reducing it to a succession of calls to \va{cons}, followed by \va{null}.

The next two rules, \rulename{6cons} and \rulename{6consi}, allocate new \va{cons} cells.
They both move $\texttt{(}\va{cons}~v_1~v_2\texttt{)}$ into the store, bound to a fresh
pair pointer (see also section~\ref{sec:semantics:quote} for a description of ``fresh''). 
The \rulename{6cons} uses a \nt{mp} variable, to indicate the pair is mutable, and the \rulename{6consi} uses a \nt{ip} variable to indicate the pair is immutable.

The rules \rulename{6car} and \rulename{6cdr} extract the components of a pair from the store when presented with a pair pointer (the \nt{pp} can be either \nt{mp} or \nt{ip}, as shown in figure~\ref{fig:grammar}).

The rules \rulename{6setcar} and \rulename{6setcdr} handle assignment of mutable pairs. 
They replace the contents of the appropriate location in the store with the new value, and reduce to \va{unspecified}. See section~\ref{sec:semantics:underspecification} for an explanation of how \va{unspecified} reduces.

\beginfig
\subfigureadjust{}
\begin{center}
\input{r6-fig-Var-set!d_.tex}
\end{center}
\caption{Variable-assignment relation}\label{fig:varsetd}
\endfig

The next four rules handle the \va{null?} predicate and the \va{pair?} predicate, and the final four rules raise exceptions when \va{car}, \va{cdr}, \va{set-car!} or \va{set-cdr!} receive non pairs.

\section{Eqv}

The rules for \va{eqv?} are shown in figure~\ref{fig:Eqv}. The first two rules cover most of the behavior of \va{eqv?}. 
The first says that when the two arguments to \va{eqv?} are syntactically identical, then \va{eqv?} produces \semtrue{} and the second says that when the arguments are not syntactically identical, then \va{eqv?} produces \semfalse{}. 
The structure of \nt{v} has been carefully designed so that simple term equality corresponds closely to \va{eqv?}'s behavior. 
For example, pairs are represented as pointers into the store and \va{eqv?} only compares those pointers.

The side-conditions on those first two rules ensure that they do not apply when simple term equality does not match the behavior of \va{eqv?}. There are two situations where it does not match: comparing two conditions and comparing two procedures. For the first, the report does not specify \va{eqv?}'s behavior, except to say that it must return a boolean, so the remaining two rules (\rulename{6eqct}, and \rulename{6eqcf}) allow such comparisons to return \semtrue{} or \semfalse{}. Comparing two procedures is covered in section~\ref{sec:semantics:underspecification}. 

\section{Procedures and application}

In evaluating a procedure call, the report leaves
unspecified the order in which arguments are evaluated. So, our reduction system allows multiple, different reductions to occur, one for each possible order of evaluation.

To capture unspecified evaluation order but allow only evaluation that
is consistent with some sequential ordering of the evaluation of an
application's subexpressions, we use non-deterministic choice to first pick
a subexpression to reduce only when we have not already committed to
reducing some other subexpression. To achieve that effect, we limit
the evaluation of application expressions to only those that have a
single expression that isn't fully reduced, as shown in the
non-terminal $F$, in figure~\ref{fig:ec-grammar}. To evaluate
application expressions that have more than two arguments to evaluate,
the rule \rulename{6mark} picks one of the subexpressions of an
application that is not fully simplified and lifts it out in its own
application, allowing it to be evaluated. Once one of the lifted
expressions is evaluated, the \rulename{6appN} substitutes its value
back into the original application.

The \rulename{6appN} rule also handles other applications whose
arguments are finished by substituting the first actual parameter for
the first formal parameter in the expression. Its side-condition uses
the relation in figure~\ref{fig:varsetd} to ensure that there are no
\sy{set!} expressions with the parameter $x_1$ as a target.
If there is such an assignment, the \rulename{6appN!} rule applies (see also section~\ref{sec:semantics:quote} for a description of ``fresh'').
Instead of directly substituting the actual parameter for the formal
parameter, it creates a new location in the store, initially bound the
actual parameter, and substitutes a variable standing for that
location in place of the formal parameter. The store, then, handles
any eventual assignment to the parameter. Once all of the parameters
have been substituted away, the rule \rulename{6app0} applies and
evaluation of the body of the procedure begins.

At first glance, the rule \rulename{6appN} appears superfluous, since it seems like the rules could just reduce first by \rulename{6appN!} and then look up the variable when it is evaluated. 
There are two reasons why we keep the \rulename{6appN}, however. 
The first is purely conventional: reducing applications via substitution is taught to us at an early age and is commonly used in rewriting systems in the literature.
The second reason is more technical. In particular, there is a subtle interaction with the \rulename{6mark} rule. 
Consider the right-hand side of the \rulename{6mark} and imagine that $\nt{e}_i$ has beem reduced to a value. At this point, we'd like to take that value and replace it back into the original application. Unfortunately, the \rulename{6appN!} does not do that. 
Instead, it will lift the value into the store and replace put a variable reference into the application, leading to another use of \rulename{6mark}, and another use of \rulename{6appN!}, which continues forever.

The rule \rulename{6$\mu$app} handles a well-formed application of a function with a dotted argument lists. 
It such an application into an application of an
ordinary procedure by constructing a list of the extra arguments. Similarly, the rule \rulename{6$\mu$app1} handles an application of a procedure that has a single variable as its parameter list.

The rule \rulename{6var} handles variable lookup in the store and \rulename{6set} handles variable assignment.

The next two rules \rulename{6proct} and \rulename{6procf} handle applications of \va{procedure?}, and the remaining rules cover applications of non-procedures and arity errors.

\beginfig
\subfigureadjust{}
\begin{center}
\input{r6-fig-Apply.tex}
\input{r6-fig-circular_.tex}
\end{center}
\caption{Apply}\label{fig:Apply}
\endfig
\subfigurestop{}

The rules in figure~\ref{fig:Apply} cover 
cover \va{apply}. 
The first rule, \rulename{6applyf}, covers the case where the last argument to
\va{apply} is the empty list, and simply reduces by erasing the
empty list and the \va{apply}. The second rule, \rulename{6applyc}
covers a well-formed application of \va{apply} where \va{apply}'s final argument is a pair. It
reduces by extracting the components of the pair from the store and
putting them into the application of \va{apply}. Repeated
application of this rule thus extracts all of the list elements passed
to \va{apply} out of the store. 

The remaining five rules cover the
various errors that can occur when using \va{apply}. The first one covers the case where \va{apply} is supplied with a cyclic list. The next four cover applying a
non-procedure, passing a non-list as the last argument, and supplying
too few arguments to \va{apply}.

\section{Call/cc and dynamic wind}

\beginfig
\begin{center}
\input{r6-fig-Call-cc--and--dynamic-wind.tex} \\
\input{r6-fig-TrimpRepoSt.tex}
\end{center}
\caption{Call/cc and dynamic wind}\label{fig:Call-cc--and--dynamic-wind}
\endfig

The specification of \va{dynamic-wind} uses 
$\texttt{(}\sy{dw}~x~e~e~e\texttt{)}$
expressions to record which dynamic-wind \var{thunk}s are active at
each point in the computation. Its first argument is an identifier
that is globally unique and serves to identify invocations of
\va{dynamic-wind}, in order to avoid exiting and re-entering the
same dynamic context during a continuation switch. The second, third,
and fourth arguments are calls to some \var{before}, \var{thunk}, and
\var{after} procedures from a call to \va{dynamic-wind}. Evaluation only
occurs in the middle expression; the \sy{dw} expression only
serves to record which \var{before} and \var{after} procedures need to be run during a
continuation switch. Accordingly, the reduction rule for an
application of \va{dynamic-wind} reduces to a call to the
\var{before} procedure, a \sy{dw} expression and a call to the
\var{after} procedure, as
shown in rule \rulename{6wind} in
figure~\ref{fig:Call-cc--and--dynamic-wind}. The next two rules cover
abuses of the \va{dynamic-wind} procedure: calling it with
non-procedures, and calling it with the wrong number of arguments. The
\rulename{6dwdone} rule erases a \sy{dw} expression when its second
argument has finished evaluating.

The next two rules cover \va{call/cc}. The rule
\rulename{6call/cc} creates a new continuation. It takes the context
of the \va{call/cc} expression and packages it up into a
\sy{throw} expression that represents the continuation. The
\sy{throw} expression uses the fresh variable $x$ to record
where the application of \va{call/cc} occurred in the context for
use in the \rulename{6throw} rule when the continuation is applied.
That rule takes the arguments of the continuation, wraps them with a
call to \va{values}, and puts them back into the place where the
original call to \va{call/cc} occurred, replacing the current
context with the context returned by the $\mathscr{T}$ metafunction.

The $\mathscr{T}$ (for ``trim'') metafunction accepts two $D$ contexts and
builds a context that matches its second argument, the destination
context, except that additional calls to the \var{before} and
\var{after} procedures
from \sy{dw} expressions in the context have been added.

The first clause of the $\mathscr{T}$ metafunction exploits the
$H$ context, a context that contains everything except
\sy{dw} expressions. It ensures that shared parts of the
\va{dynamic-wind} context are ignored, recurring deeper into the
two expression contexts as long as the first \sy{dw} expression in
each have matching identifiers ($x_1$). The final rule is a
catchall; it only applies when all the others fail and thus applies
either when there are no \sy{dw}s in the context, or when the
\sy{dw} expressions do not match. It calls the two other
metafunctions defined in figure~\ref{fig:Call-cc--and--dynamic-wind} and
puts their results together into a \sy{begin} expression.

The $\mathscr{R}$ metafunction extracts all of the \var{before}
procedures from its argument and the $\mathscr{S}$ metafunction extracts all of the \var{after} procedures from its argument. They each construct new contexts and exploit
$H$ to work through their arguments, one \sy{dw} at a time.
In each case, the metafunctions are careful to keep the right
\sy{dw} context around each of the procedures in case a continuation
jump occurs during one of their evaluations. 
Since $\mathscr{R}$,
receives the destination context, it keeps the intermediate
parts of the context in its result.
In contrast
$\mathscr{S}$ discards all of the context except the \sy{dw}s,
since that was the context where the call to the
continuation occured.

\section{Letrec}

\beginfig
\begin{center}
\input{r6-fig-Letrec.tex}
\end{center}
\caption{Letrec and letrec*}
\label{fig:Letrec}
\endfig

Figre~\ref{fig:Letrec} shows the rules that handle \sy{letrec} and \sy{letrec*} and the supplementary expressions that they produce, \sy{l!} and \sy{reinit}. As a first approximation, both \va{letrec} and \va{letrec*} reduce by allocating locations in the store to hold the values of the init expressions, initializing those locations to \sy{bh} (for ``black hole''), evaluating the init expressions, and then using \va{l!} to update the locations in the store with the value of the init expressions. They also use \va{reinit} to detect when an init expression in a letrec is reentered via a continuation.

Before considering how \sy{letrec} and \sy{letrec*} use \sy{l!} and \sy{reinit}, first consider how \sy{l!} and \sy{reinit} behave. The first two rules in figure~\ref{fig:Letrec} cover \sy{l!}. It behaves very much like \sy{set!}, but it initializes both ordinary variables, and variables that are current bound to the black hole (\sy{bh}).

The next two rules cover ordinary \sy{set!} when applied to a variable that is currently bound to a black hole. This situation can arise when the program assigns to a variable before letrec initializes it, eg \verb|(letrec ((x (set! x 5))) x)|. The report specifies that either an implementation should perform the assignment, as reflected in the \rulename{6setdt} rule or it should signal an error, as reflected in the \rulename{6setdte} rule.

The \rulename{6dt} rule covers the case where a variable is referred to before the value of a init expression is filled in, which must always be an error.

A \va{reinit} expression is used to detect a program that captures a continuation in an initialization expression and returns to it, as shown in the three rules \rulename{6init}, \rulename{6reinit}, and \rulename{6reinite}. The \va{reinit} form accepts an identifier that is bound in the store to a boolean as its argument. Those are identifiers are initially \semfalse{}. When \va{reinit} is evaluated, it checks the value of the variable and, if it is still \semfalse{}, it changes it to \semtrue{}. If it is already \semtrue{}, then \va{reinit} either just does nothing, or it raises an exception, in keeping with the two legal behaviors of \va{letrec} and \va{letrec*}. 

The last two rules in figure~\ref{fig:Letrec} put together \sy{l!} and \sy{reinit}. The \rulename{6letrec} rule reduces a \sy{letrec} expression to an application expression, in order to capture the unspecified order of evaluation of the init expressions. Each init expression is wrapped in a \sy{begin0} that records the value of the init and then uses \sy{reinit} to detect continuations that return to the init expression. Once all of the init expressions have been evaluated, the procedure on the right-hand side of the rule is invoked, causing the value of the init expression to be filled in the store, and evaluation continues with the body of the original \sy{letrec} expression.

The \rulename{6letrec*} rule behaves similarly, but uses a \sy{begin} expression rather than an application expression, since its specification mandates that the init expressions are evaluated from left to right. In addition, each init expression is filled into the store as it is evaluated, so that subsequent init expressions can refer to its value.

\section{Underspecification}\label{sec:semantics:underspecification}

\beginfig
\begin{center}
\input{r6-fig-Underspecification.tex}
\end{center}
\caption{Explicitly unspecified behavior}\label{fig:Underspecification}
\endfig

The rules in figure~\ref{fig:Underspecification} cover aspects of the
semantics that are explicitly unspecified. Implementations can replace
the rules \rulename{6ueqv}, \rulename{6uval} and with different rules that cover the left-hand sides and, as long as they follow the informal specification, any replacement is valid. Those three situations correspond to the case when \va{eqv?} applied to two procedures and when multiple values are used in a single-value context.

The remaining rules in figure~\ref{fig:Underspecification} cover the results from the assignment operations, \sy{set!}, \va{set-car!}, and \va{set-cdr!}. An implementation does not adjust those rules, but instead renders them useless by adjusting the rules that insert \va{unspecified}: \rulename{6setcar}, \rulename{6setcdr}, \rulename{6set}, and \rulename{6setd}. Those rules can be adjusted by replacing \va{unspecified} with any number of values in those rules.

So, the remaining rules just specify the minimal behavior that we know that a value or values must have and otherwise reduce to an \textbf{unknown:} state. The rule \rulename{6udemand} drops \va{unspecified} in the \sy{U} context. See figure~\ref{fig:ec-grammar} for the precise definition of \sy{U}, but intuitively it is a context that is only a single expression layer deep that contains expressions whose value depends on the value of their subexpressions, like the first subexpression of a \sy{if}. Following that are rules that discard \va{unspecified} in expressions that discard the results of some of their subexpressions. The \rulename{6ubegin} shows how \sy{begin} discards its first expression when there are more expressions to evaluate. The next two rules, \rulename{6uhandlers} and \rulename{6udw} propagate \va{unspecified} to their context, since they also return any number of values to their context. Finally, the two \va{begin0} rules preserve \va{unspecified} until the rule \rulename{6begin01} can return it to its context.

%\section*{Acknowledgments}
%Thanks to Michael Sperber for many helpful discussions of specific points in the semantics, for spotting many mistakes and places where the formal semantics diverged from the informal semantics, and for generally making it possible for us to keep up with changes to the informal semantics as it developed. Thanks also to Will Clinger for a careful reading of the semantics and its explanation.

%%% Local Variables: 
%%% mode: latex
%%% TeX-master: "paper"
%%% End: 
 \par
\chapter{Sample definitions for derived forms}
\label{derivedformsappendix}

This appendix contains sample definitions for some of the keywords
described in this report in terms of simpler forms:

\subsubsection*{{\tt cond}}
The {\cf cond} keyword (section~\ref{cond}) 
could be defined in terms of {\cf if}, {\cf let} and {\cf
  begin} using {\cf syntax-rules} (see
section~\ref{syntaxrulessection}) as follows:

\begin{scheme}
(define-syntax \ide{cond}
  (syntax-rules (else =>)
    ((cond (else result1 result2 ...))
     (begin result1 result2 ...))
    ((cond (test => result))
     (let ((temp test))
       (if temp (result temp))))
    ((cond (test => result) clause1 clause2 ...)
     (let ((temp test))
       (if temp
           (result temp)
           (cond clause1 clause2 ...))))
    ((cond (test)) test)
    ((cond (test) clause1 clause2 ...)
     (let ((temp test))
       (if temp
           temp
           (cond clause1 clause2 ...))))
    ((cond (test result1 result2 ...))
     (if test (begin result1 result2 ...)))
    ((cond (test result1 result2 ...)
           clause1 clause2 ...)
     (if test
         (begin result1 result2 ...)
         (cond clause1 clause2 ...)))))
\end{scheme}
\subsubsection*{{\tt case}}
The {\cf case} keyword (section~\ref{case}) could be defined in terms of {\cf let}, {\cf cond}, and
{\cf memv} (see library chapter~\ref{lib:listutilities}) using {\cf syntax-rules}
(see section~\ref{syntaxrulessection}) as follows:

\begin{scheme}
(define-syntax \ide{case}
  (syntax-rules (else)
    ((case expr0
       ((key ...) res1 res2 ...)
       ...
       (else else-res1 else-res2 ...))
     (let ((tmp expr0))
       (cond
         ((memv tmp '(key ...)) res1 res2 ...)
         ...
         (else else-res1 else-res2 ...))))
    ((case expr0
       ((keya ...) res1a res2a ...)
       ((keyb ...) res1b res2b ...)
       ...)
     (let ((tmp expr0))
       (cond
         ((memv tmp '(keya ...)) res1a res2a ...)
         ((memv tmp '(keyb ...)) res1b res2b ...)
         ...)))))
\end{scheme}

\subsubsection*{{\tt letrec}}
The {\cf letrec} keyword (section~\ref{letrec})
could be defined approximately in terms of {\cf let}
and {\cf set!} using {\cf syntax-rules} (see
section~\ref{syntaxrulessection}), using a helper
to generate the temporary variables
needed to hold the values before the assignments are made,
as follows:

\begin{scheme}
(define-syntax \ide{letrec}
  (syntax-rules ()
    ((letrec () body1 body2 ...)
     (let () body1 body2 ...))
    ((letrec ((var init) ...) body1 body2 ...)
     (letrec-helper
       (var ...)
       ()
       ((var init) ...)
       body1 body2 ...))))

(define-syntax letrec-helper
  (syntax-rules ()
    ((letrec-helper
       ()
       (temp ...)
       ((var init) ...)
       body1 body2 ...)
     (let ((var <undefined>) ...)
       (let ((temp init) ...)
         (set! var temp)
         ...)
       (let () body1 body2 ...)))
    ((letrec-helper
       (x y ...)
       (temp ...)
       ((var init) ...)
       body1 body2 ...)
     (letrec-helper
       (y ...)
       (newtemp temp ...)
       ((var init) ...)
       body1 body2 ...))))
\end{scheme}

The syntax {\cf <undefined>} represents an expression that
returns something that, when stored in a location, causes an exception
with condition type {\cf\&assertion} to
be raised if an attempt to read to or write from the location occurs before the
assignments generated by the {\cf letrec} transformation take place.
(No such expression is defined in Scheme.)

\subsubsection*{{\tt let-values}}
The following definition of {\cf let-values} (section~\ref{let-values})
using {\cf syntax-rules} (see section~\ref{syntaxrulessection})
employs a pair of helpers to
create temporary names for the formals.

\begin{scheme}
(define-syntax let-values
  (syntax-rules ()
    ((let-values (binding ...) body1 body2 ...)
     (let-values-helper1
       ()
       (binding ...)
       body1 body2 ...))))

(define-syntax let-values-helper1
  ;; map over the bindings
  (syntax-rules ()
    ((let-values
       ((id temp) ...)
       ()
       body1 body2 ...)
     (let ((id temp) ...) body1 body2 ...))
    ((let-values
       assocs
       ((formals1 expr1) (formals2 expr2) ...)
       body1 body2 ...)
     (let-values-helper2
       formals1
       ()
       expr1
       assocs
       ((formals2 expr2) ...)
       body1 body2 ...))))

(define-syntax let-values-helper2
  ;; create temporaries for the formals
  (syntax-rules ()
    ((let-values-helper2
       ()
       temp-formals
       expr1
       assocs
       bindings
       body1 body2 ...)
     (call-with-values
       (lambda () expr1)
       (lambda temp-formals
         (let-values-helper1
           assocs
           bindings
           body1 body2 ...))))
    ((let-values-helper2
       (first . rest)
       (temp ...)
       expr1
       (assoc ...)
       bindings
       body1 body2 ...)
     (let-values-helper2
       rest
       (temp ... newtemp)
       expr1
       (assoc ... (first newtemp))
       bindings
       body1 body2 ...))
    ((let-values-helper2
       rest-formal
       (temp ...)
       expr1
       (assoc ...)
       bindings
       body1 body2 ...)
     (call-with-values
       (lambda () expr1)
       (lambda (temp ... . newtemp)
         (let-values-helper1
           (assoc ... (rest-formal newtemp))
           bindings
           body1 body2 ...))))))
\end{scheme}


%%% Local Variables: 
%%% mode: latex
%%% TeX-master: "r6rs"
%%% End: 
 \par
\extrapart{Additional material}

The Schemers web site at
\begin{center}
{\cf http://www.schemers.org/}
\end{center}
as well as the Readscheme site at
\begin{center}
{\cf http://library.readscheme.org/}
\end{center}
contain extensive Scheme bibliographies, as well as papers,
programs, implementations, and other material related to Scheme.

%%% Local Variables: 
%%% mode: latex
%%% TeX-master: "r6rs"
%%% End: 
 \par
\chapter{Example }

\nobreak
This section describes an example consisting of the
\library{runge-kutta} library, which provides an {\cf integrate-system}
procedure that integrates the system 
$$y_k^\prime = f_k(y_1, y_2, \ldots, y_n), \; k = 1, \ldots, n$$
of differential equations with the method of Runge-Kutta.

As the \library{runge-kutta} library makes use of the \library{r6rs
  base} and the \library{r6rs promises} libraries, the library
skeleton looks as follows:

\begin{scheme}
\#!r6rs
(library (runge-kutta)
  (export integrate-system)
  (import (r6rs base)
          (r6rs promises))
  \hyper{library body})
\end{scheme}

The procedure definitions go in the place of \hyper{library body}
described below:

The parameter {\tt system-derivative} is a function that takes a system
state (a vector of values for the state variables $y_1, \ldots, y_n$)
and produces a system derivative (the values $y_1^\prime, \ldots,
y_n^\prime$).  The parameter {\tt initial-state} provides an initial
system state, and {\tt h} is an initial guess for the length of the
integration step.

The value returned by {\cf integrate-system} is an infinite stream of
system states.

\begin{schemenoindent}
(define integrate-system
  (lambda (system-derivative initial-state h)
    (let ((next (runge-kutta-4 system-derivative h)))
      (letrec ((states
                (cons initial-state
                      (delay (map-streams next
                                          states)))))
        states))))%
\end{schemenoindent}

The {\cf runge-Kutta-4} procedure takes a function, {\tt f}, that produces a
system derivative from a system state.  The {\cf runge-Kutta-4} procedure
produces a function that takes a system state and
produces a new system state.

\begin{schemenoindent}
(define runge-kutta-4
  (lambda (f h)
    (let ((*h (scale-vector h))
          (*2 (scale-vector 2))
          (*1/2 (scale-vector (/ 1 2)))
          (*1/6 (scale-vector (/ 1 6))))
      (lambda (y)
        ;; y {\rm{}is a system state}
        (let* ((k0 (*h (f y)))
               (k1 (*h (f (add-vectors y (*1/2 k0)))))
               (k2 (*h (f (add-vectors y (*1/2 k1)))))
               (k3 (*h (f (add-vectors y k2)))))
          (add-vectors y
            (*1/6 (add-vectors k0
                               (*2 k1)
                               (*2 k2)
                               k3))))))))
%|--------------------------------------------------|

(define elementwise
  (lambda (f)
    (lambda vectors
      (generate-vector
        (vector-length (car vectors))
        (lambda (i)
          (apply f
                 (map (lambda (v) (vector-ref  v i))
                      vectors)))))))

%|--------------------------------------------------|
(define generate-vector
  (lambda (size proc)
    (let ((ans (make-vector size)))
      (letrec ((loop
                (lambda (i)
                  (cond ((= i size) ans)
                        (else
                         (vector-set! ans i (proc i))
                         (loop (+ i 1)))))))
        (loop 0)))))

(define add-vectors (elementwise +))

(define scale-vector
  (lambda (s)
    (elementwise (lambda (x) (* x s)))))%
\end{schemenoindent}

The {\cf map-streams} procedure is analogous to {\cf map}: it applies its first
argument (a procedure) to all the elements of its second argument (a
stream).

\begin{schemenoindent}
(define map-streams
  (lambda (f s)
    (cons (f (head s))
          (delay (map-streams f (tail s))))))%
\end{schemenoindent}

Infinite streams are implemented as pairs whose car holds the first
element of the stream and whose cdr holds a promise to deliver the rest
of the stream.

\begin{schemenoindent}
(define head car)
(define tail
  (lambda (stream) (force (cdr stream))))%
\end{schemenoindent}

\bigskip
The following script illustrates the use of {\cf integrate-system} in
integrating the system
$$ C {dv_C \over dt} = -i_L - {v_C \over R}$$\nobreak
$$ L {di_L \over dt} = v_C$$
which models a damped oscillator.

\begin{schemenoindent}
\#! /usr/bin/env scheme-script
\#!r6rs
(import (r6rs base)
        (r6rs i/o simple)
        (runge-kutta))

(define damped-oscillator
  (lambda (R L C)
    (lambda (state)
      (let ((Vc (vector-ref state 0))
            (Il (vector-ref state 1)))
        (vector (- 0 (+ (/ Vc (* R C)) (/ Il C)))
                (/ Vc L))))))

(define the-states
  (integrate-system
     (damped-oscillator 10000 1000 .001)
     '\#(1 0)
     .01)))

(letrec ((loop (lambda (s)
                 (newline)
                 (write (head s))
                 (loop (tail s)))))
  (loop the-states))

0%
\end{schemenoindent}

This prints output like the following:

\begin{scheme}
\#(1 0)
\#(0.99895054 9.994835e-6)
\#(0.99780226 1.9978681e-5)
\#(0.9965554 2.9950552e-5)
\#(0.9952102 3.990946e-5)
\#(0.99376684 4.985443e-5)
\#(0.99222565 5.9784474e-5)
\#(0.9905868 6.969862e-5)
\#(0.9888506 7.9595884e-5)
\#(0.9870173 8.94753e-5)
\end{scheme}

%%% Local Variables: 
%%% mode: latex
%%% TeX-master: "r6rs"
%%% End: 
 \par
\chapter{Language changes}

This chapter describes most of the changes that have been made to
Scheme since the ``Revised$^5$ Report''~\cite{R5RS} was published:

\begin{itemize}
\item Scheme source code now uses the Unicode character set.
  Specifically, the character set that can be used for identifiers has
  been greatly expanded.
\item Identifiers can now start with the characters {\cf ->}.
\item Identifiers and symbol literals are now case-sensitive.
\item Bytevector literal syntax has been added.
\item The read-syntax abbreviations {\cf \sharpsign{}'} (for {\cf
    syntax}), {\cf \sharpsign\backquote} (for {\cf quasisyntax}), {\cf
    \sharpsign{},} (for {\cf unsyntax}), and , {\cf \sharpsign{},@}
  (for {\cf unsyntax-splicing} have been added.  (Section~\ref{quotesection}.)
\item The external representation of numbers can now include a
  mantissa width.
\item Literals for NaNs and infinities were added.
\item String and character literal can now use a variety of escape sequences.
\item Block and datum comments have been added.
\item The {\cf !\sharpsign{}r6rs} comment for marking report-compliant
  lexical syntax has been added.
\item Characters are now specified to correspond to Unicode scalar
  values.
\item Many of the procedures and syntactic forms of the language are
  now part of the \rsixlibrary{base} library.  Some procedures and
  syntactic forms have been moved to other libraries; see figure~\ref{r5rsmovedfigure}.

  \begin{figure*}[tb]
    \centering
    \small
    \begin{tabular}[t]{ll}
      identifier & moved to \\\hline
      {\cf assoc} & \rsixlibrary{lists} \\
      {\cf assv} & \rsixlibrary{lists} \\
      {\cf assq} & \rsixlibrary{lists} \\
      {\cf call-with-input-file} & \rsixlibrary{i/o simple} \\
      {\cf call-with-output-file} & \rsixlibrary{i/o simple} \\
      {\cf char-upcase} & \rsixlibrary{unicode} \\
      {\cf char-downcase} & \rsixlibrary{unicode} \\
      {\cf char-ci=?} & \rsixlibrary{unicode} \\
      {\cf char-ci<?} & \rsixlibrary{unicode} \\
      {\cf char-ci>?} & \rsixlibrary{unicode} \\
      {\cf char-ci<=?} & \rsixlibrary{unicode} \\
      {\cf char-ci>=?} & \rsixlibrary{unicode} \\
      {\cf char-alphabetic?} & \rsixlibrary{unicode} \\
      {\cf char-numeric?} & \rsixlibrary{unicode} \\
      {\cf char-whitespace?} & \rsixlibrary{unicode} \\
      {\cf char-upper-case?} & \rsixlibrary{unicode} \\
      {\cf char-lower-case?} & \rsixlibrary{unicode} \\
      {\cf close-input-port} & \rsixlibrary{i/o simple} \\
      {\cf close-output-port} & \rsixlibrary{i/o simple} \\
      {\cf current-input-port} & \rsixlibrary{i/o simple} \\
      {\cf current-output-port} & \rsixlibrary{i/o simple} \\
      {\cf display} & \rsixlibrary{i/o simple} \\
      {\cf do} & \rsixlibrary{control} \\
      {\cf eof-object?} & \rsixlibrary{i/o simple} \\
      {\cf eval} & \rsixlibrary{eval} \\
      {\cf delay} & \rsixlibrary{r5rs}\\
      {\cf exact->inexact} & \rsixlibrary{r5rs}\\
      {\cf force} & \rsixlibrary{r5rs}
\htmlonly \\ \endhtmlonly
\texonly
    \end{tabular}
    \qquad
    \begin{tabular}[t]{ll}
      identifier & moved to \\\hline
\endtexonly
      {\cf inexact->exact} & \rsixlibrary{r5rs}\\
      {\cf member} & \rsixlibrary{lists} \\
      {\cf memv} & \rsixlibrary{lists} \\
      {\cf memq} & \rsixlibrary{lists} \\
      {\cf modulo} & \rsixlibrary{r5rs} \\
      {\cf newline} & \rsixlibrary{i/o simple} \\
      {\cf null-environment} & \rsixlibrary{r5rs} \\
      {\cf open-input-file} & \rsixlibrary{i/o simple} \\
      {\cf open-output-file} & \rsixlibrary{i/o simple} \\
      {\cf peek-char} & \rsixlibrary{i/o simple} \\
      {\cf quotient} & \rsixlibrary{r5rs} \\
      {\cf read} & \rsixlibrary{i/o simple} \\
      {\cf read-char} & \rsixlibrary{i/o simple} \\
      {\cf remainder} & \rsixlibrary{r5rs} \\
      {\cf scheme-report-environment} & \rsixlibrary{r5rs} \\
      {\cf set-car!} & \rsixlibrary{mutable-pairs} \\
      {\cf set-cdr!} & \rsixlibrary{mutable-pairs} \\
      {\cf string-ci=?} & \rsixlibrary{unicode} \\
      {\cf string-ci<?} & \rsixlibrary{unicode} \\
      {\cf string-ci>?} & \rsixlibrary{unicode} \\
      {\cf string-ci<=?} & \rsixlibrary{unicode} \\
      {\cf string-ci>=?} & \rsixlibrary{unicode} \\
      {\cf string-set!} & \rsixlibrary{mutable-strings} \\
      {\cf with-input-from-file} & \rsixlibrary{i/o simple} \\
      {\cf with-output-to-file} & \rsixlibrary{i/o simple} \\
      {\cf write} & \rsixlibrary{i/o simple} \\
      {\cf write-char} & \rsixlibrary{i/o simple}
    \end{tabular}
    \caption{Identifiers moved to libraries}
    \label{r5rsmovedfigure}
  \end{figure*}
\item The base language has the following new procedures and syntactic
  forms: {\cf letrec*}, {\cf let-values}, {\cf let*-values}, {\cf
    real-valued?}, {\cf rational-valued?}, {\cf integer-valued?}, {\cf
    exact}, {\cf inexact}, {\cf finite?}, {\cf infinite?}, {\cf nan?},
  {\cf real->single}, {\cf real->double}, {\cf div}, {\cf mod}, {\cf
    div-and-mod}, {\cf div0}, {\cf mod0}, {\cf div0-and-mod0}, {\cf
    exact-integer-sqrt}, {\cf boolean=?}, {\cf symbol=?}, {\cf
    string-for-each}, {\cf vector-map}, {\cf vector-for-each}, {\cf
    error}, {\cf assertion-violation}, {\cf assert}, {\cf call/cc},
  {\cf identifier-syntax}.
\item The following procedures have been removed: {\cf
    char-ready?}, {\cf transcript-on}, {\cf transcript-off},
  {\cf load}.
\item The case-insensitive string comparisons ({\cf string-ci=?}, {\cf
    string-ci<?}, {\cf string-ci>?}, {\cf string-ci<=?}, {\cf
    string-ci>=?}) operate on the case-folded versions of the strings
  rather than as the simple lexicographic ordering induced by the
  corresponding character comparison procedures.
\item Libraries have been added to the language.
\item A number of standard libraries are described in a separate
  report~\cite{R6RS-libraries}.
\item Many situations that ``were an error'' now have defined or
  constrained behavior.  In particular, many are now specified in
  terms of the exception system.
\item The full numeric tower is now required.
\item The semantics for the transcendental functions has been
  specified more fully.
\item The semantics of {\cf expt} for zero bases has been refined.
\item In {\cf syntax-rules} forms, a {\cf\_} may be used in place of
  the keyword.
\item The {\cf let-syntax} and {\cf letrec-syntax} no longer introduce a
  new environment for their bodies.
\item For implementations where NaNs and/or infinities are available,
  the semantics of many arithmetic operations has been specified on
  these values consistently with IEEE~754.
\item For implementations that support a distinct -0.0, the semantics
  of many arithmetic operations with regard to -0.0 has been specified
  consistently with IEEE~754.
\item Scheme's reals now have an exact zero as their imaginary part.
\item The specification of {\cf quasiquote} has been extended.  Nested
  quasiquotations work correctly now, and {\cf unquote} and {\cf
    unquote-splicing} have been extended to several operands.
\item Immutable objects and procedures now may or may not denote
  locations.  Consequently, {\cf eqv?} is now unspecified in a few
  cases where it was specified before.
\item The mutability of the values of {\cf quasiquote} structures has
  been specified to some degree.
\item The dynamic environment of the \var{before} and \var{after}
  thunks of {\cf dynamic-wind} is now specified.
\item Various expressions that have only side effects are now allowed
  to return an arbitrary number of balues.
\item The order and semantics for macro expansion has been more fully
  specified.
\item Internal definitions are now defined in terms of {\cf letrec*}.
\item The old notion of program structure and Scheme's top-level
  environment has been replaced by top-level programs and libraries.
\item The denotational semantics has been replaced by an operational
  semantics.
\end{itemize}

%%% Local Variables: 
%%% mode: latex
%%% TeX-master: "r6rs"
%%% End: 
 \par
%\newpage                   %  Put bib on it's own page (it's just one)
%\twocolumn[\vspace{-.18in}]%  Last bib item was on a page by itself.
\renewcommand{\bibname}{References}

\bibliographystyle{plain}
\bibliography{abbrevs,rrs}

\vfill\eject


\newcommand{\indexheading}{Alphabetic index of definitions of
  concepts, keywords, and procedures}
\texonly
\newcommand{\indexintro}{The index includes entries from the library
  document; the entries are marked with ``(library)''.}
\endtexonly

\printindex

\end{document}
